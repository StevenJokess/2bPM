%% Generated by Sphinx.
\def\sphinxdocclass{report}
\documentclass[letterpaper,11pt,english]{sphinxmanual}
\ifdefined\pdfpxdimen
   \let\sphinxpxdimen\pdfpxdimen\else\newdimen\sphinxpxdimen
\fi \sphinxpxdimen=.75bp\relax
%% turn off hyperref patch of \index as sphinx.xdy xindy module takes care of
%% suitable \hyperpage mark-up, working around hyperref-xindy incompatibility
\PassOptionsToPackage{hyperindex=false}{hyperref}

\PassOptionsToPackage{warn}{textcomp}

\catcode`^^^^00a0\active\protected\def^^^^00a0{\leavevmode\nobreak\ }
\usepackage{cmap}
\usepackage{fontspec}
\defaultfontfeatures[\rmfamily,\sffamily,\ttfamily]{}
\usepackage{amsmath,amssymb,amstext}
\usepackage{polyglossia}
\setmainlanguage{english}



\setmainfont{FreeSerif}[
  Extension      = .otf,
  UprightFont    = *,
  ItalicFont     = *Italic,
  BoldFont       = *Bold,
  BoldItalicFont = *BoldItalic
]
\setsansfont{FreeSans}[
  Extension      = .otf,
  UprightFont    = *,
  ItalicFont     = *Oblique,
  BoldFont       = *Bold,
  BoldItalicFont = *BoldOblique,
]
\setmonofont{FreeMono}[
  Extension      = .otf,
  UprightFont    = *,
  ItalicFont     = *Oblique,
  BoldFont       = *Bold,
  BoldItalicFont = *BoldOblique,
]


\usepackage[Bjarne]{fncychap}
\usepackage[,numfigreset=2,mathnumfig]{sphinx}
\sphinxsetup{verbatimwithframe=false, verbatimsep=2mm, VerbatimColor={rgb}{.95,.95,.95}}
\fvset{fontsize=\small}
\usepackage{geometry}


% Include hyperref last.
\usepackage{hyperref}
% Fix anchor placement for figures with captions.
\usepackage{hypcap}% it must be loaded after hyperref.
% Set up styles of URL: it should be placed after hyperref.
\urlstyle{same}


\usepackage{sphinxmessages}
\setcounter{tocdepth}{0}



% Page size
\setlength{\voffset}{-14mm}
\addtolength{\textheight}{16mm}

% Chapter title style
\usepackage{titlesec, blindtext, color}
\definecolor{gray75}{gray}{0.75}
\newcommand{\hsp}{\hspace{20pt}}
\titleformat{\chapter}[hang]{\Huge\bfseries}{\thechapter\hsp\textcolor{gray75}{|}\hsp}{0pt}{\Huge\bfseries}

% So some large pictures won't get the full page
\renewcommand{\floatpagefraction}{.8}

\setcounter{tocdepth}{1}
% Use natbib's citation style, e.g. (Li and Smola, 16)
\usepackage{natbib}
\protected\def\sphinxcite{\citep}





% Remove top header
\usepackage[draft]{minted}
\fvset{breaklines=true, breakanywhere=true}
\setlength{\headheight}{13.6pt}
\makeatletter
    \fancypagestyle{normal}{
        \fancyhf{}
        \fancyfoot[LE,RO]{{\py@HeaderFamily\thepage}}
        \fancyfoot[LO]{{\py@HeaderFamily\nouppercase{\rightmark}}}
        \fancyfoot[RE]{{\py@HeaderFamily\nouppercase{\leftmark}}}
        \fancyhead[LE,RO]{{\py@HeaderFamily }}
     }
\makeatother


\title{To be AI PM}
\date{Feb 26, 2021}
\release{0.0.2}
\author{The contributors}
\newcommand{\sphinxlogo}{\vbox{}}
\renewcommand{\releasename}{Release}
\makeindex
\begin{document}

\pagestyle{empty}
\sphinxmaketitle
\pagestyle{plain}
\sphinxtableofcontents
\pagestyle{normal}
\phantomsection\label{\detokenize{index::doc}}


专注成为AI+金融+健身产品经理,我能求职成功吗???


\chapter{AI产品经理专业体系 {[}2{]}}
\label{\detokenize{index:ai-2}}
将网络上杂乱的内容进行有效整理,搭建自己的AI产品经理专业体系

好处:
\begin{itemize}
\item {} 
有目的方向的去扩充我的专业库,积累我的经验。

\item {} 
非常多的信息总结和跟踪,提高我二次工作的效率性。

\item {} 
我的积累形成系统化,随时能快速找出信息。

\end{itemize}


\chapter{常用}
\label{\detokenize{index:id1}}

\section{标签}
\label{\detokenize{index:id2}}\begin{quote}\begin{description}
\item[{lable}] \leavevmode
\item[{ref}] \leavevmode
\end{description}\end{quote}


\subsection{求职AI PM,百度用了我战略idea?}
\label{\detokenize{get_started:ai-pm-idea}}\label{\detokenize{get_started::doc}}
AI产品经理是直接应用或间接涉及了AI技术,进而完成相关AI产品的设计、研发、推广、产品生命周期管理等工作的产品经理。由于AI的技术领域太多、且有更多和垂直行业结合的机会,导致细分AI领域的产品经理所需要的背景和能力可能大不相同。\sphinxhref{https://www.boxuegu.com/news/4368.html}{1}%
\begin{footnote}[1]\sphinxAtStartFootnote
\sphinxnolinkurl{https://www.boxuegu.com/news/4368.html}
%
\end{footnote}

\sphinxstylestrong{电子邮箱: llgg8679@qq.com}


\bigskip\hrule\bigskip


\sphinxstylestrong{一面百度AIstudio产品经理失败}后的总结:\sphinxurl{https://github.com/StevenJokess/d2l-en-read/blob/moreme/chapter-generative-adversarial-networks/aistudio-job.md}

\begin{figure}[H]
\centering
\capstart

\noindent\sphinxincludegraphics{{baidu_kaifa}.png}
\caption{baidu开发者版}\label{\detokenize{get_started:id13}}\end{figure}

\begin{figure}[H]
\centering
\capstart

\noindent\sphinxincludegraphics{{idea_time}.png}
\caption{git时间}\label{\detokenize{get_started:id14}}\end{figure}

可以看到2020年10月3日里面就有最近\sphinxstylestrong{才beta测试}的
\sphinxurl{https://kaifa.baidu.com} 的主意!

据我能找到的最早时间:

\begin{figure}[H]
\centering
\capstart

\noindent\sphinxincludegraphics{{kaifa_online}.png}
\caption{能找到的最早时间}\label{\detokenize{get_started:id15}}\end{figure}

百度股价预测:

\begin{figure}[H]
\centering
\capstart

\noindent\sphinxincludegraphics{{baidu_gujia}.jpg}
\caption{百度股价}\label{\detokenize{get_started:id16}}\end{figure}

百度最新股价:

\begin{figure}[H]
\centering
\capstart

\noindent\sphinxincludegraphics{{baidu_gujia_newest}.png}
\caption{百度最新股价}\label{\detokenize{get_started:id17}}\end{figure}


\bigskip\hrule\bigskip



\subsubsection{\sphinxstylestrong{简历}}
\label{\detokenize{get_started:id1}}

\paragraph{自我评价:}
\label{\detokenize{get_started:id2}}\begin{itemize}
\item {} 
大四应届生,具备AI、计算机、金融的基础知识.

\item {} 
有广博的视野:(读书狂和藏书癖:阅读过上千本书,之前常用Moonreader,并现在微信读书完成2020年111书9.18完
结!,现省下48084.53元
见https://weread.qq.com/misc/booklist/358906697\_7e9fYZVah);

\item {} 
关注Springer、
packt、O’Reilly、manning、机工社、电工社、人邮社、清华社、图灵异步社区强烈推荐数学之美(第三版)

\item {} 
与时俱进的敏锐度(有效性市场、指数增长模型导致的):
良好的体力(兴奋起来熬一夜)和爆发力(30s跳绳速摇).

\item {} 
熟练使用知网、Google学术、SciHub、Stackoverflow、Google等搜索引擎.

\item {} 
会使用VScode、Python、Anaconda、pypi,并能通过自定义快速查询相应的库文档.

\item {} 
深度学习践行者:

\end{itemize}
\begin{enumerate}
\sphinxsetlistlabels{\arabic}{enumi}{enumii}{}{.}%
\item {} 
完成李沐等人写的《动手学深度学习》的GAN和DCGAN从mxnet到pytorch的翻译,且被收录到官方.

\item {} 
运行、测试了移动深度学习的数个框架.

\item {} 
撰写普惠深度学习(WIP)、人工智能产品经理相关书(WIP)

\end{enumerate}


\paragraph{教育经历}
\label{\detokenize{get_started:id3}}
学校: 山西大学

开始时间: 2016\sphinxhyphen{}09\sphinxhyphen{}01

结束时间: 2020\sphinxhyphen{}07\sphinxhyphen{}01

学历: 本科

学位: 学士

专业: 经济学类\sphinxhyphen{}金融学

专业描述: 经济与管理学院 太原

荣誉/奖项:学业奖学金(2017);三好学生(2017)

相关课程:
\begin{itemize}
\item {} 
数学分析(95);高等代数(89);概率论与数理统计(85);大学英语(90)

\item {} 
计算机基础–PS(90);计算机高级语言–C语言(100);网络金融(80);

\item {} 
微观经济学(85);宏观经济学(90);计量经济学(82);投资学(82补考);金融计量学(85);

\item {} 
会计循环实验(91);计量经济学实验(90);证券投资模拟交易(89);EXCEL计算实验(86);商业银行综合业务

\item {} 
实验(87);投资组合管理(81);财务报表分析(80);

\item {} 
金融服务营销(93);金融从业综合素质实训(92);毕业实习(88);

\end{itemize}


\paragraph{工作经历}
\label{\detokenize{get_started:id4}}
企业名称: 个人求职中

开始时间: 2020\sphinxhyphen{}12\sphinxhyphen{}01

结束时间: 2021\sphinxhyphen{}02\sphinxhyphen{}01

职位名称: 学习并撰写AI产品经理相关内容

所在部门: 个人求职中

离职原因: 对AI更热爱。感觉AI能服务更多人。

工作描述:
\begin{itemize}
\item {} 
运用docker的 \sphinxurl{https://github.com/aieye-top/d2l-book2} 包,来:

\item {} 
撰写普惠深度学习(WIP):\sphinxurl{https://github.com/aieye-top/d2cl}

\item {} 
撰写人工智能产品经理相关书(WIP):\sphinxurl{https://stevenjokess.github.io/2bPM/}

\end{itemize}


\bigskip\hrule\bigskip



\paragraph{项目经验}
\label{\detokenize{get_started:id5}}
项目名称: 移动深度学习开发、测试 开始时间: 2020\sphinxhyphen{}08\sphinxhyphen{}01 结束时间:
2020\sphinxhyphen{}09\sphinxhyphen{}01 项目描述:
\begin{itemize}
\item {} 
观看pytorch官方文档和视频,了解了基本的andriod开发体系;

\item {} 
动手完成了针对动物的图片识别项目:\sphinxurl{https://github.com/StevenJokess/pytorch-andriod-greatdemo};

\item {} 
并通过https://stevenjokess.github.io/post/pytorch\sphinxhyphen{}android/来分享经历

\item {} 
项目职责: 同时了解了arm体系和测试了其他框架:

\item {} 
\sphinxurl{https://github.com/StevenJokess/Pytorch-Kotlin-Demo}

\item {} 
\sphinxurl{https://github.com/StevenJokess/djl-android-demo}

\item {} 
\sphinxurl{https://github.com/StevenJokess/paddlelite-andriod-demo}

\end{itemize}


\bigskip\hrule\bigskip


项目名称: 动手学深度学习GAN开发者

开始时间: 2020\sphinxhyphen{}06\sphinxhyphen{}01

结束时间: 2020\sphinxhyphen{}11\sphinxhyphen{}01

项目描述:
\begin{itemize}
\item {} 
开设d2l\sphinxhyphen{}en\sphinxhyphen{}read记录自己所有的学习过程.见https://github.com/StevenJokess/d2l\sphinxhyphen{}en\sphinxhyphen{}read

\item {} 
积极参与discuss.d2l.ai,记录自己遇到的坑,被李沐(MXNet开发者)评为最活跃的参与者.

\item {} 
和mxnet的开发者表达对社区的死气沉沉的不满,并提出活跃社区建议,后被采纳开设discussion区.

\item {} 
并学习更多AI内容记录在https://github.com/StevenJokess/d2l\sphinxhyphen{}en\sphinxhyphen{}read/tree/moreme

\item {} 
项目职责: 运用谷歌、stack
overflow等编程搜索引擎,并积极参与GitHub讨论,完成GAN、DCGAN从MXNet到PyTorch的翻译

\item {} 
PR.项目可参见(点开pytorch标签的最后的“continue discussion”可见)

\item {} 
GAN:http://preview.d2l.ai/d2l\sphinxhyphen{}en/master/chapter\_generative\sphinxhyphen{}adversarial\sphinxhyphen{}networks/gan.html

\item {} 
DCGAN:http://preview.d2l.ai/d2l\sphinxhyphen{}en/master/chapter\_generative\sphinxhyphen{}adversarial\sphinxhyphen{}networks/dcgan.html

\end{itemize}


\bigskip\hrule\bigskip


项目名称: 完成学位论文

开始时间: 2020\sphinxhyphen{}02\sphinxhyphen{}01

结束时间: 2020\sphinxhyphen{}05\sphinxhyphen{}01

项目描述: 独立研究者 repo: \sphinxurl{https://github.com/StevenJokess/gra\_paper}
\begin{itemize}
\item {} 
运用知网、Google学术、SciHub等学术搜索引擎,完成文献综述和翻译.

\item {} 
由于导师没接触过Python,我独立阅读Python文档、十余本相关书籍.

\item {} 
项目职责: 运用Pandas库的DataReader()、datetime()导入股市数据.

\item {} 
to\_excel()导出,后Excel处理缺失数据与整合文件;read\_excel()读取,plt、seaborn库生成时间序列图.

\item {} 
Statsmodel库的极大似然估计下fit()出VAR模型,as\_csv()来保存结果.

\item {} 
Word完成编写、排版,共13656字的《中美贸易摩擦前后中美股市的联动性分析》

\end{itemize}

项目名称: 参加山西省跳绳竞标赛

开始时间: 2018\sphinxhyphen{}07\sphinxhyphen{}01

结束时间: 2018\sphinxhyphen{}08\sphinxhyphen{}01

项目描述:
\begin{itemize}
\item {} 
30s单摇:66;30s双摇:60;三摇:11个

\item {} 
毕业前还可单手俯卧撑、单腿深蹲(现在学AI学肥了。。)

\end{itemize}

项目职责:
\begin{itemize}
\item {} 
偶然在操场练习双摇被相中参加比赛。

\item {} 
作为非体院唯一绳没有的第四棒,在4*30男子团体单摇比赛共250个,取得市和省级铜牌

\item {} 
更多见https://user.qzone.qq.com/867907127的相册“比赛视频”的倒数第二三个视频的第四棒。

\item {} 
社团成员文案抓住大家减肥痛点、展示速摇,招新成功翻4倍。

\end{itemize}


\paragraph{培训经历}
\label{\detokenize{get_started:id6}}
开始时间: 2018\sphinxhyphen{}05\sphinxhyphen{}01 结束时间: 2018\sphinxhyphen{}06\sphinxhyphen{}01 培训机构: 山西跳绳运动协会
培训地点: 山西 培训课程: 跳绳初级裁判、初级教练 获得证书:
跳绳初级裁判证、初级教练证


\paragraph{语言能力}
\label{\detokenize{get_started:id7}}\begin{itemize}
\item {} 
语种: 英语

\item {} 
听说能力: 良好

\item {} 
读写能力: 精通

\item {} 
语言等级: 英语\sphinxhyphen{}英语六级

\end{itemize}


\paragraph{计算机技能}
\label{\detokenize{get_started:id8}}\begin{itemize}
\item {} 
技能类别: Anaconda 掌握程度: 良好

\item {} 
技能类别: VScode 掌握程度: 良好

\item {} 
技能类别: Python 掌握程度: 良好

\item {} 
技能类别: markdown 掌握程度: 良好

\item {} 
技能类别: pytorch 掌握程度: 良好

\item {} 
技能类别: mxnet 掌握程度: 良好

\item {} 
技能类别: Linux 掌握程度: 良好

\item {} 
技能类别: Android开发 掌握程度: 普通

\end{itemize}


\paragraph{专业技能}
\label{\detokenize{get_started:id9}}
技能名称: 跳绳 掌握程度: 精通


\paragraph{证书}
\label{\detokenize{get_started:id10}}\begin{itemize}
\item {} 
证书名称: 会计从业资格证 说明: 大一上获得

\item {} 
证书名称: 跳绳初级教练证 说明: 大二下获得

\item {} 
证书名称: 跳绳初级裁判证 说明: 大二上获得

\item {} 
证书名称: 证券从业资格证 说明: 大一下获得

\item {} 
证书名称: 基金从业资格证 说明: 大三下获得

\item {} 
证书名称: 普通话二甲证书 说明: 大四上获得

\end{itemize}


\subsubsection{个人基本信息}
\label{\detokenize{get_started:id11}}
接受调剂: 不接受

姓名: 蔡舒起

性别: 男

出生日期: 1998\sphinxhyphen{}08\sphinxhyphen{}11

国籍/地区: 中国

民族: 汉族

婚姻状况: 未婚

工作年限: 无经验

政治面貌: 共青团员

证件类型: 身份证

证件号码: ?

现居住地: 浙江省\sphinxhyphen{}台州市

籍贯: 浙江省\sphinxhyphen{}台州市

学历: 本科

毕业时间: 2020\sphinxhyphen{}07\sphinxhyphen{}01

学位: 学士

毕业院校: 山西大学

专业: 经济学类\sphinxhyphen{}金融学

移动电话: 1840xxxxxxx

电子邮箱: \sphinxhref{mailto:llgg8679@qq.com}{llgg8679@qq.com}


\paragraph{求职意向}
\label{\detokenize{get_started:id12}}
期望工作性质: 全职

期望行业: 互联网/电子商务/AI金融/AI健身

目前薪酬: 面议

期望薪酬: 面议

期望年薪: 面议

到岗时间: 随时

\begin{figure}[H]
\centering
\capstart

\noindent\sphinxincludegraphics{{rope}.png}
\caption{跳绳证书、六级}\label{\detokenize{get_started:id18}}\end{figure}

\begin{figure}[H]
\centering
\capstart

\noindent\sphinxincludegraphics{{resume}.jpg}
\caption{未太更新的简历}\label{\detokenize{get_started:id19}}\end{figure}


\subsection{Introduction}
\label{\detokenize{chapter_introduction/index:introduction}}\label{\detokenize{chapter_introduction/index:chap-intro}}\label{\detokenize{chapter_introduction/index::doc}}

\subsubsection{需求}
\label{\detokenize{chapter_introduction/need:id1}}\label{\detokenize{chapter_introduction/need::doc}}
每个产品都应该有自己的主路径,没有主路径的需求都是伪需求。\sphinxhref{http://www.woshipm.com/pmd/2903334.html}{2}%
\begin{footnote}[2]\sphinxAtStartFootnote
\sphinxnolinkurl{http://www.woshipm.com/pmd/2903334.html}
%
\end{footnote}

解决一个他必须要解决的问题:越重要,你的产品核心价值就越大。

迅雷的主路径就是:搜电影\sphinxhyphen{}下电影\sphinxhyphen{}看电影

满足用户的需求全过程:“边看边播”的功能;比如迅雷后来还增加了影评和打分等功能,这一切都是围绕主路径而衍生。

就像互联网产品经理刚出现的时候也没有真正的一套知识和技能体系供大家去参考,目前的互联网产品经理的知识和技能体系也是随着各大公司招聘要求相互碰撞和产品研发过程中不断摸索所得到的一个共同体系,现在的AI产品经理,由于没有大面积成熟商业产品落地,现在所面临的知识空窗期只能说更为严重。

我对伪需求的理解其实蛮简单的,就看两点:是否是围绕主路径;该需求的使用频率。


\paragraph{需求采集}
\label{\detokenize{chapter_introduction/need:id2}}
直接采集与间接采集,获取到的需求分别是一手需求与二手需求。

可以从两个角度来理解它们的差异:需求的提出者是不是有需求的人、需求是原始的还是加工过的。

直接采集的一手需求更准确,所以产品经理一定要确保手里有足够比例的需求是直接采集的,这样才能让产品本身和自己对产品的判断更接地气。

而间接采集的二手需求,就需要带着“问号”来看,思考原始需求方和转述者分别是谁,以及它有没有被曲解过。但二手需求(比如一份客户反馈周报)可以通过更多的人,收集到更多的用户声音,而且是经过梳理的,所以获取信息的效率更高。

扩展到实践层面,团队内“全员参与采集,产品人员处理”是比较可行的模式,是一种效率和准确度的兼顾方案。


\subparagraph{直接采集的途径 3\sphinxfootnotemark[3]}
\label{\detokenize{chapter_introduction/need:id3}}%
\begin{footnotetext}[3]\sphinxAtStartFootnote
\sphinxnolinkurl{http://www.woshipm.com/zhichang/459131.html}
%
\end{footnotetext}\ignorespaces \begin{itemize}
\item {} 
用户访谈提出需求

\item {} 
其他参与者和关注者反馈的需求

\end{itemize}


\subparagraph{间接采集的途径 3\sphinxfootnotemark[4]}
\label{\detokenize{chapter_introduction/need:id4}}%
\begin{footnotetext}[4]\sphinxAtStartFootnote
\sphinxnolinkurl{http://www.woshipm.com/zhichang/459131.html}
%
\end{footnotetext}\ignorespaces \begin{itemize}
\item {} 
老板提出的战略性需求

\item {} 
产品经理根据产品方向规划需求

\item {} 
推广规划的活动和数据分析出来一些需求

\end{itemize}


\paragraph{先满足哪类用户?}
\label{\detokenize{chapter_introduction/need:id5}}
如果说,普通用户有一个需求,核心用户也有一个需求,而且,两种用户群体的需求都是比较重要的大需求,那么,你觉得应该先满足哪类用户?

答案是先满足普通用户。

因为核心用户属于平台忠诚度非常高的用户群体,在产品里已经投入了较多的时间和精力,并不会因为你的需求稍微晚一些就选择离开;即使离开也只是暂时的,他还会回来的,因为这是沉没成本,而且你并没有做出严重伤害他们的行为。

不论运营还是产品,都应该重视用户留存,也必须绝对肯定的要重视,甚至在产品的爆发期,留存大于一切。


\paragraph{早期产品的三个核心问题}
\label{\detokenize{chapter_introduction/need:id6}}\begin{itemize}
\item {} 
需求:解决什么人的什么需求

\item {} 
具体形态:如何解决的

\item {} 
推广:人们怎么知道它

\end{itemize}

首先是需求,产品所解决的需求是一个多大的市场规模,是大部分人都需要的,还是仅局限在一个垂直的人群,规模多大?这个需求出现的频率如何,是每天都需要的,还是每周几次,还是隔上至少个把月甚至更长时间才能想到的?这个道理很简单,那些绝大部分人都需要的而且每天都需要的,是S级的需求,比如微信解决的是沟通这样的SSS级需求,又比如搜索、支付、影音,都是大部分人经常用到的;而那些尽管小众、但经常使用,又或者虽然使用频度不高,但也是大部分人都需要的,是次一级需求,比如教育、办公,比如购物、旅游;遇到那些不知道做给谁的、不知道多久才能想起来一次的产品,基本就算了吧。


\paragraph{需求的checklist}
\label{\detokenize{chapter_introduction/need:checklist}}
\begin{figure}[H]
\centering
\capstart

\noindent\sphinxincludegraphics{{define_need}.png}
\caption{需求的checklist}\label{\detokenize{chapter_introduction/need:id7}}\end{figure}


\subsubsection{产品}
\label{\detokenize{chapter_introduction/Product:id1}}\label{\detokenize{chapter_introduction/Product::doc}}

\paragraph{灵感}
\label{\detokenize{chapter_introduction/Product:id2}}
要把自己当成一个傻子,用一种特别挑剔、不满意的方法去试用自己的产品,绝对能够发现很多问题。我们很多产品的改进,都是来自用户看起来不理性的投诉、粗暴的回应,但是认真想一想,用户的不满背后其实都代表了一种需求。
\sphinxhref{https://www.jianshu.com/p/ef308c923f06}{5}%
\begin{footnote}[5]\sphinxAtStartFootnote
\sphinxnolinkurl{https://www.jianshu.com/p/ef308c923f06}
%
\end{footnote}


\paragraph{什么算作成功的产品?}
\label{\detokenize{chapter_introduction/Product:id3}}
用户人数多(留存、活跃、新增这个优先度递减)的产品就是成功的,这是从用户的角度看问题;可以利用信息打破阶层跨越的,这是从社会角度看问题;能为公司赚钱的就是成功,逆推回去没为公司赚钱的就不是好产品吗?许多产品人气很旺但是并不要赚钱,只是用来市场卡位的,这样不算好产品吗?这是运营的思路,不是产品经理的思路。最后一个答案,长得好看的产品,那是美工的思路,也不是产品经理的思路,甚至是易学易用、层次感,也只是交互设计师的思路,同样不是产品经理的思路。


\subparagraph{用户的维度}
\label{\detokenize{chapter_introduction/Product:id4}}\begin{itemize}
\item {} 
首先要解决用户的需求;

\item {} 
其次是要有黏性(持续不断地迭代,以满足用户不断产生的新需求);

\item {} 
要拥有不错的体验(快速、有效)。

\end{itemize}


\subparagraph{需求}
\label{\detokenize{chapter_introduction/Product:id5}}
这是一个产品之所以被称为产品的前提,产品的本质就是用来解决需求的,黏性和体验是之后的事。


\subparagraph{黏性}
\label{\detokenize{chapter_introduction/Product:id6}}
一个成功的产品,一定是不断被用户想起的产品,一旦用户产生了某种需求,就能想起你,这就是一个好的产品。有黏性的产品一定是很好的解决了某种需求,而且做到了竞品没有的高度。用户用了一次就不再使用,说明你的产品并不好,或者说干脆就是定位有了问题。


\subparagraph{优秀的用户体验}
\label{\detokenize{chapter_introduction/Product:id7}}
在这个产品同质化竞争比较严重的时代,好的用户体验就是商机,尤其是你弯道超车的策略之一。例如电商三只松鼠的用户体验:在你收到包裹的时候你就会发现每个包装坚果的箱子上都会贴着一段手写体的给快递的话:“快递叔叔我要到我主人那了,你一定要轻拿轻放哦,如果你需要的话也可以直接购买哦。”打开包裹后会发现,每一包坚果都送了一个果壳袋,方便把果壳放在里面;打开坚果的包装袋后,每一个袋子里还有一个封口夹,可以把吃了一半但吃不完的坚果袋儿封住。令你想不到的还有,袋子里备好的擦手湿巾,方便吃之前不用洗手。这些小小的变化使他们的销售额不断增长。所以说好的用户体验就是商机。


\subparagraph{商业的维度}
\label{\detokenize{chapter_introduction/Product:id8}}
成功的产品可以持续为公司创造长期的商业价值,包括但不限于用户规模、产品利润等。


\subparagraph{技术实现的维度}
\label{\detokenize{chapter_introduction/Product:id9}}
成功的产品,其产品方案利用当前的技术就可以实现,并且可以长期维护、持续完善。


\paragraph{设计原则}
\label{\detokenize{chapter_introduction/Product:id10}}\begin{itemize}
\item {} 
好的产品是有创意的

\item {} 
好的产品必须对人有用

\item {} 
好的产品是优美的

\item {} 
好的产品是容易使用的

\item {} 
好的产品是含蓄的,不招摇的

\item {} 
好的产品是诚实的

\item {} 
好的产品会经久不衰

\item {} 
好的产品不会放过任何一个细节

\item {} 
好的产品是环保的,不浪费太多资源的

\item {} 
好的产品尽可能少地体现设计(少即是多)

\end{itemize}


\paragraph{产品思维}
\label{\detokenize{chapter_introduction/Product:id11}}\begin{enumerate}
\sphinxsetlistlabels{\arabic}{enumi}{enumii}{}{.}%
\item {} 
定位、差异点

\item {} 
动力引擎

\item {} 
核心输出

\item {} 
外围价值

\item {} 
商业模式

\item {} 
价值放大

\end{enumerate}


\subparagraph{定位、差异点}
\label{\detokenize{chapter_introduction/Product:id12}}
陌陌:陌生交友 知群:企业资源


\subparagraph{动力引擎}
\label{\detokenize{chapter_introduction/Product:id13}}
直播带货:流量、价格(新形式的团购)。 知群:以招聘内推作为底层驱动力


\subparagraph{核心输出}
\label{\detokenize{chapter_introduction/Product:id14}}
知群:入行


\subparagraph{外围价值}
\label{\detokenize{chapter_introduction/Product:id15}}
知乎:交流空间、外部性


\subparagraph{商业模式}
\label{\detokenize{chapter_introduction/Product:id16}}
知群:TOP班提供可靠学习保障。


\subparagraph{价值放大}
\label{\detokenize{chapter_introduction/Product:id17}}

\paragraph{产品层次}
\label{\detokenize{chapter_introduction/Product:id18}}\begin{enumerate}
\sphinxsetlistlabels{\arabic}{enumi}{enumii}{}{.}%
\item {} 
核心产品:真正所要求——购买唇膏,不只是买嘴唇的颜色而是销售希望

\item {} 
有形产品:质量水准、功能特色、式样、品牌以及包装。

\item {} 
附加产品:提供购买零件保证书、技术、免费操作课程、快速维修服务,和询问任何问题及疑难的免费电话专线。

\end{enumerate}


\paragraph{做出来和推出去的效率}
\label{\detokenize{chapter_introduction/Product:id19}}\begin{itemize}
\item {} 
出来的效率,在管理学里专业的说法是“生产制造的可扩展性”。打比方说,一款产品如果给
10
倍的用户使用,那么这款产品在生产制造上的成本提升是多少?如果成本提升得少,就是可扩展性高。

\item {} 
推出去的效率,它的专业说法叫“销售传播的可扩展性”。同样的比方,一款产品给
10
倍的用户使用,它在销售传播上的成本提升是多少?如果成本提升得少,就是可扩展性高。

\end{itemize}

提升做出来效率的常见方法:
\begin{itemize}
\item {} 
降低复制成本,比如标准化、数字化、智能化;

\item {} 
提供基础设施,然后众包 / 外包生产过程。

\end{itemize}

提升推出去效率的常见方法:
\begin{itemize}
\item {} 
消除时间、地点等销售传播的限制因素;

\item {} 
产品数字化,减少,甚至消除物流环节;

\item {} 
提供基础设施,然后众包/外包分销过程。

\end{itemize}

关于做出来和推出去效率的提升,我们能看到一些大的趋势:

首先,产品交付从实到虚,再到虚实结合,这是因为人们不能只活在数字世界里;
其次,效率高的产品供给方,都会渐渐的演变成平台,让更多的玩家、更多的用户参与到做和推的过程中。


\paragraph{如何起步? 3\sphinxfootnotemark[6]}
\label{\detokenize{chapter_introduction/Product:id20}}%
\begin{footnotetext}[6]\sphinxAtStartFootnote
\sphinxnolinkurl{https://www.jianshu.com/p/266cd3df64d5}
%
\end{footnotetext}\ignorespaces 
一款产品的起步是有个逻辑顺序的,《产品游戏化》一书里归纳出的逻辑顺序是:习惯打造、启程、发现、精通。以下,我们把“习惯打造”模块,简称为“习惯”模块。

需要注意的是,这和一个新用户使用产品的逻辑顺序并不相同,因为用户是按照“发现、启程、习惯、精通”来使用产品的。

产品起步思维是有实用场景的,更适用于正在从小量用户逐步扩展到大量用户的产品。如果你服务的是少数大客户,第一次交付的产品就需要已经相对完整才行。


\subparagraph{习惯}
\label{\detokenize{chapter_introduction/Product:id21}}
先回到“做产品”的逻辑上来,它的第一个模块是习惯。

这要求你先打造出某个
对用户有价值的闭环,用户来了,获得价值了,下一次还愿意来。这个最小的产品模块,已经可以用来做“留存假设”的验证,所以这也算是第三轮的
MVP 了,这里的 P 代表 Product。


\subparagraph{启程}
\label{\detokenize{chapter_introduction/Product:id22}}
第二个要做的模块是启程,即用户的第一次体验。

启程模块是产品的验证对象扩展开以后,做给相对的“新手用户”的,最常见的就是各种产品里的“新手上路”模块。

之所以不用最先做启程,是因为产品的早期使用者,往往是高手行家,我们也常把这群人称作种子用户、天使用户,即便没人手把手指导,他们也能用得很溜。


\subparagraph{发现}
\label{\detokenize{chapter_introduction/Product:id23}}
然后是发现模块。有了一批新人用户之后,我们算是验证完了启程与习惯模块,这时候产品应该进入推广阶段,开始做“发现”模块。

我们要发掘出用户在何时、何地会对产品产生第一印象,会通过什么渠道第一次接触产品。如果是手机
App
的话,用户在应用商店里看到的广告、搜索产品名称、下载安装,直到第一次点击打开
App 都算是发现模块。


\subparagraph{精通}
\label{\detokenize{chapter_introduction/Product:id24}}
最后要做的是精通模块。当产品运营了一段时间之后,就会有相当数量的用户对产品了如指掌,这时候才有必要给他们打造“精通”系统,让他们不断地收到新的刺激。这是高级功能,可以考虑让高级用户参与贡献,充分利用你最热情用户的深层次需求和驱动力。

比如,服务产品里,让高级用户做志愿者,论坛里让高级用户做版主,游戏里让高级玩家做分区的督导者等等,都算是产品的精通模块。

这时候,你已经在打造上一讲里提到的个体粘性、群体粘性了,这些特性的成功,会使产品拥有自己的正反馈闭环,也常常被叫做增长飞轮。

当然也有例外,有些产品,所有用户很快就精通了,基本上,这个产品也就没啥想象力了,比如手电筒
App。


\paragraph{产品服务系统 4\sphinxfootnotemark[7]}
\label{\detokenize{chapter_introduction/Product:id25}}%
\begin{footnotetext}[7]\sphinxAtStartFootnote
\sphinxnolinkurl{https://www.jianshu.com/p/75de15c9d6b3}
%
\end{footnotetext}\ignorespaces 
“产品服务系统”能以一种集成的、有针对性的方式进行产品分类,精准地满足用户需求,有助于产品的创新。

产品服务系统的核心要点是,任何广义的产品都包含有实体部分和服务部分,三大导向,从实体到服务,实体部分越来越少,服务部分越来越多,逐渐过渡。

分三大类导向的产品服务系统,即“实体导向”“使用导向”“结果导向”。


\subparagraph{实体导向}
\label{\detokenize{chapter_introduction/Product:id26}}
第一种,实体导向的产品服务系统。这种类型是以实体为主,包含有少量服务。它的服务目的是让用户可以顺利地使用产品实体,是与实体紧密相关的。比如空调和它的上门安装、保修服务。


\subparagraph{使用导向}
\label{\detokenize{chapter_introduction/Product:id27}}
使用导向的产品服务系统,它和实体导向型产品的区别在于,供给方给你的不是所有权,而是长期独占的使用权(Lease),或者是某种条件下,一段时间的使用权(Renting/Sharing),甚至是共享的使用权(Pooling)。比如摩拜单车
1 小时使用权。

因为使用导向的情况下,用户买的并不是实体,所以相关的配套服务会多一些,以确保用户使用顺利。


\subparagraph{结果导向}
\label{\detokenize{chapter_introduction/Product:id28}}
结果导向就以服务为主了,你要买的不是一个实体,而是一种“结果”,使用实体只是为了达成结果需要用的一个过程或者一个媒介而已。比如网络广告,按点击量、按成交量付费等模式。

有时在消费完结果导向的产品后,你可能甚至感知不到实体的存在,比如付费聊天、轻咨询,甚至是寺庙里求签拜佛。


\paragraph{三种导向间的演变趋势}
\label{\detokenize{chapter_introduction/Product:id29}}

\subparagraph{用户模式}
\label{\detokenize{chapter_introduction/Product:id30}}
从实体到服务的变化意味着从“成交终止”到“成交开始”。

从实体到服务,供应者与用户的关系有越来越紧密的趋势,触点越来越多,用户尝试的成本越来越低。

在这个时代,因为社会供给越来越丰富,所以各种产品的市场会越来越供过于求,这会导致需求驱动而不是生产驱动,用户变得越来越重要。所以,我们要好好思考如何更多地接触用户,给用户创造价值,从而为公司创造更多的商业价值。

比如一个做人工智能客服机器人的生意,这是一种典型的 2B
企业服务。对小客户的交付中,实体比例更多,更偏实体导向,大多数功能让客户自助完成使用。但对
VIP
大客户的交付中,就是服务比例更多,更偏结果导向,甚至会提供外包的客服人员。

因为相对来说,小客户比较容易批量获得,而大客户需要一个个地”啃“,更需要建立长期的关系。这一点,也会体现在下面的增长模式上。

所以,从这个角度来看,越是重要的用户,就越要用服务比例高的产品服务系统来完成交付。


\subparagraph{增长模式}
\label{\detokenize{chapter_introduction/Product:id31}}
增长模式下的实体到服务,是从“数量复制”到“人尽其用”。

不同的卖法,增长的方式不同。实体更容易标准化,从而可以批量地卖给更多的用户,我把这个叫作数量复制。而服务的极致体验是个性化,所以增长的模式挖掘每个用户的更多需求,这叫作人尽其用。

这个角度给我们的启发就是,随着产品供给的极大丰富,没有被开发的用户已经越来越少了,所以我们更要思考如何在已有用户身上做文章,精细化运营。

比如一个软件,是使用导向的产品,如果它卖的是软件 1
年的使用权,就没法向数据量大的用户收更多的钱。这时候如果改为结果导向,根据数据量收费,那么既可以让数据量少的用户几乎免费使用,降低他们尝试的门槛,也可以充分赚取大客户的费用,对方也更愿意为好的结果付费。


\subparagraph{财务模式}
\label{\detokenize{chapter_introduction/Product:id32}}
在财务模式下,实体到服务的变化是从“当期收入”变为“预期收入”。

从用户模式到增长模式,再到财务模式,实体比例越来越低,会造成的必然结果是短期收入减少,资产投入增加,利润减少,但预期利润增加。

比如房企不卖房,改做长租生意了,那就没有了卖房时那一大笔的即时收入,在一段时间内的资金压力就很大。

所以,偏服务的产品服务系统,不确定性更高,更需要我们掌握新的产品创新方法,更需要有长远的眼光。


\paragraph{从单一产品到产品矩阵 2\sphinxfootnotemark[8]}
\label{\detokenize{chapter_introduction/Product:id33}}%
\begin{footnotetext}[8]\sphinxAtStartFootnote
\sphinxnolinkurl{https://www.jianshu.com/p/ed738dac00e5}
%
\end{footnotetext}\ignorespaces \begin{itemize}
\item {} 
PSF,是
Problem\sphinxhyphen{}Solution\sphinxhyphen{}Fit,问题与解决方案的匹配,这是价值假设,相当于从 0
到 1;

\item {} 
PMF,是 Product\sphinxhyphen{}Market\sphinxhyphen{}Fit,产品与市场的匹配,这是增长假设,是从 1 到
N;

\item {} 
PRF,是
Positioning\sphinxhyphen{}Resource\sphinxhyphen{}Fit,定位与资源的匹配,这是长青假设,是从 N
到正无穷。

\end{itemize}


\subparagraph{价值假设:问题与解决方案的匹配}
\label{\detokenize{chapter_introduction/Product:id34}}
PSF 要验证的是价值,即问题对不对,解决方案对不对,对应着前两轮
MVP,也就是 Paperwork 和 Prototype 阶段。

这一阶段中常见的错误有三点:
\begin{enumerate}
\sphinxsetlistlabels{\arabic}{enumi}{enumii}{}{.}%
\item {} 
问题不存在,是臆想出来的。(点子过滤器来避免)

\item {} 
解决方案不存在。根本无解的事,多思无益。(询问领域专家来避免)

\item {} 
问题也有,解决方案也有,但是问题(P)和解决方案(S)不匹配。(用户测试来避免)

\end{enumerate}


\subparagraph{增长假设:产品与市场的匹配}
\label{\detokenize{chapter_introduction/Product:id35}}
如果问题(P)和解决方案(S)匹配了,达到了
PSF,我们才算有了一个产品,也就是 PMF 的
P——Product。这时候重点就变成了后两轮 MVP,Product 和 Promotion
相关的内容了。

PMF 讲的是产品与市场的匹配,要验证的是增长,也就是产品的生产 /
分销可扩展性好不好,市场是不是足够好。

产品与市场的匹配中常见的几种错误:
\begin{enumerate}
\sphinxsetlistlabels{\arabic}{enumi}{enumii}{}{.}%
\item {} 
产品有了,但本身无法规模化。(寻求模式突破来解决)

\item {} 
没有一个相对大、不断增长的市场,导致这事儿只是个小生意,不是个大事业。(当然,“做大”是一种选择,“小而美”也是一种选择,只不过你想选哪种得先想清楚。)

\item {} 
产品和市场不匹配。比如在行,产品与市场的供需关系上出现了一个逻辑问题,即“一群有时间没钱的人,花钱买一群有钱没时间的人的时间”,这是不可能有很大增长的。(需要对行业做深入的分析研究)

\item {} 
做一件事,问题与解决方案是必须匹配上的。但是如果你觉得小而美也挺好的话,追求产品与市场的匹配(PMF
和增长)就并不是必须的。

\end{enumerate}


\subparagraph{长青假设:定位与资源的匹配}
\label{\detokenize{chapter_introduction/Product:id36}}
如果你做到了产品与市场的匹配(PMF
达到),那就算找到了一个自己公司团队的定位,也就是 PRF 的
P,Positioning,下一步就是达成 PRF,完成定位与资源的匹配来扩大战果。

这一部分,就超出了单一产品的范畴,不在四轮 MVP
框架里了。这里面也有几种常犯的错误:
\begin{enumerate}
\sphinxsetlistlabels{\arabic}{enumi}{enumii}{}{.}%
\item {} 
定位不可持续。定位是公司立身之本,即“使命、愿景、价值观”,是公司早期靠着创始团队、产品、用户之间的反复互动,逐渐打磨清晰的,它给我们的后续产品指明了大方向。如果你的定位是“最好的马车公司”,那汽车时代来临时,你该怎么办?

\item {} 
资源没能积累。随着公司、产品、用户的协同发展,应该要有某种资源像雪球一样越滚越大,形成自己的增长飞轮。比如用户越来越多,成交就越来越多,对商家的议价能力就越来越强,商品价格越来越便宜,用户就越来越多,完成闭环。这就是一个典型的增长飞轮。而有不少公司,除了不断赚点钱,没能积累下什么。

\item {} 
定位和资源不匹配。这一点阿里做得不错,使命是“让天下没有难做的生意”,重要资源是不断积累的数据,数据可以帮助生意做得更好。

\item {} 
如果成功达成了定位和资源的匹配,那我们就可以说,公司有了一个很好的产品矩阵。

\end{enumerate}


\subparagraph{单一产品在矩阵中的评价}
\label{\detokenize{chapter_introduction/Product:id37}}
矩阵中的任何一个产品,做得好的话,都要考虑和其他众多产品的关系,都要求该产品满足三个条件:可复用、能积累、善生死。
\begin{enumerate}
\sphinxsetlistlabels{\arabic}{enumi}{enumii}{}{.}%
\item {} 
可复用:就是说可以复用公司的积累,比如供应链、比如数据沉淀、比如已有用户。如果不能复用的话,你推出的第二个产品和众多竞争对手相比,就没有任何优势。

\item {} 
能积累:意味着后续产品可以为公司积累将来可复用的资源,好产品应该让整体更优,而不是单纯地消耗公司的积累。

\item {} 
善生死:说的是要有合理的生命周期管理。每一个产品,都要在该进入的时候进入,该退出的时候退出。一个公司和一个生态系统一样,资源都是有限的,有时候死亡(即释放资源)可以创造巨大的价值。

\end{enumerate}


\paragraph{产品成功 6\sphinxfootnotemark[9]}
\label{\detokenize{chapter_introduction/Product:id38}}%
\begin{footnotetext}[9]\sphinxAtStartFootnote
\sphinxnolinkurl{https://www.jianshu.com/p/111d9fcc005e?utm\_campaign=maleskine\&utm\_content=note\&utm\_medium=seo\_notes\&utm\_source=recommendation}
%
\end{footnotetext}\ignorespaces 
产品设计、竞争策略、全局商战

底层思维:整体式设计是指不能单点极致,需要整体开花;用户决策思维意味着产品设计要关注用户决策,而不是一味追求用户体验;价值本位模型是指产品设计要围绕核心价值展开,流量圈养的互联网思维并不适用。

产品创新(微观层):探索——发展——成熟的三个阶段,AI+硬件的模式:硬件赋能模式和互联网管道模式(智能音箱背后的语音平台、内容服务等)。

竞争态势(中观层,竞争产品阵营当下的格局和未来的势头):对抗(巨头争霸)、割据(多品牌分散)、创新(率先进入新领域)、延伸(大生态中延伸小生态)

商战全局(宏观层):以产品为根基,表现为价值驱动(追求先进性,如大疆,激进)、认知驱动(追求差异性,如oppo,后发制人)、购买驱动(追求经济性,如小米)三种类型。


\subsubsection{To B 1\sphinxfootnotemark[10]}
\label{\detokenize{chapter_introduction/2B:to-b-1}}\label{\detokenize{chapter_introduction/2B::doc}}%
\begin{footnotetext}[10]\sphinxAtStartFootnote
\sphinxnolinkurl{https://tanxianlian.com/2020/03/07/\%e6\%88\%91\%e7\%9a\%84to-b\%e4\%ba\%a7\%e5\%93\%81\%e6\%96\%b9\%e6\%b3\%95\%e8\%ae\%ba/}
%
\end{footnotetext}\ignorespaces 

\paragraph{如何定义B端或C端产品}
\label{\detokenize{chapter_introduction/2B:bc}}
一个产品属于B端还是C端,取决于这个产品究竟在解决什么样的问题,而不在于产品究竟会有什么样的功能。例如,IM即时通信,通常被理解为C端产品的功能,然而这个功能在某些场景下也可以被认为是B端产品的功能,如微信是典型的C端产品,但并不妨碍它发展为B端产品,企业微信便应运而生,你不能简单地说企业微信还是C端产品。
\sphinxhref{https://weread.qq.com/web/reader/40632860719ad5bb4060856k9a132c802349a1158154a83}{14}%
\begin{footnote}[11]\sphinxAtStartFootnote
\sphinxnolinkurl{https://weread.qq.com/web/reader/40632860719ad5bb4060856k9a132c802349a1158154a83}
%
\end{footnote}


\paragraph{定义}
\label{\detokenize{chapter_introduction/2B:id1}}
To
B产品主要是面向企业的软件系统产品,例如企业的ERP系统、OA系统以及线上型业务的用户系统、订单系统等。它是一个概称,既可以指单个系统,也可以指代某一个集合体系,例如一整套的解决方案。

核心着力点是:提高企业生产、销售、组织等活动的效率。

通用性不高,难参考难复制每个企业以及企业内部的各需求部门业务都不同。

一般 To B
往往的产品往往是行业内技术或者测试出身转岗更为合适,业务的熟悉程度要求高一些。
\sphinxhref{https://m.zhipin.com/mpa/html/get/share?type=4\&contentId=8eaf00b18d9c5148tnVy2t-9GVI~\&uid=5885ce18425348b00nR73NS6E1FX\&identity=0}{3}%
\begin{footnote}[12]\sphinxAtStartFootnote
\sphinxnolinkurl{https://m.zhipin.com/mpa/html/get/share?type=4\&contentId=8eaf00b18d9c5148tnVy2t-9GVI~\&uid=5885ce18425348b00nR73NS6E1FX\&identity=0}
%
\end{footnote}

针对2B产品,部门流程改变,组织架构调整,工作流程优化等
\sphinxhref{http://www.woshipm.com/pmd/1792966.html}{4}%
\begin{footnote}[13]\sphinxAtStartFootnote
\sphinxnolinkurl{http://www.woshipm.com/pmd/1792966.html}
%
\end{footnote} 产品经理站在 IT
专业上,有更多的话语权来建议业务部门产品应该如何搭建。
\sphinxhref{https://www.yuque.com/weis/pm/wkixxq}{11}%
\begin{footnote}[14]\sphinxAtStartFootnote
\sphinxnolinkurl{https://www.yuque.com/weis/pm/wkixxq}
%
\end{footnote} 收益难以量化

其产品设计的逻辑是“重流程规则、轻体验”
\sphinxhref{https://www.aiyingli.com/74015.html}{10}%
\begin{footnote}[15]\sphinxAtStartFootnote
\sphinxnolinkurl{https://www.aiyingli.com/74015.html}
%
\end{footnote}


\paragraph{用户 6\sphinxfootnotemark[16]}
\label{\detokenize{chapter_introduction/2B:id2}}%
\begin{footnotetext}[16]\sphinxAtStartFootnote
\sphinxnolinkurl{http://www.pmtalk.club/\#/article/detail/6375}
%
\end{footnotetext}\ignorespaces 
B端产品的用户较为理性;决策者以及关键干系人,会从好用、性价比、提高效率、适配公司情况等多个维度来进行综合考虑和筛选,最后选择最为合适的产品来使用。

B端产品较为复杂,并且与日常生活相关不大,多数为垂直行业属性打造,市场上少见,所以一般需要用户花费一定时间来进行学习

角色分工:B端产品至少会分为决策者、管理者、普通员工。

G端(To
Goverment)产品,是比B端产品更B端的产品,产品经理其它能力的权重更低,找到关键決策人利益诉求的权重更高(可能是隐性的,需要高难度能力)。\sphinxhref{https://zhuanlan.zhihu.com/p/127962653}{13}%
\begin{footnote}[17]\sphinxAtStartFootnote
\sphinxnolinkurl{https://zhuanlan.zhihu.com/p/127962653}
%
\end{footnote}


\paragraph{产品本质}
\label{\detokenize{chapter_introduction/2B:id3}}
To
B产品的本质是效率(生产力)工具,不管是服务于企业的采购、行政、人事、成品/库存管理、销售,本质上都是服务于提高企业的效率。这点在产品的策划、打磨、生产等各阶段都是考虑的重点。


\paragraph{过程 9\sphinxfootnotemark[18]}
\label{\detokenize{chapter_introduction/2B:id4}}%
\begin{footnotetext}[18]\sphinxAtStartFootnote
\sphinxnolinkurl{https://zhiya360.com/50903.html}
%
\end{footnotetext}\ignorespaces 
以B端产品为例,包含需求调研,竞品分析,产品规划,产品设计,跟进开发,测试上线,售前推广,客户部署,培训指导,售后跟踪,迭代优化等阶段。


\paragraph{产品分类 16\sphinxfootnotemark[19]}
\label{\detokenize{chapter_introduction/2B:id5}}%
\begin{footnotetext}[19]\sphinxAtStartFootnote
\sphinxnolinkurl{https://www.jianshu.com/p/b159b89df3f8}
%
\end{footnotetext}\ignorespaces \begin{enumerate}
\sphinxsetlistlabels{\arabic}{enumi}{enumii}{}{.}%
\item {} 
放大信息类产品
信息类平台1688、hc360等大型B2B电商平台,通过互联网的流通和放大效应,降低信息获取者的获取成本,降低信息发声者的推广成本,从而放大信息的价值;

\item {} 
提高工作协同效率产品
钉钉。OA、jira、teambition、点餐系统等皆是促进信息传递,提升企业工作协作效率。通过交互设计的手段,将不同的企业信息、任务流聚集到有效的产品之中。

\item {} 
降低固有成本类产品
智能客服系统、智能售票、财务管理软件等系统,是降低企业固有成本类产品。

\item {} 
数据挖掘类产品
随着大数据的到来,企业也越来越注重大数据对企业运营的影响,大数据相关的产品,提供准确数据统计,更全、更准确,成为企业做出正确决策的参考依据;

\end{enumerate}


\paragraph{其他}
\label{\detokenize{chapter_introduction/2B:id6}}

\subparagraph{业务渠道 18\sphinxfootnotemark[20]}
\label{\detokenize{chapter_introduction/2B:id7}}%
\begin{footnotetext}[20]\sphinxAtStartFootnote
\sphinxnolinkurl{http://reader.epubee.com/books/mobile/12/1240b863fa87878a6e1899147685e374/text00000.html}
%
\end{footnotetext}\ignorespaces 
2B业务渠道分为线下渠道和线上渠道,而线下渠道分为经销渠道、KA渠道、企业大客户渠道,线上

渠道分为1688、零售通、新通路、找钢网、找煤网、找塑料网、企业自建的B2B在线订货渠道等。


\subparagraph{收费模式}
\label{\detokenize{chapter_introduction/2B:id8}}
传统的to
B产品大多是本地部署,一次收费,后续的维护、更新等服务,按次收费。

企业外部的B端产品:平台型企业给卖家提供运营管理支持的系统

随着云计算的兴盛,Saas(Software as a
service)服务也随之兴起,云端部署+按年收费的模式开始逐渐成为主流。

经过国内外的众多实践,证明Saas云端服务+按年收费的模式有众多的优点,也是更可行有效,更能实现服务更优、利润最大化的方式。


\subparagraph{部署方式}
\label{\detokenize{chapter_introduction/2B:id9}}\begin{itemize}
\item {} 
私有化部署:软件部署在自己的IDC以及主机和存储设备中,与外网隔离

\item {} 
云部署:软件部署在第三方云服务商

\end{itemize}


\subparagraph{技术架构}
\label{\detokenize{chapter_introduction/2B:id10}}\begin{itemize}
\item {} 
B/S 更优

\item {} 
C/S

\end{itemize}


\subparagraph{业务方向:}
\label{\detokenize{chapter_introduction/2B:id11}}\begin{itemize}
\item {} 
业务支持类:企业经营管理或核心业务开展(CRM、仓配系统)

\item {} 
办公协同类:企业内部协同办公(OA office automation、HRM)

\item {} 
商家端管理:商家前台/后台/商家管理

\end{itemize}


\paragraph{迭代模式:稳定 or 常变?}
\label{\detokenize{chapter_introduction/2B:or}}
企业用户的业务在一定时间内具有连续性,因此需求也存在一定时间的延续性。在操作体验上,企业用户并不看重趣味性、更在乎便利性,因此在操作上也会形成惯性路径,即使用习惯。

因此,企业用户希望to B的产品具有一定稳定性。

但业务和需求始终都还是会有变化的,不可能始终不变,因此to
B的产品还是要保持一定的迭代节奏,只不过相比to
C产品,迭代的周期要更长,以及基于前述的原因,迭代要更多基于优化而非大改,不然就使自身丧失了当初的立身基础。


\paragraph{发展路径}
\label{\detokenize{chapter_introduction/2B:id12}}
\sphinxstyleemphasis{第一阶段:内部效率工具}

该阶段是To
B产品的创生阶段,面向的用户主要是企业内部的使用者,产品的生产者是卖方,使用者是买方,产品的被使用就能直接或间接地为企业提高生产力,使产品有存续的价值和空间。

该阶段,因为面向的用户主要是企业内部的使用者,并且产品的生产者是卖方、使用者是买方的关系,因此,产品通常是免费的。

\sphinxstyleemphasis{第二阶段:内部商业化}

在很多大型企业,例如集团公司,或者是BU结构的公司,会实行内部成本核算。

内部的效率工具经由内部成本核算,实现的是内部商业化。

企业内部的中后台系统大多都属于前面的两个阶段。

这两个阶段的to B产品有两个关键词:有限内部竞争、行政+利益驱动 。

具体来说,大公司内部可能会有多个团队进行内部竞争,开发相同的产品,以及主要靠行政命令以及利益联合作为产品推广的驱动力。

\sphinxstyleemphasis{第三阶段:外部商业化}

该阶段的产品较少。

一是外部商业化的产品,因为面向外部市场,市场化对产品本身的要求会更高;

二是to
B产品的功能和架构和企业的组织结构及业务体系是适配的,因此从内部转变为外部产品的时候,在产品架构及功能体系方面,会有很大的不同;

三是因为是面向的企业增多,彼此需求并不一致,因此需要面对更高的复杂性。


\paragraph{突破点}
\label{\detokenize{chapter_introduction/2B:id13}}
宏观上,要更多地依靠生态体系,或者联盟合作,来进行市场拓展。

例如,某销售型企业需要整套的企业在线化解决方案,公司A主打产品是销售Saas系统,并且是行业最佳,但该客户还有财务、行政Saas系统的需求。

客户担心如果选用了不同服务方的不同产品,体系割裂,数据及账号权限体系不统一,并且也不便于地实现多系统的集成,所以不愿意单独选用公司A的销售Saas产品。

如果有公司B刚好能提供该客户剩余需求的财务及行政系统,公司A和公司B合作,对各自产品进行集合,打通数据及账号权限体系,打包提供给该客户,就可以提升公司A和公司B彼此的交易成功率及市场空间。


\subparagraph{权限设计 7\sphinxfootnotemark[21]}
\label{\detokenize{chapter_introduction/2B:id14}}%
\begin{footnotetext}[21]\sphinxAtStartFootnote
\sphinxnolinkurl{https://github.com/JoJoDU/Book\_Notes/issues/2}
%
\end{footnotetext}\ignorespaces 

\subparagraph{权限表}
\label{\detokenize{chapter_introduction/2B:id15}}

\begin{savenotes}\sphinxattablestart
\centering
\begin{tabulary}{\linewidth}[t]{|T|T|T|T|}
\hline
\sphinxstyletheadfamily 
一级导航
&\sphinxstyletheadfamily 
页面
&\sphinxstyletheadfamily 
页面元素
&\sphinxstyletheadfamily 
角色1
\\
\hline
客户管理
&
门店列表
&
“编辑”按钮
&
√
\\
\hline
\end{tabulary}
\par
\sphinxattableend\end{savenotes}


\subparagraph{RBAC(role based access control)权限模型}
\label{\detokenize{chapter_introduction/2B:rbac-role-based-access-control}}
ER模型:用户、角色、用户组


\subparagraph{数据权限:各个角色能看到的数据范围}
\label{\detokenize{chapter_introduction/2B:id16}}
机构树 数据范围是当前节点及其子节点 客户地区


\paragraph{深耕细作}
\label{\detokenize{chapter_introduction/2B:id17}}
在IT行业内,很多做TO
B产品的公司是可以发展很久的,比如IBM、微软等。\sphinxhref{https://www.epubit.com/onlineEbookReader?id=0dc0f81254b5455c892a7896d0f7d0ac\&pid=9821123a37484750b6317c8c1c217500\&isFalls=true}{8}%
\begin{footnote}[22]\sphinxAtStartFootnote
\sphinxnolinkurl{https://www.epubit.com/onlineEbookReader?id=0dc0f81254b5455c892a7896d0f7d0ac\&pid=9821123a37484750b6317c8c1c217500\&isFalls=true}
%
\end{footnote}

To
B产品更重要的是对商业模式的经营和核心功能的打磨。一旦占据了市场领先地位,将比较难被替代,试想一个公司的CRM系统被替代需要付出多少的代价?先要把数据转移,然后还需要适配各个系统。

在前面产品核心竞争力的章节也提到过,ToB产品提供给用户的更多的是服务,服务包含售前、售后、文档、产品功能等多个方面,建立这一套完整的体系是需要经历很长时间打磨的,所以做ToB的产品经理要耐得住性子点点地打磨产品才有可能得到市场的认可。

对于 To B
来说,潜在用户一共就那么多,这里舍弃点、那里舍弃点,你还有多少用户?你还做个毛线?所以必须深耕细作,争取把行业通吃,toB
里面赢家通吃是很常见的。

深耕细作依赖行业理解。如果你没有参与过销售管理,你就很难明白为什么 CRM
里需要那么复杂的销售线索分配机制。

然而现在的互联网产品人,大多一毕业就进入互联网圈,没有接触行业一线的机会,也不愿意去了解。互联网来钱太容易,PM
都干不了脏活。不信你问问身边的,有几个敢去主动给用户打电话?

而那些在行业里经验丰富的人呢?互联网公司嫌弃他们又土又穷、不懂互联网,很少给他们转业的机会。这些人因为专业、技能、经验和学历的原因,不太容易进入互联网行业;即便进入了,也不可能担任重要角色。可以说很大一部分想法和创新都被封闭和埋没在了领域内部。

这么说肯定有点太抬高领域人才而贬低 PM
们了。事实上你让一个行业大佬来做互联网,大概率难有起色。无讼的创始人是全国顶级律师,产品一坨屎;iCourt
创始人是搞律师培训的,产品年收入破亿。toB
产品人需要把互联网和行业知识相结合,打造完整的产品研发和服务团队。有这能力的人,凤毛麟角。


\subparagraph{建立产品服务体系}
\label{\detokenize{chapter_introduction/2B:id18}}
建立产品服务体系是TOB产品与ToC产品的一大区别。在商业化服务场景下,光有孤零零的产品功能是无法跟客户需求匹配的,需要有一系列使用帮助教程。其中产品经理的主要工作是输出整个产品的功能说明文档,要细致到每个按钮。以作者参与的机器学习平台产品为例,单是功能介绍文档就有将近4万字。这些说明文档需要不断地随着产品功能的更新而更新,所以文档工作通常会占用产品经理大量的精力。另外,针对部分比较难以上手的产品,建议要录制使用视频,以视频解说的方式介绍产品的功能。视频教程也是目前人工智能ToB领域比较普遍的功能介绍方式。根据作者的工作经验,录制视频教程的效果会优于文档。

除了功能介绍文档等相关材料的开发工作,服务体系的建立依赖于许多支持团队的合作,产品经理在其中的角色是沟通和协调,将整个售前和售后链路打通。比如产品经理需要给售后团队明确的SLA准则(SLA指的是售后服务保障),并且培训售后团队,使售后团队在遇到用户索赔和追责的时候可以快速处理问题。在售前方面,产品经理也要协调各个售前工程师和销售团队,给前方团队输出与产品售卖相关的商业指导书,扫清产品售卖工作的障碍。

在产品对外服务的过程中,产品经理是整个体系的接口人,任何售前售后、开发端出现问题都会与产品经理联系,所以在各个团队之间的沟通和协调工作会占据很大的一部分精力。


\paragraph{产品路标规划:干系人关键问题拆解法(2B产品)4\sphinxfootnotemark[23]}
\label{\detokenize{chapter_introduction/2B:b-4}}%
\begin{footnotetext}[23]\sphinxAtStartFootnote
\sphinxnolinkurl{http://www.woshipm.com/pmd/1792966.html}
%
\end{footnotetext}\ignorespaces 
针对2B产品时,产品规划的核心往往是解决各干系人的问题,围绕着产品核心价值路径,不断汇总并提出问题。沿着客户路径,不断的去分解他们的问题,同时要寻找到解决方案。2B类产品的规划就是将各种问题和解决方案进行汇总,然后按照优先级进行罗列,最终形成产品路线图。(有点像需求优先级的判断)

首先要明确产品的核心目标,在该目标的基础上,我们自己要先拆解出几个子问题,比如涉及哪些业务部门?涉及哪些职位?怎样使用产品?使用场景是什么?等。

接下来,可以在以上问题的基础上,做各部门干系人的访谈,继续获得更细节的问题,比如部门的对接人是谁?部门需要得到什么服务支持?部门需要提供什么服务?哪个部门的需求最紧急等等。

实际工作中我们可能会分解出很多的问题,在此基础上,划分好优先级,形成一个在哪个阶段使用什么方式解决哪些干系人的什么问题的产品规划方案。


\paragraph{MVP基本原则 17\sphinxfootnotemark[24]}
\label{\detokenize{chapter_introduction/2B:mvp-17}}%
\begin{footnotetext}[24]\sphinxAtStartFootnote
\sphinxnolinkurl{https://www.niaogebiji.com/article-31885-1.html}
%
\end{footnotetext}\ignorespaces \begin{itemize}
\item {} 
突出优势:基于企业自身当前的能力优势

\item {} 
先易后难:从简单的功能开始

\item {} 
内外兼顾:有大局观,进行通盘考虑。

\end{itemize}


\paragraph{原型设计要求 5\sphinxfootnotemark[25]}
\label{\detokenize{chapter_introduction/2B:id19}}%
\begin{footnotetext}[25]\sphinxAtStartFootnote
\sphinxnolinkurl{http://www.woshipm.com/pmd/3755958.html}
%
\end{footnotetext}\ignorespaces 
对原型能力要求没那么高,基本就是一个打辅助的作用,来解释需求文档(以前我都是画个demo后直接找UI小姐姐\textasciitilde{})


\paragraph{项目管理}
\label{\detokenize{chapter_introduction/2B:id20}}
项目管理保证软件开发按计划推进、落地,保障团队的产品研发效率与质量

\begin{figure}[H]
\centering
\capstart

\noindent\sphinxincludegraphics{{project_manage}.jpg}
\caption{标准项目流程}\label{\detokenize{chapter_introduction/2B:id27}}\end{figure}


\subparagraph{工作重点}
\label{\detokenize{chapter_introduction/2B:id21}}\begin{itemize}
\item {} 
设计并优化项目管理制度:合理的规范制度可以约束产品团队行为也可以保护产品团队的权益
比如要求业务部门提交需求时提交BRD

\item {} 
负责大中型项目的立项实施

\end{itemize}


\subparagraph{如何把控项目进度}
\label{\detokenize{chapter_introduction/2B:id22}}\begin{itemize}
\item {} 
细化工作,明确交付 工作拆解,明确细化是想的负责人、交付物、时间点

\item {} 
通过机制把控进度

\end{itemize}
\begin{enumerate}
\sphinxsetlistlabels{\arabic}{enumi}{enumii}{}{.}%
\item {} 
开展定期会议:聚合项目各方人员,回顾上次会议以来的进展、遇到的苦难、下一次会议前的计划

\item {} 
每日站会

\item {} 
日报、周报:通报进展、警示风险

\end{enumerate}
\begin{itemize}
\item {} 
编写内容清晰的日报或周报
管理项目、通报进展;争取关注度和资源,解决项目中遇到的问题

\end{itemize}
\begin{enumerate}
\sphinxsetlistlabels{\arabic}{enumi}{enumii}{}{.}%
\item {} 
本周进度

\item {} 
项目风险

\item {} 
下周计划

\item {} 
整体进度

\end{enumerate}
\begin{itemize}
\item {} 
保持责任心

\end{itemize}


\paragraph{运营管理}
\label{\detokenize{chapter_introduction/2B:id23}}

\subparagraph{产品运营岗}
\label{\detokenize{chapter_introduction/2B:id24}}
SaaS:偏销售、BD职能 双边市场攻击端:商家、店铺运营,偏C端运营
内部业务系统(以下讨论方向)


\subparagraph{工作内容}
\label{\detokenize{chapter_introduction/2B:id25}}
工作目标:挖掘B端产品能力(现有功能推广、协助完成产品升级优化),帮助其余人解决业务问题(营收增长、风险控制)
\begin{itemize}
\item {} 
产品功能推广培训:线上推广宣传(消息推送、公告通知);现场培训(复杂升级改造)

\item {} 
问题解答处理:初上线的系统,组织试点用户群,搜集问题;解答迅速有效;总结共性问题,以便产品进行系统优化

\item {} 
需求采集过滤:收集一线业务人员的直接诉求,挖掘到真正会产生影响的需求,和PM持续优化产品

\item {} 
项目效果分析:对上线功能进行持续的数据分析和观察;作为中立方,考核项目效果和收益,给出客观分析

\item {} 
业务诊断分析:诊断业务,分析问题,提出解决方案

\end{itemize}


\subparagraph{业务运营岗}
\label{\detokenize{chapter_introduction/2B:id26}}\begin{itemize}
\item {} 
业务支持:审批、核对、检验

\item {} 
流程管理:保证分支机构管理的规范性和可靠性

\item {} 
策略制订:促销策略、定价策略、供应商返点策略、仓储排班策略

\item {} 
绩效考核制度制订:自顶向下

\item {} 
培训考核

\item {} 
项目管理

\item {} 
合规质检

\item {} 
数据分析

\end{itemize}


\paragraph{Buyer和User的区别}
\label{\detokenize{chapter_introduction/2B:buyeruser}}
产品经理在设计功能的时候一定要区分这个功能是提供给客户(
Buyer)还是用户(User)的,
Buyer指的是实际为产品付费的人,User指的是产品的实际使用用户。

对于ToB产品来讲, Buyer和User往往在企业是不同的角色!


\subparagraph{Buyer是决策链路的核心}
\label{\detokenize{chapter_introduction/2B:buyer}}
通常决定是否购买一款产品的人是公司的CTO或者CEO,决定购买的人是产品的客户,CTO和CEO更关注产品使用过程中的消耗以及是否能节约人力。也就是说无论是产品设计还是最终产品的营销策略,核心的问题是要提升Buyer的满意度,因为
Buyer是决定是否购买的最关键因素,User更多的是从使用层面去影响
Buyer如果想取得
Buyer的好感,首先要在售卖模式上做文章,产品的售卖是否能做到资源用量可控。比如大部分企业都是预算制,每年在某个部分的消费是提前规划好的,如果产品的售卖模式包含预付费(包年或包月)模式且包含按量付费模式,那么
Buyer在做资源预估的时候就会有更多余地。另外,CTO和CEO很关注产品在使用过程中的效果和消耗,也就是俗称的投入产出比。
很多ToB产品都会为客户设计一个看板用来观察产品的实时具体价值,这些产品的设计都是对
Buyer友好的。


\subparagraph{User决定了产品的业务深度}
\label{\detokenize{chapter_introduction/2B:user}}
既然
Buyer是决定产品购买链路最核心的因素,那么User的体验是否就不重要了?显然不是。让User体验感好,是一个产品能否在一家客户做得更深入的关键。User是产品的实际长期使用者,也是产品后期付费的推动者。
如果User验证了产品功能确实能提升自己的效率,自然会给
Buyer提供一个针对产品的正向反馈,这种反馈是产品后期能否得到续费的关键。
其实产品绝大部分的功能是要针对User设计的,提升User好感的方式也有很多种,比如在User使用产品的整个链路上,ToB产品往往会增加很多文档类的引导,目的就是提升User的好感。很多ToB产品也会把User和Buyer的使用路径通过权限做隔离,
Buyer会看到更多与产品报表相关的内容,而User则更多地看到产品功能性的内容。


\subparagraph{产品购买链路中User和Buyer之间的矛盾}
\label{\detokenize{chapter_introduction/2B:userbuyer}}
User受雇于Buyer,那么在购买决策链路中,他们之间是否也会存在矛盾呢。在许多TB产品的场景下,User和
Buyer之间是有一定矛盾的,比如人工智能算法平台这样的产品,目标客户的
Buyer一般是互联网公司的CTO,User是算法工程师。算法工程师在公司中的使命一般是开发和使用算法去解决诸如智能推荐或智能风控这样的业务问题。如果
Buyer买了算法平台这样的产品,某种意义上会替代原先算法团队的工作,这是否意味着User的工作量小了,团队价值也就没有以前那么大了。所以为了同时满足User和
Buyer的需求,产品在设计和宣传时要注意不要一味地强调替代某些人的工作,而是要把产品功能的核心放到如何去提升他人工作的效率上,这一点对于PaS层的产品尤为重要。
以上是一些针对
Buyer和User不同的产品设计理念和营销方向的分析也是ToB产品和ToC产品的主要区别之一。


\paragraph{AI PM}
\label{\detokenize{chapter_introduction/2B:ai-pm}}
关注人工智能产品周期的第一和最后一英里。B2B公司为一小部分消费者解决非常复杂的问题。以安全为例:许多支持AI/
ml的安全公司只专注于应用威胁和异常检测。尽管它们服务的公司可能非常多样化,但提供这些人工智能产品的公司明确关注\sphinxstylestrong{一到两种产品类型}——这是消费者人工智能产品很少拥有的优势。

就商业模式而言,市面上传统toB的AI科技公司,大多倾向采用SaaS订阅模式提供AI服务,如书中所言,对甲方客户公司来说降低了采购门槛,同时也降低了乙方AI服务公司的签单难度,但增加了乙方的运营压力,服务标准化,继而规模化显得生死攸关。在国内市场环境下,服务标准化很理想,现实很骨感,每家甲方公司(尤其传统大公司)都有自己的管理特色和业务特色,若需要深入到甲方客户业务中,就做不到自己的产品标准化,更别说通过标品规模化降低单位成本。既要初心、又要资金,所以选择AI应用场景几乎决定了一家toB的AI科技公司的规模,也决定了个人未来职业发展的高度和宽度。

对企业而言,人工智能产品的目标就是提高企业生产力。人工智能技术通过替代企业中的劳动力提高劳动效率和延伸劳动资料这两种方式,提升企业的生产力。\sphinxhref{https://weread.qq.com/web/reader/0c032c9071dbddbc0c06459k70e32fb021170efdf2eca12}{15}%
\begin{footnote}[26]\sphinxAtStartFootnote
\sphinxnolinkurl{https://weread.qq.com/web/reader/0c032c9071dbddbc0c06459k70e32fb021170efdf2eca12}
%
\end{footnote}


\subsubsection{To C}
\label{\detokenize{chapter_introduction/2C:to-c}}\label{\detokenize{chapter_introduction/2C::doc}}

\paragraph{定义}
\label{\detokenize{chapter_introduction/2C:id1}}
C端产品为满足个人服务,满足民生生活,以提供便捷、满足兴趣、欲望、社交、工具的需求为主,不直接为产品带来利益,基本都是免费的。但是C端产品真正所销售的是其用户群体,用户注意力,用户时间、买单方式客户或广告主;


\paragraph{方法论}
\label{\detokenize{chapter_introduction/2C:id2}}
To
C产品有众多方法论,从PC互联网到移动互联网也经历了一些演变,但大体都围绕着“用户体验”(User
Experience,简称UE/UX)阐述,百度百科上对这个词的解释是——用户在使用产品过程中建立起来的一种纯主观感受。

围绕“用户体验”,聚焦的维度通常包括内容、功能、交互、视觉等。数据驱动设计,收益可量化(UV,PV,日活,转化率),运营很重要。\sphinxhref{https://github.com/JoJoDU/Book\_Notes/issues/2}{8}%
\begin{footnote}[27]\sphinxAtStartFootnote
\sphinxnolinkurl{https://github.com/JoJoDU/Book\_Notes/issues/2}
%
\end{footnote}

但我觉得这样的方法论是比较单薄的。在产品的设计和开发之外,还应该基于“产品生命周期”的视角,将产品的运营容纳进整体的方法论之中。通用性高可参考可复制。
\sphinxhref{https://m.zhipin.com/mpa/html/get/share?type=4\&contentId=8eaf00b18d9c5148tnVy2t-9GVI~\&uid=5885ce18425348b00nR73NS6E1FX\&identity=0}{3}%
\begin{footnote}[28]\sphinxAtStartFootnote
\sphinxnolinkurl{https://m.zhipin.com/mpa/html/get/share?type=4\&contentId=8eaf00b18d9c5148tnVy2t-9GVI~\&uid=5885ce18425348b00nR73NS6E1FX\&identity=0}
%
\end{footnote}


\paragraph{按功能分类 8\sphinxfootnotemark[29]}
\label{\detokenize{chapter_introduction/2C:id3}}%
\begin{footnotetext}[29]\sphinxAtStartFootnote
\sphinxnolinkurl{https://github.com/JoJoDU/Book\_Notes/issues/2}
%
\end{footnotetext}\ignorespaces \begin{itemize}
\item {} 
工具类:独立功能解决具体需求

\item {} 
内容类:OGC(occupationally职业生产内容)、PGC(professionally专业)、UGC(user用户)

\item {} 
贡献:全新的销售渠道

\item {} 
社交类

\item {} 
平台类

\end{itemize}


\paragraph{业务渠道 14\sphinxfootnotemark[30]}
\label{\detokenize{chapter_introduction/2C:id4}}%
\begin{footnotetext}[30]\sphinxAtStartFootnote
\sphinxnolinkurl{http://reader.epubee.com/books/mobile/12/1240b863fa87878a6e1899147685e374/text00000.html}
%
\end{footnotetext}\ignorespaces 
2C业务渠道分为线下渠道、电商渠道、创新渠道,如图8\sphinxhyphen{}5所示。

线下渠道按经营主体可分为直营门店、加盟门店,按门店位置可分为商场专柜、社区店、奥莱店、

工厂折扣店、街边专卖店。

电商渠道按平台可分为天猫、淘宝、京东、苏宁、唯品会、亚马逊、拼多多、有赞、微盟,及自建

官方商城微商城、小程序、App等。

创新渠道分为社交电商和内容电商。社交电商如拼多多、抖音、今日头条、环球、微信、微博等,

内容电商如小红书、礼物说等。


\paragraph{价值 13\sphinxfootnotemark[31]}
\label{\detokenize{chapter_introduction/2C:id5}}%
\begin{footnotetext}[31]\sphinxAtStartFootnote
\sphinxnolinkurl{https://www.jianshu.com/p/b159b89df3f8}
%
\end{footnotetext}\ignorespaces 
通过获取更多用户的注意力,故C端产品的用户行为涉及核心在于\sphinxstylestrong{塑造用户行为},C端产品设计师通过交互设计、视觉设计的方式影响用户注意力,终极目标是使用户的使用行为尽在掌控之中,而不是产品去辅助用户行为;


\paragraph{用户 6\sphinxfootnotemark[32]}
\label{\detokenize{chapter_introduction/2C:id6}}%
\begin{footnotetext}[32]\sphinxAtStartFootnote
\sphinxnolinkurl{http://www.pmtalk.club/\#/article/detail/6375}
%
\end{footnotetext}\ignorespaces 
C端产品的用户比较感性;往往一款产品可以满足某个需求,或者让自己兴奋,则会直接使用。

C端产品比较重视用户体验,往往以一两个核心功能来撬动用户,希望用户能够以最短路径和时间达到“兴奋”点;

角色分工:角色单一,用户角色高度集中


\paragraph{Buyer和User是同一个人}
\label{\detokenize{chapter_introduction/2C:buyeruser}}
对于ToC产品来讲,
Buyer和User是同一个人,例如手游App,实际参与使用App的用户也是最终充值消费的用户,所以在设计ToC产品的时候,不太需要把每个功能和付费途径做拆解,只要保证用户在具体使用产品的过程中可以顺畅地进行消费即可


\paragraph{C端产品思维 11\sphinxfootnotemark[33]}
\label{\detokenize{chapter_introduction/2C:c-11}}%
\begin{footnotetext}[33]\sphinxAtStartFootnote
\sphinxnolinkurl{https://weread.qq.com/web/reader/40632860719ad5bb4060856k9a132c802349a1158154a83}
%
\end{footnotetext}\ignorespaces \begin{itemize}
\item {} 
用户思维:以用户为中心,认识产品的核心用户,明白运用人工智能是为了提升用户体验。

\item {} 
简约思维:不要在核心功能外画蛇添足,并不是所有功能加上人工智能就能成为好的功能。

\item {} 
极致思维:超越用户预期,人工智能可以赋予产品超越用户预期的功能。

\item {} 
迭代思维:小步快跑,人工智能有时不是那么完美,企业可以在小步快跑中达到目的。

\item {} 
社会化思维:用网络的方式完成分工与合作。许多需要模型学习的标注数据,同样可使用该方法来实现。

\item {} 
平台思维:建设开放、共享、共赢的平台,同时人工智能的能力输出实际上也能够为模型的不断进化提供支持。

\item {} 
跨界思维:要有大眼光,用多角度、多视野看待问题和提出解决方案,并结合人工智能的多项手段,提出综合解决方案。

\end{itemize}


\paragraph{C端产品的设计原则}
\label{\detokenize{chapter_introduction/2C:c}}
C端产品的设计十分注重细节,产品在方案设计过程中,在初期主要思考的是主体框架和流程,在后期主要注重产品细节的设计,以下是10大经典的产品设计原则,可供AI产品经理参考。
\begin{enumerate}
\sphinxsetlistlabels{\arabic}{enumi}{enumii}{}{.}%
\item {} 
状态可见或可知原则用户的任何操作,单击、滑动、按下按钮、语音唤醒等,产品应即时给出反馈。“即时”是指响应时间小于用户能忍受的等待时间,这个反馈可以是页面形式也可以是语音提示。

\item {} 
环境贴切原则产品的一切表现或表述,应该尽可能贴近用户所处的环境(年龄、学历、文化、时代背景)。系统所使用的词、短语应该是用户熟悉的概念,而不是系统术语。如个人助理中,机器与人的交谈会非常注重人性化的回复。

\item {} 
用户控制性与自由度原则不要替用户做决定,为了避免用户的误用、误碰,产品应该可以撤销或重复操作,在人工智能产品中,则可以用语音提示的方式告知用户。

\item {} 
一致性原则一致性不仅指产品中的用语、功能、操作、界面的一致,还包括产品应遵循行业规则,如智能音箱的唤醒就是一个通用操作。

\item {} 
防错原则在用户选择动作发生之前,就要防止用户有容易混淆或者错误的选择,比出现错误信息提示更好的是用更好的设计来防止此类问题发生,在语音交互系统中回声的消除、误唤起就要遵守防错原则。

\item {} 
易获取原则尽量减少用户对操作目标的记忆负荷,无论是操作动作还是选项都应该是可见的,而系统的使用说明应该是可见的或者是容易获取的。

\item {} 
灵活高效原则中级用户的数量远高于初级和高级用户的数量,这意味着企业需要为大多数用户设计,不要低估、也不可轻视,要保持灵活高效的产品设计原则。

\item {} 
审美与简约设计如果用户使用产品的习惯是浏览产品,一般动作不是读、不是看,而是浏览,那么界面就需要突出重点,弱化和剔除无关信息。如果用户是在一个无法看只能听的环境中使用产品,则需要设计简单的语音沟通功能。

\item {} 
容错原则容错指的是允许用户犯错,错误信息应该用语音表达,较准确地指出问题所在,并且提出一个用户可进行实际操作的解决方案。

\item {} 
人性化帮助原则系统帮助性提示包含①无须提示;②一次性提示;③常驻提示;④帮助文档;提示的方式可以是文本、图片和语音,如果系统不使用文档是最好的。系统提供的任何信息应当是容易去搜索的,并且专注于用户的任务,列出了具体的步骤。

\end{enumerate}


\paragraph{增长}
\label{\detokenize{chapter_introduction/2C:id7}}
根据具体的立足点,还可以把增长大概分为这样几类:用户增长、流量增长、交易增长。

用户增长,已经发展出了完整的方法论,例如AARRR模型:获取(Acquisition)\sphinxhyphen{}
激活(Activation)\sphinxhyphen{}(留存(Retention)\sphinxhyphen{} 收入(Revenue)\sphinxhyphen{}推荐(Refer)。


\paragraph{风口论}
\label{\detokenize{chapter_introduction/2C:id8}}
针对2C产品,遇到了市场新趋势,市场上竞品服务模式需要统一更改等
\sphinxhref{http://www.woshipm.com/pmd/1792966.html}{4}%
\begin{footnote}[34]\sphinxAtStartFootnote
\sphinxnolinkurl{http://www.woshipm.com/pmd/1792966.html}
%
\end{footnote}

大家平时讨论最多的都是 to C 互联网,听到最多的一个词是「风口」。为什么
to C 那么在意风口?因为 to C 强调创新和需求体量。

to C
爆发通常靠两点:更好地解决需求、创造新需求。这个过程需要不断试错,费时费力费钱。相比之下更聪明的做法肯定是抄作业、抢风口。

既然是抢风口,比的就是谁快。什么鸡巴精益创业、敏捷开发、弹性架构、人月神话,只要业务能跑起来、让运营去做增长,管你是
PHP、Python 还是易语言写出来的代码,能 Run
就行。而且初期系统挂的越多越好,挂的多说明你业务增长快,说明你火爆。越挂越有人想注册,去投资人那这理由还能加钱。

在这种氛围的长期熏陶下,to C
产品人越发重视细节、重视核心想法的表达、越发去抓大放小、越发忽略系统的顶层架构和长远战略。

另外由于 to C
病毒传播的可行性强,产品人会觉得只要发点优惠券烧钱、广告轰炸烧钱、做足微信传播,用户自然就能指数增长。当他们涉足
to B 领域时,发现这些套路根本不 Work。


\paragraph{过程 9\sphinxfootnotemark[35]}
\label{\detokenize{chapter_introduction/2C:id9}}%
\begin{footnotetext}[35]\sphinxAtStartFootnote
\sphinxnolinkurl{https://zhiya360.com/50903.html}
%
\end{footnotetext}\ignorespaces 
C端产品生命周期通常包含:需求调研、竞品分析,产品规划,产品设计、跟进开发、测试上线、冷启动期、运营推广,迭代优化等阶段。


\paragraph{产品路标规划:产品生命周期法(2C产品)4\sphinxfootnotemark[36]}
\label{\detokenize{chapter_introduction/2C:c-4}}%
\begin{footnotetext}[36]\sphinxAtStartFootnote
\sphinxnolinkurl{http://www.woshipm.com/pmd/1792966.html}
%
\end{footnotetext}\ignorespaces 
产品生命周期大家应该不会陌生,一款2C产品一般会经过引入期、成长期、成熟期和衰退期。而产品生命周期法则是按照产品不同的生命周期目标来制定产品规划。
\begin{enumerate}
\sphinxsetlistlabels{\arabic}{enumi}{enumii}{}{.}%
\item {} 
引入期,即产品MVP阶段,此阶段要用最小成本快速验证产品思路在目标用户群中的接收度,减少产品走错路的风险。在这个阶段产品规划时,就要考虑如何让目标用户快速了解和使用我们的产品,如何找到种子用户,如何快速获得用户的反馈,得到反馈后如何快速根据这些反馈进行产品迭代,如何初步的推广产品等,结合这些问题去规划该阶段的产品规划。

\end{enumerate}

冷启动通俗地说是指不通过大规模的市场推广,而是通过优质的内容或者熟人口碑传播进行产品启动的方法。冷启动可以有效地降低项目风险,但是启动速度比较慢。

冷启动的典型例子是知乎。知乎最开始就是周源凭借自己在互联网行业的人脉,以向专业人士发邀请码的形式邀请用户进行注册的,如李开复、徐小平、周鸿祎等人都是知乎的早期用户。这些人在知识的广博性及专业性上都远胜于普通用户,这与知乎“高质量知识分享社区”的定位吻合。反过来,这些人的站台,也为知乎后续长远的发展奠定了基础。知乎在2013
年才开放用户注册。

热启动,顾名思义,就是公司通过大量的资源(包含人力、资金等)投入让产品迅速启动,实现用户的爆发式增长,一般被大型公司采用。

热启动的典型例子是QQ 系的产品,如QQ 空间、QQ 邮箱等都是以QQ
为土壤迅速发展起来的。\sphinxhref{https://weread.qq.com/web/reader/8d232b60721a488e8d21e54k65132ca01b6512bd43d90e3}{7}%
\begin{footnote}[37]\sphinxAtStartFootnote
\sphinxnolinkurl{https://weread.qq.com/web/reader/8d232b60721a488e8d21e54k65132ca01b6512bd43d90e3}
%
\end{footnote}
\begin{enumerate}
\sphinxsetlistlabels{\arabic}{enumi}{enumii}{}{.}%
\setcounter{enumi}{1}
\item {} 
在产品的成长期,我们更关心的会是拉新(补贴、活动、邀请)和促进活跃,核心用户群要快速稳定地增长,同时也要保证一定的留存率。此时,产品规划的目标就是要考虑通过怎样的策略去实现上述目标,比如产品功能的优化、运营推广、产品性能优化(技术手段)等。产品优化:关注用户在每个核心页面的访问时长、核心页面的转化率及用户使用路径,不断提升产品用户体验;用户拉新和留存:每日新增用户数、次日留存率、7
日留存率、DAU(Daily Active User,日活跃用户数)、MAU(Monthly Active
User,月活跃用户数);推广:推广渠道数据,筛选出投入产出比最高的推广渠道并持续投入

\item {} 
在成熟期,产品活跃用户的增长会很缓慢,因为此阶段出现有大量的竞争对手,目标用户已被市场覆盖,或者是产品的模式等原因。在此阶段,产品规划应该去关注如何提升用户的转化率、如何提高产品的盈利能力,如果是产品自身的模式原因,就要去改善现有产品的服务、模式以及运营策略等,进一步提升产品的活跃用户。重点观测的数据指标:老用户留存率、老用户流失速度、每日新增用户数、新用户增长速度。

\item {} 
处于衰退期的产品,其实能够起死回生的几率不大,除非它的产品经理是卓越的领袖。重点观测的数据指标:每日用户流失数、用户流失速度、挽回效果数据。此阶段的规划可以尝试去发掘产品的第二春或颠覆式创新。此时如果要放弃产品,就要做好产品退出市场的相关工作。

\end{enumerate}

\begin{figure}[H]
\centering
\capstart

\noindent\sphinxincludegraphics{{product_diedai}.png}
\caption{产品迭代}\label{\detokenize{chapter_introduction/2C:id12}}\end{figure}


\paragraph{原型能力 5\sphinxfootnotemark[38]}
\label{\detokenize{chapter_introduction/2C:id10}}%
\begin{footnotetext}[38]\sphinxAtStartFootnote
\sphinxnolinkurl{http://www.woshipm.com/pmd/3755958.html}
%
\end{footnotetext}\ignorespaces 
C端的产品更重交互,所有对原型能力要求高一些,有的公司会要求产品画高保真设计图。


\paragraph{与产品价值相矛盾}
\label{\detokenize{chapter_introduction/2C:id11}}
C
端产品时常会遇到与产品价值相矛盾的情况,例如视频产品最核心的用户体验就是让用户不间断地看视频,但往往碍于公司生存压力,不得不在视频播放时插入广告金主的广告内容。


\paragraph{AI PM直接合作 10\sphinxfootnotemark[39]}
\label{\detokenize{chapter_introduction/2C:ai-pm-10}}%
\begin{footnotetext}[39]\sphinxAtStartFootnote
\sphinxnolinkurl{https://www.oreilly.com/radar/practical-skills-for-the-ai-product-manager/}
%
\end{footnotetext}\ignorespaces 
产品经理更有可能\sphinxstylestrong{直接}与功能团队合作,做更多客户驱动的工作。因为他们正在打造一款将被大众消费的人工智能产品,所以有可能(甚至是可取的)优化以实现快速实验和迭代的准确性


\paragraph{AI作用 12\sphinxfootnotemark[40]}
\label{\detokenize{chapter_introduction/2C:ai-12}}%
\begin{footnotetext}[40]\sphinxAtStartFootnote
\sphinxnolinkurl{https://weread.qq.com/web/reader/0c032c9071dbddbc0c06459k70e32fb021170efdf2eca12}
%
\end{footnotetext}\ignorespaces 
对个人端消费级产品而言,人工智能的意义在于将人类本身的感官和技能进行了技术形态的延伸。消费级产品的人工智能应用点一般集中在三个方面:一是信息采集;二是协助判断;三是协助处理。

人工智能产品是通过信息采集获取大量的数据,然后通过对数据训练得出适用性模型,接着将个人的信息数据作为输入并通过模型给出答案,因此人工智能产品必须拥有数据采集的能力,以便于进行智能化的判断。从智能穿戴设备到智能家居,从推荐引擎到预测系统,均需要通过各种传感器及输入设备获取数据。

消费级产品可以更好地帮助人类感知外部和自身的信息,能够帮助人们更为高效、快捷地处理信息。这种处理表现为两个方向:一是对个体的信息量化;二是对信息的处理进化。人们在使用一些人工智能产品的时候会发现,这些人工智能产品表现得越来越了解自己:你爱看什么类型的新闻,客户端就会给你推什么类型的新闻;你爱吃什么类型的菜,客户端就会给你推送什么类型的餐馆。原因就是这些人工智能产品已经通过采集及算法模型得出了一个量化数据的“你”。不仅如此,一些医疗保健和运动健身领域的产品通过心率、步频、身高、体重、速度、血压等信息的检测将个人信息更细致地量化。除自身信息的量化外,人们通过这些人工智能产品能够更好地感知和处理更多的数据,令信息处理能力大幅度提升。


\subsubsection{财务规划}
\label{\detokenize{chapter_introduction/money:id1}}\label{\detokenize{chapter_introduction/money::doc}}

\paragraph{产品财务规划 5\sphinxfootnotemark[41]}
\label{\detokenize{chapter_introduction/money:id2}}%
\begin{footnotetext}[41]\sphinxAtStartFootnote
\sphinxnolinkurl{http://www.woshipm.com/pmd/1792966.html}
%
\end{footnotetext}\ignorespaces 
产品财务规划是产品规划和商业模式的进一步落地,在进行产品的财务规划时,我们可以请求公司财务部门的协助,做好产品财务规划落地。

在财务规划阶段,我们要回答好如下问题,产品投入多大?成本如何?后续盈利模式如何?需要多久能够开始盈利?是否有进一步的可持续的盈利模式等。

在成本方面,要将资源分阶段投入,产品也要分迭代版本逐步退出,以降低企业的现金流压力。在产品规划中,加入产品各版本以及涉及到的资源投入,能够更好的说明产品成本信息。

在收益方面,除上述盈利模式等问题的考虑外,如果要将产品推出市场,还要考虑产品的定价策略。产品定价策略需要考虑市场本身的成熟度,客户关系积累程度以及产品本身的发展等多方面信息。


\paragraph{赚钱}
\label{\detokenize{chapter_introduction/money:id3}}
所谓“赚钱”,本质上即是:抓住他人欲望,满足他人需求,使之心甘情愿的为这份渴望而支出。

特斯联副总裁谢超告诉极客公园,在他看来,一旦研发开始投入,再想要掉转船头就很难了,「你一定要想好,未来的产出是什么」。\sphinxhref{https://tech.sina.com.cn/roll/2020-07-12/doc-iivhuipn2598506.shtml}{1}%
\begin{footnote}[42]\sphinxAtStartFootnote
\sphinxnolinkurl{https://tech.sina.com.cn/roll/2020-07-12/doc-iivhuipn2598506.shtml}
%
\end{footnote}

摆在特斯联面前的有两层选择,是做单品还是场景化解决方案,是做消费市场还是政府主导的公共事业市场。二乘二,一共有四条路可以走。做单品,可以做智能门锁等智能硬件;做场景化解决方案,则需要有对场景的深刻理解及部署能力。消费级市场和政府级市场的需求不同,企业的组织构建也就不同。

「我的原则是要用产品证明价值,让别人买单,而不是免费给别人用,这也是做为一个企业的本分」,服务金融客户的科技公司氪信的
CEO 朱明杰这样说。

金融机构一面服务广大 C
端用户,一面服务投资机构,一旦出错,是真金白银的损失,试错成本非常高。因此在金融行业,金融机构与科技公司正式签合同之前,一般双方要在实际业务场景中磨合一段时间。这段时间非常长,有的甚至长达一年,目的是为了充分验证竞标公司的技术实力,减少出错的可能。


\paragraph{赚钱几种人2\sphinxfootnotemark[43]}
\label{\detokenize{chapter_introduction/money:id4}}%
\begin{footnotetext}[43]\sphinxAtStartFootnote
\sphinxnolinkurl{https://www.sohu.com/a/409718794\_312708}
%
\end{footnotetext}\ignorespaces 
为什么2020年底还在神化AI的主力军基本上只剩培训机构和自媒体了呢?因为多数还在搞AI产品的公司都意识到了一个问题:靠AI本身是赚不到钱,只有把产品/服务卖出去才能赚到钱。

真正由 AI
驱动的产品并不多,理性认识你负责的那个模型对项目到底有多重要,可以更合理的调配工作时间与精力,也能在和外部对接时省去很多不必要的口舌。

AI赚了钱或得了利的主要是三种人:第一种是赚了风投的钱,吐血的是大大小小的孙正义。第二种是搭了巨无霸的顺风车,那些IT大厂不惜巨资做AI,不是因为AI给他们做出了赚钱的产品,而是想靠炒作AI提升股价,最终是让股民买单。大厂无一例外不敢不上,不能不鼓吹AI,无论其创始人对AI是真了解还是门外汉。他输不起,泡沫起处,你不冲浪冒险,你连游戏都玩不了,入不了局。第三种才是真正找到了市场切入点,把AI落地做成了规模化产品,占住了某个领域市场,也彰显了
AI
的威力。可惜,这第三类跟大熊猫似的,非常珍稀,而且多是九死一生侥幸生存下来的。包括特斯拉的自动驾驶,也是大难不死,现在才见到了曙光。


\paragraph{分析师3\sphinxfootnotemark[44]}
\label{\detokenize{chapter_introduction/money:id5}}%
\begin{footnotetext}[44]\sphinxAtStartFootnote
\sphinxnolinkurl{https://blogs.nvidia.cn/2017/09/30/ai-how-to-speed-up-the-analysis-of-financial-markets/}
%
\end{footnotetext}\ignorespaces 
Triumph
的数据科学家将专用数据库中的新闻传输到深度学习系统中。机器经过训练,每三毫秒即可分析一篇文章,一天能处理数十万篇,这个原来我们认为无法实现的结果,近期得以突破。

系统可以识别文章中数百个关键字。称为 GloVe
的无指导性学习算法可为每个关键字赋予一个数值,然后系统的其他模型可以理解并使用该数值。

深度学习系统最终会产生三个结果:它将文章关联至适当的股票和公司;它为每篇文章得出一个影响力得分(正面、中立和负面);它可以评估新闻影响市场的可能性。

在一段时期内,“虚假新闻”充斥着传统新闻领域,公司的数据科学家使用特定关键字和声誉好的新闻来源来提高可靠性。


\subparagraph{外包}
\label{\detokenize{chapter_introduction/money:id6}}
\sphinxurl{https://www.fiverr.com/?source=top\_nav}


\paragraph{盈利模式 4\sphinxfootnotemark[45]}
\label{\detokenize{chapter_introduction/money:id7}}%
\begin{footnotetext}[45]\sphinxAtStartFootnote
\sphinxnolinkurl{https://www.zhihu.com/question/20781934}
%
\end{footnotetext}\ignorespaces 
强调如何获取利润

每个企业能长期活下去,早期可以靠融资,可以烧钱,但一个企业不可能永远都烧钱,最终都还是要靠自己赚钱活下去,以及老股东投资了也要获得回报。

工具类:坐拥数十甚至上千万用户,却不知如何有效的将流量转化为收入——这是典型的“商业模式没想明白后遗症”!


\paragraph{互联网的盈利模式}
\label{\detokenize{chapter_introduction/money:id8}}
\sphinxhref{https://www.zhihu.com/question/20304614/answer/1608253955}{6}%
\begin{footnote}[46]\sphinxAtStartFootnote
\sphinxnolinkurl{https://www.zhihu.com/question/20304614/answer/1608253955}
%
\end{footnote}
\begin{enumerate}
\sphinxsetlistlabels{\arabic}{enumi}{enumii}{}{.}%
\item {} 
提供一个公平的、纯粹的平台,把卖家和买家,或者把创作者(creator)和用户(viewer)双方联系起来,抽取佣金或者赚点广告费。

\item {} 
提供一个平台,但是把平台本身当作商品。这种公司出售的“商品”就是在这个平台“作弊”的机会。

\item {} 
“杀猪盘”。就是既提供平台,也下场和卖家/创作者竞争。把卖家/创作者当作猪,吸引他们入场,养肥了之后再杀。

\end{enumerate}

形式:广告、实物/虚拟商品售卖、平台佣金、增值服务及金融服务


\subparagraph{流量变现:广告费、抽取佣金or利润/利息差}
\label{\detokenize{chapter_introduction/money:or}}
贩卖流量较多使用kill time产品

以创作、社交为主的平台来说,流量就是一切。所以最重要的就是吸引优秀创作者。为此,很多公司还会和创作者分享利润(revenue
sharing)。比如在播放YouTube优秀创作者视频的时候投放的广告收入,YouTube都会分一部分给创作者本身。YouTube顶流李子柒,每年光YouTube广告收费分成就有七百万元左右。而在直播平台Twitch,如果你有几万粉丝,同时在线人数保持在100以上,就可以申请成为Twitch伙伴(partner)。除了打赏、广告、带货收入之外,只要你签约不去其他平台直播,Twitch每年还会另外给你40万元左右的收入。打赏自由知识分享的

电商直播:通过品类垄断的强议价能力,获取渠道利润;通过高效的仓储物流体系压缩流通成本,进一步拓展利润空间;通过龙头效应带来的厂家营销资源投入(代理、广告、商品运营等与厂家的营销合作)


\subparagraph{广告 7\sphinxfootnotemark[47]}
\label{\detokenize{chapter_introduction/money:id9}}%
\begin{footnotetext}[47]\sphinxAtStartFootnote
\sphinxnolinkurl{https://weread.qq.com/web/reader/8d232b60721a488e8d21e54kc20321001cc20ad4d76f5ae}
%
\end{footnotetext}\ignorespaces 
形式:
\begin{itemize}
\item {} 
搜索广告典型的搜索广告是百度的竞价广告,展示在搜索结果列表中。商家竞价购买关键词,当用户搜索的内容触发关键词时,出价最高的广告就会被优先展示。

\item {} 
展示类广告展示类广告一般出现在信息类网站中的Banner、竖边、通栏等位置,而且展示类广告以品牌广告居多,即它更注重品牌曝光。

\item {} 
开屏广告部分App
的启动过程中会显示一副全屏广告,形式可能是图片,也可能是视频,展示时间为3~5秒不等,这就是开屏广告。知乎、豆瓣、小红书上等都会有这种类型的广告。

\item {} 
信息流广告信息流广告是指以文章、图片、视频等形式插入信息列表的广告,常见于内容类的产品,如百度、知乎、今日头条等App
中经常出现标记了“广告”字眼的信息。

\item {} 
视频广告在爱奇艺、优酷、腾讯视频等视频网站,如果用户没有购买会员,就会在视频播放前、播放过程中及暂停过程中看到广告,这些广告就是视频广告。

\end{itemize}

计费模式:
\begin{enumerate}
\sphinxsetlistlabels{\arabic}{enumi}{enumii}{}{.}%
\item {} 
CPM(Cost per
Mile,按千次展示收费):只要曝光就收费,不管点击、下载或注册等后续流程。这种模式适合想扩大知名度做品牌广告的广告主。早期门户网站的展示类广告基本都采用这种模式,少数开屏广告也采用这种模式。

\item {} 
CPC(Cost per
Click,按点击收费):在这种模式下,不管展示了多少次,只要用户不点击,广告主都不需要付费,只有用户点击了,广告主才需要付费。这种模式对广告主比较友好,因为首先它加大了平台作弊的难度;其次,它可以检测每个平台的流量质量,点击率高的就意味着质量高、用户精准,广告主以后可以多在这个平台投放广告。目前,这种模式常见于信息流广告和开屏广告。

\item {} 
CPA(Cost per Action,按用户行动收费):A
代表Action(行动),具体的用户行动是多种多样的,可以是下载、安装、购买等,具体是指哪种行动,需要在广告洽谈的时候,广告主和平台协商好,只有用户产生了协商好的行动,广告主才付费。

\end{enumerate}

可以看到,CPD(Cost per Download,按下载量收费)、CPI(Cost per
Install,按安装量收费)、CPS(Cost per Sales,按销售量收费)适合App
下载、增加新用户等需要明确转化行动的广告主。相对来说,CPM
适合以宣传品牌为主的广告主,而CPC 和CPA 倾向于保护广告主的利益。


\subparagraph{平台佣金}
\label{\detokenize{chapter_introduction/money:id10}}
平台模式:只提供交易平台(佣金、管理费用),卖家处理商品管理、仓储、配送、售后服务、开具发票

自营模式:买断货物,企业提供商品管理、仓储、配送、售后服务、开具发票服务;优势:服务体验优,利润率高;缺点:运营成本高

大部分平台型产品本身不拥有资产,但是通过整合资源提升服务效率获利。比较重要的盈利模式是收取平台佣金,但是收取对象不同。例如,淘宝平台和美团平台佣金的收取对象是商家,滴滴出行平台佣金的收取对象是司机,腾讯课堂平台佣金的收取对象是教师、教育机构,同花顺平台佣金的收取对象是股票投资者,映客直播平台佣金的收取对象是主播,直卖网平台佣金的收取对象是生产厂家等,这些平台的主要盈利模式就是平台佣金。平台上内容的所有权不归平台,所以这种盈利模式与实物/虚拟商品售卖的盈利模式有本质的区别。


\subparagraph{赚取中间利润}
\label{\detokenize{chapter_introduction/money:id11}}\begin{enumerate}
\sphinxsetlistlabels{\arabic}{enumi}{enumii}{}{.}%
\item {} 
实物商品例如,京东的自营商品。京东采销人员向供应商采购商品在京东上售卖,商品的所有权属于京东,京东通过售卖商品赚取中间利润。又如,网易严选在代工厂贴牌之后直接售卖商品,这也是自营实物商品的模式。

\item {} 
虚拟商品例如,猿辅导的K12
网课,所有课程都是猿辅导的在职教师录制的,属于自营模式,而且课程属于虚拟商品。又如,粉笔网的网课等也属于虚拟商品。

\item {} 
虚拟服务虚拟服务与虚拟商品不同,它不是商品,而是一种服务,如阿里云服务(宽带、云存储等服务)、百度地图(WebAPI
服务)等。

\end{enumerate}


\subparagraph{赚取利息差}
\label{\detokenize{chapter_introduction/money:id12}}
金融借贷根据服务对象的不同,金融借贷可分为消费金融和供应链金融。消费金融是指2C
业务,如京东的白条及线下婚庆公司、教育公司的分期服务等;供应链金融是指2B
业务,如京东的京小贷、京保贝等业务都是为京东商家提供贷款的业务。

沉淀资金沉淀资金的金融服务模式是指利用沉淀在平台上的资金投资或者开展其他业务而产生收益。例如,用户在京东平台购物需要实时支付,但是京东平台跟第三方商家的结算是有一定的账期的,如30
天。那么,在账期内,这些资金就会沉淀在京东平台上,京东平台就可以利用这些资金投资或开展其他业务。


\subparagraph{“作弊”的机会——增值服务}
\label{\detokenize{chapter_introduction/money:id13}}
直接收费较多使用在save time产品

Facebook的“商品”,便是在Facebook的平台上“作弊”。也就是交钱让Facebook推广你的广告帖、广告视频。Facebook对优秀创作者并不支持,反而是打压。如果你不给Facebook交推广费,即使你一直在Facebook创作优秀的内容,很多人关注,很多人转发你的内容,Facebook也会故意通过算法打压你的内容,让别人无法看到。所以在Facebook能不出推广费用而获得流量是非常非常困难的。

虽然这种盈利方式来钱快,但是也有弊端,也就是因为失去优质内容,容易丢失用户。Facebook的应对方式就是通过主打和真实世界认识的亲朋好友的联系,推广和Facebook账号绑定的聊天软件Messenger来锁住用户。很多人为了了解亲朋好友的动态,不得不捏着鼻子看着Facebook上面充斥的广告贴。即使这样,最近Facebook的北美月活跃用户(monthly
active users)也一直在下降,有成为北美人人网的趋势。

基础功能免费吸引用户,增值服务收费实现盈利,这就是增值服务这种盈利模式的拆解。例如,百度网盘基础版的上传、下载等功能都可以免费使用,百度网盘也会免费为用户提供一部分存储空间,但是用户想获得更大的存储空间、更快的下载速度等,就要购买产品会员,这就是增值服务。又如,QQ
超级会员,很多年轻人喜欢的装扮特权(如挂件头像、气泡等),以及一些热门功能特权(如消息记录漫游、3000
人超大群)都属于增值服务。再如,CCtalk
的基础营销工具及授课是可以免费使用的,但是用户想要获取短信通知、多群直播、高清授课、录制下载等高级功能,就要付费购买。


\subparagraph{“杀猪盘”}
\label{\detokenize{chapter_introduction/money:id14}}
亚马逊这个网站,既提供一个平台给小商家在上面卖东西,但是他们自己也有自营的网店业务。他们先让第三方卖家入场卖东西,让他们赚点小钱。但是这些第三方卖家的价格、浏览量、销售额全部被亚马逊平台所掌握。之后亚马逊便会通过这些数据进行“严选”,找出一些比较好赚钱的商品,直接下场与第三方卖家竞争,通过价格战消灭第三方卖家。有时候甚至会随意封禁第三方卖家。

但是如果一个公司市场占有率还很低,公司官方提供的商品/内容质量远远低于第三方卖家/创作者,还学亚马逊搞杀猪盘,竭泽而渔的话,很难说是一种明智的行为。


\paragraph{人工智能的盈利模式 9\sphinxfootnotemark[48]}
\label{\detokenize{chapter_introduction/money:id15}}%
\begin{footnotetext}[48]\sphinxAtStartFootnote
\sphinxnolinkurl{https://weread.qq.com/web/reader/0c032c9071dbddbc0c06459k65132ca01b6512bd43d90e3}
%
\end{footnotetext}\ignorespaces 

\subparagraph{通过专利技术的授权、转让或置换实现盈利}
\label{\detokenize{chapter_introduction/money:id16}}
随着经济的发展,中国已成为全球消费品生产、消费和贸易大国,中国的人工智能产品也越来越多,而支撑人工智能产品的基础是中国在人工智能技术方面的不断创新。全球人工智能领域的专利数量自2011年开始逐渐呈现爆发式增长,每年的复合增长率达到30\%以上。中国在AI方面的专利技术布局程度已经位居世界第一。正因为专利技术在人工智能产品中的重要性,在人工智能市场上,通过技术创新申请专利,并将专利技术转让已经成了一种非常重要的盈利模式。

对比一下中国和美国在人工智能领域的六个企业——腾讯、百度、阿里巴巴、IBM、微软、谷歌可知,这些企业都非常注重整体的专利布局。而且,通过比较这些企业申请的专利可知,美国企业热衷于机器学习、语音识别、语言合成处理等领域,中国企业则倾向于支付、交互技术、视频图像信息处理、智能搜索等领域。另外,六家企业都比较感兴趣的领域有无人驾驶、数据文本聚类、指纹识别等。

IBM专利布局比较全面,其中算法优化、自然语言处理、自主驾驶领域布局优势明显;微软的专利布局主要在机器学习、神经网络、音视频识别等领域;谷歌主要是在无人驾驶、语音识别、自然语言处理领域有较多专利;腾讯主要在即时通信、数据处理、支付平台、数据交易等人工智能领域展开布局;百度比较热衷在搜索业务、无人驾驶、语音识别、图像识别等领域布局;阿里巴巴则在支付平台、信息交互、广告投放等领域布局明显。

企业在申请人工智能相关专利技术时,应充分体现出理论层次性、技术创新性、工程复杂性。在撰写时应注意以下几个方面:
\begin{enumerate}
\sphinxsetlistlabels{\arabic}{enumi}{enumii}{}{.}%
\item {} 
建议突出其成果专利的创新能力;

\item {} 
建议突出技术细节,并描述技术深度;

\item {} 
建议合理运用应用场景结合技术创新。

\end{enumerate}


\subparagraph{通过输出人工智能技术实现盈利}
\label{\detokenize{chapter_introduction/money:id17}}
除通过专利技术的授权、转让或置换实现盈利外,拥有人工智能技术的公司也可以通过对外提供技术服务实现盈利。目前人工智能的技术服务体系包括了基础级技术服务、技术级技术服务和行业级技术服务三个层面。

基础级技术服务是指企业通过提供框架平台或算法平台来提供的技术服务。例如,百度AI开放平台,阿里云ET大脑、腾讯AI平台、讯飞开放平台等。提供基础级技术的公司通过推出这些平台接口,吸引更多的用户,从而进一步活跃其产品的应用,并逐渐打造起一个开发者生态,并通过生态的活跃,提供其产品在行业中的应用阿里云ET大脑提供了多项人工智能技术服务,包括ET行业大脑、人工智能解决方案、人工智能接口、算法平台、ET大脑生态等内容。

上述公司在人工智能基础技术领域内有一定的优势,因此这类公司会在具体的技术领域进行拓展延伸。当然,如果这类公司仅仅提供技术,则会有竞争力弱的困扰,所以这些仅提供技术的公司往往通过“人工智能+行业”的模式形成具体的解决方案从而实现持续的盈利。

如果人工智能类的公司既拥有技术,同时又拥有大量的数据积累,则可以通过提供人工智能产品应用实现盈利。例如,格灵深瞳将人工智能和视频监控进行结合,开发了威目视图,实现了图像识别、人车定位识别;旷视科技的Face++平台,已经是我国领先的人脸识别的服务平台。


\subparagraph{通过销售产品实现盈利}
\label{\detokenize{chapter_introduction/money:id18}}
人工智能专利技术的转让及人工智能技术服务的输出,一般都是面向企业的,而人工智能产品则是面向大众的,易形成影响力。随着人工智能的发展,人工智能的产品类型也越来越多,如人工智能机器人、智能音箱、实时翻译工具、电子商务推荐助手、医疗影像检查、智能手环、游戏等。

人工智能产品销售方向有两个:一是面向企业,即2B;二是面向个人,即2C。2B方向的产品主要以提高生产力为目标,为企业降本增效,如智能分拣机器人、智能服务员、智能客服等。2C方向的产品主要是作为人体的延伸进行辅助判断及辅助操作,如智能音箱、智能推荐系统等。天猫精灵是阿里巴巴人工智能实验室于2017年7月5日发布的人工智能产品。天猫精灵内置智能语音助手AliGenie,能听懂普通话语音指令,并实现智能家居控制、语音购物、手机充值、音乐播放等功能。2018年5月27日,阿里巴巴公布了天猫精灵的销量,销售总量超过了300万台,业绩非常优秀。


\paragraph{为何免费}
\label{\detokenize{chapter_introduction/money:id19}}
免费商业模式的本质,即交叉补贴。

前提:
\begin{enumerate}
\sphinxsetlistlabels{\arabic}{enumi}{enumii}{}{.}%
\item {} 
能不能找到补贴方?

\item {} 
从补贴方获得的收益能否覆盖免费的成本?

\item {} 
在找到能覆盖免费陈本的补贴方之前,这个时间成本是否可承受?你总不能死在找到补贴方之前吧。

\end{enumerate}

◆ 直接交叉补贴:产品之间的交叉补贴,用免费吸引你掏腰包买其他的产品。

腾讯公司马化腾就是利用免费核心产品QQ,绑架近几亿用户,从而向这些用户销售一些增值产品来赚取利润,比如QQ秀,钻等。电信运营商依靠赠送手机或话费来吸引用户,赚取流量和话费等等。

◆ 三方市场:利益主体之间的交叉补贴

媒体行业是三方市场模式的典型,广播、报纸、电视和杂志等等,用户不用付费可以免费的到信息、内容或软件。由广告商买单。即媒体将用户卖给广告商。内容消费者得到了免费,但有广告主来买单。

◆ 现在和未来之间的交叉补贴

比如滴滴打车,在推广的时候,很多人享受过免费打车的。但这个钱最终会在习惯被养成之后赚回来。

◆ 货币市场和非货币市场的交叉补贴:

任何人都可以免费得到其他人赠送的产品或服务,且不需要得到金钱回报,获得的是关注度和声誉。撰写博客,发布微博、微信等,并非出于谋取利益,而是与人分享喜怒哀乐,期待结识朋友;公益捐助,获得慈善相关的名声等等。


\paragraph{ROI}
\label{\detokenize{chapter_introduction/money:roi}}
Open source models, data and transfer learning are also enabling
businesses to more easily move models into production and to achieve an
ROI.


\paragraph{融资}
\label{\detokenize{chapter_introduction/money:id20}}
企业融资,说白了就是企业如何获得正向的现金流。因为有了钱,你就可以当个土豪“买买买”,买装备、买土地、买资源、买人才、买用户甚至买竞争对手。但是,市场上真正缺钱的都是中小企业/民营企业/初创企业这样的企业,出身差、没钱、没人才、没资源,在债权融资要不来钱的情况下,这些企业就选择股权融资。

一般来说,融资轮次的划分为种子轮、天使轮、A轮、B轮、C轮、D轮、E轮等,但根据实际情况,有些项目也会进行preA轮、A+轮、C+轮融资,不管是什么轮,其核心无非是投资人投的多少钱的问题,
\begin{itemize}
\item {} 
种子轮:种子阶段的融资人,通常只有idea和团队,但没具体产品的初始形态,投资人一般多是亲朋好友、或者创业者自掏腰包,当然现在也涌现不少种子时期投资人;倘若你的融资项目团队,有idea,马上进入最终的落地,那么就可以进行种子轮融资,一般项目融资都在100万左右,根据不同的赛道,可能从几十万到200万不等。

\item {} 
天使轮:天使阶段的项目通常是团队ready,有产品雏形,有产品初步的商业规划,却也陷入找人——做产品——没人了——找人——做产品的循环之中,如果融资项目已经起步,产品初具模样,有种子数据显示出增长趋势、留存、复购等证明。同时积累了一些核心用户,商业处于待验证的阶段,那么找天使投资人或机构,开始天使轮融资便是最为合适的,融资金额大概在300万到500万左右;

\item {} 
PreA轮:是一个夹层轮,融资人根据自身项目的成熟度,再决定是否要融资,倘若项目前期整体数据已经具有一定规模,只是未占据市场前列,那么可以进行PreA轮融资;

\item {} 
A轮:对于拥有成熟产品,完整详细的商业及盈利模式,同时在行业内拥有一定地位与口碑的项目,哪怕现阶段处于亏损状态,也可以选择专业的风险投资机构进行A轮融资,这一阶段融资人已经不可能只凭借idea融资,而是要有用户,包括月活、日活、要有自己的商业模式,有能与竞品抗衡的成熟产品,有一定的市场位置;

\item {} 
B轮:经过一轮的烧钱后,项目有较大的发展,商业模式与盈利模式均已得到很好的验证,有的已经开始盈利。此时,融资人可能需要资金支持推出新的业务、拓展新领域,那么就适合说服上一轮风险投资跟投,或寻找新的风投机构的加入,又或是吸引私募股权投资机构加入的形式,开始新的一轮的B轮融资。

\item {} 
C轮:如果此时融资人的项目十分成熟,在行业内基本可以稳坐前三把交椅,正在为上市做准备,那么就适合进行C轮融资,此时除了可以进一步拓展新业务,也可以补全商业闭环,准备上市打好基础。

\item {} 
D轮、E轮、F轮:上市后的融资;

\end{itemize}


\subsubsection{Product Mananger}
\label{\detokenize{chapter_introduction/PM:product-mananger}}\label{\detokenize{chapter_introduction/PM::doc}}

\paragraph{定义}
\label{\detokenize{chapter_introduction/PM:id1}}

\subparagraph{心态}
\label{\detokenize{chapter_introduction/PM:id2}}
只要你能够发现问题并描述清楚,转化为一个需求,进而转化为一个任务,争取到支持,发动起一批人,将这个任务完成,并创造了价值,并持续不断以主人翁的心态去跟踪、维护这个产物,那么,你就是产品经理。


\subparagraph{市场满足}
\label{\detokenize{chapter_introduction/PM:id3}}
产品经理的价值是创造一款满足产品\sphinxhyphen{}市场匹配(Product\sphinxhyphen{}market
fit,PMF)「在一个好的市场里,能用一个产品去满足这个市场」的产品
\sphinxhref{http://www.ramywu.com/work/2018/05/31/AI-PM-Interview/}{9}%
\begin{footnote}[49]\sphinxAtStartFootnote
\sphinxnolinkurl{http://www.ramywu.com/work/2018/05/31/AI-PM-Interview/}
%
\end{footnote}

为企业获得利益回报而创造顾客价值,并与顾客建立稳定的关系;为产品的市场结果负责。
\begin{itemize}
\item {} 
顾客价值,即为顾客创造的产品(包含服务,我将其归到产品里,以后所有提到产品的地方都包含服务);

\item {} 
利益回报,简单讲就是买产品赚取金钱收益,当然也可能是合理降低回报以获取品牌收益,但最终还是为了赚钱;

\item {} 
建立顾客关系,即持续赚钱,尽可能占领更多的市场。

\end{itemize}


\subparagraph{协作}
\label{\detokenize{chapter_introduction/PM:id4}}
产品经理是“将不同语言的公司的所有各种功能和角色结合在一起的粘合剂” – Ken
Norton,GV \sphinxhref{https://easyai.tech/author/xiaoqiang/page/5/}{7}%
\begin{footnote}[50]\sphinxAtStartFootnote
\sphinxnolinkurl{https://easyai.tech/author/xiaoqiang/page/5/}
%
\end{footnote}

产品经理还是一个项目的信息汇集中心,对于公司的战略方向,产品经理比其他人更早地知道。产品经理也会经常跟运营人员、销售人员开会,所以对于公司的运营节奏、销售数据,产品经理都可能提前知悉。这些都是可以帮助产品经理快速成长的点,也是产品经理不断积累人脉的基础。
\sphinxhref{https://weread.qq.com/web/reader/8d232b60721a488e8d21e54k8f132430178f14e45fce0f7}{15}%
\begin{footnote}[51]\sphinxAtStartFootnote
\sphinxnolinkurl{https://weread.qq.com/web/reader/8d232b60721a488e8d21e54k8f132430178f14e45fce0f7}
%
\end{footnote}

\begin{figure}[H]
\centering
\capstart

\noindent\sphinxincludegraphics{{business_value}.png}
\caption{为了商业价值}\label{\detokenize{chapter_introduction/PM:id29}}\end{figure}

传统产品经理泛指传统互联网产品经理,区别于宝洁(产品经理这个职能的开创者)时期的实体工业产品经理,互联网产品经理标准化是在移动互联网的商业模式成型过程中,也就是2010s这个时期,在这个时期随着移动互联网的发展,各个公司特别是BAT为代表的巨头公司细化了产品经理所需要具备的能力模型,并且基于这个能力模型去对产品人员进行量化考核
\sphinxhref{https://www.jianshu.com/p/fd466ed1bda6}{1}%
\begin{footnote}[52]\sphinxAtStartFootnote
\sphinxnolinkurl{https://www.jianshu.com/p/fd466ed1bda6}
%
\end{footnote}


\paragraph{VS Project Mananger}
\label{\detokenize{chapter_introduction/PM:vs-project-mananger}}
\sphinxstylestrong{产品经理(Product
Manager)}:负责挖掘用户需求,并提出能够同时满足用户需求和公司利益的产品方案。对产品本身负责,对用户负责,或者说对产品发布后是否受用户认可负责。与产品运营人员共背用户数、活跃度、留存率、营收等KPI指标。

多做需求快速试错:
对于大多数产品经理来说,用户是否喜欢新功能其实是未知的,不然人人都是产品经理了,免不了试错的过程。做5个新功能可能没有1个用户喜欢的,做50个呢?总能撞上几个让用户喜欢的吧。所以很多不成熟的产品经理喜欢通过增加功能点来“碰运气”。

\sphinxstylestrong{项目经理(Project
Manager)}:负责管理整个产品开发过程的项目,负责协调产品开发中的一切资源,包含人、时间、资源、成本等。是整个项目的牵头人,对项目的开发过程和按时完成预期结果负责。也就是说,项目经理的KPI应该是项目的按时完成与完成质量。用户数、活跃度等指标一概与项目经理无关,更别说干涉需求。

按时按质完成:做得越多就错得越多,很容易影响项目按时验收,以及项目结果质量。比如APP产品的开发,大家都知道修复完一个bug后,正常情况下会出9个新bug。所以项目经理是喜欢需求越少越好,这样项目复杂程度会低,工作量与风险程度都很容易控制在安全范围内。

由目标的差异推导,项目经理更强调执行,是接到一个任务,\sphinxstylestrong{正确地做事};产品经理更强调创新,是设定一个目标,\sphinxstylestrong{做正确的事}。

如果由一个人同时担任产品经理和项目经理,就很容易左右互搏。需求是自己的,当然希望都做。但开发过程中遇到难题怎么办。砍需求?改方案?自己预期中最完美的方案要舍弃还是挺心痛的。坚持把需求按原方案做完?项目延期,影响正常上线,责任谁担?


\paragraph{不是人人都是PM}
\label{\detokenize{chapter_introduction/PM:pm}}
对产品经理有洁癖的腾讯,把12级(原P4\sphinxhyphen{}1)以下的产品从业者都改名叫什么产品策划、产品运营,只有综合能力达标的那一小撮人,才配继续叫产品经理。\sphinxhref{https://m.k.sohu.com/d/495625828?channelId=1\&page=1}{8}%
\begin{footnote}[53]\sphinxAtStartFootnote
\sphinxnolinkurl{https://m.k.sohu.com/d/495625828?channelId=1\&page=1}
%
\end{footnote}


\paragraph{能力要求}
\label{\detokenize{chapter_introduction/PM:id5}}
软能力包括了最常提到的学习能力、执行能力、沟通能力、责任感、沟通表达能力、市场洞察能力、创新能力、影响力等等,这些能力是比较难以量化,需要通过具体项目推进去观察,带有一定的主观性。硬能力包括了产品规划、需求调研、需求拟定(原型、需求文档等)项目管理、商务沟通、运营数据分析、市场营销等

\begin{figure}[H]
\centering
\capstart

\noindent\sphinxincludegraphics{{PM}.jpg}
\caption{PM能力模型}\label{\detokenize{chapter_introduction/PM:id30}}\end{figure}


\paragraph{主线}
\label{\detokenize{chapter_introduction/PM:id6}}
主线是围绕产品从0\sphinxhyphen{}1\sphinxhyphen{}N全周期的具体推进,每一个环节会有具体的工具和方法,如产品规划阶段就需要用到PEST分析法、SPAN竞争与策略、波特五力模型、波士顿矩阵等,有的公司还会形成一套标准MRD(市场需求文档);再比如需求拟定环节能力细化为产品原型、交互说明、业务流、数据流、资金流等。再分享一个我目前看到过最完善的产品能力框架图,这张图把产品的能力分为了产品设计能力(市场阶段、规划阶段)、专业技能管理(产品设计阶段)、产品管理(产品开发阶段)、自我管理能力(软技能)、团队管理能力(管理能力)等五大模块

\begin{figure}[H]
\centering
\capstart

\noindent\sphinxincludegraphics{{product_life}.png}
\caption{产品生命周期}\label{\detokenize{chapter_introduction/PM:id31}}\end{figure}


\paragraph{工作内容 2\sphinxfootnotemark[54]}
\label{\detokenize{chapter_introduction/PM:id7}}%
\begin{footnotetext}[54]\sphinxAtStartFootnote
\sphinxnolinkurl{https://www.zhihu.com/question/343743405/answer/1237754321s}
%
\end{footnotetext}\ignorespaces \begin{enumerate}
\sphinxsetlistlabels{\arabic}{enumi}{enumii}{}{.}%
\item {} 
做行业洞察和市场调研,分析行业和产品的发展趋势,友商的竞品分析和客户的需求分析等,输出MRD,需求用例评审。

\item {} 
根据MRD结合公司现有的技术积累、公司战略方向、客户痛点需求和市场销售预期写PRD。
\begin{enumerate}
\sphinxsetlistlabels{\arabic}{enumii}{enumiii}{}{.}%
\item {} 
先分析业务,整理出需求用例文档,需求用例评审通过\sphinxhref{https://www.zhihu.com/question/36913495/answer/252737063}{6}%
\begin{footnote}[55]\sphinxAtStartFootnote
\sphinxnolinkurl{https://www.zhihu.com/question/36913495/answer/252737063}
%
\end{footnote}

\item {} 
用 Axure 制作原型图,原型图评审通过

\item {} 
用 PhotoShop 做出效果图,效果评审通过

\item {} 
切图出素材,再然后开始做软件架构设计,架构评审通过

\end{enumerate}

\item {} 
推动研发的开发和资源投入,项目管理(制定计划并跟踪、确定资源投入、把控质量,写周报等汇报),产品生命周期管理等

\item {} 
负责产品的推广策略、要写一堆的产品推广资料

\item {} 
负责产品经营性工作,要负责产品营销策略和产品销售业绩,所以经常要做产品经营性数据分析

\item {} 
培训、拜访客户、挖坑、填坑balabala…..等其他非核心内容工作。

\end{enumerate}


\paragraph{产品经理的角色理解 5\sphinxfootnotemark[56]}
\label{\detokenize{chapter_introduction/PM:id8}}%
\begin{footnotetext}[56]\sphinxAtStartFootnote
\sphinxnolinkurl{https://www.zhihu.com/question/31636227/answer/1251352264}
%
\end{footnotetext}\ignorespaces 
产品经理不做具体的开发工作,只是规划产品的功能和发展方向,然后去协调UI、UE、前端、开发、测试等部门,一起协同完成产品的开发。从这个意义上讲,产品经理是做协调工作的

首先我们要明确的一件事是:虽然称为产品经理,但产品经理是没有管理权限的,也就是说产品经理在公司几乎不能要求别人做什么事情,而只能是协调他人做什么事。

弄清楚了这一点,我们再来看产品经理在公司的角色,就可以归结为协调者。所谓协调者,可以从以下几个方面来理解:


\subparagraph{信息的协调者}
\label{\detokenize{chapter_introduction/PM:id9}}
在前面介绍产品经理做什么的时候,也说到产品经理会接触公司大部分的部门,因此产品经理就会收集到这些部门与自身产品相关的信息。例如产品经理可以从公司领导那里获得产品战略发展的信息;可以从UI那里那里获得LOGO含义的信息;可以从开发那里获得产品底层框架的信息,等等。当这些信息达到产品经理手里时,并不是信息的终结,而是信息分析与传递的开始。产品经理需要将这些信息转化,转化成大家需要且易懂的信息,进而再传递给需要的成员。从这个意义上讲,产品经理在公司更多扮演了信息收集者和传递者的角色。


\subparagraph{资源的协调者}
\label{\detokenize{chapter_introduction/PM:id10}}
虽然说产品经理手里没有管理权,但却在很大程度上决定产品的发展,因此产品经理可以发挥影响力来协调广泛的资源。我们都知道,产品经理需要和公司领导、UI、前段、开发、测试、客服等部门进行协调,而这些部门同事的工作基本上也都是围绕着产品经理展开的,所以两者之间是一种相互依存的关系。

在这种情况下,产品经理就可以根据产品计划来协调资源。不过,这里非常考验产品经理协调资源的能力,尤其是在产品经理手里有若干项目,或者有若干个产品经理要共享有限的资源的情况下,这时候协调的好与坏,直接决定了项目的进度与效率。

再上升一个层次看产品经理的角色,其手里可能握有产品的生杀大权。也就是说,产品经理可能会决定一个产品的成与败,一个优秀的产品经理可以化腐朽为神奇,成为人们心中的大咖,而不好的产品经理却可能化神奇为腐朽,将产品和团队带入迷茫之中。

对于很多产品小白而言,可能做的更多还是领导指派的具体事务,不过只要保持进步,终有一天会成为中流砥柱,而如果你已经小有成就,对产品也需要抱有敬畏之心,因为世界变化太快,成败往往就一瞬之间的事情。


\paragraph{产品经理接触的人}
\label{\detokenize{chapter_introduction/PM:id11}}
分两部分来说:产品规划与产品开发。


\subparagraph{就产品规划而言,产品经理接触到的人包括但不限于:}
\label{\detokenize{chapter_introduction/PM:prod-people}}\label{\detokenize{chapter_introduction/PM:id12}}
1)直线领导:

当我们做产品规划时,必然要和直线领导就方案达成共识,才能进一步向外沟通确认,因此在产品规划阶段,你需要频繁地与直线领导沟通或汇报(有时候直线领导可能不参与具体讨论,但需要知道进度)。

2)公司领导

有时候,公司领导可能是某个需求的提出者。这种情况下,产品经理(或直线领导)需要向公司领导汇报相关解决方案。

3)业务人员

如果你负责的产品有业务人员的话,那他们也是产品重要的需求方,同时他们在与客户接触中,会出现种种问题。这个时候,都需要产品经理参与解决。

4)客服人员

针对产品规划,客服人员反馈的用户数据尤为重要,因此产品经理需要频繁地与客服人员进行沟通,搜集数据,整理并转化为需求。

5)用户

用户研究是产品规划阶段的核心工作之一,也是产品经理难得的接触真正用户的机会。在这个阶段中,产品经理可以采用用户访谈、调查问卷、可用性测试等方式,多多与用户进行接触。


\subparagraph{就产品开发而言,产品经理接触的人包括但不限于:}
\label{\detokenize{chapter_introduction/PM:id13}}
1)UI/UE

当产品原型最终确定,就可以进入UI设计(多为GUI)阶段,这个时候产品经理就需要和UI探讨原型细节,进入设计阶段。用户界面是系统和用户之间进行交互和信息交换的媒介,它实现息的内部形式与人类可以接受形式之间的转换。体验其实也就是一系列感官的综合。

\begin{figure}[H]
\centering
\capstart

\noindent\sphinxincludegraphics{{UX}.png}
\caption{UX}\label{\detokenize{chapter_introduction/PM:id32}}\end{figure}

2)前端

UI设计完成后,就开始转入前端工作。对于前端而言,会更加关注细节,每一个按钮的状态变化,每一个交互细节,都需要详细说明。这块一般是由产品经理和UI共同提供的。

不过如果是移动端产品,前端基本上就不太会参与,页面切图和标注工作主要是由UI完成。

3)开发

开发的工作主要是参照需求文档来展开的,因此产品经理需要就需求文档细节与开发进行充分沟通,以保证开发工作的有效性。

研发经理:研发经理是技术研发管理职位,负责了解项目的需求,系统分析,做相关的技术选型,制定开发计划与开发规范。

架构师:架构师是软件系统和网络系统的设计师,负责确认和评估产品需求、搭建软件研发和网络系统的核心构架、并扫清主要难点。架构师着眼于“技术实现”,能对常见场景快速给出最恰当的技术解决方案,并能评估团队实现功能需求的代价。架构师分为软件架构师和系统架构师两类,分别专注于软件开发和系统运维两个阶段的系统设计。

Web前端工程师:Web前端工程师是界面研发职位,负责根据架构设计文档和界面设计稿,使用Web技术(HTML/CSS/JavaScript等)进行Web产品界面开发,并调用Server端接口实现Web应用。

APP开发工程师:APP开发工程师是APP界面研发职位,负责根据需求文档和界面设计稿开发出APP客户端界面,并调用Server端接口实现APP应用

4)测试

开发完成了项目工作,就进入了测试阶段。一般情况下,测试人员会在开始之前召开测试用例评审,然后才进入具体的测试阶段。无论是测试用例编写阶段,还是测试阶段,执行测试任务、提交测试Bug、跟进Bug修正,产品经理都是要与测试充分沟通的。

事实上,项目开发的工作是阶段性的,但产品经理与团队的接触则是全程的。从需求的发生,到项目的上线,产品经理都需要与UI、前端、开发、测试等人员充分接触,对产品需求进行沟通评估。


\paragraph{分类}
\label{\detokenize{chapter_introduction/PM:id14}}

\subparagraph{技能型产品经理}
\label{\detokenize{chapter_introduction/PM:id15}}
所谓技能型产品经理,就是对某个特定领域有很深的研究,具有较高的专业门槛。为了更直观地了解技能型产品经理,我们来看一则招聘广告:

职位描述:
\begin{itemize}
\item {} 
负责京准通(京东广告平台)创意审核系统,AI方向的优化升级相关工作;

\item {} 
从AI审核、人工审核、创意自动化等多个方向出发,提出优化改进方案,
最终实现审核时效及审核通过率的提升;

\item {} 
AI在广告投放平台的其他应用试验:包含效果优化,预算控制等。

\item {} 
了解行业整体发展趋势,定期对相关竞品进行跟踪和分析;
关注产品运营数据和用户反馈,深入发掘用户的需求,持续改进产品。

\end{itemize}

任职要求:
\begin{itemize}
\item {} 
熟悉互联网精准广告的投放流程,具备互联网商业变现或者广告行业工作经验者优先;有AI相关工作经验的优先

\item {} 
良好的需求分析、数据分析、产品设计能力,熟悉产品设计工作流程;

\item {} 
优秀的沟通协调能力,整合各相关团队资源,推动跨团队合作。
以上是京东商城招聘AI广告产品经理的招聘信息。从信息中,我们可以看到,对产品经理的要求几乎都是关于AI方面。对于此类工作,如果没有深厚的专业知识和行业经验,是很难胜任的。

\end{itemize}


\subparagraph{管理型产品经理}
\label{\detokenize{chapter_introduction/PM:id16}}
相比较技能型产品经理,管理型产品经理的要求更多偏向于规划、协调等方面。同样,我们来看下面招聘信息:

职位描述:
\begin{itemize}
\item {} 
负责规划、设计、运营管理产品,架构专车B:raw\sphinxhyphen{}latex:\sphinxtitleref{C端产品系统};

\item {} 
根据每个阶段的业务目标,确立需求的优先级,满足业务每个阶段的人员效率要求,支持业务快速发展;

\item {} 
负责具体系统项目的计划、需求和产品文档撰写,详细阐述产品功能和操作流程;

\item {} 
跟进协调与支持产品相关的技术团队完成产品开发任务,保证按时上线。

\end{itemize}

任职要求:
\begin{itemize}
\item {} 
5年以上互联网产品设计经验,有丰富的系统设计或独立业务经验的产品架构师优先;

\item {} 
良好的逻辑思维能力、系统思维和广阔的业务视野;

\item {} 
良好的表达能力、沟通能力、抗压能力和团队管理能力;

\item {} 
富有激情和强烈的创新意识和团队合作。

\end{itemize}


\paragraph{大厂VS咨询VS创业 11\sphinxfootnotemark[57]}
\label{\detokenize{chapter_introduction/PM:vsvs-11}}%
\begin{footnotetext}[57]\sphinxAtStartFootnote
\sphinxnolinkurl{https://www.bilibili.com/read/cv4579443/}
%
\end{footnotetext}\ignorespaces 

\subparagraph{大厂产品经理}
\label{\detokenize{chapter_introduction/PM:id17}}
以腾讯(商户管理)产品经理的工作职责,我们可以看到大厂的产品经理需要具备的关键技能体现在
4 方面:
\begin{enumerate}
\sphinxsetlistlabels{\arabic}{enumi}{enumii}{}{.}%
\item {} 
产品设计和运营能力

\item {} 
持续优化和运营能力

\item {} 
组织协调和跨部门协作能力

\item {} 
长期规划能力

\end{enumerate}

大厂产品经理需要具备的技能中,有 2 个关键技能非常值得大家注意:
\begin{enumerate}
\sphinxsetlistlabels{\arabic}{enumi}{enumii}{}{.}%
\item {} 
软技能

\end{enumerate}

在大公司,需要产品经理具备软技能,比如书写邮件能力、组织开会能力、整理会议纪要能力、协调资源能力。
\begin{enumerate}
\sphinxsetlistlabels{\arabic}{enumi}{enumii}{}{.}%
\setcounter{enumi}{1}
\item {} 
跨部门协作

\end{enumerate}

在大公司,各部门的职能划分非常细,比如市场、销售、运营推广、用户调研、市场调研都是由不同的部门来承接,所以大厂的产品经理在工作中,需要跟多个部门进行跨部门协作和协调,才能把产品顺利上线。


\paragraph{咨询公司产品经理 12\sphinxfootnotemark[58]}
\label{\detokenize{chapter_introduction/PM:id18}}%
\begin{footnotetext}[58]\sphinxAtStartFootnote
\sphinxnolinkurl{https://zhuanlan.zhihu.com/p/347994504}
%
\end{footnotetext}\ignorespaces \begin{enumerate}
\sphinxsetlistlabels{\arabic}{enumi}{enumii}{}{.}%
\item {} 
研究并理解客户的战略、商业模式,挖掘并揭示客户的痛点和诉求

\item {} 
帮助客户识别商业机会并建议实施方案

\item {} 
引导需求探寻和创新思考工作坊,产出客户认可的解决方案

\item {} 
创建并清楚展示方案蓝图,确保客户和交付团队理解并达成共识

\item {} 
定义关键目标、成功要素,识别风险、挑战、依赖和约束

\item {} 
有效引导和促进 Product
Owner、客户出资人、行业专家、技术团队、最终用户间的沟通和协作,保证产品从概念、到原型、到上线及运营的端到端交付

\end{enumerate}


\subparagraph{创业公司}
\label{\detokenize{chapter_introduction/PM:id19}}
创业公司的产品经理需要具备的关键技能

与大厂不同的是,创业公司产品经理的关键技能主要体现在 3 方面:
\begin{enumerate}
\sphinxsetlistlabels{\arabic}{enumi}{enumii}{}{.}%
\item {} 
领导力

\item {} 
魄力

\item {} 
凝聚力

\end{enumerate}

创业公司产品经理的工作职责有 4 个关键点:
\begin{enumerate}
\sphinxsetlistlabels{\arabic}{enumi}{enumii}{}{.}%
\item {} 
制定方向和策略

\end{enumerate}

在产品的初期,产品经理需要参与公司和产品愿景和规划的过程,从制定产品方向和策略开始,而不仅仅是考虑产品功能的设计。
\begin{enumerate}
\sphinxsetlistlabels{\arabic}{enumi}{enumii}{}{.}%
\setcounter{enumi}{1}
\item {} 
全流程参与

\end{enumerate}

创业公司的产品经理需要参与到产品的所有环节,比如从产品远景、规划、原型设计、交互设计、视觉设计、开发上线的每一个环节。
\begin{enumerate}
\sphinxsetlistlabels{\arabic}{enumi}{enumii}{}{.}%
\setcounter{enumi}{2}
\item {} 
发挥空间大

\end{enumerate}

创业公司的产品经理需要主动承担和负责产品的整个生命周期,凝聚团队成员协作,发挥空间较大。
\begin{enumerate}
\sphinxsetlistlabels{\arabic}{enumi}{enumii}{}{.}%
\setcounter{enumi}{3}
\item {} 
高风险

\end{enumerate}

大厂的产品可能是已经成型、上线、有一定数量的客户,但是创业公司的产品需要试错,并不知道产品推向市场以后的反应是怎样的,所以具有相对较大的风险。


\paragraph{结果 3\sphinxfootnotemark[59]}
\label{\detokenize{chapter_introduction/PM:id20}}%
\begin{footnotetext}[59]\sphinxAtStartFootnote
\sphinxnolinkurl{http://www.woshipm.com/pmd/3945349.html}
%
\end{footnotetext}\ignorespaces 
产品设计结果:高效快速的将需求产品化,面对同样问题或需求,更好的解决方案、更少的开发量、更快的上线。举例,用半年做出来的和用2个月做出来的同功能、扩展性、结果的东西,投资收益后者是前者的3倍,这之间的差值,是产品经理之间的差值。这里更多的强调是“把事情做对”,即事情分给你,可以以最高性价比的方式做出来,做好。

数据结果:用户对产品的使用情况,更准确、更多、更系统的挖掘用户的场景,系统性的解决场景背后的问题,并使得上线之后的产品得到更多用户的认可和使用。同样是花了2个月优化了某模块,有的产品经理可以让模块使用人数增2倍,有的产品经理只可以让模块使用人数提升20\%,有的甚至优化之后使用量还下降。这些数据之间的差值是产品经理之间的差值。

商业结果:一方面是短期带来的收入,B端的新签价值、续约价值,C端广告费,文章阅读费用等。另一方面是长期带来的战略布局价值,如产品矩阵的构建,产品架构支撑大客户的扩展,支撑在某个领域的布局等。


\paragraph{产品思维与技术思维的区别 4\sphinxfootnotemark[60]}
\label{\detokenize{chapter_introduction/PM:id21}}%
\begin{footnotetext}[60]\sphinxAtStartFootnote
\sphinxnolinkurl{http://www.woshipm.com/pmd/1629952.html}
%
\end{footnotetext}\ignorespaces 
\begin{figure}[H]
\centering
\capstart

\noindent\sphinxincludegraphics{{tech_product0}.jpeg}
\caption{技术VS产品}\label{\detokenize{chapter_introduction/PM:id33}}\end{figure}
\begin{itemize}
\item {} 
\sphinxstylestrong{产品经理}思考的是产品的用户价值和使用场景,同时还需要考虑产品所承载的业务闭环及商业价值

\item {} 
\sphinxstylestrong{工程师}看到产品设计后,在脑海里构建的是拆解后的技术实现要点,好比一栋房子的内部结构。对于一个产品,工程师需要先构建产品的技术架构,然后评估产品功能的技术成本。

\end{itemize}

\begin{figure}[H]
\centering
\capstart

\noindent\sphinxincludegraphics{{tech_product}.jpeg}
\caption{技术VS产品的分工}\label{\detokenize{chapter_introduction/PM:id34}}\end{figure}

产品经理是发现需求后做产品策略做对的产品,例如:当快手2011年开始上市场运营,而今日头条系从2016年才开始做抖音,那么如果你是技术思维的话,你准备研究比快手更好的AI模型?然后超越快手吗?

那我们看抖音的产品负责人士怎么运用产品思维做产品策略的。

首先AI技术模型全世界都是公开的,这一点从产品角度看没有门槛。

另外抖音的产品一下子发三款,分别是:
\begin{enumerate}
\sphinxsetlistlabels{\arabic}{enumi}{enumii}{}{.}%
\item {} 
跟快手一模一样的纯粹类UGC平台火山小视频;

\item {} 
较长视频西瓜视频平台;

\item {} 
做一款又类PGC优质内容的平台抖音,在同时从市场收购一款。2017年11月10日头条以10亿美元购北美音乐短视频社交平台Musical.ly,与抖音合并。

\end{enumerate}

如果头条是技术思维的话,通过技术逆向看Musical.ly源码,会不出意外发现我们也能做呀,我们技术比Musical.ly还好。

笔者建议以上思想想在AI时代做产品经理一定要买本《AI+时代产品经理的思维方法》一书,好好读读产品经理的本质是啥。

例如:上面的例子再分析,如果头条是技术思维抖音早就被2018年腾讯系的微视干死了,还哪里会等你慢慢开发一个Musical.ly。


\paragraph{Awesome}
\label{\detokenize{chapter_introduction/PM:awesome}}
\sphinxurl{https://www.yuque.com/books/share/2325abf6-ed56-4941-bf99-94edeb122076}?\#\%20\%E3\%80\%8A\%E4\%BA\%A7\%E5\%93\%81API:\%E8\%BF\%9B\%E9\%98\%B6\%E5\%85\%A8\%E6\%A0\%88PM\%E6\%89\%8B\%E5\%86\%8C\%E3\%80\%8B

社区:
\begin{itemize}
\item {} 
UCD大社区: www.ucdchina.co

\item {} 
腾讯CDC: \sphinxurl{http://cdc.tencent.com}

\item {} 
淘宝UED: \sphinxurl{http://ued.taobao.com}

\item {} 
百度UED: \sphinxurl{http://ued.baidu.com/}

\item {} 
\sphinxurl{http://www.pmtalk.club/}

\item {} 
\sphinxurl{https://www.pmcaff.com/}

\item {} 
\sphinxurl{https://www.woshipm.com/}

\item {} 
\sphinxurl{https://dh.woshipm.com/\#section-16}

\item {} 
\sphinxurl{http://www.crazypm.com/}

\end{itemize}


\paragraph{我适合当产品经理吗10\sphinxfootnotemark[61]}
\label{\detokenize{chapter_introduction/PM:id22}}%
\begin{footnotetext}[61]\sphinxAtStartFootnote
\sphinxnolinkurl{https://www.bilibili.com/video/BV1qv411B7J1}
%
\end{footnotetext}\ignorespaces \begin{enumerate}
\sphinxsetlistlabels{\arabic}{enumi}{enumii}{}{.}%
\item {} 
你要想上班

\item {} 
不轻松躺着赚钱

\item {} 
发展比稳定更重要

\item {} 
学历是影响因素

\item {} 
轻松还赚大钱不存在

\item {} 
想创业,产品是关键

\item {} 
性格偏中性些

\end{enumerate}


\paragraph{天赋 17\sphinxfootnotemark[62]}
\label{\detokenize{chapter_introduction/PM:id23}}%
\begin{footnotetext}[62]\sphinxAtStartFootnote
\sphinxnolinkurl{https://www.zhihu.com/question/22113339/answer/1418832617}
%
\end{footnotetext}\ignorespaces 
A 类:有深度思考能力或超常同理心

对产品经理来说,深度思考是指习惯思考事物背后的本质,且在同等条件下,对事物的洞察更深或更快。能深度思考的人很少见,但只有借助于深度思考,在微观场景和宏观背景下发现并理解事物的共性、差异性和各种因果关系,才能在这个现实世界中不断总结出规律和特点,提高未来决策和行为的成功率。

知人知面不知心,科学方法只能高效处理客观行为,行为背后的心理动机却无法确定和验证,这就需要产品经理带着同理心来工作。同理心是指能够站在别人的角度去思考,并准确地察觉和判断别人的感受。同理心是天赋本能,每个人多少都会有,后天也能通过刻意训练适度提高。当然,有超常同理心的人也很少见,但一旦有,做产品经理就极具优势。

世界上永远不会有两场相同的战争,产品经理也面临相同情况,永远要在变化的环境中去发现和解决新问题,这其实是一个要永远保持创造性的工作,如果产品经理的先天天赋占优,同等条件下的创造性和输出能力也会占优。

A
类产品经理很少见,这跟智商、经验、级别都不一定有关,更多是跟特殊天赋和潜力有关。事后分析一个产品或行业的得失和规律相对容易,很多人都能做得不错,但当产品和行业还处于结局不确定的发展过程中,就能更早更深地察觉到市场需求和行业方向的特质是企业最希望产品经理拥有的,这也是我们总在努力寻找
A 类产品经理的原因。

A 类人才里面当然也会有强弱之分,但是,只要符合 A
类标准就够了,甚至只符合 B
类标准,掌握了科学方法又经过充分实践历练,也够了。因为,对于大多数产品经理来说,创造成功产品的主要瓶颈还是机遇,如果能够抓住好的时代机遇,时代会推着你走。

潜力和优势来源: \sphinxhref{https://github.com/JoJoDU/Book\_Notes/issues/3}{18}%
\begin{footnote}[63]\sphinxAtStartFootnote
\sphinxnolinkurl{https://github.com/JoJoDU/Book\_Notes/issues/3}
%
\end{footnote}
\begin{itemize}
\item {} 
感兴趣的领域做到勤奋和自省

\item {} 
利他,替众人着想和想众人所想——市场导向型PM

\item {} 
产品实践经历

\end{itemize}


\paragraph{未来能成为优秀的产品经理}
\label{\detokenize{chapter_introduction/PM:id24}}\begin{enumerate}
\sphinxsetlistlabels{\arabic}{enumi}{enumii}{}{.}%
\item {} 
10\textasciitilde{}20w。目标不清晰,行动能力弱。

\item {} 
20\textasciitilde{}50w。目标清晰,行动能力强。

\item {} 
50w+。目标清晰,有干劲、胆量。

\end{enumerate}


\paragraph{“抄”,“超”,“钞” 19\sphinxfootnotemark[64]}
\label{\detokenize{chapter_introduction/PM:id25}}%
\begin{footnotetext}[64]\sphinxAtStartFootnote
\sphinxnolinkurl{https://wen.woshipm.com/question/detail/c5toar.html?sf=wipm}
%
\end{footnotetext}\ignorespaces \begin{itemize}
\item {} 
“抄”:就是抄袭,只有你研究的竞品和你现在做的业务差不多,那就直接抄,最起码人家做的这些在市场上已经验证了,用户也接收了,只要你理解了他的逻辑直接拿过来没什么问题。

\item {} 
“超”:既然抄袭了,总不能一辈子跟着后面走,产品上线后接收反馈就要有超越和优化的想法,有些地方确实用户不适合的话就需要懂脑子进行优化,超越你所抄袭的竞品。

\item {} 
“钞”:顾名思义就是钱了,只要产品做得好,肯定就可以给公司带来效益和价值,自然而然你也会得到更多的资源和奖励。

\end{itemize}


\paragraph{阶段}
\label{\detokenize{chapter_introduction/PM:id26}}\begin{itemize}
\item {} 
产品经理阶段:我自己在做这个岗位,也会服务产品经理同行。

\item {} 
产品思维阶段:我去服务泛产品经理,抽象出背后相对通用的思维方式,去影响更多人。

\item {} 
产品创新阶段:我认识到产品思维是方法,而产品创新是目的,更直接地,从想到做,从思维方式到做事方法,更落地。

\end{itemize}


\paragraph{能力模型}
\label{\detokenize{chapter_introduction/PM:id27}}
\begin{figure}[H]
\centering
\capstart

\noindent\sphinxincludegraphics{{PM_ability}.png}
\caption{产品经理能力模型}\label{\detokenize{chapter_introduction/PM:id35}}\end{figure}


\paragraph{产品经理成熟的标准是什么? 16\sphinxfootnotemark[65]}
\label{\detokenize{chapter_introduction/PM:id28}}%
\begin{footnotetext}[65]\sphinxAtStartFootnote
\sphinxnolinkurl{https://zhuanlan.zhihu.com/p/38392075}
%
\end{footnotetext}\ignorespaces 
看他做一款创新型产品时,更依赖竞品调研还是独立判断。站在巨人的肩膀上是没错,但前瞻性的方案更依赖人性洞察和市场嗅觉。


\subsubsection{AI}
\label{\detokenize{chapter_introduction/AI:ai}}\label{\detokenize{chapter_introduction/AI::doc}}

\paragraph{细分}
\label{\detokenize{chapter_introduction/AI:id1}}
\begin{figure}[H]
\centering
\capstart

\noindent\sphinxincludegraphics{{AI_class}.jpg}
\caption{AI类别}\label{\detokenize{chapter_introduction/AI:id11}}\end{figure}


\paragraph{AI在其中扮演什么角色 6\sphinxfootnotemark[66]}
\label{\detokenize{chapter_introduction/AI:ai-6}}%
\begin{footnotetext}[66]\sphinxAtStartFootnote
\sphinxnolinkurl{https://www.zhihu.com/people/hanniman-2/posts?page=2}
%
\end{footnotetext}\ignorespaces 
AI革命可以看作是生产力的革命,从生产力的角度讲,第一是将人类从现实世界的非创造性劳动当中解放出来,从而更快速的向虚拟世界迁移;第二是赋予创造性劳动更低的门槛,以建设更丰富的虚拟世界。

这两点我总结为叫做对人类的“去工具化”,就是说,人之所以为人,是有人固有的价值,比如“想象力创造力”,“理解另一个人类需求的共情能力”。这些很难被机器替代。而人类完成自我实现,却需要掌握大量复杂工具,逐渐将自己培养成工具。比如,你有很好的想象力,却不会有画笔,也难以完成一幅画作。掌握画笔本身就是“工具化”。

但是我认为AI可以帮助人类实现“去工具化”,真正“身随意动”的发挥人之所以为人的价值,具体就是依靠上述两点。


\paragraph{可能优势 9\sphinxfootnotemark[67]}
\label{\detokenize{chapter_introduction/AI:id2}}%
\begin{footnotetext}[67]\sphinxAtStartFootnote
\sphinxnolinkurl{https://pair.withgoogle.com/chapter/user-needs/}
%
\end{footnotetext}\ignorespaces \begin{itemize}
\item {} 
向不同的用户推荐不同的内容。比如为电影提供个性化的建议。

\item {} 
对未来事件的预测。例如,显示11月下旬飞往丹佛的机票价格。

\item {} 
个性化改善了用户体验。随着时间的推移,个性化的自动家用恒温器使家庭更舒适,恒温器更高效。

\item {} 
自然语言理解。听写软件要求人工智能能够很好地适应不同的语言和说话风格。

\item {} 
对一整类实体的识别。把每一张脸都编程进照片标签应用程序是不可能的,它使用人工智能来识别同一个人的两张照片。

\item {} 
检测随时间变化的低发生事件。信用卡诈骗不断演变,很少发生在个人身上,但却经常发生在一大群人身上。人工智能可以学习这些不断演变的模式,并在出现新的欺诈类型时发现它们。

\item {} 
特定领域的代理或机器人体验。对于大量用户来说,酒店预订遵循类似的模式,并且可以实现自动化,以加快过程。

\item {} 
显示动态内容比显示可预测的界面更有效。来自流媒体服务的人工智能建议会显示用户几乎不可能找到的新内容。

\end{itemize}


\paragraph{可能劣势}
\label{\detokenize{chapter_introduction/AI:id3}}\begin{itemize}
\item {} 
保持可预测性。有时候,核心体验中最有价值的部分是其可预测性,而不考虑上下文或额外的用户输入。例如,当“Home”或“Cancel”按钮停留在相同的位置时,它更容易作为逃生通道使用。

\item {} 
提供静态或有限的信息。例如,信用卡输入表单是简单的、标准的,并且对于不同的用户没有非常不同的信息需求。

\item {} 
最小化代价高昂的错误。如果错误的代价非常高,超过了成功率的小幅提高带来的好处,比如导航指南建议一条越野路线,以节省几秒钟的旅行时间。

\item {} 
完整的透明度。如果用户、客户或开发人员需要准确地理解代码中发生的一切,就像开源软件一样。人工智能并不总是能提供那种程度的解释。

\item {} 
高速和低成本的优化。对于该业务来说,开发速度和率先进入市场是否比其他任何事情都重要,包括添加人工智能将带来的价值。

\item {} 
自动化的高价值的任务。如果有人明确告诉你,他们不想要一项由人工智能自动完成或增强的任务,这是一个不要试图破坏的好任务。我们将在下面更多地讨论人们如何评价某些类型的任务。

\end{itemize}


\paragraph{人工智能行业吐槽}
\label{\detokenize{chapter_introduction/AI:id4}}

\subparagraph{鱼龙混杂}
\label{\detokenize{chapter_introduction/AI:id5}}
伴随着行业持续火热,资金流不断涌入,现状却是整个行业内对技术、业务、商务都精通的产品大咖非常之少,滥竽充数的人很多。能够对行业的技术边界了然于胸,又对这个行业的产业链、利益链有深入理解的人才不可多得,大厂哄抢。有人戏称目前很多人工智能产品都是“人工智障”,可见该行业要实现真正的产业化、产品化,还有很大的空间。


\subparagraph{概念空洞}
\label{\detokenize{chapter_introduction/AI:id6}}
我曾笑称,进入这个行业真是感觉到中华文字的博大精深,把很多早就出现的技术名词玩文字游戏包装一下,突然就变得高大上起来了。天天张口闭口“动态时空库”、“计算引擎”、“一人一档”、“端到端解决方案”、“AI赋能”等等,其实稍微了解一下就发现,“动态时空库”不就是摄像头抓拍,“计算引擎”不就是服务器、“一人一档”不就是数据分组、“端到端解决方案”不就是软硬件都有、“AI赋能”不就是算法能力。但是这个行业就是这样的现状,只有包装了才有爆点,包装了才能融资,融资了才更需要噱头去营销,你也很难说这是良性循环,还是恶性循环。AI行业的最核心还是算法,传统研发人员会在算法这个盒子外面加一层包装,用所谓的云平台、互联网接口去封装,产品设计会在研发的基础上再加一层包装,解决方案会在产品基础上再加一层包装,当用户通过品宣与销售之口了解AI时已经在怀疑AI是不是快要取代人类了。所以才导致了大众认知和现实能力之间有巨大的鸿沟,目前的行业才不断的强调AI决胜在落地。只有有开创性的产品落地,才能弥补公众认知与现实能力的缺口。


\subparagraph{方案同质化}
\label{\detokenize{chapter_introduction/AI:id7}}
如果你稍微深入的了解过这个行业,你大概会与我有同样的想法,如果从非算法人员的角度来讲,这个行业的技术类别并没有那么复杂,相比于已经发展成熟的电力行业、电子行业、通信行业,其实它的知识宽度还算单纯,相对比较容易梳理清楚。再加上行业产品同质化严重,基本上这个行业的方案就是你抄我,我抄你,谁都说自己是首创,谁都从不同的角度去宣传自己是第一。很多概念也不知道是谁第一个提出,反正渐渐的就发现行业内各家都这么说。目前整个CV领域,基本上to
B和to G就集中在安防领域,to
C就集中在手机端的图像软件处理上以及金融认证比对上了,除此之外真的很难找到什么可圈可点的应用亮点。


\paragraph{人工智能层次2\sphinxfootnotemark[68]}
\label{\detokenize{chapter_introduction/AI:id8}}%
\begin{footnotetext}[68]\sphinxAtStartFootnote
\sphinxnolinkurl{https://easyai.tech/blog/ai-pm-knowledge/}
%
\end{footnotetext}\ignorespaces 
\begin{figure}[H]
\centering
\capstart

\noindent\sphinxincludegraphics{{ceng}.jpg}
\caption{AI应用层、技术层、基础层}\label{\detokenize{chapter_introduction/AI:id12}}\end{figure}


\paragraph{人工智能几问3\sphinxfootnotemark[69]}
\label{\detokenize{chapter_introduction/AI:id9}}%
\begin{footnotetext}[69]\sphinxAtStartFootnote
\sphinxnolinkurl{https://www.sohu.com/a/364264851\_114819}
%
\end{footnotetext}\ignorespaces \begin{enumerate}
\sphinxsetlistlabels{\arabic}{enumi}{enumii}{}{.}%
\item {} 
人工智能和互联网时代的不同是什么?

\end{enumerate}

互联网主要是重构生产要素(即重构商业模式),人工智能则是升级生产要素。

比如在出行领域,出行平台直接连接了司机和乘客,重构了线上、线下的出行流程;但是人工智能则是从自动驾驶技术切入,重构了车辆和司机本身。
\begin{enumerate}
\sphinxsetlistlabels{\arabic}{enumi}{enumii}{}{.}%
\setcounter{enumi}{1}
\item {} 
人工智能没有普及的原因是什么?

\end{enumerate}

医疗领域、自动驾驶等,容错度低\sphinxhref{http://www.ramywu.com/work/2017/08/20/Product-Orientation/}{5}%
\begin{footnote}[70]\sphinxAtStartFootnote
\sphinxnolinkurl{http://www.ramywu.com/work/2017/08/20/Product-Orientation/}
%
\end{footnote},在准确率不够或样本不够多,满足不了安全需求,不敢普及。

计算特斯拉的事故率时,样本是很少的,对比基于整个社会上的车辆数和里程数。

只有等到特斯拉自动驾驶的车辆数和里程数积累到一定量级,样本足够大后,才能和人工驾驶的事故率进行比较,也才能真正证明自动驾驶是否更优于人工驾驶。
\begin{enumerate}
\sphinxsetlistlabels{\arabic}{enumi}{enumii}{}{.}%
\setcounter{enumi}{2}
\item {} 
AI
在什么场景下才能发挥出最大的作用?\sphinxhref{http://www.ramywu.com/work/2017/08/20/Product-Orientation/}{5}%
\begin{footnote}[71]\sphinxAtStartFootnote
\sphinxnolinkurl{http://www.ramywu.com/work/2017/08/20/Product-Orientation/}
%
\end{footnote}

\end{enumerate}

人工的优势是:可以解决创造性质的问题,复杂判断的问题。而 AI
的优势有哪些呢?在什么场景下才能发挥出最大的作用?

(1)数据量规模庞大,人工速度拼不过的时候,比如:
\begin{itemize}
\item {} 
在机场安防监控,肉眼一个个识别 拼不过 AI 人脸1:N快速识别;

\item {} 
快递行业尤其是双十一,每天都几百万的数据量,在做分拣时候,工业拍照扫描分拣和肉眼\sphinxhyphen{}
分拣都经常出错,10\%\sphinxhyphen{}20\%的出错率都会造成巨大的损失;

\item {} 
出版社、公众号编辑每天会处理大批量文字;

\end{itemize}

(2)简单且重复、精细的,人肉无法快速识别时,比如:

简单+重复:
\begin{itemize}
\item {} 
快递员每天都要发快递和联系收件人,而输入快递单里的手机号会很辛苦,内置系统通过快\sphinxhyphen{}
递单 OCR 识别能快速发送到联系人;

\item {} 
微信编辑写完文章还要人工做枯燥重复的文字检查,速度很慢,出错率高,急切需要提升文字的发布速度;

\end{itemize}

精细:
\begin{itemize}
\item {} 
检测人脸中两只眼睛的距离,机器是可以计算的,而肉眼做不到;

\item {} 
处理初级的错误,如形近字,肉眼也看不见如此微妙的错误;

\end{itemize}


\paragraph{在To B产品中可以替代人工劳动力的例子: 8\sphinxfootnotemark[72]}
\label{\detokenize{chapter_introduction/AI:to-b-8}}%
\begin{footnotetext}[72]\sphinxAtStartFootnote
\sphinxnolinkurl{http://www.crazypm.com/zixun/102296.html}
%
\end{footnotetext}\ignorespaces \begin{itemize}
\item {} 
腾讯觅影(\sphinxurl{http://t.cn/RYRDSmI} ):替代医生的部分职责;

\item {} 
百度Apollo(\sphinxurl{http://apollo.auto/} ):完全替代汽车驾驶员的职责;

\item {} 
商汤\sphinxhyphen{}公安人脸识别智能(\sphinxurl{http://t.cn/RYRD0zo}
):替代公安人员的部分职责;

\item {} 
网易七鱼\sphinxhyphen{}智能客服(\sphinxurl{http://t.cn/RYRDYwY} ):替代客服人员的部分职责;

\item {} 
UIzard(\sphinxurl{http://t.cn/RYRD89b} ):替代前端工程师的部分职责;

\item {} 
鲁班设计AI(\sphinxurl{http://t.cn/RYRD3y1} ):替代UI设计师的部分职责;

\item {} 
.Boomtrain的智能营销平台(\sphinxurl{http://t.cn/RYRDdYk}
):替代营销人员的部分职责;

\item {} 
京东仓库机器人(\sphinxurl{http://t.cn/RYRDsfH}
):完全替代仓库库管、分拣员、包装员等各种角色;

\item {} 
阿里巴巴天巡(\sphinxurl{http://t.cn/RYRkhsC} ):替代服务器运维人员30\%的工作;

\item {} 
Abyss Creations娃娃(\sphinxurl{http://t.cn/RCi65Q7} ):替代….(自己去看吧)

\end{itemize}

产品经理只有先除掉PC时代的上亿PV,移动互联网时代的数亿DAU,在产品经理眼中的障碍,才能看得清AI时代并解决PC和移动互联网时解决不了的痛点。


\paragraph{AI产品与传统产品的区别:}
\label{\detokenize{chapter_introduction/AI:id10}}\begin{enumerate}
\sphinxsetlistlabels{\arabic}{enumi}{enumii}{}{.}%
\item {} 
AI产品诞生的市场背景是甚至一个垂直的细分领域均有一个APP产品的市场环境,这个时候需要AI产品做到比原来的产品好上10倍的体验或者比原来的产品快10倍以上才能赢得市场的环境。

\item {} 
在做纯APP的时候是不需要考虑供应链的,但是由于广义范畴上的AI产品是\sphinxstylestrong{从数据获取到数据分析再到数据应用上,少不了硬件等外设}的采用,例如:用深度摄像头采集更多的数据,采用NB\sphinxhyphen{}IoT采集人和物体的行为数据,均需要硬件的融合。AI产品是更加考验产品经理综合素质的,除了设计管理好传统的软件上下游之外,还融入了供应链产业的深挖,例如:当你的摄像头与AI主体硬件产品出现BUG的时候,你需要联系的事摄像头生产厂商,而不像APP时代仅仅需要再成熟的手机上研发即可,这个时候需要产品调动的是摄像头整个研发甚至一个工厂来配合你。

\item {} 
需求的变化有:

\end{enumerate}
\begin{itemize}
\item {} 
例如:新零售,用户需要货来匹配人,这里需要LBS和更多智能传感器的数据来服务人。

\item {} 
例如:线下商铺原来是不知道哪个用户来逛街,哪个潜在消费者在哪个商品前停留的更久,节假日购买热销商铺结账需要排队等等需求正好使得AI产品得以展身手的时刻。

\end{itemize}


\subsubsection{AI产品经理}
\label{\detokenize{chapter_introduction/AI_PM:ai}}\label{\detokenize{chapter_introduction/AI_PM::doc}}
一个合格的AI产品经理在具备上述传统产品经理能力模型的基础上,应该熟悉AI技术的效能与边界,对AI产业的三驾马车算法、算力、数据有一定的理解(对PM来说不要求上手coding,但是要对实现原理要理解),并且对部分垂直场景业务逻辑深耕,甚至达到跨领域协作的产品境界。展开来谈,这部分,我围绕算法、算力、数据、硬件、业务五个维度来解释AI产品经理所需要的能力加成。\sphinxhref{https://www.jianshu.com/p/fd466ed1bda6}{2}%
\begin{footnote}[73]\sphinxAtStartFootnote
\sphinxnolinkurl{https://www.jianshu.com/p/fd466ed1bda6}
%
\end{footnote}


\paragraph{FMCG PM}
\label{\detokenize{chapter_introduction/AI_PM:fmcg-pm}}
快速消费品时代,产品经理其实是品牌经理,他的主要职责是负责这个品牌的市场定位、营销推广、渠道建设等等。产品本身是高度同质化的,一个产品究竟是一亿量级还是十亿量级,取决于产品经理的\sphinxstylestrong{品牌营销能力}。


\paragraph{IPM}
\label{\detokenize{chapter_introduction/AI_PM:ipm}}
IPM (Internet Product
Manager),互联网产品经理是互联网公司中的一种职能,负责互联网产品的计划和推广,以及互联网产品生命周期的演化。根据所负责的互联网产品是用户产品还是商业产品,可以分为互联网用户产品经理和互联网商业产品经理。用户产品经理最关心的是互联网用户产品的用户体验,商业产品经理最关心的是互联网商业产品的\sphinxstylestrong{流量变现能力}。

互联网产品经理还可以有如下2个分类维度:
\begin{itemize}
\item {} 
“产品形态”维度:PC客户端产品经理、网站产品经理(Web/H5)、移动端产品经理(iOS/Android)、Server产品经理……

\item {} 
“行业领域”维度:工具产品经理、社交产品经理、电商产品经理、O2O产品经理……

\end{itemize}


\subparagraph{“互联网产品经理演进史”}
\label{\detokenize{chapter_introduction/AI_PM:id1}}
消费品时代,产品经理(Product
Manager)的本质是“营销产品经理”。因为需求相对明确、产品同质化、生产标准化。
软件时代,PM本质是“项目产品经理”。因为需求相对容易明确,用户对产品体验要求不高(选择少,必须用)——PM“在管理生产上更能创造价值,沟通协调,版本控制,按时交付。”
互联网时代,PM本质是“需求产品经理”。因为需求+体验,能产生更大价值。


\subparagraph{互联网时代特性 17\sphinxfootnotemark[74]}
\label{\detokenize{chapter_introduction/AI_PM:id2}}%
\begin{footnotetext}[74]\sphinxAtStartFootnote
\sphinxnolinkurl{https://github.com/JoJoDU/Book\_Notes/issues/3}
%
\end{footnotetext}\ignorespaces 
一、
\begin{itemize}
\item {} 
信息复制分发的边际成本降低 \sphinxstyleemphasis{分析} 关键变化:连接让信息快速交互传递 –>
全新的产品类别:在线信息产品 –> 互联网PM 信息分发历史:信件 –>
报纸、杂志 –> 软件(光盘、软盘)

\item {} 
信息过载
\sphinxhref{https://weread.qq.com/web/reader/0c032c9071dbddbc0c06459k70e32fb021170efdf2eca12}{31}%
\begin{footnote}[75]\sphinxAtStartFootnote
\sphinxnolinkurl{https://weread.qq.com/web/reader/0c032c9071dbddbc0c06459k70e32fb021170efdf2eca12}
%
\end{footnote}
\sphinxstyleemphasis{分析}由于互联网的信息的快速传输、数据的海量共享,使得大量冗余的信息充斥着人们的眼球。

\item {} 
用户量巨大 多生产、多供给一份信息的边际成本降低 –>
信息传播极快;免费信息 –> 转化用户群极大扩展; \sphinxstylestrong{导致}
在线信息产品极易爆发增长
互联网使信息传播和沟通更加高效,用户更易接触、度量、比较不同的产品;
\sphinxstylestrong{导致}市场竞争激烈,出 现马太效应

\end{itemize}

二、
\begin{itemize}
\item {} 
快速迭代:市场反馈快,产品生产交付快,分发快;降低试错成本

\item {} 
数据(反馈和信息)、AB测试(实现更大规模和更快速的迭代)、千人千面(基于用户反馈数据快速验证和迭代)
\sphinxstylestrong{PM上线能力:理解用户、理解交易}

\end{itemize}

三、
\begin{itemize}
\item {} 
体验设计价值增大

\item {} 
需求 + 体验:以用户为核心,横向组织资源,按需求生产和销售

\end{itemize}


\subparagraph{互联网 VS AI}
\label{\detokenize{chapter_introduction/AI_PM:vs-ai}}
\begin{figure}[H]
\centering
\capstart

\noindent\sphinxincludegraphics{{Internet_VS_AI}.png}
\caption{互联网时代 VS AI时代\sphinxhref{https://zhuanlan.zhihu.com/p/43888627}{27}\sphinxfootnotemark[76]}\label{\detokenize{chapter_introduction/AI_PM:id33}}\end{figure}
%
\begin{footnotetext}[76]\sphinxAtStartFootnote
\sphinxnolinkurl{https://zhuanlan.zhihu.com/p/43888627}
%
\end{footnotetext}\ignorespaces 

\subparagraph{AI产品经理与之区别 8\sphinxfootnotemark[77]}
\label{\detokenize{chapter_introduction/AI_PM:ai-8}}%
\begin{footnotetext}[77]\sphinxAtStartFootnote
\sphinxnolinkurl{https://easyai.tech/blog/ai-pm-knowledge/}
%
\end{footnotetext}\ignorespaces 
\begin{figure}[H]
\centering
\capstart

\noindent\sphinxincludegraphics{{Product_center}.png}
\caption{产品}\label{\detokenize{chapter_introduction/AI_PM:id34}}\end{figure}


\subparagraph{数据集优先}
\label{\detokenize{chapter_introduction/AI_PM:id3}}
AI
PM首先创建/收集一个代表问题空间的数据集。只有这样,他们才会要求工程师对问题进行迭代,以提供需要解决的问题90\%的准确性。而不是画UX。

\sphinxurl{https://appen.com/}
\sphinxhref{https://medium.com/@fabian.kutschera/udacitys-ai-product-manager-a-review-2faba9ba3669}{26}%
\begin{footnote}[78]\sphinxAtStartFootnote
\sphinxnolinkurl{https://medium.com/@fabian.kutschera/udacitys-ai-product-manager-a-review-2faba9ba3669}
%
\end{footnote}

法律法规和国家政策:
\sphinxhref{https://www.bilibili.com/video/av800293586/}{28}%
\begin{footnote}[79]\sphinxAtStartFootnote
\sphinxnolinkurl{https://www.bilibili.com/video/av800293586/}
%
\end{footnote}
\begin{itemize}
\item {} 
数据与隐私保护(egGeneral Data ProtectionRegulation)

\item {} 
各种协议(开源软件、公开数据、公共数据集)

\item {} 
大数据、人工智能战略(eg教育领域)

\end{itemize}


\subparagraph{广泛涉猎 18\sphinxfootnotemark[80]}
\label{\detokenize{chapter_introduction/AI_PM:id4}}%
\begin{footnotetext}[80]\sphinxAtStartFootnote
\sphinxnolinkurl{http://www.woshipm.com/pd/2209024.html}
%
\end{footnotetext}\ignorespaces 
与互联网产品不同,AI产品经理需要广泛涉猎不同行业不同地域的操作习惯,借鉴硬件、软件行业的优秀的交互,不断总结,思考更为轻松、自然、平顺的产品体验。


\subparagraph{真正的“需求产品经理” 16\sphinxfootnotemark[81]}
\label{\detokenize{chapter_introduction/AI_PM:id5}}%
\begin{footnotetext}[81]\sphinxAtStartFootnote
\sphinxnolinkurl{https://mp.weixin.qq.com/s?\_\_biz=MjM5NzA5OTAwMA==\&mid=2650005725\&idx=1\&sn=75d33e7ae76805c9bd2db9d90147e27b\&chksm=bed8644a89afed5c69f23c01a86a601399269c9d604862305e94c89ba10890e951c550df0386\&scene=21\#wechat\_redirect}
%
\end{footnotetext}\ignorespaces 
AI行业(产品/市场)变化太快,而且是大调整。最多6个月,如果不去接触一线的情况,就会突然发现自己不熟悉市场了。

客户希望你解决他们的问题;他们不在乎你用的是哪种神经网络。你可能会发现自己根本不需要AI,这也没什么。

总之,机会多、难度大、变化又快又大,导致老板得承认自己的知识背景和精力有限,可能无法兼顾所有可能方向,必须让AI产品经理成为细分领域的小CEO,来做决策和承担更大压力。


\subparagraph{需求验证更为重要}
\label{\detokenize{chapter_introduction/AI_PM:id6}}
对于AI产品经理来说,产品能够达到60分的可用及格线往往比开发一款完美产品更为重要,只有我们能够通过技术验证场景,该一款产品才有可能进行商业落地,往往AI产品经理会遇到一个窘况,自己好不容易发现了一个场景需求,在个人看来通过人工智能手段能够大大提高效率,但是在经过半年甚至一年的技术验证之后发现并不能


\subparagraph{技术迭代更为快速}
\label{\detokenize{chapter_introduction/AI_PM:id7}}
研究人工智能翻译的人写的一段话,大致的含义就是他看了谷歌的一篇关于智能翻译的论文,发现自己之前所有的技术积累都已经落后了。关注前沿的行业技术更新就显得更为重要了,往往去年的某一个需求验证没有通过,但是随着时间推移又能够通过技术手段进行解决了


\subparagraph{确认自己是AI PM吗? 25\sphinxfootnotemark[82]}
\label{\detokenize{chapter_introduction/AI_PM:ai-pm-25}}%
\begin{footnotetext}[82]\sphinxAtStartFootnote
\sphinxnolinkurl{https://medium.com/@donnabella/what-does-it-mean-to-be-an-ai-product-manager-d67dc97da2e1}
%
\end{footnotetext}\ignorespaces \begin{itemize}
\item {} 
你是否首先查看了你的问题空间并收集了唯一的数据源?然后让开发者努力达到市场所要求的解决问题的准确度?

\item {} 
你了解你的AI性能/基准测试数据吗?

\item {} 
你的训练数据集是独特的和有区别的吗?

\item {} 
你的产品是否包含了所有合法的数据?

\item {} 
你是否一直在寻找额外的数据来源来继续改进你的产品?

\end{itemize}


\paragraph{AI产品经理的工作特点及发展方向:1\sphinxfootnotemark[83]}
\label{\detokenize{chapter_introduction/AI_PM:ai-1}}%
\begin{footnotetext}[83]\sphinxAtStartFootnote
\sphinxnolinkurl{https://www.boxuegu.com/news/4368.html}
%
\end{footnotetext}\ignorespaces 
把AI产品经理分为四个象限,分别是:
\begin{enumerate}
\sphinxsetlistlabels{\arabic}{enumi}{enumii}{}{.}%
\item {} 
突破算法产品经理,在大企业的研究部门、实验室,主要面向基础算法的更新迭代,强调对底层算法的逻辑理解、项目协调能力、竞品收集分析能力。在国内主要分布于BAT等一线互联网企业,或者讯飞、商汤等AI为主的企业;这类产品经理日常工作以研究为主,失败大于成功,不过没有苛刻的KPI,多为学术型人才。

\item {} 
创新产品经理,多为技术出身,在某个技术领域是个专家型人才。投入到初创公司,利用所掌握的技术能力,设计创新型产品,担任主要产品的设计工作,可以说是公司的关键人物,多是应用最新的前沿技术,结合垂直场景或领域,设计出创造型产品。

\item {} 
交付产品经理,多为产品出身,AI技术能力不是长项,但产品能力扎实,熟悉成熟AI技术,主要面向时间业务场景的AI落地,强调对行业的业务理解,典型场景分析,制定产品落地方案。

\item {} 
普及行业产品经理,多为非技术出身,熟悉成熟的AI技术能力,熟悉市场上成熟的AI产品,且具备深刻行业理解力,分析AI行业落地方向,能够很好的完成相关AI产品的拆解、分析、改造,从而制定产品整体规划方向,面向算法需求提出。

\end{enumerate}

\begin{figure}[H]
\centering
\capstart

\noindent\sphinxincludegraphics{{+AI+}.png}
\caption{+AI+\sphinxhref{https://www.bilibili.com/video/BV1MK411u7SH?p=10}{14}\sphinxfootnotemark[84]}\label{\detokenize{chapter_introduction/AI_PM:id35}}\end{figure}
%
\begin{footnotetext}[84]\sphinxAtStartFootnote
\sphinxnolinkurl{https://www.bilibili.com/video/BV1MK411u7SH?p=10}
%
\end{footnotetext}\ignorespaces 
\begin{figure}[H]
\centering
\capstart

\noindent\sphinxincludegraphics{{AIPM_ability}.png}
\caption{AI PM 能力模型}\label{\detokenize{chapter_introduction/AI_PM:id36}}\end{figure}


\paragraph{负责}
\label{\detokenize{chapter_introduction/AI_PM:id8}}\begin{itemize}
\item {} 
决定AI产品的核心功能、受众和预期用途(竞争优势)

\item {} 
评估输入数据管道,并确保它们在整个AI产品生命周期中得到维护

\item {} 
协调跨职能团队(数据工程、研究科学、数据科学、机器学习工程和软件工程)

\item {} 
决定关键界面和设计:用户界面和体验(UI/UX)和功能工程

\item {} 
将模型和服务器基础设施与现有软件产品集成

\item {} 
与ML工程师和数据科学家一起进行技术堆栈的设计和决策

\item {} 
交付AI产品并在发布后进行管理

\item {} 
与工程、基础设施和站点可靠性团队协调,以确保所有发布的特性都能得到大规模支持

\end{itemize}


\paragraph{主要能力}
\label{\detokenize{chapter_introduction/AI_PM:id9}}
随着AI行业的爆发,越来越多的产品经理进入到AI应用企业甚至是底层算法企业工作,AI产品经理主要能力:

1、
技术能力,AI产品常常要深入算法逻辑,产品经理不要求具备编码能力,但需要理解各团队的工作流程及模式特点,尤其是基础算法的业务模式。为了提需求写MRD和PRD、产品卖点、产品竞争优势、产品销售打法。
\begin{itemize}
\item {} 
算法:算法就是计算或者解决问题的步骤。想用计算机解决特定的问题,就要遵循相应的算法。

\item {} 
算力:算力简单来说就是实现算法功能的资源要求,可以按照云、边、端来分类,云即云计算、边即嵌入式、端泛指服务器类资源,而这三者的背后核心都是集成电路,也就是芯片。嵌入式硬件:包含嵌入式微处理器、存储器(SDRAM、ROM、Flash等)、通用设备接口和I/O接口(A/D、D/A、I/O等)。实现一款产品往往还包含了外围元器件,比如
GPS、气压计、超声波、PIR
等等。作为PM的你就开始想了,需要多大的算力、运行内存预计需要多少、人脸库以及视频/照片存储需要多大的空间。是手动唤醒设备还是无感的呢?分别用什么元器件可满足需求呢?满足需求的情况下,预计硬件成本是多少?性能是否足够?

\item {} 
数据:数据包括这两层理解。第一层意思通用数据分析能力,这里不仅包括针对PV/UV/访问时长/新用户人数等运营指标的数据分析,更深一点的还会包括以分布式数据处理为核心的大数据技术(hadoop/spark/Hbase/Kafka等等),当然产品经理去了解大数据知识不是为了开发,而是为了在产品设计之初就协同研发一起评估中长期的技术需求和能力边界。
监督学习:不断地用标注后的数据去训练模型,不断调整模型参数,得到指标数值更高的模型。除此之外延伸的数据清晰、数据入库、数据验证、数据可视化等工作

\item {} 
硬件:AI硬件产品经理来说,还需要关注摄像头、门禁设备、传感器等硬件知识。

\end{itemize}

\sphinxstylestrong{以无人机产品为例}

硬件方面:
\begin{itemize}
\item {} 
处理器:计算平台,需要满足飞控、视觉算法(目标跟踪、Vslam等)算力要求。

\item {} 
传感器:GPS、激光雷达、视觉传感器、陀螺仪等数据采集器件,需要满足功能效果。

\end{itemize}

算法方面:
\begin{itemize}
\item {} 
飞控算法:飞控的性能要求,PID 控制、目标跟踪、手势识别等

\item {} 
定位导航:因为飞机不是在单一环境下运行,有可能GPS信号没有/不佳,光线环境佳等。需要多传感器融合,比如视觉+气压计+超声波+GPS
。

\item {} 
虽然算法产品经理有深度,但需要更高的广度去完成一个产品设计。

\end{itemize}

数据方面:
\begin{itemize}
\item {} 
各传感器的效果决定了数据的质量。

\end{itemize}

2、
分析及沟通能力,因为AI常常会涉及底层AI算法、工程化SDK开发、业务PASS中台开发、前端业务开发、智能硬件等多个团队,因此产品经理需要具备整体思维,对端到端的整体架构及相互模式有总分认知,能清晰定位问题点,具备与业务端及技术端的翻译能力,这样才能有效定位问题点,将各团队业务有效协同。

3、业务能力,随着AI技术的日益成熟,AI的行业落地正成为重点关注的方向,因此产品经理需要有行业知识及业务落地分析能力,不用了解细化业务流程,但需要清晰典型业务场景。

对于AI产品经理来说,思考的核心可以有两个走向(开源节流),第一个走向是传统问题能否利用AI技术更低成本的解决(节流),第二个走向是是否能利用AI技术创造需求并创造付费模式(开源)。

同时,我认为术业有专攻,优秀的产品一定是在特定垂直领域反复打磨的,因此一个合格的AI产品经理对某个业务领域一定要有深耕的,才会厚积薄发创造出更有沉淀的产品,AI产品经理本质上还是一个产品经理,一定是以业务为导向的。


\subparagraph{怎么衡量“懂技术”7\sphinxfootnotemark[85]}
\label{\detokenize{chapter_introduction/AI_PM:id10}}%
\begin{footnotetext}[85]\sphinxAtStartFootnote
\sphinxnolinkurl{https://zhuanlan.zhihu.com/p/33524676}
%
\end{footnotetext}\ignorespaces 
无论你是三个阵营中的哪个,你的技术知识,应该帮助你回答下面几个问题:
\begin{enumerate}
\sphinxsetlistlabels{\arabic}{enumi}{enumii}{}{.}%
\item {} 
人工智能技术可能会给你的产品带来多大价值?因为产品永远是需求驱动,而非技术驱动。别忘了,再前沿的技术,从理论到产品落地是有巨大投入的。

\item {} 
从技术角度,将人工智能技术应用到你的产品中需要哪些资源或准备?例如需要更多的数据,更完善的算法模型?尽管很难量化这样的需求,你还是要尽可能的掌握更多信息去做判断。

\item {} 
从技术角度识别人工智能领域中的哪些理论已经有了最佳实践,即需要判断技术的成熟度。

\end{enumerate}

当你在将AI技术应用到产品中时,你应该能够给出答案:
\begin{enumerate}
\sphinxsetlistlabels{\arabic}{enumi}{enumii}{}{.}%
\item {} 
识别人工智能带来的价值是否真的被客户认可?这样的技术真的比传统技术更好吗?你需要多长时间或多少样例数据来验证你的人工智能产品已经站住脚了?

\item {} 
一旦产品上线后的效果没有预期好,你是否有备用计划?

\item {} 
任何一个机器学习功能的上线都需要占用研发80\%或更多的时间来完成对数据的准备(机器学习对数据的准备更占用时间),你是否已经和研发部门充分沟通并达成一致?

\end{enumerate}


\subparagraph{技术瓶颈 9\sphinxfootnotemark[86]–可解释性}
\label{\detokenize{chapter_introduction/AI_PM:id11}}%
\begin{footnotetext}[86]\sphinxAtStartFootnote
\sphinxnolinkurl{http://www.woshipm.com/pmd/798007.html}
%
\end{footnotetext}\ignorespaces \begin{enumerate}
\sphinxsetlistlabels{\arabic}{enumi}{enumii}{}{.}%
\item {} 
深度学习对于技术人员的经验依赖性依然很强,调参、收集数据、架构设计等没有通识的普遍规律,黑盒下的操作还是占很大比例。

\item {} 
对于每个技术背后的原理,知识体系往往存在着断层,很多过程我们是无法用语言或图像描述出来的。

\item {} 
算法可视化很苦恼,可能连设计者都无法用任何方式将内在的原理可视化给用户看。

\end{enumerate}


\subparagraph{Google总结了可解释性原则如下10\sphinxfootnotemark[87]}
\label{\detokenize{chapter_introduction/AI_PM:google10}}%
\begin{footnotetext}[87]\sphinxAtStartFootnote
\sphinxnolinkurl{https://easyai.tech/author/xiaoqiang/page/4/}
%
\end{footnotetext}\ignorespaces 
–
了解隐藏层的作用:深层学习模型中的大部分知识都是在隐藏层中形成的。在宏观层面理解不同隐藏层的功能对于解释深度学习模型至关重要。

–
了解节点的激活方式:可解释性的关键不在于理解网络中各个神经元的功能,而是在同一空间位置一起激发的互连神经元群。通过互连神经元组对网络进行分段将提供更简单的抽象级别来理解其功能。

–
理解概念是如何形成的:了解神经网络形成的深度,然后可以组合成最终输出的个体概念是可解释性的另一个关键构建块。


\subparagraph{AI产品经理与编码技术人员的关系(区别于算法技术人员)4\sphinxfootnotemark[88]}
\label{\detokenize{chapter_introduction/AI_PM:ai-4}}%
\begin{footnotetext}[88]\sphinxAtStartFootnote
\sphinxnolinkurl{http://www.woshipm.com/pmd/1629952.html}
%
\end{footnotetext}\ignorespaces 
张小龙、雷军、雷军认为其在金山的优点是勤劳,缺点是没有顺势而为,说白了什么叫顺势而为。笔者理解顺势而为就是产品思维,以用户为中心的思维再来看张小龙。

张小龙做微信的时候他指出:一个亿级用户的产品经理,无需做到透彻思考人性和产品的所有方面,但需要在极端现实主义和极端理想主义之间取得平衡。做产品力求简单美,要满足用户“贪嗔痴”。

关心的是用户!!!


\subparagraph{AI产品经理与算法技术人员的关系4\sphinxfootnotemark[89]}
\label{\detokenize{chapter_introduction/AI_PM:ai4}}%
\begin{footnotetext}[89]\sphinxAtStartFootnote
\sphinxnolinkurl{http://www.woshipm.com/pmd/1629952.html}
%
\end{footnotetext}\ignorespaces 
像灵犬的产品经理还要写灵犬反低俗助手产品的产品介绍、产品Q\&A。产品用户调研、产品推广,产品策略制定例如通过灵犬小程序产品可以收集数据来优化今日头条的本体反低俗模型产品。

什么需求?为了解决今日头条本身平台上鉴定低俗内容的

行为:检测其阅读内容的健康指数,输出对应的分数、评级和结论。不同于色情信息,处理低俗信息的一个难点在于,人们对于低俗的判断标准具有一定的主观性,合理筛选难度大。团队根据测试员的意见反馈
\begin{enumerate}
\sphinxsetlistlabels{\arabic}{enumi}{enumii}{}{.}%
\item {} 
灵犬的色彩以及对搜索框居中设计的布局可见灵犬是个单独的产品模块即可称之为独立的产品。

\item {} 
命名实体(NER)技术用来识别里面人名、地名、事物的名称等关键名词

\item {} 
灵犬反低俗助手产品的产品介绍、产品Q\&A、产品用户调研、产品推广、产品策略制定。

\end{enumerate}

AI产品经理应该能够使用设计专家使用的快速创新工具,包括用户体验模型、线框和用户调查。在这个阶段,确定产品要解决的问题或机会也是至关重要的。在他的文章“产品经理的机器学习”中,Neal
Lathia将ML问题类型分为六个类别:排名、推荐、分类、回归、聚类和异常检测。AI
PM只有在尽可能准确地确定他们想要解决的问题,并将问题归入其中一个类别之后,才能进入功能开发和实验阶段。\sphinxhref{https://www.oreilly.com/radar/practical-skills-for-the-ai-product-manager/}{22}%
\begin{footnote}[90]\sphinxAtStartFootnote
\sphinxnolinkurl{https://www.oreilly.com/radar/practical-skills-for-the-ai-product-manager/}
%
\end{footnote}

AI产品经理在需求评审【由项目经理(有单独项目经理的公司)组织产品经理、研发人员、测试人员、UI
设计人员听产品经理讲解需求的过程\sphinxhref{https://weread.qq.com/web/reader/8d232b60721a488e8d21e54kc51323901dc51ce410c121b}{12}%
\begin{footnote}[91]\sphinxAtStartFootnote
\sphinxnolinkurl{https://weread.qq.com/web/reader/8d232b60721a488e8d21e54kc51323901dc51ce410c121b}
%
\end{footnote}】的阶段,需要与算法共同明确的主要有以下几点:\sphinxhref{https://m.k.sohu.com/d/495625828?channelId=1\&page=1}{11}%
\begin{footnote}[92]\sphinxAtStartFootnote
\sphinxnolinkurl{https://m.k.sohu.com/d/495625828?channelId=1\&page=1}
%
\end{footnote}

模型目标 特征选择 数据收集 验收标准
至于数据预处理、模型选取、特征工程、调参等等部分,如果你有精力和能力去理解那自然是好的,但如果不能,只需要理解算法运作的基本原理即可。


\subparagraph{编码技术人员与算法技术人员的关系}
\label{\detokenize{chapter_introduction/AI_PM:id12}}
在软件开发中,无论是多么匪夷所思的
BUG,大都能查出具体的原因并给出修复方案,这个问题是确定的。

但在视觉模型这边,无论是多么合情合理的 bad
case,大都只能给出合理的推测:缺特定场景的数据?超参数不合适?没收敛好?那补数据、调参、重新训练之后一定能解决这个问题吗?会不会按下葫芦浮起瓢?不知道。这个问题是不确定的。


\subparagraph{VS 数据产品经理 19\sphinxfootnotemark[93]}
\label{\detokenize{chapter_introduction/AI_PM:vs-19}}%
\begin{footnotetext}[93]\sphinxAtStartFootnote
\sphinxnolinkurl{https://www.sohu.com/a/397318209\_114819}
%
\end{footnotetext}\ignorespaces 

\subparagraph{产品目标不同}
\label{\detokenize{chapter_introduction/AI_PM:id13}}
数据产品经理的产品目标是用数据确认确定性的需求;AI产品经理的产品目标是创造性的解决不确定性的产品需求。

当增长遇瓶颈;当产品不能精准的推荐给用户;当生产效率变低;当产品经理不能预测新的产品需求和新的服务需求;当人力成本变高,当有些固定流程的工作可以被机器人代替;

前类主要是数据产品经理要解决的问题,通过数据来验证产品提出的产品需求的正确性,通过上线后的数据来发现产品需要迭代改进甚至创新的点,通过数据分析,数据挖掘发现原本发现不了的产品问题,改进问题。

后类主要是AI产品经理的产品目标,AI一方面能帮人节省时间,另外能预测原本发现不了的产品和服务需求,还有AI能够解决不确定性的产品服务需求。


\subparagraph{产品过程步骤不同}
\label{\detokenize{chapter_introduction/AI_PM:id14}}
数据产品经理的数据分析的步骤一般可以分为如下6个步骤:
\begin{enumerate}
\sphinxsetlistlabels{\arabic}{enumi}{enumii}{}{.}%
\item {} 
明确分析的目的

\item {} 
数据准备

\item {} 
数据清洗

\item {} 
数据分析

\item {} 
数据可视化

\item {} 
分析报告

\end{enumerate}

AI产品经理案例:训练神经网络经典案例拆解:
\begin{enumerate}
\sphinxsetlistlabels{\arabic}{enumi}{enumii}{}{.}%
\item {} 
选定一个基础模型

\item {} 
设定初始化参数代入模型

\item {} 
用训练集对模型进行训练

\item {} 
通过一些数量指标,评估训练误差

\item {} 
如果训练误差不满足要求,继续调整参数

\item {} 
重复7–8次

\item {} 
采集新的数据,生成新的数据集。

\end{enumerate}


\subparagraph{懂不懂AI技术?懂的程度?}
\label{\detokenize{chapter_introduction/AI_PM:id15}}
产品经理是发现需求并在不确定的需求里面确定需求,而Kaggle一类的思维模式是辅助识别需求。

从算法工程师的角度切入做产品经理的人比比皆是,但是非算法工程师出身的产品负责人人数更多。

柔宇的刘博士懂柔性传感器技术,但是需要帮这种技术产品化,故此需要用柔性显示+做出柔记(柔记:一种写在真纸上,记载在AI芯片里的智能手写本),抢在三星之前发布柔性可折叠手机做出柔派(柔性技术+AI+新交互的手机和PAD同体款),早日做出柔派是要抢占用户心智。


\subparagraph{思维VS技术}
\label{\detokenize{chapter_introduction/AI_PM:vs}}
如果方向和方法错了你越执着于执行和操作,你错的越深你的产品越没有用户和客户。

我们常说:需求是洞,产品是钉子,技术选型是锤子,即AI产品经理本质核心工作是持续从用户需求出发,满足用户需求。洞察、分析、不断的满足用户需求。

真正由 AI
驱动的产品并不多,理性认识你负责的那个模型对项目到底有多重要,可以更合理的调配工作时间与精力,也能在和外部对接时省去很多不必要的口舌。


\subparagraph{分饰角色}
\label{\detokenize{chapter_introduction/AI_PM:id16}}
如何平衡炼丹、工程和业务?这得看你在实际工作中分饰几个角色。

如果三个角色都是你自己来,那么这就只是一个时间和精力分配的问题。

如果只是模型的训练和部署由你负责,那么这就是时间和精力分配+与产品有效沟通两个问题。

以上两种情况都有大量的先进经验可以借鉴,建议直接站内搜索。

如果算法工程师只负责训练,问题会相对复杂一些。


\subparagraph{缺乏角色 22\sphinxfootnotemark[94]}
\label{\detokenize{chapter_introduction/AI_PM:id17}}%
\begin{footnotetext}[94]\sphinxAtStartFootnote
\sphinxnolinkurl{https://www.oreilly.com/radar/practical-skills-for-the-ai-product-manager/}
%
\end{footnotetext}\ignorespaces 
缺乏特定的角色定义并不会阻止成功,但它确实引入了随着业务规模的扩大而积累技术债务的风险。重要的是,一个组织的整体数据战略包括路标(可能是产品管道中的阶段),标志着升级AI资源、技术和领导力的适当时间和条件。这一责任落在行政领导身上。如果没有高管的支持,强大的人工智能产品管理和工程领导力就无法蓬勃发展。


\subparagraph{AI各层次}
\label{\detokenize{chapter_introduction/AI_PM:id18}}
美团外卖:实时智能调度

\begin{figure}[H]
\centering
\capstart

\noindent\sphinxincludegraphics{{AI_cengci}.png}
\caption{层次 \sphinxhref{https://www.bilibili.com/video/BV1MK411u7SH?p=8}{15}\sphinxfootnotemark[95]}\label{\detokenize{chapter_introduction/AI_PM:id37}}\end{figure}
%
\begin{footnotetext}[95]\sphinxAtStartFootnote
\sphinxnolinkurl{https://www.bilibili.com/video/BV1MK411u7SH?p=8}
%
\end{footnotetext}\ignorespaces 

\subparagraph{流程}
\label{\detokenize{chapter_introduction/AI_PM:id19}}
泳道流程图:

\begin{figure}[H]
\centering
\capstart

\noindent\sphinxincludegraphics{{all_process}.png}
\caption{全流程}\label{\detokenize{chapter_introduction/AI_PM:id38}}\end{figure}

下图提供了与这些角色(在纵轴上)的生命周期的每个阶段(在横轴上)相关联的任务(在蓝色中)和工件(在绿色中)的网格视图:

\begin{figure}[H]
\centering
\capstart

\noindent\sphinxincludegraphics{{ms_flow_chart}.png}
\caption{微软(TDSP)}\label{\detokenize{chapter_introduction/AI_PM:id39}}\end{figure}


\subparagraph{成长路径}
\label{\detokenize{chapter_introduction/AI_PM:id20}}
\begin{figure}[H]
\centering
\capstart

\noindent\sphinxincludegraphics{{AIPM_env}.png}
\caption{不同企业背景下AI产品经理的成长环境\sphinxhref{https://weread.qq.com/web/reader/40632860719ad5bb4060856k9f6326602389f61408e3715}{30}\sphinxfootnotemark[96]}\label{\detokenize{chapter_introduction/AI_PM:id40}}\end{figure}
%
\begin{footnotetext}[96]\sphinxAtStartFootnote
\sphinxnolinkurl{https://weread.qq.com/web/reader/40632860719ad5bb4060856k9f6326602389f61408e3715}
%
\end{footnotetext}\ignorespaces 

\subparagraph{学习路线}
\label{\detokenize{chapter_introduction/AI_PM:id21}}
AI是一个技能型的职业,其主要的机会在于细分领域和交叉领域,AI产品经理所面临的最大的难度其实就是在于怎么去基于场景去定义需求。
主要的学习路线我个人认为可以分为三部走:
第一步,找到自己的兴趣点和特长,最好和自己之前有技能重合的领域,分技术大类的话其实也就是人机交互/计算机视觉/自然语言处理/生物特征识别等几个大类,这些大类又相对应的分出很多小类。
第二步就是要选择好自己的方向是基于平台类,还是聊天类亦或者是基于场景类。
最后一步其实就是实施转型了,这是最为艰难的一步,当然也是最终要的一步,这里我们重点聊聊实施转型这一步。


\paragraph{误区}
\label{\detokenize{chapter_introduction/AI_PM:id22}}

\subparagraph{不是学术研究 5\sphinxfootnotemark[97]}
\label{\detokenize{chapter_introduction/AI_PM:id23}}%
\begin{footnotetext}[97]\sphinxAtStartFootnote
\sphinxnolinkurl{https://www.zhihu.com/question/425088404/answer/1613313769}
%
\end{footnotetext}\ignorespaces 
如何确定哪些任务可以用人工智能完成?如何划分数据集?模型性能不佳时如何判断问题所在?如何判断某个改进思路的可行性?深度学习项目通常需要消耗大量的资源,与其投入一两个月的精力实现某思路结果发现性能并不尽如人意,“谋定而后动”是十分必要的。目前,直接关注深度学习在实际项目实践中经验心得的中文资料还十分匮乏,本文力图对深度学习项目实践从项目选型、数据准备、训练、模型分析、模型部署全阶段的注意事项和技巧进行梳理。

学术研究的目标。学术研究的目的是为人类认识世界提供新的知识,因此很看重独创性(novelty)和
方法可复现性。创新是学术研究的核心,同样的研究问题,无论以何种形式,已经有人发表,那么该工作的创新性将大打折扣。

项目实践的目标。而项目实践的目标通常是得到一个高性能的模型,对这个模型是否独创或可复现并不看重。这并不是说项目实践比学术研究更简单,不客气地讲,有的学术研究就是在想方设法过拟合几个基准(benchmark)数据集的测试集以在论文中展示出漂亮的数字。而项目实践对模型的泛化性能十分看重,如果过拟合测试集,只会让你的老板短暂的高兴一下之后将发现线上指标十分糟糕。此外,要想做好项目实践,除了需要读论文、复现结果、产生自己idea的能力外,还需要下载处理数据、调试代码等脏活累活,这两部分同等重要。


\subparagraph{沉溺论文 6\sphinxfootnotemark[98]}
\label{\detokenize{chapter_introduction/AI_PM:id24}}%
\begin{footnotetext}[98]\sphinxAtStartFootnote
\sphinxnolinkurl{https://zhuanlan.zhihu.com/p/73661738}
%
\end{footnotetext}\ignorespaces 
不要沉溺在论文的海洋。现在人工智能正值热潮,每年新发表的论文非常多。而机器学习是实践科学,尤其是当你不是该领域专家时,事先很难知道哪种方案在实际中效果最好,通常需要尝试很多的思路。机器学习实战的过程是思路、代码实现、实验结果的迭代循环。迭代循环的越快,取得的进展越大。不要在开始前想的过多,尤其不要一开始就想着设计和构建一个完美的系统,最初的方案要越早构建和训练越好,之后根据偏差/方差分析和误差分析确定下一步的工作方向,并进行迭代。


\subparagraph{学生思维}
\label{\detokenize{chapter_introduction/AI_PM:id25}}
读书时,我们主要是面向论文炼丹的:需求就是SOTA,数据主要靠公开数据集,我们要做的主要就是训练、迭代、刷分——这是典型的Hard
\& Clean Problem。

工业界不是这样的。公司是花钱请人来在用\sphinxstylestrong{有限的资源}解决实际问题的。

实际情况往往是:需求常生变、采集有周期、标注有成本、清洗要策略、训练要时间、测试卡标准。和算法工程师最相关的训练迭代,往往既不是最关键的,也不是最复杂的,甚至都未必是最贵的环节。

需求分析的重要性是显而易见的,需求都歪了后面再努力也是白给。

要对全流程有所理解,然后根据实际情况坚持以理服人,对产品划清能力边界,严防脑洞大开;对工程尽量提供更有确定性的消息,以及,文档例程不要过于放飞自我。


\subparagraph{唯SOTA论}
\label{\detokenize{chapter_introduction/AI_PM:sota}}
因为选择SOTA != 选择未经验证的解决方案 != 选择了高风险。

除非你在做 POC
或者预研性质的项目,否则一个项目的目标、成本、排期、风险点一定是在实施之前就基本确认的。用尽可能低的预算解决问题的方法有很多,但未经广泛验证的
SOTA 绝非上策。

在实践中,如果不能加数据,那么通常是 simple, solid work
更好用一点:花样百出 Attention
模块没有几个在多数任务中都能即插即用即涨点的;ArcFace
的核心就那么几行,用过都说好。


\subparagraph{形势简单化}
\label{\detokenize{chapter_introduction/AI_PM:id26}}
除非你即将或已经入职的公司基础设施完备、支持团队给力,不然在实际工作中难免碰到各种一地鸡毛的事情,比如:
\begin{itemize}
\item {} 
数据清洗、预处理没有合适的工具

\item {} 
实验管理没有合适的工具(MLFlow、 Weights \& Biases)

\item {} 
结果可视化没有合适的工具

\end{itemize}

Serving Infrastructure: This includes tools for model development (such
as the Cloudera Data Science Workbench, Domino Data Lab, Data Robot, and
Dataiku) and production serving infrastructure (such as Seldon,
Sagemaker, and TFX).
\sphinxhref{https://www.oreilly.com/radar/practical-skills-for-the-ai-product-manager/}{24}%
\begin{footnote}[99]\sphinxAtStartFootnote
\sphinxnolinkurl{https://www.oreilly.com/radar/practical-skills-for-the-ai-product-manager/}
%
\end{footnote}


\paragraph{难题 29\sphinxfootnotemark[100]}
\label{\detokenize{chapter_introduction/AI_PM:id27}}%
\begin{footnotetext}[100]\sphinxAtStartFootnote
\sphinxnolinkurl{https://www.bilibili.com/video/BV1xK4y1j7ep}
%
\end{footnotetext}\ignorespaces \begin{itemize}
\item {} 
通用产品vs定制方案

\item {} 
完全没有数据vs没有标注数据

\item {} 
炫酷的技术vs可怜的性能指标

\item {} 
被宣传鼓舞的期待vs低可用性的产品

\item {} 
产品整体指标vs个别 bad case

\item {} 
沟通难题:与技术人员、与客户

\end{itemize}


\paragraph{Awesome}
\label{\detokenize{chapter_introduction/AI_PM:awesome}}
\sphinxurl{https://www.yuque.com/gdbwhd/mesmph/uuuuu\#LsKnF}


\paragraph{To B/C 13\sphinxfootnotemark[101]}
\label{\detokenize{chapter_introduction/AI_PM:to-b-c-13}}%
\begin{footnotetext}[101]\sphinxAtStartFootnote
\sphinxnolinkurl{https://www.bilibili.com/video/BV1MK411u7SH?p=14}
%
\end{footnotetext}\ignorespaces 
以科大讯飞为例 ①原有语音识别技术,要做商业化
②需要结合行业找应用场景,做解决方案


\subparagraph{TO B(汽车行业)}
\label{\detokenize{chapter_introduction/AI_PM:to-b}}
核心:调硏客户需求、出解决方案

①熟悉行业,要求对行业必须深度理解
②找客户,调研客户需求、出解决方案、投标/比标


\subparagraph{TO C(小蛋机器人)}
\label{\detokenize{chapter_introduction/AI_PM:to-c}}
核心:为其他行业树立标杆

①超强创新能力 ②注重性能的稳定性


\subparagraph{度量标准}
\label{\detokenize{chapter_introduction/AI_PM:id28}}
对于没有成熟数据或机器学习实践的企业来说,定义和同意度量标准通常是困难的。

最坏的情况是企业没有任何指标。如果业务缺乏度量标准,它可能也缺乏数据基础设施、收集、治理等方面的规程。


\paragraph{道德}
\label{\detokenize{chapter_introduction/AI_PM:id29}}
在整个产品开发过程中,产品经理需要思考道德问题,并鼓励产品团队思考道德问题,但在定义问题时,这一点尤为重要。这是一个需要解决的问题吗?这个解决方案怎么可能被滥用?这些都是每个产品团队都需要思考的问题。

\sphinxurl{https://learning.oreilly.com/library/view/ethics-and-data/9781492043898/}


\paragraph{高风险}
\label{\detokenize{chapter_introduction/AI_PM:id30}}
基于深度学习的产品很难(甚至不可能)开发;这是一个典型的“高回报与高风险”的情况,在这种情况下,计算投资回报本来就很困难。


\paragraph{优秀的AI PM 32\sphinxfootnotemark[102]}
\label{\detokenize{chapter_introduction/AI_PM:ai-pm-32}}%
\begin{footnotetext}[102]\sphinxAtStartFootnote
\sphinxnolinkurl{https://weread.qq.com/web/reader/0c032c9071dbddbc0c06459kb6d32b90216b6d767d2f0dc}
%
\end{footnotetext}\ignorespaces 
人工智能产品经理的工作是分析出有价值的商业场景,并对其进行评估,评估其可行性、必要性、社会及商业价值、道德及法律框架。除分析有价值的商业场景外,人工智能产品经理还需要评估技术的可行性,评估技术能够达到的最优度,并根据内外部资源评估出产品价值与技术实现的平衡点,以及商业价值与技术成本之间的平衡点。最后,人工智能产品经理应能够通过交互体系设计出完整的产品。如果一个人能够做到以上几点,那么他会是一个非常优秀的人工智能产品经理。


\paragraph{最好的人工智能产品 22\sphinxfootnotemark[103]}
\label{\detokenize{chapter_introduction/AI_PM:id31}}%
\begin{footnotetext}[103]\sphinxAtStartFootnote
\sphinxnolinkurl{https://www.oreilly.com/radar/practical-skills-for-the-ai-product-manager/}
%
\end{footnotetext}\ignorespaces 
在最好的人工智能产品中,用户无法分辨底层模型如何影响他们的体验。他们既不知道也不关心应用程序中是否存在人工智能。以Stitch
Fix为例,它使用了多种算法方法来提供定制风格的建议。当Stitch
Fix用户与人工智能产品交互时,他们会与预测和推荐引擎交互。他们在体验过程中与之互动的信息是一种人工智能产品——但他们既不知道,也不关心,他们所看到的一切背后是人工智能。如果算法做出了完美的预测,但用户无法想象佩戴所展示的物品,该产品仍然是一个失败的产品。在现实中,ML模型远非完美,因此更有必要确定用户体验。


\paragraph{失败的人工智能产品 23\sphinxfootnotemark[104]}
\label{\detokenize{chapter_introduction/AI_PM:id32}}%
\begin{footnotetext}[104]\sphinxAtStartFootnote
\sphinxnolinkurl{https://neal-lathia.medium.com/machine-learning-for-product-managers-ba9cf8724e57}
%
\end{footnotetext}\ignorespaces \begin{enumerate}
\sphinxsetlistlabels{\arabic}{enumi}{enumii}{}{.}%
\item {} 
由于ML本质上是通过例子来训练算法,产品失败的方式呈现出各种各样的新维度(这里的图片是微软的Tay机器人,它在被放到网上后变成了种族主义者)。

\item {} 
一个计算一个人怀孕概率的算法,他们用它来发送优惠券和折扣————这个系统向一个十几岁的女孩发送了婴儿服装的优惠券,这让她的父亲非常愤怒:为什么他的女儿会被婴儿服装的优惠券盯上?但不久之后,塔吉特收到了他的道歉:女孩确实怀孕了。

\item {} 
产品的使用或应用方式与产品设计师的设想不同。无论是基于敏感推断的目标客户,还是人们刺激机器人,还是用于训练人脸检测算法的有偏见的数据集

\end{enumerate}


\subsubsection{AI产品经理的完整能力矩阵1\sphinxfootnotemark[105]}
\label{\detokenize{chapter_introduction/ability:ai1}}\label{\detokenize{chapter_introduction/ability::doc}}%
\begin{footnotetext}[105]\sphinxAtStartFootnote
\sphinxnolinkurl{https://www.jianshu.com/p/fd466ed1bda6}
%
\end{footnotetext}\ignorespaces 

\paragraph{对比产品经理能力模型}
\label{\detokenize{chapter_introduction/ability:id1}}
\begin{figure}[H]
\centering
\capstart

\noindent\sphinxincludegraphics{{AIPM_vs_PM}.png}
\caption{AIPM\_vs\_PM}\label{\detokenize{chapter_introduction/ability:id15}}\end{figure}


\paragraph{总结}
\label{\detokenize{chapter_introduction/ability:id2}}
势+道+法+术+器+核心价值观来总结,“势”即君主的统治权力,“道”就是通用软技能,“法”就是团队管理方法,“术”就是具体的实现方法,“器”就是高效的实现工具,而核心价值观是作为AI产品经理的自我修养,是内功!


\paragraph{势}
\label{\detokenize{chapter_introduction/ability:id3}}
张小龙–“产品经理是站在上帝身边的人”。同样优秀的产品,发布实时机早了可能成先烈,发布晚了可能冲入红海没法突出重围。
\begin{enumerate}
\sphinxsetlistlabels{\arabic}{enumi}{enumii}{}{.}%
\item {} 
社会变革:改朝换代,颠覆式。

\item {} 
人口结构变革:每一代人都有其个性化需求,带来大量的增量市场。

\item {} 
技术变革:触屏技术的出现,直接将手机的从联系工具变成了信息获取和处理平台

\item {} 
资本市场环境改变:共享单车在红包大战时很难进入投资人的视野,产品需要资本为其成长加速,需要持续关注资本市场。

\item {} 
产品规模变革:豆瓣和知乎,小而美,大量用户涌入,原有的打分和评价系统不适新的场景

\end{enumerate}


\paragraph{道}
\label{\detokenize{chapter_introduction/ability:id4}}\begin{itemize}
\item {} 
用户/客户敏感

\item {} 
自我驱动(主动收集需求,并把它转化成产品需求{[}8{]})/学习能力(产品经理需要懂营销,懂技术、懂运营、懂设计{[}8{]})

\item {} 
沟通表达能力(一坨屎去撕逼:回响、star、输出感悟、用心{[}5{]}、Case包装{[}7{]})

\item {} 
市场洞察能力(视野/前沿)(研究市场以了解用户需求、竞争状态、市场规模和盈利模式,发现创新或者改进产品的潜在机会;通过与用户和潜在用户进行沟通交流,明确符合该机会中的目标用户群体与特征;与直接面对用户/客户的一线同事/同行交流,获取、分析、评估用户的需求。)

\item {} 
高效执行能力
(决策力背后体现的是对过去信息的有效收集,对当下形势的清晰判断和对未来趋势的准确把握。好的产品经理总能在关键时刻,踩对点,比如我最敬佩的产品大师张一鸣,在今日头条成立初期,就定下了头条内容+算法的战略方针。{[}4{]})

\item {} 
判断趋势能力(趋势预判表现为细致的观察、信息的敏感度、成熟的推演等)

\item {} 
创新能力

\item {} 
情绪控制能力

\item {} 
时间管理能力

\item {} 
多观点整合能力

\item {} 
资源协同能力(多资源协助)

\item {} 
影响他人能力(销售方案和销售激励{[}7{]})

\item {} 
团队建设能力(跨部门协作的能力,产品价值在公司内的传递,并协作宣传方案。{[}7{]})

\item {} 
外语能力

\item {} 
产品定义和原型设计能力(产品需求文档(PRD)来进行描述,形成包含交互设计(UX)、用户界面设计(UI)的具体方案:在方案形成过程中,通常会通过原型设计是把自己的想法或者团队讨论确定的方案以具象的形式呈现出来,便于团队成员的理解或者用于用户测试,最终达成共识,形成确定的方案。)

\item {} 
系统能力(将所有资源有效打包成产品、服务提供给消费者。{[}10{]})

\end{itemize}


\paragraph{法{[}6{]}}
\label{\detokenize{chapter_introduction/ability:id5}}
优秀的产品需要优秀的产品经理,我们来看看这些大公司都是如何创造产品文化的:

腾讯公司对于产品经理的要求是10/100/1000,即产品经理每个月必须做10个用户调查,关注100个用户博客,收集反馈1000个用户体验。

阿里巴巴会把新人产品经理扔到客服去处理三个月投诉,这样产品经理就非常清楚应该做什么了。

笔者认为产品经理都应该是终生学习者,对未知充满好奇,所以分享文化是必须,这既加强了产品的沟通和汇报能力,也能让团队形成探索氛围,促进大家大开脑洞。


\paragraph{术}
\label{\detokenize{chapter_introduction/ability:id6}}
\begin{figure}[H]
\centering
\capstart

\noindent\sphinxincludegraphics{{AI_knowledge_map}.jpg}
\caption{术:产品经理知识地图}\label{\detokenize{chapter_introduction/ability:id16}}\end{figure}

文本撰写、工具使用、沟通交流等能力像是产品经理的右手,创新思维、发散思维等能力像是产品经理的左手,左手的能力将决定一个产品经理能走多高。右手一般人经过一段时间的训练都能训练出来,而左手却很难,只有接触过足够多的人见过足够事,有着丰富的阅历也许才能培养出来。{[}9{]}


\subparagraph{产品战略与产品创新阶段}
\label{\detokenize{chapter_introduction/ability:id7}}\begin{itemize}
\item {} 
市场分析:PEST分析、APPEALS方法、战略定位分析(SPAN)、麦肯锡市场细分八法;

\item {} 
竞争力分析:波士顿矩阵(BCG矩阵)、GE分析、麦肯锡三层面理论等;

\item {} 
机会判断;竞品分析画布、MRD撰写;

\item {} 
用户研究:A/B test、问卷调研、可用性测试、干系人地图、用户洋葱模型等

\end{itemize}


\subparagraph{产品规划与商业模式阶段}
\label{\detokenize{chapter_introduction/ability:id8}}
需求分析:马斯诺需求层次理论、3W2H方法、5WHY分析法、PSPS模型等

商业分析:SWOT分析、波特五力分析、精益商业画布、BRD文档;

优先级评估:火车模型、Kano模型、COD评分表方法、四象限方法、MoSCoW方法等;

数据分析:TODO:

产品规划:产品架构图、产品路线图、计划扑克工作量评估法、六西格玛、TRIZ、盈利模式设计、MVP定义、突出重点(避免认知失调);


\subparagraph{产品运营与营销阶段}
\label{\detokenize{chapter_introduction/ability:id9}}
产品运营:AARRR产品运营模型、OGSM工具、运营数据分析、灰度测试、同期群分析、网络推广优化、市场维护等;

持续了解和收集基本数据,追踪产品投放到市场上的效果和反馈,以便不断迭代优化。\sphinxhref{https://www.zhihu.com/question/31636227}{2}%
\begin{footnote}[106]\sphinxAtStartFootnote
\sphinxnolinkurl{https://www.zhihu.com/question/31636227}
%
\end{footnote}
工具:Google Analytics、百度统计、TalkingData、友盟、GrowingIO 等等。

产品营销:FABE法则、电梯演讲、产品路演等;


\subparagraph{产品生命周期管理}
\label{\detokenize{chapter_introduction/ability:id10}}\begin{itemize}
\item {} 
产品方法框架:IPD、门径管理流程、抄超钞等;

\item {} 
产品宏观思维:波士顿矩阵、多产品组合战略等;

\item {} 
团队建设:团队文化定义、组织架构建设等;

\end{itemize}


\subparagraph{AI产品方法}
\label{\detokenize{chapter_introduction/ability:ai}}\begin{itemize}
\item {} 
算法

\item {} 
算力

\item {} 
数据

\item {} 
硬件

\item {} 
业务

\end{itemize}


\paragraph{器}
\label{\detokenize{chapter_introduction/ability:id11}}\begin{itemize}
\item {} 
通用办公工具:office三件套、Xmind类思维导图、think\sphinxhyphen{}cell麦客–信息收集等;

\item {} 
产品流程设计:Visio、Processon、亿图等;

\item {} 
产品原型设计:Axure、Sketch、墨刀等;

\item {} 
数据分析工具:SQL、python、powerBI、SPSS、百度指数、talkingdata、ASO100、艾瑞指数、微博数据中心等;

\item {} 
项目管理工具:Teambition、Trello–任务管理、Demoo\sphinxhyphen{}原型展示、石墨文档、禅道–项目管理、leangoo等;

\item {} 
AI工具:Python、Tensorflow、PyTorch、MxNet等

\item {} 
主要文档:MRD、BRD、PRD;

\end{itemize}


\paragraph{核心价值观}
\label{\detokenize{chapter_introduction/ability:id12}}
这里我要援引经典的产品设计五要素图来解释AI产品经理的核心价值观。

\begin{figure}[H]
\centering
\capstart

\noindent\sphinxincludegraphics{{产品设计五要素}.png}
\caption{产品设计五要素}\label{\detokenize{chapter_introduction/ability:id17}}\end{figure}


\subparagraph{初心}
\label{\detokenize{chapter_introduction/ability:id13}}
作为AI产品经理要时刻记住自己做产品的初心,也就是最底层的战略层,一方面是这个产品的初衷是什么,想清楚了它才能走的长远,如果只是未来表层和框架的浅显需求而做设计,那这个产品设计是站不住脚的,只有从战略层进行思考,产品整体设计才经得起推敲,那时即使在部分表层有缺陷,也瑕不掩瑜,这就好像哲学终的“本我”。


\subparagraph{自我定位}
\label{\detokenize{chapter_introduction/ability:id14}}
AI产品经理的自我定位也非常重要,在我的工作经历中,看过很多产品经理,因为主观或客观的产品立场不坚定,有时候把自己做成了商务、解决方案,有时候在一些技术架构方面与研发团队钻牛角尖,但往往丢失了一个产品经理的初心,最终产品走向也不是很理想。作为产品经理,我们需要把握的是整个产品的生命线,而很多细枝末节的事情,有细分领域更专业的人去做。

{[}4{]}: \sphinxhref{https://www.zhihu.com/question/31636227/answer/1162506705}{产品经理的职业发展路径是怎样的? \sphinxhyphen{} 呱说产品的回答 \sphinxhyphen{}
知乎}%
\begin{footnote}[107]\sphinxAtStartFootnote
\sphinxnolinkurl{https://www.zhihu.com/question/31636227/answer/1162506705}
%
\end{footnote} {[}5{]}:
\sphinxurl{http://www.woshipm.com/pmd/4256992.html} {[}6{]}:
\sphinxurl{http://www.woshipm.com/pmd/693904.html} {[}7{]}:
\sphinxurl{http://www.woshipm.com/pmd/3945349.html} {[}8{]}:
\sphinxurl{http://www.woshipm.com/zhichang/459131.html} {[}9{]}:
\sphinxurl{http://www.woshipm.com/zhichang/315041.html} {[}10{]}:
\sphinxurl{http://www.woshipm.com/pmd/3130419.html}


\subsubsection{产品机会}
\label{\detokenize{chapter_introduction/opportunity:id1}}\label{\detokenize{chapter_introduction/opportunity::doc}}

\paragraph{痛点}
\label{\detokenize{chapter_introduction/opportunity:id2}}
痛点是恐惧


\subparagraph{分类 1\sphinxfootnotemark[108]}
\label{\detokenize{chapter_introduction/opportunity:id3}}%
\begin{footnotetext}[108]\sphinxAtStartFootnote
\sphinxnolinkurl{https://www.zhihu.com/question/21155472/answer/1580037628}
%
\end{footnotetext}\ignorespaces 
所谓“痛点”可分为三类:

第一类是人类普遍有所体会的某种心理上的难受,或者某些蠢蠢欲动的欲望没有得到满足的难受,这种难受常常经过外界刺激而有所强化。例如,思乡、恐高、怕抽血、窥探隐私、八卦欲等等。
第二类是体验过某种产品后,如果不买会难受,会有不满足感,可谓欲罢不能。比如喝惯可乐、玩过网络游戏为什么难以摆脱。
第三类是在购买过程中小小地难受一下,如此使得顾客最终获得产品时,强烈地对比出愉悦感来。例如,苹果新品发布时总是长龙排队,甚至有顾客不得不凌晨开始在店铺外占位。

与第一类用于化解痛点的产品不同,第三类痛点是有意设置的,它有时是因为企业稀缺的资源所造就的,企业为了将有限的资源聚焦在最具竞争力的产品或服务上,因而剔除了某些附加服务。如果顾客体验到的痛感远远小于获得产品和服务时的满足感,那么顾客就不再计较这其中的痛点。成功击中这类痛点的产品,要么消解了痛点,要么弱化了痛点。


\subparagraph{如何找到痛点 2\sphinxfootnotemark[109]}
\label{\detokenize{chapter_introduction/opportunity:id4}}%
\begin{footnotetext}[109]\sphinxAtStartFootnote
\sphinxnolinkurl{https://wiki.mbalib.com/wiki/\%E7\%97\%9B\%E7\%82\%B9\%E8\%90\%A5\%E9\%94\%80}
%
\end{footnotetext}\ignorespaces 
  在客户营销学中,消费者的痛点是指消费者在体验产品或服务过程中原本的期望没有得到满足而造成的心理落差或不满,这种不满最终在消费者心智模式中形成负面情绪爆发,让消费者感觉到痛。当设计客户体验的时候,痛点是一个必须存在的“天使”,不仅可以用来对比体验中的愉悦,还可以节省资源,摆脱束缚。原因有如下三点:
\begin{enumerate}
\sphinxsetlistlabels{\arabic}{enumi}{enumii}{}{.}%
\item {} 
痛点是基于心理感受对比的体验营销的重要手段。在体验营销中,企业需要构建让消费者足够满意和愉悦的痒点和兴奋点。对于消费者而言,在鱼和熊掌不可得兼的前提下,只能抱以遗憾,选择对其更为重要的痒点或兴奋点因素,忽视这些痛点带来的不满。这是心理学中的“残缺美”现象,因为有所期望,所以才会不满,倘若没有了期望,不满从何而来?
另外,痛点也是一个相对的概念,是基于同行业的竞争而做出的对比后形成的判断,这是和其对立的词痒点比较而来的。在销售过程中,甲企业的产品除具有相同的功能特点外,还具有一些其他的附加服务,而这些恰恰是乙企业所不具备的,这对倾向于乙品牌的消费者来说,就形成了大脑认知中的痛点现象。

\item {} 
痛点是企业聚焦战略选择的必然结果。如果说兴奋点是战略选择的话,那么痛点是基于战略上的放弃。这个观点是从企业战略角度分析的,是战略的取舍问题,是典型的聚焦特点。
在销售过程中,为了实现客户价值的转移,很多企业通过丰富的体验营销手段来吸引消费者。但是随之带来的问题就是消费者的欲望是无穷的,而企业的资源又是稀缺的,全方位无条件满足消费者的欲望是不现实的。怎么样才能实现客户让渡价值的最大化,企业必然的选择就是聚焦战略,即把所提供的产品或服务形成最闪亮的亮点,其余的附加服务都取消掉。必然的结果是消费者会感受到不满意,但倘若最为满意所带来的价值远大于这种痛点所带来的不满意,那么客户最终会淡化或忽视这种痛点。

\item {} 
痛点是一种引领企业创新的过程。在竞争中,能够在消费者心智中占有一席之地,乃是所有营销人孜孜不倦的追求。而痛点就是营销策划活动中最常用和最有效的手段之一。在动态的竞争过程中,每一个企业为了参与竞争都试图通过差异化来占领自己的区域,所以“反定位”就成了有效手段,于是竞争对手的体验营销中的痛点就成了其他企业营销创新的痒点。
对于痛点寻找需要第一,对自己的产品和服务有充分的了解,还有就是对竞争对手的产品或服务有充分的了解。第二,是对消费者消费心理有充分的解读。对于自己产品或服务和竞争对手的产品或服务的了解是用来做差异化定位的,通过细分市场区找痛点。对消费者的了解是非常重要的,因为购物的主题就是他们这些人,你只有知道他们的真正需求,然后满足他们那么你的产品或服务就是成功的,否则失败。痛点是一个长期观察挖掘的过程不可能一触而就的,这些都是细节的问题,都是消费者最关注的细节,做好这些,结果就可想而知了。

\end{enumerate}


\paragraph{爽点 3\sphinxfootnotemark[110]}
\label{\detokenize{chapter_introduction/opportunity:id5}}%
\begin{footnotetext}[110]\sphinxAtStartFootnote
\sphinxnolinkurl{https://www.jianshu.com/p/fa5e2c1f3930}
%
\end{footnotetext}\ignorespaces 
有需求,还能被即时满足,这就是爽。

外卖产品就是满足了痒点。


\paragraph{痒点}
\label{\detokenize{chapter_introduction/opportunity:id6}}
痒点满足的是人的虚拟自我。什么是虚拟自我?就是想象中那个理想的自己。

网红产品靠的就是痒点。


\paragraph{需要掌握的技能/方法:}
\label{\detokenize{chapter_introduction/opportunity:id7}}
心情看板:比如说你去北京三甲医院挂号,因为三甲医院资源紧张,你在挂号排队的时候,心情会或许会非常的低落。心情看板主要为了识别出用户在完成整个流程中心情的最低点和最高点,便于根据用户情绪起伏中发现设计机会点,给问题提供针对性的假设。

问题假设:给用户亟待解决的问题提出假设,比如说怎么才能在最短时间内挂到专家门诊号。

设计挑战:主要是为了拓展解决问题的不同思路,可以在团队内部通过头脑风暴的形式做设计挑战。


\paragraph{战略示意图}
\label{\detokenize{chapter_introduction/opportunity:id8}}
Jeroen
Kraaijenbrink的战略示意图适合制定企业的总体战略,有利于在新的产品出现时,根据新的发展战略确认其是否可行。

\begin{figure}[H]
\centering
\capstart

\noindent\sphinxincludegraphics{{stratgy_pic}.png}
\caption{战略示意图}\label{\detokenize{chapter_introduction/opportunity:id9}}\end{figure}
\begin{enumerate}
\sphinxsetlistlabels{\arabic}{enumi}{enumii}{}{.}%
\item {} 
核心价值:核心价值是指企业通过某种方式提供的产品和服务,企业需思考产品对于消费者的价值是什么,产品能够解决的问题是什么。

\item {} 
核心资源:企业的产品拥有什么资源,使得产品在市场竞争环境中具有优势,表现为产品的的核心资源和核心活动,人工智能类产品一般是核心的算法、数据等。

\item {} 
盈利模式:根据企业的核心资源,产品能够如何收取费用?从谁身上收取、怎样收取、何时收取?

\item {} 
用户需求:产品服务于什么组织与群体,产品应该满足什么需要呢?

\item {} 
价值和目标:企业要明确产品要达到什么目标,产品解决的最重要的问题是什么。需要注意的是,不要将收益或者股东利益当作产品的主要目标。

\item {} 
合作伙伴:重要的合作伙伴,能够保证企业的产品和服务更有价值。

\item {} 
风险和成本:商业模式运作过程中承担什么样的财务、社会或其他风险及企业如何管理这些风险。

\item {} 
竞争对手:客户会用比较的方式来决定是否购买你的产品和服务,你有什么样的竞争优势。

\item {} 
团队氛围:团队文化和结构是什么样的,工作的组织结构和文化环境是制定决策并完美执行的重要因素。

\item {} 
机会与威胁:市场中有什么因素影响着组织,这些因素是机会还是威胁?对未来的商业模式有着怎样的影响?

\end{enumerate}


\subsection{Idea}
\label{\detokenize{chapter_idea/index:chap-idea}}\label{\detokenize{chapter_idea/index:idea}}\label{\detokenize{chapter_idea/index::doc}}

\subsubsection{思维}
\label{\detokenize{chapter_idea/idea:id1}}\label{\detokenize{chapter_idea/idea::doc}}

\paragraph{规划思维应用职业规划}
\label{\detokenize{chapter_idea/idea:id2}}
小白需要实践的是如何通过自己的能力从资源生产者变成资源管理者,最后变成资源支配者。

处于成长期阶段的人通过自己的技能输出得到回报,属于资源生产者;
处于领导者阶段的人除了技能输出,还是有整合资源的能力,属于资源管理者;
处于自我实现阶段产品经理通过参与创业的方式触碰行业天花板,属于资源支配者。

一个人属于哪个层面,与处在该层面的时间没有直接的联系,很多在一个行业做了N年的人,如果他们一直在成长期打转,在职业生涯中从来都没有支配过资源,那么就很难获得超额的价值回报。

其实有产品思维的牛人,不一定最开始是产品经理,比如王峰最开始的时候金山是负责营销工作,因为业绩出色,后来才成为了金山词霸和金山毒霸的产品市场负责人,最后做到金山副总裁。

在金山他跨越了成长期和领导者期,财务自由之后最后离职创业就很顺其自然,后面他成立了游戏公司上市,以及现在做的区块链媒体都属于自我实现阶段。


\paragraph{逻辑思维来撕逼和思考}
\label{\detokenize{chapter_idea/idea:id3}}
逻辑思维能力被简单地理解为“任何一件事情都可以归纳出一个中心论点,而这个中心论点可以由3到7个论据进行支撑,每一个论据本身又可以成为一个论点。”

在逻辑思维中,由论据到论点的推理过程就叫作推理,推理是最根本的逻辑关系。

论点的成立与否是逻辑思维的主题,而论点的成立与否又直接源自论据的真假,如果论据是假的,那么得到的论点也是假的。

提升交流质量很重要一步就是用逻辑思考对方的观点是否正确,否者只能淹没在信息中。

还有的朋友在吵架撕逼的时候,用逻辑思维找出对方的论据错误和论据到论点的推理错误,往往能在撕逼中脱颖而出。


\paragraph{敏捷思维对抗不确定性}
\label{\detokenize{chapter_idea/idea:id4}}
有一次一个研发同学和我聊起买房的事情,他手里有些积蓄,想在杭州买一套大户型的房子,但是还差很多钱,于是他想着把钱攒够了在去买房。

但是我们知道杭州房价过去几年一直处于上升通道,用敏捷思维思考这个事情,我给他的建议是用自己手里的钱可以先买一个小户型的房子,有机会再换成大户型。

先解决从无到有,再去解决从小到大,如果继续等下去付出的成本只会越来越高。在面对重大不确定的重大选择时,最稳妥的方式就是用敏捷思维先解决核心需求。

利用敏捷思维要求每次只做最重要的事情,满足用户和环境不断的变化,并且努力把每一件小事情做好,期待未来回报奇点的到来。

身边大多数牛人拥有迭代思维还意味着他们很乐观,因为纵使对当前的产品不满意,但是他们会想各种办法,哪怕每天进步一点点,去不断迭代自己,而不是把时间浪费在对未知的恐惧。


\paragraph{运营思维做销售}
\label{\detokenize{chapter_idea/idea:id5}}
一个小米同学和我讲,在小米初创期间,黎万强一手打造了小米的社区模式,但是初期的阿黎在互联网圈中并不知名,小米初期也没有很强的品牌影响力。

所以阿黎只能组织团队,注册了很多小号,在各个论坛发布了很多MIUI的帖子,刚好用户也有需求,才慢慢积累的种子用户。

初期做法是不是和做微商有点类似?


\paragraph{有限理性}
\label{\detokenize{chapter_idea/idea:id6}}

\subparagraph{原因}
\label{\detokenize{chapter_idea/idea:id7}}\begin{itemize}
\item {} 
信息获取能力有限

\item {} 
信息处理能力有限

\item {} 
禀赋偏好导致的个体差异

\item {} 
环境的不确定因素

\end{itemize}


\subparagraph{理性决策三要素}
\label{\detokenize{chapter_idea/idea:id8}}\begin{itemize}
\item {} 
理性的信念:与真实世界一致的信念,对自我认知的认知;人类最大的理性,就是理解自身认知能力的局限性

\item {} 
理性的目标:约束条件下的价值(总效用)最大化

\item {} 
理性的行动:给定目标下,寻找最优的解决方案 {[}1{]}:
\sphinxurl{http://www.woshipm.com/zhichang/4114367.html}

\end{itemize}


\subsubsection{数据分析1\sphinxfootnotemark[111]}
\label{\detokenize{chapter_idea/data:id1}}\label{\detokenize{chapter_idea/data::doc}}%
\begin{footnotetext}[111]\sphinxAtStartFootnote
\sphinxnolinkurl{http://www.woshipm.com/data-analysis/2696737.html}
%
\end{footnotetext}\ignorespaces 
比如像很多互联网公司都成立了大数据团队,收集用户的社交、电商、搜索行为等数据,通过所搜集的大数据来制定商业决策依据,以及通过数据挖掘形式,找到创新产品的机会。

大的互联网公司在满足自己内部决策需求的同时,也成了了大数据部门给其它公司进行赋能,比如蚂蚁金服的\sphinxstylestrong{数据产品芝麻信用},不仅能够成为蚂蚁内部各种金融产品的信用审核依据,也开放给了很多行业如出行、金融、共享服务公司等,极大提高了基于信用服务的门槛和便捷性。

通过数据收集处理分析驱动产品的价值验证、功能优化和业务决策


\paragraph{分析方法}
\label{\detokenize{chapter_idea/data:id2}}

\subparagraph{决策支持}
\label{\detokenize{chapter_idea/data:id3}}
决策支持是通过简单的求和以及易于理解的分析模型,帮助用户做出决策,比如对比本月同比和环比用户平均消费金额,从而决定通过什么决策活动来提高本月的用户平均消费金额。比如建立一个广告投入因素和新增用户的关系模型,就能够预测投入多少广告额,能带来多少新增用户。

简单的关系模型产品经理是能通过Excel表格分析出来的,如柱状图、折线图等。

如果一项因素引发问题的因素很复杂,则需要建立一个由多个因素组成的预测模型。通过这个模型,我们可以观察模型中某个因素对整体结果造成的影响。预测模型需要用到的统计方法有交叉列表统计、统计学假设检验
、多元回归分析等,这个阶段大部分产品经理都需要求助数据分析师的帮助了。


\subparagraph{系统优化}
\label{\detokenize{chapter_idea/data:id4}}
系统优化指的是帮助用户构建让计算机执行的方案算法,常用的系统优化方法有机器学习。

相比简单模型的决策模型,系统通过机器学习方法分析出系统中更详细的因素,比如系统优化能分析出广告投入多少金额,能带来新用户的快速增长,以及广告投放中具体什么投放渠道,效果最好。

机器学习的优势在于能从数据中学习出其本身包含的模式和规律,并以此来建立模型。比今日头条,就是通过分析我们过去浏览的记录,利用机器学习建立模型,从而给我们推荐类似的内容。系统优化用到的统计方法有逻辑回归分析、聚类、主成分分析、决策树分析等。


\paragraph{解决对策}
\label{\detokenize{chapter_idea/data:id5}}
把预估的数据代入到决策模型中,进行模拟仿真,来评估执行决策结果的成本以及决策风险;并相互沟通这种有依据的成本。


\paragraph{业务数据2\sphinxfootnotemark[112]}
\label{\detokenize{chapter_idea/data:id6}}%
\begin{footnotetext}[112]\sphinxAtStartFootnote
\sphinxnolinkurl{http://www.woshipm.com/pmd/3657472.html}
%
\end{footnotetext}\ignorespaces 
AI产品也需要采用类似数据埋点的方式去收集产品投放前后的业务指标差异,比如:GMV差异、点击率差异、转化率差异。首先为了验证产品是否对业务产生了价值,用一个粗略的公式表示AI产品的业务价值,其次是为了分析产品的哪些品功能存在优化空间,最后还可以驱动业务决策,例如例如推荐系统在电商商品推荐和广告推荐中的应用。

AI产品价值=(提高的时效*时效成本+GMV提升)\sphinxhyphen{}(AI硬件资源成本+研发成本)


\paragraph{数据沉淀}
\label{\detokenize{chapter_idea/data:id7}}
AI产品除了收集业务指标数据指导产品是否需要优化,还需要进一步做好训练数据沉淀工作。AI技术在投入试点到成熟推广,训练数据一直都是必不可少的,尤其是真实场景的数据对算法迭代更是起到“致命”的作用。

因此,如果能够源源不断的回收实际场景数据并且清洗标注,就可以提升算法准确率指标,最终提高产品使用效果,例如:可以考虑通过以下流程来实现。


\paragraph{放入真实商业环境 3\sphinxfootnotemark[113]}
\label{\detokenize{chapter_idea/data:id8}}%
\begin{footnotetext}[113]\sphinxAtStartFootnote
\sphinxnolinkurl{https://www.cnwebe.com/articles/43675.html}
%
\end{footnotetext}\ignorespaces 
不止GMV=DAU\sphinxstyleemphasis{转化率}客单价
\begin{enumerate}
\sphinxsetlistlabels{\arabic}{enumi}{enumii}{}{.}%
\item {} 
剔除虚假证据

\item {} 
深入发现问题

\item {} 
挖掘潜在因素

\item {} 
观察长期趋势

\end{enumerate}


\paragraph{工具:}
\label{\detokenize{chapter_idea/data:id9}}
神策分析、GrowingIO、友盟这种工具平台


\subsubsection{谈判}
\label{\detokenize{chapter_idea/negotiation:id1}}\label{\detokenize{chapter_idea/negotiation::doc}}
产品经理需要谈判的地方也很多,比如涉及跨部门合作需要对接资源的时候,涉及部门利益的交换,比如与上下游合作的供应商有业务往来需要谈判。


\paragraph{本质}
\label{\detokenize{chapter_idea/negotiation:id2}}
产品经理在跨部门沟通需要拿资源的时候,这个主动权在资源方,要做到不卑不吭,这里的双赢心态第一要素是要突出谈判结果对谈判方是有利的,然后对公司是有利的,最后才说对自己是有利的。

产品经理在与供应商谈判的时候,这时主动权在自己这方,这个时候也需要可以同时与多家同级别的供应商进行谈判,突出迂回策略。


\paragraph{过程}
\label{\detokenize{chapter_idea/negotiation:id3}}
主动:不迫不及待 最好先引导对方先说出自己的条件
语言上的不接受,还要再情绪表达上,适当的沉默
借鉴行业标准,行业标准是法律法规、先前的惯例、市场行业价格。


\paragraph{谈判技巧}
\label{\detokenize{chapter_idea/negotiation:id4}}\begin{itemize}
\item {} 
黑脸白脸

\item {} 
虚构领导

\item {} 
最后恭喜

\end{itemize}


\paragraph{协议2\sphinxfootnotemark[114]}
\label{\detokenize{chapter_idea/negotiation:id5}}%
\begin{footnotetext}[114]\sphinxAtStartFootnote
\sphinxnolinkurl{http://terms.aliyun.com/legal-agreement/terms/suit\_bu1\_ali\_cloud/suit\_bu1\_ali\_cloud201802261104\_19214.html?spm=a2c4g.11186623.2.11.58bf5c39z5GukP}
%
\end{footnotetext}\ignorespaces 
免费:不排除日后收取费用的可能,将提前10个自然日通过在网站内合适版面发布公告或发送站内通知等方式公布收费政策及规范。仍使用阿里云服务的,应按届时有效的收费政策付费并应遵守届时公布的有效的服务条款。拒绝支付服务费的,有权不再向您提供服务,并有权利不再继续保留您的业务数据。

完整性和保密性:您对的数据以及进入和管理口令、密码的完整性和保密性负责。因您维护不当或保密不当致使上述数据、口令、密码等丢失或泄漏所引起的损失和后果均由您承担。

数据:当服务期届满、服务提前终止(包括双方协商一致提前终止,其他原因导致的提前终止等)或您发生欠费时,除法律法规明确规定、主管部门要求或双方另有约定外,仅在一定的缓冲期(以您所订购的服务适用的专有条款、产品文档、服务说明等所载明的时限为准)内继续存储您的用户业务数据(如有),缓冲期届满将删除所有用户业务数据,包括所有缓存或者备份的副本,不再保留您的任何用户业务数据。用户业务数据一经删除,即不可恢复;您应承担数据因此被删除所引发的后果和责任,您理解并同意,阿里云没有继续保留、导出或者返还用户业务数据的义务。


\subsubsection{品牌}
\label{\detokenize{chapter_idea/brand:id1}}\label{\detokenize{chapter_idea/brand::doc}}

\paragraph{品牌管理}
\label{\detokenize{chapter_idea/brand:id2}}

\subparagraph{定义}
\label{\detokenize{chapter_idea/brand:id3}}
针对企业产品和服务的品牌,综合地运用企业资源,通过计划、组织、实施、控制来实现企业品牌战略目标的经营管理过程。

品牌是一种错综复杂的象征。它是品牌属性、名称、包装、价格、历史、信誉,广告方式的无形总称。品牌同时也是消费者对其使用者的印象,以其自身的经验而有所界定。产品是工厂生产的东西;品牌是消费者所购买的东西。产品可以被竞争者模仿,但\sphinxstylestrong{品牌则是独一无二}的,产品极易迅速过时落伍,但成功的品牌却能持久不坠,品牌的价值将长期影响企业。


\subparagraph{金三角}
\label{\detokenize{chapter_idea/brand:id4}}
金三角=领导能力+分析能力+沟通与协调能力
\begin{enumerate}
\sphinxsetlistlabels{\arabic}{enumi}{enumii}{}{.}%
\item {} 
领导能力:品牌经理必须是一个能驱动事情发生的领导者,虽然他领导的对象可能根本就不是他的下属,他需要驱动多个部门、甚至老板一起完成品牌目标。

\item {} 
分析能力:品牌经理必须是个发现机会的高手,分析中找到品牌的成长机会。

\item {} 
沟通与协调能力:品牌经理必须在口头和书面沟通上都具备突出的资质,以支持其行动。

\end{enumerate}


\subparagraph{银三角}
\label{\detokenize{chapter_idea/brand:id5}}
银三角=执行能力+创新能力+掌握专业技能的能力
\begin{enumerate}
\sphinxsetlistlabels{\arabic}{enumi}{enumii}{}{.}%
\item {} 
执行能力:想到还要做到,品牌经理要在关键时刻推动事情发生,自己动手+请别人动手,关键是看到结果。

\item {} 
创新能力:品牌经理不能因循守旧,要能够突破过去,不断创新。

\item {} 
掌握专业技能的能力:这个不用多说,品牌经理必须是专家+杂家的组合,专业范畴的市场研究、广告管理、促销管理、产品管理、品牌战略规划都是基本功,还需要汲取其它专业部门的经验和专业知识,才能全面发展。

\end{enumerate}


\paragraph{品牌资产 1\sphinxfootnotemark[115]}
\label{\detokenize{chapter_idea/brand:id6}}%
\begin{footnotetext}[115]\sphinxAtStartFootnote
\sphinxnolinkurl{http://reader.epubee.com/books/mobile/e2/e22be26cde02a62274cac6fa3d3c6fb5/text00006.html?fromPre=last}
%
\end{footnotetext}\ignorespaces 
品牌意识越强,消费者在选购产品时越会想到该品牌,最终购买该品牌的可能性也越大。
同时,品牌意识也会影响品牌联想与品牌形象的形成及其强度。

品牌资产的主要项目有:
\begin{enumerate}
\sphinxsetlistlabels{\arabic}{enumi}{enumii}{}{.}%
\item {} 
品牌意识(brand name awareness)

\item {} 
品牌忠诚(brand loyalty)

\item {} 
感知质量(perceived quality)

\item {} 
品牌联想(brand associations)

\end{enumerate}

\begin{figure}[H]
\centering
\capstart

\noindent\sphinxincludegraphics{{brand_asset}.jpg}
\caption{品牌资产}\label{\detokenize{chapter_idea/brand:id7}}\end{figure}


\subparagraph{品牌意识(brand name awareness) 2\sphinxfootnotemark[116]}
\label{\detokenize{chapter_idea/brand:brand-name-awareness-2}}%
\begin{footnotetext}[116]\sphinxAtStartFootnote
\sphinxnolinkurl{http://reader.epubee.com/books/mobile/e2/e22be26cde02a62274cac6fa3d3c6fb5/text00007.html}
%
\end{footnotetext}\ignorespaces 
指的是一个品牌在消费者心中的强度。假如在消费者心中布满了心理看板,每一个看板对应一个品牌的话,品牌在消费者心中意识的强弱就是对应看板的大小。品牌意识是根据消费者对一个品牌的不同的记忆方式进行测量的,从再认(以前曾见过这一品牌吗)到回忆(这类产品你能记起哪些品牌),再到“第一回忆”(第一个回忆出的品牌),最后到支配(唯一回忆出的品牌)。然而,心理学家和经济学家认为,再认和回忆不止是记得一个品牌的信号。

品牌再认(recognition)反映了从过去的接触中获得的熟悉度。再认不必记得在哪里见过该品牌,该品牌为什么与其他品牌有所不同,甚至不需要知道该品牌所属的产品类别。它只需要消费者记得曾经见过该品牌即可。


\subsubsection{费米推理}
\label{\detokenize{chapter_idea/decompose:id1}}\label{\detokenize{chapter_idea/decompose::doc}}
费米解答问题的方式是推理思维,估算只是方法,对问题分解是推理的必然过程。这个思维过程就是——哪些真实存在的因素导致这样的事情发生?

就是在信息不完整的情况下,凭借对对象事物的深刻理解和洞察,科学地作出一些假设使得问题得以简化,复杂的程度得以降低,从而得到符合或接近实际的估计。

费米推论是将复杂、困难的问题分解成小的、可以解决的部分,从而以最直接的方法迅速解决问题。

我们将问题分解为可知和不可知的部分,对尚未确定的答案的部分一概视为未知,说出自己的假设并加以验证。大胆猜想,用于试错。宁愿迅速发现错误,也不要用模糊的措辞隐藏错误。对于“不可知的因素”,凭直觉和经验给出结果,最后,得出结论。


\subsubsection{转换成本1\sphinxfootnotemark[117]}
\label{\detokenize{chapter_idea/convert:id1}}\label{\detokenize{chapter_idea/convert::doc}}%
\begin{footnotetext}[117]\sphinxAtStartFootnote
\sphinxnolinkurl{http://www.woshipm.com/pmd/3431686.html}
%
\end{footnotetext}\ignorespaces 

\paragraph{社交货币}
\label{\detokenize{chapter_idea/convert:id2}}
社交货币也有分类,分为内部和外部,外部比较典型的就是利用物质与用户产生连接或者增加其社交价值。相对而言,用户喜欢贪一些小便宜,他的关系网里面就会聚集一些同类爱好的用户,这种情况的传播其实是在增加他的价值。

你的关系网里面没有这样的人,则需要谨慎分享。比如你的人脉圈里都是一些比较清高、爱面子的人,这时候外物货币就不太适合对这群人起作用。

这时候,你要做社交货币,就要从内部出发,比如打造荣耀、归属感、等级制度等等来自精神层面的内部货币。


\paragraph{核心存量}
\label{\detokenize{chapter_idea/convert:id3}}
每个产品都必然有自己的核心存量,也就是你给用户提供的核心价值,成本就是围绕核心价值来展开。

核心存量:当你的产品失去这个存量后,用户将会迅速流失。


\paragraph{连接度}
\label{\detokenize{chapter_idea/convert:id4}}
你加了多少个人\textasciicircum{}即社交关系的链接度,这里的连接度分为深度和宽度。

深度:你的通讯录里面有没有人跟你关系很好。
宽度:你的通讯录里面有多少个好友。

很多人都想着再出一个社交的app来打败微信,你去看看那些想要打败微信的社交软件,它们活的怎么样了?

腾讯很清楚,微信最大的威胁绝对不是同类社交app,而是其他领域的潜力巨无霸,比如:抖音,就是腾讯完全没有预料到的,让腾讯措手不及。

很多老板就是喜欢盯着对手的动作来展开行动,产品经理也很难办,这种情况的老板陷入了知识的诅咒,他们的焦点必然会盯着对手。他们甚至不懂产品,不懂运营,但是你懂啊,你要给他讲清楚这些道理,你甚至可以直接说“我比你专业,请听我的”,老板为啥花钱请你?因为你专业。


\subsubsection{商业思维}
\label{\detokenize{chapter_idea/business:id1}}\label{\detokenize{chapter_idea/business::doc}}
创业活动,不断创新的商业模式、线上线下的商业营销活动、除了教育等几个少数行业,大部分产业成熟的标志是实现商业化等等


\paragraph{商业}
\label{\detokenize{chapter_idea/business:id2}}
商业是一种有组织的提供顾客所需产品与服务的一种行为,它最本质的内容就是通过产品与服务交换实现盈利。


\paragraph{商业模式画布 1\sphinxfootnotemark[118]}
\label{\detokenize{chapter_idea/business:id3}}%
\begin{footnotetext}[118]\sphinxAtStartFootnote
\sphinxnolinkurl{http://www.woshipm.com/pmd/2180363.html}
%
\end{footnotetext}\ignorespaces 
商业模式画布(BMC)是著名商业模式创新作家、商业顾问亚历山大·奥斯特瓦德在2008年提出的概念。

商业画布是一种能够帮助创业者催生创意、降低猜测、确保他们找对了目标用户合理解决问题的工具。

商业画布不仅能够提供更多灵活多变的计划,还更容易满足用户的需求。更重要的是它可以将商业模式中的元素标准化井强调元素间的相互作用。

\begin{figure}[H]
\centering
\capstart

\noindent\sphinxincludegraphics{{business_draw}.png}
\caption{商业画布}\label{\detokenize{chapter_idea/business:id22}}\end{figure}
\begin{enumerate}
\sphinxsetlistlabels{\arabic}{enumi}{enumii}{}{.}%
\item {} 
客户细分(Customer
Segments):为谁服务?谁来买单?大众/小众市场、利基市场、区隔化市场、多元化市场、多边平台市场。

\item {} 
价值主张(Value
Propositions):服务或产品有什么价值?颠覆式创新、更快更好、个性定制、专注把事情做好、优秀的设计、价格优势、削减成本、抑制风险、连接、方便易用等特点。

\item {} 
渠道通路(Channels):认知、评估、购买、传递、售后;通路有:搜索引擎、公众平台、应用商店、线下资源等。

\item {} 
客户关系(Customer
Relationships):借助客户口碑传播获客从而维持持续收入

\item {} 
核心资源(Key
Resources):实体资产用户基数、知识产权、人力资源、金融资产、经营资质、用户基数

\item {} 
关键业务(Key
Activities):具体如何服务客户(驱动你做出产品、需求变化)

\item {} 
重要合作(Key
Partnerships):非竞争者之间的战略联盟关系、与竞争者之间的战略合作关系、为开发新业务而构建的合资关系、以及买卖关系。

\item {} 
收入来源(Revenue
Streams):售卖实体产品、使用权收费、租凭收费、“中介”收费、广告收费。

\item {} 
成本结构(Cost Structure):成本驱动型是越少越好

\end{enumerate}

不用纠结商业模式画布是不是最好的商业模式模型,只要将其作为商业模式设计入门的初步理解材料即可
\sphinxhref{https://www.zhihu.com/question/21472586s}{2}%
\begin{footnote}[119]\sphinxAtStartFootnote
\sphinxnolinkurl{https://www.zhihu.com/question/21472586s}
%
\end{footnote}

工具:\sphinxhref{https://bms.your01.com/}{BMS(商业模式沙盘:Business Mode
Sandboxie)}%
\begin{footnote}[120]\sphinxAtStartFootnote
\sphinxnolinkurl{https://bms.your01.com/}
%
\end{footnote}


\paragraph{多层次 11\sphinxfootnotemark[121]}
\label{\detokenize{chapter_idea/business:id4}}%
\begin{footnotetext}[121]\sphinxAtStartFootnote
\sphinxnolinkurl{https://weread.qq.com/web/reader/0c032c9071dbddbc0c06459k1c3321802231c383cd30bb3}
%
\end{footnotetext}\ignorespaces 
将人工智能产品的规划、设计、实践与商业模式画布相结合,提出在产品、市场和效益3个层面进行人工智能产品设计,并列出了设计过程中涉及的14项指标。在这3个层面上,通过14项指标,产品经理可以轻松构建人工智能产品画布,如下图所示。人工智能产品画布可以帮助产品经理高效地确定产品规划、厘清产品脉络、确定产品结构,从而提升人工智能产品的设计效率。

\begin{figure}[H]
\centering
\capstart

\noindent\sphinxincludegraphics{{chapter_idea/../img/business_cengci}.png}
\caption{多层次分析}\label{\detokenize{chapter_idea/business:id23}}\end{figure}
\begin{enumerate}
\sphinxsetlistlabels{\arabic}{enumi}{enumii}{}{.}%
\item {} 
\sphinxstylestrong{产品层面}:产品层面主要包括一些产品实现的细节:一是人工智能产品的实现方案,包括产品目标、范围、可行性及关键功能架构;二是人工智能产品应用的具体行业和场景,并确定该产品在该场景下实现的效能指标和价值指标;三是实现人工智能产品的技术选型、算法分析和技术指标设计等。

\item {} 
\sphinxstylestrong{市场层面}:人工智能产品是否成功关键在于产品是否可以获得市场的认可。即使产品非常优秀,如果没有被市场和客户认可,一切投入也都将化为乌有。产品经理在产品规划过程中应从市场层面完成针对产品使用者、购买者、影响者、决策者等的客群分析,完成竞争对手分析,完成产品定价策略规划,以及完成渠道规划。市场层面产品规划最关键的内容是确定产品价值主张。产品价值主张不仅指明了产品方向,而且关系到产品的成败。产品价值主张包括产品带来的社会价值、生产力价值等。

\item {} 
\sphinxstylestrong{效益层面}:在一个产品概念创立之初,需要建立人工智能产品的效益目标,效益目标可以从经济效益、社会效益等不同层面进行考量,作为产品经理要着重对产品的成本和收入进行分析。产品经理应对效益目标进行拆解,计算投入产出比,如果投入产出比不够理想,则研发该产品没有意义。

\end{enumerate}


\paragraph{商业切入口}
\label{\detokenize{chapter_idea/business:id5}}
最初小米公司只是做手机的,当手机做的不错的时候,他们发现可以基于自己IOT的研发能力,做其它智能产品;其它产品基于小米的IOT平台,能够实现研发,品牌,供应链等联动优势,最后形成一个基于智能硬件的智能网络生态,最大化的实现了产品的商业价值。


\paragraph{商业化idea}
\label{\detokenize{chapter_idea/business:idea}}
那些值得为之探索商业模式的idea应该源于创始人对某些事物长期思考和体会得到的一些不同寻常的见解。长期来看,
新的创业机会一定是技术创新引起的,
而商业化idea往往拼的是如何理解新技术给社会带来的变化。


\paragraph{目的 3\sphinxfootnotemark[122]}
\label{\detokenize{chapter_idea/business:id6}}%
\begin{footnotetext}[122]\sphinxAtStartFootnote
\sphinxnolinkurl{https://www.zhihu.com/question/348474416/answer/841775222}
%
\end{footnotetext}\ignorespaces 
有些技术人天生有player的属性,所谓player,就是不满足于只做pawn的人。

技术只是手段,那手段用上了必然要有其目的。盈利?赚吆喝?布局?完成投资人的任务?Whatever
you name it.

所以不论是承担项目研发,技术负责,产品负责还是自己创业,有商业思维的人,总会比没有的人看问题深一层,跟他们聊天往往能感受到犀利,不同于纯技术思维的那种“愣”劲儿。


\paragraph{商业模式 7\sphinxfootnotemark[123]}
\label{\detokenize{chapter_idea/business:id7}}%
\begin{footnotetext}[123]\sphinxAtStartFootnote
\sphinxnolinkurl{https://36kr.com/p/1721542885377}
%
\end{footnotetext}\ignorespaces 
商业模式就是指企业设计完整的商业逻辑,从而实现企业的生存价值。

设计合理的商业模式,就要充分考虑和运用企业运行的内外要素,从而形成一个完整的高效率商业化运行系统,它可以保持产品独特的核心竞争力,并通过最优形式满足客户需求、实现自身价值,与此同时达成持续盈利的目标。
\sphinxhref{https://weread.qq.com/web/reader/40632860719ad5bb4060856kc0c320a0232c0c7c76d365a}{10}%
\begin{footnote}[124]\sphinxAtStartFootnote
\sphinxnolinkurl{https://weread.qq.com/web/reader/40632860719ad5bb4060856kc0c320a0232c0c7c76d365a}
%
\end{footnote}

至少包含了四个方面:产品模式、用户模式、推广模式,最后才是盈利模式。一句话,商业模式是你能提供一个什么样的产品,给什么样的用户创造什么样的价值,在创造用户价值的过程中,用什么样的方法获得商业价值。


\subparagraph{产品模式}
\label{\detokenize{chapter_idea/business:id8}}
所有的商业模式都要建立在产品模式的基础之上,没有了对产品和用户的思考,公司是不可能做大的,这样的公司注定也走不了多远。


\subparagraph{用户模式}
\label{\detokenize{chapter_idea/business:id9}}
你一定要找到对你的产品需求最强烈的目标用户。如果你说自己的产品是普世的产品,是放之四海而皆准的产品,这就说明你没有认真思考过。
\begin{itemize}
\item {} 
YY语音:帮助这些游戏工会在游戏对战中多对多沟通

\item {} 
UC手机浏览器:解决省流量的问题,因为当时手机流量很贵,网速慢且资费高

\end{itemize}


\subparagraph{推广模式}
\label{\detokenize{chapter_idea/business:id10}}
即使产品做得再好,如果只靠自然的口碑,只要还没接触到大多数目标用户,就有可能先被互联网巨头盯上。人家一模仿一捆绑,你多年的心血就算白费了。

真正的推广模式是要根据你的用户群,根据你的产品,去设计相应的推广方法。而不是拿钱去刷地铁、刷公交、刷路牌广告。

真正的推广是对产品的不断完善和提升。在推广的过程中,你要研究市场,和目标用户打交道,了解用户真正的需求,了解用户使用产品时遇到的困惑和问题,然后再反馈到产品上进行改进,由此不断帮助产品调整和完善。


\subparagraph{盈利模式}
\label{\detokenize{chapter_idea/business:id11}}
Google的两个天才创始人做搜索引擎,好几年找不到赚钱的方法,只能是给雅虎这类的门户网站提供搜索技术服务来赚点糊口的钱。

Overture创造的付费点击模式,确实为广告客户创造了商业价值,但作为寄生于搜索引擎的企业,Overture却并没有为用户创造价值。反而是Google将搜索引擎的用户价值和Overture的付费点击模式完美地结合在了一起。


\subparagraph{价值层面}
\label{\detokenize{chapter_idea/business:id12}}
BCG的价值3层面把商业模式分成了价值定位和价值传导2个大的层面,每个层面又包括3个小的具体模块,需要分别设计和规划。
\sphinxhref{https://weread.qq.com/web/reader/40632860719ad5bb4060856kc0c320a0232c0c7c76d365a}{10}%
\begin{footnote}[125]\sphinxAtStartFootnote
\sphinxnolinkurl{https://weread.qq.com/web/reader/40632860719ad5bb4060856kc0c320a0232c0c7c76d365a}
%
\end{footnote}


\paragraph{交易模型}
\label{\detokenize{chapter_idea/business:id13}}
以交易为基本单元来研究产品,目标是建立可持续交易的互惠模型


\subparagraph{企业、用户、产品关系}
\label{\detokenize{chapter_idea/business:id14}}
用户选择产品:效用\sphinxhyphen{}成本>0:
\begin{itemize}
\item {} 
直接成本:付出的时间、金钱、数据、态度等

\item {} 
间接成本:为了促成交易,付出的搜寻成本

\end{itemize}

企业生产产品:收益\sphinxhyphen{}成本>0
\begin{itemize}
\item {} 
收益:现金收入、增加未来收益的各方信任、品牌声誉等

\end{itemize}


\subparagraph{效用(欲望的满足程度)的三个属性}
\label{\detokenize{chapter_idea/business:id15}}\begin{itemize}
\item {} 
多样性:时间、欲望、心里感觉、情绪、认知

\item {} 
无限性:需求永远无法被完全满足,因为需求是会变得越来越大的

\item {} 
个体性:人会受到情境、禀赋、偏好、认知等影响,所以同一个产品带来的效用,对于不同的人来说差距很大,信息的完全性及原有的思维框架会影响每个人对效用的判断

\end{itemize}


\subparagraph{交易成本}
\label{\detokenize{chapter_idea/business:id16}}
交易成本:完成一笔交易时,交易双方在买卖前后所产生的各种与此交易相关的成本。也可以理解为”所有买方(卖方)付出但是卖方(买方)没有收到的成本。

降低交易成本
\begin{itemize}
\item {} 
标准化:把供给品尽量变成标准品,降低了度量成本,降低了不确定性带来的决策成本和保障成本

\item {} 
线上化:降低了企业与用户触发、服务、维护等成本

\end{itemize}


\paragraph{三级火箭 9\sphinxfootnotemark[126]}
\label{\detokenize{chapter_idea/business:id17}}%
\begin{footnotetext}[126]\sphinxAtStartFootnote
\sphinxnolinkurl{https://www.jianshu.com/p/ff38ced05cbd}
%
\end{footnotetext}\ignorespaces 
互联网商业就是产品、流量、转化率三个词。

第一级:搭建高频头部流量 第二级:沉淀某类用户的商业场景
第三级:完成商业闭环


\subparagraph{例子}
\label{\detokenize{chapter_idea/business:id18}}
360的一级火箭是免费杀毒工具;二级火箭是从免费杀毒工具变为网络安全平台(360安全浏览器、360安全网址导航);三级火箭就是它最终承载的商业闭环(从安全浏览器和网址导航的广告收入)。

搜狗现在的一级火箭是来自腾讯的头部流量;二级火箭是内置搜索,通过庞大的使用场景去释放更多搜索的需求。三级火箭即商业变现。

逻辑思维第一级火箭是罗振宇坚持了多年的免费脱口秀;第二级火箭是得到APP,沉淀用户的商业场景;第三级火箭,得到APP里面的付费课程。

小米的一级火箭是手机;二级火箭是一系列的零售场景(小米商城、小米之家、小米小店);三级火箭是一个高利润的产品。

你要赚利润的东西,并非是他人要赚钱的地方。面对这样的竞争者,传统的生意套路会失效。你以为亚朵在做酒店,其实亚朵在做社群共创的实景电商。


\subparagraph{必要条件}
\label{\detokenize{chapter_idea/business:id19}}\begin{enumerate}
\sphinxsetlistlabels{\arabic}{enumi}{enumii}{}{.}%
\item {} 
三级火箭递推一定是高频推低频。

\item {} 
通过一级火箭获得大量用户之后,要快速开展一个能够沉淀用户的商业场景。

\item {} 
操控三级火箭的人,一定是个势能积累到一定程度的人。(首先要有强大的融资能力;其次在头部流量铺开的时候要有短时间聚拢资源的能力)

\item {} 
操盘三级火箭的人一定是个狠人。(一级火箭就是抢别人流量,要能够承受他人指责)

\end{enumerate}


\subparagraph{原理}
\label{\detokenize{chapter_idea/business:id20}}
火箭级数越多,需要的燃料越少。但每增加一级,不可控程度越高。就像做商业,模型过于复杂,变现链条过长,就容易玩脱。

所以,三级火箭是一个成本和可控性平衡后的选择。


\subparagraph{AI产品的商业化}
\label{\detokenize{chapter_idea/business:ai}}\begin{itemize}
\item {} 
基于企业服务费的商业路径:参照行业内对手的收费模式,是按单收费,还是按配置收费
(朝头部客户去做,有大订单,投入产出比高,eg:金融领域千万订单)

\item {} 
基于互联网玩法的商业路径:小度音箱的模式,先近于免费抢占市场,后割韭菜。(SAAS服务一视同仁,找代理铺量)

\end{itemize}


\paragraph{阿里云视觉智能开放平台 4\sphinxfootnotemark[127] 5\sphinxfootnotemark[128] 6\sphinxfootnotemark[129]}
\label{\detokenize{chapter_idea/business:id21}}%
\begin{footnotetext}[127]\sphinxAtStartFootnote
\sphinxnolinkurl{https://help.aliyun.com/document\_detail/143096.html?spm=a2c4g.11186623.6.548.1a4a53cblCY4Zg}
%
\end{footnotetext}\ignorespaces %
\begin{footnotetext}[128]\sphinxAtStartFootnote
\sphinxnolinkurl{https://developer.aliyun.com/article/778839?spm=a2c6h.12873581.0.dArticle778839.5de439932BzTaX\&groupCode=viapi}
%
\end{footnotetext}\ignorespaces %
\begin{footnotetext}[129]\sphinxAtStartFootnote
\sphinxnolinkurl{https://help.aliyun.com/document\_detail/182962.html?spm=a211p3.14020179.J\_7524944390.13.738f4b58g1fD6Y}
%
\end{footnotetext}\ignorespaces 
商业化提供了预付费QPS、后付费、预付费资源包、按量付费四种收费模式。

离线SDK介绍:阿里云视觉智能开放平台的离线SDK可以为终端设备提供AI能力,目前支持提供OCR、美颜、分割等常用AI能力的离线SDK。阿里云视觉智能开放平台通过license授权方式管理离线SDK。

准备工作:在安装和使用阿里云SDK前,确保您已经注册阿里云账号并生成访问密钥(AccessKey)。详情请参见创建AccessKey。


\subsubsection{批判性思维1\sphinxfootnotemark[130]}
\label{\detokenize{chapter_idea/critical:id1}}\label{\detokenize{chapter_idea/critical::doc}}%
\begin{footnotetext}[130]\sphinxAtStartFootnote
\sphinxnolinkurl{http://www.woshipm.com/pmd/284339.html}
%
\end{footnotetext}\ignorespaces 
《学会提问:批判性思维指南》(尼尔·布朗)

信息大爆炸时代,我们每天都会接收到很多的观点,批判性的去阅读变得异常重要。本书提供了一个批判思维的完整结构

什么是论题?什么是结论? 理由是什么? 哪些词语有歧义?
什么是价值观冲突?什么是价值观假设? 什么是描述性假设?
推理中存在谬误吗? 这些证据的可信度有多大? 你发型干扰性原因了吗?
统计数据是否具有欺骗性? 哪些重要数据被遗漏了? 什么结论可能是合理的?


\subsubsection{注意力}
\label{\detokenize{chapter_idea/attention:id1}}\label{\detokenize{chapter_idea/attention::doc}}
小便池里假苍蝇,溅出率降低60\%。

微博的小红点、发布(橙色) 淘宝里的立即购买(红色)、加入购物车(橙色)
分类导航上下差5\sphinxhyphen{}10\%


\paragraph{AIDA模型}
\label{\detokenize{chapter_idea/attention:aida}}
AIDA模式也称“爱达”公式,是艾尔莫·李维斯Elmo Lewis
在1898年首次提出总结的推销模式,是西方推销学中一个重要的公式,它的具体函义是指一个成功的推销员必须把顾客的注意力吸引或转变到产品上,使顾客对推销人员所推销的产品产生兴趣,这样顾客欲望也就随之产生,尔后再促使采取购买行为,达成交易。

AIDA是四个英文单词的首字母:A为Attention,即引起注意;I为Interest,即诱发兴趣;D为Desire,即刺激欲望;最后一个字母A为Action,即促成购买。

AIDA模型可以看作是用户做购买决策的整个行为过程,从引起注意到最终的购买行为的过程。

尽管AIDA模型是为线下推销场景提出的,但放在线上电商鼎盛的今天,依旧有很大的参考价值,下面,我们就来分析一下现有的线上产品是如何运用AIDA模型来辅助用户购买决策的。


\subsubsection{Goal}
\label{\detokenize{chapter_idea/goal:goal}}\label{\detokenize{chapter_idea/goal::doc}}

\paragraph{清晰的学习的目标}
\label{\detokenize{chapter_idea/goal:id1}}\begin{itemize}
\item {} 
可行性:这个级别的产品经理要能对一个需求或问题给出高可行性的解决方案,大约是市场上的P6级,小领域的熟练执行者

\item {} 
创造:要能为一个需求或问题找到最优解。这需要不断洞察环境、用户的持续变化和趋势

\item {} 
权衡:能跳出单个需求,从全局角度考虑权衡取舍。即使一个需求为真、可行、有最优解,最重要的还是决定当下要不要做,分配多少资源做

\item {} 
变迁:能跳出当下,对世事变迁敏感,对需求或问题做决策时,形成习惯性地预判(并反思自己的判断是否正确)能力

\item {} 
方法论:这个级别要求有成体系的优质方法论输出。这个层级主要是追求“影响力”和“确定性”

\end{itemize}


\paragraph{no code1\sphinxfootnotemark[131]}
\label{\detokenize{chapter_idea/goal:no-code1}}%
\begin{footnotetext}[131]\sphinxAtStartFootnote
\sphinxnolinkurl{https://github.com/cedrickchee/knowledge/blob/master/courses/fast.ai/deep-learning-part-1/2019-edition/lesson-7-resnet-unet-gan-rnn.md}
%
\end{footnotetext}\ignorespaces 
So as long we need to code, we failed that because 99.8\% of the world
can’t code. The main goal would be to get to a point where it’s not a
library but a piece of software that doesn’t required code. It certainly
shouldn’t require a lenghty hardworking course like this one. I want to
get rid of the course, get rid of the code. I want to make it so you can
do usual stuff quickly and easily. That’s may be 5 years, may be longer.


\subsubsection{技术理解力}
\label{\detokenize{chapter_idea/understand_tech:id1}}\label{\detokenize{chapter_idea/understand_tech::doc}}

\paragraph{为何需要}
\label{\detokenize{chapter_idea/understand_tech:id2}}
在传统意义上,产品经理根据需求特性,抽象产品特性,思考产品逻辑,制定产品目标、愿景、实施计划,拟定详细的文档,按照交互\sphinxhyphen{}设计\sphinxhyphen{}重构\sphinxhyphen{}前后台开发\sphinxhyphen{}测试验收上线的流程,一步步推进即可。但看似合理且被大多数产品经理认为是理所当然的流程中,却隐藏着技术理解层面严重的bug。


\paragraph{对软件设计的理解问题}
\label{\detokenize{chapter_idea/understand_tech:id3}}\begin{itemize}
\item {} 
面向过程,是指以任务/事件为中心,进行软件设计。

\item {} 
面向对象,是指以事物为中心的软件设计。

\end{itemize}

搭乘地铁从T站到F站的简单例子来说明:

如果用面向过程的设计方式,会将地铁启动、运行、到站等一系列的动作设定为过程,也许还要设定地铁维修的过程。然后将这所有的\sphinxstylestrong{过程按照逻辑}串在一起,完成一个任务。

如果用面向对象的方式设计,那直接将地铁定义为\sphinxstylestrong{对象},地铁的颜色、形状等则为\sphinxstylestrong{属性},地铁的运行和到站就是地铁的\sphinxstylestrong{方法},也即地铁的行为,而不是过程。


\paragraph{对需求实施的理解问题}
\label{\detokenize{chapter_idea/understand_tech:id4}}
曾因为一个简单页面的图文修改,对技术人员嗤之以鼻,但当了解内情后才发现,不仅仅是修改html的问题,还涉及到css、json、数据库读取修改以及数据字段的调整。所以对于需求的理解,从产品经理和技术人员角度而言,所看到的大小和范围,也许就像冰山一样,水面和水底有很大的区别。

在这种技术层面的改动要大于产品预期的情况,难免就会产生分歧。为了尽量使需求的实施理解,也能保持同步,可以参考以下要素:
\begin{enumerate}
\sphinxsetlistlabels{\arabic}{enumi}{enumii}{}{.}%
\item {} 
参加技术人员的概要设计评审:当产品需求提到技术层面时,一般技术人员会对需求进行概要设计、评审、详细设计及评审、开发实施等环节。当然产品经理一般不会在技术层面介入太深,但为了尽量使需求不偏离目标,参加技术层面的概要设计评审,是很好的一个选择,虽然对于多数产品经理而言,不一定能全听懂技术在概要设计层面的讨论。参加概要设计评审可以了解需求在启动技术设计时,涉及到的相关系统、干系人、内外部团队等,大致了解技术实施层面的困难、瓶颈和资源需求。以减少用户类型、路径等环节的偏差。

\item {} 
提前向技术同步产品的远期愿景:同步产品愿景和长期版本目标,可以是在需求刚出现时,也可以是在交互设计时,但个人感觉最晚不能晚于技术的概要设计。提前同步产品愿景,可以在技术人员做技术设计时,能确定数据、架构、迭代以及预留字段,更能确定技术实现方式,是按照较大的系统实施,还是按照简单的逻辑实施,因为很多时候,技术的实现方式有多种选择。以免产品的期望是宏伟大厦,因为没有提前同步给技术,导致技术在打地基时,按照普通的平房实施了。

\item {} 
了解需求中的关键点:这一点需要在每一次技术沟通中进行确认,但尽量在技术概要设计前了解清楚,这也就是参加技术概要设计评审的重要性所在。了解需求的关键点,了解了相关困难、瓶颈、资源需求等,对于需求实施的排期、时间节点评估则会掌握的比较清晰。

\end{enumerate}


\paragraph{项目进度推进}
\label{\detokenize{chapter_idea/understand_tech:id5}}
产品和技术都转换思维,首先是了解对方的想法,然后是从对方角度思考,共同发掘问题和困难所在,再去解决。这样提前预估、制定时间节点、共同督促的推进方式,才能使项目推进更顺利。
\begin{enumerate}
\sphinxsetlistlabels{\arabic}{enumi}{enumii}{}{.}%
\item {} 
根据需求的关键点,把控项目进度:前文提到,了解需在技术实施环节的关键节点,目的就是为了整体把控需求,防止在关键节点掉链子。有时是需要产品协助,或是督促技术打通关键节点的问题,有时则是因为前期的评估和了解,提前将实施中关键节点可能存在的问题消化掉。

\item {} 
需求实施的“时间最小单元”不能太久:需求实施的“时间最小单元”,我把它定义为,需求实施过程中,可以标识为里程碑或是有明确交付物的最短时间。例如一个H5的登录注册功能的开发,判断每个输入框信息输入格式是否准确,将信息提交至数据库,数据库写入数据并返回是否正确写入,给用户对应的反馈,这些每个环节的开发所需时间,都可以理解为一个时间最小单元。按照正常的逻辑,这样的时间最小单元,建议是0.5天至3天,最好不超过3天。

\item {} 
时不时的讨论推进的困难和进度:对于推进实施中的需求,不能当成一个完全交出去的任务,更不能当“甩手掌柜”,而是应该参照时间最小单元,不时的讨论推进中是否存在困难,应如何解决困难,询问时间最小单元中的推进进度,如有没有进度,则可能需要调整计划了。

\end{enumerate}


\paragraph{什么是技术架构?}
\label{\detokenize{chapter_idea/understand_tech:id6}}
架构就是对系统中的实体以及实体之间的关系所进行的抽象描述,是一系列的决策,架构也是产品的结构和愿景。

系统架构是概念的体现,是对物/信息的功能与形式元素之间的对应情况所做的分配,是对元素之间的关系以及元素同周边环境之间的关系所做的定义。

做好架构是个复杂的任务,也是个很大的话题,本篇就不做深入了。有了架构之后,就需要让干系人理解、遵循相关决策。


\paragraph{单体应用和微服务}
\label{\detokenize{chapter_idea/understand_tech:id7}}
同样的,在早期大部分应用不会考虑到技术架构,但随着用户增加和未来性能要求则需要重构,这就需要到技术资深的架构师。而市面上的架构主要分为下面三类

单体应用程序:应用程序的全部功能被一起打包作为单个单元或应用程序.这个单元可以是JAR、WAR、EAR,或其他一些归档格式,但其全部集成在一个单一的单元.
微服务:微服务是一个新兴的软件架构,就是把一个大型的单个应用程序和服务拆分为数十个的支持微服务。一个微服务的策略可以让工作变得更为简便,它可扩展单个组件而不是整个的应用程序堆栈,从而满足服务等级协议。


\subparagraph{单体应用}
\label{\detokenize{chapter_idea/understand_tech:id8}}

\subparagraph{优点}
\label{\detokenize{chapter_idea/understand_tech:id9}}\begin{enumerate}
\sphinxsetlistlabels{\arabic}{enumi}{enumii}{}{.}%
\item {} 
方便调试,代码都在一起;

\item {} 
没有分布式开销,所有服务都在本地容器内;

\item {} 
中小型项目可以快速迭代,不需要太多资源。单体应用缺点:

\end{enumerate}


\subparagraph{缺点}
\label{\detokenize{chapter_idea/understand_tech:id10}}\begin{enumerate}
\sphinxsetlistlabels{\arabic}{enumi}{enumii}{}{.}%
\item {} 
可复用性差:服务被打包在应用中,功能不易复用;

\item {} 
系统启动慢,一个进程包含了所有的业务逻辑,涉及到的启动模块过多,导致系统的启动、重启时间周期过长。

\item {} 
线上问题修复周期长;任何一个线上问题修复需要对整个应用系统进行全面升级。

\end{enumerate}


\subparagraph{面向服务架构(SOA)}
\label{\detokenize{chapter_idea/understand_tech:soa}}\label{\detokenize{chapter_idea/understand_tech:id12}}\label{\detokenize{chapter_idea/understand_tech:id11}}

\subparagraph{企业服务总线(ESB)}
\label{\detokenize{chapter_idea/understand_tech:esb}}
ESB是面向服务架构(SOA)的核心构成部分,指传统数据连接技术(web、xml、中间件技术)结合的产物,简单来说,就是一根管道,用来连接各个服务节点,为了集成不同系统,不同协议的服务,服务总线做了消息的转化解释和路由工作,让不同的服务互联互通;是一个具有标准接口、实现了互连、通信、服务路由。


\subparagraph{特点}
\label{\detokenize{chapter_idea/understand_tech:id13}}\begin{enumerate}
\sphinxsetlistlabels{\arabic}{enumi}{enumii}{}{.}%
\item {} 
系统集成:从系统角度讲,解决了企业系统与系统间通信问题,把原来散乱、无规划的系统间的网状结构梳理成规整,可治理的系统。在梳理时则需要引用一些产品,常用的是企业服务总线(ESB)、技术规范、服务管理规范。主要解决核心问题,无序变有序。

\item {} 
系统的服务化:从功能角度讲,把业务转换成可复用、可组装的服务,通过服务的编排实现业务的快速复制。目的是把原先固有的业务功能转变为通用的业务服务,实现快速复用。主要解决的核心问题,原来固有业务可复用。

\item {} 
业务的服务化:从企业的角度讲,把原来职能化的企业架构转变为服务化的企业架构,进一步提升企业的对外服务能力。把一个业务单元封装成一项服务。主要解决的核心问题是高效。

\end{enumerate}


\subparagraph{优点}
\label{\detokenize{chapter_idea/understand_tech:id14}}\begin{enumerate}
\sphinxsetlistlabels{\arabic}{enumi}{enumii}{}{.}%
\item {} 
数据统一,共享数据库,使服务接口使用同一的数据模型的数据,确保数据一致性

\item {} 
灵活性较高,缩短产品和服务的上线时间,降低了开发与改变流程的成本系统

\item {} 
由子系统组成,系统易于重构

\end{enumerate}


\subparagraph{缺点}
\label{\detokenize{chapter_idea/understand_tech:id15}}\begin{enumerate}
\sphinxsetlistlabels{\arabic}{enumi}{enumii}{}{.}%
\item {} 
技术不匹配,在某些情况并不能轻松对操作平台进行重新打包,原因是业务功能结构需求不匹配

\item {} 
系统间交互需要使用远程通讯 ,一定程度上降低了响应速度

\end{enumerate}


\subparagraph{微服务架构}
\label{\detokenize{chapter_idea/understand_tech:id17}}\label{\detokenize{chapter_idea/understand_tech:id16}}\label{\detokenize{chapter_idea/understand_tech:id18}}

\subparagraph{优点}
\label{\detokenize{chapter_idea/understand_tech:id19}}\begin{enumerate}
\sphinxsetlistlabels{\arabic}{enumi}{enumii}{}{.}%
\item {} 
分而治之;单个服务功能内聚,复杂性低;方便团队的拆分和管理;

\item {} 
单独部署,独立开发;

\item {} 
易于扩展,某一项服务的性能达到瓶颈,只需增加该服务的节点数即可,其它服务不改变

\item {} 
易于维护,每个微服务的职责单一,复杂性降低,不会牵一发而动全身

\end{enumerate}


\subparagraph{缺点}
\label{\detokenize{chapter_idea/understand_tech:id20}}\begin{enumerate}
\sphinxsetlistlabels{\arabic}{enumi}{enumii}{}{.}%
\item {} 
开发难度大,前期服务的定义和拆分需要较大工作量,每个服务都需要单独部署,运维、测试成本增加;

\item {} 
跨服务的调用通常是不同的机器,甚至是不同的机房,开发人员需要处理超时、网络异常等问题,原来的函数调用改为服务调用。

\item {} 
效率相对低,团队依赖强,一个服务的版本延迟会拖慢整个应用的开发周期。

\item {} 
需要分布式事务的支持。

\end{enumerate}


\subparagraph{中台 10\sphinxfootnotemark[132]}
\label{\detokenize{chapter_idea/understand_tech:id21}}%
\begin{footnotetext}[132]\sphinxAtStartFootnote
\sphinxnolinkurl{https://www.jianshu.com/p/a5894e8ba3f3}
%
\end{footnotetext}\ignorespaces 
中台是随着公司业务高速发展,组织不断膨胀的过程中暴露的问题需要解决。将企业的核心能力随着业务不断发展以数字化形式沉淀到平台,形成以服务为中心,由业务中台和数据中台构建起数据闭环运转的运营体系,供企业更高效的进行业务探索和创新。
中台做到前后分离,后台统一提供数据接口,前台实现业务流转。
\begin{enumerate}
\sphinxsetlistlabels{\arabic}{enumi}{enumii}{}{.}%
\item {} 
中台与微服务的区别:

\end{enumerate}
\begin{itemize}
\item {} 
中台是提升企业的能力的复用,一种方法论/思想。

\item {} 
微服务是独立开发、维护、部署的小型业务组件,一种技术架构。

\end{itemize}
\begin{enumerate}
\sphinxsetlistlabels{\arabic}{enumi}{enumii}{}{.}%
\setcounter{enumi}{1}
\item {} 
中台与微服务的关系:

\end{enumerate}
\begin{itemize}
\item {} 
微服务架构,是实现中台思想的落地的重要手段。

\end{itemize}
\begin{enumerate}
\sphinxsetlistlabels{\arabic}{enumi}{enumii}{}{.}%
\setcounter{enumi}{2}
\item {} 
中台解决的核心问题:

\end{enumerate}
\begin{itemize}
\item {} 
为减少重复业务系统开发及实现系统数据共享一个技术平台底座,将多年技术沉淀的价值最大化,统一各个业务部门或系统重复使用、重复建设的功能和系统统一规划和管理。

\end{itemize}
\begin{enumerate}
\sphinxsetlistlabels{\arabic}{enumi}{enumii}{}{.}%
\setcounter{enumi}{3}
\item {} 
什么时候需要中台:

\end{enumerate}
\begin{itemize}
\item {} 
如阿里:淘宝,有订单、库存、评价、积分、物流等业务系统。天猫也有订单、库存、评价、积分、物流等业务系统。1688,也有类似业务系统。多个系统有重复业务系统需要建设,且系统间数据不能完全共享,系统各自运行。此时使用技术中台以及业务中台,来实现业务重用及数据共享,把技术沉淀价值最大化。

\end{itemize}

AI中台,更多见:\sphinxurl{https://aieye-top.github.io/d2cl/chapter\_deploy/AI-zhongtai.html}


\subparagraph{SaaS}
\label{\detokenize{chapter_idea/understand_tech:saas}}
SaaS是一个服务需求方的完整解决方案产品产品,如它为顾客提供了完整的端到端解决方案,如计算后台到客户操作终端;\sphinxhref{http://www.woshipm.com/pd/4090455.html}{11}%
\begin{footnote}[133]\sphinxAtStartFootnote
\sphinxnolinkurl{http://www.woshipm.com/pd/4090455.html}
%
\end{footnote}


\paragraph{与测试相关的专业名词}
\label{\detokenize{chapter_idea/understand_tech:id22}}\begin{enumerate}
\sphinxsetlistlabels{\arabic}{enumi}{enumii}{}{.}%
\item {} 
提测研发人员在开发完某个功能之后,把代码打包并提交给测试人员开始测试,就叫提测。

\item {} 
复现之前测试发现的Bug
再次出现,就叫复现。能否复现对于研发人员排查Bug非常重要。

\item {} 
测试用例测试用例是指测试人员根据PRD
撰写的测试流程及事项。例如,知乎要上线一个收藏文章的功能,点击“收藏”按钮,该文章就出现在收藏列表中,并且“收藏”按钮变为“已收藏”按钮,这就是一条测试用例;点击“已收藏”按钮,“已收藏”按钮就变为“收藏”按钮,同时该文章从收藏列表中消失,这就是另外一条测试用例。

\item {} 
功能测试功能测试是单一功能的测试,如某次迭代要做一个分享功能,功能测试就是测试分享这个功能是否符合PRD
的要求。

\item {} 
回归测试可以将回归测试理解为整体测试。例如,某次迭代要上线一个分享功能,需要测试一下这个功能是否会影响其他功能的正常使用,所以回归测试要测试的就是整个产品的所有功能。

\item {} 
测试报告测试报告是指在测试完成之后,由测试人员撰写的说明Bug
均已修复,可以上线的邮件或报告。

\end{enumerate}


\paragraph{懂技术,更得懂AI的局限 4\sphinxfootnotemark[134]}
\label{\detokenize{chapter_idea/understand_tech:ai-4}}%
\begin{footnotetext}[134]\sphinxAtStartFootnote
\sphinxnolinkurl{https://www.chinaventure.com.cn/news/114-20191210-350906.html}
%
\end{footnotetext}\ignorespaces 
除了基本的产品技能还要掌握AI基础技术知识,如NLP自然语言、DL深度学习、ML机器学习、大数据等

AI公司的产品里一类是应用AI技术的垂直业务产品,另外一类是AI服务的平台产品。前者负责AI能力在细分领域的应用;后者则是对AI能力的汇总和包装,例如各种AI开放平台、各种云计算平台,这就要求产品经理必须熟知公司内部的AI技术能力,还要有能力作为售前支持,为使用方提供技术咨询。

当实现一款功能的设计的时候,最基础的认知就是要首先确定什么能做什么不能做,对于可见的一些服务,比方说手机APP中的用户使用用链路来讲,一个功能能否实现是比较容易确定的。但是如果是AI类产品的设计,需要涉及到对算法以及数据的理解,只有当产品经理真正了解每种算法的玩法以及数据的使用链路,才可以将功能做活,保留高鲁棒性。

大部分的AI产品的服务对象是to B端的企业用户,
B端用户和C端用户的使用行为习惯是截然不同的,所以就有很多C端的产品转向B端出现的水土不服。


\paragraph{机器学习增加了不确定性 7\sphinxfootnotemark[135]}
\label{\detokenize{chapter_idea/understand_tech:id23}}%
\begin{footnotetext}[135]\sphinxAtStartFootnote
\sphinxnolinkurl{https://www.oreilly.com/radar/what-you-need-to-know-about-product-management-for-ai/}
%
\end{footnotetext}\ignorespaces 
有了机器学习,我们通常会得到一个在统计上比简单技术更准确的系统,但也有一个缺点,那就是一小部分模型预测总是错误的,有时会以难以理解的方式出现。

这种转变需要在软件工程实践中进行根本性的改变。用看似相似的输入输出对数据集训练的相同神经网络代码可以给出完全不同的结果。相同代码生成的模型输出将随着训练数据的大小(标记示例的数量)、网络训练参数和训练运行时等内容的变化而变化。这对软件测试、版本控制、部署和其他核心开发过程有严重的影响。

对于任何给定的输入,相同的程序不一定会产生相同的输出;输出完全取决于模型是如何训练的。对训练数据进行更改,用相同的代码重复训练过程,您将从模型中得到不同的输出预测。也许差别很细微,也许差别很大,但它们是不同的。

在这种不确定性之下,开发过程本身还存在着进一步的不确定性。很难预测一个人工智能项目需要多长时间。对传统软件来说,预测开发时间已经够困难的了,但至少我们可以根据过去的经验做出一些一般性的猜测。我们知道“进步”是什么意思。使用人工智能,你通常不知道会发生什么,直到你尝试它。花上几周甚至几个月的时间才能找到可行的方法,将模型的准确率从70\%提高到74\%,这种情况并不少见。很难说最大的模型改进是来自更好的神经网络设计、输入特征还是训练数据。你经常不能告诉经理模型将在下周或下个月完成;你的下一次尝试可能会成功,或者你可能会受挫好几个星期。你常常不知道某件事是否可行,直到你做了实验。


\paragraph{技术可行性 9\sphinxfootnotemark[136]}
\label{\detokenize{chapter_idea/understand_tech:id24}}%
\begin{footnotetext}[136]\sphinxAtStartFootnote
\sphinxnolinkurl{https://wiki.mbalib.com/wiki/\%E6\%8A\%80\%E6\%9C\%AF\%E5\%8F\%AF\%E8\%A1\%8C\%E6\%80\%A7}
%
\end{footnotetext}\ignorespaces 
技术可行性是指决策的技术和决策方案的技术不能突破组织所拥有的或有关人员所掌握的技术资源条件的边界。

做技术可行性分析时需注意全面考虑系统开发过程所涉及的所有技术问题,尽可能采用成熟技术,慎重引入先进技术,着眼于具体的开发环境和开发人员,技术可行性评价等问题。


\subparagraph{精确定义}
\label{\detokenize{chapter_idea/understand_tech:id25}}
“可行”的一个重要部分是精确定义。正如杰里米·乔丹所说:“一个明确定义的问题已经解决了一半。”如果你能非常精确地说出你想要完成的事情,并把它分解成更简单的问题,你就有了一个良好的开端。Jordan有一些很好的建议:从自己动手解决问题开始。如果你想帮助客户整理他们手机上的图片,花点时间整理你的手机上的图片。与真正的客户面谈,看看他们想要什么。建立一个他们可以用真实数据尝试的原型。最重要的是,不要认为“我们想帮助客户组织图片”是一个充分的问题陈述。它不是;你必须更详细地了解你的客户是谁,他们想如何组织他们的图片,他们可能有什么样的图片,他们想如何搜索,等等。


\paragraph{数据标记}
\label{\detokenize{chapter_idea/understand_tech:id26}}
看看您可以多快地为ML算法构建一个带有标记的基准数据集以及明确、狭窄定义的精度目标。数据标记的便捷性是机器学习是否具有成本效益的一个很好的代理。如果您可以在产品的正常用户活动中构建数据标记(例如,标记垃圾邮件),那么您就有机会收集足够多的输入\sphinxhyphen{}输出对来训练您的模型。否则,您将为标记数据的外部服务烧钱,而且在您进行第一次演示之前的前期成本很容易成为项目中最昂贵的部分。没有大量的原始数据和标注的训练数据,解决大多数人工智能问题是不可能的。


\paragraph{“BUG”一词}
\label{\detokenize{chapter_idea/understand_tech:bug}}
“BUG”一词在工程师与产品经理的角度中可能也存在偏差,在工程师的角度看,它是因为代码或者逻辑出错而导致的功能性错误,因为不影响产品功能的优化,所以不是BUG。而产品经理的角度看,认为是影响用户体验的产品BUG,本质上是交互设计问题,在加载过程中需要对用户有所提示使得产品体验更好。两种角度,两种观点,而书中告诉了我们解决方法。


\subsubsection{运营销}
\label{\detokenize{chapter_idea/yunyingxiao:id1}}\label{\detokenize{chapter_idea/yunyingxiao::doc}}

\paragraph{产品运营 1\sphinxfootnotemark[137]}
\label{\detokenize{chapter_idea/yunyingxiao:id2}}%
\begin{footnotetext}[137]\sphinxAtStartFootnote
\sphinxnolinkurl{https://baike.baidu.com/item/\%E4\%BA\%A7\%E5\%93\%81\%E8\%BF\%90\%E8\%90\%A5/1978562}
%
\end{footnotetext}\ignorespaces 
产品与运营的关系是:产品负责界定和提供长期用户价值,运营负责创造短期价值,并完善长期价值。很多产品的长期价值往往用户一时半会儿感知不到,需要运营创造一些短期价值去刺激用户使用和体验,并根据用户的持续反馈调整、迭代、优化来完善长期价值。

运营相当于一个保姆,是一项从内容建设,用户维护,活动策划三个层面来管理产品内容和用户的职业。
\begin{enumerate}
\sphinxsetlistlabels{\arabic}{enumi}{enumii}{}{.}%
\item {} 
内容建设:建立标准, 挽留、裂变:防止劣质的内容驱逐优质的用户

\item {} 
用户维护:挽留:建立完善Q\&A机制,解决用户投诉和困难;拉新:主动邀请有价值的用户来使用产品。

\item {} 
活动策划:煽动用户互动,加强产品品牌。产品运营最接近用户,需求质量高。

\end{enumerate}


\paragraph{营销}
\label{\detokenize{chapter_idea/yunyingxiao:id3}}

\subparagraph{营销很好,没有盈利}
\label{\detokenize{chapter_idea/yunyingxiao:id4}}\begin{enumerate}
\sphinxsetlistlabels{\arabic}{enumi}{enumii}{}{.}%
\item {} 
提高用户量;

\item {} 
客单价(用户价值);

\item {} 
成本管理;

\item {} 
增加资产的周转率;

\item {} 
寻找“增长杠杆”。

\end{enumerate}


\subparagraph{饥饿营销}
\label{\detokenize{chapter_idea/yunyingxiao:id5}}
真正目的不是为了利润,而是为了品牌附加值。

前提:
\begin{enumerate}
\sphinxsetlistlabels{\arabic}{enumi}{enumii}{}{.}%
\item {} 
产品具备不可替代性

\item {} 
消费者心智不成熟

\item {} 
市场竞争不激烈。

\end{enumerate}

副作用:
\begin{enumerate}
\sphinxsetlistlabels{\arabic}{enumi}{enumii}{}{.}%
\item {} 
客户流失。过度饥饿营销,就是将客户“送”给竞争对手。

\item {} 
顾客反感。过度饥饿营销,会让消费者饿到冷静,觉得被愚弄,对品牌产生厌恶。

\end{enumerate}


\subsubsection{设计思维 1\sphinxfootnotemark[138]}
\label{\detokenize{chapter_idea/design:id1}}\label{\detokenize{chapter_idea/design::doc}}%
\begin{footnotetext}[138]\sphinxAtStartFootnote
\sphinxnolinkurl{https://weread.qq.com/web/reader/8d232b60721a488e8d21e54kaab325601eaab3238922e53}
%
\end{footnotetext}\ignorespaces 
产品经理的核心工作是产品设计。产品设计的重点在于提升产品的用户体验,让用户喜欢并且坚持使用我们的产品。

提升产品的用户体验分两个维度。第一个维度是提升产品的可用性,即让产品确实能解决用户的需求“痛点”,这部分内容我会在第4
章中详细讲解。第二个维度是提升产品的易用性,即让产品对用户来说易于学习和使用、记忆负担小,且产品的使用满意度高。要想提升产品的易用性,需要仔细打磨产品交互,而交互设计七大定律和尼尔森十大原则就是在解决产品易用性的问题。


\paragraph{交互设计七大定律}
\label{\detokenize{chapter_idea/design:id2}}\begin{enumerate}
\sphinxsetlistlabels{\arabic}{enumi}{enumii}{}{.}%
\item {} 
费茨定律

\item {} 
7±2 法则

\item {} 
奥卡姆剃刀原理

\item {} 
接近法则

\item {} 
希克定律

\item {} 
特斯勒定律

\item {} 
新乡重夫:防错原则

\end{enumerate}


\paragraph{尼尔森十大原则 2\sphinxfootnotemark[139]}
\label{\detokenize{chapter_idea/design:id3}}%
\begin{footnotetext}[139]\sphinxAtStartFootnote
\sphinxnolinkurl{https://weread.qq.com/web/reader/0c032c9071dbddbc0c06459k37632cd021737693cfc7149}
%
\end{footnotetext}\ignorespaces 
\sphinxstylestrong{尼尔森十大可用性原则}是人机交互学博士尼尔森(Nielsen)在分析了200多个可用性问题后提炼出的十项通用型原则,是产品设计与用户体验设计的重要参考标准,值得深入研究与运用。
\begin{enumerate}
\sphinxsetlistlabels{\arabic}{enumi}{enumii}{}{.}%
\item {} 
\sphinxstylestrong{状态可见原则(Visibility of System
Status)}:在产品使用过程中应该让用户知道发生了什么,并在合适的时间做出合适的反馈。

\item {} 
\sphinxstylestrong{环境贴切原则(Match between System and the Real
World)}:在进行产品设计时应使用用户熟悉的语言体系、操作模式,尽量遵循现实世界中符合逻辑的交互过程。

\item {} 
\sphinxstylestrong{用户可控原则(User Control and
Freedom)}:对于用户的一些误操作提供二次确认和错误修正功能,这样可以提高产品的可控性,否则对产品进行一些关键性操作可能带来毁灭性的打击。例如,数据删除操作,如果没有二次确认,可能会造成严重的影响。

\item {} 
\sphinxstylestrong{一致性原则(Consistency and
Standards)}:在进行产品设计时采用的语言体系、操作模式、视觉风格、组件样式等应保持统一。

\item {} 
\sphinxstylestrong{防错原则(Error
Prevention)}:在进行产品设计时,应提供防止误操作的机制,减少用户犯错的可能。例如,当关键数据没有填写时,保存按钮置灰等。

\item {} 
\sphinxstylestrong{无回忆原则(Recognition Rather
thanRecall)}:在产品使用过程中,应尽量减少用户对操作过程的记忆负荷,所有的动作和处理都应该是可见的。用户在一个页面进行操作和处理时,无须记忆上一个页面的内容。

\item {} 
\sphinxstylestrong{灵活高效原则(Flexibility and Efficiency of
Use)}:在产品设计过程中,应充分考虑操作的灵活性和信息传递的高效性,简化操作过程,减少信息传递过程,为用户提供更便捷的操作方式。就像在电子商务系统中寻找商品一样,如果商品种类过多,用户需要逐级查找才可以寻找到目标商品,为了避免出现分类过多的问题,电子商务系统提供了虚拟分类、分类导航等多种方式让用户可以便捷地找到目标商品。

\item {} 
\sphinxstylestrong{易扫原则(Aesthetic and Minimalist
Design)}:保留主要信息展示,尽量避免无关信息影响主要信息,保证主要信息的简洁和美观。互联网用户浏览界面的动作不是读,也不是看,而是扫。易扫,意味着突出重点,弱化和剔除无关信息。

\item {} 
\sphinxstylestrong{容错原则(Help Users Recognize,Diagnose,and Recover From
Errors)}:产品设计时应充分考虑用户可能出现的错误,并设计功能帮助用户从错误中恢复,将损失降到最低。如果无法自动挽回,则应提供详尽的说明文字和指导方向。

\item {} 
\sphinxstylestrong{人性化帮助原则(Help
andDocumentation)}:优秀的产品不需要帮助文档,如果必须要提供帮助文档,则以流程化、图形化的形式提供。

\end{enumerate}


\paragraph{违反原则}
\label{\detokenize{chapter_idea/design:id4}}
之前版本,支付宝支付在美团付款页面都处于折叠状态。————阿里巴巴就撤离了美团,便再次将目标投向了“饿了么”,之后美团就一直是依靠腾讯这颗“大树”生存。
\sphinxhref{https://new.qq.com/omn/20200801/20200801A0CZYF00.html}{4}%
\begin{footnote}[140]\sphinxAtStartFootnote
\sphinxnolinkurl{https://new.qq.com/omn/20200801/20200801A0CZYF00.html}
%
\end{footnote}


\subsection{skill}
\label{\detokenize{chapter_skill/index:skill}}\label{\detokenize{chapter_skill/index:chap-skill}}\label{\detokenize{chapter_skill/index::doc}}

\subsubsection{商业化产品“七步设计法” 1\sphinxfootnotemark[141]}
\label{\detokenize{chapter_skill/7steps:id1}}\label{\detokenize{chapter_skill/7steps::doc}}%
\begin{footnotetext}[141]\sphinxAtStartFootnote
\sphinxnolinkurl{http://www.woshipm.com/pd/3784247.html}
%
\end{footnotetext}\ignorespaces 
方法论试图提炼出从0到1设计商业化产品的通用思路,重点介绍宏观设计逻辑,即具备普适性的全链路商业化产品设计流程,而不倾向于介绍某款单品类产品的设计细节。但也会在各个流程环节会特别说明适用的产品,比如适合to
C产品还是to B产品,或者适合广告产品、会员产品还是其他商业化产品。


\paragraph{如何设计一款专业的、具备竞争力的商业化产品呢?}
\label{\detokenize{chapter_skill/7steps:id2}}
你们可能会从不同的视角给出不同的答案,这可能是岗位差异造成的。然而,当我们从公司的视角考虑商业化产品设计时,就必须具备全局思维,即面向商业化全链路来设计可闭环变现的产品。

在商业流通市场中,无论商业化产品属于哪种形态或者哪个种类,产品所遵循的商业化底层逻辑是相通的,即无论你的变现模式是卖流量、卖软件、卖服务,还是其他的变现模式,最终你卖的都是“商品”,都需要遵循商业化规律和一些普适的商业操作标准。


\paragraph{“七步设计法”}
\label{\detokenize{chapter_skill/7steps:id3}}
第一步,输出BRD、资源评估报告等。
第二步,输出产品设计规划、解决方案设计规划等。
第三步,输出PRD(产品需求文档)、产品交付清单等。
第四步,输出产品报价单、SKU目录、财务模型等。
第五步,输出售卖渠道策略、售卖政策、销售协议等。
第六步,输出各类商业化产品包装资料、商业化产品官网等。
第七步,输出产品发布管理文档、各类产品培训资料等。


\subsubsection{BRD 1\sphinxfootnotemark[142]}
\label{\detokenize{chapter_skill/BRD:brd-1}}\label{\detokenize{chapter_skill/BRD::doc}}%
\begin{footnotetext}[142]\sphinxAtStartFootnote
\sphinxnolinkurl{http://www.woshipm.com/pmd/178527.html}
%
\end{footnotetext}\ignorespaces 

\paragraph{定义}
\label{\detokenize{chapter_skill/BRD:id1}}
BRD指的是商业需求文档(Business Requirement
Document)。在这篇的文档当中不会有详细的产品规划,只会有基于市场调查和用户需求调查的产品构思。

BRD
是给谁看的呢?老板、投资人、股东,目的是让他们知道这款产品如何给公司盈利。BRD
的撰写侧重点是需求描述、盈利模式。产品总监或者产品VP(Vice
President,副总监)才需要写BRD,初级产品经理基本接触不到BRD

这篇文档常以PPT等形式,以数据说服他们,帮产品的立项,来获得公司资源支持。


\paragraph{顺势而为 7\sphinxfootnotemark[143]}
\label{\detokenize{chapter_skill/BRD:id2}}%
\begin{footnotetext}[143]\sphinxAtStartFootnote
\sphinxnolinkurl{https://www.jianshu.com/p/a4b1fd94b49a}
%
\end{footnotetext}\ignorespaces 
\begin{figure}[H]
\centering
\capstart

\noindent\sphinxincludegraphics{{point_filter}.png}
\caption{点子过滤}\label{\detokenize{chapter_skill/BRD:id16}}\end{figure}


\paragraph{产品立项流程 4\sphinxfootnotemark[144]}
\label{\detokenize{chapter_skill/BRD:id3}}%
\begin{footnotetext}[144]\sphinxAtStartFootnote
\sphinxnolinkurl{https://www.bilibili.com/video/BV1254y1D7Ht?from=search\&seid=14167562900175777805}
%
\end{footnotetext}\ignorespaces 
项目概述 商业价值 项目的目标 项目风险 项目干系人、组织和其他产品


\paragraph{前期调研 3\sphinxfootnotemark[145]}
\label{\detokenize{chapter_skill/BRD:id4}}%
\begin{footnotetext}[145]\sphinxAtStartFootnote
\sphinxnolinkurl{https://www.bilibili.com/video/BV1wz4y1y7sg}
%
\end{footnotetext}\ignorespaces 
人群特征 需求特点 用户价值 体量规模 竞争优势 产品定位 资源能力 成本收益


\paragraph{产品目的 2\sphinxfootnotemark[146]}
\label{\detokenize{chapter_skill/BRD:id5}}%
\begin{footnotetext}[146]\sphinxAtStartFootnote
\sphinxnolinkurl{http://www.woshipm.com/pmd/21446.html}
%
\end{footnotetext}\ignorespaces 
提出一个清晰、简明的价值主张,让它很容易被接受,要让产品团队、管理人员、用户、市场人员清楚的明白这个产品到底是什么意图。

考虑“velevator pitch”(电梯间演讲、电梯行销)测试。


\paragraph{一个中心}
\label{\detokenize{chapter_skill/BRD:id6}}
收益!来生存


\paragraph{多个基本点}
\label{\detokenize{chapter_skill/BRD:id7}}\begin{enumerate}
\sphinxsetlistlabels{\arabic}{enumi}{enumii}{}{.}%
\item {} 
市场调查报告;

\item {} 
竞争对手报告;

\item {} 
用户需求调研报告;

\item {} 
产品功能构思;

\item {} 
产品运营构思;

\item {} 
收益分析;

\item {} 
风险分析;

\item {} 
其他。

\end{enumerate}


\paragraph{例子}
\label{\detokenize{chapter_skill/BRD:id8}}
\begin{figure}[H]
\centering
\capstart

\noindent\sphinxincludegraphics{{BRD}.jpg}
\caption{支付宝用户事业部产品提案模板}\label{\detokenize{chapter_skill/BRD:id17}}\end{figure}


\paragraph{可行性评估 3\sphinxfootnotemark[147]}
\label{\detokenize{chapter_skill/BRD:id9}}%
\begin{footnotetext}[147]\sphinxAtStartFootnote
\sphinxnolinkurl{https://www.bilibili.com/video/BV1wz4y1y7sg}
%
\end{footnotetext}\ignorespaces 
MVP:核心是试错,有反馈渠道。
PMF:Dropbox用视频介绍产品的功能,来测试反馈。核心功能内测

最短时间 最小成本 可行性验证


\paragraph{可用性测试}
\label{\detokenize{chapter_skill/BRD:id10}}

\subparagraph{确定测试目标:}
\label{\detokenize{chapter_skill/BRD:id11}}\begin{itemize}
\item {} 
产品设计方案

\item {} 
测试功能点

\item {} 
A/B测试

\item {} 
产品改进方案测试

\end{itemize}


\subparagraph{定义用户:}
\label{\detokenize{chapter_skill/BRD:id12}}\begin{itemize}
\item {} 
可从用户访谈和问卷调查中选择

\item {} 
存量用户或者新用户

\end{itemize}


\subparagraph{测试过程记录:}
\label{\detokenize{chapter_skill/BRD:id13}}\begin{itemize}
\item {} 
录屏、录音和摄像

\item {} 
记录A/B选项结果

\item {} 
页面埋点追踪

\item {} 
过程中的疑惑点,改进点及时其他特殊情况

\end{itemize}


\subparagraph{结果分析根据过程记录总结、修改方案,如:}
\label{\detokenize{chapter_skill/BRD:id14}}\begin{itemize}
\item {} 
通过统计分析追踪结果

\item {} 
AB测试结果得出改进方案

\end{itemize}


\paragraph{四轮 MVP 框架}
\label{\detokenize{chapter_skill/BRD:mvp}}
VUCA 的中文含义分别对应着易变性、不确定性、复杂性和模糊性。V:Volatility
易变性U:Uncertainty 不确定性C:Complexity 复杂性A:Ambiguity 模糊性

如今VUCA时代信息无时无刻不在变化,用户的需求无时无刻不在变化。

\begin{figure}[H]
\centering
\capstart

\noindent\sphinxincludegraphics{{MVP}.png}
\caption{MVP框架}\label{\detokenize{chapter_skill/BRD:id18}}\end{figure}
\begin{enumerate}
\sphinxsetlistlabels{\arabic}{enumi}{enumii}{}{.}%
\item {} 
Paperwork:产出物是纸面研究的结论,用的方法是 Discovery
Sprint,探索冲刺。

\item {} 
Prototype:在方案层面“先发散,后收敛”,做出原型,获得反馈后,不断修正原型,用的方法叫
Design Sprint,设计冲刺。

\item {} 
Product:验证的重点是真实产品是否可以培养出用户习惯,用户愿意用,能更高效地解决用户需求、创造价值,并且让用户愿意反复使用。这时候,我们会关注某些和用户留存有关的指标。

\item {} 
Promotion:做小规模推广尝试,测试渠道,逐步确定优选渠道,降低分销成本。对应的方法论是
Distribution Sprint,分销冲刺。

\end{enumerate}

注意:
\begin{enumerate}
\sphinxsetlistlabels{\arabic}{enumi}{enumii}{}{.}%
\item {} 
用户参与都是必须的

\item {} 
过滤器的开口应该越来越小

\item {} 
在每一轮停留的时间、投入的资源也不尽相同

\item {} 
这四轮走完,产品也才刚刚上路

\end{enumerate}


\paragraph{项目风险 RAID 4\sphinxfootnotemark[148]}
\label{\detokenize{chapter_skill/BRD:raid-4}}%
\begin{footnotetext}[148]\sphinxAtStartFootnote
\sphinxnolinkurl{https://www.bilibili.com/video/BV1254y1D7Ht?from=search\&seid=14167562900175777805}
%
\end{footnotetext}\ignorespaces \begin{itemize}
\item {} 
Risk风险:会对项目产生负面影响的事件,事件可能发生的概率和随之对项目带来的影响

\item {} 
Assumption假设:知群可以预想到的因素,一旦发生就会促进项目成功(但不发生就没有促进效果)

\item {} 
Issues问题:在项目中任何不怡当的,需要管理和解决的事情,这些事情需要持续跟踪并记录

\item {} 
Dependence依赖:任何项目所依赖的或者依赖该项目的事件和工作,需要记录依赖实现的时间

\end{itemize}


\paragraph{PEST分析 5\sphinxfootnotemark[149]}
\label{\detokenize{chapter_skill/BRD:pest-5}}%
\begin{footnotetext}[149]\sphinxAtStartFootnote
\sphinxnolinkurl{https://zh.wikipedia.org/wiki/PEST\%E5\%88\%86\%E6\%9E\%90}
%
\end{footnotetext}\ignorespaces 
政治因素包含了租税政策、劳工法律、环境管制、贸易限制、关税与政治稳定。
经济因素有经济增长、利率、汇率和通货膨胀率。
社会因素通常着重在文化观点,另外还有健康意识、人口成长率、年龄结构、工作态度及安全需求。
科技因素包含生态与环境方面,决定进入障碍和最低有效生产水准,影响委外购买决策。科技因素着重在研发活动、自动化、技术诱因和科技发展的速度。

PEST分析与外部总体环境的因素互相结合就可归纳出SWOT分析中的机会与威胁。


\paragraph{波特五力分析 6\sphinxfootnotemark[150]}
\label{\detokenize{chapter_skill/BRD:id15}}%
\begin{footnotetext}[150]\sphinxAtStartFootnote
\sphinxnolinkurl{https://zh.wikipedia.org/wiki/PEST\%E5\%88\%86\%E6\%9E\%90}
%
\end{footnotetext}\ignorespaces 
波特五力分析来定义出一个市场吸引力高低程度。

来自买方的议价能力、来自供应商的议价能力、来自潜在进入者的威胁、来自替代品的威胁和潜在竞争者的威胁
—
共同组合而演变出影响公司的第五种力量:来自现有竞争者的威胁。而每一种力量都由数项指标决定:
\begin{enumerate}
\sphinxsetlistlabels{\arabic}{enumi}{enumii}{}{.}%
\item {} 
来自买方的议价能力(Bargaining power of customers)

\item {} 
来自供应商的议价能力(Bargaining power of suppliers)

\item {} 
来自潜在进入者的威胁(Threat of new entrants)

\item {} 
来自替代品的威胁(Threat of substitutes)

\item {} 
来自现有竞争者的威胁(Competitive rivalry)

\end{enumerate}


\subparagraph{来自买方的议价能力(Bargaining power of customers)}
\label{\detokenize{chapter_skill/BRD:bargaining-power-of-customers}}\begin{itemize}
\item {} 
买方集中度(buyer concentration to firm concentration ratio)

\item {} 
谈判杠杆(bargaining leverage)

\item {} 
买方购买数量(total buyer volume)

\item {} 
买方相对于厂商的转换成本(buyer switching costs relative to firm
switching costs)

\item {} 
买方获取资讯的能力(buyer information availability)

\item {} 
买方垂直整合(bargaining leverage,backward vertical
integration)的程度或可能性

\item {} 
现存替代品(availability of existing substitute products or
services)

\item {} 
消费者价格敏感度(buyer price sensitivity)

\item {} 
总消费金额(price of total purchase)

\end{itemize}


\subparagraph{来自供应商的议价能力(Bargaining power of suppliers)}
\label{\detokenize{chapter_skill/BRD:bargaining-power-of-suppliers}}\begin{itemize}
\item {} 
供应商相对于厂商的转换成本 (switching costs of firms in the
industry)

\item {} 
投入原料的差异化程度

\item {} 
现存的替代原料(presence of substitute inputs)

\item {} 
供应商集中度 (supplier concentration)

\item {} 
供应商垂直整合(bargaining leverage,forward vertical
integration)的程度或可能性

\item {} 
原料价格占产品售价的比例

\end{itemize}


\subparagraph{来自潜在进入者的威胁(Threat of new entrants)}
\label{\detokenize{chapter_skill/BRD:threat-of-new-entrants}}\begin{itemize}
\item {} 
消费者对替代品的偏好倾向

\item {} 
替代品相对的价格效用比

\item {} 
消费者的转换成本

\item {} 
消费者认知的品牌差异

\end{itemize}


\subparagraph{来自现有竞争者的威胁(Competitive rivalry)}
\label{\detokenize{chapter_skill/BRD:competitive-rivalry}}\begin{itemize}
\item {} 
现有竞争者的数目

\item {} 
产业成长率(industry growth)

\item {} 
产业存在超额产能的情况

\item {} 
退出障碍 (exit barrier)

\item {} 
竞争者的多样性 (diversity of rivals)

\item {} 
资讯的复杂度和不对称

\item {} 
品牌权益 (brand equity)

\item {} 
每单位附加价值摊提到的固定资产

\item {} 
大量的广告需求

\item {} 
不同的产品 (product differences)

\end{itemize}


\subsubsection{用户需求研究 1\sphinxfootnotemark[151]}
\label{\detokenize{chapter_skill/users_analysis:id1}}\label{\detokenize{chapter_skill/users_analysis::doc}}%
\begin{footnotetext}[151]\sphinxAtStartFootnote
\sphinxnolinkurl{http://www.woshipm.com/operate/3627874.html}
%
\end{footnotetext}\ignorespaces 

\paragraph{什么是用户?}
\label{\detokenize{chapter_skill/users_analysis:id2}}
用户不是人,而是多个需求的集合。某个产品完全满足了某个人在某个场景下的某类需求,那么就可以说该场景下的这个人就是产品的一个用户


\paragraph{用户的五个属性 11\sphinxfootnotemark[152]}
\label{\detokenize{chapter_skill/users_analysis:id3}}%
\begin{footnotetext}[152]\sphinxAtStartFootnote
\sphinxnolinkurl{https://www.jianshu.com/p/02df7160b7b0}
%
\end{footnotetext}\ignorespaces \begin{itemize}
\item {} 
异质性:每一个用户的偏好、认知、拥有的资源是不一样的,

\item {} 
情境性:没有情境就没有用户,同一用户在不同的情境下会有不容的反应和行为

\item {} 
可塑性:用户的偏好和认知会随着外界不同的信息刺激发生变化和演化

\item {} 
自利性:追求个人总效用最大化

\item {} 
有限理性:虽然追求理性,但是能力有限、判断经常出错,所以只能做到有限的程度

\end{itemize}


\paragraph{UCD VS BCD 4\sphinxfootnotemark[153]}
\label{\detokenize{chapter_skill/users_analysis:ucd-vs-bcd-4}}%
\begin{footnotetext}[153]\sphinxAtStartFootnote
\sphinxnolinkurl{https://www.bilibili.com/video/BV1wz4y1y7sg?p=2}
%
\end{footnotetext}\ignorespaces \begin{itemize}
\item {} 
UCD(User Centered Design): 以用户为中心的产品设计

\item {} 
BCD(Boss Centered Design): 以老板为中心的产品设计

\end{itemize}

谁最了解用户,谁最有发言权!


\paragraph{为什么了解用户}
\label{\detokenize{chapter_skill/users_analysis:id4}}
涉及到方向性问题的时候,在混沌的信息中,在开放的无边界的信息中,找到适合的方向,这本身不是A/B测试能搞定的事情

基于对用户的深刻洞察,才能谈价值发现,产品规划,产品设计,上线运营等。


\paragraph{同理心地图}
\label{\detokenize{chapter_skill/users_analysis:id5}}
同理心地图给设计团队提供了一个思考框架,是帮助团队整理对用户的认识的一项工具。它帮助团队整合所观察和调研到的人和事物,并协助挖提出对用户深层次的理解目标、需求、观点、痛处、态度、行为)同时,同理心地图也是一个团队协同设计的工具,确保每位团队成员对使用者的理解都是相同的。

\begin{figure}[H]
\centering
\capstart

\noindent\sphinxincludegraphics{{empathy_map}.png}
\caption{empathy\_map}\label{\detokenize{chapter_skill/users_analysis:id26}}\end{figure}


\paragraph{用户故事 6\sphinxfootnotemark[154]}
\label{\detokenize{chapter_skill/users_analysis:id6}}%
\begin{footnotetext}[154]\sphinxAtStartFootnote
\sphinxnolinkurl{https://www.bilibili.com/video/BV1254y1D7Ht?from=search\&seid=14167562900175777805}
%
\end{footnotetext}\ignorespaces 
用户故事的用途是以用户的视角描述其通过使用软件产品想要实现的任务和获得的价值。故事不同于传统需求规格说明书,以简化的形式促进团队交流,降低修改成本、灵活调整接受变化,同时故事以验收驱动的定义形式让所有干系人入对最终的目标建立共识。

以用户的语言来描述用户故事,以识别真正的用户故事而不是解决方案

即:用户故事——谁,在什么情况下,碰到了什么问题,有什么感受和情绪,现在又是怎么做的,现在的做法中又有哪些痛点,等等。
\sphinxhref{https://www.jianshu.com/p/60e79d46dde5}{9}%
\begin{footnote}[155]\sphinxAtStartFootnote
\sphinxnolinkurl{https://www.jianshu.com/p/60e79d46dde5}
%
\end{footnote}

用户故事可分为三个层次:
\begin{enumerate}
\sphinxsetlistlabels{\arabic}{enumi}{enumii}{}{.}%
\item {} 
“主题”用户故事

\item {} 
“大”用户故事

\item {} 
“可开发”的用户故事

\end{enumerate}


\subparagraph{用户故事的INVEST原则}
\label{\detokenize{chapter_skill/users_analysis:invest}}\begin{enumerate}
\sphinxsetlistlabels{\arabic}{enumi}{enumii}{}{.}%
\item {} 
独立性(Independent)— 要尽可能的让一个用户故事独立于其他的用户故事。

\item {} 
可协商性(Negotiable)—
一个用户故事的内容要是可以协商的,用户故事不是合同。

\item {} 
有价值(Valuable)—
每个故事必须对客户具有价值(无论是用户还是购买方)。

\item {} 
可以估算性(Estimable)—开发团队需要去估计一个用户故事以便确定优先级,工作量,安排计划。

\item {} 
短小(Small)—
一个好的故事在工作量上要尽量短小,最好不要超过10个理想人/天的工作量,至少要确保的是在一个迭代或Sprint中能够完成。

\item {} 
可测试性(Testable)—一个用户故事要是可以测试的,以便于确认它是可以完成的。

\end{enumerate}


\paragraph{用户生态}
\label{\detokenize{chapter_skill/users_analysis:id7}}
在任何一个产品领域,用户都是多种多样的。所以,第二步我们要梳理用户生态。你需要了解,在产品所涉及的领域中,有哪几种用户,他们之间的关系是什么。

还要注意三点:
\begin{enumerate}
\sphinxsetlistlabels{\arabic}{enumi}{enumii}{}{.}%
\item {} 
颗粒度:某种用户可以继续细分,但分到什么程度,还没有定论,需要根据实际情况进行分析。比如家长要不要细分为爸爸和妈妈。

\item {} 
考虑边界:不同的用户和产品发生的关系有强有弱,最广义的用户,是指所有和产品有关系的人。那么,用户是否都要纳入我们日常的用户生态图,就是你需要考虑的“边界”。

\item {} 
优先级:已经被画在用户生态图中的用户,也是有重要、有次要,我们肯定是先照顾最重要的。

\end{enumerate}


\paragraph{用户画像}
\label{\detokenize{chapter_skill/users_analysis:id8}}
用户画像,我们要用一些关键特征来描述一个重要的用户群体。它可以帮助整个产品创新团队时刻牢记我们的产品是为谁服务的。

用户画像包含:
\begin{enumerate}
\sphinxsetlistlabels{\arabic}{enumi}{enumii}{}{.}%
\item {} 
基本信息,给这类用户的代表起个看起来真实的名字,选一个照片,设定性别、年龄、职业、日常的兴趣爱好。

\item {} 
描述用户的特定信息,也就是与产品领域相关的信息,比如生活方式、价值取向、心理预期等。

\item {} 
选几句在收集用户故事的时候,听到的用户说的有代表性的话,增强真实感。

\end{enumerate}

\begin{figure}[H]
\centering
\capstart

\noindent\sphinxincludegraphics{{QQ_users}.jpg}
\caption{QQ早期用户画像数据}\label{\detokenize{chapter_skill/users_analysis:id27}}\end{figure}


\subparagraph{人的五个层次 10\sphinxfootnotemark[156]}
\label{\detokenize{chapter_skill/users_analysis:id9}}%
\begin{footnotetext}[156]\sphinxAtStartFootnote
\sphinxnolinkurl{https://www.jianshu.com/p/85ec807c56d3}
%
\end{footnotetext}\ignorespaces \begin{enumerate}
\sphinxsetlistlabels{\arabic}{enumi}{enumii}{}{.}%
\item {} 
感知层

\item {} 
角色框架层

\item {} 
资源结构层

\item {} 
人的能力圈

\item {} 
一个人对存在感的定义(这是一个人的内核,就是他对他自己为什么而存在,到底是怎么感知的。)

\end{enumerate}

存在感对于人就像生存对于动物一样,是触发情绪和推动行动的开关。最内核是存在感,它的外面一层是能力圈。如果一个人的存在感满足了,其实他的能力圈就不会再扩充了。

如果你明确知道自己想成为一个什么样的存在,你就会不断地改变自己的能力圈,改变自己的资源,甚至改变自己的样子。


\paragraph{用户旅程}
\label{\detokenize{chapter_skill/users_analysis:id10}}
如果说用户画像是静态的,那我们最后做的用户旅程,就是让用户“动起来”。

选一个重要的用户,思考他在解决相应问题的时候,都会碰到什么状况,做什么事,有什么感受和情绪。这时候,“有没有产品”依然不是重点,重点还是关注用户的言行举止。

用户旅程分为三段:
\begin{enumerate}
\sphinxsetlistlabels{\arabic}{enumi}{enumii}{}{.}%
\item {} 
做某事前的准备;

\item {} 
做某事的过程;

\item {} 
做完某事之后。

\end{enumerate}


\subparagraph{用户旅程地图}
\label{\detokenize{chapter_skill/users_analysis:id11}}
用户旅程地图(User Journey Map是和用户画像
persona)相辅相成的工具,用户画像代表的是具体的族群,而体验地图是分析这个族群为了实现某个目标而经历的过程的可视化呈现工具。它用于了解和解决客户需求和痛点,在这个过程中用户可能会使用多个设备和渠道(例如网站,手机app,社交媒体,电话,线下客等)
\begin{enumerate}
\sphinxsetlistlabels{\arabic}{enumi}{enumii}{}{.}%
\item {} 
阶段:用户实现某个目标所经历的具体步骤

\item {} 
行动:每一个步骤下用户所产生的具体行为习惯

\item {} 
想法:用户在这个过程中的想法和体会

\item {} 
情感曲线:用户在这个过程中不同阶段的情感波动

\item {} 
机会点:我们洞察到的能够改进的机会

\item {} 
改进点:将每个改进点对应的相应的责任人身上

\end{enumerate}


\paragraph{Persona 文档指导}
\label{\detokenize{chapter_skill/users_analysis:persona}}
\begin{figure}[H]
\centering
\capstart

\noindent\sphinxincludegraphics{{persona}.png}
\caption{Persona 文档}\label{\detokenize{chapter_skill/users_analysis:id28}}\end{figure}


\paragraph{价值主张画布}
\label{\detokenize{chapter_skill/users_analysis:id12}}
\begin{figure}[H]
\centering
\capstart

\noindent\sphinxincludegraphics{{value_map}.png}
\caption{价值主张画布}\label{\detokenize{chapter_skill/users_analysis:id29}}\end{figure}


\paragraph{故事板 7\sphinxfootnotemark[157]}
\label{\detokenize{chapter_skill/users_analysis:id13}}%
\begin{footnotetext}[157]\sphinxAtStartFootnote
\sphinxnolinkurl{http://acadeck.com/?p=411}
%
\end{footnotetext}\ignorespaces 
故事板可以帮助用户预测并探索产品的用户体验,透过故事板的情境模拟以利设计师在设计过程中能去推测出使用者在使用过程中可能会遇上的问题,且帮助了解用户目前与问题相关的动机和经验,便于设计师能更进一步确立设计目标


\paragraph{同期群分析(Cohort Analysis)}
\label{\detokenize{chapter_skill/users_analysis:cohort-analysis}}
主要目的是分析相似群体随时间的变化(比如用户的回访)随看开发迭代的演进,产品上线第一个月使用你的产品的用户与第五个月使用你产品的用户感受到的体验是很不一样的。
我们把在同一个时间段(产品阶段)使用产品的用户划为同一期,针对他们的分析叫做同期群分析


\paragraph{需要了解到什么度}
\label{\detokenize{chapter_skill/users_analysis:id14}}
至少在你们公司,你应该是你们公司用户的专家,即其他人想要了解用户对某些场景或问题的看法时,如果想到咨询一个人的话,第一个想到的是你,那么你就是你们公司的用户专家。可以不断的问自己一个问题“自己是否可以称为用户专家,是否足够的洞察用户”,这需要时间的积累,在实践中回答这个问题,并不断的通过实践给出一个肯定的答案。


\paragraph{怎么衡量了解的度}
\label{\detokenize{chapter_skill/users_analysis:id15}}
最简单直接的方法是假设验证法,即给定一个场景,给出你对用户的判断,然后以实际结果验证你的判断。不断的实践来提高对用户判断的准确度。

当给出任何场景,你对用户的判断八九不离十,知道用户是否存在这个问题?多少用户存在这个问题?用户当前是怎么解决这个问题的?是否值得做?做了之后用户是否能从原来的习惯中迁移过来?


\paragraph{指标 4\sphinxfootnotemark[158]}
\label{\detokenize{chapter_skill/users_analysis:id16}}%
\begin{footnotetext}[158]\sphinxAtStartFootnote
\sphinxnolinkurl{https://www.bilibili.com/video/BV1wz4y1y7sg?p=2}
%
\end{footnotetext}\ignorespaces 
需求量、强度、频次、痛点、Arpu、期望、现有解决方案
\begin{itemize}
\item {} 
需求量:大众、小众

\item {} 
强度:刚需、弱需 越刚越容易付费

\item {} 
频次:高、低频

\end{itemize}

\begin{figure}[H]
\centering
\capstart

\noindent\sphinxincludegraphics{{need_analysis}.png}
\caption{need\_analysis}\label{\detokenize{chapter_skill/users_analysis:id30}}\end{figure}
\begin{itemize}
\item {} 
痛点:解决某个需求时很难受的地方

\item {} 
Arpu:用户价值。烧饼一两块、化妆品百来块、增高药上万

\item {} 
期望:超预期。才能拉新。

\end{itemize}

\begin{figure}[H]
\centering
\capstart

\noindent\sphinxincludegraphics{{lawyer_analysis}.png}
\caption{lawyer\_analysis}\label{\detokenize{chapter_skill/users_analysis:id31}}\end{figure}

为了需求找技术。


\paragraph{研究内容 4\sphinxfootnotemark[159]}
\label{\detokenize{chapter_skill/users_analysis:id17}}%
\begin{footnotetext}[159]\sphinxAtStartFootnote
\sphinxnolinkurl{https://www.bilibili.com/video/BV1wz4y1y7sg?p=2}
%
\end{footnotetext}\ignorespaces \begin{itemize}
\item {} 
用户特征:性别、年龄、职业、地域、学历、消费能力。TOFA(传统/时尚、节俭/花钱)

\item {} 
需求情景:在什么时候用,用的时候会发生什么?饿了么,来不及停止接单、在意配送时间准时保。

\item {} 
需求动机:聊天、结婚、约炮?微信熟人、陌陌陌生人不需加好友。

\item {} 
显性/隐性需求:隐性又是更重要

\item {} 
关注因素:在意什么?菜品口味、价格、送餐速度、干净卫生。大众用综合排序

\item {} 
认知过程:不知道》知道》了解》产生兴趣》学习

\item {} 
行为习惯:用户通常怎么做?由于认知决定。SICAS(Sense、Interest、Communication、Action、Share)
FOGG(motivation、ability、trigger)

\item {} 
行为心理:为何这么做?货比三家、怕吃亏上当。

\item {} 
使用过程:用户使用你产品或服务的过程。

\item {} 
决定因素:重大行为的决策。陌陌上找你喝酒,怕是酒托、仙人跳..

\end{itemize}


\subparagraph{需求情景–情节}
\label{\detokenize{chapter_skill/users_analysis:id18}}
主线:筛选饭店、点餐、支付、等餐、就餐。

分支:
\begin{enumerate}
\sphinxsetlistlabels{\arabic}{enumi}{enumii}{}{.}%
\item {} 
筛选饭店的方式:搜索、好评、默认推荐

\item {} 
查看送餐小哥什么时候能送到?要不要催促下?

\item {} 
饭到了很难吃,要不要给个差评?

\end{enumerate}

异常:退餐流程,这个流程中,又可以细分出N个情景

用户体验、满意度、需求满足程度
\begin{enumerate}
\sphinxsetlistlabels{\arabic}{enumi}{enumii}{}{.}%
\item {} 
产品不同,研究的内容和方法也会有差异,需要活学活用。

\item {} 
这条线上的每一个点,都会关联到你的业务流程设计,产品功能设计,运营策略,付费转化策略,营销推广的策略和\sphinxstylestrong{N个细节}。

\item {} 
产品的每一个细节,都跟用户需求有着千丝万缕的联系。大到产品定位规划,商业模式,竞争策略,小到每一句营销的文案编写,UI设计图那个字需要加大加粗,某个位置需要一个小图标。

\end{enumerate}


\paragraph{洞察用户时常犯错误}
\label{\detokenize{chapter_skill/users_analysis:id19}}
1)以偏概全,因自己或周边人经常遇到某些场景,就以为绝大多数人会遇到类似场景,很感性的认知。举例:你朋友圈的热点可能真的只是你朋友圈的热点,在你父母那,在你高中同学那,在别的行业的大学同学那,甚至同行业同事那里,大家的热点都是有差异的。

2)常识性错误,比如我们知道一般老板会查看下属工作情况,老板也更关心公司的业绩统计数据,然后我们就可能认为下属资料和业绩统计分析会有较高的用户重合度,其实不一定,因为查看下属资料这个大概率是管理员做,但统计分析这种业务员也可以查看,甚至是老板指派专人管理。

3)过于相信数据,比如AI技术可以实现一些功能的自动化,我们通过自动化的开关来判定用户有没有使用,也通过用户对自动化数据的修改来判断用户是否真的将自动化使用起来。但数据表现都很好,不代表用户满意,用户可能只是不知道你给他自动做来那么多事情,甚至知道了,也觉得数据是错的,但选择忽略而已,需要更多的从用户真实的反馈中得到。

4)静态的看待用户的行为,无论我们做用户访谈,还是用户调研,得出的数据和内容是基于当时用户状态及对产品的了解,而用户在产品或服务使用的过程中,是会随时间的变化而变化的。比如对于C端用户勋章挑战类的功能,刚开始可能用户比较喜欢,参与度较高,但随着参与次数的提升,部分用户会有疲劳感,这在产品的设计中,就要考虑随时间周期变化的用户的反馈。


\paragraph{如何将用户需求转换为产品需求?}
\label{\detokenize{chapter_skill/users_analysis:id20}}
首先保持二八原则,只有普遍用户的需求,才能内化为产品的需求。比如某个需求就一个用户需要,其他大多数用户都不需要,你就不需要做。

通过现象看本质,收集用户需求以后,多为自己几个为什么,找到用户的动机。

例如:用户在沙漠中需要水,你就要问自己用户为什么需要水?用户有可能口渴了,那这时候你给他水就好,如果用户是因为太热,你能不能给他防晒服,甚至考虑一下用户体验,觉得防晒服太麻烦,提供防晒霜。有时候一个人并不能完全洞察用户的动机,需要团队的其他人员一起头脑风暴,甚至多问提这个需求的原始用户几个为什么,直到找到真正动机为止,然后结合产品本身衡量需求的性价比,最后综合团队实力,需求急切度确定最终产品需求。


\subparagraph{需求提取 3\sphinxfootnotemark[160]}
\label{\detokenize{chapter_skill/users_analysis:id21}}%
\begin{footnotetext}[160]\sphinxAtStartFootnote
\sphinxnolinkurl{https://blog.csdn.net/eickandy/article/details/80294224}
%
\end{footnotetext}\ignorespaces 
“如果我最初问消费者他们想要什么,他们应该是会告诉我,‘要一匹更快的马!’”

——这是亨利·福特的一句经典名言,如今我们在《乔布斯传》里又见到了它。

客户需求有显性需求和隐性需求两大类。我们通过市场调查得知的往往都是一些诸如“我要一匹更快的马”这类显性需求。客户的显性需求并不是客户真正的需求。企业需要根据所收集的显性需求信息进行深度挖掘和捕获,以了解客户的隐性需求是什么,进而分析出客户的真正需求是什么(例如:用更短的时间、更快地到达目的地)。这就是一个需求分析的过程。


\subparagraph{Y 模型 12\sphinxfootnotemark[161]}
\label{\detokenize{chapter_skill/users_analysis:y-12}}%
\begin{footnotetext}[161]\sphinxAtStartFootnote
\sphinxnolinkurl{https://www.jianshu.com/p/2af332aaa017}
%
\end{footnotetext}\ignorespaces 
\begin{figure}[H]
\centering
\capstart

\noindent\sphinxincludegraphics{{Y_model}.png}
\caption{Y 模型}\label{\detokenize{chapter_skill/users_analysis:id32}}\end{figure}

配图中你可以看到一个大写的字母 Y,有三个线段、四个节点。
\begin{itemize}
\item {} 
“节点
1”代表的是用户需求场景,经常被简称为用户需求。这是起点,是表象,是表面的需求,是用户的观点和行为。

\item {} 
“节点
2”是用户需求背后的目标和动机,是用户言行的原因。不过产品经理在思考用户目标时也要综合考虑公司、产品的目标。

\item {} 
“节点 3”是产品功能,是解决方案,是技术人员能看懂的描述。

\item {} 
“节点
4”是人性与价值观,或者说是用户心智,是需求的最深层体现,是需求的本质。

\end{itemize}

Y 模型的不同阶段,各自需要回答一些问题,可以总结为 6 个 W 和 3 个 H。

“节点 1”这个阶段的问题主要是
Who(目标用户是谁)、What(需求表现为什么)和
Where/When(何时何地,什么情况下)。

“节点 1”到“节点 2”和“节点 2”到“节点
4”这个阶段,对应的是对用户需求的层层深入。这个阶段要回答 Why
这个问题——要不停地往下深入挖掘需求,了解用户为什么会有这样的言行、为什么会有这样的目标和动机。

“节点 4”到“节点 2”再到“节点 3”的过程中,你要想清楚
How——也就是要想清楚问题应该怎么解决。这个叫浅出,先深入后浅出。

“节点 3”中,要回答 Which、How many、How much 三个问题。

Which
是指选哪一个方案,做哪一个功能,这背后其实是对价值的判断,比如怎么评估性价比和优先级。How
many 是指这一次做多少个功能,考验的是对迭代周期,产品包大小的把控。How
much
原意是多少钱,这里引申为多少资源,是对时间、金钱、团队等资源的评估。
\begin{quote}

在一个不成熟的领域或全新的市场,只做“节点
1、2、3”是玩得转的。但表层需求很快就会被相似的跟进产品满足,随着市场的成熟,产品很快会陷入同质化竞争和价格战,最终整个市场变成红海。而破局的方法就在“节点
4”(用户心智)。
\end{quote}

作为初创团队,你要做的重要事情只有两件,一是和用户交流,二是开发产品。
———保罗·格雷厄姆

创新会面临两大风险,一是市场风险,就是说你做的产品有没有人需要,二是技术风险,就是说你能不能把东西做出来。

和用户交流,就是 Y
模型的前半段,是解决市场风险,对应的心法就是用心听;而开发产品,就是 Y
模型的后半段,是解决技术风险,对应的心法是不要照着做。


\paragraph{研究方法 5\sphinxfootnotemark[162]}
\label{\detokenize{chapter_skill/users_analysis:id22}}%
\begin{footnotetext}[162]\sphinxAtStartFootnote
\sphinxnolinkurl{https://www.bilibili.com/video/BV1wz4y1y7sg?p=3}
%
\end{footnotetext}\ignorespaces \begin{itemize}
\item {} 
用户访谈

\item {} 
问卷调查:问卷设计一般都需要产品经理完成,然后可以找专业调研公司去实施。

\item {} 
体验与观察

\item {} 
焦点小组

\item {} 
成为用户

\item {} 
参与式设计

\item {} 
卡片分类

\item {} 
亲和图法

\item {} 
可用性测试

\item {} 
数据分析

\item {} 
满意度调查

\end{itemize}


\paragraph{结果产出 5\sphinxfootnotemark[163]}
\label{\detokenize{chapter_skill/users_analysis:id23}}%
\begin{footnotetext}[163]\sphinxAtStartFootnote
\sphinxnolinkurl{https://www.bilibili.com/video/BV1wz4y1y7sg?p=3}
%
\end{footnotetext}\ignorespaces \begin{itemize}
\item {} 
用户画像

\item {} 
用户应用情景

\item {} 
主线情景

\item {} 
分支情景

\item {} 
异常情景

\item {} 
认知过程

\item {} 
关注因素

\item {} 
行为心理

\item {} 
决策心理

\item {} 
任务流程分析

\item {} 
逻辑与权重

\end{itemize}


\paragraph{用户访谈}
\label{\detokenize{chapter_skill/users_analysis:id24}}
在求职和工作中会有什么作用呢?
\begin{enumerate}
\sphinxsetlistlabels{\arabic}{enumi}{enumii}{}{.}%
\item {} 
在求职中,用户访谈是说服面试官的武器。

\end{enumerate}

在面试中,面试官可能会质疑你需求的合理性。例如,面试官可能会说:“我觉得你这个约别人骑行的需求是个伪需求。”如果你没有做用户访谈,那么你只能说:“我觉着这个需求肯定是存在的,因为我就有这个需求。”这样是不是显得底气不足?但是,如果你做过用户访谈,你就可以自信地说:“这个需求确实存在,因为我访谈了20
个目标用户,其中85\%的用户提到自己有这个需求,主要场景有两个,一个是当自己不会修车时可以找人帮忙,另一个是他们觉得一个人骑行很无聊。”这样的回答是不是更有说服力?
\begin{enumerate}
\sphinxsetlistlabels{\arabic}{enumi}{enumii}{}{.}%
\setcounter{enumi}{1}
\item {} 
在工作中,用户访谈是验证需求合理性的方法之一。

\end{enumerate}

作为产品经理,我们要“发现需求”,而不是“创造需求”。这就要求我们通过科学、严谨的方法去挖掘需求,而不是用拍脑袋的方式决定有什么需求。比较严谨的方式有两种,一种是数据分析,另一种是用户访谈。因此,在掌握了用户访谈的方法后,我们就可以在以后的工作中设计出更符合用户预期的产品。


\paragraph{消费心理 8\sphinxfootnotemark[164]}
\label{\detokenize{chapter_skill/users_analysis:id25}}%
\begin{footnotetext}[164]\sphinxAtStartFootnote
\sphinxnolinkurl{https://wiki.mbalib.com/wiki/\%E6\%B6\%88\%E8\%B4\%B9\%E8\%80\%85\%E5\%BF\%83\%E7\%90\%86}
%
\end{footnotetext}\ignorespaces \begin{itemize}
\item {} 
比附大腕以成就品牌:蒙牛一开始绑定伊利、借助内蒙古的优秀品牌。

\item {} 
通过情感联系来打造品牌:“钻石恒久远,一颗永留传”–“钻石有价,爱情无价”;贝尔电话–“女儿说她爱我们。”

\item {} 
掌握消费者的需求和心理:静的让你日日夜夜都感觉不到–伊莱克斯冰箱;“谁杀了兔子乔丹”–我穿耐克鞋,我是英雄

\item {} 
通过事件营销推广品牌:“水仙花”实验,向观众展示了水仙花在农夫山泉天然水和纯净水中的生长状况;刘翔与可口可乐,销量一度上升了30%。

\item {} 
珍惜消费群体:“飘柔”产品曾领先于洗发行业,为了低档市场的销售量,研发了“飘柔”系列的淋浴液及香皂,降低了品牌的市场价值。

\end{itemize}


\subsubsection{优先级 1\sphinxfootnotemark[165]}
\label{\detokenize{chapter_skill/priority:id1}}\label{\detokenize{chapter_skill/priority::doc}}%
\begin{footnotetext}[165]\sphinxAtStartFootnote
\sphinxnolinkurl{https://www.bilibili.com/video/BV1254y1D7Ht?from=search\&seid=14167562900175777805}
%
\end{footnotetext}\ignorespaces 

\paragraph{why? 2\sphinxfootnotemark[166]}
\label{\detokenize{chapter_skill/priority:why-2}}%
\begin{footnotetext}[166]\sphinxAtStartFootnote
\sphinxnolinkurl{https://zhuanlan.zhihu.com/p/22067195}
%
\end{footnotetext}\ignorespaces 
人类拥有无穷的欲望,却只拥有有限的资源。熊掌与鱼,不可兼得!


\paragraph{对「优先级」的定义关系产品成败}
\label{\detokenize{chapter_skill/priority:id2}}
苹果早期 iPhone
的设计是优先级控制的典范。手机正面只有一个按键的设计,尽管现在看来似乎是理所当然的,但在当时却是非常勇敢和有争议的决定。我曾深入研究过
iOS
系统早期的设计,在很多地方的取舍做的非常到位,能够大胆砍掉之前手机系统常见的功能和界面元素,让重点变得更重点,让需要突出的内容变得更突出。

而相比之下,同一时期诺基亚的系统尽管拥有大量功能,在呈现给用户时并没有处理好优先级,对于用户相对要更复杂,如果读者还有印象,可以想象一下打开一个联系人,看看与之对应的常常的功能菜单。而那个时候微软的
Windows Mobile 系统,则是大量将 PC
上的体验搬到手机上,用户的认知资源和系统有限的显示和交互资源之间并不匹配。


\paragraph{结构化优先级}
\label{\detokenize{chapter_skill/priority:id3}}\begin{enumerate}
\sphinxsetlistlabels{\arabic}{enumi}{enumii}{}{.}%
\item {} 
信息优先级要关注内容的组织关系和轻重缓急

\item {} 
视觉优先级重在引导用户视线和行为轨迹

\item {} 
交互优先级要区分主线和支线任务

\item {} 
需求优先级要平衡用户目标和商业目标

\item {} 
用户优先级要界定核心用户是谁

\end{enumerate}


\subparagraph{信息优先级}
\label{\detokenize{chapter_skill/priority:id4}}
余额宝:收益、总金额。字号大,想让你看!麦肯兹金字塔。


\subparagraph{视觉优先级}
\label{\detokenize{chapter_skill/priority:id5}}\begin{itemize}
\item {} 
报纸上的文字大小、颜色、区块

\item {} 
海报朝着商品上看。

\end{itemize}


\subparagraph{交互优先级}
\label{\detokenize{chapter_skill/priority:id6}}
区分主线和支线,\sphinxstylestrong{突出主线任务}
\begin{itemize}
\item {} 
读书app,点一下会设置,再点一下只能前后。读书!

\item {} 
支付宝,收付钱。

\item {} 
滴滴,预约用车、现在用车。感觉车很多。

\end{itemize}

建立故事,记忆更深刻。

豌豆荚几亿钱分十几个人!


\subparagraph{需求优先级}
\label{\detokenize{chapter_skill/priority:id7}}
需求优先级要平衡用户目标和商业目标。
\begin{itemize}
\item {} 
用户目标:炫耀。读取型号、找对应图片、程序合成

\item {} 
商业目标:更多用户。使用豌豆荚截图+\sphinxstylestrong{网址}。留个空间给裁。

\end{itemize}


\paragraph{项目范围优先级制定一沟通计划}
\label{\detokenize{chapter_skill/priority:id8}}
By who、Who、How、Why、When、What。

制定沟通计划的目的是为项目交付周期的交流和相互支持提供指导。在敏捷项目里,面对面交流比文档要好,但是依然会有一些共享文件,比如报告和项目计划,需要留下档案。


\paragraph{四象限 3\sphinxfootnotemark[167]}
\label{\detokenize{chapter_skill/priority:id9}}%
\begin{footnotetext}[167]\sphinxAtStartFootnote
\sphinxnolinkurl{https://www.bilibili.com/video/BV1254y1D7Ht?from=search\&seid=14167562900175777805}
%
\end{footnotetext}\ignorespaces 
优先级顺序:重要、紧急(立即做)>重要、不紧急(时间表)>不重要、紧急(委派)>不重要、不紧急(排除)

重要程度大致的排序如下:
\sphinxhref{https://weread.qq.com/web/reader/40632860719ad5bb4060856ke3632bd0222e369853df322}{4}%
\begin{footnote}[168]\sphinxAtStartFootnote
\sphinxnolinkurl{https://weread.qq.com/web/reader/40632860719ad5bb4060856ke3632bd0222e369853df322}
%
\end{footnote}
\begin{itemize}
\item {} 
不做会造成严重问题和恶劣影响的;

\item {} 
做了会产生巨大好处和极佳效果的;

\item {} 
同重要合作对象或投资人有关的;

\item {} 
同核心用户利益有关的;

\item {} 
同大部分用户权益有关的;

\item {} 
同效率或成本有关的;

\item {} 
同用户体验有关的。

\end{itemize}

紧急程度大致的排序如下:
\begin{itemize}
\item {} 
不做错误会持续发生,然后造成严重影响;

\item {} 
在一定时间内可控,但长期会有糟糕的影响;

\item {} 
做了立刻能解决很多问题、产生正面的影响;

\item {} 
做了在一段时间后可以有良好的效果。

\end{itemize}


\subsubsection{行业分析}
\label{\detokenize{chapter_skill/industry_analysis:id1}}\label{\detokenize{chapter_skill/industry_analysis::doc}}

\paragraph{目的}
\label{\detokenize{chapter_skill/industry_analysis:id2}}
产品经理需要随时了解市场和行业的变化,提前做好对行业变化的预判工作,找到机会点。这样,才能保证我们做出的产品能够更好的顺应市场。


\paragraph{分析出机会点}
\label{\detokenize{chapter_skill/industry_analysis:id3}}
产品小白如何快速做行业分析,找到机会点,最后能够做出分析结论:我们在哪些产品链条上的市场规模如何?存在哪些机会点?我们会面临上下游及竞品等哪些挑战?从中可以得到哪些商业价值?(需要日常练习,积累经验)

最后,针对我们的目标用户的典型需求,进行MVP。


\paragraph{套路}
\label{\detokenize{chapter_skill/industry_analysis:id4}}
\begin{figure}[H]
\centering
\capstart

\noindent\sphinxincludegraphics{{industry_module}.png}
\caption{industry\_module}\label{\detokenize{chapter_skill/industry_analysis:id13}}\end{figure}


\paragraph{行业分析的思路如下:}
\label{\detokenize{chapter_skill/industry_analysis:id5}}\begin{enumerate}
\sphinxsetlistlabels{\arabic}{enumi}{enumii}{}{.}%
\item {} 
了解市场规模(这个市场有多大?市场变化趋势如何?这个市场有哪些参与者?各参与者的行业影响力如何?细分市场趋势如何?指标:市场销售总量,年复合增长率,标志性现象看前瞻性)

\item {} 
产业地图分析(这个市场现状如何?行业如何运转?行业如何拆分?上下游关系如何?所在的竞争环境如何?细分市场的产品如何?指标:细分市场比例,增长率)

\item {} 
行业典型产品分析(针对我们要做的产品,来了解竞品怎么做?竞品核心业务逻辑如何?功能架构如何?运营动作如何?用户评价如何?优缺点如何?)

\item {} 
用户分析(针对我们要做的产品,来分析用户画像,从中挖掘需求,用户行为如何?机会点是什么?)

\item {} 
总结(分析结论:我们在哪些产品链条上的市场规模如何?存在哪些机会点?我们会面临上下游及竞品等哪些挑战?从中可以得到哪些商业价值?)

\end{enumerate}


\subparagraph{三四规则 5\sphinxfootnotemark[169]}
\label{\detokenize{chapter_skill/industry_analysis:id6}}%
\begin{footnotetext}[169]\sphinxAtStartFootnote
\sphinxnolinkurl{https://weread.qq.com/web/reader/40632860719ad5bb4060856k283328802332838023a7529}
%
\end{footnotetext}\ignorespaces 
三四规则可用于分析企业在一个成熟市场中的竞争地位,它将参与市场竞争的企业分为三类,分别是领先者、参与者和生存者。三四规则描述了这样一个市场规律:在有影响力的领先者之中,企业的数量绝对不会超过三个,而在这三个企业之中,最有实力的竞争者的市场份额又不会超过最小者的四倍。

一般来说,领先者是指市场占有率在15\%以上、可以对市场变化产生重大影响的企业,体现在价格、产量等方面;参与者一般是指市场占有率为5\%~15\%的企业,这些企业虽然不能对市场产生重大的影响,但是它们是市场竞争的有效参与者;生存者一般是局部细分市场的填补者,这些企业的市场占有率都非常低,通常小于5\%。

三四规则的成立也有两个假定条件。①
在任何两个竞争者之间保持2∶1的市场份额均衡点时,无论哪个竞争者要增加或减少市场份额,都显得不切实际而且得不偿失。②
当市场份额小于最大竞争者的1/4时,就不可能有效参与竞争。

我们通过“三四规则”可以了解一些市场规律,倘若两个竞争者拥有几乎相同的市场份额,在竞争时谁能提高相对市场份额,谁就能同时取得在产量和成本两个方面的增长,而在任何主要竞争者的激烈争夺情况下,最有可能受到伤害的却是市场中最弱的生存者。


\paragraph{市场调查 3\sphinxfootnotemark[170]}
\label{\detokenize{chapter_skill/industry_analysis:id7}}%
\begin{footnotetext}[170]\sphinxAtStartFootnote
\sphinxnolinkurl{https://baike.baidu.com/item/\%E5\%B8\%82\%E5\%9C\%BA\%E8\%B0\%83\%E6\%9F\%A5/170622\#:~:text=\%E5\%B8\%82\%E5\%9C\%BA\%E8\%B0\%83\%E6\%9F\%A5\%E6\%98\%AF\%E6\%8C\%87\%E7\%94\%A8,\%E6\%8F\%90\%E4\%BE\%9B\%E5\%AE\%A2\%E8\%A7\%82\%E3\%80\%81\%E6\%AD\%A3\%E7\%A1\%AE\%E7\%9A\%84\%E4\%BE\%9D\%E6\%8D\%AE\%E3\%80\%82}
%
\end{footnotetext}\ignorespaces 
市场调查是指用科学的方法,有目的、系统地搜集、记录、整理和分析市场情况,了解市场的现状及其发展趋势,为企业的决策者制定政策、进行市场预测、做出经营决策、制定计划提供客观、正确的依据。


\paragraph{关注以下几个重要指标}
\label{\detokenize{chapter_skill/industry_analysis:id8}}\begin{itemize}
\item {} 
TAM:即Total Available
Market,总有效市场或者市场规模,这是行业空间的天花板。然而,这是一个庞大的、基本没用的数字;

\item {} 
SAM:即Serviceable Available
Market,可服务市场,在基于公司内外部资源的客观条件下,所能服务到的市场范围。这个数字小了很多,基本有点用了;

\item {} 
SOM:即Serviceable Obtainable
Market,可获得市场,在能服务到的市场范围内,有能力拿下来的市场范围。这个数字进一步缩小,可以作为业务目标了;

\item {} 
Market Share:市场占有率,关注该产品在TAM中所占有份额;

\item {} 
Market Growth:市场成长性,关注整个行业TAM的增长或下降趋势;

\item {} 
Market Net
Value:公司实际收入,基于SOM所推断出来的公司实际收入(非流水,流水有可能只是过账户一道手,不一定是收入)。

\end{itemize}


\paragraph{行业数据分析}
\label{\detokenize{chapter_skill/industry_analysis:id9}}
和企业内部数据(特别是财务数据)相比,完全不是一个量级的准。

\begin{figure}[H]
\centering
\capstart

\noindent\sphinxincludegraphics{{industry_data}.png}
\caption{行业分析数据}\label{\detokenize{chapter_skill/industry_analysis:id14}}\end{figure}


\paragraph{渠道获取}
\label{\detokenize{chapter_skill/industry_analysis:id10}}\begin{enumerate}
\sphinxsetlistlabels{\arabic}{enumi}{enumii}{}{.}%
\item {} 
内部市场、运营部门、管理层等信息收集

\item {} 
艾瑞咨询、DCCI互联网数据中心、Alexa、Appstore

\item {} 
竞争对手网站、交流互动平台、产品历史更新版本、促销活动、最新调整、招聘信息等

\item {} 
竞争对手的季度/年度财报

\item {} 
行业媒体平台新闻、论坛、QQ群等

\item {} 
调查核心用户、活跃用户、普通用户不同需求弥补和代替的产品

\item {} 
使用对方的产品、客服咨询、技术问答等等

\item {} 
搜索国外同行业的官网及行业信息订阅(市场竞争可能不大,但盈利模式和功能定义用户群体具有一定前瞻性和市场趋势导向性)

\end{enumerate}


\paragraph{成为一个行业专家}
\label{\detokenize{chapter_skill/industry_analysis:id11}}
如何快速深入一个行业,笔者基于自身经验,罗列了如下6个维度:行业特点、行业运行趋势、商业模式、竞争力因素分析、行业整合、政府管制。以个人/家庭服务机器人为例。

\begin{figure}[H]
\centering
\capstart

\noindent\sphinxincludegraphics{{PM_industry}.jpg}
\caption{如何快速深入一个行业}\label{\detokenize{chapter_skill/industry_analysis:id15}}\end{figure}

\begin{figure}[H]
\centering
\capstart

\noindent\sphinxincludegraphics{{dive_industry}.jpg}
\caption{深入了解行业:点线面}\label{\detokenize{chapter_skill/industry_analysis:id16}}\end{figure}


\paragraph{框架}
\label{\detokenize{chapter_skill/industry_analysis:id12}}
腾讯5G生态计划负责人
余一列出了一个做行研的基本框架:(1)确定研究目标;(2)圈定已有资料的概览范围,上市公司财报及分析报告、咨询公司报告、数据机构资料、行业专业网站、政府网站、招聘网站、媒体网站等;(3)需要圈定时间和目标,不要迷失在资料中。(4)输出初步框架,行业现状(规模、结构、阶段)、行业趋势(发展推动要素、推动力分析)、竞争格局、其他。(5)业内访谈
,产业链、公司、专家、技术。(6)输出。

Envolve Group Co\sphinxhyphen{}founder
刘嘉培Alex详细拆解了查阅报告材料和思考的六个步骤,即要Top\sphinxhyphen{}down地思考一个行业:第一,先看整体市场规模,再看CAGR年复合增速,并思考:a)
容得下几家巨头公司? b) 增长的驱动因素是什么?
第二,了解最新资本市场活动:投资总额、IPO数量、兼并收购数量,思考:a)
行业受资本青睐吗?为什么?b)
大家是想靠估值倍数、分红、增长、并购重组挣钱?第三,利用MECE的方式把市场分割成多个不同的赛道:a)
关注不同赛道的行业规模、增速、市场活动、趋势、龙头、商业模式b)
考虑行业上下游之间的关系:整合还是分散?竞争还是合作?会一家独大还是百花齐放?c)
不同赛道里面最容易出现商业模式成熟、盈利模式清晰的公司的是哪个?
第四,关注最新行业动态、趋势和“催化剂”:a)
趋势是利好还是利空?对巨头有利还是对挑战者有利?b) 看行业垂直媒体c)
看公司研报。第五,研究行业巨头3\sphinxhyphen{}5家,新兴挑战者企业8\sphinxhyphen{}15家,做总结:a)
总结领先产品、品牌策略、用户positioning,b)
总结商业模式、盈利模式、经营模型、竞争策略,c)
二级市场估值倍数和市值变化规律,d) Where they started and how they got
here。第六,从创始人、投资人、客户/用户、投行咨询四个角度问:a)如果现在进入市场的话,会怎么做?b)如果投股票、收购公司、天使投资、债券的话,分别怎么投?为什么?在赌什么?c)
作为一个用户,最希望看到的是什么?为什么?

在实操过程中,Red Tripod captial Investment Director Vivian
Young特别强调了供需分析的重要性:一般的行业分析员大部分的时间就是在做供需分析,分析时要注意当前的供求结构关系,区分国内还是全球,存量还是增量。另外,需要重点关注:需求周期,产能周期,需求传导的逻辑,传导的节奏等。除了供需,还要研究行业未来发展趋势,其中,政策影响很关键。

除了具体的方法与步骤,阿尔法公社投资经理Gang
Liu还提醒大家,在做行研的时候不能求快,要以慢为快,在有限的时间段里,花更多的时间在研究上,方式方法重要,但执行同样重要。同时,要敬畏专业性,尽可能的找到这个领域的一线从业者或者专家,多跟他们交流。交叉验证,保持思辨性很重要。


\subsubsection{产品体验报告 1\sphinxfootnotemark[171]}
\label{\detokenize{chapter_skill/experience_report:id1}}\label{\detokenize{chapter_skill/experience_report::doc}}%
\begin{footnotetext}[171]\sphinxAtStartFootnote
\sphinxnolinkurl{https://www.jianshu.com/p/9fff898ce6bd}
%
\end{footnotetext}\ignorespaces 
1.需求分析 1.1 产品定义 1.2 用户需求 2.功能分析

3.竞品分析


\subsubsection{竞争分析}
\label{\detokenize{chapter_skill/compete_analysis:id1}}\label{\detokenize{chapter_skill/compete_analysis::doc}}

\paragraph{竞品思维误区}
\label{\detokenize{chapter_skill/compete_analysis:id2}}\begin{enumerate}
\sphinxsetlistlabels{\arabic}{enumi}{enumii}{}{.}%
\item {} 
你所以为的竞品,以为是领导,只是自己认为的竞品,别人根本没把你当竞品。

\item {} 
只把关注点集中在了那几个竞争对手上,而企业的竞争是多方面的,复杂的。

\item {} 
运用竞争研究资料时,自己反而会有意无意的模仿。

\end{enumerate}


\paragraph{竞争思维}
\label{\detokenize{chapter_skill/compete_analysis:id3}}
企业之间也是如此,当发生竞争的时候,彼此的利益就会发生转移,是在抢夺利益。

当用竞争思维去分析时,就不会去把关注点全集中在那几个头部品牌的表面分析上,会更深入的研究竞争者和竞争局面。这才是研究竞争的真正意义。

用竞争思维去研究的时候,自然就会看那些真正跟自己形成竞争的品牌,关注竞争对手同时,也看整个大局。

所以我说是做竞争研究分析,不做竞品研究分析。


\paragraph{怎么做}
\label{\detokenize{chapter_skill/compete_analysis:id4}}
6个要点
\begin{enumerate}
\sphinxsetlistlabels{\arabic}{enumi}{enumii}{}{.}%
\item {} 
行业的总体竞争态势;

\item {} 
画竞争地图;

\item {} 
行业领导者梳理;

\item {} 
识别核心竞争者(细分出竞争能力:资金、渠道、产品技术、品牌);

\item {} 
识别核心竞争者的传播策略;

\item {} 
识别异军突起的企业;

\end{enumerate}


\paragraph{美团 2\sphinxfootnotemark[172]}
\label{\detokenize{chapter_skill/compete_analysis:id5}}%
\begin{footnotetext}[172]\sphinxAtStartFootnote
\sphinxnolinkurl{https://www.bilibili.com/video/BV1wz4y1y7sg?p=5}
%
\end{footnotetext}\ignorespaces \begin{enumerate}
\sphinxsetlistlabels{\arabic}{enumi}{enumii}{}{.}%
\item {} 
将有限资源投放在二线城市市场,最后杀回一线

\item {} 
制空权:线上时o2o的入口,看出C端用户规模时关键

\item {} 
看重用户体验、留住用户。

\end{enumerate}


\paragraph{竞争战略}
\label{\detokenize{chapter_skill/compete_analysis:id6}}
行业内:领先者、跟随者、新加入者
\begin{itemize}
\item {} 
差异化:淘宝(C2C、性价比)、京东(B2C、配送快、购物体验好)、美丽说蘑菇街(女性电商导购到淘宝)、拼多多(拼一刀、货找人)

\item {} 
跟随:慧聪网跟随淘宝

\item {} 
成本优势:京东VS苏宁国美:偏远郊区来仓储VS闹市门店来摆卖

\item {} 
免费:360VS金山瑞星卡巴斯基,互联网思维:培养用户来赚钱VS软件思维:付软件。

\item {} 
好人/坏人:囚徒困境,好人+以牙还牙。

\end{itemize}


\paragraph{SWOT分析 3\sphinxfootnotemark[173]}
\label{\detokenize{chapter_skill/compete_analysis:swot-3}}%
\begin{footnotetext}[173]\sphinxAtStartFootnote
\sphinxnolinkurl{https://weread.qq.com/web/reader/0c032c9071dbddbc0c06459k37632cd021737693cfc7149}
%
\end{footnotetext}\ignorespaces 
别高估自己、低估对手!

SWOT分析,又称态势分析,它是一种常用的战略规划分析模型,主要针对产品研发过程中涉及的一些关键要素进行分析。

SWOT分析是从优势(Strengths)、劣势(Weakness)、机会(Opportunity)、威胁(Threats)四个方面进行的。

其中优势和劣势是针对自身而言的,即内部资源的优劣势;而机会和威胁则是针对外部而言的,可从市场、环境、社会等方面分析产品存在的机会和面临的威胁。产品经理通过一系列调查将这些内容列举出来,并依照矩阵的形式进行排列,然后将产品设计过程中涉及的各种因素相互匹配加以分析,进而得出一系列相应的结论。

SWOT分析通常在产品规划期内使用,用以确定产品的战略方

SW分析:着眼于企业内部,由于企业是一个整体,且资源分散在各个部门或岗位之中,因此进行SW分析时必须从企业整个价值链中的每个环节入手,将涉及的每个环节与竞争对手进行对比,对比时也需要针对企业和产品进行特别分析。例如,设计一款智能音箱,需要分析企业能力,如产品部门的设计能力、研发部门的技术储备、生产制造工艺、生产能力、资金支持情况、人力资源储备、市场渠道、销售渠道等。分析产品时可以从新颖度、识别精度、制造工艺、价格优势等方面入手。
OT分析:随着社会、政治、经济、科技等方面的变化和发展,企业所处的环境也更为开放。由于这种外部环境变化是不受控制的,因此在产品分析过程中OT分析也变得至关重要。外部环境既存在威胁,也存在机会。环境威胁指的是由环境中一种不利的发展趋势所形成的不可预知的挑战,如果不采取相应的措施,这种不利趋势可能导致产品的优势被削弱。环境机会则是可能对产品形成优势的因素,通过这些机会,产品的竞争力将得到提高。


\subsubsection{竞品分析 1\sphinxfootnotemark[174]}
\label{\detokenize{chapter_skill/goods_analysis:id1}}\label{\detokenize{chapter_skill/goods_analysis::doc}}%
\begin{footnotetext}[174]\sphinxAtStartFootnote
\sphinxnolinkurl{http://www.woshipm.com/pmd/1842636.html}
%
\end{footnotetext}\ignorespaces 

\paragraph{目的}
\label{\detokenize{chapter_skill/goods_analysis:id2}}
工作目标的不同,需要解决的问题不同。竞品分析的方式方法也会不同,最终把问题解决了才是有效的竞品分析!

用户分级来提升粘性活跃度和内容质量。
年龄大看不清分析竞品的字号大小,但目标群体年轻人就不重要了。


\paragraph{分类 4\sphinxfootnotemark[175]}
\label{\detokenize{chapter_skill/goods_analysis:id3}}%
\begin{footnotetext}[175]\sphinxAtStartFootnote
\sphinxnolinkurl{https://www.zhihu.com/question/39005837/answer/167081923}
%
\end{footnotetext}\ignorespaces \begin{itemize}
\item {} 
功能分析:主要是列出自己产品的功能和同级别产品的功能,从中发现成本差异,研发能力差异。

\item {} 
交互分析:交互其实已经包含工业设计中,但是有必要单独拿出来分析,因为决定了用户体验。是用已经发展多年的按键,触屏,手机控制还是用目前最火的语音交互?为什么?

\item {} 
设计分析:在工业设计上,设计语言不仅是公司战略级的事务,也是产品极为重要的元素之一。设计领域还是有规可循的,把该类产品历史发展的设计风格和现在市场的风格研究一番,就能略知一二。还应该讨论,设计元素的变更有什么原因,目前的技术是否能够支持。

\end{itemize}


\paragraph{原则}
\label{\detokenize{chapter_skill/goods_analysis:id4}}\begin{itemize}
\item {} 
海盗指标

\item {} 
场景分析

\item {} 
关键流程分析

\item {} 
尼尔森法则

\end{itemize}


\paragraph{步骤}
\label{\detokenize{chapter_skill/goods_analysis:id5}}

\subparagraph{产品定位规划阶段}
\label{\detokenize{chapter_skill/goods_analysis:id6}}
要解决的问题是:
\begin{itemize}
\item {} 
我们要怎么做?

\item {} 
如何来做才能赚到更多的钱?

\item {} 
如何比竞品做的更好?

\end{itemize}

竞品是怎么做的?主要包含,用户细分,产品定位产品模式,业务模型,盈利模型,付费的转化漏斗模型等等。


\subparagraph{产品设计阶段}
\label{\detokenize{chapter_skill/goods_analysis:id7}}\begin{itemize}
\item {} 
我们的产品如何设计?

\item {} 
业务流程是什么?

\item {} 
都有那些功能?

\item {} 
功能逻辑的细节是什么?

\end{itemize}

这个阶段的竞品分析,更关注的是,以上这些内容竞品是怎么做的?


\subparagraph{产品的优化迭代阶段}
\label{\detokenize{chapter_skill/goods_analysis:id8}}
任何竞品分析都不可能是静态的,整个市场在变动,分析也应该长期保持更新:

我们的产品存在哪些问题? 如何优化改进?
可以通过数据分析,并与竞品进行对比分析来,发现自身的问题

根据产品的问题确定竞品分析的方式和内容


\paragraph{竞品定义}
\label{\detokenize{chapter_skill/goods_analysis:id9}}
蛋糕 = 需求量*ARPU,僧多粥少?

竞品一词来源于经济学领域,是指对竞争对手产品的优劣势进行比较分析。随着互联网时代的到来,现在的竞品用处更加宽泛。追根溯源竞品分析更像是工业时代企业管理里面的标杆管理,都是一种见贤思齐的人性体现,我更愿意接受竞品分析是战略规划工具这一说法。

竞品理念最早的成功案例要说富士施乐(FUJI
XEROX)公司,施乐公司通过不断的同当时的日本企业进行对标,最后在复印机市场成功逆袭日本佳能。施乐公司成功后,这个方法风靡美国很多大型公司。


\paragraph{对什么?}
\label{\detokenize{chapter_skill/goods_analysis:id10}}
出发点不对的强制竞品分析任务只会适得其反,带着目的与初衷愿景去做竞品分析,做出来的结果才有意义。


\paragraph{和谁对?}
\label{\detokenize{chapter_skill/goods_analysis:id11}}
这五力分别指,同行业内现有竞争者的竞争能力、潜在竞争者进入的能力、替代品的替代能力、供应商的讨价还价能力、购买者的讨价还价能力。我们可能对直接竞争对手关注颇多,但殊不知竞品无处不在。

重新定义用户视角下的竞品分类。

做竞品分析,是为了从竞争者那里抢用户!

用户认为我们是什么,把我们归到哪一类,我们就应该在这个范围内去找自己的竞品,这之后,才谈得上抢用户。

分析“用户”!

知识付费类的APP:喜马拉雅APP,也可以是混沌大学APP
消遣无聊时光:喜马拉雅了,还可能是花椒、蜻蜓FM。
学习:选择线下大学、培训班、学习类图书、音像制品等等
没有明确的需求方向:所有可能抢占用户钱包和时间的产品都可以作为彼此的竞品。“得到”的竞品可能就是“王者荣耀”


\paragraph{怎么对}
\label{\detokenize{chapter_skill/goods_analysis:id12}}
常规的一些资料权威来源主要是一些有公信力的第三方咨询或报告公司,比如:艾瑞咨询、Talkingdata、IDC、麦肯锡、易观智库、企鹅智库、CNNIC、百度指数、友盟数据、App
Annie等。

行业协会与企业官网也是不错的选择,一般是行业内的自律组织,很多行业协会都会定期在网站上发布业内活动、重大新闻、大型会晤、科技进步以及行业数据。虽然可能有水分,但是相对比较直观量化。很多官网都会详细介绍拳头产品,甚至提供完整的产品手册(新手说明、常见问题、技术文档等),你可以和自己的产品比对,快速找出优缺点。


\paragraph{竞品分类}
\label{\detokenize{chapter_skill/goods_analysis:id13}}
\begin{figure}[H]
\centering
\capstart

\noindent\sphinxincludegraphics{{goods_env}.png}
\caption{竞品生态的组成}\label{\detokenize{chapter_skill/goods_analysis:id19}}\end{figure}

对于问题和方案的异同,我们可以用象限概念来帮助理解,把与自己产品有关系的潜在竞争对手分为四大类:问题同方案同、问题同方案异、问题异方案同、问题异方案异。

问题同方案同:直接竞品的厮杀通常是渐进式的创新,此消彼长
问题同方案异:用不同方案解决相似问题的产品,往往会成为行业里颠覆巨头的下一代产品。要特别关注!
问题异方案异:产品存在跨行业迁移,京东原来只卖 3C
数码,积累了用户和基础设施之后,卖起书来一点都不比当当差、星巴克原来只卖咖啡,2019
年也推出了茶饮料。
问题异方案异:占用了相似的不可再生资源,比如时间、金钱、人才等。产品产业链条中的上下游。任何行业里的某个角色,如果做大做强了,都很可能忍不住要占据产业链条里更多的位置


\subparagraph{竞品分类举例 2\sphinxfootnotemark[176]}
\label{\detokenize{chapter_skill/goods_analysis:id14}}%
\begin{footnotetext}[176]\sphinxAtStartFootnote
\sphinxnolinkurl{https://www.bilibili.com/video/BV1wz4y1y7sg?p=4}
%
\end{footnotetext}\ignorespaces 
烧饼 工艺、口味、为啥好?
\begin{enumerate}
\sphinxsetlistlabels{\arabic}{enumi}{enumii}{}{.}%
\item {} 
直接竞品:边上家的烧饼

\item {} 
间接竞品:卖麻花

\item {} 
潜在竞品:卖臭豆腐的也想卖烧饼

\end{enumerate}
\begin{itemize}
\item {} 
一共有多少人跟我抢这块蛋糕?(竞争形势)

\item {} 
最好的几个是谁?(用户规模、融資、口碑)

\item {} 
他们用什么方法抢的?(产品模式)

\item {} 
他们产品有多少功能?(需求分析)

\item {} 
他们盈利模式是如何设计的?

\item {} 
运营转化策略是什如何推广的?

\item {} 
他们抢到了多少?

\item {} 
他们的发展曲线

\item {} 
竞品的优点和缺点

\item {} 
参考他们的转化漏斗模型:用户量\sphinxhyphen{}》活跃\sphinxhyphen{}》转化

\end{itemize}

定位规划 功能设计 优化迭代

为我所用!


\paragraph{ToB的「无竞品」}
\label{\detokenize{chapter_skill/goods_analysis:tob}}
即使市场上存在性质类似的产品,作为普通用户想访问和使用也不是那么容易

外部公开的相似ToB产品设计资料资料可能很少,但对内的话,如果稍微留心搜索寻找一下,是可以通过内网的论坛、云盘、设计交流站点、设计稿预览站点还有不定期举办的内部专业分享等,找到前人对于类似项目的设计文档与经验总结的,给自己的设计思路带来启发。

这些小的模块很多在我们熟悉的ToC产品里都能找到影子,具体到交互设计模式很多都是通用的

一边学、一边猜、一边悟,通过收集资料,不断分析拼凑自己的产品版图


\paragraph{竞品选择策略}
\label{\detokenize{chapter_skill/goods_analysis:id15}}\label{\detokenize{chapter_skill/goods_analysis:id16}}
产品生命周期有所了解。主要包括四个发展阶段:导入期、成长期、成熟期和衰退期。

在品类不同的发展期间,用户对品类的认知是不一样的,对应的,竞品选择策略也是不一样的。

导入期:家用轿车是更快的马车、不用马拉的马车
成长期:A领导,竞品很可能还是马车。第二、第三抢领导品牌A
成熟期:沃尔沃代表安全,宝马代表驾驶的乐趣,可能还有消费者认为,日系品牌代表省油等等
衰退期:需求下降,换一个赛道,特性的创新


\paragraph{步骤}
\label{\detokenize{chapter_skill/goods_analysis:id17}}

\subparagraph{第一种刚起步,从0\sphinxhyphen{}1}
\label{\detokenize{chapter_skill/goods_analysis:id18}}
step1:找到优质竞品

行业热门、人气最旺、融资最多、最具特色
\begin{itemize}
\item {} 
关键词,搜索:全部>只找最优秀的几个(前10)

\item {} 
行业调研过程中发现的优秀竞品

\item {} 
基础数据查找,进行筛选

\end{itemize}

step2:锁定核心竞品

step3:确认分析维度
\begin{itemize}
\item {} 
产品不同、行业不同、业不同、产品关注点不同,你需要跟老大沟通的

\item {} 
产品概述(介绍这款产品的业努,公司背景)

\item {} 
产品模式(模式分析,优劣对比)

\item {} 
用户细分(用户模式,用户画像)

\item {} 
基本运营现状(用户量、日活月活、单量等指标)

\item {} 
盈利模式(讲清楚,讲细,都多少种,多少钱,角色差异)

\item {} 
核心业务流程、核心功能、亮点(点,要细节,要细节,要细节)

\end{itemize}

step4:横向对比分析

step5:借鉴与规避竞品分析总结,结合我们自身情况,可以吸收的
\begin{itemize}
\item {} 
产品模式、用户细分、盈利模式、特色亮点也许是融资最多的

\item {} 
核心业务流程

\item {} 
核心功能

\item {} 
竞品总结(借鉴与规避)

\end{itemize}


\subsubsection{MRD 1\sphinxfootnotemark[177]}
\label{\detokenize{chapter_skill/MRD:mrd-1}}\label{\detokenize{chapter_skill/MRD::doc}}%
\begin{footnotetext}[177]\sphinxAtStartFootnote
\sphinxnolinkurl{http://www.woshipm.com/pmd/131946.html}
%
\end{footnotetext}\ignorespaces 

\paragraph{定义}
\label{\detokenize{chapter_skill/MRD:id1}}
BRD:这值得吗?好处在哪?

MRD的定义:通过BRD明确这件事值得做之后,描述应该怎么做,并说明这么做的原因。如果说BRD是抛出了论题,那么MRD就是要我们用论点来支撑BRD,同时通过论证来得出我们采用什么样的方式获得BRD的商业目标。就是经过一系列分析后,拿出一套我们认为最适合干这个事情的方法以及指导实施的文档。


\paragraph{最优解基本条件 2\sphinxfootnotemark[178]}
\label{\detokenize{chapter_skill/MRD:id2}}%
\begin{footnotetext}[178]\sphinxAtStartFootnote
\sphinxnolinkurl{https://www.bilibili.com/video/BV1wz4y1y7sg}
%
\end{footnotetext}\ignorespaces \begin{enumerate}
\sphinxsetlistlabels{\arabic}{enumi}{enumii}{}{.}%
\item {} 
最懂用户

\item {} 
比竞争对手做的更好

\item {} 
要让用户知道你

\end{enumerate}

勤奋+努力+人脉资源+。。。


\paragraph{MRD的阅读对象}
\label{\detokenize{chapter_skill/MRD:mrd}}
MRD的汇报对象:未来参与产品的各个层级的同事,包括产品经理自己。
\begin{enumerate}
\sphinxsetlistlabels{\arabic}{enumi}{enumii}{}{.}%
\item {} 
MRD是最完善的产品诞生分析描述文档。

\item {} 
以后的一段时间,产品的各个衍生文档、产品依据、团队判断,都要参考MRD。

\item {} 
产品参与成员了解产品的各种背景、数据和方法的依据。

\end{enumerate}


\paragraph{主要内容}
\label{\detokenize{chapter_skill/MRD:id3}}

\subparagraph{文档说明}
\label{\detokenize{chapter_skill/MRD:id4}}\begin{itemize}
\item {} 
文档说明:文档基本信息;文档修改记录;文档目的;文档概要。

\item {} 
文档基本信息:公司名称,产品名称,文档创建日期,创建人,创建人联系方式,部门、职位。

\item {} 
文档修改记录:日期,版本、修改人、修改记录、审核人。

\item {} 
文档目的:用于说明产品的市场、用户、产品规划、核心目标、产品路线图、项目规划等。

\item {} 
文档概要:文档说明;市场说明;用户说明;产品说明。

\end{itemize}


\subparagraph{市场说明(industry\_analysis)}
\label{\detokenize{chapter_skill/MRD:industry-analysis}}\begin{itemize}
\item {} 
市场说明:摘要(可选);

\item {} 
现在市场存在的问题和机会:根据需要撰写–两到三点:产品方面–形态复杂,用户体验差;技术方面–(语音压缩技术不成熟,外资搜索引擎对中文理解不够深刻);运营方面–(产业链篇下游,重实体,轻线上,造成瓜分旅行社利润,形成对立);用户方面(新的需求的出现,需求明显);商业模式方面。

\item {} 
目标市场分析:市场规模;市场特征;发展趋势(未来2\sphinxhyphen{}5年的发展评测);时间边界(这个市场的持续时间预估)。

\item {} 
市场分析结论:一般来说,这里会得到一个比较有市场商业价值的结论。

\end{itemize}


\paragraph{用户说明(users\_analysis)}
\label{\detokenize{chapter_skill/MRD:users-analysis}}\begin{itemize}
\item {} 
用户说明:目标用户群体(要求准确:年龄段、收入、地区、学历);

\item {} 
目标群体特征:共性的为主分析;

\item {} 
建立虚拟用户角色:形象化,常用用户特征,用户名称,用户技能、与产品相关的用户特征。

\item {} 
制作用户角色卡片:对用户归类划分,抽取典型角色,能代表目标用户。

\item {} 
用户场景分析:演示性的场景,用户在时间、地点,完成的某个事的故事。

\item {} 
用户动机总结;用户目标总结(明确实质);分析影响用户使用的主要因素。

\item {} 
注意:从技术层面剖析市场,洞察用户心理案例分析。动机和目标是不一致的。

\end{itemize}


\subparagraph{产品说明(goods\_analysis 、compete\_analysis)}
\label{\detokenize{chapter_skill/MRD:goods-analysis-compete-analysis}}\begin{itemize}
\item {} 
产品定位:我们用什么样的产品满足用户或用户市场;针对什么用户,做什么事。

\item {} 
产品的核心目标:解决目标市场、用户的核心需求;核心目标的工作级别最高。

\item {} 
产品结构:整体结构,不是功能结构。是产品的核心目标,市场定位,产品定位的直接体现。

\item {} 
产品的路线图;以时间为节点的任务导向。

\item {} 
产品功能性需求:用户注册、留言等等。

\item {} 
非功能性需求:有效性、性能、扩展性、安全性、健壮性、兼容性、可用性、运营需求、用户体验。

\end{itemize}


\paragraph{优秀MBR}
\label{\detokenize{chapter_skill/MRD:mbr}}\begin{itemize}
\item {} 
逻辑性强(有论点,论据,论证);

\item {} 
把抽象的东西形象化出来;

\item {} 
数据可靠,分析有理;有把握的主观,无把握的客观;

\item {} 
用词行文,简洁明了;

\item {} 
合理的产品进度分析;

\item {} 
重视非功能需求;

\item {} 
解释专业名词。

\end{itemize}


\paragraph{市场调研能力}
\label{\detokenize{chapter_skill/MRD:id5}}

\subsubsection{PRD}
\label{\detokenize{chapter_skill/PRD:prd}}\label{\detokenize{chapter_skill/PRD::doc}}
商业需求文档(BRD)和市场需求文档(MRD)主要是前期项目调研的产物,而研发和测试主要是以产品需求文档(PRD)为蓝本来开展工作的


\paragraph{定义1\sphinxfootnotemark[179]}
\label{\detokenize{chapter_skill/PRD:id1}}%
\begin{footnotetext}[179]\sphinxAtStartFootnote
\sphinxnolinkurl{http://www.woshipm.com/pmd/3319375.html}
%
\end{footnotetext}\ignorespaces 
PRD是产品汪思考后的产品方案呈现载体,主要用于沟通研发、测试、设计、运营等,也会用于存档便于查看和回溯。

PRD 主要是给开发人员、UI
设计人员、测试人员等看的,目的是让开发人员知道如何按照产品经理的思路为需求撰写代码,让UI
设计人员知道他们需要输出哪些UI
设计稿,让测试人员知道测试用例中需要包含哪些测试点。PRD
的撰写侧重点是需求描述、需求逻辑、需求原型。PRD
是所有产品经理都会接触到的,也是产品经理平时写得最多的文档,所以大家一定要好好研究。


\subparagraph{战略层——用户需求\&产品目标}
\label{\detokenize{chapter_skill/PRD:id2}}
将需求背景、用户需求写在开头,后续撰写都要围绕这个核心,相当于定了方向,也可以指引阅读者更好更快的了解需求。
不管是定量目标还是定性目标,都要落实到PRD上,这是整个项目团队工作的预期成果。

比如36氪的目标用户就是创业者、投资人、互联网职场人和学生,创业者要找36氪报道项目,投资人要通过看36氪找可以投资的好项目,互联网职场人和学生要看36氪开拓视野,了解圈内大事,以补充自己的知识储备量。
另外注意,考虑用户时,如果是多边平台型产品,就不仅要考虑 C 端,还要顾及
B 端用户,比如广告主、机构负责人等等。

给公司带来多大价值:分金钱收益和流量收益,结合时间维度分别说明短期收益和长期收益。比如3个月内聚集用户,半年内展开招商,一年内有持续现金流等等。
注意:在描述收益时,尽量「量化」收益数值,尤其是需要和公司历史收益对比,和同行业其他家收益对比,和行业规模对比,才能做到具备一定说服力。\sphinxhref{https://www.yuque.com/weis/pm/el10i7}{5}%
\begin{footnote}[180]\sphinxAtStartFootnote
\sphinxnolinkurl{https://www.yuque.com/weis/pm/el10i7}
%
\end{footnote}


\subparagraph{范围层——功能规格\&内容需求}
\label{\detokenize{chapter_skill/PRD:id3}}
战略层中明确了用户需求和产品目标后,范围层就要确定做哪些功能、提供什么内容来实现产品目标,可以将各个功能点以及它们的优先级用脑图或excel画出来(工具随意),相当于把整个需求分拆成各个模块,便于理解和评估工作。


\subparagraph{结构层——交互设计\&信息架构}
\label{\detokenize{chapter_skill/PRD:id4}}
范围层把需求分拆之后,结构层再把各个部分用流程图或思维脑图连接起来,让阅读者能抽象出产品的使用流程或信息架构。

偏交易、工具的产品需求可以使用流程图,偏内容的产品需求可以使用思维脑图,此处放一个点餐交易的局部流程图,仅做示例参考。


\subparagraph{框架层——界面设计\&导航设计}
\label{\detokenize{chapter_skill/PRD:id5}}
结构层将需求分拆模块,框架层梳理整体流程,相信这时候PRD读者已经对需求有了清晰的思路,框架层主要对每个模块或页面的逻辑、布局做详细阐述。


\subparagraph{表现层——视觉设计}
\label{\detokenize{chapter_skill/PRD:id6}}
表现层主要包括但不限于视觉、听觉、触觉等内容,需要做一些高保真的设计,建议与UED、用户体验部门联合产出,这里不做过多展开。


\paragraph{过程2\sphinxfootnotemark[181]}
\label{\detokenize{chapter_skill/PRD:id7}}%
\begin{footnotetext}[181]\sphinxAtStartFootnote
\sphinxnolinkurl{http://www.woshipm.com/pmd/3516749.html}
%
\end{footnotetext}\ignorespaces 
先找研发对一下需求,连接上下游的关系。然后再写,把层次关系梳理出现,再用图表或流程图表现。拆分组块业务逻辑,梳理业务上下游。

目录:
\begin{enumerate}
\sphinxsetlistlabels{\arabic}{enumi}{enumii}{}{.}%
\item {} 
产品背景

\item {} 
名词解释

\item {} 
产品综述

\item {} 
用户故事

\item {} 
需求详述

\item {} 
评审记录

\item {} 
其他问题描述

\end{enumerate}


\subparagraph{产品背景}
\label{\detokenize{chapter_skill/PRD:id8}}

\subparagraph{背景概述}
\label{\detokenize{chapter_skill/PRD:id9}}
含义解释:背景概述是用简单的语言大概概括一下大的背景,让人知道我们本次要讲的内容大概是什么。

描述正文:官网为用户提供产品试用,目前,完整的试用流程如下:

用户在官网进行注册,填写申请试用表单。商务(运营)在管理后台,对用户的申请进行授权操作(允许/拒绝)。

注释:这样的背景描述,是将云官网,本次的产品需求,用业务流程串联起来,从前端到后端。从业务流程出发,将业务串联起来,这是一种非常好的方式。用一个事件,将涉及的所有产品功能都串联起来,让本次讨论有主线。

一般来说,这段话的作用在于让人阅读后明白我们为什么要花时间做这件事,以及明白了这件事的意义所在。重点在WHY,关于WHY的重要性,大家可以看一个演讲叫做How
great leaders inspire action。

既然是完善和优化,那么产品经理就需要向运营、向研发证明,为什么这样的修改,相较于原来的流程,更好。

Example: \sphinxurl{http://www.woshipm.com/pd/4167090.html}


\paragraph{AI产品}
\label{\detokenize{chapter_skill/PRD:ai}}
二者的差异主要集中在需求内容的部分。而需求内容这方面的差异又因PM角色的不同而有所不同。


\paragraph{测试完整性3\sphinxfootnotemark[182]}
\label{\detokenize{chapter_skill/PRD:id10}}%
\begin{footnotetext}[182]\sphinxAtStartFootnote
\sphinxnolinkurl{http://www.woshipm.com/pmd/21446.html}
%
\end{footnotetext}\ignorespaces 
现在你有一个PRD草稿,你需要测试它的完整性。工程师是否可以充分了解并达到目标?OA
Team(质量管理团队)是否有足够的信息来做出测试计划,是否可以开始做案例?

当投资人或相关人审核了PRD,确定了各个需要说明的方面,所有的问题得到解决,现在你就可以按PRD进行产品开发。


\paragraph{管理产品3\sphinxfootnotemark[183]}
\label{\detokenize{chapter_skill/PRD:id11}}%
\begin{footnotetext}[183]\sphinxAtStartFootnote
\sphinxnolinkurl{http://www.woshipm.com/pmd/21446.html}
%
\end{footnotetext}\ignorespaces 
解决所有PRD中存在问题,如果不在PRD中就写进去。你的任务就是迅速解决问题并记录在PRD。

如果你做了你的工作并准备记录在PRD,项目审查就会变得非常简单,因为任何一个部份都历历在目。

记住PRD是一个“活”的文件,在要跟踪记录在产品开发期间的所有功能过程。最后你会发现很多额外的东西,如果你认为是必要的就在PRD中写进。

confluence:这个不多说了,非常好用的一款产品PRD在线编辑软件。\sphinxhref{http://www.woshipm.com/pmd/913343.html}{4}%
\begin{footnote}[184]\sphinxAtStartFootnote
\sphinxnolinkurl{http://www.woshipm.com/pmd/913343.html}
%
\end{footnote}


\paragraph{COD评分表方法}
\label{\detokenize{chapter_skill/PRD:cod}}

\paragraph{信息架构}
\label{\detokenize{chapter_skill/PRD:id12}}
组织系统、标签系统、导航系统、搜索系统


\subsubsection{原型设计}
\label{\detokenize{chapter_skill/prototype_design:id1}}\label{\detokenize{chapter_skill/prototype_design::doc}}
重点:可视+互动

在产品定义阶段,用户参与产品设计和体验,解决75\%的问题才是最高效的。

原型的定义:是产品方案的输出物
\sphinxhref{https://www.zhihu.com/question/55997614/answer/615628989}{1}%
\begin{footnote}[185]\sphinxAtStartFootnote
\sphinxnolinkurl{https://www.zhihu.com/question/55997614/answer/615628989}
%
\end{footnote},将想法转化为高效传达给其他人(程序员和ui)或者与用一起测试的形式,并且随看时间的推移,可以持续进行调整。你可以用它来自如的测试未完成的想法,从而达到最佳的结果。接受反馈,将反馈再次融入到你的原型设计中。


\paragraph{原型设计和评审}
\label{\detokenize{chapter_skill/prototype_design:id2}}
由产品经理主导的。基于运营的需求设计原型,在原型设计完后,要经过内部评审和外部评审。

在内部评审中,产品经理要召集数据中台的架构师、模型设计师、数据开发工程师、后端开发工程师、前端开发工程师、UI设计师、测试工程师,说明整个功能的价值和详细的业务流程、操作流程,确保大家理解一致。

接下来,产品经理和运营人员要针对原型做一次外部评审,把有歧义的地方一并解决。

对于比较重要的功能,产品经理需要发邮件让运营人员进一步确认,并同步给所有的产品/运营人员,保证大家的口径一致。


\paragraph{原型分类}
\label{\detokenize{chapter_skill/prototype_design:id3}}
低保真原型:线框图,不可交互,打印在纸上,展示主业务流程;

中保真原型:可简单交互(点击页面跳转),简单配色并配图,但字体、配色和配图不完美;

高保真原型:开发和设计参与,原型可交互,可录入信息,可以实现所有流程,有动画效果,字体、配色和配图与真实产品一致;


\paragraph{反馈圈分类}
\label{\detokenize{chapter_skill/prototype_design:id4}}
团队成员或朋友:团队成员,公司员工,社交圈朋友。特点:容易找到,易与沟通。使用低保真原型,了解主要流程是否存在问题。

行业专家:相关领域产品经理和产品专家,他们了解目标市场。特点:能提出逻辑清晰的意见。使用低,中保真原型。

客户:目标市场的客户。使用中、高保真原型,他们会提供非常棒的反馈意见

客户的客户:特定人群应用,如投资人等。使用高保真原型展示,获取真实反馈。


\paragraph{信息收集工具}
\label{\detokenize{chapter_skill/prototype_design:id5}}\begin{itemize}
\item {} 
记笔记;

\item {} 
手机录音;

\item {} 
在用户允许的情况下,录入表情;

\item {} 
原型操作过程手机或电脑页面录制。

\end{itemize}


\paragraph{工具推荐}
\label{\detokenize{chapter_skill/prototype_design:id6}}\begin{itemize}
\item {} 
低保真\sphinxhyphen{}线框\sphinxhyphen{}纸质原型:可以用于产品团队内部交流使用,通过头脑风暴,跑通住业务流程。确定MVP(Minimum
Viable Product, MVP)方案,确定中保真原型方案。

\item {} 
中保真原型–墨刀或axure:对于朋友、专家和客户可以提供具有交互功能的可录入信息的中保真原型,设计尽量保证完成,以便于手机用户的全部操作行为。

\item {} 
高保真原型:开发资源富裕的团队,如微信团队,可以实现每个想法的高保真原型快速输出和范围测试。

\end{itemize}


\subparagraph{墨刀}
\label{\detokenize{chapter_skill/prototype_design:id7}}
墨刀\sphinxhref{https://zhuanlan.zhihu.com/p/33997501}{3}%
\begin{footnote}[186]\sphinxAtStartFootnote
\sphinxnolinkurl{https://zhuanlan.zhihu.com/p/33997501}
%
\end{footnote}是一个款移动端原型制作工具,原型制作制作方便,可以快捷添加一些简单的交互动画,信息录入,逻辑判断等功能,支持微信分享,便于传播手机信息。

挑一种学一下就可以了(最易学习的是墨刀,Sketch学会了用起来最顺手快捷,Axure应该算是最不好用的一个了


\paragraph{产品演进路线图 2\sphinxfootnotemark[187]}
\label{\detokenize{chapter_skill/prototype_design:id8}}%
\begin{footnotetext}[187]\sphinxAtStartFootnote
\sphinxnolinkurl{https://www.bilibili.com/video/BV1254y1D7Ht?from=search\&seid=14167562900175777805}
%
\end{footnotetext}\ignorespaces 
产品演进路线图是为了给产品的利益相关者以及团队呈现产品长远发展的构思。使团队内部和利益相关者对项目的长远发展一目了然并达成共识

近期:为了满足需求必须要包含的功能 中期:关注范围广、有一定的灵活性的功能
远期:顶层设计、宽泛的范围、更加灵活的功能


\paragraph{验证(Validate) 4\sphinxfootnotemark[188]}
\label{\detokenize{chapter_skill/prototype_design:validate-4}}%
\begin{footnotetext}[188]\sphinxAtStartFootnote
\sphinxnolinkurl{https://www.jianshu.com/p/cb6ae5a3f3fa}
%
\end{footnotetext}\ignorespaces 
验证产品原型


\paragraph{题目 5\sphinxfootnotemark[189]}
\label{\detokenize{chapter_skill/prototype_design:id9}}%
\begin{footnotetext}[189]\sphinxAtStartFootnote
\sphinxnolinkurl{https://blog.nowcoder.net/n/9bd8651faead4a73ae344be0b74128de}
%
\end{footnotetext}\ignorespaces 
Axure文件的后缀名RP(Rapid Prototyping)指快速原型
\begin{itemize}
\item {} 
Action sheet:动作菜单/动作面板/行动列表

\item {} 
Picker:选择器/拾取器

\item {} 
Toast:吐司提示
产品设计中常把出现在屏幕中、用以引起注意,短暂出现后消失的提醒控件叫做

\item {} 
Status bar:全局修改状态栏

\item {} 
More: \sphinxhref{https://www.jianshu.com/nb/9076183}{6}%
\begin{footnote}[190]\sphinxAtStartFootnote
\sphinxnolinkurl{https://www.jianshu.com/nb/9076183}
%
\end{footnote}

\end{itemize}


\subsubsection{沟通、协作、项目推动能力}
\label{\detokenize{chapter_skill/collaborate:id1}}\label{\detokenize{chapter_skill/collaborate::doc}}
人工智能的应用模式是对产品功能的赋能,因此在推进产品上线的过程中,跨团队、跨部门沟通是一种常态,有时候因为数据分散在各个业务模块,这时就更考验AI产品经理凝聚团队的能力,以及清楚表达项目价值的能力。

有的AI产品经理的日常工作就是需要和多方沟通、协调各方资源、达成产品目标。有人说“在产品的需求评审的时候,能够应答如流”是AI产品经理个人的荣耀时刻,但“顺利完成需求评审,跟进产品上线、把握项目进度,产品得到用户的认可”这个过程才是整个团队最为骄傲的时刻。AI产品经理的工作不仅表现出AI产品经理具备表达能力、逻辑能力和执行力,表现出跟进产品开发上线过程中的为人处事的能力,其实也是其个人人格魅力的很好体现。

对于产品经理来说,人工智能的产品应用显然会比一个普通的功能设计复杂得多,从项目初期的数据评估、设定项目的目标、可行性方案跟踪,到项目中期的研发、设计、测试、协调,到最后项目上线的效果评估,每一个环节都考验着AI产品经理沟通、协作和项目推动的能力。

TODO:\sphinxurl{https://weread.qq.com/web/reader/40632860719ad5bb4060856ka5732aa0226a5771bce9dc4}


\subsubsection{迭代优化}
\label{\detokenize{chapter_skill/upgrade_manage:id1}}\label{\detokenize{chapter_skill/upgrade_manage::doc}}

\paragraph{需求管理背景}
\label{\detokenize{chapter_skill/upgrade_manage:id2}}
内部资源永远是紧张的,外部竞争却时刻在发生。所谓内部资源,主要是指研发资源。与AI相关的研发工程师在大多数企业中都是处于稀缺的状态,如何“好钢用在刀刃上”,那就考验产品经理的决策能力了。

当产品经理面对多个需求时该怎样做呢?我们没有办法把所有的需求同时做好,与其接收多个需求同时做,可能每个需求和功能都做不好,不如抓住一个需求做到完整。


\paragraph{需求管理}
\label{\detokenize{chapter_skill/upgrade_manage:id3}}
需求管理——加深对业务和产品的理解;需求优先级、工作迭代计划——业务准确判断力

需求排序\sphinxhref{http://www.woshipm.com/pd/1887717.html}{1}%
\begin{footnote}[191]\sphinxAtStartFootnote
\sphinxnolinkurl{http://www.woshipm.com/pd/1887717.html}
%
\end{footnote}——第一要务就是分清产品需求的主次,解决问题的顺序:


\paragraph{需求来源部分}
\label{\detokenize{chapter_skill/upgrade_manage:id4}}
不同方向的需求来源略有不同,总体来说,产品经理的需求来源有以下几个方面:
\begin{itemize}
\item {} 
业务需求:由业务方,比如 BD、编审、运营等,直接提出的业务需求;

\item {} 
数据挖掘:通过数据挖掘和分析,发现的问题或需求;

\item {} 
竞对调研:通过分析竞对的产品,发现竞对比我们有优势或值得学习借鉴的地方;

\item {} 
实地观察:不论是 B 端产品还是 C
端产品,其实都有大量的机会实地观察用户行为。比如直接陪访城市运营等;

\item {} 
战略需要:俗话说「老大拍的」,这类是由公司 leader
直接安排的需求,通常表示公司未来发展方向或急需解决的关键问题;

\item {} 
其他还包括客服、商服反馈的问题,通过在线渠道或用户访谈获得的需求,还有微博、朋友圈吐槽等各种
SNS
途径。不论哪种途径获取的需求,产品经理都应该有一个自己的需求池,统一记录各种想法。

\end{itemize}

在不同需求来源中,最看重的产品经理自己发现/评估的需求,这是评价产品经理需求决策能力的重要指标。


\paragraph{收集和整理需求}
\label{\detokenize{chapter_skill/upgrade_manage:id5}}
收集和整理需求,就是将需求统一记录到需求池,方便团队内/间的传播。每个团队建议维护自己的需求池地址。

需求池只是一个备忘,记录下来的需求不一定都要做,也未必是值得做的需求才记录下来。


\subparagraph{需求收集}
\label{\detokenize{chapter_skill/upgrade_manage:id6}}\begin{itemize}
\item {} 
需求来源:见上。

\item {} 
需求内容:交互体验优化、业务调整要求、业务管理要求

\item {} 
需求采集:1V1面谈、问卷调研、轮岗实习

\item {} 
需求背后的 \sphinxstylestrong{真实问题} 以及 \sphinxstylestrong{需求的价值}

\item {} 
需求背后的真正问题是什么?

\item {} 
问题是否有简单快速的解法?

\item {} 
问题影响面有多大?个案问题是否值得研究解决

\item {} 
共性问题,优先级和紧急程度?

\end{itemize}


\subparagraph{需求池}
\label{\detokenize{chapter_skill/upgrade_manage:id7}}
企业建立需求池是为了让AI产品经理能够了解在一定时间内该业务的整体需求,从而能够更加全面地评估需求,需求池可以按照周或月的频次进行更新,这样更便于观察。


\subparagraph{需求池管理}
\label{\detokenize{chapter_skill/upgrade_manage:id8}}\begin{enumerate}
\sphinxsetlistlabels{\arabic}{enumi}{enumii}{}{.}%
\item {} 
目的:实现清晰准确的需求管理、迭代计划管理,使项目进度透明

\item {} 
需求池管理模板:

\end{enumerate}
\begin{itemize}
\item {} 
业务线/对应的系统

\item {} 
需求类型:产品需求、产品需求(插入)、技术需求、技术需求(插入)、线上bug

\item {} 
主题:需求概述

\item {} 
背景:原因引发

\item {} 
内容:需求具体描述

\item {} 
预期收益:用户、产品、市场的数据指标

\item {} 
来源:需求提出者

\item {} 
提出时间

\item {} 
优先级:重要紧迫度、目标、作用对象;需要与业务方在原则上达成一致

\item {} 
迭代版本

\item {} 
业务负责人

\item {} 
产品经理

\item {} 
研发负责人

\item {} 
测试负责人

\item {} 
状态

\item {} 
计划上线时间

\item {} 
实际上线时间

\item {} 
前端开始/结束日期

\item {} 
前端研发工作量

\item {} 
发版计划

\end{itemize}


\paragraph{如此困难}
\label{\detokenize{chapter_skill/upgrade_manage:id9}}\begin{itemize}
\item {} 
相比去做更普适的项目,做那些你最喜欢的、自己会用的产品更令人满足。

\item {} 
相比去做直接对你的目标产生影响的项目,把注意力集中在那些聪明有趣的主意上更有诱惑力。

\item {} 
相比去做自己已经有信心的项目,去钻研新的想法更令人兴奋。(有些情况也可能反过来)

\item {} 
相比去做语权大的需求,拒绝伪需求冒着得罪人的更稳妥

\item {} 
相比追求面面俱到,只把主线功能做好显得多么简陋

\end{itemize}


\paragraph{需求属性的评估}
\label{\detokenize{chapter_skill/upgrade_manage:id10}}

\subparagraph{三维度}
\label{\detokenize{chapter_skill/upgrade_manage:id11}}\begin{itemize}
\item {} 
需求价值评估:需求的价值分为两个维度,一个维度是创造效益,另一个维度是节省成本,通过对需求背景的影响因子给定假设条件进行评估,就能大致估算出这个需求可以产生的预期效果;例如:当前的妥投率为95\%,未妥投原因中收件地址错误的原因占比75\%,假设通过增加收件人地址校验,解决85\%,那么需求的预期收益就是提升3.18\%的妥投率。

\item {} 
需求难度评估:难度可以从是否需要设计新模块开发、原有模块改造量、开发实现难度、是否涉及关联系统改造几个方面进行考量;如果一个需求既设计算法选项、参数优化、训练数据标注、模块封装、总控接入、前端改造那么这个需求的实现难度就属于很高的程度了。

\item {} 
需求周期评估:需求拆解后本项目组的开发周期加上关联系统排期、开发的周期就能确定需求的实现周期。

\end{itemize}


\subparagraph{RICE 5\sphinxfootnotemark[192]}
\label{\detokenize{chapter_skill/upgrade_manage:rice-5}}%
\begin{footnotetext}[192]\sphinxAtStartFootnote
\sphinxnolinkurl{https://weread.qq.com/web/reader/40632860719ad5bb4060856ke3632bd0222e369853df322}
%
\end{footnotetext}\ignorespaces 
RICE SCORE =
R\sphinxstyleemphasis{I}C/E,根据RICE评分即可对需求进行排序,该方法比较适用于大型项目,在一般项目中不常使用。


\subparagraph{Reach(接触数量)}
\label{\detokenize{chapter_skill/upgrade_manage:reach}}
接触数量是指用每个时间段的用户数或事件数来衡量,考察一个需求在一定时间段内会影响多少用户。这可能是“每季度客户数量”或“每月交易数量”,尽可能使用产品指标的实际测量结果


\subparagraph{Impact(影响程度)}
\label{\detokenize{chapter_skill/upgrade_manage:impact}}
影响程度是对目标产生可观影响的需求,以此来预估这个项目对个人产生的影响。可以分为巨大影响、高、中、低、极低几个标准。


\subparagraph{Confidence(信心指数)}
\label{\detokenize{chapter_skill/upgrade_manage:confidence}}
有些需求有创意但无数据支持而显得不明确,我们在评估时可以把信心指数考虑进去,可以分高为100\%、中为80\%、低为50\%三个档次。


\subparagraph{Effort(投入精力)}
\label{\detokenize{chapter_skill/upgrade_manage:effort}}
为了迅速行动并且事半功倍,估算项目需要团队的所有成员(产品、设计和工程)的总时间。投入精力的预估单位是人/月。


\subparagraph{MoSCoW}
\label{\detokenize{chapter_skill/upgrade_manage:moscow}}
美其名曰“全面”,以全面来打入市场。最终却是“样样做,样样差”。must
have、should have、could have、won’t have模型。

\begin{figure}[H]
\centering
\capstart

\noindent\sphinxincludegraphics{{MoSCoW}.png}
\caption{MoSCoW}\label{\detokenize{chapter_skill/upgrade_manage:id16}}\end{figure}
\begin{itemize}
\item {} 
位于“1”(Must have):用户价值高,难度低,优先制作

\item {} 
位于“2”(won’t have):用户价值低,难度高,延后制作甚至不做;

\item {} 
位于“3”、“4”:如果交货时间紧,“可以有”将第一批被删除,“应该有”紧随其后。

\end{itemize}


\subparagraph{Kano帮你找到用户满意度2\sphinxfootnotemark[193]}
\label{\detokenize{chapter_skill/upgrade_manage:kano2}}%
\begin{footnotetext}[193]\sphinxAtStartFootnote
\sphinxnolinkurl{https://www.huaweicloud.com/articles/280202e7d83cd36df93e5f027939cbaa.html}
%
\end{footnotetext}\ignorespaces 
一种在不同阶段按产品目标倒退需求优先级的思维方式,它将需求分为三类:

它将需求分为三类:
\begin{enumerate}
\sphinxsetlistlabels{\arabic}{enumi}{enumii}{}{.}%
\item {} 
基础功能。代表产品进入市场的基本门槛,保证能够满足用户普遍需求的最低标准。然而在后续的研发若投入大量精力,并不会显著提高用户的满意度或建立产品的竞争门槛,因此此类需求优先级较低。

\item {} 
性能需求。即在实现基础功能后,为了提升和优化产品性能的需求。这类需求可以在一定程度上提升用户满意度,但其他竞争对手同时也会在这方面持续投入,ROI通常为线性。

\item {} 
尖叫(兴奋)功能。用户使用产品后能够感受到喜悦和兴奋,这种产品可能是非常有创造性的,也有可能带来

\end{enumerate}

属性的成熟程度和情绪反应之间呈线性关系,主要针对于如易用性、成本、娱乐价值和安全性这样的产品特征。

狩野纪昭(Noriaki
Kano)将五种情绪反应可视化为图中的曲线,其中,y轴是情绪反应,x轴是特征的成熟程度。情绪反应的强度由特征如何充分呈现和其成熟程度驱动。

将需求划分为必备型、期望型、魅力型、无差异型、反向型五类,分别以英文字母M、O、A、I、R表示。
\begin{itemize}
\item {} 
必备型需求(M):需求满足时,用户不会感到满意。需求不满足时,用户会很不满意。

\item {} 
期望型需求(O):需求满足时,用户会感到很满意。需求不满足时,用户会很不满意。

\item {} 
魅力型需求(A):该需求超过用户对产品本来的期望,使得用户的满意度急剧上升。即使表现的不完善,用户的满意度也不受影响。

\item {} 
无差异型需求(I):需求被满足或未被满足,都不会对用户的满意度造成影响。

\item {} 
反向型需求(R):该需求与用户的满意度呈反向相关,满足该要求,反而会使用户的满意度下降。

\end{itemize}

better\sphinxhyphen{}worse系数:
\begin{itemize}
\item {} 
Better系数=(期望数+魅力数)/(期望数+魅力数+必备数+无差异数)

\item {} 
Worse系数= \sphinxhyphen{}1*(期望数+必备数)/(期望数+魅力数+必备数+无差异数)

\end{itemize}

Better系数越接近1,表示该具备度越高该需求对用户满意度提升的影响效果越大。Worse系数越接近\sphinxhyphen{}1,表示具备度越低该需求对用户满意度造成的负面影响越大。

\sphinxurl{http://www.woshipm.com/pd/4383131.html}


\subparagraph{维格斯法}
\label{\detokenize{chapter_skill/upgrade_manage:id12}}
该方法将需求分为4个维度来进行评估。
\begin{itemize}
\item {} 
实现需求给客户带来的收益。

\item {} 
不实现需求给客户带来的损害。(不做会怎么样?发现用户在解决这个问题的不满)

\item {} 
实现需求所需要耗费的成本。

\item {} 
实现需求的风险。其中收益和损害是从客户角度出发的,而成本和风险则是从实现角度出发的,是逻辑较清晰且通用的方法。

\end{itemize}


\subparagraph{优先级指数量表}
\label{\detokenize{chapter_skill/upgrade_manage:id13}}
优先级指数=(需求急迫性+功能价值+需求普遍性+数据支持度+资源准备度)/(开发成本+技术实现难度)

\begin{figure}[H]
\centering
\capstart

\noindent\sphinxincludegraphics{{need_judge}.png}
\caption{优先级指数量表}\label{\detokenize{chapter_skill/upgrade_manage:id17}}\end{figure}


\bigskip\hrule\bigskip



\paragraph{产品迭代管理}
\label{\detokenize{chapter_skill/upgrade_manage:id14}}
软件的持续优化、升级:充分利用研发资源,正确认识技术优化所需的资源,升级研发效率
\begin{enumerate}
\sphinxsetlistlabels{\arabic}{enumi}{enumii}{}{.}%
\item {} 
研发资源管理:研发人力资源安排图(时间、负责人、项目模块)

\item {} 
技术优化资源分配

\end{enumerate}
\begin{itemize}
\item {} 
初创:权利开发业务功能,10\%用来技术优化

\item {} 
瓶颈:业务需求满足疲态,技术架构、设计缺陷出现问题,50\%技术优化

\item {} 
重构:80\%资源做技术重构

\item {} 
稳定:10\%\sphinxhyphen{}20\%资源持续做技术优化

\end{itemize}
\begin{enumerate}
\sphinxsetlistlabels{\arabic}{enumi}{enumii}{}{.}%
\item {} 
双周迭代
局限性:MVP不一定能在迭代周期内交付、跨端项目复杂研发节奏相互依赖、难以准确预估工作投入

\end{enumerate}


\paragraph{技巧 4\sphinxfootnotemark[194]}
\label{\detokenize{chapter_skill/upgrade_manage:id15}}%
\begin{footnotetext}[194]\sphinxAtStartFootnote
\sphinxnolinkurl{http://www.woshipm.com/pmd/4161341.html}
%
\end{footnotetext}\ignorespaces \begin{enumerate}
\sphinxsetlistlabels{\arabic}{enumi}{enumii}{}{.}%
\item {} 
平衡
如果是面向B端业务,那么所有业务线对自己的需求都是关注且紧迫的;这时候就需要学会平衡每个业务的需求,不能被业务完全牵着走,这样对产品规划会有极大影响。

\end{enumerate}

那么每当这时候,就需要可以一起拉通多个业务,来集中评估各方的诉求,宣导团队的资源是有限性的;让业务之间来取舍他们之间的优先级,这就是让“决策”转移到业务自身上。
\begin{enumerate}
\sphinxsetlistlabels{\arabic}{enumi}{enumii}{}{.}%
\setcounter{enumi}{1}
\item {} 
替代
任何的产品解决方案都是有备选方案的,那么在当前无法尽快满足用户的前提;可以优先采用临时方案,先满足用户最核心的需求,把其他延伸性需求先砍掉,待到条件成熟在上线完整需求。

\end{enumerate}

而这就是采用“替代”的方式,一定程度上去满足用户需求,这对挽留用户、提升用户口碑有极大帮助。
\begin{enumerate}
\sphinxsetlistlabels{\arabic}{enumi}{enumii}{}{.}%
\setcounter{enumi}{2}
\item {} 
延迟
这是一个不太“厚道”的方式,前面提到用户对他们自身的需求都关注比较强烈,受限当前的规划和限制条件,确实无法那无法尽快满足的情况,但用户是不会理解买账的;那么如何去“安抚‘用户情绪呢,把负面情绪尽可能降到最低?

\end{enumerate}

这就靠一个“拖”字决,可以先给对方的一个明确信号:我们会做(类似先画个饼)。

但是由于一些原因(把这些问题夸大),需要稍微往后一些才能支持,那么这个往后的时间就可以相对灵活可变的,在这里也主要是安抚用户的情绪为主。


\subsubsection{流程图1\sphinxfootnotemark[195]}
\label{\detokenize{chapter_skill/flow_chart:id1}}\label{\detokenize{chapter_skill/flow_chart::doc}}%
\begin{footnotetext}[195]\sphinxAtStartFootnote
\sphinxnolinkurl{http://www.woshipm.com/pd/818876.html}
%
\end{footnotetext}\ignorespaces 
画流程应该是最基本最必要的技能。但往往大多数新产品人却对流程的概念异常模糊。

下面将讲解定义、分类以及画法,及各种流程图的特点。


\paragraph{流程图}
\label{\detokenize{chapter_skill/flow_chart:id2}}
流程——顾名思义:水流的路程;事物进行中的次序或顺序的布置和安排。流程是自然而然就存在的,它可以不规范,可以不固定,可以充满问题。

图——Chart 或者 Diagram,
是将基本固化有一定规律的流程进行显性化和书面化,从而有利于传播与沉淀、流程重组参考。

由两个及以上的步骤,完成一个完整的行为的过程,可称之为流程;注意是两个及以上的步骤。

流程图的核心就在于如何排布事物进行的次序,不同的顺序可能造成截然不同的结果。

流程不可或缺的因素:对象、输入、动作、输出。
\begin{itemize}
\item {} 
对象就是执行人,也就是产品中的用户;

\item {} 
输入可以理解为前提、前置条件;

\item {} 
动作,就是产品中的操作,可以是点击、输入,等等;

\item {} 
输出,可以理解为结果、动作的目的。

\end{itemize}


\paragraph{目的}
\label{\detokenize{chapter_skill/flow_chart:id3}}\begin{enumerate}
\sphinxsetlistlabels{\arabic}{enumi}{enumii}{}{.}%
\item {} 
流程图为产品设计基石,可以保证产品的使用逻辑合理顺畅

\item {} 
展现了基于用户选择的状态、页面视图和内容,更好地传达需求,用流程图来更好地表达产品逻辑

\item {} 
查漏补缺,检验是否有遗漏的分支流程

\end{enumerate}


\paragraph{组件}
\label{\detokenize{chapter_skill/flow_chart:id4}}
\begin{figure}[H]
\centering
\capstart

\noindent\sphinxincludegraphics{{flow_chart_part}.png}
\caption{组件}\label{\detokenize{chapter_skill/flow_chart:id16}}\end{figure}


\paragraph{分类}
\label{\detokenize{chapter_skill/flow_chart:id5}}
流程图以描述对象分类,包括:业务流程图、页面流程图、功能流程图、数据流程图等。作为产品,经常谈的是业务流程图;作为交互设计师,则比较关心页面流程图;而作为系统分析师,数据流程图最关键。
\sphinxhref{http://www.woshipm.com/pd/675174.html}{4}%
\begin{footnote}[196]\sphinxAtStartFootnote
\sphinxnolinkurl{http://www.woshipm.com/pd/675174.html}
%
\end{footnote}

流程图其实是传统的管理业务流程图,包含基本流程图和跨职能流程图(泳道图)两种。


\subparagraph{业务流程图(Transaction Flow Diagram, TFD)}
\label{\detokenize{chapter_skill/flow_chart:transaction-flow-diagram-tfd}}
“大象塞进冰箱”–分解

抽象地描述事物进行的次序和顺序,不涉及具体操作与执行细节。在互联网软件行业通常指脱离产品设计的用户行为流程。业务流程图是一种系统分析人员都懂的共同语言,
用来描述系统组织结构、业务流程。

以医院挂号流程为例:

\begin{figure}[H]
\centering
\capstart

\noindent\sphinxincludegraphics{{hospital_flow}.png}
\caption{医院挂号流程}\label{\detokenize{chapter_skill/flow_chart:id17}}\end{figure}


\subparagraph{绘制思路一般是:}
\label{\detokenize{chapter_skill/flow_chart:id6}}\begin{itemize}
\item {} 
首先将业务按阶段划分,比如电商类可以分为下单和支付,单车类可以分为提车、骑行和停车;

\item {} 
然后列出每个阶段参与的功能模块,比如下单阶段,就有商品查看、登录/注册、信息记录、个人中心等功能。

\item {} 
最后按照时间顺序,画出业务需求在各个功能模块之间的流转情况。

\end{itemize}

有两个原则:
\begin{itemize}
\item {} 
先思考主干流程,再思考分支流程,主干流程逻辑准确,分支流程全面无遗漏;

\item {} 
表达清楚后台产生的各种判断及相应的前端展示,这将作为接口设计的重要根据。

\end{itemize}

业务流程图重要的是描述谁在什么条件下做了什么事。

而页面流程图是具体到了网站、系统、产品功能设计的时候,表现页面之前的流转关系——用户通过什么操作进了什么页面及后续的操作及页面。


\subparagraph{页面流程图(Page Flow Diagram)}
\label{\detokenize{chapter_skill/flow_chart:page-flow-diagram}}
定义:指电子产品具体所呈现的页面跳转流程图。其承载了业务流程图所包含的业务流转信息。

页面流程图依然是包含在业务流程图的。这恰恰符合定义中的要求,同时也印证了页面流程图的正确性。

我们将抽象的业务,映射在了具象的页面上,用软件的页面承载起了业务需求。而以上就是由业务流程图到页面流程图的转化过程。

\begin{figure}[H]
\centering
\capstart

\noindent\sphinxincludegraphics{{service_flow_chart}.jpg}
\caption{电商购物的业务流程图}\label{\detokenize{chapter_skill/flow_chart:id18}}\end{figure}
\begin{itemize}
\item {} 
对于设计师或产品经理的好处:

\end{itemize}
\begin{enumerate}
\sphinxsetlistlabels{\arabic}{enumi}{enumii}{}{.}%
\item {} 
页面流程图一张页面助你讲完完整的用户与系统的交互故事,借助它,你更容易知道流程中的潜在地雷是什么,哪里的效率比较低,有助于系统化、全局化、周全性的思考

\item {} 
细化工作量的基础,通过页面流程图可准确评估需要多少张页面。

\item {} 
聚焦:页面流程图中的每个页面都不必追求精细——你的目标是规划行为路径,而不是单页面交互设计,所以完全无需考虑页面内容、布局。所以你会更加聚焦于用户目标和任务的完成。不必过早陷入细节。

\item {} 
关键是很快。线框图有可能有几十张,你画起来没那么快,而且一旦进入细节,则还需要慢慢深究。但是页面流程图也许就是几个小时的事情。你就可以对整个项目心中有数了。

\end{enumerate}
\begin{itemize}
\item {} 
对于开发工程师的好处:

\end{itemize}
\begin{enumerate}
\sphinxsetlistlabels{\arabic}{enumi}{enumii}{}{.}%
\item {} 
可作为评估工作量的重要依据——可帮助他们对工作量也心中有数。

\item {} 
可做为开展代码工作的重要参考——特别是前端开发,必须得知道每一种操作指向什么页面。

\item {} 
他们会映射功能逻辑,会给你更多好的建议。

\end{enumerate}


\subparagraph{功能流程图(Function Flow Diagram)}
\label{\detokenize{chapter_skill/flow_chart:function-flow-diagram}}
定义:指单页面内或多页面之间的功能操作流程,其包含在页面流程中。

任何功能都是被包含在页面内的,但一个页面内往往不止一个功能,所以单单页面流程图可能无法完整表达所有流程,而这时就需要用功能流程图来更加具体表达每个页面内所包含的功能。

相比于业务流程图,功能流程图的特点是:
\begin{itemize}
\item {} 
只展现用户的操作,不展现后台的判断;

\item {} 
只展现正常流程,不展现异常流程;

\item {} 
只可查看用户的工作流程,无法作为开发的参考。

\end{itemize}


\subparagraph{数据流程图(Data Flow Diagram)}
\label{\detokenize{chapter_skill/flow_chart:data-flow-diagram}}
定义:特指软件产品中,描述数据在不同节点被处理的过程所画的图表。主要表达计算机程序对于业务的实现原理。用户在功能流程图中的每一个操作,对应都会反映在数据流程图中。同时,数据流程图也可以叫程序流程图(Program
Flow Diagram)。

它是一种能全面地描述信息系统逻辑模型的主要工具。它可以利用少数几种符号综合的反映出信息在系统中的流动、处理和存储的情况。数据流程图具有抽象性和概括性。

每个流程图中都有一个核心伴随着不同操作在整个系统中不断流转。比如业务流程图大多以人为核心,每个节点都是在传递人的不同行为。而页面流程图和功能流程图也类似,都是以人的操作行为为核心,在不同页面和功能间进行流转。但数据流程图不同,它是以数据为核心,展示整个系统中,数据是如何被处理的。其更偏技术思维,更多的是展现后台程序的实现原理。所以,常常是开发人员绘制此图,而产品经理涉及较少。


\subparagraph{理解业务}
\label{\detokenize{chapter_skill/flow_chart:id7}}
分别展示了一个产品的业务流程、页面流程、功能流程和数据流程。从中可以发现,由业务到页面,再到功能,再到数据处理,是顺序拓展的。一个产品的页面或功能,不是凭空出现的,而是依据业务层的各个节点和流程进行设计的。这就是为什么在做产品设计时一定要先理解业务的原因。

尽量将业务、页面、功能和数据区分清楚,并且逐层递进,不要把多种类型的流程图混杂一起。这样反而会将思想搞得混乱。


\paragraph{颗粒度}
\label{\detokenize{chapter_skill/flow_chart:id8}}
流程图的细致程度。

我在画流程图时也常常会犹豫纠结,这个功能点用不用描写得更详细?这条分支用不用标出来?这个和服务器的交互事件用不用在流程图体现?等等这些问题,也都是产品经理在日常画图时会遇到的。


\paragraph{流程图的结构}
\label{\detokenize{chapter_skill/flow_chart:id9}}
流程图中大致包含四种结构:顺序结构、条件结构(又称选择结构)、循环结构。基本上大多数流程图都是由这三种结构组成的。


\paragraph{线框图 2\sphinxfootnotemark[197]}
\label{\detokenize{chapter_skill/flow_chart:id10}}%
\begin{footnotetext}[197]\sphinxAtStartFootnote
\sphinxnolinkurl{https://www.bilibili.com/video/BV1254y1D7Ht?from=search\&seid=14167562900175777805}
%
\end{footnotetext}\ignorespaces 
线框图只需要使用线条、方框和灰阶色彩填充,是低保真设计图。
*线框图主要呈现主体信息群,勾勒结构和布局,表达用户交互界面的主视觉和描述。
线框图是一种低保真且静态的呈现方式,产品经理通常使用纸笔来表达自己的想法。
包括:内容大纲:这个产品包含什么内容信息结构、布局:这个产品的内容怎么放用户交互界面:这个产品用户怎么操作


\paragraph{更多图}
\label{\detokenize{chapter_skill/flow_chart:id11}}
蜘蛛图 气泡图 散布图 用例图 信息架构图 线框图
实体关系图\sphinxhref{http://www.woshipm.com/pmd/3864.html}{5}%
\begin{footnote}[198]\sphinxAtStartFootnote
\sphinxnolinkurl{http://www.woshipm.com/pmd/3864.html}
%
\end{footnote}


\paragraph{案例}
\label{\detokenize{chapter_skill/flow_chart:id12}}

\subparagraph{分享购物车}
\label{\detokenize{chapter_skill/flow_chart:id13}}
“发起者”角度

\begin{figure}[H]
\centering
\capstart

\noindent\sphinxincludegraphics{{share_shopping}.png}
\caption{流程图}\label{\detokenize{chapter_skill/flow_chart:id19}}\end{figure}

节点分别是:
\sphinxhref{https://coffee.pmcaff.com/article/2714966199749760/pmcaff?utm\_source=forum}{3}%
\begin{footnote}[199]\sphinxAtStartFootnote
\sphinxnolinkurl{https://coffee.pmcaff.com/article/2714966199749760/pmcaff?utm\_source=forum}
%
\end{footnote}
\begin{enumerate}
\sphinxsetlistlabels{\arabic}{enumi}{enumii}{}{.}%
\item {} 
用户是作为起点,来开始;

\item {} 
抵达的第一个页面,是购物车;

\item {} 
在购物车,有“一键分享”的按钮;

\item {} 
点击完“一键分享”后,吊起商品选择确认页面;支持“取消”商品的勾选;

\item {} 
用户点击确认后,吊起好友筛选列表;

\item {} 
在好友筛选列表中,选中某一个特定的好友;弹出“确认”或“取消”按钮;

\item {} 
用户点击“确认”后,则把之前选择好的商品商品列表发给Ta;

\end{enumerate}


\subparagraph{登录注册流程图 4\sphinxfootnotemark[200]}
\label{\detokenize{chapter_skill/flow_chart:id14}}%
\begin{footnotetext}[200]\sphinxAtStartFootnote
\sphinxnolinkurl{http://www.woshipm.com/pd/675174.html}
%
\end{footnotetext}\ignorespaces 
\begin{figure}[H]
\centering
\capstart

\noindent\sphinxincludegraphics{{login_flow_chart}.png}
\caption{登录注册流程图}\label{\detokenize{chapter_skill/flow_chart:id20}}\end{figure}

一个大的流程就是由许多小流程(一个流程一个小模块)组成,每个小流程(常用的,每个App流程基本改动不太大的)可反复使用,提高工作效率,这就有点像面向对象的封装思想。


\subparagraph{AI落地}
\label{\detokenize{chapter_skill/flow_chart:ai}}
一个AI产品从需求到落地,大概需要经历以下环节:

需求分析→数据采集→数据清洗→数据标注→训练迭代→测试验证→交付模型→生产环境部署


\subparagraph{常见的绘制流程图的工具}
\label{\detokenize{chapter_skill/flow_chart:id15}}
(1)在线工具
\begin{itemize}
\item {} 
ProcessOn:\sphinxurl{https://www.processon.com/}

\item {} 
draw.io:\sphinxurl{https://www.draw.io/}

\end{itemize}

(2)客户端
\begin{itemize}
\item {} 
Microsoft Visio

\item {} 
edraw亿图

\item {} 
xmind

\item {} 
omniGraffle(mac)

\item {} 
StarUML

\end{itemize}


\subsubsection{TFD}
\label{\detokenize{chapter_skill/TFD:tfd}}\label{\detokenize{chapter_skill/TFD::doc}}
业务流程图(TFD)是一种描述管理系统内各单位、人员之间的业务关系,作业顺序和管理信息流向的图表。

主要功能

根据以下表格将产品的主要功能做一概括描述,也就是在拿到需求时,产品团队设计需求时所列出的对应模块以及每个模块下面的功能。

4.5 法务说明

凡涉及隐私权、知识产权、专利权、商标、服务条款(TOS)、版权、合同责任、客户沟通等方面需要

标明来源。(法务应提供协助)

4.6 性能需求

技术提供

4.7 安全需求

技术提供

4.8 兼容需求

技术提供

4.9 时间显示说明

举例:

六 附录

6.1 名词解释

6.2 原型

6.3 其它


\subsubsection{资源管理}
\label{\detokenize{chapter_skill/resource_manage:id1}}\label{\detokenize{chapter_skill/resource_manage::doc}}
资源管理(数据集、算法模型、策略、软硬件资源、需求管理)


\paragraph{产品研发规划1\sphinxfootnotemark[201]}
\label{\detokenize{chapter_skill/resource_manage:id2}}%
\begin{footnotetext}[201]\sphinxAtStartFootnote
\sphinxnolinkurl{http://www.woshipm.com/pmd/220940.html}
%
\end{footnotetext}\ignorespaces 
研发规划离我们就很近了,首先研发规划的输入应该是产品规划或产品年度规划,产品年度规划的内容需要通过产品研发规划落地。研发规划可以理解成大的项目计划,那自然应该包括项目范围和需求收集,需求分析和优先级的评估,研发资源的投入和成本,研发功能点,研发进度计划。如果存在多个子产品,还必须考虑研发产品组合规划,然后才是子产品的研发计划。

研发计划需要进一步细化到短周期的产品研发项目和版本,每一个版本有明确的项目范围和交付功能,有明确的资源申请和资源拖入,研发的每个版本最终通过立项后要严格按项目管理的方式进行计划,执行和跟踪。

研发规划里面另外一个重点就是技术规划,很多时候我们的研发规划和技术规划是放在一起的,但是不同的产品研发往往是涉及到相同的技术和平台。那么这部分应该抽取出来,根据产品开发方法论,本身也是包括了产品,平台和技术三个层面的内容。平台层和技术层如果足够强大,那么产品本身的灵活和可配置性也越强。

资源管理和管道管理在产品组合管理里面是一个重要内容,在研发规划里面一定要考虑到对资源的需求,特别是涉及到多产品规划的时候,资源冲突往往是一个很头疼的事情。而研发资源往往又不适合在同一时间兼顾多个产品或项目。资源规划本身包括了对资源技能的要求,时间的要求和数量的要求。这些都需要提前考虑到。


\paragraph{产品财务规划}
\label{\detokenize{chapter_skill/resource_manage:id3}}
产品财务规划是产品战略规划和商业模式中盈利模式分析和规划内容的进一步落地。财务规划必须回答问题是产品投入多大,需要多少成本?后续盈利模式如何,需要多久能够开始盈利。是否有进一步的可持续的盈利模式,还有哪些增值点,这些都必须考虑到。

资源一定要分阶段投入,产品也最好能够分迭代版本逐步退出。要知道产品研发周期越长,那么成本投入越大,企业本身的现金流压力也就越大。公司做产品规划,做产品的目的只有一个就是合法的赚取最大化的利润。这是很正常的一个事情,一方面是回报员工和股东,一方面是公司可持续长久发展。

产品财务规划首先是产品的投资回收期和内部收益率分析,其次是产品投入预算,预算的进一步分解等。有预算后续有执行,就容易进一步的按产品核算成本和收益。对于产品推出市场,又必须考虑产品的定价策略,产品定价策略需要考虑市场本身成熟度,客户关系积累,产品本身的发展多方面信息进行完善。


\subsubsection{数据分析}
\label{\detokenize{chapter_skill/data_analysis:id1}}\label{\detokenize{chapter_skill/data_analysis::doc}}

\paragraph{经营分析}
\label{\detokenize{chapter_skill/data_analysis:id2}}
用户画像系统的标签数据通过API进入分析系统后,可以丰富分析数据的维度,支持进行多种业务对象的经营分析。


\paragraph{数据类型:}
\label{\detokenize{chapter_skill/data_analysis:id3}}\begin{itemize}
\item {} 
用户数据分析

\item {} 
活动数据分析

\item {} 
流量数据分析

\item {} 
销售数据分析

\item {} 
内容数据分析

\item {} 
商品数据分析

\item {} 
订单数据分析

\item {} 
渠道数据分析

\end{itemize}


\subparagraph{用户数据分析}
\label{\detokenize{chapter_skill/data_analysis:id4}}
用户数量:
\begin{itemize}
\item {} 
新用户数

\item {} 
老用户数

\item {} 
新/老用户数量比;

\end{itemize}

用户质量:
\begin{itemize}
\item {} 
新增用户:第一次启动应用的用户;

\item {} 
每日新增用户 DNU(Daily New Users):每日应用中的新登入用户数量

\item {} 
新增独立用户:全体应用的新增用户的总和(去重)

\item {} 
活跃用户 AU(Active
Users):当天启动一次的用户即为活跃用户,含新用户和老用户;

\item {} 
活跃独立用户:当天应用的活跃用户总和(去重)

\item {} 
DAU:DAU(Daily Active
User)日活跃用户数量。常用于反映网站、互联网应用或网络游戏的运营情况。

\item {} 
MAU:MAU(monthly active users)月活跃用户人数。

\item {} 
用户参与度

\item {} 
沉睡

\end{itemize}

这些用户价值指标,会导向一个最终的产品指标——付费用户:
\begin{itemize}
\item {} 
客单价

\item {} 
PU ( Paying User):付费用户

\item {} 
APA(Active Payment Account):活跃付费用户数

\item {} 
ARPU(Average Revenue Per User) :平均每用户收入

\item {} 
ARPPU (Average Revenue Per Paying User): 平均每付费用户收入

\end{itemize}

\begin{figure}[H]
\centering
\capstart

\noindent\sphinxincludegraphics{{AARRR}.jpg}
\caption{AARRR}\label{\detokenize{chapter_skill/data_analysis:id13}}\end{figure}


\subparagraph{渠道数据分析}
\label{\detokenize{chapter_skill/data_analysis:id5}}
用户活跃:
\begin{itemize}
\item {} 
活跃用户:UV、PV

\item {} 
新增用户:注册量、注册同环比

\end{itemize}

用户质量:

留存:次日/7日/30日留存率
\begin{itemize}
\item {} 
用户留存率:在互联网行业中,用户在某段时间内开始使用应用,经过一段时间后,仍然继续使用该应用的用户,被认作是留存用户。这部分用户占当时新增用户的比例即是留存率,会按照每隔1单位时间(例日、周、月)来进行统计。

\end{itemize}

用户留存率中的40\sphinxhyphen{}20\sphinxhyphen{}10法则:如果你想让游戏、应用的DAU超过100万,那么日留存率应该大于40\%,周留存率和月留存率分别大于20\%和10\%。
\begin{itemize}
\item {} 
次日留存率:(当天新增的用户中,在往后的第1天还活跃的用户数)/第一天新增总用户数;

\item {} 
第2日留存率:(第一天新增用户中,在往后的第2天还有活跃的用户数)/第一天新增总用户数;

\item {} 
第7日留存率:(第一天新增的用户中,在往后的第7天还有活跃的用户数)/第一天新增总用户数;

\item {} 
第30日留存率:(第一天新增的用户中,在往后的第30天还有活跃的用户数)/第一天新增总用户数。

\end{itemize}

渠道收入:
\begin{itemize}
\item {} 
订单:订单量、日均订单量、订单同环比

\item {} 
营收:付费金额、日均付费金额、金额同环比

\item {} 
用户:人均订单量、人均订单金额

\end{itemize}


\subparagraph{流量分析}
\label{\detokenize{chapter_skill/data_analysis:id6}}\begin{itemize}
\item {} 
流量来源;

\item {} 
流量数量:UV、PV;

\item {} 
流量质量:浏览深度(UV、PV)、停留时长、来源转化、ROI(投资回报率,return
on investment);

\end{itemize}


\subparagraph{PV > UV:}
\label{\detokenize{chapter_skill/data_analysis:pv-uv}}\begin{itemize}
\item {} 
PV(访问量):即Page View,
具体是指网站的是页面浏览量或者点击量,页面被刷新一次就计算一次。如果网站被刷新了1000次,那么流量统计工具显示的PV就是1000
。

\item {} 
UV(独立访客):即Unique
Visitor,访问您网站的一台电脑客户端为一个访客。00:00\sphinxhyphen{}24:00内相同的客户端只被计算一次。

\end{itemize}

\begin{figure}[H]
\centering
\capstart

\noindent\sphinxincludegraphics{{UV}.png}
\caption{UV拆分}\label{\detokenize{chapter_skill/data_analysis:id14}}\end{figure}
\begin{itemize}
\item {} 
另外就是APP的埋点数据,这个功能的点击率是多少?这个功能有多少人打开,又有多少人使用了?有多少人在频繁使用这个功能?等等,这些埋点数据要时常关注。结合数据变化来反思功能设计的问题,从而优化产品。

\end{itemize}


\subparagraph{数据埋点}
\label{\detokenize{chapter_skill/data_analysis:id7}}
定义注入代码频段,一边网站分析工具能够准确捕捉用户行为

B端埋点工具:Google Analytics(GA)、百度统计

数据分析:
\begin{itemize}
\item {} 
基本分析:网站的全体用户、分群用户、个体用户的浏览行为进行全面准确地监控和分析,从而优化站点
内容,提高留存率、转化率等;可统计:访客来源、设备信息、访客属性、页面访问量、停留时长、流转去向等

\item {} 
桑基图(能量分流图)

\item {} 
概要的迅速观察用户的整体访问路径和习惯,以及在哪些页面、什么情况下用户会终端访问

\item {} 
Cohort分析图(队列分析):留存分析法

\item {} 
访客分析

\item {} 
客观分析全面的用户行为数据

\item {} 
热力图

\item {} 
页面不同区域的热度图表

\end{itemize}

\sphinxstyleemphasis{B端和C端数据埋点的区别}

诉求
\begin{itemize}
\item {} 
B端(尤其业务系统):观察研究用户对各项产品功能的接受程度、使用情况、用户操作习惯等,进一步评估功能是否合理,能否帮助用户提升效率

\item {} 
C端:提升用户体验,细致的、全面的数据埋点

\end{itemize}

方案
\begin{itemize}
\item {} 
B端:web埋点(URL访问、跳转、按钮点击、文本框录入)

\item {} 
C端:app(交互行为进行细致的埋点,全面掌握用户的动作)

\end{itemize}


\subparagraph{产品数据分析}
\label{\detokenize{chapter_skill/data_analysis:id8}}\begin{itemize}
\item {} 
搜索功能:搜索人数/次数、搜索功能渗透率、搜索关键词;

\item {} 
关键路径漏斗等产品功能设计分析;

\end{itemize}


\paragraph{警惕指标作弊}
\label{\detokenize{chapter_skill/data_analysis:id9}}\begin{itemize}
\item {} 
DAU(日活跃用户数):买垃圾流量,做各种不靠谱的活动。

\item {} 
下载量:虚假宣传,夸大产品价值。

\item {} 
注册用户数:不考虑留存的注册返现。

\item {} 
活跃度:在“分子 / 分母”的公式上做文章,在分子、分母的定义上玩花样。

\item {} 
人均 PV:一篇文章分 N 页,人均停留时间也类似。

\item {} 
点击率:软件下载站上,各种花花绿绿的“下载”按钮,点好几次也不一定能点到真的下载链接。

\item {} 
使用时长:后台运行,或者故意“迷惑”用户,让用户无法快速完成任务。

\item {} 
付费用户数:首单 1 分钱。

\item {} 
复购率:首单 9 块 9,第二单 1 毛。

\item {} 
不只是制订指标的人,哪怕经常完成指标的你,也一定对上面这些投机取巧的做法深恶痛绝。但人性使然,我们不能去正面挑战它。

\end{itemize}

真正的成功指标可以反映出用户的“非受迫、无诱导的成功行为”。衡量指标要在执行开始前制订,而不是过程中根据“做的情况”调整。如果没有重大变化,不可以不断调整目标
\begin{itemize}
\item {} 
非受迫:用户没有被逼着做没价值的事情,比如有些 App
里的签到才能获得某个价值;

\item {} 
无诱导:用户的行为不是“奖励就有,没奖励就没有”,比如有红包才会转发;

\item {} 
成功行为:指的是指标考察的行为,本身就为用户创造了价值,而不只对公司有价值。

\end{itemize}


\subparagraph{商品数据分析}
\label{\detokenize{chapter_skill/data_analysis:id10}}\begin{itemize}
\item {} 
商品动销:GMV、客单价、下单人数、取消购买人数、退货人数、各端复购率、购买频次分布、运营位购买转化;

\item {} 
商品品类:支付订单情况(次数、人数、趋势、复购)、访购情况、申请退货情况、取消订单情况、关注情况/;

\end{itemize}


\subparagraph{订单数据分析}
\label{\detokenize{chapter_skill/data_analysis:id11}}\begin{itemize}
\item {} 
订单指标:总订单量、退款订单量、订单应付金额、订单实付金额、下单人数;

\item {} 
转化率指标:新增订单/访问UV、有效订单/访问UV;

\end{itemize}


\paragraph{AI 数据}
\label{\detokenize{chapter_skill/data_analysis:ai}}
从“数据”这个角度来说,从收集(TTS,3个月)、分析(看大量聊天对话数据,才能自己提炼规则feature)、应用(产品早期,数据的价值甚至大过技术模型算法)到测试(产品需求、TE测试、用户使用,数据集都是不一样的,越来越不可控)等等,每个环节都有很大不同。

从结果看,即使是大公司中级产品经理(总监级),也至少3\sphinxhyphen{}6个月来适用AI产品工作,甚至都很难有自己真正独到而深入的理解认知


\subparagraph{和基线比较 5\sphinxfootnotemark[202]}
\label{\detokenize{chapter_skill/data_analysis:id12}}%
\begin{footnotetext}[202]\sphinxAtStartFootnote
\sphinxnolinkurl{https://neal-lathia.medium.com/machine-learning-for-product-managers-ba9cf8724e57}
%
\end{footnotetext}\ignorespaces 
我们通常孤立地看待早期/MVP产品:我们只做一件事,然后把它推出去看顾客的反应。

ML产品是不同的,因为性能总是相对的——即使是第一次迭代。例如,如果您的高级ML算法是95\%的准确性,但您的简单基线是94\%的准确性,那么您投资了大量的工作,以获得1\%的收益。另一方面,如果您的ML算法的准确率是75\%,但简单的基线是50\%,那么您已经取得了巨大的飞跃。

这里有两点很重要:首先,性能总是相对于某些东西:您需要一个基线(baseline)。其次,为了能够进行比较,您需要定义良好的指标。

在ML产品中,这些指标通常分为离线指标(例如,“算法预测历史数据的准确性有多高?”)和在线指标(例如,“当我们使用这种算法部署产品时,我们能获得多少转化率?”)。


\subsubsection{管理项目}
\label{\detokenize{chapter_skill/project_manage:id1}}\label{\detokenize{chapter_skill/project_manage::doc}}
产品和开发的是同样的团队和同样的人,但在驱动产品和驱动项目这两件事情上,最好还是有所差别。至少产品更关注的是产品、功能、方向和反馈;而项目则更关注进度、质量和测试等。
\begin{itemize}
\item {} 
做好评估。几乎所有项目最终未按计划执行,其最根本原因就是在项目开始阶段,没有对需求、技术、产品有足够充分的了解,也就没有后续开发中的可控力度。高估和低估都是有问题的,所以我们常用的做法就是非常重视前期的评估,宁愿多花时间,并且对有模糊边界或者有挑战的问题,留足buffer。

\item {} 
将计划落实到可执行的单元和可执行的人。有了评估,然后就是将计划落实到足够力度的任务,以任务驱动开发过程,任务落实到责任人,任务要标明截止日期。在此,通过一定的工具来管理,是十分必要而可控进度的。例如我们基于自主产品PingCode
的任务驱动方式,就可以很好的将开发计划落实到任务和可执行的人,以直观的方式来告诉负责人项目整体的状态、执行者的情况、被delay的事情有哪些。总之,工具的辅助需要团队开发想法的驱动。(再多说一句:PingCode不仅可以进行以上的项目管理,还覆盖了项目、任务、需求、缺陷、迭代规划、测试、目标管理的研发管理全流程。)

\end{itemize}


\subsubsection{Research}
\label{\detokenize{chapter_skill/research:research}}\label{\detokenize{chapter_skill/research::doc}}

\paragraph{工具}
\label{\detokenize{chapter_skill/research:id1}}
谷歌学术、MIT/卡梅隆/北大清华图书馆资源库、公司内部知识库(如商汤/阿里)


\paragraph{博客}
\label{\detokenize{chapter_skill/research:id2}}\begin{itemize}
\item {} 
\sphinxurl{https://medium.com/}

\item {} 
\sphinxurl{https://towardsdatascience.com/}

\item {} 
\sphinxurl{https://ai.googleblog.com/}

\item {} 
\sphinxurl{https://deepmind.com/blog/?category=research}

\end{itemize}


\paragraph{论文}
\label{\detokenize{chapter_skill/research:id3}}

\subparagraph{顶级会议}
\label{\detokenize{chapter_skill/research:id4}}\begin{itemize}
\item {} 
NLP:ACL / NAACL / EMNLP …

\item {} 
CV: CVPR / ICCV / ECCV …

\item {} 
ML: ICML / ICLR / NIPS / AAAI …

\item {} 
Data Mining \& Information Retrieval: WWW / KDD / IJCAI …

\end{itemize}


\subparagraph{Sci\sphinxhyphen{}Hub}
\label{\detokenize{chapter_skill/research:sci-hub}}
\sphinxurl{http://www.sci-hub.io/}

备用站点:\sphinxurl{http://www.sci-hub.cc/}

中国版以及备用站点:\sphinxurl{http://www.sci-hub.cn/}、\sphinxurl{http://www.sci-hub.xyz/}


\subparagraph{谷歌学术}
\label{\detokenize{chapter_skill/research:id5}}
谷歌学术网址,\sphinxurl{http://scholar.glgoo.org/}、\sphinxurl{https://xs.glgoo.net/}、\sphinxurl{http://scholar.hedasudi.com/}

也有镜像网站合集http://www.dirmor.com/


\subparagraph{Library Genesis}
\label{\detokenize{chapter_skill/research:library-genesis}}
Library
Genesis号称是帮助全人类知识无版权传播的计划。网站上论文很多,下载方便,还有很多外文书籍和中文书籍,几乎每天都在更新。\sphinxurl{http://gen.lib.rus.ec/}


\subparagraph{arXiv}
\label{\detokenize{chapter_skill/research:arxiv}}
www.arxiv.org

arXiv,
是archive(归档)的意思,是一个由康乃尔大学维护的免费的多学科论文预出版(preprint)数据库。所谓预出版,就是说论文还没有经过同行评审,文责自负,文章质量参差不齐,所以一般不会作为正式的学术成果。不过有的学科习惯上先把文章公开到arXiv上,然后再提交到会议上。
arXiv 是我们搜索、浏览和下载学术论文的重要工具。近 30 年来,arXiv
为公众和研究社区提供了开放获取学术论文的服务。这些论文涉及物理学的庞大分支和计算机科学的众多子学科,如数学、统计学、电气工程、定量生物学和经济学等等。

Chrome extension that adds video explanations to research papers on
arxiv.org: \sphinxurl{https://github.com/amitness/papers-with-video}


\subparagraph{arxiv\sphinxhyphen{}sanity}
\label{\detokenize{chapter_skill/research:arxiv-sanity}}
\sphinxurl{http://www.arxiv-sanity.com/}

感兴趣相关度排序、个人图书馆、推荐系统、都在看什么\sphinxhref{https://cloud.tencent.com/developer/article/1473703}{2}%
\begin{footnote}[203]\sphinxAtStartFootnote
\sphinxnolinkurl{https://cloud.tencent.com/developer/article/1473703}
%
\end{footnote}

\sphinxurl{https://arxiv.xixiaoyao.cn/}


\subparagraph{semanticscholar3}
\label{\detokenize{chapter_skill/research:semanticscholar3}}
\sphinxurl{https://www.semanticscholar.org/}


\subparagraph{metacademy}
\label{\detokenize{chapter_skill/research:metacademy}}
\sphinxurl{https://metacademy.org/}

metacademy看作一副机器学习和人工智能的知识图谱,在上面搜索任意机器学习或者人工智能的知识概念,它会告诉你学习这个知识点需要什么前置知识,需要多长时间掌握,并罗列出相关的课程。


\subparagraph{AMiner3}
\label{\detokenize{chapter_skill/research:aminer3}}\label{\detokenize{chapter_skill/research:id6}}
输入关键词就能找到对应的学者: \sphinxurl{https://www.aminer.cn/}

\sphinxurl{https://www.aminer.cn/ai2000/country/China}


\subparagraph{论文}
\label{\detokenize{chapter_skill/research:id7}}
\sphinxurl{https://openreview.net/}

CV领域Paper论文常见单词(一) \sphinxhyphen{} 王大东的文章 \sphinxhyphen{} 知乎
\sphinxurl{https://zhuanlan.zhihu.com/p/58860096}

\sphinxurl{https://ying-zhang.github.io/misc/2016/we-love-paper/}


\paragraph{代码}
\label{\detokenize{chapter_skill/research:id8}}

\subparagraph{Papers with Code1\sphinxfootnotemark[204]}
\label{\detokenize{chapter_skill/research:papers-with-code1}}%
\begin{footnotetext}[204]\sphinxAtStartFootnote
\sphinxnolinkurl{https://www.jiqizhixin.com/articles/2020-10-09-5}
%
\end{footnotetext}\ignorespaces 
机器学习资源网站 Papers with Code
自创立以来,凭借丰富的开放资源和卓越的社区服务,成为机器学习研究者最常用的资源网站之一。网站还根据细分领域的指标对论文进行了整理和排序。以图像配准(Image
Registration)为例。指标、论文、代码,安排得明明白白。

2019 年底,Papers with Code 正式并入 Facebook
AI。最近,它又有了新举措:与论文预印本平台 arXiv 展开合作,支持在 arXiv
页面上添加代码链接。

Browse State\sphinxhyphen{}of\sphinxhyphen{}the\sphinxhyphen{}Art: \sphinxurl{https://paperswithcode.com/sota}


\subparagraph{Github \sphinxhyphen{} Awesome3}
\label{\detokenize{chapter_skill/research:github-awesome3}}
以 awesome 为名,进行某个领域资料的高质量整合。


\subparagraph{实践}
\label{\detokenize{chapter_skill/research:id9}}
推荐网站:\sphinxurl{https://www.kaggle.com/}


\paragraph{mathpix}
\label{\detokenize{chapter_skill/research:mathpix}}
手写/截图 转 LaTex公式:\sphinxurl{https://mathpix.com/}

LaTex如果所有公式都要自己手打还是很痛苦的。(虽然很多时候一篇Deep
Learning方向的paper公式数量只有十个左右(这还是在强行加上LSTM等被翻来覆去写烂的公式的情况下))

\sphinxurl{http://deepdive.nn.157239n.com/}


\paragraph{报告}
\label{\detokenize{chapter_skill/research:id10}}
\sphinxurl{http://www.zft-park.com.cn/index.php?m=Article\&a=show\&id=384}


\subsubsection{估值}
\label{\detokenize{chapter_skill/Valuation:id1}}\label{\detokenize{chapter_skill/Valuation::doc}}

\paragraph{尽职调查}
\label{\detokenize{chapter_skill/Valuation:id2}}
上裁判文书网、执行信息网、信用中国扒个涉诉涉执行情况


\paragraph{上市1\sphinxfootnotemark[205]}
\label{\detokenize{chapter_skill/Valuation:id3}}%
\begin{footnotetext}[205]\sphinxAtStartFootnote
\sphinxnolinkurl{https://www.bilibili.com/video/av21295743/}
%
\end{footnotetext}\ignorespaces 
问:中国大型的互联网公司为什么纷纷跑到海外上市呢?下面请中国法学理事会理事李可书博士为大家解答一下境内外上市的区别在哪里。

  李可书:我们注意到中国大型的互联网公司,往往选择赴海外上市,而且美国上市是优选的,这是一个目前很常见的现象。
  那么,我们再看为什么,比如说阿里,它为什么选择在美国上市,这是一个我们需要去思索的点,首先我想分析一下它为什么选择在美国上市的原因。第一个原因,是因为在A股上市准入门槛比较高,规定的条件比较多,这种条件因为阿里满足不了,首先第一个条件,我国的《证券法》和《公司法》规定的非常明确,在A股上市的公司必须是依照法律规定,在中国境内设立的股份有限公司。而阿里巴巴呢,是注册在海外的,它这种外资的身份,使得它不满足A股上市的要求,这是第一个原因,因为A股要求必须注册在国内的企业。
  第二个要求,它也满足不了,是什么呢?关于A股上市的发行对象的要求,对于A股的上市来说,要求将股东的人数控制在200人,大家知道阿里之前大量的通过职工持股,收了自己员工大量的股票,如果需要将股东人数进行压缩的话,那么需要大量清理职工持股,这也是阿里不希望做到的。所以呢,基于这两点法律的要求,阿里很难去完全满足,这是第一个阿里当时考虑在美国上市的原因之一。
  第二个原因,是因为审核的时间,因为在国内上市呢,我们国内目前审核制,注册制没有正式实行之前一直是审核制,审核制决定了我们的审核是需要时间的,正常是两到三年,这个时间对于大量的互联网公司,包括阿里在内的大量互联网公司他们觉得时间太长,耗不起这个时间,而在国外呢,比如在美国上市,实际上是非常快的,一般来讲是半年左右就完成上市。所以从时间上考虑,在美国上市,这种境外上市,比在A股上市来说有一定的吸引力。这个是阿里当时选择在美国上市的第二个原因。
  第三个原因,是关于A股对于公司的盈利有明确的财务指标的要求,包括阿里在内的京东也一样,大家知道京东前几年一直没有盈利,所以它无法满足A股上市的要求,这是第三个原因。
  所以基于这些原因,导致包括阿里、京东这些大量的互联网公司选择在境外上市,当然未来他们是否选择回归,这可能是未来我们需要考虑的点,在我国的这种A股市场的制度进行变更之后。


\paragraph{商汤科技}
\label{\detokenize{chapter_skill/Valuation:id4}}
成立于2014年的商汤科技,仅仅3年时间估值就暴涨到20亿美金。2017年7月,商汤科技宣布完成4.1亿美元B轮融资,创下当时全球人工智能领域单轮融资额记录。

2018年4月,商汤科技完成阿里巴巴集团领投的6亿美元C轮融资,再次创下全球人工智能领域融资记录;一个月后,商汤科技再度获得6.2亿美元C+轮融资;三个多月后,商汤科技再度获得软银10亿美金的融资,估值也飙升至60亿美金。

现在绝大部分技术型的、平台型的公司还是一To
B的场景,但投资机构却把它们当作To
C的公司来投。这样的公司,后续还需要多轮的融资支持成长。如果天使轮一下子把估值做到1亿,那A轮总得3亿,做到F轮怎么办?


\paragraph{Labby Inc2\sphinxfootnotemark[206]}
\label{\detokenize{chapter_skill/Valuation:labby-inc2}}%
\begin{footnotetext}[206]\sphinxAtStartFootnote
\sphinxnolinkurl{https://www.labbyinc.com/labbys-ai-enabled-optical-sensing-technology-secures-usd-480-000-seed-investment}
%
\end{footnotetext}\ignorespaces 
Labby Inc, an early\sphinxhyphen{}stage startup specializing in AI\sphinxhyphen{}enabled optical
sensing solutions for raw milk testing, today announced it has raised
\$480,000 in seed funding. AgriTech Capital, a strategy and investment
firm specializing in innovation and technology in the agribusiness
sector.

The global dairy farming industry loses \$32 billion annually due to
mastitis infections. At a minimum, twenty\sphinxhyphen{}five percent of cows each year
are impacted regardless of how well managed a farm is. Farmers lack a
way to quickly and easily identify mastitis at an early stage so they
can take preventative measures to reduce the impact on yields. With
Labby’s solution, farmers and dairy processors finally have a way to
quickly and easily test raw milk gaining visibility into animal health,
milk quality, and feed efficiency, enabling them to optimize their
operations.


\subsubsection{产品定价 1\sphinxfootnotemark[207]}
\label{\detokenize{chapter_skill/price:id1}}\label{\detokenize{chapter_skill/price::doc}}%
\begin{footnotetext}[207]\sphinxAtStartFootnote
\sphinxnolinkurl{http://www.woshipm.com/pd/3817590.html}
%
\end{footnotetext}\ignorespaces 
商业化产品的定价链条如下图所示。将定价链条的上下游关联在一起形成的核心定价工作主要有三项:第一,制定计费方式;二,制定产品价格;三,产品SKU管理。


\paragraph{计费}
\label{\detokenize{chapter_skill/price:id2}}
计费方式是产品定价不可分割的一部分,并且是产品定价的基础。不同类型的商业化产品的计费方式不同,比如实物产品、软件产品、广告产品、内容产品、会员产品、数据产品等。

需要说明的是,在各类产品越来越精细化的价格管理背后,计费方式的管理也更加精细化,既包括内部计费控制,又包括给予客户的计费说明。下图所示为融云的计费方式,其中每个计费标准都通过独立页面做单独说明(不做展示)。


\subsection{experience}
\label{\detokenize{chapter_experience/index:experience}}\label{\detokenize{chapter_experience/index:chap-exper}}\label{\detokenize{chapter_experience/index::doc}}

\subsubsection{职业发展路径}
\label{\detokenize{chapter_experience/career_path:id1}}\label{\detokenize{chapter_experience/career_path::doc}}
\sphinxstylestrong{懂用户、会权衡}


\paragraph{什么样的人适合做产品经理?}
\label{\detokenize{chapter_experience/career_path:id2}}
1、喜欢体验各种新鲜事情,遇到不喜欢的设计有自己独特的想法;2、逻辑清晰,能用简短的语言描述复杂的事物;3、喜欢与人沟通,有很强的的同理心去揣测身边人的心理活动;4、对新奇的事物有新鲜感,喜欢追根刨底的问问题;5、喜欢发起辩论,用自己的思考和逻辑说服别人;6、注重细节体验,能从细节中发现问题,并对细节瑕疵不能忍。


\paragraph{实习生}
\label{\detokenize{chapter_experience/career_path:id3}}
理解产品可以从理解基础技术开始。

这里提到的基础技术包括了客户端技术分类,例如Android、iOS、H5或者微信小程序:它们各自的技术特点是什么。比如为什么开发Android和iOS应用的是两个不同的技术职能,而开发H5的是一个技术职能。

另外,产品实习生需要对产品底层的技术通信原理进行理解。

比如当我们使用客户端产品发送一条消息时,这条消息经过了哪些环节后被另一个客户端收到——这里就涉及了什么是服务端,客户端和服务端的主要职能和通信机制是什么。先从大局观上对互联网产品的技术框架有一个基本认知。

建立互联网产品技术认知,划分清楚技术职能,了解各技术的特点和应用场景,就能胜任最基本的产品工作了。

产品助理的本职: \sphinxhref{http://www.woshipm.com/pmd/415296.html}{4}%
\begin{footnote}[208]\sphinxAtStartFootnote
\sphinxnolinkurl{http://www.woshipm.com/pmd/415296.html}
%
\end{footnote}
\begin{enumerate}
\sphinxsetlistlabels{\arabic}{enumi}{enumii}{}{.}%
\item {} 
需求文档:将需求整理成规范的文档

\item {} 
框线图的完善:将草图,概念图,补充成为整套的框线文件

\item {} 
材料收集:收集市场,运营,案例等产品经理所需要的一些材料

\item {} 
基础的用户调研:收集用户的反馈信息,观察用户基于产品,基于市场产生的言行,并及时向产品经理反馈

\item {} 
反馈技巧:助理必须要学会的是如何向产品经理良好沟通反馈,

\end{enumerate}

晋升能力则是以下四个技能
\begin{enumerate}
\sphinxsetlistlabels{\arabic}{enumi}{enumii}{}{.}%
\item {} 
团队沟通:与设计,研发,QA,运营等团队中的其他角色能进行针对项目的良好沟通,形成讨论氛围

\item {} 
进度跟进:能清楚掌握到项目的执行进度,已经完成情况,剩余时间,延期风险评估。

\item {} 
产品发布:着手了解各个渠道,平台的产品发布规则,准备相应的发布资料

\item {} 
小模块的独立策划:这个环节需要的不仅仅是创新思维,更重要的是在一个小模块的环境下,去探索产品思维,要知道产品思维是个很全很宽的面性思维

\end{enumerate}


\paragraph{产品经理}
\label{\detokenize{chapter_experience/career_path:id4}}
过年的时候,大家会在微信收发红包,微信红包就是一个具体的功能模块,如果你在微信做产品经理,那或许就要从负责一个功能模块开始历练了。

要能建立完整的技术基础概念认知,能从技术角度对产品方案进行初步评估和判断。

面试考核的重点:
\begin{itemize}
\item {} 
执行力:初级产品经理最重要的就是执行力,因为大部分的情况下,产品的大方向不由他控制,只负责局部的数据,用户需求往往比较明显,所以对于需求的把握能力要求并不高,能深度的做好用户调研和反馈,快速的迭代并提升数据就可以了,而以上的这些,就要求应聘者有强大的内驱力,可以有力的推动项目内成员达成目标。

\item {} 
综合能力:以逻辑能力、沟通表达能力为主,逻辑能力是PM安家立命之本,对于初级产品经理来说,能不能理清楚功能模块和整个产品的关系非常重要,除此之外,功能的设计和迭代的节奏,也非常考验产品经理的逻辑能力,一个页面会遇到几种使用场景?不同场景之间的关系是什么?如何让一个页面同时满足多种入口和多种需求?没有优秀的逻辑,处理这些问题的时候,就会有纰漏。

\item {} 
交互设计:国内很多的一线互联网企业都有专业的交互设计师,相处过很多tx的PM,都会在入司后问到交互设计师在哪?但个人认为,PM应该兼顾交互设计师的工作,特别是初创型企业,大部分都没有专职的交互设计师。对于初级产品经理来说,可以把单个模块的交互做完整,输出整洁、清晰的产品需求交付物就算合格了,面试官可以让面试者带一些相关的设计产出,并当面提问,面试的效果就比较好。

\end{itemize}


\paragraph{高级产品经理}
\label{\detokenize{chapter_experience/career_path:id5}}
如果你从产品经理提升为高级产品经理,将会负责微信整个支付功能,也就是一条产品线,除了微信红包,还有涉及到支付的其他功能,比如钱包、收付款等模块。

面试考核的重点:
\begin{itemize}
\item {} 
需求把控能力:这个阶段的产品经理,往往是企业招聘回来之后负责新产品的,那么对于需求的把控能力就非常的重要,把控不单单是指理解,还要包括控制,好的产品是有节奏的,特别是涉及多个部门的资源和排期,很有一种带着镣铐跳舞的感觉。
如果是我面试这部分的产品经理,我会直接问他的产品经历,重点推敲几个核心逻辑
他的产品经历,重点推敲几个核心逻辑
1、“为什么要做这个产品,需求是什么?” 2、“用户的核心场景是怎样的?”
3、“做起来之后,对业务线有什么价值?”

\item {} 
\sphinxstylestrong{资源协调、项目推动能力}:带独立的产品,和做模块是不一样的,做一个小模块,评审通过,点对点找开发沟通就可以了,但是独立的产品包含的是一整个打包的功能List,其中涉及的开发量也往往不是一个开发可以完成的,而前后端的对接,各种语言的通讯等细节都决定了排期和节奏,这些对于一个产品经理的资源协调能力要求很高,定什么里程碑,开发之间要什么时候对接,测试什么时候进行,版本回滚的机制和风险方案,这些都是考验一个产品经理资源协调,项目推动能力的地方。

\end{itemize}


\paragraph{产品总监}
\label{\detokenize{chapter_experience/career_path:id6}}
当你从高级产品经理晋升为产品总监,你就不只需要负责微信支付产品线,还要肩负微信涉及到移动支付领域的整体工作。微信支付涉及移动支付领域的工作不只是微信内部的产品上线和协调工作,还涉及到外部协调和对接,比如说与金融机构的协调。(根据百度百科的定义:移动支付是指移动客户端利用手机等电子产品来进行电子货币支付,移动支付将互联网、终端设备、金融机构有效地联合起来,形成了一个新型的支付体系。)

对于高阶产品经理,能从业务角度和产品发展角度对技术架构进行预判,能掌握新技术的基本原理并加以运用到产品和业务中,是产品综合实力的一种体现,能做出在时间、资源、效率上最优的产品决策。


\paragraph{事业部负责人}
\label{\detokenize{chapter_experience/career_path:id7}}
除了要具备产品总监的能力还要懂运营和渠道、资金和财务,对业务业绩负责;


\paragraph{产品副总裁}
\label{\detokenize{chapter_experience/career_path:id8}}
如果你从产品总监,升为产品副总裁,那就需要负责微信产品部门的整体工作,不只包括微信支付,还有小程序、微信公众平台、微信广告等。


\paragraph{产品CEO}
\label{\detokenize{chapter_experience/career_path:ceo}}
在整个产品经理职业发展路径中,如果你最后担任产品CEO角色,就像张小龙,不仅负责整个微信产品部门,还会负责腾讯的其他产品或业务,比如说FoxMail(QQ邮箱)。

这个层次需要的是资源整合能力、管理能力以及对商业的精准判断。

\begin{center}\sphinxincludegraphics{{path}.jpg}\end{center} \sphinxincludegraphics{{PM_top}.jpg}


\paragraph{分类}
\label{\detokenize{chapter_experience/career_path:id9}}\begin{itemize}
\item {} 
执行类产品经理:指只掌握需求生产能力的产品经理;

\item {} 
筹划类产品经理:指开始参与市场工作的产品经理。

\end{itemize}

\begin{figure}[H]
\centering
\capstart

\noindent\sphinxincludegraphics{{PM_class}.png}
\caption{产品经理能力\sphinxhref{http://www.woshipm.com/pmd/2466877.html}{5}\sphinxfootnotemark[209]}\label{\detokenize{chapter_experience/career_path:id13}}\end{figure}
%
\begin{footnotetext}[209]\sphinxAtStartFootnote
\sphinxnolinkurl{http://www.woshipm.com/pmd/2466877.html}
%
\end{footnotetext}\ignorespaces 
\sphinxstylestrong{对比程序员的成长路径}
\begin{quote}
\begin{quote}\begin{description}
\item[{width}] \leavevmode
400px

\end{description}\end{quote}
\end{quote}

\begin{figure}[H]
\centering
\capstart

\noindent\sphinxincludegraphics{{coder_path}.png}
\caption{coder path}\label{\detokenize{chapter_experience/career_path:id14}}\end{figure}


\paragraph{了解产品流程 2\sphinxfootnotemark[210]}
\label{\detokenize{chapter_experience/career_path:id10}}%
\begin{footnotetext}[210]\sphinxAtStartFootnote
\sphinxnolinkurl{http://www.woshipm.com/zhichang/906380.html}
%
\end{footnotetext}\ignorespaces 
对于一年以下产品经验的应届生,我会让他开始独立做运营类的需求,一般这样的需求比较简单,涉及的关联系统也会单一,对核心业务的要求也没那么高,逻辑思维上也比较简洁,这也是他了解产品流程,业务流程最快的方式,而且运营类活动活动周期短,反馈快,他能快速知道自己的不足之处,快速提升产品思维,数据意识和沟通效率,快速高效的反馈,是其快速成长的关键。


\paragraph{职级晋升 3\sphinxfootnotemark[211]}
\label{\detokenize{chapter_experience/career_path:id11}}%
\begin{footnotetext}[211]\sphinxAtStartFootnote
\sphinxnolinkurl{https://www.yuque.com/weis/pm/lto95c}
%
\end{footnotetext}\ignorespaces 
晋升和职级标准制定的理性目标应该是为公司发展服务。

最合理的标准需要考虑公司内部业务和人才的现状、未来发展预期,来决定公司未来一段时间应该侧重激励什么。比如侧重短期绩效,则人人争先,短期内公司会有较强的战斗力;如果注重潜力,优先选拔高潜年轻人,则对公司的长期竞争力有利;如果注重专业能力,则公司的产品质量或技术含量会领先;如果注重协调沟通和文化价值观,则公司的组织能力和大规模作战能力会有优势。

公司制定晋升和职级标准,还要考虑内部的文化历史惯性和理解能力,以及外部大众的接受度,考虑在相关人才市场上的稀缺性和企业的竞争力。兼顾了上述约束条件,还最有利于公司短、中、长期发展目标的,才是理性的晋升和职级标准。

产品经理绩效的定义可以差别很大,体验、收入、增长、创新、进度、效率、产品架构设计、组织建设、业务方满意度等均可作为判断标准,收入还可以分为侧重短期数字指标和长期总收入最大化。对产品经理能力的定义也可以差别很大,专业能力、业务能力、管理能力就是三种完全不同的发展方向,但它们都可能创造巨大价值,所以要把合适的人放在合适的岗位上。

资深产品经理的级别升高,在企业里越来越重要,他的素质、潜力、品性的重要性(相对专业能力)会越来越高,这是因为高阶产品经理通常是一个中枢岗位,要协调很多团队间的工作,要权衡很多员工和很多用户间的利益分配。
有些人的职级高,可能是因为他负责产品的业务规模大,或者团队规模大,或者给边缘业务的优待(边缘业务难吸引优秀人才,需要额外福利)。这样的晋升明规则或潜规则本身没有错,是符合企业利益的,但总会有聪明人会钻漏洞,比如拼命地招人以扩大团队规模,或者拼命做大业务规模以追求不健康的增长(一般是不计
ROI 的高额营销资源投入,或透支公司整体的品牌口碑)。

职级晋升看重领域经验、工龄、履历背景的企业也是有的,如果追求业务稳定发展,这也没什么错。还有些情况是因为稀缺性,某些人才很稀缺,就容易获得更高的薪酬和级别。还有些情况是,员工被猎头或朋友诱惑得到了好的工作机会,想离职,那么企业为了挽留他而给他加薪升级是很常见的。也有些公司的薪酬级别对应关系较严格,有的部门要招进某个高薪人才,就会给他申报更高职级。也有些人因为项目烂尾(不是他的过错)补偿晋级,或者被调去边缘岗位而补偿晋级。还有一些职级错配的原因,可能是评审有随机性,或者某人是擅长做
PPT
的演讲型选手,或做出把他人的业绩说成是自己业绩的作弊行为,或者领导强推特批帮助晋升等。


\paragraph{空降}
\label{\detokenize{chapter_experience/career_path:id12}}
空降高阶产品经理,成功率天然就是低的。这是因为,产品经理这个职业既需要纵向深入理解业务,又需要横向跟很多团队深度协作,所以空降高阶人员天然就要付出很高的熟悉成本和磨合成本。产品经理做决策还无法都用数据和事实说话,必须依赖知识和数据背后的判断和理念,而空降新人不可能与原有团队总是达成共识,这也使得基层产品经理遇到上级换人和技术运营搭档换人时,如同跳槽一样难以适应。于是,空降高阶产品经理的常见结果就是走一批原来的下属产品经理。只有在这几种情况下,空降高阶产品经理的成功率会高一些:任务是复制一个产品;开始一个新产品;灾后重建,原产品出了大问题,人心思变;有巨大新要素成熟,给产品带来创造巨大新价值的机会。


\subsubsection{产品经理的一天}
\label{\detokenize{chapter_experience/1Day:id1}}\label{\detokenize{chapter_experience/1Day::doc}}
结合产品经理基本工作流程来看这个问题,会更容易理解一些。虽然具体的产品开发工作不用产品经理做,但产品经理也绝对做不了甩手掌柜。在有产品开发时,他需要时刻关注产品的进度,进行问题确认,必要的时候协调资源;在没有产品开发时,他需要进行规划,同时还要关注市场及竞品的变化,以能够及时洞察产品的发展趋势。

把以上的这段文字转换成场景,基本上产品经理的一天就能呈现在我们面前。


\paragraph{场景}
\label{\detokenize{chapter_experience/1Day:id2}}
早上在上班通勤的路上,产品经理可能会打开新闻客户端,关注自己感兴趣或与工作相关的内容,必要的话会把重要信息或链接记在备忘录里。

到公司以后,规划自己一天的工作,打开电脑首先查看一下邮箱,邮箱里有四五封邮件,其中两封邮件是测试发的bug信息,需要沟通确认;有一封邮件是协同部门发来的,内容主要是得到了一些用户反馈,需要满足新的需求,需要进行评估;还有一封会议通知邮件,下午三点要开某产品需求沟通会议。

然后产品经理的一天也就围绕这几封邮件开始了。

开产品经理组内晨会:每天的产品经理组内晨会都是产品经理跟领导沟通的好机会,如果产品经理在工作中遇到了自己解决不了的问题,要学会寻求领导的帮助。

上午他可能会先去和测试沟通确认一下两个bug该如何修改,沟通的过程中又发现了新的问题,所以后来和测试的沟通就变成了和测试、开发、前端等同事的沟通。问题解决了,时间也过去了半个多小时。

解决了测试bug的问题,产品经理需要好好想想协同部门提的需求。对产品经理而言,需要很慎重地对待需求,有的需求不一定要满足,而有的则必须快速响应。经过初步分析,这个需求是需要满足的,但如何做还需要和领导沟通一下。因此,产品经理就去找领导沟通,沟通后最终形成了一个初步的方案,产品经理以此回复了邮件。而此时已经上班两个多小时了。

产品经理刚发完邮件,着手开始准备下午开会资料时,电话响了。电话是客服同事打来的,说用户使用出现了问题,需要马上解决。产品经理只好先暂时放下手里的活,去解决用户问题。用户问题解决了,时间基本上也就到中午了。

下午的工作内容相对比较整,简单说就是准备开会、开会。不过,在这个过程中,还是时不时需要跟项目成员确认信息;收到其他同事的微信或电话。下午的会开得还算成功,不过有些需求的细节还是需要调整。会议开完,产品经理就开始着手修改工作了。等修改完了,差不多也就到了下班的时间。

其实上面说了那么多,总结起来讲产品经理的一天就是由洞察趋势、内部沟通、整理信息、产品思考四大部分组成,其中沟通会占大部分时间,形式有面对面、电话、会议等等。所以沟通能力对产品经理来讲,尤为重要。


\paragraph{小结}
\label{\detokenize{chapter_experience/1Day:id3}}
关于产品经理工作流程,我们可以归纳为想、写、说、做、改五个字。任何一个阶段,都由人、物、信息三种元素组成,产品经理的工作也都以此展开。


\subsubsection{行业巨变}
\label{\detokenize{chapter_experience/issue:id1}}\label{\detokenize{chapter_experience/issue::doc}}

\subsubsection{盒马鲜生}
\label{\detokenize{chapter_experience/hema:id1}}\label{\detokenize{chapter_experience/hema::doc}}

\paragraph{观察}
\label{\detokenize{chapter_experience/hema:id2}}
“观察”的方式:

保持空无,抛下预设 ↓ 用客体视角觉察出自己内心与行为的关系 ↓
再试着深入“阅读”他人内心与行为的关系 ↓ 结合规律,分析出外界真实的需要 ↓
在生活与工作中做出策略调整或反应 ↓ 保持练习,达到情商和洞察力的提高


\paragraph{练习“关联性”思维的方式:}
\label{\detokenize{chapter_experience/hema:id3}}
抛开过去那种任何事都想着“自己干”的想法,问自己3个问题:
\begin{enumerate}
\sphinxsetlistlabels{\arabic}{enumi}{enumii}{}{.}%
\item {} 
我现在要做的事情,有没有利他性?

\item {} 
可以不可以与他人形成合力?

\item {} 
最终取得的成果,能不能多方共享?

\end{enumerate}


\bigskip\hrule\bigskip


如果3个问题想清楚了没问题,那么不怕拒绝,厚着脸皮干就完了!

日常要留心,自己和他人身上,有哪些可以“做成事”的资源,这并不是要人学会自利,而是需要培养自己的协作性。自己的专业知识,钱,甚至体力,时间,人脉圈,都是能一起互相协作的资源。

除了人与人的资源关联性,还可以培养物与物相互跨界联系的能力。

比如在阿里,训练公关的新闻策划能力,就有一种称之为“两只试管法”的日常思考方法,你可以想象成左手握一个产品试管,右手握一个情绪试管,然后两种试剂倒在了一起,产生神奇的化学反应。

比如:

盒马鲜生(线下的果蔬生鲜服务设施/一种都市快节奏生活方式)+
房价(情绪饱满的高敏感民生话题)= 品牌概念:盒区房

进口水果 + 北上广的生活压力(情绪饱满的消费焦虑)= 热门话题:车厘子自由


\subsubsection{Jamin}
\label{\detokenize{chapter_experience/Jamin:jamin}}\label{\detokenize{chapter_experience/Jamin::doc}}

\paragraph{GOAL:learn}
\label{\detokenize{chapter_experience/Jamin:goal-learn}}

\paragraph{靠拢互联网}
\label{\detokenize{chapter_experience/Jamin:id1}}

\paragraph{作品证明自己}
\label{\detokenize{chapter_experience/Jamin:id2}}

\paragraph{内推}
\label{\detokenize{chapter_experience/Jamin:id3}}

\subsubsection{taobao}
\label{\detokenize{chapter_experience/taobao:taobao}}\label{\detokenize{chapter_experience/taobao::doc}}
也许张勇最能理解蒋凡,因为他们都是那种,在关键时刻孤独地扮演过“扳道工”角色的人,无论当时对他们来说,自己在不在最重要的位置上。

在蒋凡身上,有着外界所说的“一眼看穿底层逻辑”的能力。也是当下信息爆炸的时代,一种透过乱七八糟的消息迷雾,看到复杂事物中最简单的常识的能力。

“无”招胜有招《笑傲江湖》里风清扬传给令狐冲的第一句话。


\paragraph{拼多多为什么能够在阿里眼皮下迅速崛起呢!?}
\label{\detokenize{chapter_experience/taobao:id1}}
如果说是把握了下沉市场还是流于表面,你用矛盾的观点看本质:

第1点,2015\textasciitilde{}2017年间,大量阿里生态内的小小B端的角色,如底层商家、淘客、羊毛党因为阿里战略调整,对外发生了外溢,这些互联网游牧民走到哪,哪里就形成了新的细小供应链。这些人离开阿里要吃饭啊,这是最主要矛盾。

第2点,低价智能机和微信支付相结合,带来了小镇青年整体电商用户盘子扩大,这些人的日常时间要怎么打发,身边可能连个高级商场都没有,这是次主要矛盾。

这些东西,身处五环内的你在那个年代里,光看数据是不会马上发现的,只有靠细微的洞察才能感知到:

1)快递小哥的包裹里是不是开始有了别的平台的商品?

2)老家父母亲戚的朋友圈,是不是很多东西变了?

3)地方台的的综艺节目里面,广告赞助商是不是出现了不认识的牌子?(可惜很多北上广人不看电视)

4)那些像游牧民族一样的羊毛党,被你屏蔽朋友圈的微商妈妈又在忙什么?

透过现象看本质,拼多多就是抓住了这些要素悄悄长大的。


\paragraph{怎么抓“主要矛盾”}
\label{\detokenize{chapter_experience/taobao:id2}}\begin{enumerate}
\sphinxsetlistlabels{\arabic}{enumi}{enumii}{}{.}%
\item {} 
首先是重新平衡天猫、淘宝的重心,执两用中,平衡“大多数用户”和B端之间的消费和供给,这不是拿捏尺度的平面问题,而是一个对顶层架构重新分析、设计的立体问题。

\item {} 
选用模式更适合五环外市场的聚划算做渠道下沉,向低线城市渗透、并且覆盖全年龄段,尽快封堵挤压拼多多的继续扩张

\item {} 
发力短视频、抖音、网红,直播这些内容场景,再通过大数据精准推送,通过占领用户时间,赢得市场,让B端人群比如主播网红下沉去填补C端的使用手机时间。

\item {} 
带领品牌商家下沉。之前很多品牌集中在打一二线市场,原有的渠道网络对于下沉市场是滞后的。但随着阿里的强势运营,优质的中部商家做敲门砖品牌迅速得以下沉。提前占住山头,让对手仰攻。

\end{enumerate}

随着最近淘宝特价推出,结合淘宝、聚划算、天猫、淘小铺全面出击,阿里军团的刀枪剑戟朝向了同一个方向,B端搭建架构,C端占领时间,蒋凡完成了对北上广人群和下沉市场的一记全垒打!


\subsubsection{tencent}
\label{\detokenize{chapter_experience/tencent:tencent}}\label{\detokenize{chapter_experience/tencent::doc}}

\paragraph{运营转产品}
\label{\detokenize{chapter_experience/tencent:id1}}
运营:短期价值 产品:长期价值


\paragraph{总结}
\label{\detokenize{chapter_experience/tencent:id2}}
心态:认识自己,而非彻底否定 把握可控:自我介绍和提问环节 不设限:多尝试
低谷期:学 规划:把握住运气


\paragraph{简历坑}
\label{\detokenize{chapter_experience/tencent:id3}}

\subparagraph{做}
\label{\detokenize{chapter_experience/tencent:id4}}
更关注工作经历 简历是产品,HR即用户


\subparagraph{投}
\label{\detokenize{chapter_experience/tencent:id5}}
越新越可能,不合适也可以投


\paragraph{面试坑}
\label{\detokenize{chapter_experience/tencent:id6}}
多试试,现场面试


\paragraph{公司}
\label{\detokenize{chapter_experience/tencent:id7}}
小公司:多面手 大公司:专才

{[}1{]}: \sphinxhref{https://zhuanlan.zhihu.com/p/257044198}{从半年低谷期到入职腾讯产品经理,他的经验是什么?\sphinxhyphen{}
知群群星闪耀时}%
\begin{footnote}[212]\sphinxAtStartFootnote
\sphinxnolinkurl{https://zhuanlan.zhihu.com/p/257044198}
%
\end{footnote}


\subsubsection{蚂蚁借呗的金融产品解析}
\label{\detokenize{chapter_experience/ant_jiebei:id1}}\label{\detokenize{chapter_experience/ant_jiebei::doc}}
信用贷款是指以借款人的信誉发放的贷款,借款人不需要提供担保。

其特征就是:借款人无需提供抵押品或第三方担保仅凭自己的信誉就能取得贷款,并以借款人信用程度作为还款保证的。

信用贷成为了众多平台变现的方式:对于变现困难的平台来讲,可以借鉴信用贷的形式,实现金融变现。

一探究竟:
\begin{enumerate}
\sphinxsetlistlabels{\arabic}{enumi}{enumii}{}{.}%
\item {} 
市场规模:艾瑞咨询预测,未来两年互联网消费金融2019年的规模将达到3.4万亿元

\item {} 
蚂蚁借呗:根据风控和准入标准筛选额度,定位为小额、高息,面向支付宝用户,门槛,无抵押。

\item {} 
产品特征:对比传统的流程(线下网点,打印个人信用资料)支付宝申请简便、当日放款、自动还款防忘。

\item {} 
准入条件:门槛–个人实名认证、芝麻信用在600分以上,无在阿里其它平台内有不良记录和纠纷

\item {} 
放款机构:重庆市蚂蚁商城小额贷款有限公司和广发银行股份有限公司合作联合放贷

\item {} 
授信额度:授信额度从1千到30万不等,用户在阿里生态内购物、理财、捐款、转账、上传学历车辆信息等能提高用户的授信额度,用户变更收货地址、多次借贷、核心联系人逾期会降低自己的授信额度。

\item {} 
可用额度:等于或者小于授信金额;授信额度减去已经提取额度,即下次可贷金额

\item {} 
借款期限:非固定期限(按天计算)和固定期限(按月计算:3、6、12个月)。

\item {} 
借款利率:
日计算利率,蚂蚁借呗的借款日利率为0.015\%\sphinxhyphen{}0.6\%的区间,乘以365天即年化5.5\%\sphinxhyphen{}21.9\%。

\item {} 
征信情况:蚂蚁借呗主要依靠人行征信和芝麻信用评断用户的信用情况

\item {} 
贷款审批:流程快慢与数据、分控有关,流程分“借款申请”、“借款审核”与“放款还款”

\item {} 
还款方式:蚂蚁借呗目前只先息后本和每月等额两种还

\item {} 
结清管理:提前部分结清或者全部结清,
一般结清需要根据提前还款金额收一定比例的手续费,蚂蚁借呗目前没有收取提前结清手续费,但是在贷款期限内,提前结清手续费的收取标准也可能产生变化。恶意频繁多次的提前还款,来达到套现、刷分的目的,有可能导致借呗账户额度被降低或关闭。

\item {} 
贷后管理:如果用户有多笔按月借款类贷款,那么蚂蚁统一把还款日固定成统一的一天

\item {} 
逾期管理:还款日扣划或者冻结用户的网商银行账户,支付宝账号以及在阿里平台内其它应收款项,直到用户的欠款还清,否则一直持续的扣划或者冻结相关账号;逾期后蚂蚁借呗的未偿还本金贷款利率按照放款利率提高50\%,如果资金未按贷款用途使用,那么罚息在放款利率水平上提升100\%,未偿还的利息按照罚息利率计算复率;如果用户逾期严重,那么借呗可以通知用户的关联人逾期情况,同时该笔借款会被蚂蚁借呗委托给第三方催收公司和律师事务所,进行电话催收甚至是法律诉讼。同时逾期信息也将会计入芝麻信用和上报至人行征信系统。

\end{enumerate}


\subsubsection{diudiu}
\label{\detokenize{chapter_experience/diudiu:diudiu}}\label{\detokenize{chapter_experience/diudiu::doc}}

\paragraph{求职}
\label{\detokenize{chapter_experience/diudiu:id1}}
一次就够,1/2000:\sphinxurl{https://baike.baidu.com/item/\%E5\%BC\%A0\%E9\%A2\%96/3405319}


\paragraph{动态迭代}
\label{\detokenize{chapter_experience/diudiu:id2}}
了解评价,作品找工作

「快速迭代」:缩短决策与行动之间的周期,先让自己快速动起来。\sphinxhref{https://zhuanlan.zhihu.com/p/146486072}{1}%
\begin{footnote}[213]\sphinxAtStartFootnote
\sphinxnolinkurl{https://zhuanlan.zhihu.com/p/146486072}
%
\end{footnote}


\paragraph{作品核心}
\label{\detokenize{chapter_experience/diudiu:id3}}
核心的用户需求 产品定位 规划


\paragraph{话题引子}
\label{\detokenize{chapter_experience/diudiu:id4}}
分析(用户、需求等)–》落地(设计、输出)


\paragraph{规划路径}
\label{\detokenize{chapter_experience/diudiu:id5}}
不纠结

\sphinxurl{https://izhiqun.com/web/share/experience}


\paragraph{关注潜力}
\label{\detokenize{chapter_experience/diudiu:id6}}

\paragraph{准备}
\label{\detokenize{chapter_experience/diudiu:id7}}

\paragraph{不闭门造车}
\label{\detokenize{chapter_experience/diudiu:id8}}

\subsubsection{钉钉}
\label{\detokenize{chapter_experience/dingding:id1}}\label{\detokenize{chapter_experience/dingding::doc}}
2014年,阿里经历了强推社交产品“来往”的巨大挫折,在智能手机全国开始普及的年代,因为社交用户基数大,而且极度高频的入口级特性,社交产品所能带来的安全感是各大互联网厂商都极度渴望的,所以你可以理解为什么马化腾会把微信横空出世称为:\sphinxstylestrong{抢到第一张移动互联网船票}。

绝望会让一个人抛弃原有的脑子里对世界所有的理解,进入一种彻底放空和内省状态,这时候才能静下心来观察和阅读世界真正的需要。

作为一个产品经理可能会反思,任何大而广的东西一定有弱点,如果说微信的社交面是一条横线,需要观察寻找的,是哪里可以诞生一条尚未挖掘的纵线。

静心向内看就会有答案,那就是阿里生态圈的万千小B企业,如果你进入用户的心中去“观察”他们的想法。你就会用心眼看到后面的答案。

钉钉的成功最深处,是在碎片化办公的大环境下,人性中饱含的对深度工作专注和效率的追求。而在这一点上,无论是老板还是员工,只要他还算是
“想做事的人” 那就是共通的!

钉钉所有的员工,入职后第一课就是被要求,放下已知,带着空杯进入那些小B企业中,同工同吃,“观察”和阅读用户内心真正的需要。


\subsubsection{4年产品工作总结}
\label{\detokenize{chapter_experience/4years:id1}}\label{\detokenize{chapter_experience/4years::doc}}
关于学习成长,关于交流分享,关于职场选择。


\subsubsection{剧本杀}
\label{\detokenize{chapter_experience/jubensha:id1}}\label{\detokenize{chapter_experience/jubensha::doc}}
“剧本杀”最初源自线下游戏“谋杀之谜”,是一款 LARP
(实时角色扮演)游戏。不同游戏的剧本内容各不相同,但是玩法基本大同小异:

游戏开始阶段,每一名玩家选择扮演剧本中的一个角色,其中有一名玩家会在其他玩家不知情的情况下\sphinxstylestrong{扮演凶手},其他玩家需要在故事情节以及所搜寻到的证据的分析推理下,\sphinxstylestrong{共同找出真凶}。

剧本杀多是以封闭和半封闭半开放的剧本为主。封闭的剧本好比爬楼梯,一步步的探索最终获得事件的真相;开放的剧本就好比寻宝,一丝丝的痕迹与线索拼凑在一起得知最后的事实。


\paragraph{起源与走红1\sphinxfootnotemark[214]}
\label{\detokenize{chapter_experience/jubensha:id2}}%
\begin{footnotetext}[214]\sphinxAtStartFootnote
\sphinxnolinkurl{http://www.woshipm.com/it/1374466.html}
%
\end{footnotetext}\ignorespaces 
起源于风靡欧美的线下派对,2016年首档明星推理综艺秀\sphinxstylestrong{《明星大侦探》}第三季播出结束后播放量突破32亿次,收获了大批“剧本杀”粉丝。

2018年上半年,随着几款连麦推理社交游戏的上架,“剧本杀”迅速走红


\paragraph{游戏环节分析}
\label{\detokenize{chapter_experience/jubensha:id3}}
五个环节:
\begin{enumerate}
\sphinxsetlistlabels{\arabic}{enumi}{enumii}{}{.}%
\item {} 
选择角色:共侦探、嫌疑人、真凶。进入游戏,默认进入语音群聊,并选择想要扮演的剧本角色房间后,此阶段时间较短,待所以选角完成,进入下一阶段

\item {} 
人物剧本:所有玩家阅读各自人物剧本,剧本内容包含人物过往经历、重要线索、时间点等情节。剧本阅读完毕,玩家们在语音群聊房间内依次自我介绍,并讲述过往经历及重要线索。尽量入戏,你就是这个人物,不要莫得感情的读剧本。\sphinxhref{https://www.murdermysterypa.com/thread-3693-1-1.html}{2}%
\begin{footnote}[215]\sphinxAtStartFootnote
\sphinxnolinkurl{https://www.murdermysterypa.com/thread-3693-1-1.html}
%
\end{footnote}

\item {} 
搜集线索:所有玩家共同阅读线索信息,同时就具体线索展开讨论。以发问、闲聊等方式透露给其他玩家(关键信息可隐瞒)。此阶段包含对多个角色的线索搜集及玩家讨论(可以选择分享以甩锅或保留以掩盖)的过程,直至所有的线索搜集并讨论完成,方可结束。桌游版剧本杀通常将证据卡片分布在不同场景和人物身上,每位玩家对每个场景和人物有搜索限制(每个本不一样)。\sphinxhref{https://www.zhihu.com/question/270386766/answer/615240050}{5}%
\begin{footnote}[216]\sphinxAtStartFootnote
\sphinxnolinkurl{https://www.zhihu.com/question/270386766/answer/615240050}
%
\end{footnote}严格以上,只是为玩家提供一些思路去范围缩小或怀疑对象!误导类、决定性类、无用类和一般线索,别轻易定性,思考怎样串联完整的证据链\sphinxhref{https://zhuanlan.zhihu.com/p/66137913}{8}%
\begin{footnote}[217]\sphinxAtStartFootnote
\sphinxnolinkurl{https://zhuanlan.zhihu.com/p/66137913}
%
\end{footnote}。

\item {} 
圆桌讨论:搜证阶段结束后,玩家们再次共同讨论并寻找故事真凶,此阶段\sphinxstylestrong{不提供}任何剧本及线索,所以各玩家首先需要\sphinxstylestrong{回忆}之前各个阶段的逻辑推理再作讨论。因为之后会共同投票,更加鼓励玩家把自己的想法和推理讲出来,让其他玩家信服

\item {} 
最终投票:就圆桌阶段讨论的结果,玩家们共同投票,选择凶手。投票阶段一般会计时,玩家需要在计时结束前完成投票。投票结束后,公布结局真相,并给出答案解析,游戏结束。

\end{enumerate}

目前主流的“剧本杀”游戏,玩家的主体精力基本放在“搜集线索”环节,需要进行多次搜证并对线索加以讨论。在房间人数上大多采用多人局,最多支持8人,其中《我是谜》app还提供了1\sphinxhyphen{}2人局,在极大程度解决了用户匹配的问题。游戏基本采用纯音频形式进行游戏,未来或许会出现类似“狼人杀”的视频面杀形式。


\paragraph{世界观→动机→逻辑→线索4\sphinxfootnotemark[218]}
\label{\detokenize{chapter_experience/jubensha:id4}}%
\begin{footnotetext}[218]\sphinxAtStartFootnote
\sphinxnolinkurl{https://www.zhihu.com/question/270386766/answer/692364483}
%
\end{footnotetext}\ignorespaces \begin{itemize}
\item {} 
世界观:隐藏在背景故事和个人故事里,从故事的完整性上,更好对照人物关系,动机。

\item {} 
人物:服装(带面具,面纱,口罩,有胡子)、发色、服装(样式,颜色)、身份(和死者事件的关系及目的)
凶手:
\sphinxstyleemphasis{禁忌}:群攻(没有逻辑地拉仇恨)、乱私聊(暴露关系)、线索(作案销毁物证、第一时间搜)、时间线(读剧本作案卡壳)、多嘴(爆出当时只有凶手才知道)。\sphinxstyleemphasis{正确}:栽赃第二嫌疑人(引诱其他玩家盘问、隔山观虎斗)、骗拉友军(洗清别人嫌疑、后期隐藏)、散布假证据、沉着冷静流畅地编(记下其他角色作案时间,别冲突\sphinxhref{https://zhuanlan.zhihu.com/p/66137913}{8}%
\begin{footnote}[219]\sphinxAtStartFootnote
\sphinxnolinkurl{https://zhuanlan.zhihu.com/p/66137913}
%
\end{footnote})

\item {} 
动机:分外、内因。路人(双眼能佐证所有人的时间线、驳不在场证明)、动机证人(知道过去别人不知道的)、昏迷失忆(剧本交互导致的)

\item {} 
逻辑:无故放弃杀害、动机和行为不符。是个人故事、时间线,动机的综合体。

\item {} 
线索:不会骗人,佐证你的逻辑线,动机,世界观。明辨线索为一次行为形成的还是多次行为形成的。注意人设是否前后不一。关键信息:尸体死状,从伤口等找到凶器,\sphinxstylestrong{注意}多种凶器都有可能造成,结合时间线和动机综合分析!

\end{itemize}


\paragraph{用户分析}
\label{\detokenize{chapter_experience/jubensha:id5}}
“剧本杀”游戏的玩家群体,从技术竞技性、社交追求度和游戏排他性可以大致分为以下几类:竞技型玩家、社交型玩家、语音型玩家。

群体细分:
\begin{itemize}
\item {} 
竞技性玩家:喜爱游戏本身,极度追求技术,对游戏剧本的内容,游戏结果以及整体流程体验会更加看重

\item {} 
社交型玩家:青睐社交属性较重,热爱结交朋友,更看重社交方面的体验

\item {} 
语音型玩家:热爱包含语音连麦功能的游戏,更关注连麦时语音的质量

\end{itemize}


\paragraph{游戏关键要素}
\label{\detokenize{chapter_experience/jubensha:id6}}
线下游戏迁移至线上app
\begin{itemize}
\item {} 
好评:例如,“玩法很新颖”、“有一种自己演电影的感觉”、“很锻炼逻辑思维能力”

\item {} 
问题:例如,“希望剧本的筛选能更用心”、“有些人麦的回音很大”、“无法闭麦”等。

\end{itemize}
\begin{enumerate}
\sphinxsetlistlabels{\arabic}{enumi}{enumii}{}{.}%
\item {} 
剧本内容:依靠粉丝网友的投稿,奖励机制吸引优质写手,签约长期合作的剧本作者。合理精彩的剧本会带来更好的游戏体验

\item {} 
实时互动:除了阅读剧本,搜集线索的过程,90\%以上的游戏体验核心都在语音聊天上。网络不稳定等原因导致语言不通畅

\item {} 
音质效果:因为语音连麦聊天的问题,杂音、丢音、回声、噪声等问题严重影响游戏体验

\item {} 
稳定流畅:网络异常问题导致的游戏过程断续,音视频技术可调用第三方的SDK(满足快速上线与稳定)

\end{enumerate}


\paragraph{对产品经理的好处3\sphinxfootnotemark[220]}
\label{\detokenize{chapter_experience/jubensha:id7}}%
\begin{footnotetext}[220]\sphinxAtStartFootnote
\sphinxnolinkurl{http://www.woshipm.com/pmd/3064843.html}
%
\end{footnotetext}\ignorespaces \begin{enumerate}
\sphinxsetlistlabels{\arabic}{enumi}{enumii}{}{.}%
\item {} 
锻炼逻辑思维能力、辩论能力:以理服人

\item {} 
锻炼独立思考、自我判断能力:辨别伪装

\item {} 
锻炼记忆力:沟通需求不遗漏,后置位发言的人很有优势的原因。

\item {} 
锻炼随机应变能力:应对措手不及的突击、质问、威胁\sphinxhref{https://www.zhihu.com/question/270386766/answer/415647339}{7}%
\begin{footnote}[221]\sphinxAtStartFootnote
\sphinxnolinkurl{https://www.zhihu.com/question/270386766/answer/415647339}
%
\end{footnote}

\item {} 
锻炼团队配合能力:协调好团队,以共同决策。信息闭塞

\item {} 
锻炼辨别真伪需求能力:分辨表面和内在实质

\item {} 
交朋友:放松工作压力的同时社交

\item {} 
锻炼撒谎能力:见人说人话见鬼说鬼话。

\item {} 
锻炼抗压能力:项目组其他岗位(设计、开发、测试、运营)的,随时都可能并发产生。破绽后洗白。

\item {} 
锻炼控制情绪的能力:以理服人,防自爆

\item {} 
锻炼表演能力:换角度思考别人的问题,才能演好自己的角色

\item {} 
锻炼观察能力:表情不自然、言行怪异\sphinxhref{https://www.zhihu.com/question/270386766/answer/655939057}{6}%
\begin{footnote}[222]\sphinxAtStartFootnote
\sphinxnolinkurl{https://www.zhihu.com/question/270386766/answer/655939057}
%
\end{footnote}

\item {} 
锻炼交流能力:避免信息闭塞的自我相信。追问、私聊、讨论

\end{enumerate}


\subsubsection{失败}
\label{\detokenize{chapter_experience/fail:id1}}\label{\detokenize{chapter_experience/fail::doc}}

\paragraph{别死}
\label{\detokenize{chapter_experience/fail:id2}}
互联网本身竞争就激烈,往往是老大和老二打架,最后死的却是老三,前几年风光无限的明星产品经理,

过几年之后就沦为一个故事家,出来打情怀牌,吃老本的人也大把大把的存在。

这还是好的情况,很多做了N年的产品经理,本职工作并不能有很好的突破,行业不行,公司不行,自己空有改变世界的心,可惜最后也沦为了重复造轮子的工具人。

但是不可否认,那些处在金字塔尖,动不动做着行业排名数一数二产品经理,日子还算过得滋润。

那些没有成功的产品经理也并不代表他们不优秀,可能就差一次机会而已。

现在之所以会出现一些唱衰产品经理的声音,主要是和疫情经济影响和移动互联网增长乏力带来的双重影响。


\subsubsection{AI PM的隐性工作1\sphinxfootnotemark[223]}
\label{\detokenize{chapter_experience/recessive_work:ai-pm1}}\label{\detokenize{chapter_experience/recessive_work::doc}}%
\begin{footnotetext}[223]\sphinxAtStartFootnote
\sphinxnolinkurl{https://medium.com/@liwdai/ai-pm-\%E4\%B9\%8B\%E9\%9A\%90\%E6\%80\%A7\%E9\%83\%A8\%E5\%88\%86\%E7\%9A\%84\%E5\%B7\%A5\%E4\%BD\%9C-be6de08d1c05}
%
\end{footnotetext}\ignorespaces 
「产品经理的工作包括显性部分和隐性部分。为外界所知的显性部分,通常其实只占这个工作的十分之一左右;不为外人道的隐性部分,则占了这个工作的十分之九甚至更多。」


\paragraph{隐性工作 — PM}
\label{\detokenize{chapter_experience/recessive_work:pm}}
一般来说,偏 PM 的隐形部分的工作主要包括以下几个方面:
控制项目进度,按时交付产品; 团队协作与沟通; 产品开发周期中的问题处理。


\subsection{project}
\label{\detokenize{chapter_project/index:project}}\label{\detokenize{chapter_project/index:chap-project}}\label{\detokenize{chapter_project/index::doc}}

\subsubsection{AI行业分析 1\sphinxfootnotemark[224]}
\label{\detokenize{chapter_project/AI_industry_analysis:ai-1}}\label{\detokenize{chapter_project/AI_industry_analysis::doc}}%
\begin{footnotetext}[224]\sphinxAtStartFootnote
\sphinxnolinkurl{http://www.woshipm.com/pd/873240.html}
%
\end{footnotetext}\ignorespaces 
\begin{center}\sphinxincludegraphics{{2020_AI}.png}\end{center} \sphinxincludegraphics{{iresearch_AI}.png}


\paragraph{优势:极快、极简}
\label{\detokenize{chapter_project/AI_industry_analysis:id1}}
人工智能可以处理人1秒中可以想出答案的问题,这个问题还需要有以下几个特点:大规模,重复性,限定领域,快速反馈。

人工智能产品设计要以操作极度简单为标准,但是前端的简单代表后端的复杂,系统越复杂,才能越智能。

同样,人工智能的发展依赖于产业生态的共同推进,上游芯片提供算力保障,中游人工智能厂商着力研发算法模型,下游应用领域提供落地场景

\begin{figure}[H]
\centering
\capstart

\noindent\sphinxincludegraphics{{qushi}.png}
\caption{趋势}\label{\detokenize{chapter_project/AI_industry_analysis:id12}}\end{figure}


\paragraph{BAT}
\label{\detokenize{chapter_project/AI_industry_analysis:bat}}
百度A(AI)B(Big
data)C(Cloud)战略,阿里腾讯也有各自云服务,大数据中心,人工智能实验室,这些大公司胜在基础架构层、数据量和资本优势上,拥有大量的人工智能科学家,可以持续优化算法,提升算法模型的准确度。


\paragraph{准确性}
\label{\detokenize{chapter_project/AI_industry_analysis:id2}}

\subparagraph{需要达到99.9999\%}
\label{\detokenize{chapter_project/AI_industry_analysis:id3}}
如手术机器人,自动驾驶技术,智慧交通等,这些产品和服务直接关系到人的生死,要求具有极高的准确度,需要AI科学家持续的优化,只有达到近乎百分之百的准确度才会商用。


\subparagraph{达到99\%或者95\%就可以}
\label{\detokenize{chapter_project/AI_industry_analysis:id4}}
如面部识别,语音机器人,无人机农药喷洒,艺术设计,搜索引擎,精准营销等,这些产品和服务对于精确度要求不高,因为即使不精确也不会直接造成人员伤亡。


\paragraph{垄断程度}
\label{\detokenize{chapter_project/AI_industry_analysis:id5}}

\subparagraph{高}
\label{\detokenize{chapter_project/AI_industry_analysis:id6}}
行业的垄断程度越高,头部公司的体量越大,最初可能因为缺乏AI技术而采购技术,当技术环境成熟,BAT和google这类公司开源了大量技术后,行业垄断型公司会则会搭建自己的AI团队,搭建自己的大数据,云计算和AI实验室,以运营商行业为例,资源垄断型市场,三家独大,每家都在搭建自己的大数据分析平台,也在搭建自己的人工智能实验室。


\subparagraph{低}
\label{\detokenize{chapter_project/AI_industry_analysis:id7}}
如衣食住行相关的制造业和零售行业,因为分散,他们有需求,但是没有足够体量和资本自己搭建AI团队,所以他们会将AI技术作为一项工具,以合理的价格采购成套服务,来实现+AI的升级。

如同当年的互联网+和+互联网一样,也会演化出AI+和+AI的发展方向。


\subparagraph{象限图}
\label{\detokenize{chapter_project/AI_industry_analysis:id8}}
我认为第一象限因为BAT拥有科学家优势,虽然垄断程度高的企业很有钱,但是因为BAT有数据优势和科学家优势,在这个领域BAT优势明显,可以向企业提供独特的AI服务,提升垄断企业效率,这部分产品需要靠AI科学家驱动。

第三象限虽然技术门槛低,垄断程度低,会出现大量小AI公司进入这个市场,BAT进入这个市场拥有足够的品牌优势,因为市场需求量较大,BAT可以考虑做开放平台,为有垂直领域的AI公司体统底层服务,如果自己来做,这部分服务和产品将是运营和产品来主要驱动。

第二象限暂时来看不太适合进场,第四象限垄断企业会自己组建AI团队来做,我们能看到,手机制造这个还不算垄断的行业中,因为资本实力雄厚,各个厂家已经在组建自己的AI研发团队。

\begin{figure}[H]
\centering
\capstart

\noindent\sphinxincludegraphics[width=600\sphinxpxdimen]{{产品象限}.png}
\caption{产品象限}\label{\detokenize{chapter_project/AI_industry_analysis:id13}}\end{figure}


\paragraph{应用场景2\sphinxfootnotemark[225]}
\label{\detokenize{chapter_project/AI_industry_analysis:id9}}%
\begin{footnotetext}[225]\sphinxAtStartFootnote
\sphinxnolinkurl{https://www.zhihu.com/question/57373956/answer/155398900}
%
\end{footnotetext}\ignorespaces 
1.场景比较规范,2.需要经验,
3.且数据量大,4.但是反复度高的工作岗位,5.如果监管准入门槛比较低就更好。
1和5可促进快速落地,2、3、4适合深度学习复现场景。

医疗+AI,门槛着重考虑;安防+AI,门槛重在渠道,和海康;无人驾驶,需要规范,市场、大众、政府、产品供应、交通设施等都需要规范。


\paragraph{2B}
\label{\detokenize{chapter_project/AI_industry_analysis:b}}

\subparagraph{民营企业}
\label{\detokenize{chapter_project/AI_industry_analysis:id10}}
赚更多的钱 转型的决心和行动力:只要技术是有用的,可以提升效率或压缩成本的
途径:BAT可以考虑在尽可能多民营企业家聚集的场合,推广真实高效的+AI产品和服务


\subparagraph{国营企业}
\label{\detokenize{chapter_project/AI_industry_analysis:id11}}
国营企业即承担创造价值的责任,也同时承担着保证国有资产不流失的责任,组织内部员工多是对上级和自己的职位负责,所以创新一定要稳妥
用友和亚信等软件开发团队多是长期驻厂,提供运维服务和新需求开发
核心诉求是不犯错,未必有功,但求无过

AIStartups: \sphinxurl{https://github.com/lipiji/AIStartups}


\paragraph{More:}
\label{\detokenize{chapter_project/AI_industry_analysis:more}}
\begin{figure}[H]
\centering
\capstart

\noindent\sphinxincludegraphics{{data_AI_industry}.jpg}
\caption{data\_AI\_industry}\label{\detokenize{chapter_project/AI_industry_analysis:id14}}\end{figure}

\sphinxurl{https://mattturck.com/data2020/}

\sphinxurl{https://daxueconsulting.com/category/artificial-intelligence-industry-in-china/}

\sphinxurl{https://www.ulapia.com/reports/search?query=AI}
\sphinxurl{https://www.iyiou.com/search?p=\%E4\%BA\%BA\%E5\%B7\%A5\%E6\%99\%BA\%E8\%83\%BD}


\subsubsection{AI公司}
\label{\detokenize{chapter_project/AI_company:ai}}\label{\detokenize{chapter_project/AI_company::doc}}

\paragraph{理解企业}
\label{\detokenize{chapter_project/AI_company:id1}}
\sphinxurl{https://www.zhihu.com/market/paid\_column/1251475507390050304/section/1251475513652604928}

“那些口号喊得响、低估场景挑战、高估技术能力的公司大多会在泡沫中死掉。”GMIS全球机器智能峰会后,刘维在接受网易智能专访时这样评价AI创业现状。
\sphinxhref{https://mp.weixin.qq.com/s?\_\_biz=MzI3NTU3ODk1MQ==\&mid=2247484933\&idx=1\&sn=e7b99f0686f5f4c6f9d41bc22a012881\&chksm=eb03ef2ddc74663bc8f0ccca0f64c71a72e9e5583986806f81d86a799beca3d56ac970f461f9\&scene=21\#wechat\_redirect}{9}%
\begin{footnote}[226]\sphinxAtStartFootnote
\sphinxnolinkurl{https://mp.weixin.qq.com/s?\_\_biz=MzI3NTU3ODk1MQ==\&mid=2247484933\&idx=1\&sn=e7b99f0686f5f4c6f9d41bc22a012881\&chksm=eb03ef2ddc74663bc8f0ccca0f64c71a72e9e5583986806f81d86a799beca3d56ac970f461f9\&scene=21\#wechat\_redirect}
%
\end{footnote}


\paragraph{国际互联网}
\label{\detokenize{chapter_project/AI_company:id2}}
\begin{figure}[H]
\centering
\capstart

\noindent\sphinxincludegraphics{{international_AI}.png}
\caption{国际互联网企业产业布局图谱\sphinxhref{https://weread.qq.com/web/reader/40632860719ad5bb4060856kc9f326d018c9f0f895fb5e4}{10}\sphinxfootnotemark[227]}\label{\detokenize{chapter_project/AI_company:id5}}\end{figure}
%
\begin{footnotetext}[227]\sphinxAtStartFootnote
\sphinxnolinkurl{https://weread.qq.com/web/reader/40632860719ad5bb4060856kc9f326d018c9f0f895fb5e4}
%
\end{footnotetext}\ignorespaces 

\paragraph{中国公司总览 7\sphinxfootnotemark[228]}
\label{\detokenize{chapter_project/AI_company:id3}}%
\begin{footnotetext}[228]\sphinxAtStartFootnote
\sphinxnolinkurl{https://daxueconsulting.com/ai-landscape-china/}
%
\end{footnotetext}\ignorespaces 
\begin{figure}[H]
\centering
\capstart

\noindent\sphinxincludegraphics{{AI_company}.jpg}
\caption{AI公司}\label{\detokenize{chapter_project/AI_company:id6}}\end{figure}


\paragraph{阵营 4\sphinxfootnotemark[229]}
\label{\detokenize{chapter_project/AI_company:id4}}%
\begin{footnotetext}[229]\sphinxAtStartFootnote
\sphinxnolinkurl{https://zhuanlan.zhihu.com/p/33524676}
%
\end{footnotetext}\ignorespaces 
人工智能公司(主要针对创业公司)主要分为三个阵营:\sphinxhref{https://www.sohu.com/a/364264851\_114819}{5}%
\begin{footnote}[230]\sphinxAtStartFootnote
\sphinxnolinkurl{https://www.sohu.com/a/364264851\_114819}
%
\end{footnote}
\begin{enumerate}
\sphinxsetlistlabels{\arabic}{enumi}{enumii}{}{.}%
\item {} 
研究核心技术的AI公司(Core AI
Companies)核心人工智能,主要针对人工智能基础设施的搭建。产品经理侧重于对底层技术框架的理解。

\item {} 
应用人工智能公司(Application AI
Companies):通常的表现形式是提供一种基础功能,客户可以通过调用封装好的API进行对自身产品的武装或填充,而无需自己研究基础功能。因为往往对于一些中小公司而言,拥有的数据量有限,无法通过机器学习技术完成对每一个基础功能的建模和应用部署,因此需要借助这样公司提供的开放API能力,然后自身做垂直应用。产品对行业的理解力和对行业趋势的洞察力是核心;应用AI技术公司的商业模式以TO
B为主,产品经理的KPI是项目回款,因此产品经理需要有一定的商务技能(售前、销售);同时因为需要定制化开发,产品经理要明确区分标准化产品和定制化产品;

\item {} 
行业人工智能公司(Industry AI
Companies):三个阵营中最接近终端用户的公司,提供垂直领域的AI服务,帮助用户解决具体场景中的具体问题。产品对行业的理解力和对行业趋势的洞察力是核心

\end{enumerate}

核心人工智能公司往往对产品经理在技术层面要求最高,应用人工智能其次,行业垂直应用人工智能公司是对产品经理的业务深度或行业理解深度要求最高。


\paragraph{Baidu}
\label{\detokenize{chapter_project/AI_company:baidu}}
2016 ACL Fellow 百度 CTO
王海峰。\sphinxhref{https://www.jiqizhixin.com/articles/2019-11-28-4}{1}%
\begin{footnote}[231]\sphinxAtStartFootnote
\sphinxnolinkurl{https://www.jiqizhixin.com/articles/2019-11-28-4}
%
\end{footnote}

作为百度集团首席技术官,王海峰负责百度搜索、语音搜索、图像搜索、信息流、手机百度、小度机器人、自然语言处理、知识图谱、互联网数据挖掘等业务,并曾创始了百度语音、图像、推荐及个性化、深度学习、度秘等多个技术方向。由王海峰领导研发的百度翻译产品目前支持
28 种语言、756 个方向的自动翻译,并于 2015 年 5
月上线了全球首个融合神经网络机器翻译和统计机器翻译模型的大规模在线翻译系统。其领导的「基于大数据的互联网机器翻译核心技术及产业化」还荣获了
2015 年国家科技进步奖,这也是我国互联网企业首次获得该奖项。

2016 年,王海峰当选 ACL
Fellow,成为了首位获此荣誉的中国大陆科学家。会士评选委员会在对王海峰的评语中写道:王海峰在机器翻译、自然语言处理和搜索引擎技术领域,在学术界和工业界都取得了杰出成就,对于
ACL 在亚洲的发展也做出了卓越贡献。

在前沿技术领域,拥有七大实验室、聚集数十位世界顶级AI科学家的百度研究院,正在聚焦前瞻基础研究,探索技术前沿方向。比如,在量子领域,除“量桨”之外,百度还研发出了国际领先、国内第一的云上量子脉冲系统“量脉”。在区块链领域,百度超级链XuperChain实现了核心技术自主可控,专利申请量200多件。在工业物联网安全领域,百度AIoT安全能力覆盖六大场景,覆盖终端设备超过1.5亿。

在芯片方面,百度自主研发的百度昆仑AI芯片、百度鸿鹄AI芯片,正在为新基建提供可靠的动力。百度昆仑芯片是业界实际性能最高的AI芯片,也是首次在工业领域大规模应用的中国自研AI芯片。百度鸿鹄芯片作为远场语音芯片,适配于车载、智能家具等场景,一颗芯片就能解决所有语音交互问题,是一个足以改变行业的技术革新。

百度智能云已经取得了一系列傲人的成果:智慧金融已服务近200家金融客户,涉及营销、风控等十几个金融场景;智慧医疗的产品已经服务300多家医院和超过1500家基层医疗机构,服务人次超过了2500万;智慧城市则已经逐渐落地北京海淀、重庆、苏州等城市,成为新一代城市智能基础设施,让城市变得更智慧;智慧能源领域中,企业级AI中台、知识中台在国家电网、南方电网等头部客户落地应用,支撑20多个业务场景,覆盖两条特高压智能化线路、150多个智慧变电站、4万多条输电线路的监拍智能化,每天代替人工巡视能源路线超过7万公里;智能制造领域中,已覆盖14大行业,30余家企业,16个合作伙伴,触达32类垂直场景,在3C、汽车、钢铁、能源等行业已规模落地。

智能交通方面,百度Apollo依托百度领先的AI能力,接连中标重庆、合肥、阳泉等地车路协同新基建项目,Apollo
Robotaxi自动驾驶出租车服务也已在长沙全面开放试运营。第一季度,知名研究公司Navigant
Research将百度Apollo列为全球四大自动驾驶领域领导者之一。\sphinxhref{http://www.mysecretrainbow.com/ai/17083.html}{6}%
\begin{footnote}[232]\sphinxAtStartFootnote
\sphinxnolinkurl{http://www.mysecretrainbow.com/ai/17083.html}
%
\end{footnote}

在工业互联网领域,百度智能云在四季度分别与贵州省贵阳市政府、山东省济南市工信局签署战略合作,打造地方级AI工业互联网平台,全面推动企业数字化、智能化转型,助力当地经济形成新的经济增长点。\sphinxhref{http://finance.eastmoney.com/a/202102181812494141.html}{8}%
\begin{footnote}[233]\sphinxAtStartFootnote
\sphinxnolinkurl{http://finance.eastmoney.com/a/202102181812494141.html}
%
\end{footnote}

\begin{figure}[H]
\centering
\capstart

\noindent\sphinxincludegraphics{{baidu_AI}.png}
\caption{Baidu AI}\label{\detokenize{chapter_project/AI_company:id7}}\end{figure}


\paragraph{Aliyun}
\label{\detokenize{chapter_project/AI_company:aliyun}}
机器学习PAI
Studio\sphinxhref{https://www.aliyun.com/product/bigdata/product/learn}{2}%
\begin{footnote}[234]\sphinxAtStartFootnote
\sphinxnolinkurl{https://www.aliyun.com/product/bigdata/product/learn}
%
\end{footnote}
PAI\sphinxhyphen{}Studio为开发者提供可视化的机器学习实验开发环境,帮助用户实现0代码开发人工智能相关服务。内置数百个成熟的机器学习算法,覆盖商品推荐、金融风控、广告预测等场景。

PAI平台提供将常规机器学习模型/深度学习模型一键发布为Restful
API的功能,支持用户通过http请求调用模型服务做实时预测,并且提供蓝绿部署、模型版本管理、在线调试等服务功能。
每个模型在部署的时候由用户选择占用的资源:CPU部署单位为Quota,GPU部署目前支持P4卡(NVidia
Tesla P4 GPU卡),部署单位为P4Quota

部署单位说明: 1Quota为1核CPU、4G内存 1P4quota为1P4卡、8核CPU、32G内存

每个模型服务费用=部属单位数量\sphinxstyleemphasis{部属单位单价}时长

模型一旦部署并处于running状态就会开始计费,请切记及时停止无用的模型服务,以免造成不必要的费用开销
后付费服务类型 价格 区域 CPU服务 0.4元/Quota/小时 华东2、华北2
GPU服务(P4卡) 8元/P4Quota/小时 华东2、华北2


\paragraph{More}
\label{\detokenize{chapter_project/AI_company:more}}
\sphinxurl{https://www.tusimple.com/} \sphinxurl{https://github.com/amusi/CV-Company-List}
开源:\sphinxurl{https://www.oschina.net/company} \sphinxurl{http://www.birdbot.cn/}

AI 证券:

\sphinxurl{http://search.stcn.com/was5/web/search?token=0.1584090199903.75\&channelid=252914\&searchword=AI\&catid=\&order=rel\&before=\&after=};


\bigskip\hrule\bigskip


阿波罗的官网地址是: \sphinxurl{http://apollo.auto/}

源代码,文档与数据下载地址为: \sphinxurl{https://github.com/apolloauto}


\subsubsection{AI 金融}
\label{\detokenize{chapter_project/AI_Finance:ai}}\label{\detokenize{chapter_project/AI_Finance::doc}}

\paragraph{背景}
\label{\detokenize{chapter_project/AI_Finance:id1}}\begin{enumerate}
\sphinxsetlistlabels{\arabic}{enumi}{enumii}{}{.}%
\item {} 
金融科技的生态是三个相互牵制的部分:

\end{enumerate}

公司/银行——监管——资本
\begin{enumerate}
\sphinxsetlistlabels{\arabic}{enumi}{enumii}{}{.}%
\setcounter{enumi}{1}
\item {} 
金融科技发展:

\end{enumerate}

20世纪70年代 业务电子化 20世纪80年代 前台电子化(ATM机等) 20世纪90年代
金融业务互联网化(实现了高效连接) 21世纪 金融科技
\begin{enumerate}
\sphinxsetlistlabels{\arabic}{enumi}{enumii}{}{.}%
\setcounter{enumi}{2}
\item {} 
中国金融科技发展

\end{enumerate}

IT系统——支付——信贷——大金融——生活

\begin{center}\sphinxincludegraphics{{AI_finance}.png}\end{center} \sphinxhref{https://www.donews.com/news/detail/4/3084506.htmls}{5}%
\begin{footnote}[235]\sphinxAtStartFootnote
\sphinxnolinkurl{https://www.donews.com/news/detail/4/3084506.htmls}
%
\end{footnote}


\paragraph{为什么是金融+AI而非AI+金融呢4\sphinxfootnotemark[236]}
\label{\detokenize{chapter_project/AI_Finance:aiai-4}}%
\begin{footnotetext}[236]\sphinxAtStartFootnote
\sphinxnolinkurl{https://tanxianlian.com/2020/05/15/\%e9\%87\%91\%e8\%9e\%8dai\%e7\%9a\%84\%e6\%9c\%aa\%e6\%9d\%a5\%e7\%95\%85\%e6\%83\%b3/}
%
\end{footnotetext}\ignorespaces 
这两者的前后连接顺序体现的是主动和被动,引导和被引导的关系。之所以是金融+AI,体现的是这是金融行业对AI技术主动性的利用需求,而非被动型的推动。


\paragraph{为什么金融可以+AI呢?}
\label{\detokenize{chapter_project/AI_Finance:id2}}
因为金融业务开展的基础本质上是基于信息(数据)进行,AI可以对数据进行更好的利用,从而提升金融业务的效率。

具体来说,有这么几个主要的业务范围:
\begin{itemize}
\item {} 
KYC(客户了解):具体包括客户背景信息调查、客户核身、用户画像、客户偏好等;

\item {} 
交易决策:例如信贷领域的风控,理财领域的风险等级评估、产品推荐,保险领域的保险方案设计、理赔验证,以及所有细分领域都包含的反欺诈等;

\item {} 
客户服务:例如售前营销、售后服务等。

\end{itemize}


\paragraph{金融新基建2\sphinxfootnotemark[237]}
\label{\detokenize{chapter_project/AI_Finance:id3}}%
\begin{footnotetext}[237]\sphinxAtStartFootnote
\sphinxnolinkurl{https://www.leiphone.com/news/202012/7ovvkzByXnPQjnlD.html}
%
\end{footnotetext}\ignorespaces 
在金融新基建榜中,乐信、水滴、弘玑Cyclone、洞见、同盾五家公司凭借各自优势在众多优秀竞争者中脱颖而出。

他们分别荣获“最佳新消费AI平台奖”、“最佳保险科技数据中台奖”、“最佳智能自动化平台方案奖”、“最佳隐私计算平台奖”和“最佳智能分析决策奖”。


\paragraph{智能投顾1\sphinxfootnotemark[238]}
\label{\detokenize{chapter_project/AI_Finance:id4}}%
\begin{footnotetext}[238]\sphinxAtStartFootnote
\sphinxnolinkurl{https://zhuiyi.ai/solution/securities}
%
\end{footnotetext}\ignorespaces 
智能投顾,用服务新模式,打造差异化品牌
业务同质化让券商竞争激烈,企业希望通过服务的创新打造出差异化,吸引更多用户。追一AIForce的智能投顾助手YIFA提供了实时个股诊断、多条件筛选的能力,再结合快速交易能力,让投资者随时掌握个股动态,抓住转瞬即逝的交易机会。

智能投顾助手积累了行业头部的3000+常用知识点,让客户能在自营券商APP中闭环完成查询、交易和学习。创新的服务模式在不断增加客户信任度和粘性,提升品牌价值。

低成本高质量的智能外呼 有效覆盖更多场景
证券行业的高频度服务,让每个用户触点上的服务质量,
成为决定券商运营效率和客户满意度的关键。

外呼可以提供各类电话沟通服务,包括开户的回访、对离职员工名下的客户进行回访、风险抽查、满意度调查、新股中签缴费提醒等等。他还能提供自动的业务咨询等经纪服务,既降低人力成本,又提升服务能力,提升覆盖度。

客户画像师, 挖掘数字金矿价值
大量的客户数据和运营数据在碎片化的场景中,难以获得有效沉淀与分析利用。

追一AIForce的客户画像师Feature,基于强大的语义理解能力,可以分析每一通外呼电话和各个渠道的客户交互内容。打破数据黑盒,将信息整理为结构化的数据,构建出消费者画像,从而辅助决策或主动服务,实现精细化运营与精准营销。


\paragraph{AI 在金融领域落地面临困难和挑战3\sphinxfootnotemark[239]}
\label{\detokenize{chapter_project/AI_Finance:ai-3}}%
\begin{footnotetext}[239]\sphinxAtStartFootnote
\sphinxnolinkurl{http://www.ramywu.com/work/2018/05/18/AI-in-Finance-Survey/}
%
\end{footnotetext}\ignorespaces \begin{enumerate}
\sphinxsetlistlabels{\arabic}{enumi}{enumii}{}{.}%
\item {} 
深度学习模型的构建比较困难
目前并没有成熟的理论对深度学习模型的构造提供指导,主要还是依靠研究学者不断实验、不断探索

\item {} 
深度学习模型的稳健性和适用性有待商榷
深度学习模型能否适用于特定领域的分析和预测,需要大量实验进行验证。目前相关理论研究还处于对单一模型的优化处理,并没有提炼出通用的规律性方法和框架,从而限制了最终模型的稳健性和广泛适用性。

\item {} 
深度学习模型较难正确地阐述金融数据分析结果背后的经济学原理
深度学习模型在分析金融数据时,削弱了利用经济学解释最终结果的因果关系、以及隐藏于数据背后的经济学原理。

\end{enumerate}


\paragraph{2020 金融AI}
\label{\detokenize{chapter_project/AI_Finance:id5}}
金融科技进入“强监管”时代,行业合规有序发展◆金融科技行业正式进入“强监管”时代,市场的喧嚣与浮躁开始隐退,各类机构在探索创新与合规的平衡中不断前行。首份金融科技发展顶层文件出台,明确金融科技创新与服务的边界,整个行业进入合规有序发展阶段。金融机构积极拥抱金融科技,通过调整内部信息技术架构、成立科技子公司,推动技术从后台走向前台和中台,赋能业务发展。金融科技出海热潮持续进化,一批以提供获客、风控、运营等金融技术服务的企业开始扬帆远航,寻求新的发展机遇。整体来看,监管规范、新技术与金融业的融合应用、技术驱动下的经营模式与业务合作模式创新都是行业普遍关注和积极实践的焦点。

亿欧智库认为金融科技2020年十大关键词为:金融开放、金融科技监管、监管科技、消费金融、小微金融、开放银行、第三方支付、财富管理、保险科技。


\subsubsection{AI 健身}
\label{\detokenize{chapter_project/AI_fit:ai}}\label{\detokenize{chapter_project/AI_fit::doc}}

\paragraph{行业分析}
\label{\detokenize{chapter_project/AI_fit:id1}}
根据CBNData\&天猫发布的《2020健身大企业新趋势研究》,国人健身意识逐年增高,预计2020年健身器材市场规模将达500亿元。90后为主的新消费群体以及已婚已育女性成为健身大器械下单的主要人群。在器材中方面,跑步机成为家用场景下健身器材的中流砥柱。

张卓彧从投资角度进一步分析认为,国内健身市场正从休闲化向专业化发展,居家健身器材品牌需要硬件和内容双向升级,这更加抬高了健身器械市场的进入壁垒,赛道会逐步跑出头部玩家。而目前,国内健身器械品牌不仅在国内有广阔的发展土壤,并且有出海机会。

“比起欧美,中国有供应链优势。2020年,美国健身器材硬件进口市场中,来自中国大陆的产品比例占到64.5\%,来自中国台湾的产品占28.1\%。但未来当供应链发展到一定成熟阶段时,成本优势就没有太大差异了。所以健身器械品牌还需要做产品性能的提升和品牌溢价,未来健身器材智能化是一个发展方向。”张卓彧表示。

根据前瞻产业研究院发布的《2019年中国体育市场现状及发展趋势分析》统计,2019年,中国数字体育月活跃用户超1.2亿人。其中数字健身用户超2000万人。《2019\sphinxhyphen{}2020中国健身房市场发展白皮书》指出,Keep疫情期间月活跃用户量较2019年同期增长超过20\%。


\paragraph{痛点 3\sphinxfootnotemark[240]}
\label{\detokenize{chapter_project/AI_fit:id2}}%
\begin{footnotetext}[240]\sphinxAtStartFootnote
\sphinxnolinkurl{https://post.smzdm.com/p/andllwop/}
%
\end{footnotetext}\ignorespaces 
健身行业目前存在着诸多痛点:传统的健身来回通勤时间成本太高、健身教练鱼龙混杂、普遍缺乏数字化功能,或者是可以与用户实时交互的功能。所以健身过程比较枯燥,且不益于科学精准指导,用户粘性和健身积极性都不高。


\paragraph{例子}
\label{\detokenize{chapter_project/AI_fit:id3}}
2017年,天猫魔盒的研发者杜武平,就自立门户打造了一款全息互动训练系统——魔力智屏(Smart
Wall)。通过全触控的墙屏和地屏,对场内的人实现心率监测、运动检测、AI评分、墙地屏互动等功能。

坐落在北京望京区域的VENTO,就通过AI设备会对用户进行弹跳力、灵敏度、肌群柔韧度测试,然后将数据写入智能手环,与智能健身器材进行连接,来陪伴客户的整个运动过程,并实现自动化的调节。


\paragraph{失败}
\label{\detokenize{chapter_project/AI_fit:id4}}
也相继关闭了北京和上海的两家门店,单纯的智能设备始终无法取代人工,给用户带来私教式的指导。尽管节省了人力成本,却无力吸引到足够的会员来支撑运营。


\paragraph{产品体验 3\sphinxfootnotemark[241]}
\label{\detokenize{chapter_project/AI_fit:id5}}%
\begin{footnotetext}[241]\sphinxAtStartFootnote
\sphinxnolinkurl{https://post.smzdm.com/p/andllwop/}
%
\end{footnotetext}\ignorespaces 

\subparagraph{myShape}
\label{\detokenize{chapter_project/AI_fit:myshape}}
\sphinxurl{http://myshape.ai/mirror}

myShape包含了自研的3D动捕技术、姿态识别纠错算法、运动力学的专业知识以及一套强交互性的健身内容。

包含了自研的3D动捕技术、姿态识别纠错算法、运动力学的专业知识以及一套强交互性的健身内容。

TODO:


\paragraph{华为智慧屏 22\sphinxfootnotemark[242]}
\label{\detokenize{chapter_project/AI_fit:id6}}%
\begin{footnotetext}[242]\sphinxAtStartFootnote
\sphinxnolinkurl{https://consumer.huawei.com/cn/support/content/zh-cn00977206/}
%
\end{footnotetext}\ignorespaces 
适用版本:HarmonyOS 1.1, HarmonyOS 2.0

华为智慧屏的一大特色是拥有一个AI慧眼,除了支持视频通话,还有一个特色功能AI
健身,将你的客厅变身健身中心,智慧屏化身私教,操作步骤也很简单,只需语音说“我要健身”即可打开此功能,开启智慧健身第一步。\sphinxhref{https://zhuanlan.zhihu.com/p/87779620}{24}%
\begin{footnote}[243]\sphinxAtStartFootnote
\sphinxnolinkurl{https://zhuanlan.zhihu.com/p/87779620}
%
\end{footnote}

您可以按照以下方式操作:

1、在智慧屏主页选择 AI健身 。

3、进入AI健身应用,选择 推荐/健身专区/肩颈放松/瑜伽专区/广场舞专区
任意页签。

4、选择任意健身课程,在课程详情界面,选择 智能模式 或 普通模式
,根据界面提示,开始健身。

智能模式:摄像头实时收集您的健身动作和运动情况,并显示在智慧屏上,通过智能分析及时给出反馈和指导。

普通模式:智慧屏不会显示您的健身情况,您可以根据智慧屏的画面指导自行健身。

在智慧屏AI健身应用中,您可以选择健身训练课程,摄像头实时记录您的健身动作,并及时提醒指导,让您在家也能科学运动健身。

在屏幕上看到自己的健身姿态,可以和视频动作对比。

华为团队自研HUAWEI
Fitness算法,可以实时识别和追踪人体的骨骼关节点,通过关节点的位置变化识别用户的健身动作,并通过各个肢体在完成健身动作中的姿态变化客观地评价用户动作是否标准(完美,真棒,很好,加油。),让用户的运动更科学,效果更好

距离可能比较近,摄像头无法拍摄到用户的全身,此时比较难识别和评价用户的健身动作。作了针对性处理,只要摄像头捕捉到的用户肢体能满足当前健身动作的最核心的评价需求,即可完成动作的识别、评价和指导功能。


\subparagraph{3D体型追踪仪 VisbodyFit、Visbody D}
\label{\detokenize{chapter_project/AI_fit:d-visbodyfitvisbody-d}}
三维人体扫描国家标准起草单位唯一企业身份
\sphinxhref{https://www.visbodyfit.com/a/xinwenzixun/2021/0219/142.html}{20}%
\begin{footnote}[244]\sphinxAtStartFootnote
\sphinxnolinkurl{https://www.visbodyfit.com/a/xinwenzixun/2021/0219/142.html}
%
\end{footnote}
166项三维人体扫描技术领域专利数量居全球第一
清华\sphinxhyphen{}伯克利深圳学院人体视觉领域 唯一技术合作伙伴
西安电子科技大学成立图像视觉实验室唯一合作伙伴
谷歌、上海交大在顶级会议ECCV 联合发表体测领域专业论文
牛耳奖“人工智能领域年度最佳技术创新奖”
阿里巴巴诸神之战全AIoT赛道全球总决赛亚军\sphinxhref{https://www.visbodyfit.com/a/xinwenzixun/2020/0622/29.html}{16}%
\begin{footnote}[245]\sphinxAtStartFootnote
\sphinxnolinkurl{https://www.visbodyfit.com/a/xinwenzixun/2020/0622/29.html}
%
\end{footnote}
体测领域唯一一家国家级高新技术企业
西安市硬科技之星示范企业/西安市瞪羚企业 AIWIN医疗赛道TOP3
华为鲲鹏云生态合作伙伴

2017年,维塑以多年的技术研究为基础,成功研发了国内首款3D智能体测设备——维塑3D体型追踪仪。站在维塑上转一圈,即可收获个人专属的360°
可旋转3D人体模型,还有9项毫米级身体围度数据与12项身体成分分析数据。智能3D体态评估功能可以直观判断体态问题,帮助健身人群高效运动同时避免运动损伤等多种问题。维塑目前已销往全国200多个城市,用户超百万。目前VisbodyFit已经覆盖全国165
个城市超过
1500家健身机构,以先进设计理念,高新技术,优秀的用户体验打破了体测领域国外产品称霸的局面
\sphinxhref{https://www.visbodyfit.com/a/xinwenzixun/2020/0622/30.html}{17}%
\begin{footnote}[246]\sphinxAtStartFootnote
\sphinxnolinkurl{https://www.visbodyfit.com/a/xinwenzixun/2020/0622/30.html}
%
\end{footnote}
被超级猩猩、金吉鸟、李欣普拉提等知名品牌全线采购
\sphinxhref{https://www.visbodyfit.com/a/xinwenzixun/2020/1221/138.html}{21}%
\begin{footnote}[247]\sphinxAtStartFootnote
\sphinxnolinkurl{https://www.visbodyfit.com/a/xinwenzixun/2020/1221/138.html}
%
\end{footnote}

在2018年,维塑在三维人体扫描领域拥有百余项国家专利,获得了相关专利数量全国第一全球第三的成绩,并持有三维数字成像技术、基于深度学习的数字人体分析技术、云平台技术、BDA人体成分算法等多项核心技术。\sphinxhref{https://www.visbodyfit.com/a/xinwenzixun/2020/0622/31.html}{15}%
\begin{footnote}[248]\sphinxAtStartFootnote
\sphinxnolinkurl{https://www.visbodyfit.com/a/xinwenzixun/2020/0622/31.html}
%
\end{footnote}

首次新增“4象限重心平衡检测”,通过足底压力的分布检测人体重心平衡情况。
\sphinxhref{https://www.visbodyfit.com/a/xinwenzixun/2020/0611/27.html}{18}%
\begin{footnote}[249]\sphinxAtStartFootnote
\sphinxnolinkurl{https://www.visbodyfit.com/a/xinwenzixun/2020/0611/27.html}
%
\end{footnote}

维塑专利技术能实现从视觉传感器捕捉的人体信息中提取:生命体征(身高、体温、心跳以及血压等),身体成分(体脂率、骨骼肌、基础代谢以及无机盐含量等),身体围度信息、3D体型体态信息以及体能体质(包括体适能与关节灵活度分析等在内的人体动态评估)等多维度数据。

维塑3D体测仪被超级猩猩、金吉鸟、李欣普拉提等知名品牌全线采购

维塑科技希望通过搭建身体云平台,将用户、机构、行业各方的需求链接在一起,开启智能健康管理新局面:硬件设备将数据上传至云端进行计算和存储,个人用户可通过手机查看检测结果;企业客户可通过客户端进行管理;不同业态的合作伙伴可通过API接口集成数据,打破孤岛,实现价值最大化。\sphinxhref{https://www.visbodyfit.com/a/xinwenzixun/2020/0622/29.html}{16}%
\begin{footnote}[250]\sphinxAtStartFootnote
\sphinxnolinkurl{https://www.visbodyfit.com/a/xinwenzixun/2020/0622/29.html}
%
\end{footnote}

VisbodyFit赋能于各类大健康领域机构,包括医院相关科室、康复中心、健身中心、瑜伽馆、美容塑形机构以及家庭健身,帮助人们更好地掌控身体健康、塑造自己的健康生活。

2020年4月,维塑新品Visbody D正式发售。Visbody
D将作为顶级体测设备面世,带来更专业更高端的体测技术,并针对后疫情时期的客户需求做了更深的挖掘。\sphinxhref{https://www.visbodyfit.com/a/xinwenzixun/2020/1221/138.html}{21}%
\begin{footnote}[251]\sphinxAtStartFootnote
\sphinxnolinkurl{https://www.visbodyfit.com/a/xinwenzixun/2020/1221/138.html}
%
\end{footnote}
\begin{enumerate}
\sphinxsetlistlabels{\arabic}{enumi}{enumii}{}{.}%
\item {} 
Visbody
D支持手势操作,用户只需挥手即可完成选择及确认操作。手势识别技术的难点在于准确率。每台Visbody
D出厂均经过万次测试,确保用户拥有便捷流畅的体验。

\item {} 
除头部、肩部、骨盆的评估外,Visbody
D新增X/O/K/D腿型及膝关节评估,并增加不良体态可能导致的风险预警,让体态评估更加符合用户需求及认知水平。

\item {} 
加高精度压力传感器,通过足底压力的分布检测人体重心平衡情况

\item {} 
Visbody
D身体成分检测技术已升级为医疗级,可检测细胞内外液含量,通过细胞外液和细胞内液比率可判断是否存在体液失衡,从而评估营养情况,判断免疫力水平。

\item {} 
3年前,本地存储+单次结果+纸质报告还是体测的唯一选择。维塑首次推出云端档案+历史数据查看+微信报告功能

\end{enumerate}

\begin{figure}[H]
\centering
\capstart

\noindent\sphinxincludegraphics{{Visbody}.jpg}
\caption{Visbody D \& R}\label{\detokenize{chapter_project/AI_fit:id17}}\end{figure}


\subparagraph{FITURE魔镜}
\label{\detokenize{chapter_project/AI_fit:fiture}}
FITURE(成都拟合未来科技有限公司)致力于通过科技帮助大众建立健康的生活方式。提供“硬件+AI+内容+服务”一体式健康生态

FITURE魔镜解决的痛点不只是场景、成本等问题,它其实是为用户提供智能健康综合解决方案的科技产品。获得ELLEMEN
2020理容大奖榜单之“年度最佳科技产品”

“FITURE Motion
Engine”智能运动追踪系统,是适应各种极端场景的人体检测模型、高精度的姿态识别模型,是抽象化连续姿态的动作识别引擎,使得你站在这面魔镜前,无需穿戴任何产品或传感器辅助,你的一举一动都会被镜面上的摄像头和传感器捕捉,这些信息会成为判断标准,系统会通过屏幕里的AI教练会实时指导你的的动作姿势。

在家单独锻炼难免会缺少氛围,FITURE魔镜的游戏化互动健身课程,提供即刻置身虚拟游戏世界的沉浸式体验,为家庭健身增添了许多趣味性,用户甚至可以发起线上挑战赛,有社交的健身才更有动力。

FITURE于2020年9月底完成了A轮融资,并刷新了全球健身行业A轮融资的纪录,成为红杉资本、腾讯、C资本、凯辉基金、黑蚁资本、CPE(中信产业基金)、BAI(贝塔斯曼亚洲投资基金)、全明星基金等头部基金追捧的宠儿。
\sphinxhref{https://coffee.pmcaff.com/article/13654236\_j}{8}%
\begin{footnote}[252]\sphinxAtStartFootnote
\sphinxnolinkurl{https://coffee.pmcaff.com/article/13654236\_j}
%
\end{footnote}

NFS2020年度CEO峰会暨猎云网创投颁奖盛典上,FITURE
入选“2020年度最具投资价值创新企业TOP20”。

在此之外,AI健身myShape,已于今年9月推出旗下首个智能健身镜,并与器械公司乔山合作推出健身镜产品。而近乎同一时间,健身O2O平台沸腾时刻也推出智能健身镜。同样在中国香港的健身市场,刚刚上市的健身内容公司OliveX,发布了第一款健身镜产品。

抛开产品细节,各家健身镜的功能近乎相似。镜子硬件、课程内容、AI交互,构成核心的功能模块。相比跑步机、单车,做家庭健身镜在成为更多中国健身公司的选择。\sphinxhref{https://www.visbodyfit.com/a/xinwenzixun/2020/1130/132.html}{14}%
\begin{footnote}[253]\sphinxAtStartFootnote
\sphinxnolinkurl{https://www.visbodyfit.com/a/xinwenzixun/2020/1130/132.html}
%
\end{footnote}


\subparagraph{面试问题参考: 9\sphinxfootnotemark[254]}
\label{\detokenize{chapter_project/AI_fit:id7}}%
\begin{footnotetext}[254]\sphinxAtStartFootnote
\sphinxnolinkurl{https://coffee.pmcaff.com/article/2729281195713664/pmcaff?utm\_source=forum}
%
\end{footnotetext}\ignorespaces \begin{enumerate}
\sphinxsetlistlabels{\arabic}{enumi}{enumii}{}{.}%
\item {} 
请问在独立推动项目时,遇到的最困难的经历是什么样的?请详细说明为什么这些点难以解决,及你的应对策略。

\item {} 
你认为疫情下,用户对日常锻炼的需求会有什么变更?长期健身者的用户画像会有什哪些可能的变化?

\item {} 
有没有自己比较满意的项目经历,解释为什么会觉得满意。是什么原因让项目成功?

\end{enumerate}


\subparagraph{Keep AI 虚拟教练 6\sphinxfootnotemark[255]}
\label{\detokenize{chapter_project/AI_fit:keep-ai-6}}%
\begin{footnotetext}[255]\sphinxAtStartFootnote
\sphinxnolinkurl{https://coffee.pmcaff.com/article/12061874\_j}
%
\end{footnotetext}\ignorespaces 
为了追寻我们最初的目标:让用户能够最有效的运动。让更多的人运动起来。

人们日益增长的健身需求,和健身教练数量不足之间的矛盾。为用户提供个性化的专属训练计划

Keep 展示了
TOF(深度摄像头)动作打分的新技术。通过拍摄用户运动过程,Keep 的 APP
可以在你锻炼的时候进行动作指导,并实时提示标准度。

Keep
成立了人工智能研究院,秦曾昌博士任首席科学家兼人工智能研究院的院长。在和一家国内手机厂商合作,很快就会推出基于深度摄像头的应用。「即使是一张平面的照片,我们也可以重建出
3D 的人体姿态。」

新品Q60智能电视,搭载了Keep AI大屏互动健身产品,基于Keep
动作库和数据模型,结合Keep的AI算法,借助电视端摄像头功能,在捕捉用户动作轨迹的同时进行标准度打分和提供健身指导。同时,Keep
AI大屏互动健身产品整套操作系统中加入了智能语音控制,通过对话便可控制开始、结束、切换动作等流程。\sphinxhref{https://coffee.pmcaff.com/article/13242929\_j}{7}%
\begin{footnote}[256]\sphinxAtStartFootnote
\sphinxnolinkurl{https://coffee.pmcaff.com/article/13242929\_j}
%
\end{footnote}


\paragraph{快快第二代智能健身系统搭 10\sphinxfootnotemark[257]}
\label{\detokenize{chapter_project/AI_fit:id8}}%
\begin{footnotetext}[257]\sphinxAtStartFootnote
\sphinxnolinkurl{https://coffee.pmcaff.com/article/13646585\_j}
%
\end{footnotetext}\ignorespaces 
配合上课的智能运动臂带,0.01秒监测人体脉搏波形,媲美医用级心率带,每次课后都会进行72项指标的运动评估,为用户带来了“千人千面”智能健身体验;快快智能运动膝带,采用双腿4颗9轴传感器,支持跑步模式+课程模式,提高运动效率还能保护膝盖安全。

快快第二代智能健身系统搭载的165吋智慧大屏,两个乒乓桌的面积大,可以做到1:1真人等比上课,自研解码算法,在国内率先实现了60fps高帧率物理分辨率6K直播技术。在运动过程中,系统还支持每秒数百组数据实时回显,可满足多人多组PK赢取金币,让运动娱乐化。

可以说,快快第二代智能健身系统通过搭建高清数字矩阵和终端智慧教室,加速了健身场景的智能化改造,用自己的AI能力去给更多产品赋能,进入到了一个技术和数据推动的阶段。


\paragraph{Freeletics应用程序 12\sphinxfootnotemark[258]}
\label{\detokenize{chapter_project/AI_fit:freeletics-12}}%
\begin{footnotetext}[258]\sphinxAtStartFootnote
\sphinxnolinkurl{https://ai.51cto.com/art/202011/632683.htm}
%
\end{footnotetext}\ignorespaces 
Freeletics应用程序,它利用一系列AI流程来创建自定义锻炼,然后对其进行维护和修改以优化用户的喜好和发展。Freeletics应用程序首先收集少量个人数据。然后,它与其他用户和锻炼的海量数据库进行交叉引用,以创建建议的开始程序。

然后,该应用程序将跟踪用户的进度并接受反馈,以继续使他们的锻炼达到满意程度。无论是一般健身,针对单个肌肉群或身体部位,减肥或其他健身目标,Freeletics都使用机器学习。通过机器学习,用户将获得反馈,该反馈将使用户长期坚持的常规操作归零。其他应用程序使用人体姿势估计来检测和分析体育活动中的人体姿势。


\paragraph{Asensei智能穿戴}
\label{\detokenize{chapter_project/AI_fit:asensei}}
Asensei是一家智能服装开发人员,提供一套衬衫和裤子,能够跟踪用户进行的诸如弓步,下蹲和类似常规动作等身体运动的运动。Asensei智能服装使用运动捕捉和AI技术将用户的角度和运动范围与可接受的运动形式规范进行比较,并可以实时纠正用户以养成良好的运动习惯。

Sensoria提供了类似的基于AI的可穿戴系统,该系统专门针对慢跑和跑步而设计。Sensoria平台从智能服装(Sensoria自己的服装或其他具有IoT功能的服装)中收集数据。数据可测量一系列运动和生物特征。这包括心率,脚着地和脚踏的速度以及跑步时的冲击力。

Sensoria分析不仅为锻炼程序提供了优化和改进的建议,而且可以监视和发现等待中的潜在伤害并识别动力学链中的薄弱环节。Sensoria系统的设计既注重健康又适合从事积极生活方式的用户。


\paragraph{运动技术的可穿戴设备}
\label{\detokenize{chapter_project/AI_fit:id9}}
纳迪(Nadi)是一位专注于瑜伽的智能服装创造者,使用许多相同的AI和身体捕捉技术来生产智能瑜伽裤。

这些绑腿以无线方式连接到用于移动设备的应用程序,并且该移动应用程序提供了通过预定例程进行的瑜伽教程。同时,绑腿本身利用一系列轻柔的振动来提供指导,以指导例程的每个步骤都应关注身体的哪些部位。


\subparagraph{健身环大冒险}
\label{\detokenize{chapter_project/AI_fit:id10}}
去年10月18日上市时,官方销售价格(含)为79.99美元,约合560元人民币。2月22日,京东平台上的游戏报价已达1799+元,较最初的官方定价提升3倍,一度卖到断货。由此,《健身环大冒险》被网友们戏称为“2020年度最佳理财产品”。3月12日,游戏在淘宝平台售价跌为1248元。尽管后期有小幅上涨,但当人们开始复工复产后,价格就一直回落,最终下降到1000元左右。

AI私教又不够人性化,不能检测到玩家的每一个动作是否标准,也不晓得玩家是否穿着拖鞋、瘫在床上来偷懒作弊。


\subparagraph{超级猩猩APP 21\sphinxfootnotemark[259]}
\label{\detokenize{chapter_project/AI_fit:app-21}}%
\begin{footnotetext}[259]\sphinxAtStartFootnote
\sphinxnolinkurl{https://www.visbodyfit.com/a/xinwenzixun/2020/1221/138.html}
%
\end{footnotetext}\ignorespaces 
\begin{figure}[H]
\centering
\capstart

\noindent\sphinxincludegraphics{{fit_app}.png}
\caption{来源:MobData研究院公开资料}\label{\detokenize{chapter_project/AI_fit:id18}}\end{figure}

区别于超级猩猩的线上直播课程,目前所有App端的课程对有邀请资格的用户免费。根据超级猩猩线上负责人刻奇透露,本次公测随机邀请部分活跃猩友参与体验,首日邀请用户超过500人,并将逐步扩大邀请范围。

而社群功能在超级猩猩App中被设定为「圈子」,目前没有对全体用户开放创建圈子的权限,仅认证后的教练和官方可以创建,这样主要是为了方便形成良好的讨论氛围,降低用户探索成本,之后会逐步开放。

此外,用户还可在APP上通过教练账户页面直接预约课程,还可以在课中和课后,使用门禁密码推送和照片推送功能。据官方称,推出APP旨在了解用户需求、完善服务场景和提高用户体验。

App是精细化运营的需求。超级猩猩全国门店数量约100家,累计会员数量100万人。从服务会员体量上来说,仅靠微信生态的线上运营用户的方式已不能满足超级猩猩更加精细化的服务需求,另外目前超级猩猩扩店的进度早已放缓,从而企业的重心转移到更精细化的用户运营上,接下来就是如何用线上做存量和增量。

疫情期间,超级猩猩在「一直播」平台直播,一次直播几个教练带动十万人参与运动,明星教练的效率前所未有的高,但效益是怎样的呢?

App是精细化运营的需求。超级猩猩全国门店数量约100家,累计会员数量100万人。从服务会员体量上来说,仅靠微信生态的线上运营用户的方式已不能满足超级猩猩更加精细化的服务需求,另外目前超级猩猩扩店的进度早已放缓,从而企业的重心转移到更精细化的用户运营上,接下来就是如何用线上做存量和增量。

而在2020年12月主打“零售课程”已经拿到D轮融资的超级猩猩截至2020年底累计开店数达113家。


\paragraph{优质教练}
\label{\detokenize{chapter_project/AI_fit:id11}}
就国内健身器械商转型对标的Peloton而言,用户月活高达21.1次,相比于去年同期的12.6次,翻了将近一倍。

高月活的背后,是平台上的优质教练。类似于SoulCycle的明星教练,在美国社媒Reddit和Instagram上,用户对于Peloton明星教练,如粉丝般的狂热追逐,热度不亚于Netflix的演员,对明星教练和优质内容的认可,是构成用户几乎每天使用Peloton的关键。


\paragraph{更多 23\sphinxfootnotemark[260]}
\label{\detokenize{chapter_project/AI_fit:id12}}%
\begin{footnotetext}[260]\sphinxAtStartFootnote
\sphinxnolinkurl{https://36kr.com/p/1087630409318663}
%
\end{footnotetext}\ignorespaces 
咕咚、75派推出智能跳绳,通过传感器计数,还可将数据同步到手机APP,乔山旗下品牌Matrix
Fitness引入iFit交互平台,金史密斯、云麦加入小米生态链体系,舒华跑步机也加入到华为DFH闭环生态圈等。


\paragraph{技术能力}
\label{\detokenize{chapter_project/AI_fit:id13}}
依赖于开发者在三维动态捕捉、深度学习建模等领域的技术能力
\sphinxhref{https://www.tmtpost.com/4257148.html}{1}%
\begin{footnote}[261]\sphinxAtStartFootnote
\sphinxnolinkurl{https://www.tmtpost.com/4257148.html}
%
\end{footnote}

微软研究院利用消费者手中的智能手机相机进行远程医疗等领域的非接触式生理测量。测量人体随时间而产生的细微变化,而通过捕捉这些肉眼无法察觉的变化能够提供非常多的生理信息。\sphinxhref{https://ai.51cto.com/art/202012/633705.htm}{13}%
\begin{footnote}[262]\sphinxAtStartFootnote
\sphinxnolinkurl{https://ai.51cto.com/art/202012/633705.htm}
%
\end{footnote}


\paragraph{体感游戏 4\sphinxfootnotemark[263]}
\label{\detokenize{chapter_project/AI_fit:id14}}%
\begin{footnotetext}[263]\sphinxAtStartFootnote
\sphinxnolinkurl{https://www.infoq.cn/article/qiciiwtdpujamorfuijq}
%
\end{footnotetext}\ignorespaces 
体感游戏不仅仅意味着手机产品可以打开全新的游戏类别,更多的是可以与智能穿戴、大屏产品以及其他类似于外置手柄一类的外设连带售卖。这与手机厂商近来拓宽产品线、增加
IoT SKU 的经营理念的相符的。在 PC
版“吃鸡”风行时,游戏配置对于硬件的高要求甚至掀起了一阵换机潮。或许一款优秀的手机体感游戏,也能带动很多
IoT 产品的出售。


\paragraph{问题 11\sphinxfootnotemark[264]}
\label{\detokenize{chapter_project/AI_fit:id15}}%
\begin{footnotetext}[264]\sphinxAtStartFootnote
\sphinxnolinkurl{http://www.woshipm.com/ai/990247.html}
%
\end{footnotetext}\ignorespaces \begin{enumerate}
\sphinxsetlistlabels{\arabic}{enumi}{enumii}{}{.}%
\item {} 
用户决策成本高,健身房售课缴费动辄包年包季,可看到健身效果却需要3\sphinxhyphen{}6个月。

\item {} 
从业人员复制率低,培养一名优秀教练的难度不亚于培养一名医生,现如今整个健身行业都以销售为导向,教练缺乏经验,自然也不能给用户提供好的服务。

\item {} 
行业中同质化竞争严重,标准化的团操、标准化的设备、标准化的收费模式之下,商家很难建立起壁垒,也容易陷入无底线的竞争。

\end{enumerate}


\paragraph{机会 11\sphinxfootnotemark[265]}
\label{\detokenize{chapter_project/AI_fit:id16}}%
\begin{footnotetext}[265]\sphinxAtStartFootnote
\sphinxnolinkurl{http://www.woshipm.com/ai/990247.html}
%
\end{footnotetext}\ignorespaces \begin{itemize}
\item {} 
用户层面的体验机会。像Enflux、肌动科技和其他智能设备、智能健身房的出现,可以对用户健身的体验和效率进行改变,逐渐减少健身中的用户决策成本。

\item {} 
行业层面的壁垒机会。当AI带来健身体验的个性化,用户在进行决策时考虑也就不仅仅是场馆距离家近不近、费用多少这些容易被无底线竞争破坏的因素,而开始考虑这家健身房的解决方案是否适合自己。

\item {} 
产业层面的模式机会。当体验更好的智能健身解决方案普及度越来越高时,整个健身行业也就不再囿于卖课、卖卡、卖加盟这样单一的盈利方式。在未来专门适用于健身的“体育云”、健身算法训练师、更丰富的智能设备等等都可能出现,整个健身产业的营收模式会变得更加丰富。

\end{itemize}


\paragraph{Amesome}
\label{\detokenize{chapter_project/AI_fit:amesome}}
\sphinxurl{https://36kr.com/projectDetails/79445}


\subsubsection{AI 硬件 1\sphinxfootnotemark[266]}
\label{\detokenize{chapter_project/AI_hardware:ai-1}}\label{\detokenize{chapter_project/AI_hardware::doc}}%
\begin{footnotetext}[266]\sphinxAtStartFootnote
\sphinxnolinkurl{https://www.jianshu.com/p/111d9fcc005e?utm\_campaign=maleskine\&utm\_content=note\&utm\_medium=seo\_notes\&utm\_source=recommendation}
%
\end{footnotetext}\ignorespaces 

\paragraph{方案+研究+生产}
\label{\detokenize{chapter_project/AI_hardware:id1}}
做方案:
\begin{enumerate}
\sphinxsetlistlabels{\arabic}{enumi}{enumii}{}{.}%
\item {} 
研究市场需求

\item {} 
出MDR,定义市场需求(定目标人群、定卖点、定关键特性、定价)

\item {} 
和ID(工业设计)团队一起出方案,出效果图

\item {} 
拿方案在内部评审可行性,直到比较有把握

\item {} 
ID开始设计和打样,做Mockup

\item {} 
做市场调查,找用户提提意见

\item {} 
拿方案去对客户宣讲,看看客户的反馈(一般是合作过的老客户,比如运营商、渠道商;新客户都是拿成品去拓展的)

\item {} 
根据市场调研反馈修改设计

\end{enumerate}

研发和生产阶段:
\begin{enumerate}
\sphinxsetlistlabels{\arabic}{enumi}{enumii}{}{.}%
\item {} 
ID团队出正式的方案(中间需要反复修改设计、去喷漆厂调色、做高保真Mockup)

\item {} 
MD(结构设计)出结构方案

\item {} 
硬件出硬件方案

\item {} 
MD和硬件一起选购元器件,定BOM

\item {} 
软件研发软件方案

\item {} 
开模,需多次修模

\item {} 
量产

\end{enumerate}


\paragraph{硬件AI PM}
\label{\detokenize{chapter_project/AI_hardware:ai-pm}}
硬能力核心流程如下图:

\begin{figure}[H]
\centering
\capstart

\noindent\sphinxincludegraphics{{hardwareAI_PM}.png}
\caption{硬件AI PM}\label{\detokenize{chapter_project/AI_hardware:id26}}\end{figure}


\paragraph{硬件PM VS 互联网PM}
\label{\detokenize{chapter_project/AI_hardware:pm-vs-pm}}

\subparagraph{知识面不同: 2\sphinxfootnotemark[267]}
\label{\detokenize{chapter_project/AI_hardware:id2}}%
\begin{footnotetext}[267]\sphinxAtStartFootnote
\sphinxnolinkurl{http://www.woshipm.com/pmd/1815501.html}
%
\end{footnotetext}\ignorespaces \begin{itemize}
\item {} 
软件产品经理通常需要关注市场和用户调研和分析、产品设计、用户体验、UI设计、开发技术、数据分析、运营维护、拉新促活和召回、商业变现、推荐推送等方面的知识。

\item {} 
硬件产品经理除了需要关注市场和用户调研和分析、产品设计之外还需要关注、方案设计、ID开模、电子电路、成本控制、包装设计、质量把控、营销渠道、售后服务等各方面知识。

\end{itemize}


\subparagraph{合作人不同:}
\label{\detokenize{chapter_project/AI_hardware:id3}}
\begin{figure}[H]
\centering
\capstart

\noindent\sphinxincludegraphics{{PM_Internet_VS_hardware}.png}
\caption{硬件PM VS 互联网PM}\label{\detokenize{chapter_project/AI_hardware:id27}}\end{figure}
\begin{itemize}
\item {} 
互联网产品经理一般合作的团队成员主要有设计、开发、测试、运营、市场这几个主要岗位,如果是大公司还有需求分析师、交互设计师、用户研究员等相关岗位。

\item {} 
硬件产品经理一般是和ID设计、平面设计师、结构设计师、电子工程师、软件工程师、采购、品控、销售、售后、技术支持、仓库管理员、供应商、代工厂、模具成、SMT厂、包装厂等相关职位以及合作伙伴进行合作。因为需要面对很多外部人员以及多方合作,所以硬件产品经理也需要具有来更好地协调能力和规划能力。

\end{itemize}


\subparagraph{定位不同:参谋 VS 领班 3\sphinxfootnotemark[268]}
\label{\detokenize{chapter_project/AI_hardware:vs-3}}%
\begin{footnotetext}[268]\sphinxAtStartFootnote
\sphinxnolinkurl{http://www.woshipm.com/pmd/134575.html}
%
\end{footnotetext}\ignorespaces \begin{itemize}
\item {} 
硬件团队中职能团队要广的多,老板难以掌控所有的工作,所以必须专设两类助手——产品经理(找业务方向)和项目经理(推动项目执行)。尽管PM也没有管人的”实权“,但因为身背业绩,所以在业务的发言权要大很多——比如例会上,一般是老板坐镇,所有团队负责人参与,然后产品经理和项目经理轮流汇报工作。在这个过程中,任何问题都可能被提出来,由对应职能部门负责人解释。一旦被被挑战,压力非常大。

\item {} 
互联网团队通常采用扁平化管理,产品、设计、研发、运营都是彼此协作的
职能团队,职位高低上并无不同。由于团队中一般不专设项目经理,就把PM抓来同时兼任。由于这种”项目经理“并无分配资源的”实权“,又必须推动项目实
施,所以”产品汪“天天跪舔”攻城狮“就一点都不奇怪了。

\end{itemize}


\subparagraph{KPI不同:收入考核 VS 用户考核}
\label{\detokenize{chapter_project/AI_hardware:kpi-vs}}\begin{itemize}
\item {} 
硬件产品没有互联网那么多眼花缭乱的商业模式。硬件PM的KPI非常简单粗暴,就是收入和利润是否达标。无论市场眼光准不准、方案完成度高不高、产品有没有竞争力、情怀能不能打动人,最终都体现在一件事上——能不能把东西卖掉并赚到钱!

\item {} 
互联网产品一般不向用户收费,很多产品在上线多年后都不能变现,用收入考核产品经理根本不现实。所以产品经理一般是与运营一起背用户增长、流量等。

\end{itemize}


\subparagraph{关注点不同:商业价值 VS 用户体验}
\label{\detokenize{chapter_project/AI_hardware:vs}}\begin{itemize}
\item {} 
互联网PM更关注完整的用户体验,对用户心理的揣摩更细腻,比如用切换卡导航还是抽屉导航,用图标按钮还是文字链等等。

\item {} 
硬件PM虽然也关注用户体验,但更需要衡量每一点改善的投入产出比,如果不能带来销量的提升就尽可能砍掉。比如我从来没见过哪个传统电视机厂家把遥控器做的好用一些,因为这对销售帮助极小。

\end{itemize}


\subparagraph{项目追求不同:高完成度VS快速上线}
\label{\detokenize{chapter_project/AI_hardware:id4}}\begin{itemize}
\item {} 
发布硬件产品,则需要从市场到设计、研发、生产、销售、售后全部团队的重度参与,一旦量产就不能容许有严重BUG。所以在立项起各阶段的检查就需要极为严格,在方案达标后必须及时关闭迭代,保证产品的高完成度。硬件产品试错成本很高,必须谨慎计划,严肃执行。

\item {} 
互联网常常拉一拨人几个月就上线顺便公测,效果不错就加大投入,效果不佳就赶紧换方向或砍掉。上线后就算有严重BUG,也不算事儿,有问题下一版升级就好了。互联网产品试错成本很低,可以小步快跑,随时掉头

\end{itemize}


\subparagraph{成本意识不同:非常关注成本VS不太关注成本}
\label{\detokenize{chapter_project/AI_hardware:id5}}\begin{itemize}
\item {} 
硬件对配置斤斤计较:由于每件硬件产品最终都是一笔交易,所以控制成本就极为重要。因此从元器件采购到确定BOM都需要PM深度参与,当然专业意见还是技术部门出,订货过程由项目经理跟。

\item {} 
互联网产品不考虑成本:一般初期投入成本很低(只有一些人力投入),用户增长的边际成本也极低。所以互联网PM基本不太需要考虑成本,更关心如何吸引用户。

\end{itemize}


\paragraph{互联网思维对硬件的束缚}
\label{\detokenize{chapter_project/AI_hardware:id6}}
互联网的“反硬件基因”,能用app解决的事情,坚决不用硬件


\paragraph{AI产品商业简史}
\label{\detokenize{chapter_project/AI_hardware:ai}}\begin{itemize}
\item {} 
三大应用:语音、视觉、机器翻译

\item {} 
四大品类:智能音响、家庭机器人、翻译机、AI相机

\end{itemize}


\paragraph{智能音响商业史}
\label{\detokenize{chapter_project/AI_hardware:id7}}
Echo诞生之初着力点是运算能力和高音质,因此价格199美金。此时市场为蓝海,通过降价策略吸引了一部分观众后,用户体验后感觉良好。随后google感到危机加入战场,凭借其互联网搜索引擎的家底占据了一部分市场。随着技术的成熟,价格下降成为趋势,通过低价位,echo进一步巩固市场地位,google随后也向低档下手。最后屏幕音响扰乱市场。


\subparagraph{智能音响商业策略}
\label{\detokenize{chapter_project/AI_hardware:id8}}
三大基本策略:应用渗透(以产品服务的渗透率为第一目标)+生态延伸(自身产品生态的延伸,如HomePod)+价值割据(围绕用户价值改进产品,建立优势巩固壁垒)

企业需要考虑的:战略贯穿(以引流为目的,盈利优先级低)+结局导向(确保能落地)+全局商战(企业的利弊权衡+需求程度和用户接受度+软件加硬件加商业化运作)


\subparagraph{智能机器人发展史}
\label{\detokenize{chapter_project/AI_hardware:id9}}
14年的妖风初起,资本夸大机器人市场;15年巅峰在望,消费级机器人品类增加;16年由盛转衰,回归理性;17年资本降温,强弩之末,或突围或止损;18年回首,一地鸡毛。

底层需求与价值:教育(早教机器人、编程机器人)、娱乐(玩具)、效率(扫地机器人、音响)

极点产品设计:极点形态(用户选择)、极点功能(需求)、合理的价格区间

商用机器人市场大于消费级机器人,仅仅炫技而无法落地的机器人很难生存

商用机器人:物流机器人+农场机器人+安防机器人+公关+外骨骼+医疗


\subparagraph{智能翻译机发展史}
\label{\detokenize{chapter_project/AI_hardware:id10}}
翻译机相对app的卖点:使用可靠性、识别准确性、操作简易性

翻译机的商业策略:产品演化+抢占市场


\subparagraph{机器视觉产品应用}
\label{\detokenize{chapter_project/AI_hardware:id11}}
技术赋能,给老产品带来新体验


\paragraph{互联网思维做不好AI硬件}
\label{\detokenize{chapter_project/AI_hardware:id12}}
\begin{figure}[H]
\centering
\capstart

\noindent\sphinxincludegraphics{{Internet_VS_hardware}.png}
\caption{互联网思维和硬件思维的差异}\label{\detokenize{chapter_project/AI_hardware:id28}}\end{figure}
\begin{itemize}
\item {} 
功能:互联网思维是设计功能、满足需求,硬件领域则是要达到用户的预期。如Echo

\item {} 
设计:软件产品强调极致,硬件产品关注全局整体性。如AirPods

\item {} 
价格:互联网的免费思维背后是流量,硬件则应该一开始就考虑产品定位和定价。硬件的渗透是渐进的,无法复制软件的导流。

\item {} 
开发机制:软件快节奏,容错能力强;硬件重质量,容错能力弱

\end{itemize}


\paragraph{为什么体验好的产品,卖不好}
\label{\detokenize{chapter_project/AI_hardware:id13}}
体验最好不代表最合适。合适的重要性远大于体验,提升体验着成本的上身,用户只买对的,不买贵的。销量是衡量产品最普适的指标。


\paragraph{AI硬件的3种模式}
\label{\detokenize{chapter_project/AI_hardware:ai3}}\begin{itemize}
\item {} 
硬件模式(AI+硬件):硬件是主体,AI可有可无。如智能手机的拍照、无人机的镜头、加入新技术的玩具、带有语音助手的耳机、智能手表、智能家居(音响主控,其他的家电被连接)、相机、眼镜

\item {} 
互联网模式(AI+管道):智能音响、智能翻译机、智能电视、家庭监控

\item {} 
资本模式(资本+梦想):下一代交互/运算平台(消费者购买的是产品,不会为梦想买单,如TNT)+AI机器人(技术or泡沫?)

\end{itemize}


\paragraph{AI硬件创新}
\label{\detokenize{chapter_project/AI_hardware:id14}}\begin{itemize}
\item {} 
用户决策:轻决策(门槛低、风险低、心动就会买,忌花里胡哨抬高价格)+重决策(门槛高、风险高、没有必要就不买,不能妥协性能)

\item {} 
产品演进:压缩成本(轻决策,平民化)+提高价值(重决策,价值穿透)

\item {} 
推广路径:轻决策,依靠平价,快速渗透;重决策,穿透核心价值用户后才能抵达大众,强贯穿

\end{itemize}


\paragraph{智能硬件}
\label{\detokenize{chapter_project/AI_hardware:id15}}
智能硬件看似复杂,拆解出来脉络很清晰。包含硬件(HW)、软件(SW)、外观(ID)、结构(MD)、互联网平台。

其中软件包含板级支持包(BSP)、底层引导程序(bootloade)、系统与应用程序、算法,这些不展开来讲,找固件打包的工程师就
OK ,一般所有的程序都汇总到他那儿了。

作为项目经理,不太需要进行深入的了解,当然能够深入更好,但作为产品经理还是更深入一点较好。

互联网平台,这个包含云服务、后台、App、小程序等。常见的是前三个。跟进对应的工程师就好。

\begin{figure}[H]
\centering
\capstart

\noindent\sphinxincludegraphics{{hardwareAI_flow_chart}.png}
\caption{AI 智能硬件流程图}\label{\detokenize{chapter_project/AI_hardware:id29}}\end{figure}


\paragraph{项目研发}
\label{\detokenize{chapter_project/AI_hardware:id16}}
项目研发分为EVT阶段、DVT阶段、PVT阶段、MP阶段和维护阶段

EVT 阶段:(Engineering Verification
Test),指工程验证。一般在工程样机之前的研发行为,我都称之为工程验证。

这个阶段,目的是工程验证。尽可能的发现设计问题,方案对比。

最终拿到的是工程样机,用于样机整机测试,判定是否可以开模。

DVT 阶段:(Design Verification
Test),指设计验证测试。最终拿到的是试产的整机样机,用于多方联调,验证优化。

上一个阶段,完成产品的雏形,这个阶段继续上个阶段的设计开发、优化。MD
详细设计完成,开始投模、试模、修模、颜色调制等。

试产模具,组装整机,进行硬件/结构的整机测试。软硬件、结构、互联网平台多方联调。比如软硬件的稳定性、可靠性、性能等;软件与互联网平台(云服务/App等)联调测试;硬件与结构的联调测试,比如散热、结构强度等。

另外,这在这阶段关于产品的贴纸、说明书、包装等可以开始设计/打样,然后等待,因为这些时间周期比较短。

如果软硬件状态比较理想,在这个阶段尽早安排认证。因为认证周期非常长,基本在
40 天左右,别等到产品快要量产了,认证还没出来,影响销售。

总之,这个阶段就是联调、测试、试模、打板、试产。

PVT 阶段 :(Process Verification
Test),指生产验证。进行小批量产,摸清生产工艺,测试工艺,为大批量产做准备。

这个阶段依然会进行各种验证,以及解决上一阶段遗留的一些小问题。但主要的精力放在一致性、设计(细节,比如按键手感不好,干涉等)调整上。

各部门处于生产支持模式,比如工程部制作
SOP(标准作业程序),结构部帮忙解决生产上的结构问题。与生产相关的测试工具、生产工具、烧录工具、产测工具的支持。

所有的生产支持文件规定当送到工厂,量产软件/量产硬件BOM/量产结构BOM,结构/元器件终版签样。

总之,这个阶段就是为了保证产品量产。
量产顺利,效率高,不良率最低,产品一致性够高。


\paragraph{智能硬件设计流程 5\sphinxfootnotemark[269]}
\label{\detokenize{chapter_project/AI_hardware:id17}}%
\begin{footnotetext}[269]\sphinxAtStartFootnote
\sphinxnolinkurl{https://weread.qq.com/web/reader/40632860719ad5bb4060856k98f3284021498f137082c2e}
%
\end{footnotetext}\ignorespaces 
智能硬件从智能穿戴设备开始,在智能硬件领域已经扩展出了诸如智能电视、智能家居、智能汽车、医疗健康、智能玩具、机器人等人工智能应用。如今比较典型的智能设备包括Google
Glass、三星Gear、Fitbit、麦开水杯、咕咚手环、Tesla、无屏电视等。智能硬件涉及领域广泛,与此相关的行业也非常多。一个完整的智能硬件产品通常拥有一个包含双流程的产品设计流程,如下图所示。

\begin{figure}[H]
\centering
\capstart

\noindent\sphinxincludegraphics{{hardware_design}.png}
\caption{智能硬件产品设计流程}\label{\detokenize{chapter_project/AI_hardware:id30}}\end{figure}


\subparagraph{需求分析}
\label{\detokenize{chapter_project/AI_hardware:id18}}
确认整体的业务场景、了解应用的技术、明确需要满足人什么需求,甚至对整体市场的情况进行评估,这个阶段是AI产品经理调研需求定义产品的阶段,是一个智能硬件产品生命周期的开始。


\subparagraph{产品形态定义}
\label{\detokenize{chapter_project/AI_hardware:id19}}
AI产品经理在这个阶段需要完成产品的整体方案,包括硬件和软件的相关功能,将产品形态,以文字、图片、模型等方式展示出来,完成对产品形态的定义。


\subparagraph{双流程设计需求}
\label{\detokenize{chapter_project/AI_hardware:id20}}
采集并完成产品方案设计后,会按照硬件设计和软件设计流程同步进行。在硬件的设计流程中,会涉及一些更加专业的流程,如BOM规划、ID工艺。


\subparagraph{BOM规划}
\label{\detokenize{chapter_project/AI_hardware:bom}}
BOM(Bill of
Material)指的是硬件产品所需物料明细表,BOM详细记录了一个项目所用到的所有材料及相关属性,母件与所有子件的从属关系、单位用量及其他属性,在有些系统称为材料表或配方料表。当AI明确产品经理的需求后,工业设计团队和研发团队会分工设计产品的结构、外观,包括对核心部件的选择,从而完成BOM规划,通过合理的BOM规划,可以最大限度地减少资源浪费,通过物料清单,AI产品经理能够了解基本的成本。


\subparagraph{ID工艺}
\label{\detokenize{chapter_project/AI_hardware:id}}
ID设计指的是工艺产品设计,主要指的是产品外观设计,该部分会有专业人员进行设计,ID设计需要考虑产品的美观、易用等性能。


\subparagraph{结构工艺}
\label{\detokenize{chapter_project/AI_hardware:id21}}
完成BOM规划和ID设计后,设计团队会进行结构工艺,如注射开模,然后进行小量试产,就会产生试用产品。

智能硬件还包括软件设计流程,该部分流程同大部分互联网产品设计流程一致,配合产品功能,需要进行软件功能设计,包括方案设计,如有相关操作界面,还需要增加界面设计、完成开发测试的流程。①
方案设计。软件设计部分需要了解数据存储方式和数据交互方式;硬件产品部分数据是存储在本地的,这与常见的互联网产品不同,如智能音箱唤醒词,需要特别注意的是,由于数据存储方式的差异而产生的边界情况。②
界面设计。智能硬件的屏幕不再是标准的手机界面,如可能是如手表的圆形界面,此外色彩呈现和交互方式也与手机有所不同,AI产品经理需要确认产品的载体及支持的展示方式,太过复杂的效果可能无法呈现。③
开发测试。软件部分的开发测试主要侧重于进行数据逻辑的验证,在硬件设计和软件设计阶段完成后,会进行软硬联调的工作。


\paragraph{智能硬件成本预估}
\label{\detokenize{chapter_project/AI_hardware:id22}}
智能硬件产品的成本主要包括原材料成本、生产成本和第三方成本。


\subparagraph{原材料成本}
\label{\detokenize{chapter_project/AI_hardware:id23}}
原材料成本是产品的直接成本,是组成产品的所有原材料的成本之和,一个硬件产品的原材料成本通常包含如下几种。
\begin{enumerate}
\sphinxsetlistlabels{\arabic}{enumi}{enumii}{}{.}%
\item {} 
PCB成本,PCB物料成本和PCB板上元器件成本,包括IC(主IC、电源管理IC、RF
IC、其他类IC)、存储(FLASH、RAM)、屏幕、电池、电阻电容电感的物料成本等。

\item {} 
结构物料成本,包括产品上盖、下盖、中框和按键等。

\item {} 
配件物料成本,包括电源适配器、数据线、耳机等,适配器基本为标配。

\item {} 
包装物料成本,包括外包装、内纸托等。

\item {} 
文档类物料成本,包括使用说明书,法规类说明文档等。

\end{enumerate}


\subparagraph{生产成本}
\label{\detokenize{chapter_project/AI_hardware:id24}}
生产成本指的是将产品原材料组装生产、研发、成品过程中所产生的费用,主要包括以下几项费用。
\begin{enumerate}
\sphinxsetlistlabels{\arabic}{enumi}{enumii}{}{.}%
\item {} 
生产组装费用,烧录、SMT、插件、包装费用。

\item {} 
生产检测类费用,产品性能类测试费用、产品法规类检测费用、产品品质类检测费用等。③
批量生产费用,工厂的一切日常活动都反映到机器产能和人工产能上,批量和产能越高,费用越低。

\item {} 
研发成本,主要为人力成本。

\item {} 
ID设计成本,产品外观设计费用;外观手模制作费用。

\item {} 
模具开模成本,产品ID开模、结构物料开模(屏蔽罩等)费用。

\item {} 
物料打样成本,样机制作成本。

\end{enumerate}


\subparagraph{第三方成本}
\label{\detokenize{chapter_project/AI_hardware:id25}}
由产品生产方支付给第三方公司的费用,主要包括:第三方专利费用、第三方软件授权费用、服务器费用、流量费用、云费用等。


\subsubsection{产品整个流程1\sphinxfootnotemark[270]}
\label{\detokenize{chapter_project/process:id1}}\label{\detokenize{chapter_project/process::doc}}%
\begin{footnotetext}[270]\sphinxAtStartFootnote
\sphinxnolinkurl{http://www.woshipm.com/pd/313514.html}
%
\end{footnotetext}\ignorespaces 

\paragraph{研究背景 3\sphinxfootnotemark[271]}
\label{\detokenize{chapter_project/process:id2}}%
\begin{footnotetext}[271]\sphinxAtStartFootnote
\sphinxnolinkurl{http://www.woshipm.com/pd/841065.html}
%
\end{footnotetext}\ignorespaces 
1、提高研发计划性
产品开发流程每个环节都涉及时间排期,这些时间管理要素可以有效控制项目时间表。

2、提高研发效率
通过明确开发团队每个角色的职责和协作方式,让每个成员只需严格按照规范做好自己的工作即可高效协作,降低沟通成本。

3、保证产品质量
通过确保每个环节的输入输出结果,让最终产出的产品得到有效保证。

4、及时发现问题 通过各环节过程数据,方便管理人员深入了解问题。


\paragraph{组建团队}
\label{\detokenize{chapter_project/process:id3}}
产品研发核心团队通常由产品经理(1名)、研发经理(1名)、研发人员(5\sphinxhyphen{}10名)组成。产品开发涉及的职责分配到各位成员身上。

角色见::ref:\sphinxcode{\sphinxupquote{prod\_people}}


\paragraph{工作流程2\sphinxfootnotemark[272]}
\label{\detokenize{chapter_project/process:id4}}%
\begin{footnotetext}[272]\sphinxAtStartFootnote
\sphinxnolinkurl{http://www.woshipm.com/zhichang/459131.html}
%
\end{footnotetext}\ignorespaces 
收集产品需求(需求池)→ 评审需求→ 竞品分析 → 产品原型设计 → Demo评审→
UI评审→ 开发跟踪→上线前的测试 → 产品上线后的bug收集 →
对客服的培训,可根据实际情况酌情进行调整。

\begin{figure}[H]
\centering
\capstart

\noindent\sphinxincludegraphics{{process}.png}
\caption{流程}\label{\detokenize{chapter_project/process:id35}}\end{figure}


\paragraph{七步}
\label{\detokenize{chapter_project/process:id5}}\begin{enumerate}
\sphinxsetlistlabels{\arabic}{enumi}{enumii}{}{.}%
\item {} 
市场调研

\item {} 
需求管理

\item {} 
产品设计

\item {} 
产品研发

\item {} 
产品测试

\item {} 
产品发布上线

\item {} 
项目跟进优化

\end{enumerate}


\paragraph{市场调研}
\label{\detokenize{chapter_project/process:id6}}
市场调查:分析行业现状和市场规模,发现并掌握目标市场和用户需求的变化趋势;
用户调研:通过用户访谈,可用性测试,调查问卷,数据分析的方法对用户需求进行挖掘和分析;
竞品分析:剖析产品的竞争对手,对其产品进行用户体验分析。
盈利分析:估算产品成本,验证产品需求。


\paragraph{需求管理}
\label{\detokenize{chapter_project/process:id7}}
产品规划:确定目标市场、产品定位、发展规划及路线图;
提炼需求:由市场或运营部门收集的需求来进行分析,提炼核心功能;
根据竞品分析,市场调研来对功能进行优先级排序;
版本规划:每个版本重点开发什么,预估研发进度,上线日期。


\paragraph{产品设计}
\label{\detokenize{chapter_project/process:id8}}
编写产品需求文档,确认产品周期。
产品原型要做的是梳理和完善产品需求流程,降低团队沟通成本。
跟设计师确立产品设计规范,从视觉效果角度确立选用图标类型,文字大小,模块间距,宽高大小等。


\paragraph{产品研发}
\label{\detokenize{chapter_project/process:id9}}
组织讨论,对需求进行评估及确认研发周期、提测时间、预发布时间点、正式发布时间点;
App的开发环境搭配,确定技术框架,以及研发各种基础系统等;
跟踪和推动项目进度,确保项目计划的完成;
布局产品运营工具,方便后期分析与用户跟踪。


\paragraph{产品测试}
\label{\detokenize{chapter_project/process:id10}}
测试周期是直接跟着开发周期一起做;
测试设备:确定要兼容的系统版本,手机品牌类型,手机分辨率等;
按照产品需求文档进行测试。


\paragraph{产品发布上线}
\label{\detokenize{chapter_project/process:id11}}
发布环境的搭建,包括预发布环境、生产环境、灰度发布环境的准备等工作。

而正式上线的工作,则包括数据库上线、程序文件上线等工作。

应用商店ASO优化并根据不同的应用商店作出调整; 协作运营部门做产品推广。


\paragraph{项目跟进优化}
\label{\detokenize{chapter_project/process:id12}}
根据用户反馈对功能进行改进,对用户体验进行优化;
对产品数据进行监控,分析产品运营效果、用户使用行为及需求,以便对产品进行持续性优化和改进;
根据数据挖掘新需求。
\begin{enumerate}
\sphinxsetlistlabels{\arabic}{enumi}{enumii}{}{.}%
\item {} 
研发工作总结:需要对产品研发过程做总结,不论是产品上的还是流程配合上的,为后续加强沟通协作、产品运营打好基础。

\item {} 
上线后收集用户反馈:为了更好的收集用户反馈,需要在所有产品上都增加反馈入口,以便用户提交反馈内容。每天都需要花费相当比例的时间去浏览,并将反馈建议\sphinxstylestrong{转化产品需求点加入需求池。}

\item {} 
产品体验可用性测试:邀请一批真实的典型客户,针对典型场景使用产品,用户研究员在一旁观察、聆听、记录,从而发现产品中存在的可用性缺陷。必须性,因为产品运营团队的员工往往潜意识里会认为用户一定会怎样操作,但是事实上用户很大概率上都不会按照他们希望的进行操作,甚至会陷入茫然根本用不下去。而通过可用性测试,就可以找到问题点,通过优化体验设计\sphinxstylestrong{降低用户使用门槛}。

\item {} 
运维系统优化:升级版本上线工作、服务监控、应用状态统计、日常服务状态巡检、突发故障处理、服务日常变更调整、集群管理、服务性能评估优化、数据库管理优化、随着应用PV增减进行应用架构的伸缩、安全、运维开发等工作。

\end{enumerate}


\paragraph{人工智能规划流程}
\label{\detokenize{chapter_project/process:id13}}
业内较为常见的设计流程是CRISP\sphinxhyphen{}DM(Cross\sphinxhyphen{}Industry Standard Process for
Data Mining,跨行业数据挖掘标准流程)

\begin{figure}[H]
\centering
\capstart

\noindent\sphinxincludegraphics{{CRISP-DM}.png}
\caption{CRISP\sphinxhyphen{}DM}\label{\detokenize{chapter_project/process:id36}}\end{figure}

在1996年的时候,SPSS,戴姆勒\sphinxhyphen{}克莱斯勒和NCR公司发起共同成立了一个兴趣小组,目的是为了建立数据挖掘方法和过程的标准。并在1999年正式提炼出了CRISP\sphinxhyphen{}DM流程。

这个流程确定了一个数据挖掘项目的生命周期,包括以下六个阶段:
\begin{enumerate}
\sphinxsetlistlabels{\arabic}{enumi}{enumii}{}{.}%
\item {} 
商业理解:了解进行数据挖掘的业务原因和数据挖掘目标。

\item {} 
数据理解:深入了解可用于挖掘的数据。

\item {} 
数据准备:对待挖掘数据进行合并,汇总,排序,样本选取等操作。

\item {} 
建立模型:根据前期准备的数据选取合适的模型。

\item {} 
模型评估:使用在商业理解阶段设立的业务成功标准对模型进行评估。

\item {} 
结果部署:使用挖掘后的结果提升业务的过程。

\end{enumerate}

是否可以继续进行下一个阶段取决于是否有达到最初的业务目标,如果业务目标没有达到,那么就要考虑是否是数据不够充分或算法需要调整,一切都以业务目标为导向。

AI项目在产品开发过程和AI产品本身都需要一个“反馈循环”。因为人工智能产品本质上是基于研究的,所以实验和迭代开发是必要的。传统软件开发的输入和结果通常是确定的,而人工智能开发周期是概率性的。不管项目管理框架是什么,这都需要对项目的建立和执行方式进行一些重要的修改。\sphinxhref{https://www.oreilly.com/radar/bringing-an-ai-product-to-market/}{5}%
\begin{footnote}[273]\sphinxAtStartFootnote
\sphinxnolinkurl{https://www.oreilly.com/radar/bringing-an-ai-product-to-market/}
%
\end{footnote}


\subparagraph{商业理解}
\label{\detokenize{chapter_project/process:id14}}
海微购是一家从事跨境电商业务的创业公司,在前几年抓住了海淘的趋势,用户量和交易额都还不错。在新的财年,公司希望能在去年的基础上将GMV提高10\%,并以此为目标制定新一年的工作计划。


\subparagraph{了解客户和确定业务目标}
\label{\detokenize{chapter_project/process:id15}}
在内部,人工智能项目经理必须让利益相关者参与进来,以确保与最重要的决策者和顶级业务指标保持一致。

产品经理必须确保AI项目收集关于客户行为的定性信息。因为这可能不是直观的,需要注意的是,传统的数据测量工具在测量规模上比情绪更有效。对于大多数AI产品,产品经理\sphinxstylestrong{对点击率(CTR)和其他量化指标的兴趣不如对AI产品对用户的效用}感兴趣。因此,传统的产品研究团队必须与人工智能团队合作,以确保将正确的直觉应用到人工智能产品开发中,因为人工智能从业者可能缺乏适当的技能和经验。ctr很容易测量,但是如果您构建了一个旨在优化这类指标的系统,您可能会发现该系统牺牲了实际的实用性和用户满意度。在这种情况下,无论AI产品对这些指标的贡献有多好,它的产出最终都不会服务于公司的目标。

根据电商零售额公式(零售额=流量转化率客单价*复购率),公司认为:在获客成本较高的市场环境,以及本公司经营的海淘产品复购周期较长的情况下,应优先提高转化率和客单价两项指标。根据SMART目标制定原则,确定下一次迭代的产品目标为:猜你喜欢模块中的商品点击量需提高20\%,交叉销售额提高10\%。

如果你没有做适当的研究,你很容易将注意力集中在错误的度量上。我们采访的一家中型数字媒体公司报道称,他们的营销、广告、战略和产品团队曾经想要建立一个人工智能驱动的用户流量预测工具。市场营销团队建立了第一个模型,但因为它来自市场营销,所以该模型针对点击率和潜在客户转化率进行了优化。广告团队更感兴趣的是每潜在成本(CPL)和终身价值(LTV),而策略团队则与企业指标(收益影响和总活跃用户)保持一致。结果,很多工具的用户都不满意,尽管人工智能运行得很完美。最终的结果是开发了针对不同指标进行优化的多种模型,并重新设计了工具,以便能够将这些输出清晰、直观地呈现给不同类型的用户。


\subparagraph{确定数据挖掘目标}
\label{\detokenize{chapter_project/process:id16}}
需要明确数据挖掘的问题是一个分类问题,聚类问题还是一个预测问题,以便于后续的建模阶段选择合适的算法。另外,还需要确定的是数据挖掘的范围,是针对所有用户大范围调整,还是先针对小规模的部分用户进行A/B
Test,待验证成功之后再全面推行。

数据挖掘成功的标准是什么,例如:推荐的准确率要提高40\%,或用户的流失概率降低20\%等,通过可量化的指标评估整个工作的效果。


\subparagraph{制定项目计划}
\label{\detokenize{chapter_project/process:id17}}
根据具体的,可量化的方案组织相关的干系人来评估工作量。根据工作量倒排项目计划表,将目标拆解到更小的时间颗粒度,并指定相关责任人进行任务跟进

\begin{figure}[H]
\centering
\capstart

\noindent\sphinxincludegraphics{{AI_plan}.png}
\caption{项目计划}\label{\detokenize{chapter_project/process:id37}}\end{figure}

在这个阶段需要明确各个环节的交付产物,并识别可能的项目风险,提前制定风险应对计划。

例如:本公司缺乏某方面的\sphinxstylestrong{数据,需要从外部获取,}或者相关人员配置不足,需要招聘或借调人力资源的支持等等。在项目进行的过程中持续监控,以确保项目的正常进行。


\subparagraph{数据准备}
\label{\detokenize{chapter_project/process:id18}}

\subparagraph{数据探索和实验}
\label{\detokenize{chapter_project/process:id19}}
人工智能项目的这一阶段费时费力,但能否完成是未来成功的最重要指标之一。产品需要平衡资源投资和在没有充分了解数据环境的情况下继续发展的风险。获取数据通常很困难,尤其是在受监管的行业。一旦获得了相关数据,理解什么是有价值的,什么是简单的噪音就需要严格的统计和科学。人工智能产品经理可能不会自己做研究;他们的角色是指导数据科学家、分析师和领域专家对数据进行以产品为中心的评估,并为有意义的实验设计提供信息。我们的目标是对存在的数据有一个可衡量的信号,对数据相关性有一个可靠的洞察,并对在哪里集中精力设计特性有一个清晰的愿景。


\subparagraph{数据处理和特征工程}
\label{\detokenize{chapter_project/process:id20}}
数据处理和特征工程是每个AI项目中最困难也是最重要的阶段。人们普遍认为,在一个典型的产品开发周期中,数据科学家80\%的时间都花在特性工程上。自动化和深度学习的趋势和工具确实减少了构建原型(如果不是实际产品的话)所需的时间、技能和努力。尽管如此,构建一个卓越的特性管道或模型架构总是值得的。AI产品经理应该确保项目计划考虑到所需的时间、精力和人员。


\subparagraph{数据理解}
\label{\detokenize{chapter_project/process:id21}}
明确了业务目标之后,我们需要针对数据挖掘的问题收集相关的数据,并对数据进行初步的处理,目标是熟悉数据,探索数据与数据之间的内在联系,并识别数据的质量是否有问题。

用户画像,选择典型的主要客户——例如:最近有过购买记录,并且在一定时间内持续活跃的用户。而不能选择已经流失的,或者是从来没有购买记录的无效客户。

对于参与建模的特征,需要选择那些与业务目标息息相关的数据,以购买商品转化为例:从业务经验来看与之相关的数据有用户的月均消费额度、用户的商品偏好、商品的曝光率、好评率等等。


\subparagraph{构建产品}
\label{\detokenize{chapter_project/process:id22}}

\subparagraph{AI产品交互设计}
\label{\detokenize{chapter_project/process:ai}}
AI产品经理必须从一开始就成为设计团队的一员,以确保产品提供所需的结果。考虑产品的使用方式是很重要的。在最好的人工智能产品中,用户无法分辨底层模型如何影响他们的体验。他们既不知道也不关心应用程序中是否存在人工智能。以Stitch
Fix为例,它使用了多种算法方法来提供定制风格的建议。当Stitch
Fix用户与人工智能产品交互时,他们会与预测和推荐引擎交互。他们在体验过程中与之互动的信息是一种人工智能产品——但他们既不知道,也不关心,他们所看到的一切背后是人工智能。如果算法做出了完美的预测,但用户无法想象佩戴所展示的物品,该产品仍然是一个失败的产品。在现实中,ML模型远非完美,因此更有必要确定用户体验。

要做到这一点,产品经理必须确保设计与工程同等重要。设计师更倾向于用户行为的定性研究。显示用户满意度的信号是什么?如何开发出令用户满意的产品?苹果公司通过iPod、iPhone和iPad产品开创的设计理念,即制造“可以正常工作”的东西,是他们业务的基础。这就是你需要的,你一开始就需要输入。界面设计不是事后添加的东西。


\subparagraph{建模和评估}
\label{\detokenize{chapter_project/process:id23}}
人工智能项目的建模阶段是令人沮丧和难以预测的。这个过程本质上是迭代的,有些AI项目在这一点上失败了(原因很充分)。这一步之所以困难很容易理解:很少有朝着目标稳步前进的感觉。你不断地试验,直到有效果;这可能发生在第一天,或者第100天。当没有有形的“产品”可以展示给每个人的劳动和投资时,AI产品经理必须激励团队成员和利益相关者。一种保持动力的策略是推动短期的突破,以超越表现基线。另一种方法是启动多个线程(甚至可能是多个项目),这样一些线程就能够演示进度。


\subparagraph{原型及MVP}
\label{\detokenize{chapter_project/process:mvp}}
Entrepreneurial product managers are often associated with the phrase
“Move Fast and Break Things.” AI product mangers live and die by
“Experiment Fast So You Don’t Break Things Later.” Take any social media
company that sells advertisements. The timing, quantity, and type of ads
displayed to segments of a company’s user population are overwhelmingly
determined by algorithms. Customers contract with the social media
company for a certain fixed budget, expecting to achieve certain
audience exposure thresholds that can be measured by relevant business
metrics. The budget that is actually spent successfully is referred to
as fulfillment, and is directly related to the revenue that each
customer generates. Any change to the underlying models or data
ecosystem, such as how certain \sphinxstylestrong{demographic features are weighted},
can have a dramatic impact on the social media company’s revenue.
Experimenting with new models is essential–but so is yanking an
underperforming model out of production. This is only one example of why
rapid prototyping is important for teams building AI products. AI PMs
must create an environment in which continuous experimentation and
failure are allowed (even celebrated), along with supporting the
processes and tools that enable experimentation and learning through
failure.

Qualitative data collection tools (such as SurveyMonkey, Qualtrics, and
Google Forms) should be joined with interface prototyping tools (such as
Invision and Balsamiq), and with data prototyping tools (such as Jupyter
Notebooks) to form an ecosystem for product development and testing.


\subparagraph{数据质量和标准化}
\label{\detokenize{chapter_project/process:id24}}
在大多数组织中,数据质量要么是工程问题,要么是IT问题;除非它阻碍了下游流程或项目,否则产品团队很少处理它。这种关系对于开发AI产品的团队来说是不可能的。“垃圾输入,垃圾输出”也适用于人工智能,所以优秀的人工智能pm必须关心数据的健康状况。

原则:
\begin{itemize}
\item {} 
小心“数据清理”方法会破坏数据。如果它改变了底层数据的核心属性,那么它就不是数据清理。

\item {} 
寻找数据中的特性(例如,来自遗留系统的数据会截断文本字段以节省空间)。

\item {} 
在计划和实施数据收集时,要了解糟糕的下游标准化的风险(例如任意词干提取、停止词删除)。

\item {} 
确保数据存储、关键管道和查询都有适当的文档记录,具有结构化的元数据和易于理解的数据流。

\item {} 
考虑时间如何影响您的数据资产,以及季节性影响和其他偏差。

\item {} 
要理解用户体验选择和调查设计可能会引入数据偏差和人为因素。

\end{itemize}


\subparagraph{通过技术领导来加强AI产品管理}
\label{\detokenize{chapter_project/process:id25}}
除了优秀的产品感觉、UI/X体验、客户知识、领导技能等,不太可能每个AI
PM都拥有世界级的技术直觉。但不要因此而产生悲观情绪。因为一个人不可能是所有事情的专家,AI
pm需要与技术领导者(例如,技术领导者或首席科学家)建立合作关系,后者了解技术的现状,熟悉当前的研究,并相信技术领导者受过教育的直觉。

找到这个关键的技术合作伙伴是很困难的,特别是在今天竞争激烈的人才市场。然而,并不是一切都失去了:有许多优秀的技术产品领导者伪装成有能力的工程经理。


\subparagraph{模型评估}
\label{\detokenize{chapter_project/process:id26}}
开始最后的部署阶段之前,最重要的事情是彻底的评估模型,根据在商业理解阶段定义的业务目标来评估我们努力的成果。


\subparagraph{评估结果}
\label{\detokenize{chapter_project/process:id27}}
数据挖掘没有达成业务目标的结果,也不一定意味着建模的失败,有多种可能性导致不成功的结果。

例如:业务目标一开始定得不够合理,与业务目标密切相关的数据未收集到,数据准备出现了错误,训练数据和测试数据不具备代表性等等。这时候我们就要回到之前的步骤,来检视到底是哪个环节出现了问题。


\subparagraph{结果部署}
\label{\detokenize{chapter_project/process:id28}}
建模的作用是从数据中找到知识,获得的知识需要以便于用户使用的方式重新组织和展现,这就是结果部署阶段的工作。根据业务目标的不同,结果部署简单的可能只需要提交一份数据挖掘报告即可,也有可能复杂到需要将模型集成到企业的核心运营系统当中。


\subparagraph{部署前}
\label{\detokenize{chapter_project/process:id29}}
不违反某些指标阈值是非常重要的。这些度量通常被称为护栏指标,它们确保了产品分析不会给决策者错误的信号,让他们知道什么才是对业务真正重要的。

拼车公司:人工智能产品可以轻易地减少来自难以到达地点的用户的请求,从而减少平均拾取时间。然而,这种行为会对公司的整体业务结果产生负面影响,并最终减缓服务的采用。

当一个措施成为目标时,它就不再是一个好的措施(古德哈特定律)。任何衡量标准都会被滥用。“让我们想想如何滥用拾取时间度量”


\subparagraph{制定部署计划}
\label{\detokenize{chapter_project/process:id30}}
根据业务要求和运算性能的的不同,部署的模型可分为:离线模型,近线模型和在线模型三种

离线模型一般适用于重量级的算法,部署于大数据集群仓库,运算的时间需要以小时计,并且时效上通常是T+1的。

近线模型适用于轻量级算法,一般在内存和Redis(一种支持Key\sphinxhyphen{}Value等多种数据结构的存储系统,适用于高速消息队列)中进行,运算的速度以秒为单位。而在线模型则通常根据业务规则来设置,在内存中运行,运行的速度以毫秒计。

另外,部署还需要考虑不同编程语言对于算法模型的调取兼容性,在这个阶段算法工程师需要与原有业务的开发工程师进行联调协作,确保业务系统能够正确的调用算法模型结果。


\subparagraph{部署}
\label{\detokenize{chapter_project/process:id31}}
与传统的软件工程项目不同,AI产品经理必须大量参与构建过程。工程经理通常负责确保软件产品的所有组件都被正确地编译成二进制文件,并按照版本仔细地组织构建脚本,以确保可再现性。许多成熟的DevOps过程和工具,经过多年成功的软件产品发布的磨练,使这些过程更加易于管理,但它们是为传统的软件产品开发的。在ML/AI生态系统中根本不存在类似的工具和过程;即使这样,它们也很少成熟到可以大规模使用。因此,人工智能项目经理必须采取高度接触的、定制的方法来指导人工智能产品的生产、部署和发布。


\subparagraph{模型监督和维护}
\label{\detokenize{chapter_project/process:id32}}
对于AI产品,模型性能和应用程序性能必须同时监控。当AI产品执行超出规范时触发的警报可能需要以不同的方式传递;如果没有AI团队的支持,现场SRE团队可能无法诊断模型或数据管道的技术问题。

我们知道算法模型是基于历史数据得来的,但是在模型部署并运行一段时间之后,可能业务场景已经发生了变化,原有的模型已经无法满足当前的业务需要。

例如,一款帮助服装制造商了解该购买哪种材料的人工智能产品将随着时尚的变化而过时。如果AI产品成功了,它甚至可能导致这些变化。您必须检测模型何时过时,并根据需要对其进行重新培训。

这就需要我们在模型部署上线的同时,同步上线模型的监督和维护系统,持续跟踪模型的运行状况。

该框架允许部署的模型不间断地运行,同时根据总体样本评估第二个模型。如果第二种型号的性能优于原来的,它可以简单地被替换掉——通常没有任何停机时间!

当发现模型结果在出现短期异常值时,排查异常的原因,例如:运营活动或者节假日等因素,当发现模型长期表现不佳时,则要考虑是否是用户和产品的数据构成已经发生了变化。如果是因为数据构成已经发生变化,则需要重新通过CRISP\sphinxhyphen{}DM流程构建新的模型。


\subparagraph{持续集成和持续部署}
\label{\detokenize{chapter_project/process:id33}}\begin{itemize}
\item {} 
\sphinxurl{https://martinfowler.com/articles/cd4ml.html}

\item {} 
\sphinxurl{https://www.oreilly.com/library/view/continuous-delivery-reliable/9780321670250/}

\end{itemize}


\subparagraph{生成最终报告}
\label{\detokenize{chapter_project/process:id34}}
最后,不要忘了向项目的相关干系人发送一份最终的项目总结报告。


\subsubsection{数据处理流程拆解}
\label{\detokenize{chapter_project/Data Process:id1}}\label{\detokenize{chapter_project/Data Process::doc}}

\paragraph{数据工作 2\sphinxfootnotemark[274]}
\label{\detokenize{chapter_project/Data Process:id2}}%
\begin{footnotetext}[274]\sphinxAtStartFootnote
\sphinxnolinkurl{https://weread.qq.com/web/reader/40632860719ad5bb4060856ka1d32a6022aa1d0c6e83eb4}
%
\end{footnotetext}\ignorespaces 
数据准备可以分为3个阶段,数据来源—数据定义—数据交付,在这3个阶段中,需要AI产品经理具备规划、收集、整理数据的能力。在数据准备的过程中,我们可以根据不同阶段考虑以下问题。


\subparagraph{数据来源}
\label{\detokenize{chapter_project/Data Process:id3}}\begin{itemize}
\item {} 
数据源。数据源在哪?这些数据是否存在不同的地方,以及如何进行关联?

\item {} 
数据规模。数据是否能够或足够进行建模,有多少数据来描述这个场景?

\item {} 
数据更新。数据是怎么更新的?这些数据的维度是什么?

\end{itemize}


\subparagraph{数据定义}
\label{\detokenize{chapter_project/Data Process:id4}}\begin{itemize}
\item {} 
数据清洗。用什么样的方法清洗和整理数据?

\item {} 
输入和输出。设置什么样的“输入”和“输出”,能够保证测试集训练出来的机器能更好地运用在实际场景中?

\item {} 
数据标注。如何更迅速高效地标注数据?不同的输入方式之间有什么层级关系?我们应该用什么形式来展现这些层级关系?

\end{itemize}


\subparagraph{数据交付}
\label{\detokenize{chapter_project/Data Process:id5}}\label{\detokenize{chapter_project/Data Process:id6}}
在产品的实际交互中,应记录哪些数据? 用什么样的形式提供数据?
如何使用数据,通过接口或是批量推送?

获取正确数据的过程并非我们想象中那么简单,因为不同团队对于数据的维护程度是不一样的,在获取正确数据的过程中,沟通成本是非常高的,在AI产品构建过程中,除了业务场景的调研,数据的准备环节也是很重要的,因为数据是AI产品设计的基础,所有的工作自始至终都是围绕数据展开的。


\paragraph{数据处理流程拆解}
\label{\detokenize{chapter_project/Data Process:id7}}
一、数据标注

数据的质量直接会影响到模型的质量,因此数据标注在整个流程中绝对是非要重要的一点。

1、一般来说,数据标注部分可以有三个角色

标注员:标注员负责标记数据。

审核员:审核员负责审核被标记数据的质量。

管理员:管理人员、发放任务、统计工资。

只有在数据被审核员审核通过后,这批数据才能够被算法同事利用。

2、数据标记流程

任务分配:假设标注员每次标记的数据为一次任务,则每次任务可由管理员分批发放记录,也可将整个流程做成“抢单式”的,由后台直接分发。

标记程序设计:需要考虑到如何提升效率,比如快捷键的设置、边标记及边存等等功能都有利于提高标记效率。

进度跟踪:程序对标注员、审核员的工作分别进行跟踪,可利用“规定截止日期”的方式淘汰怠惰的人。

质量跟踪:通过计算标注人员的标注正确率和被审核通过率,对人员标注质量进行跟踪,可利用“末位淘汰”制提高标注人员质量。

而且,模型的效果,需要在这两个指标之间达到一个平衡。

测试同事需要关注特定领域内每个类别的指标,比如针对识别人脸的表情,里面有喜怒哀乐等分类,每一个分类对应的指标都是不一样的。测试同事需要将测试的结果完善地反馈给算法同事,算法同事才能找准模型效果欠缺的原因。同时,测试同事将本次模型的指标结果反馈给产品,由产品评估是否满足上线需求。

2、数据标记流程

任务分配:假设标注员每次标记的数据为一次任务,则每次任务可由管理员分批发放记录,也可将整个流程做成“抢单式”的,由后台直接分发。

标记程序设计:需要考虑到如何提升效率,比如快捷键的设置、边标记及边存等等功能都有利于提高标记效率。

进度跟踪:程序对标注员、审核员的工作分别进行跟踪,可利用“规定截止日期”的方式淘汰怠惰的人。

质量跟踪:通过计算标注人员的标注正确率和被审核通过率,对人员标注质量进行跟踪,可利用“末位淘汰”制提高标注人员质量。

对于众包平台来讲,国外首选亚马逊众包平台,ImageNet就是通过这个平台进行标注的。而国内也有百度众包、京东众智、龙猫数据等众包平台可供选择。

二、模型训练

这部分基本交由算法同事跟进,但产品可依据需求,向算法同事提出需要注意的方面;

举个栗子:

背景:一个识别车辆的产品对大众车某系列的识别效果非常不理想,经过跟踪发现,是因为该车系和另外一个品牌的车型十分相似。那么,为了达到某个目标(比如,将精确率提高5\%),可以采用的方式包括:

补充数据:针对大众车系的数据做补充。值得注意的是,不仅是补充正例(“XXX”应该被识别为该大众车系),还可以提供负例(“XXX”不应该被识别为该大众车系),这样可以提高差异度的识别。

优化数据:修改大批以往的错误标注。

产品将具体的需求给到算法工程师,能避免无目的性、无针对性、无紧急程度的工作。

三、模型测试

测试同事(一般来说算法同事也会直接负责模型测试)将未被训练过的数据在新的模型下做测试。

如果没有后台设计,测试结果只能由人工抽样计算,抽样计算繁琐且效率较低。因此可以考虑由后台计算。

一般来说模型测试至少需要关注两个指标:

精确率:识别为正确的样本数/识别出来的样本数

召回率:识别为正确的样本数/所有样本中正确的数

举个栗子:全班一共30名男生、20名女生。需要机器识别出男生的数量。本次机器一共识别出20名目标对象,其中18名为男性,2名为女性。则

精确率=18/(18+2)=0.9

召回率=18/30=0.6

再补充一个图来解释:

四、产品评估

“评估模型是否满足上线需求”是产品必须关注的,一旦上线会影响到客户的使用感。

因此,在模型上线之前,产品需反复验证模型效果。为了用数据对比本模型和上一个模型的优劣,需要每次都记录好指标数据。

假设本次模型主要是为了优化领域内其中一类的指标,在关注目的的同时,产品还需同时注意检测其他类别的效果,以免漏洞产生。


\subsubsection{原型检验1\sphinxfootnotemark[275]}
\label{\detokenize{chapter_project/inspect:id1}}\label{\detokenize{chapter_project/inspect::doc}}%
\begin{footnotetext}[275]\sphinxAtStartFootnote
\sphinxnolinkurl{http://www.woshipm.com/pmd/21446.html}
%
\end{footnotetext}\ignorespaces 
容易犯一个常见的错误,他们对产品设计规范太有信心,结果一旦得到beta的测试他们就必须调整产品。但是肯定beta测试版并不是进行重大改变的时候,所以才会有许多首次发布的产品离目标太远。


\paragraph{可行性测试}
\label{\detokenize{chapter_project/inspect:id2}}
一个直接的问题就是产品是否可以开发,你的工程师和设计师应当介入技术的可行性调查和探索可用办法。有些办法是行不通的,但是有其他的办法可行是非常有希望的。
工程师会发现在产品的某个阶段不可能逾越,现在知道比以后知道要好。


\paragraph{可用性测试}
\label{\detokenize{chapter_project/inspect:id3}}
产品设计师将要和你紧密工作共同提出产品功能,让它能适应不同的用户。可用性测试常常会找出遗漏的产品要求,同时确认产品最初的要求是否是必须的。在你拿出一个成功的用户体验之前需要多做一些测试工作。可用性的目的是在真正的用户身上测试,从产品目标用户得到质量反馈的测试是非常艺术和科学的。当然产品经理和产品设计将模仿使用,但是实际是没有人能取代真实的目标用户。


\paragraph{概念测试(Product Concept Testing)}
\label{\detokenize{chapter_project/inspect:product-concept-testing}}
光是可用和可行是不足的。真正的问题是你的用户想要购买吗—你的用户有多喜欢\sphinxhyphen{}你做的有什么价值。这测试可能与可用性测试联系在一起。

对于一部份小产品,您的想法写在纸就足够了,但是对于多数产品,为了预计产品是否达到目标,复杂用户互作用或新技术的使用、某种形式原型都是非常重要的。

原型也许是一个物理设备,或者它也许是软件产品的一个预览版本。关键是它需要足够现实,您能用原型在实际目标顾客身上测试,并且他们可以给您质量反馈。

以前做原型主要有两个障碍。第一是缺乏良好的原型工具,需要花费很多的时间制作原型;另一个是管理方不知道原型和真实产品的区别,在不可预计的情况下,按照最终产品来要求原型。


\subsubsection{产品文档}
\label{\detokenize{chapter_project/product_document:id1}}\label{\detokenize{chapter_project/product_document::doc}}

\paragraph{问题}
\label{\detokenize{chapter_project/product_document:id2}}\begin{itemize}
\item {} 
需求描述不清楚:前端新增页面是新开页面,还是当前页面打开。

\item {} 
逻辑不通:产生需求变更,不仅引起众怒,研发人员也质疑产品经理的能力。

\item {} 
需求遗漏:功能的排序规则不对,然后让研发修改。

\end{itemize}


\paragraph{目标}
\label{\detokenize{chapter_project/product_document:id3}}\begin{enumerate}
\sphinxsetlistlabels{\arabic}{enumi}{enumii}{}{.}%
\item {} 
灵活运用不同文档格式描述功能、业务等,而不拘泥于框架,将你的想法最大程度用文档呈现;

\item {} 
使你的文档逻辑有条理、框架更清晰、重点更突出,减少与开发人员的沟通成本;

\item {} 
利用不同的规范化文档,在接收需求、管理需求、验收需求的过程中提高工作效率。

\end{enumerate}


\subsubsection{Roadmap1\sphinxfootnotemark[276]}
\label{\detokenize{chapter_project/Roadmap:roadmap1}}\label{\detokenize{chapter_project/Roadmap::doc}}%
\begin{footnotetext}[276]\sphinxAtStartFootnote
\sphinxnolinkurl{http://www.woshipm.com/pd/3591815.html}
%
\end{footnotetext}\ignorespaces 
产品路线图是传达产品战略方向的生动、可视化文档。这些高层次的总结将所有涉及产品管理和产品开发的活动部分结合在一起。他们传达了两个愿望(通过产品愿景)和计划(通过倡议),同时也将产品战略与公司战略联系起来。有效地完成后,产品路线图将传达您正在构建的内容背后的“原因”。


\paragraph{目标}
\label{\detokenize{chapter_project/Roadmap:id1}}

\subparagraph{实现以下几点:}
\label{\detokenize{chapter_project/Roadmap:id2}}\begin{itemize}
\item {} 
描述产品愿景和战略

\item {} 
提供执行战略的指导文件

\item {} 
使内部利益相关者保持一致

\item {} 
促进对选项和方案规划的讨论

\item {} 
沟通产品开发计划的进展和状态

\item {} 
帮助向外部利益相关者(包括客户)传达您的战略

\end{itemize}


\subparagraph{还可帮助:}
\label{\detokenize{chapter_project/Roadmap:id3}}\begin{itemize}
\item {} 
IT团队可以使用路线图来规划和跟踪架构项目、软件实现和各种其他项目。

\item {} 
营销团队可以使用路线图来规划从营销策略到内容日历的所有内容。

\item {} 
业务和业务运营团队可以使用路线图来规划扩展和其他组织范围的目标。

\item {} 
DevOps可能会使用路线图来更好地连接他们负责整合的组织的各个部分。

\item {} 
工程团队可以使用路线图来计划冲刺。

\end{itemize}


\paragraph{长什么样}
\label{\detokenize{chapter_project/Roadmap:id4}}\begin{enumerate}
\sphinxsetlistlabels{\arabic}{enumi}{enumii}{}{.}%
\item {} 
产品愿景
你的产品愿景,通常被称为产品愿景宣言或使命宣言,是一个简洁的、高层次的、有抱负的陈述。

\end{enumerate}

为什么你的产品存在,如果你的产品成功了,世界将会变成怎样?

它是一盏指路明灯——在当下遥不可及,但却始终代表着预定的方向。

在Roadmap上阐述产品愿景,一方面有助于我们的产品始终围绕着这个核心价值观而不偏离轨道,另一方面,愿景是我们激励其他部门的重要力量。
\begin{enumerate}
\sphinxsetlistlabels{\arabic}{enumi}{enumii}{}{.}%
\setcounter{enumi}{1}
\item {} 
说明
通常需要说明roadmap上时间段的起点和终点分别代表什么意思。时间段不同颜色代表什么意思,比如可以是不同颜色代表不同的技术平台,也可以是代表不同的市场。

\item {} 
关键细分
这个关键细分取决于你所在的行业以及你的受众,通常来讲,如果是客户的话,可能是他们在购买你产品的时候主要考虑的几个因素之一。比如手机的话,可以是“米系列”,“红米系列”,汽车的话,可以是“轿车”,“SUV”和“MPV”等。关键细分可以是一层,也可以是多层,但最好不要超过三层。

\item {} 
产品信息
通常分为当前在售和未来推出。这里可以在产品名称下面加一些关键特征参数,主打的卖点或关键购买因素等信息。如下图里面12M,1.55μm是CMOS的一些关键特征参数。汽车的话诸如“2.0T,涡轮增压,190马力,国VI,双擎”。这样可以方便受众快速的找到自己关心的产品的进展。

\end{enumerate}


\paragraph{总结}
\label{\detokenize{chapter_project/Roadmap:id5}}
有利且危险:可以帮助产品经理更好地可视化和沟通产品战略,形成健康的权衡。如果他们太过细节和死板,则会造成很多工作浪费,给客户和管理层产生一些无法实现的期望。


\subsubsection{Scrum1\sphinxfootnotemark[277]}
\label{\detokenize{chapter_project/Scrum:scrum1}}\label{\detokenize{chapter_project/Scrum::doc}}%
\begin{footnotetext}[277]\sphinxAtStartFootnote
\sphinxnolinkurl{https://ones-ai.gitbooks.io/ones-ai}
%
\end{footnotetext}\ignorespaces 
瀑布式(Waterfall)开发:概念设计>>设计>>编程>>测试修正

迭代周期可能短到一周,但分析、设计、编码、集成和测试等开发阶段一应俱全。而在采用瀑布开发的项目中,单个阶段也许就得花费数年时间。

与瀑布模式相比,敏捷放弃了从开始到结束控制项目形态的总体计划和规范,而选择了许多中途修正。


\paragraph{动因}
\label{\detokenize{chapter_project/Scrum:id1}}
来自于市场的压力(逼迫我们用更少的成本来制作更符合真实客户需求的软件)

用Scrum的开发模式小步快跑,可以以速度最快、最经济的方式逐步加入满足消费者的功能,明确用户价值。

瀑布式开发的致命症结就在于它的预先设计(BDUFs,Big Design
Up\sphinxhyphen{}Front)无法完全清晰,并且变更成本巨大


\paragraph{Scrum流程}
\label{\detokenize{chapter_project/Scrum:scrum}}
在发布产品获得用户反馈后才正式开始——由真实用户反馈影响需求规划、产品迭代的“小步快跑”方式,才是Scrum的精髓所在


\subparagraph{产品Backlog}
\label{\detokenize{chapter_project/Scrum:backlog}}
最顶端的核心需求拆分成同样颗粒度大小的需求——这样才能准确估计工作量——然后把需求排入版本中。


\subparagraph{User Story}
\label{\detokenize{chapter_project/Scrum:user-story}}
随着讨论和了解的细节增多,能够浮现出更多关于产品设计的细节和场景,也便越能进一步准确估算每个需求的工作量——不确定的特性是无法精确估算难度的。


\subparagraph{Sprint}
\label{\detokenize{chapter_project/Scrum:sprint}}
固定的时间限,大约2\textasciitilde{}4周,每个版本产出功能性的垂直切片,他们各自就像一个个小型项目


\subparagraph{Release}
\label{\detokenize{chapter_project/Scrum:release}}
进入到准上市状态,这称作一次发布。一个典型的发布需要经历2\textasciitilde{}4个月,节奏类似于一个经典项目中的里程碑。


\paragraph{步骤}
\label{\detokenize{chapter_project/Scrum:id2}}\begin{enumerate}
\sphinxsetlistlabels{\arabic}{enumi}{enumii}{}{.}%
\item {} 
通过创建需求池项目,进行对产品Backlog(需求列表)的管理。

\item {} 
建立发布计划和Sprint迭代计划,清晰地区分阶段性目标。

\item {} 
把需求排入相应的版本中,进行对发布计划和迭代的任务管理。

\end{enumerate}


\paragraph{角色}
\label{\detokenize{chapter_project/Scrum:id3}}

\subparagraph{Product Owner}
\label{\detokenize{chapter_project/Scrum:product-owner}}
负责明确目标、提出需求并排出优先级


\subparagraph{Scrum Master}
\label{\detokenize{chapter_project/Scrum:scrum-master}}
项目经理一般也是版本负责人,他会控制需求变更、增加需求的数量,提前预告风险,以及明确版本交付质量。

PM必须提醒大家每个版本都是产品的垂直切片,不能推迟到下一个版本才修复上个版本的bug和资源。


\subparagraph{Team}
\label{\detokenize{chapter_project/Scrum:team}}
Scrum相关著述推荐团队有7\textasciitilde{}9个成员(Schwaber 2004

将大团队拆分成多个小型Scrum团队负责某个模块,比如关卡原型制作团队。每个敏捷团队都包含所有职能的开发人员,设计、开发、测试等,他们共同为团队的产品交付物负责。


\paragraph{每日站会}
\label{\detokenize{chapter_project/Scrum:id4}}
一种有效的方法是让每个将要构建系统的人都在一个房间里,并让团队对每个项目的难度、人数和影响形成一致的估计。然后,您可以创建一个影响和易用性图表,根据投资回报对每个项目进行排序,并相应地进行优先级排序。在现实中,确定优先级是一个混乱和不稳定的过程,因为项目经常有依赖性,面临人员限制或与其他涉众截止日期的冲突。为了与其他团队或优先级保持一致,经常需要减少范围或牺牲质量。

目的:
\begin{enumerate}
\sphinxsetlistlabels{\arabic}{enumi}{enumii}{}{.}%
\item {} 
让所有团队成员统一步调,一起向着“完成Sprint”的目标进发;

\item {} 
承诺第二天要完成的工作,并重申团队对Sprint目标的承诺;

\item {} 
识别出团队面临的所有障碍;

\item {} 
使团队成员成为“一条绳子上的蚂蚱“,每个人都需要了解其它人面临的困难,以便在会后找到解决方法。

\end{enumerate}

Q:为什么一定要是”站“会呢?

A:因为要通过站立的形式让大家明确,这是一个需要快速解决的短会,而非冗长艰苦的会议。


\paragraph{反馈}
\label{\detokenize{chapter_project/Scrum:id5}}
销售,服务环节,运营环节,甚至公司团队成员本身,在ONES系统上我们可以用反馈池进行详尽记录:


\paragraph{敏捷开发 2\sphinxfootnotemark[278]}
\label{\detokenize{chapter_project/Scrum:id6}}%
\begin{footnotetext}[278]\sphinxAtStartFootnote
\sphinxnolinkurl{https://www.jianshu.com/p/e53974f9cbc9}
%
\end{footnotetext}\ignorespaces 
把一个大的产品功能模块拆解为多个相互独立的小功能模块,每次只上线一部分功能,在保证产品可用的前提下,一步步迭代完成整个功能的上线,这种方式就叫敏捷开发。
\sphinxhref{https://weread.qq.com/web/reader/8d232b60721a488e8d21e54kc51323901dc51ce410c121b}{4}%
\begin{footnote}[279]\sphinxAtStartFootnote
\sphinxnolinkurl{https://weread.qq.com/web/reader/8d232b60721a488e8d21e54kc51323901dc51ce410c121b}
%
\end{footnote}

例如,一款新的电商App要做一个购物车功能,第一个版本上线最基础的商品结算功能,第二个版本上线移除商品功能,第三个版本上线降价商品提醒功能等,不要在一次迭代中把所有的功能都做完。


\subparagraph{宣言}
\label{\detokenize{chapter_project/Scrum:id7}}\begin{itemize}
\item {} 
个体和交互胜过过程和工具。

\item {} 
可以工作的软件胜过面面俱到的文档。

\item {} 
客户合作胜过合同谈判。

\item {} 
响应变化胜过遵循计划。

\end{itemize}


\subparagraph{计划}
\label{\detokenize{chapter_project/Scrum:id8}}\begin{itemize}
\item {} 
任何过大的素材都应该被分解成小一点的部分,任何过小素材都应该和其它小的素材合并。

\item {} 
如果知道了开发速度,客户就能够对每个素材的成本有所了解。

\item {} 
迭代期间用户素材的实现顺序属于技术决策范畴。

\item {} 
一旦迭代开始,客户就不能改变该迭代期内需要实现的素材。

\end{itemize}


\subparagraph{测试}
\label{\detokenize{chapter_project/Scrum:id9}}
编写单元测试是一种验证行为,更是一种设计行为。同样,它更是一种编写文档的行为。编写单元测试避免了相当数量的反馈循环,尤其是功能验证方面的反馈循环。

首先编写测试可以:
\begin{itemize}
\item {} 
程序中的每一项功能都有测试来验证它的操作的正确性。

\item {} 
迫使我们使用不同的观察点。

\item {} 
迫使自己把程序设计为可测试的,从而迫使我们解除软件中的耦合。(forces
us to decouple the software)

\item {} 
作为一种无价的文档形式。

\end{itemize}


\subparagraph{重构}
\label{\detokenize{chapter_project/Scrum:id10}}
每一个软件模块都有三个职责:
\begin{enumerate}
\sphinxsetlistlabels{\arabic}{enumi}{enumii}{}{.}%
\item {} 
它运行起来所完成的功能。

\item {} 
它要应对变化。

\item {} 
要和阅读它的人进行沟通。

\end{enumerate}


\paragraph{项目看板}
\label{\detokenize{chapter_project/Scrum:id11}}
看板方法源自丰田的“及时生产”JIT=just\sphinxhyphen{}in\sphinxhyphen{}time)系统。

项目看板清晰地展示了:需求池中的哪些功能待开发;哪些功能进入UI设计阶段;哪些需求在开发阶段;哪些需求在测试阶段;哪些需求已经上线;哪些需求需要延期等。项目看板可以明确哪类问题需要谁去跟进,从而保证项目按照项目排期表稳步推进

看板方法可以动态显示瓶颈:你之所以能找到这些瓶颈,是因为限制了在制品(work\sphinxhyphen{}in\sphinxhyphen{}progress,
WIP)的数量会显示出瓶颈。
卡片代表了工作项,列代表了开发工序,卡片会从第一步工序流动到最后一步。每一列顶部的数字用来限制每一列最多允许放置卡片的数量。

\begin{figure}[H]
\centering
\capstart

\noindent\sphinxincludegraphics{{kanban}.png}
\caption{看板}\label{\detokenize{chapter_project/Scrum:id15}}\end{figure}

一些列分割成了两列,这是为了用来说明正在进行中的项与哪些已经完成并准备好被下游工序拉走的项。


\subparagraph{项目排期表}
\label{\detokenize{chapter_project/Scrum:id12}}
项目排期表为了保证项目按时上线,会使用项目排期表确定每个参与者的具体工作内容及起止时间。项目排期表示例如图所示。

\begin{figure}[H]
\centering
\capstart

\noindent\sphinxincludegraphics{{project_table}.png}
\caption{项目排期表}\label{\detokenize{chapter_project/Scrum:id16}}\end{figure}


\paragraph{敏捷产品 5\sphinxfootnotemark[280]}
\label{\detokenize{chapter_project/Scrum:id13}}%
\begin{footnotetext}[280]\sphinxAtStartFootnote
\sphinxnolinkurl{https://zhiya360.com/135801.html}
%
\end{footnotetext}\ignorespaces 
对于产品小团体交付给设计小团体前,我们要做需求、方案、原型三个方面的敏捷冲刺
\begin{enumerate}
\sphinxsetlistlabels{\arabic}{enumi}{enumii}{}{.}%
\item {} 
需求敏捷

\item {} 
方案敏捷

\item {} 
原型敏捷

\end{enumerate}


\subparagraph{需求敏捷}
\label{\detokenize{chapter_project/Scrum:id14}}
所有公司都用专门的问题反馈线:客户\sphinxhyphen{}>Customer Service\sphinxhyphen{}>Support
Engineer\sphinxhyphen{}>PM\sphinxhyphen{}>SDM\sphinxhyphen{}>SDE


\subsubsection{需求价值的分析方法}
\label{\detokenize{chapter_project/separate_need:id1}}\label{\detokenize{chapter_project/separate_need::doc}}
对需求的分析可以从用户、场景、需求、流程四个维度进行快速拆解。
\begin{enumerate}
\sphinxsetlistlabels{\arabic}{enumi}{enumii}{}{.}%
\item {} 
用户
用户是谁,有什么特性,用户画像具体是什么样的?如果对用户不了解还需要进一步做用户研究。

\item {} 
场景
用户在什么场景中有需求,场景可以是用户在什么时间和地点,会发生什么,如果不了解场景,最好到用户的使用场景中,当面去看用户怎么操作。

\item {} 
需求
用户的需求是无穷的,需求的背后一定是要结合用户具体的场景,脱离了场景的需求都是伪需求。

\item {} 
流程
流程决定了此需求和其它需求的关系,如何让用户最爽的满足一系列需求,就形成了流程,在B端是业务流程图,在C端是产品架构设计。

\end{enumerate}

我在日常的产品工作中,不大可能拿很多模型分析需求,论证需求,这样比较低效,公众号留言
需求画布,下载最新的需求画布模板。新手可以照着模板思考每个需求价值,老鸟时间长了表格可以扔掉,对每个需求心理默念即可。


\paragraph{A/B testing}
\label{\detokenize{chapter_project/separate_need:a-b-testing}}
一种统计方法,用于将两种或多种技术进行比较,通常是将当前采用的技术与新技术进行比较。A/B测试不仅旨在确定哪种技术的效果更好,而且还有助于了解相应差异是否具有显著的统计意义。AB测试通常是采用一种衡量方式对两种技术进行比较,但也适用于任意有限数量的技术和衡量方式。

A/B测试,在基于web的软件开发中经常使用,对于在生产中评估模型性能非常有用。

\begin{figure}[H]
\centering
\capstart

\noindent\sphinxincludegraphics{{A_B_Testing}.png}
\caption{A/B测试\sphinxhref{https://time.geekbang.org/column/intro/100065501}{4}\sphinxfootnotemark[281]}\label{\detokenize{chapter_project/separate_need:id2}}\end{figure}
%
\begin{footnotetext}[281]\sphinxAtStartFootnote
\sphinxnolinkurl{https://time.geekbang.org/column/intro/100065501}
%
\end{footnotetext}\ignorespaces 
\sphinxhref{https://experimentguide.com/}{Trustworthy Online Controlled Experiments : A Practical Guide to A/B
Testing}%
\begin{footnote}[282]\sphinxAtStartFootnote
\sphinxnolinkurl{https://experimentguide.com/}
%
\end{footnote}


\subsubsection{价值 1\sphinxfootnotemark[283]}
\label{\detokenize{chapter_project/valuable:id1}}\label{\detokenize{chapter_project/valuable::doc}}%
\begin{footnotetext}[283]\sphinxAtStartFootnote
\sphinxnolinkurl{http://www.woshipm.com/pmd/4339402.html}
%
\end{footnotetext}\ignorespaces 

\paragraph{产品创新的价值}
\label{\detokenize{chapter_project/valuable:id2}}\begin{itemize}
\item {} 
发明价值

\end{itemize}

发明价值是能被称为创新的最基础的前提,也就是说你得弄点世上原本不存在的新东西出来。

我对发明和创新这两个词有着不同的理解:发明,只需要东西新,但创新,还需要“有用”。

有很多被发明出来的科研成果都被束之高阁,只能在实验室里等待适合的应用场景,你可以说它们有发明价值,但暂时还不是创新。因为它们还没有没找到用户价值。注意,这不是说科研不重要,科学研究当然很重要,如果拉长时间的尺度,你会发现底层的发明经常能带来重大的产业变革。只不过,在商业环境下,我们做产品,不能局限于发明价值。
\begin{itemize}
\item {} 
用户价值

\end{itemize}

任何一个产品要达到第二层价值,都不容易。毫不夸张地说,每年上市的新产品中,有
90\% 都缺乏用户价值,是“没人需要的新玩意”。

可你该如何达成用户价值呢?

这需要我们去理解用户、去深挖需求、感受场景,分析竞品等等,这叫想清楚;再把问题转化为合适的解决方案,多快好省地做出来;还需要推出去,让尽可能多的目标用户用上我们的产品。这是做好产品创新,最最基础的标准动作。

用户价值(主观效用)具备的特征:
\sphinxhref{https://www.jianshu.com/p/02df7160b7b0}{3}%
\begin{footnote}[284]\sphinxAtStartFootnote
\sphinxnolinkurl{https://www.jianshu.com/p/02df7160b7b0}
%
\end{footnote}
\begin{enumerate}
\sphinxsetlistlabels{\arabic}{enumi}{enumii}{}{.}%
\item {} 
认知依存:用户的认知决定了他的偏好

\item {} 
情境依存:有情境才有用户,脱离情境就没有用户

\item {} 
经验反馈:具有经验反馈演化的特征,是不断变化的

\end{enumerate}
\begin{itemize}
\item {} 
商业价值

\end{itemize}

有用户价值的产品中,又有很大一部分没法赚钱盈利,它们要么靠团队的情怀加资金积蓄、要么靠一个更大组织里的其他团队来输血、要么靠风险投资人对未来可能性的认可。

但这些都只能短期解决问题,长期来看,是无法支撑的。所以,一个可以长期独立生存的产品,一定要有\sphinxstylestrong{自我造血}能力。没法创造商业价值的产品,只能是一个真正的闭环产品的部分模块。

一个产品是否有商业价值,也是评判一个产品人是否具备“端到端”能力的标准之一,在大公司里做产品总监、甚至产品副总裁,也未必能训练到这个能力,但要自己创业成功,必须具备这个能力。
\begin{itemize}
\item {} 
社会价值

\end{itemize}

社会价值在于,一个产品不但自给自足,还产生了正向的外部性影响,可以让更多的社会角色受益。

产品有了社会价值,也就意味着,它在该领域的生态中,占据了相对重要的生态位,不那么容易“死”掉了。


\paragraph{Naive?}
\label{\detokenize{chapter_project/valuable:naive}}
“只要是为社会疯狂创造价值的企业,它的收入、利润早晚会兑现。”

做产品经理也一样,只要一直做的是有价值的产品,个人价值也终将会兑现。


\paragraph{价值分析维度}
\label{\detokenize{chapter_project/valuable:id3}}
产品价值分析维度分为四部分:分别是方案闭环、价值闭环、资源闭环、财务闭环。
\begin{enumerate}
\sphinxsetlistlabels{\arabic}{enumi}{enumii}{}{.}%
\item {} 
方案闭环
某个需求和产品是否能够满足用户的基本需求,大部分初中级产品每天的工作就是接需求和做需求,方案做了一两年,大部分产品经理都能够做到一个方案思考逻辑没毛病,方案自然闭环。

\item {} 
价值闭环
一个产品能够做到方案闭环,最多只能说这是一个功能,如果一个产品能称为产品,它必定要有价值闭环,价值是对于用户有用、好用。

\item {} 
资源闭环
大部分产品失败是在资源闭环上存在缺陷,说白了能力大于理想,特别是B端产品,涉及供应链或者各种商户资源,如果没有资源把商户快速圈到平台中,就无法解决交易供给问题,产品功能做得再好,最后也成不了。

\item {} 
财务闭环
移动互联网十年之路,何其辉煌,如果回到2010年,大家能想到大部分移动互联网公司其实都不赚钱,赚钱的公司最后活下来靠的是金融业务。不赚钱,有用户没有收入,财务收入cover不了财务支出,这是大部分互联网公司的痛。

\end{enumerate}


\paragraph{产品价值实现路径}
\label{\detokenize{chapter_project/valuable:id4}}\begin{enumerate}
\sphinxsetlistlabels{\arabic}{enumi}{enumii}{}{.}%
\item {} 
mvp

\end{enumerate}

mvp指的是用\sphinxstylestrong{最小的成本}做对用户有价值的最小可用产品(对没有太强相关的\sphinxstylestrong{克制!})

当不知道做什么的时候,用户需求也把握不准的时候,先做做一个简单的产品方案去试点,这就是MVP。

MVP的产品不是单一的形态。只要可以让你的用户直观地感知到或者让他们实际使用起来,能激发他们真实的使用体验和感受就是可行的。
\begin{enumerate}
\sphinxsetlistlabels{\arabic}{enumi}{enumii}{}{.}%
\item {} 
pmf

\end{enumerate}

pmf指的是产品与市场匹配的产品

判断产品的价值和市场的价值是否匹配,但是如何判断PMF的临界点呢,怎么才算是产品价值和市场匹配呢?

可以参考C端,30\%的新用户次日留存,当然不同的产品类型这个数据可能有差异,如果实在不清楚,找用户做个问卷调查,看看他们对产品的真实反馈。你觉得我怎么样?貌似是向用户的一句表白,大部分收到的反馈是痛彻心扉。

因为高手从来不问你觉得我怎么样,他们不需要用户说,自己能感受到。
\begin{enumerate}
\sphinxsetlistlabels{\arabic}{enumi}{enumii}{}{.}%
\setcounter{enumi}{2}
\item {} 
Aha moment

\end{enumerate}

Aha moment是指产品使用户眼前一亮的时刻,是用户发现产品核心价值的时刻

当产品pmf满足之后,就可以尝试寻找产品的Aha
moment了,一般可以通过数据分析去发现产品的ahamoment。

在数据分析的时候去找那些用户使用了的功能或者服务,产品留存数据特别好,如果不使用,产品留存就会比较差的临界点。比如Facebook
的aha moment 是新用户有6个好友,留存率就特别高。

从pmf到Aha moment 是连续的过程,不能从mvp跨越pmf直接达到 Aha moment。

丁香医生的少楠他给我讲了丁香医生当时是怎么发展起来的,起先团队也不知道怎么做,当时产品模型搭建好之后,看似有很多方向,但是却一块也不强。

没有天生的赢家,在产品成功之前,楠哥和大部分人一样,也是懵逼的。接着看数据发现青春痘问诊数据特别好,所以后来丁香医生就专攻青春痘问诊,包括现在青春痘问诊也是核心业务,这就是找到产品的ahamoment,然后快速的复制。

反观有些产品,别说aha
moment了,pmf都没有,留存率特别低,所以只能回到用户中去再次调研,继续调整产品。就像挖一口井一样,有些地方天生不适合挖井,有些地方有地下水,但是没等到最后,团队因为投入问题而解散,这是大部分失败产品的命运。
\begin{enumerate}
\sphinxsetlistlabels{\arabic}{enumi}{enumii}{}{.}%
\item {} 
产品反馈
\sphinxhref{https://coffee.pmcaff.com/article/2753643635311744/pmcaff?utm\_source=forum}{4}%
\begin{footnote}[285]\sphinxAtStartFootnote
\sphinxnolinkurl{https://coffee.pmcaff.com/article/2753643635311744/pmcaff?utm\_source=forum}
%
\end{footnote}

\end{enumerate}

能够方便的集成在产品中,支持安卓iOS公众号小程序和WEB端,兼容PC视图和移动端视图\sphinxhyphen{}>可以识别反馈用户的身份,环境信息\sphinxhyphen{}>可以选择反馈分类,支持图文反馈\sphinxhyphen{}>有完善的通知机制和自动化机制\sphinxhyphen{}>所有反馈集中在后台显示,可回复可流转\sphinxhyphen{}>反馈回复能通知到用户\sphinxhyphen{}>用户可对回复进行评价,是否解决用户问题\sphinxhyphen{}>有完善的数据统计机制,能统计问题/回复数量,可统计运营人员工作量\sphinxhyphen{}>能导出完整的反馈/回复数据,便于进一步分析,构建用户画像,指导产品迭代。

如果是你来设计,是否能设计的比这个流程还有完善,如果要完成这些功能,需要多长的研发周期

埋点:进行二次或三次的拆解,了解每个节点数据流向,后续才能建立数据模型,分析用户操作。

异常监控:人力有限,我们无法做到24小时实时盯着数据。若功能中有影响用户核心操作的功能点,还需要让研发帮忙建立异常监控,以保证产品核心流程能够满足用户正常使用。
\begin{enumerate}
\sphinxsetlistlabels{\arabic}{enumi}{enumii}{}{.}%
\item {} 
完善的产品社区体系

\end{enumerate}

MIUI社区对于MIUI的成功来说可谓功不可没。

能在产品社区持续反馈产品问题,与产品方深入沟通自己的需求和使用场景,做任何调研都积极配合,有这么一帮铁杆种子用户,产品想不做成都难。

不过社区是一把双刃剑,良好的社区氛围能感染很多用户成为产品铁粉儿,成为产品持续优化迭代的源泉。但难就难在怎么做好社区氛围,氛围不好的社区就会成为一个产品负面消息的集散地。

良好的社区氛围和社区产品的体验以及社区运营能力有很大关系。

社区产品必须具备基本的登录、发贴、回贴、消息、个人主页功能,而帖子审核、置顶、关闭评论、点赞、标签、举报又是非常基础的运营需求。


\paragraph{价值的选择问题}
\label{\detokenize{chapter_project/valuable:id5}}
我经历的好几家公司都是以市为目的。之前在公司的时候,经常有投资者来公司参观,公司的很多决策也会按照给投资人描绘的进行布局,有些项目根本不赚钱,但是为了让故事好看,这些项目还是一定要做。

产品的工作向来不是如此纯粹,还得和内心做抗争,还得被程序员误解需求不过脑子,哀莫大于心死,一个项目死了,收拾心情,换个项目又是一条好汉。

从A股股王茅台对比美国的股王苹果,可见一二,国内的产品氛围大多还是销售(流量)驱动,国内的环境离真正的产品价值驱动还有一段距离要走。

在大环境不好的时候,做选择,选择相对价值高的事情。人和人的区别就是从做事是否有价值中拉开差距的,价值的累加就是赢的感觉,谁也不想做输家。


\subsubsection{推广产品的必由之路:运营思维的运用}
\label{\detokenize{chapter_project/widespread:id1}}\label{\detokenize{chapter_project/widespread::doc}}

\paragraph{运营的本质}
\label{\detokenize{chapter_project/widespread:id2}}
产品经理主要是负责产品设计阶段的工作,运营是负责产品推广的工作,把产品当成一个孩子来看,产品设计阶段就是“生孩子”,产品推广阶段就是“把孩子养大,而且还必须养好”。

产品设计阶段主要是为用户提供长期价值,在运营阶段主要是完善产品长期价值,并提供给用户短期价值。


\paragraph{体系的建立}
\label{\detokenize{chapter_project/widespread:id3}}
比如天猫京东等电商平台为了让商家在618和双11中更好的准备打折促销活动,会在大促之前发布商家作战地图,商家作战地图纵向涵盖了商家所有的活动,比如推广、视觉、商品、社交、内容、用户等,时间跨度包含了筹备期、蓄水期、预热期
、售卖期 、爆发期、返场期、总结复盘等。

运营模型的整理的关键来自于用户的关键行为,比如上图中我们思考的用户从哪里来(线上和线下),用户如何认知我们(认知),如何让用户进店(进店),用户进店之后怎么转化(购买),用户怎么沉淀下来(社群),以及最后如何让用户给我们的产品做自传播(传播)。


\subparagraph{指标确认}
\label{\detokenize{chapter_project/widespread:id4}}
在互联网中,运营指标指的是需要成功完成活动的数据指标,如电商运营中关于流量性的指标独立访客数(UV)、页面访问数(PV)、成交金额(GMV)、销售金额等。


\subparagraph{模块划分}
\label{\detokenize{chapter_project/widespread:id5}}
为了达到运营活动的运营效果需要把整个运营活动按职能进行拆分。比如常见的运营职能有内容运营、数据运营、活动运营、用户运营、渠道运营、市场运营、会员运营、社群运营、商家运营等。


\subsubsection{APP体验}
\label{\detokenize{chapter_project/APP_experience:app}}\label{\detokenize{chapter_project/APP_experience::doc}}

\subsection{Interview}
\label{\detokenize{chapter_interview/index:interview}}\label{\detokenize{chapter_interview/index:chap-interview}}\label{\detokenize{chapter_interview/index::doc}}

\subsubsection{CV}
\label{\detokenize{chapter_interview/CV:cv}}\label{\detokenize{chapter_interview/CV::doc}}
\begin{center}\sphinxincludegraphics{{CV}.jpg}\end{center} \sphinxincludegraphics{{CV2}.png}


\paragraph{5大问题}
\label{\detokenize{chapter_interview/CV:id1}}\begin{itemize}
\item {} 
简历过度包装

\item {} 
简历过分冗长

\item {} 
内容含糊不清

\item {} 
数据敏感度差

\item {} 
借用他人的成果

\end{itemize}


\subparagraph{简历过度包装}
\label{\detokenize{chapter_interview/CV:id2}}

\subparagraph{简历过分冗长}
\label{\detokenize{chapter_interview/CV:id3}}

\subparagraph{内容含糊不清}
\label{\detokenize{chapter_interview/CV:id4}}
抓主要放次要

AI产品经理的基础还是产品经理,更重要的还是为什么做这个产品,这个产品解决了什么问题,你在其中扮演了什么角色,做出了哪些重要策略,算法模型存在哪些劣势你是如何通过产品手段进行用户体验层面的提升等等。


\subparagraph{数据敏感度差}
\label{\detokenize{chapter_interview/CV:id5}}
每个项目都要有合适的数据指标的说明(包括基础指标和对比指标),并且自己对指标有深刻的合逻辑的理解。


\subparagraph{借用他人的成果}
\label{\detokenize{chapter_interview/CV:id6}}

\subsubsection{面试官1\sphinxfootnotemark[286]}
\label{\detokenize{chapter_interview/interviewer:id1}}\label{\detokenize{chapter_interview/interviewer::doc}}%
\begin{footnotetext}[286]\sphinxAtStartFootnote
\sphinxnolinkurl{https://mp.weixin.qq.com/s?\_\_biz=MzIyOTAyOTEyNw==\&mid=2649631891\&idx=1\&sn=e5069d1c3e77ebd781da522ad787fb48\&chksm=f05268fbc725e1edc0987f8c94c5e497fd043177f819d0fe00aae3e6bf60423156ada713f83c\&mpshare=1\&scene=1\&srcid=0411XEbDZhF7chwQG50zMtJA\&key=52f65e2fc335f0816695259594ca021e3d2476a2cafaa96f3994e7588555cfa65895fc5e48257cb85115b1e2a25142ef955e698982337df732dbd52505ff6ffd2769e5aa3847377e4fb6a6594941e866\&ascene=0\&uin=OTYyNDg4NjIx\&devicetype=iMac+MacBookPro14\%2C1+OSX+OSX+10.12.5+build(16F2073)\&version=12020810\&nettype=WIFI\&lang=zh\_CN\&fontScale=100\&pass\_ticket=BRibOyqRAz6gRljQC9sbQ9pSXaaPwwqIN7vjp9uDpWetLencjvDMAKSRN\%2FIVeI4k}
%
\end{footnotetext}\ignorespaces 
如果也能在交流中搞清楚面试官的风格、偏好、真实需求,再决定说什么,怎么说,这就好比打听到了标底再投标,无往而不利。


\paragraph{真实需求}
\label{\detokenize{chapter_interview/interviewer:id2}}
想招个人,从github上把开源项目A抄下来,然后攥弄攥弄接到自己的数据库上,每天生成个报表页面给领导看。这个,才叫真实需求。知道了这个,你要做的,不是照上面那串词儿买一系列的《二十一天精通某某某》临时抱佛脚,而是去github上把A项目弄清楚,同时别忘了在自我介绍的时候说一句“我对A项目挺熟的”。


\paragraph{流水作业型}
\label{\detokenize{chapter_interview/interviewer:id3}}
有些面试官的风格,是简单直接的:一上来二话不说,扔过来一个问题让你自个儿想,他跑到一边继续闷头回邮件。

一般是业务繁忙的中层干部,为了快点儿搞定面试任务,设计了一条流水线,翻来覆去就那么几个问题,给个“Pass”或“No
Pass”了事。他根本不怎么跟你交流,怎么对付呢?其实在这种流水线里,多数问题都是套路活,在各大公司的面试题集锦里都能找到原形。所以,为了顺利兑付这样的面试,有目的的刷题,是必不可少的。

另外,这种风格的面试官,喜欢的是踏踏实实干活的人,所以在态度上,要表现得沉静乃至木讷,特别要流露出对频繁跳槽的厌倦,和对007式加班生活无限的憧憬。

还有最关键的一点,为了不让他觉得你“太行”,就算是这些题目你对答如流,也千万要做出冥思苦想的样子,而且要找个不太紧要的地方,卖个破绽,让面试官在居高临下的指摘和纠正中心满意足地写下“Pass”。


\paragraph{职业经理型}
\label{\detokenize{chapter_interview/interviewer:id4}}

\paragraph{骗方案}
\label{\detokenize{chapter_interview/interviewer:id5}}
网上搜了一些相关吐槽,发现互联网圈和非互联网圈都狠普遍。设计师被骗设计方案,市场被骗推广方案,产品经理被骗产品方案,被骗PRD,运营被骗运营方案。

我们产品圈流传着这样一个段子,当某个产品经理接到老板任务时候,不是需求分析,不是竞品分析,而是让HR找一些人来面试,面试时候把问题抛出来,看面试者怎么解决,一天面试下来,把方案汇总一下,就可以找老板汇报了。所以你会在拉钩或者boss直聘上看到,一个岗位一直挂那儿,总感觉没人去应聘的,长时间HR不处理简历,突然有一天喊你去面试了,面试完了就杳无音信了。哎,吃相真丑!

当然一些大公司也不例外,我就知道某大型电商平台也特别喜欢干这事。但是骗学生就有点过分了,无非是浪费人家劳动力。甚至某知名企业校园招聘时候,让面试者做企业宣传视频并传播,根据视频质量和视频的播放量决定复试名额,然而潜规则是,早就定好复试的人选,你视频做的好和不好跟你进复赛没啥关系,不过让学生做了一次免费推广罢了。


\subparagraph{骗方案的套路}
\label{\detokenize{chapter_interview/interviewer:id6}}\begin{enumerate}
\sphinxsetlistlabels{\arabic}{enumi}{enumii}{}{.}%
\item {} 
你自己的产品/运营方案,你觉得怎样?怎么修改?

\item {} 
整理一个完整的方案吧/刚刚那个需求,写个PRD发给我吧。

\item {} 
套信息

\item {} 
要以往的PRD/要以往的设计稿等等

\item {} 
HR电话催

\end{enumerate}


\subparagraph{给予面试者最基本的尊重}
\label{\detokenize{chapter_interview/interviewer:id7}}
一个免费的方案背后,是候选人的经验积淀,时间成本以及对工作的憧憬。


\subparagraph{我们要看到面试者的诚意}
\label{\detokenize{chapter_interview/interviewer:id8}}
如果拿方案来体现所谓的诚意,实在以偏概全了吧!面试者顶着烈日/冒雨前来面试不能体现诚意?打扮得体,一身正装过来面试不能体现诚意?面试前搜集了公司资料,对公司产品有所了解没有诚意?我作为产品经理,都投资了一把你们的理财产品没有诚意?偏偏是你要给我出方案,才是我的诚意,这是什么逻辑!

面试尊重是相互的,诚意也是相互的,我准时准点到了,面试官是不是也准时开始面试?我做了充分准备,调研了很多公司资料过来面试,面试官是不是面试前也仔细阅读了我的简历?我认认真真的面试,面试官是不是真诚和我沟通还是敷衍的过下场子而已呢?


\subparagraph{甄别,是不是骗方案}
\label{\detokenize{chapter_interview/interviewer:id9}}
\sphinxstylestrong{对于一些机密性的信息,尽量打马虎眼},这起码是对你上家公司的尊重,因为面试官不可能一上来问你一些很机密的东西,如果感觉有可能被骗了,不妨用一些假数据来迷惑一下,专业的运营经理会对你的假数据很敏感,多问几个就会找到破绽的,他们神情会表现出来的,而套信息的,基本都是记下来,然后不仔细问了,因为他们套信息的目的已经达到了。

尽可能的保护自己作品,他们问你要方案,如果你很想进这家公司,\sphinxstylestrong{尽量以自己做成PPT}下次我来当面演示给大家看。其实只要不是居心叵测的公司,他们很乐意这样,因为能够更具体了解面试者的实力。

你也可以拒绝,人家问为什么,机灵点,我和上家公司签了保密协议之类的,若对方原形毕露耍无赖非要你发,面试出来就可以把这家公司拉黑了。不到万不得已尽量不要撕破脸,毕竟一个圈子的,圈子很小,名声很重要。


\subparagraph{这个职位的理想人选是什么样的? 3\sphinxfootnotemark[287]}
\label{\detokenize{chapter_interview/interviewer:id10}}%
\begin{footnotetext}[287]\sphinxAtStartFootnote
\sphinxnolinkurl{https://www.yuque.com/weis/pm/up33vm}
%
\end{footnotetext}\ignorespaces 
我真的很喜欢这个问题,因为这个问题以一种全新的方式诠释了招聘方对你的期望。如果你的面试官可以凭空创造一个填补这个职位的理想人选,那会是什么样的人?有时候他们会觉得你就很理想,但有时候他们也会说一些和你的背景、技能或偏好都不相符的人选。你可以通过这种方式很好地了解你是否适合这家公司。


\subparagraph{你如何衡量开发团队、个人或公司成功与否?}
\label{\detokenize{chapter_interview/interviewer:id11}}
这又是一个过程问题。我想知道我的工作以及团队的工作会被如何评估。如果他们无法回答这个问题,那就换个思路,问他们是怎么判断自己做得不好的。在我看来,如果看不到做得好的地方,只能看到做错的地方,那这绝对是一个危险信号。如果你不知道成功是什么样子,那你怎么能成功呢?


\subparagraph{新员工的入职计划是什么?你该如何让新入职成员融入团队?}
\label{\detokenize{chapter_interview/interviewer:id12}}
除非你是初次求职,不然我认为这些问题应该放在最后。头一次找工作的人要了解重要的入职计划和培训计划。但老油条也可以通过这些问题了解他们的答案。我想知道他们会如何帮助新的开发人员开始工作。他们是否思考过如何让新入职的职员更容易地融入新公司?当然,如果公司没有这些计划也并无大碍,因为大部分公司都没有。

类似问题还有,他们是否会雇佣初级开发人员以及他们如何与这些开发人员共事,但问这个问题的前提是我们已经工作过一段时间了——我们并不是职场菜鸟。我已经工作快三年了,但我并不想给任何人任何建议。高级工程师可以提这个问题,他们不容易被误认为菜鸟。这样他们可以知道这家公司是如何看待员工价值的。


\subsubsection{HR}
\label{\detokenize{chapter_interview/HR:hr}}\label{\detokenize{chapter_interview/HR::doc}}

\paragraph{应聘前}
\label{\detokenize{chapter_interview/HR:id1}}
已知信息:企业的业务模式,企业服务的客户,应聘岗位名称,工作职责和任职要求

你的诉求:产品面试整个阶段应该了解什么信息,向谁了解
\begin{enumerate}
\sphinxsetlistlabels{\arabic}{enumi}{enumii}{}{.}%
\item {} 
应聘者的时间相对是自由的,而HR分配到使用boss进行沟通邀约的时间是相对固定并且较少的
\sphinxhref{https://wen.woshipm.com/question/detail/5tfpes.html?sf=wipm}{7}%
\begin{footnote}[288]\sphinxAtStartFootnote
\sphinxnolinkurl{https://wen.woshipm.com/question/detail/5tfpes.html?sf=wipm}
%
\end{footnote}

\item {} 
一般大厂会直接招实习,但是这家企业没有明确说是实习岗位,而是挂产品助理,外加产品经理的标签来加持,证明企业在行业内的竞争力应该一般,至少在人才吸引方面不会太强;

\item {} 
因为应聘者并没有表现出对岗位对公司的诚意,即到场面试。boss上的所有操作,目的是邀约而不是面试,这是两个概念,所以在什么环节干什么事,想要了解岗位的具体情况,就要到场面试,无论对于HR更好的了解你还是你更充分的了解企业,面试都是最佳选择,既体现尊重又便于双方高效的相互判断。

\end{enumerate}


\paragraph{HR面试是拿Offer前的最后一关 4\sphinxfootnotemark[289]}
\label{\detokenize{chapter_interview/HR:hroffer-4}}%
\begin{footnotetext}[289]\sphinxAtStartFootnote
\sphinxnolinkurl{https://weread.qq.com/web/reader/8d232b60721a488e8d21e54k66f3299023a66f041e16858}
%
\end{footnotetext}\ignorespaces 
业务面试重点考察求职者的能力能否胜任产品经理的工作,HR
面试从其他方面对求职者做最终的把关,并且由于HR
负责控制公司的整体人力成本,因此与薪酬相关的内容也是在HR
面试中完成的,如果求职者和HR
没有就薪资达成一致,那么也会导致面试失败。因此,要想拿到Offer,即使轻松通过了业务面试也不要大意,HR
面试依然要好好准备。


\paragraph{谈薪资 8\sphinxfootnotemark[290]}
\label{\detokenize{chapter_interview/HR:id2}}%
\begin{footnotetext}[290]\sphinxAtStartFootnote
\sphinxnolinkurl{http://www.woshipm.com/zhichang/2301423.html}
%
\end{footnotetext}\ignorespaces 
“我相信公司会给我一个合理的薪资安排。”

增加面试官的工作负担,并将决策风险转给了面试者。对于这种不愿意当风险承担者(价高就被拒)的情况,对于一些要求比较高的或者说看中主人翁精神的面试官可能直接就会拒绝你,去选择其他明确的候选人。

正确做法应该是你给出一个你的期望薪资,并补上一句“我可以接受一定程度的薪资浮动”。


\paragraph{考察重点分析}
\label{\detokenize{chapter_interview/HR:id3}}
传统行业HR 面试更加注重求职者的稳定性,所以HR
所提的问题更多的是私人问题(如家庭成员、籍贯等),因为HR
要确认求职者在未来几年内可以为公司持续创造价值。而互联网行业的人员流动性较大,所以互联网行业的HR
对求职者稳定性的考察会稍微弱一点,而是会重点考察其他方面。


\paragraph{性格}
\label{\detokenize{chapter_interview/HR:id4}}
求职者最好是一个相对开朗、善于交际的人,也就是具有优秀的沟通能力。求职者还必须是一个对一切充满好奇、敢于质疑、追求卓越、能掌控大局,同时又心思缜密、执行能力很强的人。


\subparagraph{最大的缺点是什么?}
\label{\detokenize{chapter_interview/HR:id5}}
对现实的惨状太真实地吐露。


\subparagraph{你最大的兴趣爱好是什么?}
\label{\detokenize{chapter_interview/HR:id6}}
看书思辨与指点江山。


\subparagraph{如果在需求评审时,研发人员说无法实现这个需求,你会怎么做}
\label{\detokenize{chapter_interview/HR:id7}}
在遇到这个问题时不用害怕,求职者可以分3步来解决这个问题。
\begin{enumerate}
\sphinxsetlistlabels{\arabic}{enumi}{enumii}{}{.}%
\item {} 
了解无法实现的原因,是当前技术发展水平的限制,还是公司研发水平的限制?一般来说,只要是竞品可以实现的需求,理论上就不存在当前技术发展水平的限制,而更可能是公司研发水平的限制。

\item {} 
如果确实无法实现,那么产品经理可咨询研发人员,看\sphinxstylestrong{有哪些替代方案}。如果直线走不通,就可以尝试绕路。只要是能实现产品效果的方案,就可以对其进行评估。

\item {} 
如果有替代方案,就评估替代方案对项目的影响。例如,评估替代方案是否会导致项目延期、是否会导致产品方案变更。如果这些都在产品经理可接受的范围内,就可以适当地妥协。

\end{enumerate}


\paragraph{P8的薪资}
\label{\detokenize{chapter_interview/HR:p8}}
P8的薪资=基本薪资(月薪4\sphinxhyphen{}5万,一年45\sphinxhyphen{}60万)+年终+股票。那P8能年薪200万左右,靠的是什么呢?不是靠基本薪资,是年终+股票,尤其是股票占大头。。股票要分4年才能拿完,满2年可以拿一半,如果满2年就走人,另外一半拿不到。另外股票要交税,45\%,拿到手并不多的。

股票的价值来源于哪里,来源于行业的红利+阿里这个平台的成功,反馈回员工身上的。平台利润+行业红利提供了远超出你本身能力的金钱收获。

Y哥,2K对于你来讲,太小的钱了。但对于学生来说,2K意味着很多。我不知道你有没有在大学食堂吃过饭,这些学生一顿饭才花8块钱。有些学生为了省1,2块钱,盘子里都是米饭,都看不见菜。2K块钱,不是小钱,是他们2个月的生活费。对于你们来说,为什么不花这2K块钱,买来这个学生的感激涕零,为公司全力付出。就像做饭一样,已经做了8分熟,为什么不加把火,做成10分熟。少2K块钱,他心不甘,又去看这个公司那个公司,心在摇摆。咱们能不能多花2K块钱,让他彻底心安,一心一意等待入职。\sphinxhref{https://www.zhihu.com/people/guosheng-hu/answers/by\_votes}{2}%
\begin{footnote}[291]\sphinxAtStartFootnote
\sphinxnolinkurl{https://www.zhihu.com/people/guosheng-hu/answers/by\_votes}
%
\end{footnote}


\paragraph{禁忌离职原因1\sphinxfootnotemark[292]}
\label{\detokenize{chapter_interview/HR:id8}}%
\begin{footnotetext}[292]\sphinxAtStartFootnote
\sphinxnolinkurl{http://www.woshipm.com/zhichang/459131.html}
%
\end{footnotetext}\ignorespaces 
(1)因为收入低而离职

这样回答会让HR觉得你计较个人得失,认为你工作的
意义就是为了收入,并会猜想如果有更高收入的地方你会毫不犹豫的离职,而对你做出负面判断。

(2)因为分配不公平而离职

绩效工资、浮动奖金等是很多企业用来刺激员工提高工作效率的手段,用以体现努力和结果的结合,加之工资保密制度的实施,面试者用此作为借口会让HR以为你有喜欢打探别人隐私的嫌疑。

(3)因为人际关系复杂而离职

团队精神是大多数企业要求员工要具备的素质,拿人际关系复杂做原因HR可能会觉得你在人际交往中有所欠缺,没办法很好的融入群体。

(4)因为上司的为人问题而离职

一味的讲述上司的毛病,会在一定程度上说明你是缺乏工作上的适应性,同时,HR也会联想到你遇见麻烦的客户时会不会也凭好恶行事。

跳槽,只不过是一种双方解除之前的契约,重新和另外一家公司签订契约的过程而已,直面这个本质,你会抛弃很多顾虑。

我们在和其他公司合作时候,并不是说我和这家合作公司关系不好我就不合作了,而是这个渠道带来的ROI太低了,回报远远低于我的付出,那么我们会毅然决然的解除和这个渠道的合作。同样的,当你不开心了,当你觉得拿的少了,都不是你终止这份契约的标准,唯一的标准就是,你的付出回报比太低了,低于你的底线了,此时就解除这份契约吧。\sphinxhref{http://www.woshipm.com/zhichang/906380.html}{5}%
\begin{footnote}[293]\sphinxAtStartFootnote
\sphinxnolinkurl{http://www.woshipm.com/zhichang/906380.html}
%
\end{footnote}


\paragraph{选择}
\label{\detokenize{chapter_interview/HR:id9}}
Offer
的选择是综合考虑行业、城市、公司、待遇等多个方面的因素而得出的一个最终结果。


\subparagraph{行业}
\label{\detokenize{chapter_interview/HR:id10}}
行业发展:互联网教育、医疗有很大的发展空间,因为这两个方向的需求是刚需


\subparagraph{城市}
\label{\detokenize{chapter_interview/HR:id11}}
互联网发展最好的5个城市分别是北京、深圳、杭州、上海和广州


\subparagraph{个人兴趣}
\label{\detokenize{chapter_interview/HR:id12}}
仅考虑发展趋势还不够,还要结合自己的个人兴趣,你只有从内心喜欢这个行业,才能发挥最大的主观能动性,最大限度地发挥自己的创意,才更有可能升职加薪。最理想的状态是自己喜欢的方向恰巧未来的发展空间很大。如果二者不能很好地融合,那么我的建议是行业发展的优先级大于个人兴趣。


\subparagraph{公司}
\label{\detokenize{chapter_interview/HR:id13}}
要综合比较公司规模和业务线的重要程度。如果是大型公司的核心业务和创业型公司的核心业务对比,就选择大型公司;如果是大型公司的边缘业务和创业型公司的核心业务对比,就选择创业型公司;如果是大型公司的边缘业务和创业型公司的边缘业务对比,就选择大型公司。


\subparagraph{待遇}
\label{\detokenize{chapter_interview/HR:id14}}
在考虑待遇时,不能只看月薪,而要综合考虑。要了解清楚季度奖、年终奖的数额,有无饭补、房补、交通补助,公积金的缴纳基数及比例,有无加班费等。


\subparagraph{有什么需要我特别注意的部门吗?}
\label{\detokenize{chapter_interview/HR:id15}}
在大公司,他们认为「销售」部门需要特别注意,他们会让你知道销售就像是这里的神祇,不要惹恼他们。在小一点的公司,他们会告诉你这方面没什么需要注意的。通过这个问题你可以了解一些第一天工作要知道的事——实际发号施令的是谁?是否存在有些人觉得不值但也有些人很喜欢的项目?如果他们不介意告诉你一些秘辛,这会在你入职的头几周帮到你。这个问题也表示你很想融入公司,想和周围的人进行适当的沟通。


\subsubsection{令人喜爱的新人1\sphinxfootnotemark[294]}
\label{\detokenize{chapter_interview/new_like:id1}}\label{\detokenize{chapter_interview/new_like::doc}}%
\begin{footnotetext}[294]\sphinxAtStartFootnote
\sphinxnolinkurl{http://www.woshipm.com/pmd/284339.html}
%
\end{footnotetext}\ignorespaces 

\paragraph{心态}
\label{\detokenize{chapter_interview/new_like:id2}}
在解决需求的过程当中,对产品产生关注和信任,在对工作方法和心态都了解之后,自然而然的就变成产品经理了。产品经理是对某个问题提出一系列解决方案和推动落实的那个人,你要对用户负责,对产品负责,更要对你自己负责


\paragraph{懂行}
\label{\detokenize{chapter_interview/new_like:id3}}
如果大家平时就能写一下文章,或者对一些新出的产品给一些自己的见解或者改进方案,不妨整理好放在iPad里面,面试带过去效果会很好!了解行业动态推荐关注:虎嗅网、36氪等科技媒体,学习产品基本功,推荐关注:人人都是产品经理.


\paragraph{经历}
\label{\detokenize{chapter_interview/new_like:id4}}
撬开BAT大门的基本配置=1\sphinxhyphen{}2份有料的互联网实习+1次以上的项目经历+个人作品或高质量比赛大奖。


\paragraph{识人}
\label{\detokenize{chapter_interview/new_like:id5}}
过来人的帮助不仅能够让你获取更多的内部信息和资源(很多实习机会都是内部消化的),还可以在关键的时候给你一些比较全面的辅导和指引。所以可以通过参加一些活动,或者进入一些交流的微信群、QQ群等认识一些在互联网工作的师兄师姐或者HR


\paragraph{包装}
\label{\detokenize{chapter_interview/new_like:id6}}
得到了讯息,看到了机会,也有经历之外,还需要懂得如何把最好、最真实的自己呈现出来,这就很需要个人修养了。


\paragraph{学习}
\label{\detokenize{chapter_interview/new_like:id7}}
学会Xmind,Mindmanager,Visio,PS,Axure这些软件,另一方面还要多看书


\subsubsection{常见问题}
\label{\detokenize{chapter_interview/question:id1}}\label{\detokenize{chapter_interview/question::doc}}

\paragraph{如何验证学习效果?}
\label{\detokenize{chapter_interview/question:id2}}
给自己编30道面试题,对着镜子模拟面试,直到能得满分为止。再换下面的30道,这样一轮轮下来,遍历自己的认知边界。发现边界太窄,就继续扩张边界。

面试AI团队,最忌讳对AI认知过浅。咱们换位思考,假设你是面试官,遇到的求职者讲不清朴素贝叶斯、SVM、NLP瓶颈边界、神经网络类型和原理,不能深入分析某个AI场景……公司又不是慈善机构,凭什么给他工作机会?


\paragraph{专业能力}
\label{\detokenize{chapter_interview/question:id3}}
至于产品经理的专业能力,面试时一般从四个角度判断:
\begin{enumerate}
\sphinxsetlistlabels{\arabic}{enumi}{enumii}{}{.}%
\item {} 
批判性思维;

\item {} 
同理心;

\item {} 
用户模型;

\item {} 
产品技能熟练度。

\end{enumerate}


\paragraph{介绍一下你自己}
\label{\detokenize{chapter_interview/question:id4}}
介绍一下自己的姓名,年龄、毕业院校,工作经历。简单的介绍,保持在三分钟以内,给面试官问问题的时间。

工作经历主要讲一些你牛逼的工作经历,例如:你加入XX公司以后,销售额增加了多少、用户翻了多少倍…这样一些。有些人工作经历比较多,3年跳了好几家公司,建议你合并一下,不然面试官会觉得你这个人没有定力,在其他家公司干的时间都不长,在我公司能干多久?

至于你的毕业院校牛逼的肯定要说出来,如果觉得学校不好,不好意思说那就不说吧。

只能简单说几句,几年职业生涯一笔带过,自我介绍不超过1分钟的。说明这个候选人在描述一个事情上是说不好的,不懂得组织文字和语句,这样工作时往往不能很好的表达自己的观点。\sphinxhref{https://www.yuque.com/weis/pm/emr7ca}{10}%
\begin{footnote}[295]\sphinxAtStartFootnote
\sphinxnolinkurl{https://www.yuque.com/weis/pm/emr7ca}
%
\end{footnote}

讲的和简历上写的不一样,或者漏掉重要的经历。这种候选人需要特别注意,后面要多询问,如果不是口误或者忘了的话,简历很可能就是过度包装的。


\paragraph{真正的选拔}
\label{\detokenize{chapter_interview/question:id5}}
我真正想听的是这种问题:
\begin{itemize}
\item {} 
你过往经历里方向背景是B端还是C端?

\item {} 
你设计的系统处理业务是属于电商类?社区类还是文娱?

\end{itemize}

这张图怎么用呢?举例来说如果一家公司要求学历为大专及以上,那么如果你在面试的自我介绍中直截了当的告诉面试官我毕业于某某大学,那么面试官心中这里就直接在给你加上了学历匹配的5分。

\begin{figure}[H]
\centering
\capstart

\noindent\sphinxincludegraphics{{pipei}.png}
\caption{匹配阶梯}\label{\detokenize{chapter_interview/question:id12}}\end{figure}

根据上面六项做完匹配后,根据最终的得分我们可以将候选人直接分为这几类档:

\begin{figure}[H]
\centering
\capstart

\noindent\sphinxincludegraphics{{dang}.png}
\caption{候选人分档}\label{\detokenize{chapter_interview/question:id13}}\end{figure}

直观来说与其招到一个能把原型画到100分但是薪资相对于其他90分左右的产品提高了30\%,对于企业来说是没有多少意义的。

反之如果在高层级产品上选对了人,他能在重大决策上有100分的判断,例如在直播还在百花齐放的时代,有人先转型涉足了短视频领域做出了抖音,虽然他的薪资成本提高了30\%但是带来的意义是巨大的。

因此在公司层面来说对执行层人员的选择更多是选合适,而不是一定要非常优秀。

这里我把这种现象称之为\sphinxstylestrong{低端饱和竞争}。

所以大家在后面的产品发展道路上除了要去磨练自己的一般性技能外,还要去主动学习更高层次的产品经理所要掌握的技能,不要用战术上勤奋掩盖战略上懒惰。


\subparagraph{战略}
\label{\detokenize{chapter_interview/question:id6}}

\paragraph{一是围绕他过去做过的某件事(一般是某个产品或某个功能,也可以是其他)进行深入提问。}
\label{\detokenize{chapter_interview/question:id7}}\begin{itemize}
\item {} 
你开始为什么要做这个东西?

\item {} 
你是怎么想到的?

\item {} 
针对以后的情况你考虑了哪些方面和做了些什么?

\item {} 
开始做的过程中遇到了什么问题?

\item {} 
你是怎么解决的?发现了什么?

\item {} 
什么东西和原来理解的不一样了?

\item {} 
又发现了哪些相关的洞察,你和外部市场上的其他人对其认知不一样?

\item {} 
在迭代过程中又是怎么做的?

\item {} 
你现在对这件事情的认知有什么变化?

\item {} 
如果再回到当初,你还会做什么?

\end{itemize}

TODO: \sphinxurl{https://www.infoq.cn/article/kkfAX67GAlsRSP65JvBy}


\paragraph{为什么想做产品经理2\sphinxfootnotemark[296]}
\label{\detokenize{chapter_interview/question:id8}}%
\begin{footnotetext}[296]\sphinxAtStartFootnote
\sphinxnolinkurl{http://www.woshipm.com/zhichang/315041.html}
%
\end{footnotetext}\ignorespaces 
热爱: TODO
个人经历——自己设计的快记账项目,成功将自己理解的需求转化为了具体的功能实现,上线后数据情况良好并得到了用户良好的评价反馈。


\paragraph{你用了我们的产品么?对我们的产品有啥建议?}
\label{\detokenize{chapter_interview/question:id9}}
正确回答:(先吹嘘一番)公司做的这个业务市场规模很大,很有前景,而且我们公司是做的比较好的,产品体验也不错,尤其XX地方,设计的很好,用户体验很棒,但个人认为在一些细节上还有优化的空间,XX功能如果XX做的话会更好一些。


\paragraph{你最喜欢的产品是什么?3\sphinxfootnotemark[297]}
\label{\detokenize{chapter_interview/question:id10}}%
\begin{footnotetext}[297]\sphinxAtStartFootnote
\sphinxnolinkurl{http://www.woshipm.com/pmd/2891945.html}
%
\end{footnotetext}\ignorespaces 
用户体验的五个层次出发,显得有条理:战略、范围、结构、框架、表现

比如从战略层,产品的核心是解决了什么用户痛点,这个市场规模能有多大。那么你回答的信息整合和用户调性,本质上都是它的产品定位以及配套的业务范围,也许是整个团队需要努力保持的核心竞争力。

结构方面,关注流和推荐流的层级设计,比如抖音是推荐主导,微博是关注主导,也能体现产品设计的侧重点。

构架和表现层,其实移动互联网时代的交互已经很久没有大的变化了,所以大多数正常产品挑不出大毛病,偶尔有一些细节彩蛋就会让人很惊喜,比如你说的blabla……”


\paragraph{问清题目4\sphinxfootnotemark[298]}
\label{\detokenize{chapter_interview/question:id11}}%
\begin{footnotetext}[298]\sphinxAtStartFootnote
\sphinxnolinkurl{https://zhuanlan.zhihu.com/p/108911948\#\%E4\%B8\%80\%E4\%B8\%AA\%E9\%9D\%9E\%E5\%B8\%B8\%E7\%AE\%80\%E5\%8D\%95\%E7\%9A\%84\%E4\%BE\%8B\%E5\%AD\%90}
%
\end{footnotetext}\ignorespaces 
许多面试官在面试的时候,会故意先抛出一个模糊的问题。实际上,他们希望面试者能够经过一些询问理解问题。在这个过程中,面试者能够展现出自己对问题的分析能力以及沟通的能力。前者的重要性参见编程珠玑第一章:明确问题,战役就成功了90\%。后者的重要性在于,问清题目的这个交流过程与面试者入职之后与同事讨论问题的形式非常类似。显而易见,一个能够很难沟通的面试者也很难成为一个很好沟通的同事。


\paragraph{To B的产品,跟To c 的产品,在设计产品过程中有什么不同?5\sphinxfootnotemark[299]}
\label{\detokenize{chapter_interview/question:to-b-to-c-5}}%
\begin{footnotetext}[299]\sphinxAtStartFootnote
\sphinxnolinkurl{https://zhuanlan.zhihu.com/p/33524676}
%
\end{footnotetext}\ignorespaces 
toB
公司产品要保证的是交付给客户的解决方案的结果和质量。\sphinxhref{http://www.ramywu.com/work/2018/04/09/Get-Ready-For-AI-PM/}{8}%
\begin{footnote}[300]\sphinxAtStartFootnote
\sphinxnolinkurl{http://www.ramywu.com/work/2018/04/09/Get-Ready-For-AI-PM/}
%
\end{footnote}所以
AI PM
在推动产品落地的过程中,需要各种团队协作、跨部门沟通、向上汇报等,因为工作目标是交付项目,在项目管理本身有很多蛮内隐知识在里面够学的了。
对2C平台来说,每个阶段的侧重点是不同的,前期更注重日活,后期看GMV。\sphinxhref{https://m.k.sohu.com/d/495625828?channelId=1\&page=1}{7}%
\begin{footnote}[301]\sphinxAtStartFootnote
\sphinxnolinkurl{https://m.k.sohu.com/d/495625828?channelId=1\&page=1}
%
\end{footnote}

自提:

为何前面有快消产品经理所注重的品牌产品经理的内容?


\subsubsection{不能去}
\label{\detokenize{chapter_interview/not_go:id1}}\label{\detokenize{chapter_interview/not_go::doc}}

\paragraph{公司级别的划分标准}
\label{\detokenize{chapter_interview/not_go:id2}}
在业内并没有严格的公司级别的划分标准,一般来说,已上市的集团型公司为大型公司,如阿里巴巴、腾讯、百度、美团、小米、网易等;经过C
轮以上融资但未上市的独角兽公司为中型公司;经过C
轮以下融资或者未经过融资的公司为创业型公司,这类公司非常多。


\paragraph{隐患}
\label{\detokenize{chapter_interview/not_go:id3}}
大多数创业公司,一定一定会进入瓶颈期,这时各种隐患都会冒出来,比如加班多、薪酬不够多(还有猎头来挖人,一对比就。。)、流程制度不规范、士气不高、融资没结论等等。这时,就需要价值观、职业素养、公司凝聚力来救驾了


\paragraph{不能去的创业公司:}
\label{\detokenize{chapter_interview/not_go:id4}}\begin{enumerate}
\sphinxsetlistlabels{\arabic}{enumi}{enumii}{}{.}%
\item {} 
上一轮融资已经烧差不多,自身盈利能力还没起来,需要靠下一轮融资继续经营。如果这时候融不到钱,就会大裁员,作为新员工、应届生,估计第一个被裁。

\item {} 
不好好面试的创业公司。创业公司的早期员工需要是精兵强将,否则养人的费用远高于员工的产出,本身就不盈利的话会死很快。所以从面试流程可以看出这家公司是不是认真的在招每一个人。

\item {} 
不签署劳工合同、不交五险一金的创业公司。合同是保护员工的,不签署对员工很不利,五险一金更不用说了。

\end{enumerate}


\paragraph{老提当年}
\label{\detokenize{chapter_interview/not_go:id5}}
创始人实则是一个码人的角色,不要说自己是什么什么出身,那即代表你远离你的「出身」很久了,long
long ago 的 story 不必再提,好汉不提当年弱,更不要提当年勇。

码人,就是让一直擅长某一个领域的大神、大牛、Expert
在某一个公司管理部门发挥其价值。


\subsubsection{工作}
\label{\detokenize{chapter_interview/offer:id1}}\label{\detokenize{chapter_interview/offer::doc}}
多申请了几家:如果手里没有颇具竞争力的
offer,你在薪资谈判中就会处于严重的劣势。面试练习也的确给了我很大的帮助。

我了解的公司更多了。居然有那么多不错的工作我都没有考虑过!一些最有趣的职位(以及最好的
offer)来自我最初没有考虑过的公司。

我找工作花了挺长时间(从投出第一份简历到接受 offer
大约有半年时间),但我也是精疲力尽:那几个月我基本都是在机场、酒店、面试间度过的,不断接电话、和
HR 谈薪资。

不要指望在这个时期做什么工作。正如一个同事所说:「你的头脑总是被那些招聘反馈占满,没有多余的精力去想
ICML。」

几乎所有的面试官都会抽时间询问职位、团队或公司的相关问题,我喜欢询问工作与生活的平衡、工作的难点或他们对工作不太喜欢的地方,大多数人都会诚实坦率地回答这些问题,这会展示未来工作一些尖锐的问题所在。


\paragraph{公司认可什么价值}
\label{\detokenize{chapter_interview/offer:id2}}
商业价值:
\begin{itemize}
\item {} 
业务价值

\item {} 
技术价值

\item {} 
工程价值

\end{itemize}


\paragraph{公司如何评判价值}
\label{\detokenize{chapter_interview/offer:id3}}\begin{itemize}
\item {} 
业务成就

\item {} 
技术难题

\item {} 
工程设施

\end{itemize}


\bigskip\hrule\bigskip


我认为面试人员常常故意不具体说明一些问题,就是想要看看我的反应,并且愿意提供帮助或与我讨论细节。这种面试从不像是一种对抗过程,而更像是同事之间的讨论。

关于招聘人员

我从没有见过像招聘人员这样将胡萝卜加大棒的把戏玩得如此炉火纯青的一群人。

他们会告诉你「这是我们能给的最高待遇」(一周后就变了),还会跟你强调「他们会为你破例,因为你是一个非常优秀的求职者」(而实际上提供给你的东西和别人相差无几)。

他们还会凭空捏造出严格的
deadline(五分钟之后就会告诉你他们完全接受延期),他们还会告诉你,他们不会重新谈判(但只要你拿出更有竞争力的
offer,他们还是会重新跟你谈),还会说他们在过去两个月内在面试其他候选人(但他们绝对不会放弃跟你谈)。

感觉招聘人员总是想确定我是不是真的想选择另一份 offer
而不是他们的,还是只是利用他们与另一家公司重新谈判。我想这一切都在意料之中,而我最好的建议是始终保持礼貌、耐心和执着。


\bigskip\hrule\bigskip


先分享一下我的经验吧。我帮我们组和其他机器学习组面过50多个candidates(含校招或社招、全职或实习)。面试前一般会有prebrief,就是hiring
manager会描述对candidate的机器学习或者深度学习(DL)的水平的预期,然后面试官相应地设计面试题。

假设candidate真的把整本书吃透了,根据我面试的数据:

Breadth方面:

能过。

Depth方面:

如果candidate本身做过深度学习项目,我一般会结合这个项目问一些深一点的问题,大部分问题回归本质后,答案都在书里。

如果candidate无DL项目经验,我会从一个基础的问题开始和candidate聊,一般这个candidate会有意无意地显示出自己比较擅长的方面,然后我会在这方面问越来越深的问题,判断candidate在最擅长的方面能达到什么水平。假设这个方面与深度学习有关,我一般再深挖的问题无非与方法细节、数学分析、实现细节有关,这些书里都涉及了。

现在工业界用深度学习比较普遍,所以大部分机器学习工作都直接要求深度学习背景了。比如prebrief的时候很多hiring
manager会明确告诉我机器学习breadth/depth只需要关注candidate在深度学习方面的水平。但如果一个职位强调传统机器学习方法的话,我觉得在本书基础上可以根据职位需求多准备一些传统机器学习的基础知识。


\subsubsection{招不到合适}
\label{\detokenize{chapter_interview/hire:id1}}\label{\detokenize{chapter_interview/hire::doc}}
做产品、招牛人、自成长,都是要知道“边界”。

如何才能知道“边界”呢?

从表面看,是格局+正直。站得高,才能看得远,而有清澈的心,才能借由阳光,看到边界。

从本质看,是谦虚。因为边界是动态的,每一步,内外环境都在变,即使前一天看得清楚,后一天都可能一叶障目。

就像当初朱啸虎主动去找(投)滴滴,不仅因为这个方向的市场空间大,更因为”滴滴用了唯一正确的切入方式“——四个不做(不做黑车、不做加价、不做帐户、不做硬件)。

做产品、招牛人、自成长,都是要知道“边界”。

如何才能知道“边界”呢?

从表面看,是格局+正直。站得高,才能看得远,而有清澈的心,才能借由阳光,看到边界。

从本质看,是谦虚。因为边界是动态的,每一步,内外环境都在变,即使前一天看得清楚,后一天都可能一叶障目。


\subsection{Dive}
\label{\detokenize{chapter_dive/index:dive}}\label{\detokenize{chapter_dive/index:chap-dive}}\label{\detokenize{chapter_dive/index::doc}}
​


\subsubsection{纵向深度挖掘}
\label{\detokenize{chapter_dive/zongshen:id1}}\label{\detokenize{chapter_dive/zongshen::doc}}
第2阶段:AI领域的纵向深度挖掘 ————————————————
1.模式识别:图像识别、人脸识别、语音识别、视频识别等;
2.自然语言处理(NLP):自然语言理解(NLU)、自然语言生成(NLG)、知识图谱等
3.智能语音:语音识别、语音合成、声纹识别
4.智能软硬件:芯片、智能软硬件项目流程;
5.专家系统:基于专家知识架构的AI综合应用 6.机器人:基于生物智能的AI应用
7.其他领域的纵向深度挖掘\#​AI产品经理​\#​产品经理


\subsubsection{专家}
\label{\detokenize{chapter_dive/expert:id1}}\label{\detokenize{chapter_dive/expert::doc}}
\sphinxurl{https://www.zaih.com/falcon/mentors/2bye7ddr3cg?from=tutor\_like\_tutor}

金融AI产品经理: \sphinxurl{https://www.linkedin.com/in/molly-xu-28032614b/}

产品经理课程大纲: \sphinxurl{https://www.yuque.com/836488572/vgdm9i/xc28u7}

\sphinxurl{http://reader.epubee.com/books/mobile/bb/bb028d2a403e97a9eb16c4d233e03aef/cover1.html?fromPre=last}


\subsubsection{蚂蚁金服AI首席科学家漆远1\sphinxfootnotemark[302]}
\label{\detokenize{chapter_dive/qi_yuan:ai1}}\label{\detokenize{chapter_dive/qi_yuan::doc}}%
\begin{footnotetext}[302]\sphinxAtStartFootnote
\sphinxnolinkurl{https://tech.antfin.com/community/articles/144}
%
\end{footnotetext}\ignorespaces 
中科院硕士、美国麻省理工学院博士兼博士后、普渡大学计算机系和统计系终身教授,曾赴剑桥大学、哥伦比亚大学、伦敦城市大学、杜克大学、SAMSI、布朗大学等名校和研究院做访问学者

他获得美国科学基金NSF
Career奖;拿了微软的牛顿研究突破奖;在人工智能的顶级会议AAAI做过大会tutorial;曾是机器学习顶级会议ICML的领域主席。


\paragraph{错过}
\label{\detokenize{chapter_dive/qi_yuan:id1}}
2003年的一天,拉里·佩奇来到麻省理工学院招人。漆远对这个新兴公司也动了心,“我们一起吃了饭,但我一心想做学术,想当老师。”漆远说了谢谢,并没有去。那个拉里·佩奇,他跟另一个创始人成立的公司叫谷歌。

同一年,漆远去了英国,在剑桥大学的微软实验室做研究。他帮那里的一位“超级大牛”Chris
Bishop审了几章书,然后他的名字就出现在了前言的致谢里,那本书就是国际上机器学习的一本经典课本《模式识别与机器学习》。

还是在英国,在伦敦城市大学的盖茨比中心有位漆远非常喜欢的老师,在剑桥实习后,漆远到这个实验室待了3个月。大胜李世石的“阿尔法狗”,就是这个实验室后来几位毕业生领导的杰作。

2004年,漆远有个朋友跟他说,有个很好的“泡妞网站”,正从哈佛和麻省理工学院所在的波士顿地区开始推广。他的同学毕业后陆陆续续有些去了那里工作,如今已经财务自由,进入提前退休模式。漆远也没去。那个网站就是Facebook。


\paragraph{加入蚂蚁金服人工智能}
\label{\detokenize{chapter_dive/qi_yuan:id2}}
如今漆远领导的团队里,有大量从美国回来的博士,一些出自高校,一些则是从美国的大企业跳槽而来。由于这个领域发展迅速,很多人手里握着不少工作机会。他对记者说,自己从美国招了不少人回来,但不是“说服”他们,而是“吸引”他们。

2013年,漆远加入阿里巴巴集团并担任副总裁,和另外一名负责人在王坚博士的领导下创建了阿里巴巴DST(数据科学与技术研究院);2015年担任蚂蚁金服集团副总裁、首席数据科学家,其人工智能团队正在研发虚拟机器人。他领导着一个机器学习与人工智能团队从事深度学习、加强学习等人工智能领域的前沿研究和应用。

自己从美国招了不少人回来,但不是“说服”他们,而是“吸引”他们。


\paragraph{普惠金融}
\label{\detokenize{chapter_dive/qi_yuan:id3}}
金融智能的目标主要在三方面:风控信用决策、降低服务成本、和提高用户体验。”

漆远在谈及金融科技的核心时说道:“为此,我们搭建金融智能平台,服务我们的业务并赋能生态伙伴。”蚂蚁金融智能应用范围很广,包括智能客服、交易风控、商家营销、车险图像定损、还有线上贷款310模式(3分钟申请,1秒钟放贷,零人工干预)、反欺诈反套现、乃至基金智能推荐等等。

他特别强调了人工智能技术在小额贷款风控系统中的应用。

过去几年,漆远所带领的蚂蚁金服人工智能团队在自身场景和外部合作伙伴客户场景中都全面开花。对于普通用户而言,无论是保险还是理财,这些业务的背后都有着人工智能技术的应用。

以定损宝为例,2018年5月,定损宝技术版本正式升级,包括将图像识别升级成准确率更高的视频识别,将开放技术平台,从与保险公司一对一理赔系统对接升级成未来保险公司可自助接入定损宝。


\paragraph{融合共创}
\label{\detokenize{chapter_dive/qi_yuan:id4}}
漆远表示,当前 AI 发展过程中的挑战在于融合共创,因此蚂蚁金服希望把自己在
AI 方面的技术对外开放出来。

中和农信是一家专注农村扶贫贷款的机构,漆远的AI团队把他们开发的保护数据隐私的共享机器学习平台分享给中和农信,使得双方可以在保护各自数据隐私的情况下开展基于双方的加密数据来做机器学习。经统计,蚂蚁的共享多方AI风控技术帮助中和农信把农村小额贷款风控效果提升了一倍,同时大规模提升了贷款效率。在数据隐私保护在全世界都变得越来越重要的当下,蚂蚁金服的保护隐私的共享学习技术有着广泛的应用前景。

在共享学习的情况下,大规模提升了中和农信的风控能力,充分保障了数字化贷款业务,实现了
1+1>2 的作用。

漆远在招聘时会告诉对方,那些“相信未来”的人适合回来。想成就一番事业的人,可以来;想要朝九晚五的生活,不要来。“我们是创业公司,很多变化,很多压力,你自己要有心理准备。同时,也有很多机会。”


\paragraph{金融、保险2\sphinxfootnotemark[303]}
\label{\detokenize{chapter_dive/qi_yuan:id5}}%
\begin{footnotetext}[303]\sphinxAtStartFootnote
\sphinxnolinkurl{http://www.fortunechina.com/ztjj/c/2018-11/30/content\_320739.htm}
%
\end{footnotetext}\ignorespaces 
AI要落地,除了平台就是场景,场景非常非常必要。普惠金融这个场景就特别适合AI。普惠要服务很多人、很多中小企业,这里面一定是技术驱动的。人是没有办法做普惠的。而蚂蚁金服恰恰就做的是普惠金融。

人工智能技术在金融领域中的应用,更多的价值来自于提高效率和成本降低。在汽车保险相关的业务中,人工智能已经可以判断车辆的损伤和可修复程度,从而大规模提高保险机构的效率,在一年内节省75万小时的定损员工时。漆远简单换算了一下效率提升所带来的成本节省,表示人工智能技术能够帮助降低10亿元人民币的成本。

业内人士预计,人工智能会替代金融业许多信息收集分析加工的工作和风险定价的功能,特别是那些不内嵌于人们生活场景之中的纯信息分析加工工作。\sphinxhref{http://www.iwshang.com/articledetail/252117}{3}%
\begin{footnote}[304]\sphinxAtStartFootnote
\sphinxnolinkurl{http://www.iwshang.com/articledetail/252117}
%
\end{footnote}

人工智能在保险领域的应用。他以蚂蚁金服的退货险为例,称早期的审核验证依靠人工,接入海量的数据之后,模型大规模提升了退货风险预测的准确度,进而推动了退货险的诞生。类似的事情也在小额贷款业务中发生,将深度学习和强化学习技术应用到传统统计模型之后,模型会更全面地考虑数据之间的关联。

“定损宝”————利用图像技术的车辆定损产品。当车主在行驶过程中不幸遇上了一个小车祸,自己爱车的损伤后需要保险公司定损赔偿。他不再需要耗费精力走联系定损员等繁琐流程,而仅需要将车辆损伤部位拍张照片上传,“定损宝”就可以根据图片对车辆损坏程度定损。这一技术极大的节约了车险公司高昂的定损员培训等其他人力的支出。

从图像的噪音去除、类目识别,到目标检测、原因判断,再到程度判断(损坏程度)、目标跟踪,之后对目标进行分割、多图融合,最终生成决策并进行验证。“定损宝”能够将案件的平均处理成本降低至150元,同时可减少50\%的作业量,更可以解决偏远地区过高峰时期定损员人力不足的问题。\sphinxhref{https://posts.careerengine.us/p/5aa35373b50d5f700921cc43}{5}%
\begin{footnote}[305]\sphinxAtStartFootnote
\sphinxnolinkurl{https://posts.careerengine.us/p/5aa35373b50d5f700921cc43}
%
\end{footnote}


\paragraph{客服}
\label{\detokenize{chapter_dive/qi_yuan:id6}}
人工智能技术带来的效率提升在客服行业也有体现。漆远介绍,目前支付宝的全球用户达到9亿,如果按照传统方式组建客服团队,就需要数万名客服人员满足需求。但目前支付宝的客服规模不到一万,95\%的客服需求由机器人解决。此外,漆远还表示,把衡量真人客服的标准应用在机器人上后,机器人的表现比人类更好。

蚂蚁金服于2017年8月正式对外全面开放以人工智能技术为核心的智能客服的能力。其中结合结合用户行为轨迹的语义匹配模型采用了LSTM+DSSM(Long
Short\sphinxhyphen{}Term Memory + DeepStructured Semantic
Model)的算法创新。该技术首先通过LSTM对用户行为轨迹做一个编码,通过深度排序模型,结合用户之前的历史操作,做到“未问先答”。

借助这项技术,蚂蚁金服双十一智能客服自助服务的比例高达惊人的97\%,目前人工智能客服助理的回答满意度也已经超过了人工客服,系统整体在降低成本的同时服务质量还有了显著的提升。

支付宝人工智能客服“小蚂答”、蚂蚁财富的社区机器人“乐于助人的安娜”\sphinxhref{https://posts.careerengine.us/p/5aa35373b50d5f700921cc43}{5}%
\begin{footnote}[306]\sphinxAtStartFootnote
\sphinxnolinkurl{https://posts.careerengine.us/p/5aa35373b50d5f700921cc43}
%
\end{footnote}


\paragraph{推荐}
\label{\detokenize{chapter_dive/qi_yuan:id7}}
除了读学术论文,他还向笔者推荐了几本没有数学公式的书,比如Crowd
Intelligence(《群体智慧》);Made to Stick(《粘性》);和Good to
Great(《从好到卓越》) 。


\paragraph{蚂蚁金服急需的人才4\sphinxfootnotemark[307]}
\label{\detokenize{chapter_dive/qi_yuan:id8}}%
\begin{footnotetext}[307]\sphinxAtStartFootnote
\sphinxnolinkurl{https://cloud.tencent.com/developer/article/1111954}
%
\end{footnotetext}\ignorespaces 
CSDN:问一些大家都迫切想知道的问题。蚂蚁金服现在估值600亿美金,很多人也希望进入里面工作。您对人工智能团队的要求是什么样的?什么样的人才能够进入到蚂蚁金服的和您一起来工作呢?

漆远:对团队的要求是,既叫座又叫好。

首先能够解决实际问题,见效果,从问题出发,不是拿着锤子找钉子。

希望有技术深度,当然这里面需要平衡,有的同学算法多一点,有的搞工程多一点。

我们的团队不是一个刷单的团队,刷各种外面的公开比赛,我们是真正要解决实际问题,一方面提升蚂蚁金服甚至服务整个阿里经济体,解决大家遇到的核心的AI问题;一方面我们要产生新的产品、新的服务,能够造成新的增长点,这是目标。

这就直接映射到我们对人的需求上来。

我希望加入我们团队的人,首先能够对机器学习技术本身有真正的热爱,没有热爱就比较难做。因为技术说起来很高大上,真正做起来需要投入的精力,不是短期的,也不是表层的。

第二,对于人才我们既需要全栈型的,也需要对某技术特别钻深的。如果两个都很强,那就更好了。


\subsubsection{支付宝}
\label{\detokenize{chapter_dive/alipay:id1}}\label{\detokenize{chapter_dive/alipay::doc}}

\paragraph{历史}
\label{\detokenize{chapter_dive/alipay:id2}}
自从 2008
年首次引入水电煤缴费服务以来,支付宝一直在把线上线下各类场景的服务纳入到版图中来,比如政务服务,阿里系的饿了么、飞猪等生活服务,以及通过生活号和小程序等工具提供的第三方服务等。

在2020 年 3 月 10
日的改版中,除了原先最醒目的扫一扫、收付款、卡包等四大功能以外,首页应用中心展示的应用从
11 个增加为 14
个。在应用下方,新增的服务推荐包括吃穿住行、生活服务、金融产品等。

从用户体验的五个层次看支付宝的改版:\sphinxstylestrong{战略层→范围层→结构层→框架层→表现层。}
\sphinxhref{http://www.woshipm.com/pmd/172931.html}{4}%
\begin{footnote}[308]\sphinxAtStartFootnote
\sphinxnolinkurl{http://www.woshipm.com/pmd/172931.html}
%
\end{footnote}


\paragraph{战略层:连接人与服务}
\label{\detokenize{chapter_dive/alipay:id3}}
BAT三家的发家史都是在做“连接”这件事。百度连接人与信息,腾讯连接人与人,阿里连接人与商品,这是第一阶段的连接。


\paragraph{范围层:连接和“钱”有关的一切服务}
\label{\detokenize{chapter_dive/alipay:id4}}
工具型产品最重要的一点,构建工具使用情景,让用户在这个场景下首先想到这个产品。所以工具型产品的运营,首要的就是使用场景化运营的思维,用户不会因为看到支付宝想到去购物,而会在购物的时候想到用支付宝。

当把支付的场景无限延伸,在此之前一切需要出现“钱”的地方,都换成支付宝来取代呢?

最早的时候,支付宝是PC时代上淘宝时的那个第三方担保的支付工具。后来,手机上出现了支付宝钱包,就是不用输入银行卡号的电子钱包。再后来,有了充话费、买电影票、信用卡还款等生活服务,有了理财产品。接下来呢?继续贯穿到所有的消费场景。

从这个角度上讲,支付宝的方向就是要继续做好并强化这件事:连接和“钱”有关的一切服务。


\paragraph{结构层:支付场景拆解}
\label{\detokenize{chapter_dive/alipay:id5}}
从支付宝9.0来看,白崎总结了一下目前支付宝已涵盖的使用场景:
\begin{itemize}
\item {} 
基础功能:付款、收款、转账、汇款

\item {} 
生活服务:手机充值、信用卡还款、外卖、生活缴费(水电燃气电视固话宽带)、城市服务(违章查询、医院挂号等)、电影、打车、机票、景区门票、游戏充值、寄快递、公交卡充值(目前仅支持羊城通)、校园一卡通充值、缴学费、话费卡转让、爱心捐赠。

\item {} 
理财:余额宝、招财宝、娱乐宝、彩票、股票、理财小工具(汇率计算、存款收益、房贷、记账)

\item {} 
商业:众筹、阿里系电商(淘宝、天猫)

\item {} 
社交:红包、亲密付

\item {} 
国民征信体系(最高级):虚拟信用卡(花呗)、个人信用评级(芝麻信用分)

\end{itemize}

百度也做百度钱包这个产品,我们曾给钱包团队提建议说,赶紧支持北京公交一卡通的充值啊,这么大一块市场难道学不会吗?一上线绝对秒杀支付宝。但是我们得到的答案是:idea并不重要,这个点子被无数的人提过,相信支付宝团队也曾想到过。但是为什么大家都没有做呢?不是因为技术上实现不了,而是人家北京公交集团不愿意跟你连接,你就拿它没辙儿。

理财服务里,受国内A股行情影响,预计近期支付宝会在股票这个版块上继续重点发力。一旦连接成功,前景非常可观。A股市场里,股票账户里的钱是放在第三方存管银行的,支付宝要做的是什么呢?取代银行!以后大家炒股的钱都直接由支付宝托管。

再谈到最高级的国民征信体系,这是阿里作为企业在做政府的事情。虚拟信用卡,直接借钱给你花,牢牢地把你绑在支付宝上,这是干倒银行的另一步。芝麻信用分,现在大家还看不到它的威力,等到以后出国办签证,贷款买房,通通看你的信用分,这是在美国已经很成熟的一套体系,本来是政府的事情,支付宝却抢先来完成。


\paragraph{深耕数字经济:转移到本地生活 1\sphinxfootnotemark[309]}
\label{\detokenize{chapter_dive/alipay:id6}}%
\begin{footnotetext}[309]\sphinxAtStartFootnote
\sphinxnolinkurl{https://www.sohu.com/a/405889196\_114819}
%
\end{footnotetext}\ignorespaces 
支付宝的战略转型:淡化金融标签,将美食、酒店、电影等本地生活服务入口置顶,使产品服务加快渗透人们的日常生活。

在疫情期间,我们日常使用支付宝建康码出行、水电煤缴费、外卖生鲜订购、果蔬商超医药、团购等入口。已经能感受到这个支付app在不知不觉中成为了一个多功能的线上生活服务平台。

体验:各项功能入口横向排布,下拉呈现更多,界面清晰干净,使用简单。点进功能二级页面以后,呈现更多,手机号登录即可;如果之前在关联app上注册过信息,无须重复注册填写,自动呈现;默认支付宝支付;用后退出页面即可,无粘性。总体来说,整洁干净,操作方便
\sphinxhref{https://www.zhihu.com/question/380276570}{2}%
\begin{footnote}[310]\sphinxAtStartFootnote
\sphinxnolinkurl{https://www.zhihu.com/question/380276570}
%
\end{footnote}

意图正在于以金融服务为抓手,在C端向消费延伸,在B端向服务延伸,扩展更多的线下服务与商家场景来拉动新的增长点,做大生态。

支付宝加码布局本地生活生态位,剑指美团,我们可以预见两大巨头对战,本地生活领域硝烟再起。

一些产品管理者还会苦恼:如何制定并管理产品路线、如何根据产品生命周期规划产品目标、如何针对性培养下属、如何分析数据驱动迭代业务?


\paragraph{支付能力 3\sphinxfootnotemark[311]}
\label{\detokenize{chapter_dive/alipay:id7}}%
\begin{footnotetext}[311]\sphinxAtStartFootnote
\sphinxnolinkurl{http://www.woshipm.com/it/632590.html}
%
\end{footnotetext}\ignorespaces 
2017 年 5 月,支付宝率先推出了“收钱码”,
这称得上是一款扭转战局的产品,蓝色二维码一时间风行于全国各地商户的柜台上。同年,支付宝在杭州实现了全线刷码出行,年底正式接入了杭州地铁,之后,支付宝从大本营杭州走了出去。\sphinxhref{https://www.chinaz.com/2020/0312/1116987.shtml}{5}%
\begin{footnote}[312]\sphinxAtStartFootnote
\sphinxnolinkurl{https://www.chinaz.com/2020/0312/1116987.shtml}
%
\end{footnote}

「走出淘系」就是支付宝最初的开放过程,把支付能力作为水电煤气一样的连接物,开放给所有行业,哪怕面向是京东(现已下线)、唯品会这些和阿里系电商业务产生竞争关系的公司都开放。

支付宝还开放了新技术场景下的支付能力,比如近期开放给小米和华为的VR
Pay能力,在购物、直播、游戏、社交等VR场景中,用户可以通过凝视、点头、触摸等控制方式完成交易。这里面不仅有密码、指纹等传统校验身份的方式,还有包含声波、虹膜等生物识别技术的支付能力。生物识别会解决未来生活中非常基础的一个问题,那就是如何真实有效得去识别一个人,这种基础能力的研究可不是一天两天可以完成的,所以说开放不仅仅是一种策略,更是一种能力。


\paragraph{财富号 3\sphinxfootnotemark[313]}
\label{\detokenize{chapter_dive/alipay:id8}}%
\begin{footnotetext}[313]\sphinxAtStartFootnote
\sphinxnolinkurl{http://www.woshipm.com/it/632590.html}
%
\end{footnotetext}\ignorespaces 
财富号是支付宝近期宣布的一个开放项目,即面向基金、保险、银行等传统金融机构提供开放平台,帮助金融机构更好地服务客户。支付宝的财富号形态和微信的公众号有些类似,支付宝提供基础功能和技术服务,财富号的运营维护交给金融机构。按照支付宝对外公布的资料,财富号里提供产品推荐、互动直播、问答社区、企业动态等模块,满足金融机构对于产品售卖、客户维系、营销活动等一系列连接性的需求。相信除此之外,支付宝会开放一部分原生流量给首批入驻的财富号,这意味着金融机构还将有机会通过支付宝平台获取新客户。


\subsubsection{暗物智能}
\label{\detokenize{chapter_dive/dm-ai:id1}}\label{\detokenize{chapter_dive/dm-ai::doc}}

\paragraph{名字来源}
\label{\detokenize{chapter_dive/dm-ai:id2}}
朱松纯教授认为,人工智能领域存在智能“暗物质”,包括功能、物理、因果、意图、价值等,就是人工智能领域的“Dark”,而Dark
Beyond
Deep。\sphinxhref{https://www.dm-ai.cn/news/\%e5\%a4\%a7\%e5\%92\%96\%e4\%ba\%91\%e9\%9b\%86\%e5\%9b\%be\%e7\%81\%b5\%e5\%a4\%a7\%e4\%bc\%9a\%ef\%bc\%8c\%e6\%9a\%97\%e7\%89\%a9\%e6\%99\%ba\%e8\%83\%bd\%e5\%a4\%a7\%e6\%94\%be\%e5\%bc\%82\%e5\%bd\%a9/}{6}%
\begin{footnote}[314]\sphinxAtStartFootnote
\sphinxnolinkurl{https://www.dm-ai.cn/news/\%e5\%a4\%a7\%e5\%92\%96\%e4\%ba\%91\%e9\%9b\%86\%e5\%9b\%be\%e7\%81\%b5\%e5\%a4\%a7\%e4\%bc\%9a\%ef\%bc\%8c\%e6\%9a\%97\%e7\%89\%a9\%e6\%99\%ba\%e8\%83\%bd\%e5\%a4\%a7\%e6\%94\%be\%e5\%bc\%82\%e5\%bd\%a9/}
%
\end{footnote}
大数据背后的物体功效、因果链条、行为动机、价值决策、内心认知等人工智能领域的“暗物质”–“小数据、大任务”


\paragraph{历程}
\label{\detokenize{chapter_dive/dm-ai:id3}}
\sphinxurl{https://www.dm-ai.cn/aboutus/}

\sphinxurl{https://www.tianyancha.com/company/3205042145}


\paragraph{好评}
\label{\detokenize{chapter_dive/dm-ai:id4}}
「AI中国」机器之心2020年度榜单中凭借出色的技术创新能力与优秀的商业落地成果,成功入选“最强人工智能公司TOP30”。要求入选企业具备成熟的商业模式,主营业务在近三年保持较高增长率,并在其主要关注的细分市场领域有成熟的产品服务,且已获得该领域主导型市场地位。
\sphinxhref{https://www.dm-ai.cn/news/\%e4\%ba\%a7\%e4\%b8\%9a\%e8\%90\%bd\%e5\%9c\%b0\%e6\%88\%90\%e6\%9e\%9c\%e7\%aa\%81\%e5\%87\%ba\%ef\%bc\%81\%e6\%9a\%97\%e7\%89\%a9\%e6\%99\%ba\%e8\%83\%bd\%e5\%85\%a5\%e9\%80\%89\%e6\%9c\%80\%e5\%bc\%ba\%e4\%ba\%ba\%e5\%b7\%a5\%e6\%99\%ba\%e8\%83\%bd/}{1}%
\begin{footnote}[315]\sphinxAtStartFootnote
\sphinxnolinkurl{https://www.dm-ai.cn/news/\%e4\%ba\%a7\%e4\%b8\%9a\%e8\%90\%bd\%e5\%9c\%b0\%e6\%88\%90\%e6\%9e\%9c\%e7\%aa\%81\%e5\%87\%ba\%ef\%bc\%81\%e6\%9a\%97\%e7\%89\%a9\%e6\%99\%ba\%e8\%83\%bd\%e5\%85\%a5\%e9\%80\%89\%e6\%9c\%80\%e5\%bc\%ba\%e4\%ba\%ba\%e5\%b7\%a5\%e6\%99\%ba\%e8\%83\%bd/}
%
\end{footnote}

“未来独角兽”创新企业榜单定位于发掘具有成为“独角兽”的潜力创新企业,面向市场估值1亿美元至10亿美元之间的创新企业;“高精尖”企业榜单则定位于有核心技术创新,具备较强行业影响力的技术企业。\sphinxhref{https://www.dm-ai.cn/news/\%e6\%9c\%aa\%e6\%9d\%a5\%e5\%8f\%af\%e6\%9c\%9f\%ef\%bc\%81\%e6\%9a\%97\%e7\%89\%a9\%e6\%99\%ba\%e8\%83\%bd\%e5\%90\%8c\%e6\%97\%b6\%e5\%85\%a5\%e9\%80\%89\%e5\%b9\%bf\%e5\%b7\%9e\%e6\%9c\%aa\%e6\%9d\%a5\%e7\%8b\%ac\%e8\%a7\%92\%e5\%85\%bd/}{2}%
\begin{footnote}[316]\sphinxAtStartFootnote
\sphinxnolinkurl{https://www.dm-ai.cn/news/\%e6\%9c\%aa\%e6\%9d\%a5\%e5\%8f\%af\%e6\%9c\%9f\%ef\%bc\%81\%e6\%9a\%97\%e7\%89\%a9\%e6\%99\%ba\%e8\%83\%bd\%e5\%90\%8c\%e6\%97\%b6\%e5\%85\%a5\%e9\%80\%89\%e5\%b9\%bf\%e5\%b7\%9e\%e6\%9c\%aa\%e6\%9d\%a5\%e7\%8b\%ac\%e8\%a7\%92\%e5\%85\%bd/}
%
\end{footnote}

德勤“2019广州高科技高成长20强暨广州明日之星”、“2019中国高科技高成长50强暨中国明日之星”榜单相继揭晓,DMAI凭借在细分领域取得的领先优势和巨大的发展潜力同时荣获“广州明日之星”和“中国明日之星”
\sphinxhref{https://www.dm-ai.cn/news/\%e6\%9a\%97\%e7\%89\%a9\%e6\%99\%ba\%e8\%83\%bd\%e8\%8d\%a3\%e8\%86\%ba2019\%e5\%be\%b7\%e5\%8b\%a4\%e6\%98\%8e\%e6\%97\%a5\%e4\%b9\%8b\%e6\%98\%9f/}{10}%
\begin{footnote}[317]\sphinxAtStartFootnote
\sphinxnolinkurl{https://www.dm-ai.cn/news/\%e6\%9a\%97\%e7\%89\%a9\%e6\%99\%ba\%e8\%83\%bd\%e8\%8d\%a3\%e8\%86\%ba2019\%e5\%be\%b7\%e5\%8b\%a4\%e6\%98\%8e\%e6\%97\%a5\%e4\%b9\%8b\%e6\%98\%9f/}
%
\end{footnote}


\paragraph{融资情况}
\label{\detokenize{chapter_dive/dm-ai:id5}}
\sphinxurl{https://www.tianyancha.com/brand/b5c88428508}

2019\sphinxhyphen{}03\sphinxhyphen{}28已完成数千万美元Pre\sphinxhyphen{}A轮融资。本轮融资由赛领领投,IDG、鼎晖、高捷、将门等投资机构共同参与投资。

2021\sphinxhyphen{}01\sphinxhyphen{}28:5亿人民币 A轮


\paragraph{领导}
\label{\detokenize{chapter_dive/dm-ai:id6}}
公司的使命是“Lift Humanity with Cognitive AI
Platforms(以强认知AI平台提升人类福祉)”。

DMAI创始人朱松纯教授是全球著名计算机视觉专家、统计与应用数学家,于1996年获哈佛大学计算机博士学位,在国际顶级期刊和会议上发表论文260余篇,获得多个国际学术奖项,如三次问鼎计算机视觉领域的马尔奖、获得国际认知科学学会颁发的认知建模奖等等。
马尔奖获得者、赫尔姆霍茨奖获得者、UCLA教授、IEEE
Fellow。More:\sphinxhref{https://zhuanlan.zhihu.com/p/245049303}{16}%
\begin{footnote}[318]\sphinxAtStartFootnote
\sphinxnolinkurl{https://zhuanlan.zhihu.com/p/245049303}
%
\end{footnote}

年轻人要能沉得住气,做人做事都要能坚守信念,一辈子只做一件事,把它做好,就能有所成就。性格决定命运,要特别坚韧,Harry也说过,脸皮要厚一点,要经得住老师和同行的批评。聪明的学生尤其要能克服这个问题。

据公开资料显示,林倞曾担任商汤科技执行研发总监及研究院副院长,在人工智能领域拥有丰富的实战经验,多次领导和实施大规模高并发的AI应用项目,成功服务于亿级别终端用户。
\sphinxhref{https://www.dm-ai.cn/news/\%e5\%bc\%95\%e9\%a2\%86\%e5\%bc\%ba\%e8\%ae\%a4\%e7\%9f\%a5ai\%e8\%87\%aa\%e4\%b8\%bb\%e5\%88\%9b\%e6\%96\%b0\%ef\%bc\%8c\%e6\%9a\%97\%e7\%89\%a9\%e6\%99\%ba\%e8\%83\%bd\%e4\%ba\%ae\%e7\%9b\%b8\%e7\%ac\%ac22\%e5\%b1\%8a\%e9\%ab\%98\%e4\%ba\%a4\%e4\%bc\%9a/}{3}%
\begin{footnote}[319]\sphinxAtStartFootnote
\sphinxnolinkurl{https://www.dm-ai.cn/news/\%e5\%bc\%95\%e9\%a2\%86\%e5\%bc\%ba\%e8\%ae\%a4\%e7\%9f\%a5ai\%e8\%87\%aa\%e4\%b8\%bb\%e5\%88\%9b\%e6\%96\%b0\%ef\%bc\%8c\%e6\%9a\%97\%e7\%89\%a9\%e6\%99\%ba\%e8\%83\%bd\%e4\%ba\%ae\%e7\%9b\%b8\%e7\%ac\%ac22\%e5\%b1\%8a\%e9\%ab\%98\%e4\%ba\%a4\%e4\%bc\%9a/}
%
\end{footnote}

“我们目前正在开发AI+教育的产品,用强认知交互提升教育行业效率,提供个性化的智适应解决方案”。DMAI董事长助理、大中华区运营副总裁董乐表示,未来DMAI将逐渐进行AI+各个产业领域的迁移,辐射健康、新零售、娱乐等垂直领域。
\sphinxhref{https://www.dm-ai.cn/news/\%e6\%9a\%97\%e7\%89\%a9\%e6\%99\%ba\%e8\%83\%bddmai\%e8\%90\%bd\%e6\%88\%b7\%e5\%8d\%97\%e6\%b2\%99-\%e6\%89\%93\%e9\%80\%a0\%e6\%96\%b0\%e4\%b8\%80\%e4\%bb\%a3\%e4\%ba\%ba\%e5\%b7\%a5\%e6\%99\%ba\%e8\%83\%bd\%e4\%bc\%81\%e4\%b8\%9a/}{4}%
\begin{footnote}[320]\sphinxAtStartFootnote
\sphinxnolinkurl{https://www.dm-ai.cn/news/\%e6\%9a\%97\%e7\%89\%a9\%e6\%99\%ba\%e8\%83\%bddmai\%e8\%90\%bd\%e6\%88\%b7\%e5\%8d\%97\%e6\%b2\%99-\%e6\%89\%93\%e9\%80\%a0\%e6\%96\%b0\%e4\%b8\%80\%e4\%bb\%a3\%e4\%ba\%ba\%e5\%b7\%a5\%e6\%99\%ba\%e8\%83\%bd\%e4\%bc\%81\%e4\%b8\%9a/}
%
\end{footnote}

高级研发总监梁小丹
\sphinxhref{https://www.dm-ai.cn/news/\%e5\%a4\%a7\%e5\%92\%96\%e4\%ba\%91\%e9\%9b\%86\%e5\%9b\%be\%e7\%81\%b5\%e5\%a4\%a7\%e4\%bc\%9a\%ef\%bc\%8c\%e6\%9a\%97\%e7\%89\%a9\%e6\%99\%ba\%e8\%83\%bd\%e5\%a4\%a7\%e6\%94\%be\%e5\%bc\%82\%e5\%bd\%a9/}{5}%
\begin{footnote}[321]\sphinxAtStartFootnote
\sphinxnolinkurl{https://www.dm-ai.cn/news/\%e5\%a4\%a7\%e5\%92\%96\%e4\%ba\%91\%e9\%9b\%86\%e5\%9b\%be\%e7\%81\%b5\%e5\%a4\%a7\%e4\%bc\%9a\%ef\%bc\%8c\%e6\%9a\%97\%e7\%89\%a9\%e6\%99\%ba\%e8\%83\%bd\%e5\%a4\%a7\%e6\%94\%be\%e5\%bc\%82\%e5\%bd\%a9/}
%
\end{footnote}:“吴文俊人工智能科学技术奖”,青橙奖,被英国创新基金会评为AI领域杰出女科学家,2016年获中山大学工学博士学位,于2018年10月在卡内基梅隆大学做博士后研究员。已将研究应用于计算机视觉领域中大规模物体检测和分割、多粒度人物解析以及视觉与自然语言处理交叉领域中的人机交互自然问答和医疗诊断对话系统中。研究通过将人类常识和结构化信息结合于智能推理过程,从而使智能系统具备全局认知推理、高鲁棒性和解释性等能力,推动人工智能研究从感知层走向认知层。大三时,第一次听林倞教授的计算机视觉课,觉得特别酷炫。而后硕博期间师从林老师,潜心做计算机视觉的研究。
中国计算机学会CCF 优秀博士论文奖(全国每年仅10人)、 国际人工智能学会ACM
优秀博士论文奖(全国每年仅2人)、
国家级2011高性能计算协同创新中心优秀博士论文奖(全国每年仅10人)、
全球FashionAI\sphinxhyphen{}衣服服饰关键点检测学术竞赛 第二名

DMAI联合创始人兼首席运营官董乐表示,“DMAI将强交互、强认知的核心AI能力与教育场景做深度融合,是引领教育行业系统性变革的内生变量。公司也将汇聚政府、企业、高校等各方力量,持续革新教育行业生态圈,联合推动人工智能教育的发展。”

DMAI研究总监陈崇雨
\sphinxhref{https://www.dm-ai.cn/news/\%e5\%bc\%ba\%e8\%ae\%a4\%e7\%9f\%a5ai\%e8\%9e\%8d\%e9\%80\%9aar\%ef\%bc\%8cdmai\%e4\%b8\%8e\%e6\%96\%b0\%e8\%8a\%82\%e5\%a5\%8f\%e8\%be\%be\%e6\%88\%90\%e6\%88\%98\%e7\%95\%a5\%e5\%90\%88\%e4\%bd\%9c/}{14}%
\begin{footnote}[322]\sphinxAtStartFootnote
\sphinxnolinkurl{https://www.dm-ai.cn/news/\%e5\%bc\%ba\%e8\%ae\%a4\%e7\%9f\%a5ai\%e8\%9e\%8d\%e9\%80\%9aar\%ef\%bc\%8cdmai\%e4\%b8\%8e\%e6\%96\%b0\%e8\%8a\%82\%e5\%a5\%8f\%e8\%be\%be\%e6\%88\%90\%e6\%88\%98\%e7\%95\%a5\%e5\%90\%88\%e4\%bd\%9c/}
%
\end{footnote}

DMAI首次提出人工智能的“五层认知架构”,旨在实现机器与人类在多模态下的交流与协作,达到人机共生共存的目标。
\sphinxhref{https://www.dm-ai.cn/news/dmai\%e6\%90\%ba\%e5\%bc\%ba\%e8\%ae\%a4\%e7\%9f\%a5ai\%e4\%ba\%ae\%e7\%9b\%b82019\%e4\%b8\%96\%e7\%95\%8c\%e4\%ba\%ba\%e5\%b7\%a5\%e6\%99\%ba\%e8\%83\%bd\%e5\%a4\%a7\%e4\%bc\%9a-\%e6\%88\%90\%e4\%b8\%ba\%e4\%b8\%8a\%e6\%b5\%b7ai/}{9}%
\begin{footnote}[323]\sphinxAtStartFootnote
\sphinxnolinkurl{https://www.dm-ai.cn/news/dmai\%e6\%90\%ba\%e5\%bc\%ba\%e8\%ae\%a4\%e7\%9f\%a5ai\%e4\%ba\%ae\%e7\%9b\%b82019\%e4\%b8\%96\%e7\%95\%8c\%e4\%ba\%ba\%e5\%b7\%a5\%e6\%99\%ba\%e8\%83\%bd\%e5\%a4\%a7\%e4\%bc\%9a-\%e6\%88\%90\%e4\%b8\%ba\%e4\%b8\%8a\%e6\%b5\%b7ai/}
%
\end{footnote}


\paragraph{科研}
\label{\detokenize{chapter_dive/dm-ai:id7}}
吴田富:直接问能不能读研究生,还有很多邮件是撒传单式的,我都直接删除。仅仅提出自己思考的问题,就像我1991年申请研究生院的个人陈述;他在这个领域发表了一些颇有影响力的文章,敢啃硬骨头,比如
top\sphinxhyphen{}down/bottom\sphinxhyphen{}up 算法调度问题,
就是大家说了几十年但是没有人敢去做的事。

朱松纯教授实验室学生王文冠获得2018年ACM中国优秀博士论文奖

ACM TURC 2019 Conference Best Paper由朱松纯教授实验室学生解旭斩获

《EagleEye: Fast Sub\sphinxhyphen{}net Evaluation for Efficient Neural Network
Pruning》提出了一种性能极高的剪枝算法“鹰眼(EagleEye):暗物智能研究副总监苏江博士与高级研究员李百林领衔,联合中山大学团队共同完成。EagleEye主要包含3个模块:策略生成、通道裁剪、自适应批归一化评估模块。


\paragraph{商业落地}
\label{\detokenize{chapter_dive/dm-ai:id8}}
暗物智能在打磨核心技术平台的同时,率先探索出将强认知AI与多模态人机交互技术应用于产业实践的道路。通过深入行业场景,暗物智能已实现强认知AI技术在教育、新零售、泛娱乐等多个行业的商业落地。


\subparagraph{教育}
\label{\detokenize{chapter_dive/dm-ai:id9}}
朱松纯领衔的暗物智能创始团队一直致力于计算机视觉、机器人技术和人工智能的AOG表征和建模。目前,该方法已成功运用于“AI+教育”产品。依托AOG模型,暗物智能自主研发基于产生式模型的自动出题、解题与讲题模块。系统可以自动从教材、讲义与教案等素材中构建教学知识库,并通过归纳学习方法获取解题规则。而针对变异题型,系统将基于价值函数,在因果状态空间中自动推演出最优解题路径,所有解题路径与规则自动构建AOG表达。

基于AOG的强认知AI算法引擎,提出了一种基于语义对齐的树结构应用题求解器。此技术通过树结构来强化表达和显式约束题目文本中的语义信息,挖掘与解题有关的各类数学知识和必要的常识。有了“读题”的能力,DMAI通用求解器不仅在准确率上实现了质的提升,还在教育智能化产品中扮演“中枢”角色。从层层递进的解题演示,到精细至每一个步骤的自动批改,谙心助教作为首个融入认知AI能力的学习服务产品,背后离不开DMAI自动求解技术的支撑。

暗物智能在教育领域已形成学龄前、K12、在线教育、职业教育的用户服务全生命周期闭环,为题拍拍、腾讯作业君、豌豆思维等多家头部客户提供成熟的产品及解决方案。在美国推出的认知AI智能早教产品AILA,销量长期在亚马逊等主流平台中处于前列,好评率超过谷歌同类产品。在拓展国际市场的同时,通过打造AILA中国版本,积极推进内容生态本土化,已同国内多家知名幼教机构达成合作。

以AILA为代表的新兴早教智能产品,首次将认知人工智能技术引入家庭伴学场景。产品聚焦儿童认知潜能的开发,结合优质原创教学内容,为孩子提供真正个性化、自适应的陪伴学习体验,并联合喜马拉雅、声希AI课等合作伙伴共同打造智慧早教生态。

而DMAI打造的各类教育产品涵盖教研分析管理、陪伴自主学习、深度互动等多个关键核心环节,可以实现教学课堂全景时空解译,提供知识追踪诊断、交互练习、引导陪伴学习等强认知AI
能力,实现真正的因材施教。与单点功能的应用不同,DMAI依托具备认知和推理能力的新一代人工智能操作系统,打造构建知识和任务的内容生产平台,致力于“教、学、评、测、练”全流程的提升。

DMAI将与全通教育在人工智能教学以及人工智能公共服务等方面开展深入合作

DMAI
2019年的产品规划包括智慧课堂、知识内容生产系统、知心家教等等,到2020年,公司还将推出知心家教桌面学习终端、教育陪伴机器人解决方案等。

谙心助教率先实现的AI自动解题与讲题功能,不仅是破解“作业难”的“杀手锏”

谙心课堂融合计算机视觉、语音识别、自然语言理解等多模态AI技术,以“AI可视化教学全过程分析”为核心,对课堂氛围、教学交互、教学激励、学生发言等进行多维度分析,为线上教学场景提供全时空教学过程把控和教学质量分析。

智慧教学产品运用于华南师范大学附属南沙小学、实验小学,助力南沙打造“强认知人工智能+智慧教育”创新应用先导示范区

AILA Sit \&
Play这一融合了歌曲、游戏、声音教学法的变革型产品获得包括众多零售商在内的一致好评。一名创意产品销售商的负责人表示,很期待为所有来店的新父母们展示这种创意性的AI+教育产品。

暗物智能研发了具有高度自适应能力的桌面机器人获得“Mom’s Choice Gold
Award”等多项大奖


\subparagraph{新零售}
\label{\detokenize{chapter_dive/dm-ai:id10}}
暗物智能针对新零售领域,推出行业首个强交互智能导航机器人。此外,暗物智能还基于认知AI技术,赋能趣互联等多家智慧物联服务商,打造集合智能调度、精准推荐、智能导购等功能的新一代认知AI智慧零售解决方案。


\subparagraph{电子游戏智能NPC}
\label{\detokenize{chapter_dive/dm-ai:npc}}
利用认知AI技术平台在人机交互、可迁移性等方面的优势,暗物智能推出行业首个以认知AI驱动的电子游戏智能NPC引擎,为吉比特等知名游戏开发商打造会思考、强交互、自主协作的高质量游戏NPC大脑。


\subparagraph{公共服务 12\sphinxfootnotemark[324]}
\label{\detokenize{chapter_dive/dm-ai:id11}}%
\begin{footnotetext}[324]\sphinxAtStartFootnote
\sphinxnolinkurl{https://www.dm-ai.cn/news/\%e6\%89\%93\%e9\%80\%a0\%e4\%b8\%aa\%e6\%80\%a7\%e5\%8c\%96\%e4\%ba\%a4\%e4\%ba\%92\%e4\%bd\%93\%e9\%aa\%8c\%ef\%bc\%8c\%e5\%bc\%ba\%e8\%ae\%a4\%e7\%9f\%a5ai\%e5\%8a\%a9\%e5\%8a\%9b\%e7\%96\%ab\%e6\%83\%85\%e9\%98\%b2\%e6\%8e\%a7/}
%
\end{footnotetext}\ignorespaces 
文本深度语义匹配:以DMAI自主训练的强认知深度语义匹配模型为基础,可以准确理解不同句法结构下的语义信息,精准匹配问答答案,实现高效互动。

基于医疗知识图谱的推断问答:基于结构化知识构建的新冠肺炎医疗知识图谱是疫情助手的核心模块,也是它的“最强大脑”。如同医护人员的线下诊断一般,系统能够通过自然语言理解技术解析用户的问题,在医疗知识图谱中进行智能推理得出答案并回复用户。

意图猜测和引导:对于用户提到的系统掌握范围以外的问题,疫情通达助手也能结合语义理解分析结果,引导用户询问类似问题,并主动推荐用户可能关注的问题。

DMAI打造的AI防疫测温解决方案不仅从技术上提升检测效率,切实保障全校师生安全,也带来了更加便捷的交互体验。校园防疫涉及学校、教室、学生家庭等多方面的沟通,任何一方的信息缺失都可能导致“防线”出现漏洞。该方案与学生信息管理平台相连接,便于家长和老师实时查看学生体温数据,省心更放心。\sphinxhref{https://www.dm-ai.cn/uncategorized/\%e5\%a4\%a7ai\%e6\%97\%a0\%e7\%96\%86\%ef\%bc\%8cdmai\%e4\%b8\%ba\%e6\%8a\%97\%e7\%96\%ab\%e7\%8c\%ae\%e6\%99\%ba/}{15}%
\begin{footnote}[325]\sphinxAtStartFootnote
\sphinxnolinkurl{https://www.dm-ai.cn/uncategorized/\%e5\%a4\%a7ai\%e6\%97\%a0\%e7\%96\%86\%ef\%bc\%8cdmai\%e4\%b8\%ba\%e6\%8a\%97\%e7\%96\%ab\%e7\%8c\%ae\%e6\%99\%ba/}
%
\end{footnote}


\paragraph{DM AutoML13\sphinxfootnotemark[326]}
\label{\detokenize{chapter_dive/dm-ai:dm-automl13}}%
\begin{footnotetext}[326]\sphinxAtStartFootnote
\sphinxnolinkurl{https://www.dm-ai.cn/news/\%e4\%b8\%8d\%e9\%9c\%80\%e8\%a6\%81ai\%e4\%b8\%93\%e5\%ae\%b6\%e4\%b9\%9f\%e8\%83\%bd\%e5\%ae\%9e\%e7\%8e\%b0\%e6\%99\%ba\%e8\%83\%bd\%e5\%8c\%96\%ef\%bc\%81dmai\%e6\%8e\%a8\%e5\%87\%ba\%e7\%a7\%81\%e6\%9c\%89\%e5\%8c\%96automl\%e4\%ba\%91\%e5\%b9\%b3/}
%
\end{footnotetext}\ignorespaces \begin{itemize}
\item {} 
硬件自适应能力:基于预设的资源约束(参数规模、计算复杂度、推理延时等),能够自适应调整网络结构,自动优化AI模型的效率与性能。

\item {} 
模型智能调参能力:DM
AutoML云平台的模型通过基于因果关系推理的超参数调优,摆脱人工训练深度模型不透明、难以溯源的“炼丹”过程,以更短的时间匹配最优的超参组合。

\item {} 
模型多终端一站式部署能力:实现了深度模型从开发到部署的全自动化流程,支持caffe2、tf\sphinxhyphen{}lite等端上部署。

\item {} 
私有化平台部署能力:与多数基于云服务的AutoML提供方不同,DM
AutoML云平台直接部署在客户私有服务器上,无需依赖第三方云服务,从源头上确保业务数据无外泄,让企业安全、安心地使用AI能力。

\end{itemize}

DM
AutoML云平台不仅提升了准确率,更节省了大量计算资源和人工调参成本。以分类任务场景为例,仅需一台8GPU服务器不间断运行一周,DM
AutoML云平台在零人工模型、无人工干预的状态下,输出的模型在ImageNet验证集上准确率高达77.3\%。而同等模型规模下,作为最前沿的神经网络结构搜索算法,谷歌的EfficientNetB0在投入数千TPU的计算资源,并经过相关领域AI专家的反复调参后,准确率仅能达到76.3\%。


\paragraph{金融领域}
\label{\detokenize{chapter_dive/dm-ai:id12}}
依托核心操作系统DMOS,发挥认知推理引擎准确度高、决策可解释的独特技术优势,构筑智能化金融生态闭环。同时,运用知识图谱、规则学习、逻辑推理等强认知AI技术,融合专家经验、监管规则,打造人机共治的智能风控系统。


\paragraph{AR}
\label{\detokenize{chapter_dive/dm-ai:ar}}
DMAI将通过多模态交互AI芯片、教育多模态分析技术、自适应学习系统、早教学习机等基于强认知AI的教育智能化产品与技术,与新节奏AR体感教育系统等产品融合共享,建设“AI+AR”双重赋能的教育服务体系。

新节奏是中国最早将动作识别技术和AR技术运用在儿童游戏化教学的公司,在AR和VR内容生成与互动方面拥有领先的数字化应用案例。2019年1月,新节奏成为教育部装备中心在人工智能儿童教育装备领域的唯一合作伙伴。


\paragraph{社招}
\label{\detokenize{chapter_dive/dm-ai:id13}}
\sphinxurl{https://sc.hotjob.cn/wt/DMAI/web/index/webPositionN310!getOnePosition?postId=111201\&recruitType=2\&brandCode=1\&importPost=0\&columnId=2}



\renewcommand{\indexname}{Index}
\printindex
\end{document}