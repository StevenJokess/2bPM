%% Generated by Sphinx.
\def\sphinxdocclass{report}
\documentclass[letterpaper,10pt,english]{sphinxmanual}
\ifdefined\pdfpxdimen
   \let\sphinxpxdimen\pdfpxdimen\else\newdimen\sphinxpxdimen
\fi \sphinxpxdimen=.75bp\relax
%% turn off hyperref patch of \index as sphinx.xdy xindy module takes care of
%% suitable \hyperpage mark-up, working around hyperref-xindy incompatibility
\PassOptionsToPackage{hyperindex=false}{hyperref}

\PassOptionsToPackage{warn}{textcomp}


\usepackage{cmap}
\usepackage{fontspec}
\defaultfontfeatures[\rmfamily,\sffamily,\ttfamily]{}
\usepackage{amsmath,amssymb,amstext}
\usepackage[english]{babel}



\setmainfont{FreeSerif}[
  Extension      = .otf,
  UprightFont    = *,
  ItalicFont     = *Italic,
  BoldFont       = *Bold,
  BoldItalicFont = *BoldItalic
]
\setsansfont{FreeSans}[
  Extension      = .otf,
  UprightFont    = *,
  ItalicFont     = *Oblique,
  BoldFont       = *Bold,
  BoldItalicFont = *BoldOblique,
]
\setmonofont{FreeMono}[
  Extension      = .otf,
  UprightFont    = *,
  ItalicFont     = *Oblique,
  BoldFont       = *Bold,
  BoldItalicFont = *BoldOblique,
]


\usepackage[Sonny]{fncychap}
\usepackage[,numfigreset=2,mathnumfig]{sphinx}

\fvset{fontsize=\small}
\usepackage{geometry}


% Include hyperref last.
\usepackage{hyperref}
% Fix anchor placement for figures with captions.
\usepackage{hypcap}% it must be loaded after hyperref.
% Set up styles of URL: it should be placed after hyperref.
\urlstyle{same}


\usepackage{sphinxmessages}
\setcounter{tocdepth}{0}


\usepackage{ctex}
\setmainfont{Source Serif Pro}
\setsansfont{Source Sans Pro}
\setmonofont{Source Code Pro}
\setCJKmainfont[BoldFont=Source Han Serif SC SemiBold]{Source Han Serif SC}
\setCJKsansfont[BoldFont=Source Han Sans SC Medium]{Source Han Sans SC Normal}
\setCJKmonofont{Source Han Sans SC Normal}
\addto\captionsenglish{\renewcommand{\chaptername}{}}
\addto\captionsenglish{\renewcommand{\contentsname}{目录}}
\usepackage[draft]{minted}
\fvset{breaklines=true, breakanywhere=true}
\setlength{\headheight}{13.6pt}
\makeatletter
\fancypagestyle{normal}{
\fancyhf{}
\fancyfoot[LE,RO]{{\py@HeaderFamily\thepage}}
\fancyfoot[LO]{{\py@HeaderFamily\nouppercase{\rightmark}}}
\fancyfoot[RE]{{\py@HeaderFamily\nouppercase{\leftmark}}}
\fancyhead[LE,RO]{{\py@HeaderFamily }}
}
\makeatother
\CJKsetecglue{}
\usepackage{zhnumber}


\title{To be AI PM}
\date{Apr 17, 2021}
\release{0.0.2}
\author{The contributors}
\newcommand{\sphinxlogo}{\vbox{}}
\renewcommand{\releasename}{Release}
\makeindex
\begin{document}

\pagestyle{empty}
\sphinxmaketitle
\pagestyle{plain}
\sphinxtableofcontents
\pagestyle{normal}
\phantomsection\label{\detokenize{index::doc}}

\begin{itemize}
\item {} 
背景:传统的招聘方由于求职方是否了解AI产品经理的职位要求以及公司情况而要求简历、设立面试环节(原有解决方案),而由于简历篇幅有限只能展示部分、时间有限往往只能出几个问题,不能确保全面考察到了面试者的能力,而这时求职方往往只有为数不多的时间去应答,极度考验临场反应来对答案进行表述,相比之前的积攒的能力这种情况有很强的随机性。

\item {} 
动机:为了公司(UCD)能尽快全面了解到我个人的能力,以最快捷最有效地招募到人才(哈哈哈,王婆卖瓜),而写出这本书(此时书即预备产品,我即产品经理\sphinxhref{https://mp.weixin.qq.com/s?\_\_biz=MjM5MzE3MDQ3Mw==\&mid=2650404998\&idx=5\&sn=3717d423a70a8b1b860ad6f5a81f4f72\&chksm=be964dc089e1c4d6414995b77c3d16cbf616d1302ab86387c822c90c2b42aab8716940b4f82b\&scene=21\#wechat\_redirect}{14}%
\begin{footnote}[1]\sphinxAtStartFootnote
\sphinxnolinkurl{https://mp.weixin.qq.com/s?\_\_biz=MjM5MzE3MDQ3Mw==\&mid=2650404998\&idx=5\&sn=3717d423a70a8b1b860ad6f5a81f4f72\&chksm=be964dc089e1c4d6414995b77c3d16cbf616d1302ab86387c822c90c2b42aab8716940b4f82b\&scene=21\#wechat\_redirect}
%
\end{footnote})。

\item {} 
目标用户主要是招聘方,等待您需求的验证!同时可以给一些想一同成为AI金融产品经理的同学(不过你是我的竞品哎)提供参考。

\item {} 
同时欢迎讨论与反馈\sphinxhref{http://www.ramywu.com/work/2018/04/09/How-to-Learn-AI-PM-Tacit-Knowledge/}{6}%
\begin{footnote}[2]\sphinxAtStartFootnote
\sphinxnolinkurl{http://www.ramywu.com/work/2018/04/09/How-to-Learn-AI-PM-Tacit-Knowledge/}
%
\end{footnote}:\sphinxurl{https://github.com/StevenJokess/2bPM/issues}

\end{itemize}


\chapter{成为AI PM的路径}
\label{\detokenize{index:ai-pm}}
\begin{figure}[H]
\centering
\capstart

\noindent\sphinxincludegraphics{{2bPM}.png}
\caption{成为AI PM}\label{\detokenize{index:id11}}\end{figure}


\chapter{产品经理}
\label{\detokenize{index:id1}}
产品经理是一个技能积累要求高,需要广度的职业,
你不能不了解产品设计,心理学,经济学,技术趋势,财务知识等等。其他职能的技能经验积累会在一个明确的方向内,积累一级的经验可以升一级,产品则需要十个方向十级经验量才能升一级。
\sphinxhref{https://zhuanlan.zhihu.com/p/30984881}{3}%
\begin{footnote}[3]\sphinxAtStartFootnote
\sphinxnolinkurl{https://zhuanlan.zhihu.com/p/30984881}
%
\end{footnote}

当下时代,行业和市场的变化异常迅速,每一个行业的模式每个月都在进化,旧模式将无法适用。
假设你今天在金融,或在广告,或在电商,你以为自己以后可以以多年金融产品经验、多年广告产品经验、多年电商经验,可以开始吃老本,工作可以轻松而游刃有余,还是醒醒,不现实,马上被淘汰。

产品经理是一个停不下来的职业,不论是一直做一个类型的业务,或者不断变换业务,当对手在进化的时候,当行业趋势在改变的时候,你就有一堆东西要从头学起了。

真的一点安全感都没有吗,那工作中的经历会不会留下什么?会。

你要给自己搭建一个书库,把每一天,每一份工作遇到的东西,变成你的藏品放在你的书架上。这个书库分门别类,便于你日后查找,你不论遇到什么问题,都能查到资料来应对,反而还会觉得每次做不同的东西特别有乐趣,好挑战。这个书库叫做“知识体系”。

\sphinxstylestrong{有了知识体系,有了方法论,才能有资格、有机会去创造那些还未出现的产品形态}


\chapter{AI产品经理专业体系}
\label{\detokenize{index:ai}}
没有实际经验,是断无可能获得产品/实验方面的知识的。这种观点,我坚决不认同。虽然我之前并没有任何产品或
A/B
测试的经验,但我相信,这些技能是可以通过阅读、倾听、思考和总结来获得。毕竟,这与我们在学校里接受教育的方式也一样不是?\sphinxhref{https://www.infoq.cn/article/IPDVRNxwJVsx3ZGrgwzW}{9}%
\begin{footnote}[4]\sphinxAtStartFootnote
\sphinxnolinkurl{https://www.infoq.cn/article/IPDVRNxwJVsx3ZGrgwzW}
%
\end{footnote}

将网络上杂乱的内容进行有效整理(互联网上充斥着各种资源,主动降噪(知识卸载和去重\sphinxhref{http://www.woshipm.com/zhichang/4156203.html}{10}%
\begin{footnote}[5]\sphinxAtStartFootnote
\sphinxnolinkurl{http://www.woshipm.com/zhichang/4156203.html}
%
\end{footnote})来过滤出最优质的内容以飨自己。\sphinxhref{https://www.zhihu.com/question/29342383/answer/44323650}{7}%
\begin{footnote}[6]\sphinxAtStartFootnote
\sphinxnolinkurl{https://www.zhihu.com/question/29342383/answer/44323650}
%
\end{footnote}),用逻辑(从框架到细节、前后顺序)搭建自己的AI产品经理专业体系
\sphinxhref{http://www.woshipm.com/zhichang/3945751.html}{1}%
\begin{footnote}[7]\sphinxAtStartFootnote
\sphinxnolinkurl{http://www.woshipm.com/zhichang/3945751.html}
%
\end{footnote}

好处:
\begin{itemize}
\item {} 
有目的方向的去扩充我的专业库,积累我的经验。

\item {} 
非常多的信息总结和跟踪,提高我二次工作的效率性。

\item {} 
使我的积累形成系统化,随时能快速找出信息。

\end{itemize}

框架的思考方式避免我们的遗漏,有益于创造一个无错的产品,但是“魔鬼在于细节”,对于细节的\sphinxstylestrong{洞察}或许更能打动用户。\sphinxhref{https://zhuanlan.zhihu.com/p/34673277}{8}%
\begin{footnote}[8]\sphinxAtStartFootnote
\sphinxnolinkurl{https://zhuanlan.zhihu.com/p/34673277}
%
\end{footnote}


\section{信息与知识}
\label{\detokenize{index:id2}}\begin{itemize}
\item {} 
什么叫信息,能够消除不确定性的数据。

\item {} 
当你把能消除某类问题的不确定性的信息,整合成体系,才是知识\sphinxhref{https://www.zhihu.com/question/443911275}{11}%
\begin{footnote}[9]\sphinxAtStartFootnote
\sphinxnolinkurl{https://www.zhihu.com/question/443911275}
%
\end{footnote}

\end{itemize}


\section{「长期主义」}
\label{\detokenize{index:id3}}
「长期主义」价值观:用长远的目光分析事物的发展走向,什么能产生更广泛更长期的影响?有且只有一条标准,那就是是否在创造真正的价值,这个价值是否有益于社会的整体繁荣。
\sphinxhref{https://zhuanlan.zhihu.com/p/340824405}{17}%
\begin{footnote}[10]\sphinxAtStartFootnote
\sphinxnolinkurl{https://zhuanlan.zhihu.com/p/340824405}
%
\end{footnote}

当下流行的碎片化学习要坚持「长期主义」,在建立个人系统化的知识网络的基础上,有规划地持续不断地学习,进而提高学习质效,满足其个性化的知识需求,完成个人知识管理

从碎片化的时间里学习到知识 \sphinxhyphen{}>
知识体系\sphinxhref{https://www.zhihu.com/market/paid\_magazine/1153593677761941504/section/1153593764286263296}{16}%
\begin{footnote}[11]\sphinxAtStartFootnote
\sphinxnolinkurl{https://www.zhihu.com/market/paid\_magazine/1153593677761941504/section/1153593764286263296}
%
\end{footnote}


\section{知识体系}
\label{\detokenize{index:id4}}
知识体系有三个特性:目标性、体系性以及抽象性,多使用逻辑树进行构建。\sphinxhref{http://www.woshipm.com/zhichang/4156203.html}{10}%
\begin{footnote}[12]\sphinxAtStartFootnote
\sphinxnolinkurl{http://www.woshipm.com/zhichang/4156203.html}
%
\end{footnote}
\begin{itemize}
\item {} 
目标是指方向聚焦,体系有重点。体系是指结构完整、层次分明,前2层分支的平衡性越好,归纳程度也越强。

\item {} 
而抽象则是知识体系是知识从具体到抽象的表现,呈现了知识的特征或本质。

\item {} 
再进一层,什么是好的知识体系呢?除了上述所描述的有目标、够全面、抽象程度高,还有一点则是知识体系应该是相互独立但非无限穷尽的。

\end{itemize}

有的知识仅须停留在知道,有的则需要运用并且创新。

把你已经知道的东西梳理一遍。如何梳理呢?以你能够说出某个知识点的影响因素,以及它对其他事物的影响为准。顺着这样的知识点捋一遍,这个网络就是你已经构建完成的知识网络\sphinxhref{https://www.zhihu.com/market/paid\_magazine/1153593677761941504/section/1153593764286263296}{16}%
\begin{footnote}[13]\sphinxAtStartFootnote
\sphinxnolinkurl{https://www.zhihu.com/market/paid\_magazine/1153593677761941504/section/1153593764286263296}
%
\end{footnote}。


\section{用技能分工代替职业分工}
\label{\detokenize{index:id5}}
要每一个项目都组建一支产品、交互、设计、前端、后台、运营的团队肯定是不现实。我的想法就是团队的每个人经可能多的学习技能,走全栈型路线。在王者里面每个英雄\sphinxstylestrong{可以带三个技能},如果有人能带五个技能,基本上可以一个顶两个用了。产品、交互还有运营基本上可以\sphinxstylestrong{合并为一个人}来统筹负责,技术部分\sphinxstylestrong{用跨平台的解决方案简化为一到两个人来负责}。Full
Stack后眼界和见识提高\sphinxhref{https://www.zhihu.com/question/22613861/answer/34122925}{12}%
\begin{footnote}[14]\sphinxAtStartFootnote
\sphinxnolinkurl{https://www.zhihu.com/question/22613861/answer/34122925}
%
\end{footnote},这样\sphinxstylestrong{大幅简化了人员的沟通成本},缩减了项目切换过程中所需要的冷却值。而且从市场的反馈到产品的迭代更新间的\sphinxstylestrong{反射弧更加的快速}。
\sphinxhref{http://dyin.tech/}{2}%
\begin{footnote}[15]\sphinxAtStartFootnote
\sphinxnolinkurl{http://dyin.tech/}
%
\end{footnote}


\chapter{全栈产品经理}
\label{\detokenize{index:id6}}
全栈产品经理必须完全理解/了解自己业务系统,从底层基础架构,到数据中间件,到API逻辑,到
Web 和 Mobile 实现,到 UX 和
UI,到产品逻辑,到部署和发布计划的制定。\sphinxhref{https://www.zhihu.com/question/22613861/answer/118658853}{13}%
\begin{footnote}[16]\sphinxAtStartFootnote
\sphinxnolinkurl{https://www.zhihu.com/question/22613861/answer/118658853}
%
\end{footnote}

\begin{figure}[H]
\centering
\capstart

\noindent\sphinxincludegraphics{{full_stack_PM_skill_tree}.png}
\caption{“全栈产品经理”的技能树}\label{\detokenize{index:id12}}\end{figure}


\chapter{减人加速}
\label{\detokenize{index:id7}}
须知人数越少,沟通成本越低,执行效率越高。就像我对自己连续推进6个产品项目(2个主场1个客场3个玩票)的能力很是自负,但前提得是我兼任PM/交互/QA,并且主持搭建内容框架。如果给我再配一两个交互设计师,一两个QA,内容框架也交出去,再和他们反复沟通确认,我的速度反而被拉了下来。这不仅苛求个人能力,也要求你在大包大揽时,还能接受其中的粗活笨活。如果把我手上各种杂活剥离出去,起码能设3个岗位——并且速度拖慢50\%。
技术研发也是如此,减人才能提速。又要减人又要代码能力,大牛就得分担杂活。\sphinxhref{https://zhuanlan.zhihu.com/p/291532809}{15}%
\begin{footnote}[17]\sphinxAtStartFootnote
\sphinxnolinkurl{https://zhuanlan.zhihu.com/p/291532809}
%
\end{footnote}


\chapter{AI商业化逻辑}
\label{\detokenize{index:id8}}
2017年AI人才结构失衡。特别是发论文的人和做产品或解决方案研发的人,比例严重倒挂。
\sphinxhref{https://www.zhihu.com/question/279550559}{4}%
\begin{footnote}[18]\sphinxAtStartFootnote
\sphinxnolinkurl{https://www.zhihu.com/question/279550559}
%
\end{footnote}

当时最好的AI技术人员更愿意在大学里搞学术,或在企业研究院里发论文,或在创业公司拿着高薪担任首席科学家之类的职位,但AI商业化落地亟需的AI应用开发人员、AI架构设计人员、AI产品和解决方案设计人员等,人才市场上几乎是空白。规模巨大的计算机或相关专业毕业生、有经验的前后端工程师、架构设计师、产品经理的人群中,当时懂得AI特别是懂得AI商业化逻辑的人,真是少之又少。

“我们认为,AI商业化的最佳途径对百度来说,就是平台化、生态化。一个开放的生态最具活力最有竞争力。”陆奇说道。\sphinxhref{https://ai.baidu.com/forum/topic/show/492818}{5}%
\begin{footnote}[19]\sphinxAtStartFootnote
\sphinxnolinkurl{https://ai.baidu.com/forum/topic/show/492818}
%
\end{footnote}


\chapter{功能点:}
\label{\detokenize{index:id9}}

\section{常用}
\label{\detokenize{index:id10}}
标签: :lable:、:ref:


\section{TODO:}
\label{\detokenize{index:todo}}
google analysis.


\subsection{求职AI PM,百度用了我战略idea?}
\label{\detokenize{get_started:ai-pm-idea}}\label{\detokenize{get_started::doc}}
AI产品经理是直接应用或间接涉及了AI技术,进而完成相关AI产品的设计、研发、推广、产品生命周期管理等工作的产品经理。由于AI的技术领域太多、且有更多和垂直行业结合的机会,导致细分AI领域的产品经理所需要的背景和能力可能大不相同。\sphinxhref{https://www.boxuegu.com/news/4368.html}{1}%
\begin{footnote}[20]\sphinxAtStartFootnote
\sphinxnolinkurl{https://www.boxuegu.com/news/4368.html}
%
\end{footnote}

\begin{figure}[H]
\centering
\capstart

\noindent\sphinxincludegraphics{{resume_nophone}.jpg}
\caption{简历}\label{\detokenize{get_started:id13}}\end{figure}


\subsubsection{点击下载整页简历PDF(带URL)\sphinxfootnotemark[21]}
\label{\detokenize{get_started:pdf-url}}%
\begin{footnotetext}[21]\sphinxAtStartFootnote
\sphinxnolinkurl{https://github.com/StevenJokess/2bPM/blob/master/蔡舒起-AI产品经理-GAN(MXNet-PyTorchTF2开发者)\_nophone.pdf}
%
\end{footnotetext}\ignorespaces 
领英(交个朋友呗):\sphinxurl{https://www.linkedin.com/in/\%E8\%88\%92\%E8\%B5\%B7-\%E8\%94\%A1-b609001b7/}




\bigskip\hrule\bigskip


细节:

\sphinxstylestrong{一面百度AIstudio产品经理失败}后的总结:\sphinxurl{https://github.com/StevenJokess/d2l-en-read/blob/moreme/chapter-generative-adversarial-networks/aistudio-job.md}

\begin{figure}[H]
\centering
\capstart

\noindent\sphinxincludegraphics{{baidu_kaifa}.png}
\caption{baidu开发者版}\label{\detokenize{get_started:id14}}\end{figure}

\begin{figure}[H]
\centering
\capstart

\noindent\sphinxincludegraphics{{idea_time}.png}
\caption{git时间}\label{\detokenize{get_started:id15}}\end{figure}

可以看到2020年10月3日里面就有最近\sphinxstylestrong{才beta测试}的
\sphinxurl{https://kaifa.baidu.com} 的主意!

据我能找到的最早时间:

\begin{figure}[H]
\centering
\capstart

\noindent\sphinxincludegraphics{{kaifa_online}.png}
\caption{能找到的最早时间}\label{\detokenize{get_started:id16}}\end{figure}

百度股价预测:

\begin{center}\sphinxincludegraphics{{baidu_gujia}.jpg}\end{center} :depth:300px

百度最新股价:

\begin{center}\sphinxincludegraphics{{baidu_gujia_newest}.png}\end{center} :depth:200px


\bigskip\hrule\bigskip


\sphinxurl{https://www.overleaf.com/project/603dfbba8126ff225dc18564}


\subsubsection{\sphinxstylestrong{简历}}
\label{\detokenize{get_started:id1}}

\paragraph{自我评价:}
\label{\detokenize{get_started:id2}}\begin{itemize}
\item {} 
健身跳绳、计算机AI、金融的基础

\item {} 
《动手学深度学习》GAN、DCGAN从MXNet到PyTorch的开发

\item {} 
PyTorch Android demo等四种移动深度学习框架复现

\item {} 
Docker部署的d2lbook2写 《To be AI PM》

\item {} 
证书:省跳绳铜牌、初级教练、裁判证;会计、证券、基金从业;六级、二甲

\item {} 
快速学习:与时俱进、千书阅读

\end{itemize}


\paragraph{工作经历}
\label{\detokenize{get_started:id3}}
企业名称: 个人求职中

开始时间: 2020\sphinxhyphen{}12\sphinxhyphen{}01

结束时间: 2021\sphinxhyphen{}02\sphinxhyphen{}01

职位名称: 学习并撰写AI产品经理相关内容

所在部门: 个人求职中

离职原因: 对AI更热爱。感觉AI能服务更多人。

工作描述:
\begin{itemize}
\item {} 
运用docker的 \sphinxurl{https://github.com/aieye-top/d2l-book2} 包,来:

\item {} 
撰写普惠深度学习(WIP):\sphinxurl{https://github.com/aieye-top/d2cl}

\item {} 
撰写人工智能产品经理相关书(WIP):\sphinxurl{https://stevenjokess.github.io/2bPM/}

\end{itemize}


\bigskip\hrule\bigskip



\paragraph{项目经验}
\label{\detokenize{get_started:id4}}
项目名称: 移动深度学习开发、测试 开始时间: 2020\sphinxhyphen{}08\sphinxhyphen{}01 结束时间:
2020\sphinxhyphen{}09\sphinxhyphen{}01 项目描述:
\begin{itemize}
\item {} 
观看pytorch官方文档和视频,了解了基本的andriod开发体系;

\item {} 
动手完成了针对动物的图片识别项目:\sphinxurl{https://github.com/StevenJokess/pytorch-andriod-greatdemo};

\item {} 
并通过https://stevenjokess.github.io/post/pytorch\sphinxhyphen{}android/来分享经历

\item {} 
项目职责: 同时了解了arm体系和测试了其他框架:

\item {} 
\sphinxurl{https://github.com/StevenJokess/Pytorch-Kotlin-Demo}

\item {} 
\sphinxurl{https://github.com/StevenJokess/djl-android-demo}

\item {} 
\sphinxurl{https://github.com/StevenJokess/paddlelite-andriod-demo}

\end{itemize}


\bigskip\hrule\bigskip


项目名称: 动手学深度学习GAN开发者

开始时间: 2020\sphinxhyphen{}06\sphinxhyphen{}01

结束时间: 2020\sphinxhyphen{}11\sphinxhyphen{}01

项目描述:
\begin{itemize}
\item {} 
开设d2l\sphinxhyphen{}en\sphinxhyphen{}read记录自己所有的学习过程.见https://github.com/StevenJokess/d2l\sphinxhyphen{}en\sphinxhyphen{}read/tree/moreme

\item {} 
积极参与discuss.d2l.ai,记录自己遇到的坑,被李沐(MXNet开发者)评为最活跃的参与者.

\item {} 
和mxnet的开发者表达对社区的死气沉沉的不满,并提出活跃社区建议,后被采纳开设discussion区.(\sphinxurl{https://github.com/apache/incubator-mxnet/issues/18931})

\item {} 
项目职责: 运用谷歌、stack
overflow等编程搜索引擎,并积极参与GitHub讨论,完成GAN、DCGAN从MXNet到PyTorch的翻译

\item {} 
PR.项目可参见(点开pytorch标签的最后的“continue discussion”可见)

\item {} 
GAN:http://d2l.ai/chapter\_generative\sphinxhyphen{}adversarial\sphinxhyphen{}networks/gan.html

\item {} 
DCGAN:http://d2l.ai/chapter\_generative\sphinxhyphen{}adversarial\sphinxhyphen{}networks/dcgan.html

\end{itemize}


\bigskip\hrule\bigskip


项目名称: 完成学位论文

开始时间: 2020\sphinxhyphen{}02\sphinxhyphen{}01

结束时间: 2020\sphinxhyphen{}05\sphinxhyphen{}01

项目描述: 独立研究者 repo: \sphinxurl{https://github.com/StevenJokess/gra\_paper}
\begin{itemize}
\item {} 
运用知网、Google学术、SciHub等学术搜索引擎,完成文献综述和翻译.

\item {} 
由于导师没接触过Python,我独立阅读Python文档、十余本相关书籍.

\item {} 
项目职责: 运用Pandas库的DataReader()、datetime()导入股市数据.

\item {} 
to\_excel()导出,后Excel处理缺失数据与整合文件;read\_excel()读取,plt、seaborn库生成时间序列图.

\item {} 
Statsmodel库的极大似然估计下fit()出VAR模型,as\_csv()来保存结果.

\item {} 
Word完成编写、排版,共13656字的《中美贸易摩擦前后中美股市的联动性分析》

\end{itemize}

项目名称: 参加山西省跳绳竞标赛

开始时间: 2018\sphinxhyphen{}07\sphinxhyphen{}01

结束时间: 2018\sphinxhyphen{}08\sphinxhyphen{}01

项目描述:
\begin{itemize}
\item {} 
30s单摇:66;30s双摇:60;三摇:11个

\item {} 
毕业前还可单手俯卧撑、单腿深蹲(现在学AI学肥了。。)

\end{itemize}

项目职责:
\begin{itemize}
\item {} 
偶然在操场练习双摇被相中参加比赛。

\item {} 
作为非体院唯一绳没有的第四棒,在4*30男子团体单摇比赛共250个,取得市和省级铜牌

\item {} 
更多见 \sphinxurl{https://www.bilibili.com/video/BV1Wf4y167Kp?pop\_share=1}
的第四棒。

\item {} 
社团成员文案抓住大家减肥痛点、展示速摇,招新成功翻4倍。

\end{itemize}


\paragraph{培训经历}
\label{\detokenize{get_started:id5}}
开始时间: 2018\sphinxhyphen{}05\sphinxhyphen{}01 结束时间: 2018\sphinxhyphen{}06\sphinxhyphen{}01 培训机构: 山西跳绳运动协会
培训地点: 山西 培训课程: 跳绳初级裁判、初级教练 获得证书:
跳绳初级裁判证、初级教练证


\paragraph{语言能力}
\label{\detokenize{get_started:id6}}\begin{itemize}
\item {} 
语种: 英语

\item {} 
听说能力: 良好

\item {} 
读写能力: 精通

\item {} 
语言等级: 英语\sphinxhyphen{}英语六级

\end{itemize}


\paragraph{计算机技能}
\label{\detokenize{get_started:id7}}\begin{itemize}
\item {} 
技能类别: Anaconda 掌握程度: 良好

\item {} 
技能类别: VScode 掌握程度: 良好

\item {} 
技能类别: Python 掌握程度: 良好

\item {} 
技能类别: markdown 掌握程度: 良好

\item {} 
技能类别: pytorch 掌握程度: 良好

\item {} 
技能类别: mxnet 掌握程度: 良好

\item {} 
技能类别: Linux 掌握程度: 良好

\item {} 
技能类别: Android开发 掌握程度: 普通

\end{itemize}


\paragraph{专业技能}
\label{\detokenize{get_started:id8}}
技能名称: 跳绳 掌握程度: 精通


\paragraph{证书}
\label{\detokenize{get_started:id9}}\begin{itemize}
\item {} 
证书名称: 会计从业资格证 说明: 大一上获得

\item {} 
证书名称: 跳绳初级教练证 说明: 大二下获得

\item {} 
证书名称: 跳绳初级裁判证 说明: 大二上获得

\item {} 
证书名称: 证券从业资格证 说明: 大一下获得

\item {} 
证书名称: 基金从业资格证 说明: 大三下获得

\item {} 
证书名称: 普通话二甲证书 说明: 大四上获得

\end{itemize}


\subsubsection{个人基本信息}
\label{\detokenize{get_started:id10}}
接受调剂: 不接受

姓名: 蔡舒起

性别: 男

出生日期: 1998\sphinxhyphen{}08\sphinxhyphen{}11

国籍/地区: 中国

民族: 汉族

婚姻状况: 未婚

工作年限: 无经验

政治面貌: 共青团员

证件类型: 身份证

证件号码: ?

现居住地: 浙江省\sphinxhyphen{}台州市

籍贯: 浙江省\sphinxhyphen{}台州市

学历: 本科

毕业时间: 2020\sphinxhyphen{}07\sphinxhyphen{}01

学位: 学士

毕业院校: 山西大学

专业: 经济学类\sphinxhyphen{}金融学

移动电话: 1840xxxxxxx

电子邮箱: \sphinxhref{mailto:llgg8679@qq.com}{llgg8679@qq.com}


\paragraph{求职意向}
\label{\detokenize{get_started:id11}}
期望工作性质: 全职

期望行业: 互联网/电子商务/AI金融/AI健身

目前薪酬: 面议

期望薪酬: 面议

期望年薪: 面议

到岗时间: 随时


\paragraph{教育经历}
\label{\detokenize{get_started:id12}}
学校: 山西大学

开始时间: 2016\sphinxhyphen{}09\sphinxhyphen{}01

结束时间: 2020\sphinxhyphen{}07\sphinxhyphen{}01

学历: 本科

学位: 学士

专业: 经济学类\sphinxhyphen{}金融学

专业描述: 经济与管理学院 太原

荣誉/奖项:学业奖学金(2017);三好学生(2017)

相关课程:
\begin{itemize}
\item {} 
数学分析(95);高等代数(89);概率论与数理统计(85);大学英语(90)

\item {} 
计算机基础–PS(90);计算机高级语言–C语言(100);网络金融(80);

\item {} 
微观经济学(85);宏观经济学(90);计量经济学(82);投资学(82补考);金融计量学(85);

\item {} 
会计循环实验(91);计量经济学实验(90);证券投资模拟交易(89);EXCEL计算实验(86);商业银行综合业务

\item {} 
实验(87);投资组合管理(81);财务报表分析(80);

\item {} 
金融服务营销(93);金融从业综合素质实训(92);毕业实习(88);

\end{itemize}

\begin{figure}[H]
\centering
\capstart

\noindent\sphinxincludegraphics{{rope}.png}
\caption{跳绳证书、六级}\label{\detokenize{get_started:id17}}\end{figure}

书单分享: \sphinxurl{https://weread.qq.com/misc/booklist/358906697\_7e9fYZVah}


\subsection{入门}
\label{\detokenize{chapter_introduction/index:chap-intro}}\label{\detokenize{chapter_introduction/index:id1}}\label{\detokenize{chapter_introduction/index::doc}}

\subsubsection{需求}
\label{\detokenize{chapter_introduction/need:need}}\label{\detokenize{chapter_introduction/need:id1}}\label{\detokenize{chapter_introduction/need::doc}}
人类为了维持生活机能或健全心灵的基础满足条件,更多见\sphinxhref{https://zh.wikipedia.org/wiki/\%E9\%9C\%80\%E6\%B1\%82\%E5\%B1\%82\%E6\%AC\%A1\%E7\%90\%86\%E8\%AE\%BA}{需求层次理论}%
\begin{footnote}[22]\sphinxAtStartFootnote
\sphinxnolinkurl{https://zh.wikipedia.org/wiki/\%E9\%9C\%80\%E6\%B1\%82\%E5\%B1\%82\%E6\%AC\%A1\%E7\%90\%86\%E8\%AE\%BA}
%
\end{footnote}

需求是产品产生的前提,如果没有需求,产品也就不会存在,从而产品经理这个岗位也就不会存在了。
\sphinxhref{https://www.zhihu.com/pub/reader/119980992/chapter/1284104614460440576}{9}%
\begin{footnote}[23]\sphinxAtStartFootnote
\sphinxnolinkurl{https://www.zhihu.com/pub/reader/119980992/chapter/1284104614460440576}
%
\end{footnote}

需求总是含混的,做需求相关的工作,就是不断地降低含混。——需求蛋模型
《探索需求》
\sphinxhref{https://www.yinxiang.com/everhub/note/f9ab87ee-73e6-4241-9428-9507cbfd007f}{7}%
\begin{footnote}[24]\sphinxAtStartFootnote
\sphinxnolinkurl{https://www.yinxiang.com/everhub/note/f9ab87ee-73e6-4241-9428-9507cbfd007f}
%
\end{footnote}

需求决定产品核心价值:解决的需求越重要,你的产品核心价值就越大。


\paragraph{别做「传话筒」}
\label{\detokenize{chapter_introduction/need:id2}}
产品经理作为需求「收纳箱」,要接收来自四面八方的需求,然后将需求串起来并进行有效的规划后,形成产品需求。可是很多时候,产品经理只是被动地接收需求,对需求不进行任何的分析和转换,直接就放到了产品规划清单里,等待着开发上线,似乎一切就这样完成了。
\sphinxhref{https://www.zhihu.com/pub/reader/119980992/chapter/1284104607329615872}{8}%
\begin{footnote}[25]\sphinxAtStartFootnote
\sphinxnolinkurl{https://www.zhihu.com/pub/reader/119980992/chapter/1284104607329615872}
%
\end{footnote}

如果只是简单传个话,公司直接用一支录音笔是否更合适呢?而且录音笔的保真率至少能够达到
95\%,而人的保真率可能连 80\% 都达不到。

最糟情况,东西做出来了,对方不认可,而怪罪:
\begin{itemize}
\item {} 
需求方提需求没说清楚(手机端搞成PC)

\item {} 
研发部实现不了(无第三方稳定接口、自研耗时长)

\item {} 
老板定的(先抓紧上线PC端,再慢慢研究手机端)

\end{itemize}


\paragraph{产品需求举例 13\sphinxfootnotemark[26]}
\label{\detokenize{chapter_introduction/need:id3}}%
\begin{footnotetext}[26]\sphinxAtStartFootnote
\sphinxnolinkurl{http://www.shuahuangpu.com/articles/110937.html}
%
\end{footnotetext}\ignorespaces \begin{itemize}
\item {} 
滴滴解决了用户想要便捷出行的基本出行需求

\item {} 
抖音解决了新时代下用户的娱乐需求,表达展示自己的需求

\item {} 
美团外卖解决了用户懒和吃的需求,用户无需出门就能吃到想吃的东西

\end{itemize}


\paragraph{需求or伪需求}
\label{\detokenize{chapter_introduction/need:or}}
“客户不是想买一个1英寸的电钻,而是想要一个1英寸的钻孔!”——市场营销学教授
西奥多·莱维特
\begin{enumerate}
\sphinxsetlistlabels{\arabic}{enumi}{enumii}{}{.}%
\item {} 
是否是围绕主路径;迅雷:搜电影\sphinxhyphen{}下电影\sphinxhyphen{}看电影,“边看边播”、影评和打分\sphinxhref{http://www.woshipm.com/pmd/2903334.html}{2}%
\begin{footnote}[27]\sphinxAtStartFootnote
\sphinxnolinkurl{http://www.woshipm.com/pmd/2903334.html}
%
\end{footnote}

\item {} 
该需求的使用频率;必须高频

\end{enumerate}


\paragraph{需求来源}
\label{\detokenize{chapter_introduction/need:id4}}
商业\sphinxhyphen{}>市场\sphinxhyphen{}>用户\sphinxhyphen{}>需求\sphinxhref{https://zhuanlan.zhihu.com/p/25965712}{14}%
\begin{footnote}[28]\sphinxAtStartFootnote
\sphinxnolinkurl{https://zhuanlan.zhihu.com/p/25965712}
%
\end{footnote}


\paragraph{需求职责}
\label{\detokenize{chapter_introduction/need:id5}}
需求管理、需求定义、需求确认、需求跟踪等与需求相关的职责都是公司对产品经理最基本的要求。原因是产品经理是对公司产品的负责人,而产品是为用户解决某种特定需求的,因此即使我们来到了人工智能时代,产品依然是围绕用户需求定义的,这个本质是不变的。\sphinxhref{https://zhuanlan.zhihu.com/p/36871139}{12}%
\begin{footnote}[29]\sphinxAtStartFootnote
\sphinxnolinkurl{https://zhuanlan.zhihu.com/p/36871139}
%
\end{footnote}


\paragraph{需求采集}
\label{\detokenize{chapter_introduction/need:id6}}
直接采集与间接采集,获取到的需求分别是一手需求与二手需求。

可以从两个角度来理解它们的差异:需求的提出者是不是有需求的人、需求是原始的还是加工过的。比喻:生孩子与养孩子;

直接从用户处采集的一手需求更准确,所以产品经理一定要确保手里有足够比例的需求是直接采集的,这样才能让产品本身和自己对产品的判断更接地气。

而从销售、营销、服务人员\sphinxhref{https://quizlet.com/129588206/\%E4\%BA\%BA\%E4\%BA\%BA\%E9\%83\%BD\%E6\%98\%AF\%E4\%BA\%A7\%E5\%93\%81\%E7\%BB\%8F\%E7\%90\%86-\%E7\%AC\%94\%E8\%AE\%B0-flash-cards/}{6}%
\begin{footnote}[30]\sphinxAtStartFootnote
\sphinxnolinkurl{https://quizlet.com/129588206/\%E4\%BA\%BA\%E4\%BA\%BA\%E9\%83\%BD\%E6\%98\%AF\%E4\%BA\%A7\%E5\%93\%81\%E7\%BB\%8F\%E7\%90\%86-\%E7\%AC\%94\%E8\%AE\%B0-flash-cards/}
%
\end{footnote}等间接采集的二手需求,有可能被扭曲,就需要带着“问号”来看,思考原始需求方和转述者分别是谁,以及它有没有被曲解过。但二手需求(比如一份客户反馈周报)可以通过更多的人,收集到更多的用户声音,而且是经过梳理的,所以获取信息的效率更高。

扩展到实践层面,团队内“全员参与采集,产品人员处理”是比较可行的模式,是一种效率和准确度的兼顾方案。


\subparagraph{用户需求vs产品需求}
\label{\detokenize{chapter_introduction/need:vs}}\begin{itemize}
\item {} 
用户需求:用户自以为的需求,并且经常表达为用户的解决方案;

\item {} 
产品需求:经过我们分析,找到的真实需求,并且表达为产品的解决方案;

\item {} 
需求分析:从用户需求出发,找到用户内心真正的渴望,再转化为产品需求的过程;

\end{itemize}

用户需求\sphinxhyphen{}(需求分析)–》产品需求


\subparagraph{直接采集的途径 3\sphinxfootnotemark[31]}
\label{\detokenize{chapter_introduction/need:id7}}%
\begin{footnotetext}[31]\sphinxAtStartFootnote
\sphinxnolinkurl{http://www.woshipm.com/zhichang/459131.html}
%
\end{footnotetext}\ignorespaces \begin{itemize}
\item {} 
用户访谈提出需求(问题集中体现在视觉、交互或一些漏洞等体验侧层面)

\item {} 
其他参与者和关注者反馈的需求

\end{itemize}


\subparagraph{间接采集的途径 3\sphinxfootnotemark[32]}
\label{\detokenize{chapter_introduction/need:id8}}%
\begin{footnotetext}[32]\sphinxAtStartFootnote
\sphinxnolinkurl{http://www.woshipm.com/zhichang/459131.html}
%
\end{footnotetext}\ignorespaces \begin{itemize}
\item {} 
老板提出的战略性需求(往往体验侧的,给老板时间去深度思考)

\item {} 
产品经理根据产品方向规划需求

\item {} 
推广规划的活动和数据分析出来一些需求(反馈能够产生实际收益,更关注业务的优化空间)

\end{itemize}


\paragraph{满足需求 6\sphinxfootnotemark[33]}
\label{\detokenize{chapter_introduction/need:id9}}%
\begin{footnotetext}[33]\sphinxAtStartFootnote
\sphinxnolinkurl{https://quizlet.com/129588206/\%E4\%BA\%BA\%E4\%BA\%BA\%E9\%83\%BD\%E6\%98\%AF\%E4\%BA\%A7\%E5\%93\%81\%E7\%BB\%8F\%E7\%90\%86-\%E7\%AC\%94\%E8\%AE\%B0-flash-cards/}
%
\end{footnotetext}\ignorespaces 
由于理想与现实的差距产生需求,所以满足需求有3种方式
\begin{enumerate}
\sphinxsetlistlabels{\arabic}{enumi}{enumii}{}{.}%
\item {} 
改变现状

\item {} 
降低理想

\item {} 
转移需求

\end{enumerate}


\subparagraph{降低理想}
\label{\detokenize{chapter_introduction/need:id10}}

\subparagraph{商业化与用户体验冲突}
\label{\detokenize{chapter_introduction/need:id11}}
对于中小企业的产品经理来说,生存才是第一要义,因此中小企业的产品经理一定要优先考虑商业化。产品经理只有在对利益诉求没有那么高的情况下,才会优先考虑用户体验。\sphinxhref{https://www.zhihu.com/pub/reader/119980992/chapter/1284104619489460224}{10}%
\begin{footnote}[34]\sphinxAtStartFootnote
\sphinxnolinkurl{https://www.zhihu.com/pub/reader/119980992/chapter/1284104619489460224}
%
\end{footnote}


\subparagraph{先满足哪类用户?}
\label{\detokenize{chapter_introduction/need:id12}}
如果说,普通用户有一个需求,核心用户也有一个需求,而且,两种用户群体的需求都是比较重要的大需求,那么,你觉得应该先满足哪类用户?

答案是先满足普通用户。

因为核心用户属于平台忠诚度非常高的用户群体,在产品里已经投入了较多的时间和精力,并不会因为你的需求稍微晚一些就选择离开;即使离开也只是暂时的,他还会回来的,因为这是沉没成本,而且你并没有做出严重伤害他们的行为。

不论运营还是产品,都应该重视用户留存,也必须绝对肯定的要重视,甚至在产品的爆发期,留存大于一切。


\subparagraph{小众需求}
\label{\detokenize{chapter_introduction/need:id13}}
区别是小众还是大部分用户的需求,优先满足大部分用户的需求。\sphinxhref{https://www.zhihu.com/question/59911327/answer/259895734}{15}%
\begin{footnote}[35]\sphinxAtStartFootnote
\sphinxnolinkurl{https://www.zhihu.com/question/59911327/answer/259895734}
%
\end{footnote}


\subparagraph{转移需求}
\label{\detokenize{chapter_introduction/need:id14}}
需求延期


\paragraph{需求的checklist}
\label{\detokenize{chapter_introduction/need:checklist}}
\begin{figure}[H]
\centering
\capstart

\noindent\sphinxincludegraphics{{define_need}.png}
\caption{需求的checklist}\label{\detokenize{chapter_introduction/need:id22}}\end{figure}


\paragraph{需求管理附加值}
\label{\detokenize{chapter_introduction/need:id15}}
通过对需求各项属性的统计,进行需求管理;统计提交人的需求数量、提交时间等信息、每个模块的需求数量、每个分类的需求数量等;


\paragraph{需求管理软件}
\label{\detokenize{chapter_introduction/need:id16}}
\begin{figure}[H]
\centering
\capstart

\noindent\sphinxincludegraphics{{need_list}.png}
\caption{产品需求list}\label{\detokenize{chapter_introduction/need:id23}}\end{figure}

除了excel,还有Mantis、Mercury Interactive公司的Quality
Center、IBM的Rational RequisitePro等;


\paragraph{需求评审}
\label{\detokenize{chapter_introduction/need:id17}}
统一思想,明确需求,确定实现过程的会议

\begin{figure}[H]
\centering
\capstart

\noindent\sphinxincludegraphics{{need_who_judge}.jpg}
\caption{需求评审参与人员}\label{\detokenize{chapter_introduction/need:id24}}\end{figure}


\paragraph{量化需求 11\sphinxfootnotemark[36]}
\label{\detokenize{chapter_introduction/need:id18}}%
\begin{footnotetext}[36]\sphinxAtStartFootnote
\sphinxnolinkurl{http://www.xmamiga.com/3573/s}
%
\end{footnotetext}\ignorespaces 

\subparagraph{为什么要量化需求}
\label{\detokenize{chapter_introduction/need:id19}}
基于概率 –> 需求量化 –> 技术可行性预研 –> 得出结论 –> 开发、测试、上线
–> 复盘

在产品开始之前提出量化标准,方便对工作成果进行衡量; 一般有三种结果:
\begin{itemize}
\item {} 
存在“小数据”或若标注的情况 –>
保持上线时间不变,需求更改:在算法精度上进行妥协【尽量避免】;

\item {} 
存在“小数据”或若标注的情况 –>
保持量化标准不变:申请更多的资源【尽量避免】;

\item {} 
基于现有资源在规定时间内可以实现量化要求;

\end{itemize}


\subparagraph{需要考虑的点}
\label{\detokenize{chapter_introduction/need:id20}}\begin{itemize}
\item {} 
预研期间:衡量数据质量、算力资源、上线时间,在算法精度上给出合理量化标准,或者要求增加资源投入;

\item {} 
开发、测试、上线后:对量化的目标进行精准地验证,进行 A/B
测试时可以比较 A、B 两个方案的效果;

\item {} 
复盘期间:总结量化评估经验,和研发团队沟通,了解团队技术实力和算法能力边界。争取量化更靠谱,减少需求变更和额外申请资源。

\end{itemize}


\paragraph{示例}
\label{\detokenize{chapter_introduction/need:id21}}
同花顺问财功能需求分析文档:\sphinxurl{https://www.jianshu.com/p/130fb4f1036a}


\subsubsection{产品}
\label{\detokenize{chapter_introduction/Product:id1}}\label{\detokenize{chapter_introduction/Product::doc}}

\paragraph{什么是产品?}
\label{\detokenize{chapter_introduction/Product:id2}}
任何事情都可以是产品,任何满足人们实际需求的有形商品或无形服务都是产品。\sphinxhref{https://www.zhihu.com/question/19563363/answer/1592063177}{26}%
\begin{footnote}[37]\sphinxAtStartFootnote
\sphinxnolinkurl{https://www.zhihu.com/question/19563363/answer/1592063177}
%
\end{footnote}

任何让用户挑不出 Bug 并有高留存高活跃度的行为举措都是产品思维。

你的公众号是产品,你写的文章也是产品,甚至,你今天要穿一套怎样的衣服去一家怎样的公司,说怎样的话去面试,都是产品思维。


\paragraph{灵感}
\label{\detokenize{chapter_introduction/Product:id3}}
要把自己当成一个傻子,用一种特别挑剔、不满意的方法去试用自己的产品,绝对能够发现很多问题。我们很多产品的改进,都是来自用户看起来不理性的投诉、粗暴的回应,但是认真想一想,用户的不满背后其实都代表了一种需求。
\sphinxhref{https://www.jianshu.com/p/ef308c923f06}{5}%
\begin{footnote}[38]\sphinxAtStartFootnote
\sphinxnolinkurl{https://www.jianshu.com/p/ef308c923f06}
%
\end{footnote}


\paragraph{什么算作成功的产品?}
\label{\detokenize{chapter_introduction/Product:id4}}

\subparagraph{常见错误视角:}
\label{\detokenize{chapter_introduction/Product:id5}}\begin{itemize}
\item {} 
用户:用户人数多(留存、活跃、新增这个优先度递减)

\item {} 
社会:可以利用信息打破阶层跨越的

\item {} 
运营:人气很旺但是并不要赚钱,只是用来市场卡位的

\item {} 
美工:长得好看的产品

\item {} 
UI/UE:易学易用、层次感

\end{itemize}

\begin{figure}[H]
\centering
\capstart

\noindent\sphinxincludegraphics{{good_product}.png}
\caption{好产品}\label{\detokenize{chapter_introduction/Product:id54}}\end{figure}


\subparagraph{核心:要、能、赚!}
\label{\detokenize{chapter_introduction/Product:id6}}\begin{itemize}
\item {} 
要!–需求/用户:有用、好用、爱用

\item {} 
能!–技术/开发:能做、好做、做好

\item {} 
赚!–商业/企业:活命、赚钱、持续

\end{itemize}


\subparagraph{用户的维度}
\label{\detokenize{chapter_introduction/Product:id7}}\begin{enumerate}
\sphinxsetlistlabels{\arabic}{enumi}{enumii}{}{.}%
\item {} 
有用(理性 IQ):要解决用户的需求

\item {} 
好用(感性 AQ):要拥有不错的体验(快速、有效、简洁地交互)

\item {} 
爱用(人性
EQ):要有黏性(持续不断地迭代,以满足用户不断产生的新需求;产品对用户的吸引力沉淀为对公司极具价值的品牌忠诚度,产品和用户有了情感链接,形成黑话“亚文化现象”)、且能打动人,去调动人原始的感觉和情绪,带来美好的感觉。\sphinxhref{https://t.qidianla.com/1173713.html}{17}%
\begin{footnote}[39]\sphinxAtStartFootnote
\sphinxnolinkurl{https://t.qidianla.com/1173713.html}
%
\end{footnote}

\end{enumerate}


\subparagraph{需求}
\label{\detokenize{chapter_introduction/Product:id8}}
这是一个产品之所以被称为产品的前提,产品的本质就是用来解决需求的,黏性和体验是之后的事。需求可用技术满足。


\subparagraph{优秀的用户体验}
\label{\detokenize{chapter_introduction/Product:id9}}
在这个产品同质化竞争比较严重的时代,好的用户体验就是商机,尤其是你弯道超车的策略之一。例如电商三只松鼠的用户体验:在你收到包裹的时候你就会发现每个包装坚果的箱子上都会贴着一段手写体的给快递的话:“快递叔叔我要到我主人那了,你一定要轻拿轻放哦,如果你需要的话也可以直接购买哦。”打开包裹后会发现,每一包坚果都送了一个果壳袋,方便把果壳放在里面;打开坚果的包装袋后,每一个袋子里还有一个封口夹,可以把吃了一半但吃不完的坚果袋儿封住。令你想不到的还有,袋子里备好的擦手湿巾,方便吃之前不用洗手。这些小小的变化使他们的销售额不断增长。所以说好的用户体验就是商机。

\sphinxstylestrong{五个层次:}\sphinxhref{https://www.bilibili.com/video/BV1wt411Y7zh}{15}%
\begin{footnote}[40]\sphinxAtStartFootnote
\sphinxnolinkurl{https://www.bilibili.com/video/BV1wt411Y7zh}
%
\end{footnote}

购买体验–>功能体验–>易用体验–>品牌体验–>情感体验
\begin{itemize}
\item {} 
购买体验:ofo(考虑到校园用户对价格敏感,所以身份免押金)VS
摩拜单车(299的押金)

\item {} 
功能体验:摩拜单车(以四年免维护为目标,以质量为重,价格昂贵且笨重的实心轮胎,街上稍有损坏的)VS
ofo(虽然投放多,普通的气胎,容易损坏,周破损率达20\%)

\item {} 
易用体验:摩拜单车(后续优化单车,推出轻骑版;智能化系统、GPS定位寻找单车)
VS
之前的摩拜单车(笨重的实心轮胎)、ofo(开始时采取机械锁\sphinxhref{https://vickydyy.github.io/2019/06/20/6-20\%EF\%BC\%9A\%E4\%BA\%A7\%E5\%93\%81\%E6\%80\%9D\%E8\%80\%83\%EF\%BC\%88\%E4\%B8\%80\%EF\%BC\%89/}{21}%
\begin{footnote}[41]\sphinxAtStartFootnote
\sphinxnolinkurl{https://vickydyy.github.io/2019/06/20/6-20\%EF\%BC\%9A\%E4\%BA\%A7\%E5\%93\%81\%E6\%80\%9D\%E8\%80\%83\%EF\%BC\%88\%E4\%B8\%80\%EF\%BC\%89/}
%
\end{footnote})

\item {} 
品牌体验:哈罗单车(借助蚂蚁金服的资金、流量互补,实现饿了吗、骑车运动的互补)
摩拜单车(接入美团App、更名为美团单车、流量互补,实现美团外卖、骑车运动的互补\sphinxhref{https://finance.sina.com.cn/tech/2020-12-15/doc-iiznezxs7048324.shtml}{22}%
\begin{footnote}[42]\sphinxAtStartFootnote
\sphinxnolinkurl{https://finance.sina.com.cn/tech/2020-12-15/doc-iiznezxs7048324.shtml}
%
\end{footnote})
VS ofo(滴滴要求绝对控制力,不考虑合并和收购,最后。。)

\item {} 
情感体验:?

\end{itemize}

\sphinxstylestrong{功能体验的更多维度}\sphinxhref{https://coffee.pmcaff.com/article/1329730610781312/pmcaff?utm\_source=forum}{13}%
\begin{footnote}[43]\sphinxAtStartFootnote
\sphinxnolinkurl{https://coffee.pmcaff.com/article/1329730610781312/pmcaff?utm\_source=forum}
%
\end{footnote}
\begin{itemize}
\item {} 
丰富度如何?

\item {} 
反应速度如何?

\item {} 
扩展性如何?

\item {} 
精致度如何?

\item {} 
Bug多不多?

\end{itemize}


\subparagraph{黏性}
\label{\detokenize{chapter_introduction/Product:id10}}
一个成功的产品,一定是\sphinxstylestrong{不断被用户想起的产品},一旦用户产生了某种需求,就能想起你,这就是一个好的产品。有黏性的产品一定是很好的解决了某种需求,而且做到了竞品没有的高度。用户用了一次就不再使用,说明你的产品并不好,或者说干脆就是定位有了问题。

动态演化:
\begin{itemize}
\item {} 
个体价值—>个体黏性—>群体黏性一>生态黏性

\item {} 
活靶子 —>护城河 —>增压器 一> 培养皿

\end{itemize}


\subparagraph{技术实现的维度}
\label{\detokenize{chapter_introduction/Product:id11}}
成功的产品,其产品方案利用当前的技术就可以实现,并且可以长期维护、持续完善。

技术需考虑自己的团队开发能力(新团队否、能否解决技术难点),需求考虑项目周期(是否需要简化、砍),来如期完成项目。\sphinxhref{https://zhuanlan.zhihu.com/p/24855458}{12}%
\begin{footnote}[44]\sphinxAtStartFootnote
\sphinxnolinkurl{https://zhuanlan.zhihu.com/p/24855458}
%
\end{footnote}


\subparagraph{商业的维度}
\label{\detokenize{chapter_introduction/Product:id12}}
成功的产品可以持续为公司创造长期的商业价值,包括但不限于用户规模、产品利润等。


\subparagraph{示例}
\label{\detokenize{chapter_introduction/Product:id13}}
为什么幼年淘宝是个好产品
\sphinxhref{https://weread.qq.com/web/reader/8d632bc07208ed1c8d697c4k37632cd021737693cfc7149}{9}%
\begin{footnote}[45]\sphinxAtStartFootnote
\sphinxnolinkurl{https://weread.qq.com/web/reader/8d632bc07208ed1c8d697c4k37632cd021737693cfc7149}
%
\end{footnote}


\paragraph{从 0 到 1 做产品 14\sphinxfootnotemark[46]}
\label{\detokenize{chapter_introduction/Product:id14}}%
\begin{footnotetext}[46]\sphinxAtStartFootnote
\sphinxnolinkurl{https://www.zhihu.com/market/paid\_column/1312360599620358144/section/1312363033470443520}
%
\end{footnotetext}\ignorespaces 
如果一定要高度概括腾讯产品经理从 0 到 1
做产品的方法,我们可以简化为三步:

找——比——试。
\begin{enumerate}
\sphinxsetlistlabels{\arabic}{enumi}{enumii}{}{.}%
\item {} 
找,多方挖掘,找到需求;

\item {} 
比,对比环境和自身,确定产品策略;

\item {} 
试,快速尝试,实践才出真知。

\end{enumerate}


\paragraph{产品思维}
\label{\detokenize{chapter_introduction/Product:id15}}\begin{enumerate}
\sphinxsetlistlabels{\arabic}{enumi}{enumii}{}{.}%
\item {} 
定位、差异点

\item {} 
动力引擎

\item {} 
核心输出

\item {} 
外围价值

\item {} 
商业模式

\item {} 
价值放大

\end{enumerate}


\subparagraph{定位、差异点}
\label{\detokenize{chapter_introduction/Product:id16}}
陌陌:陌生交友 知群:企业资源


\subparagraph{动力引擎}
\label{\detokenize{chapter_introduction/Product:id17}}
直播带货:流量、价格(新形式的团购)。 知群:以招聘内推作为底层驱动力


\subparagraph{核心输出}
\label{\detokenize{chapter_introduction/Product:id18}}
知群:入行


\subparagraph{外围价值}
\label{\detokenize{chapter_introduction/Product:id19}}
知乎:交流空间、外部性


\subparagraph{商业模式}
\label{\detokenize{chapter_introduction/Product:id20}}
知群:TOP班提供可靠学习保障。


\subparagraph{价值放大}
\label{\detokenize{chapter_introduction/Product:id21}}

\paragraph{产品层次}
\label{\detokenize{chapter_introduction/Product:id22}}\begin{enumerate}
\sphinxsetlistlabels{\arabic}{enumi}{enumii}{}{.}%
\item {} 
核心产品:真正所要求——购买唇膏,不只是买嘴唇的颜色而是销售希望

\item {} 
有形产品:质量水准、功能特色、式样、品牌以及包装。

\item {} 
附加产品:提供购买零件保证书、技术、免费操作课程、快速维修服务,和询问任何问题及疑难的免费电话专线。

\end{enumerate}


\paragraph{做出来和推出去的效率}
\label{\detokenize{chapter_introduction/Product:id23}}\begin{itemize}
\item {} 
出来的效率,在管理学里专业的说法是“生产制造的可扩展性”。打比方说,一款产品如果给
10
倍的用户使用,那么这款产品在生产制造上的成本提升是多少?如果成本提升得少,就是可扩展性高。

\item {} 
推出去的效率,它的专业说法叫“销售传播的可扩展性”。同样的比方,一款产品给
10
倍的用户使用,它在销售传播上的成本提升是多少?如果成本提升得少,就是可扩展性高。

\end{itemize}

提升做出来效率的常见方法:
\begin{itemize}
\item {} 
降低复制成本,比如标准化、数字化、智能化;

\item {} 
提供基础设施,然后众包 / 外包生产过程。

\end{itemize}

提升推出去效率的常见方法:
\begin{itemize}
\item {} 
消除时间、地点等销售传播的限制因素;

\item {} 
产品数字化,减少,甚至消除物流环节;

\item {} 
提供基础设施,然后众包/外包分销过程。

\end{itemize}

关于做出来和推出去效率的提升,我们能看到一些大的趋势:
\begin{itemize}
\item {} 
首先,产品交付从实到虚,再到虚实结合,这是因为人们不能只活在数字世界里;

\item {} 
其次,效率高的产品供给方,都会渐渐的演变成平台,让更多的玩家、更多的用户参与到做和推的过程中。

\end{itemize}


\paragraph{如何起步? 3\sphinxfootnotemark[47]}
\label{\detokenize{chapter_introduction/Product:id24}}%
\begin{footnotetext}[47]\sphinxAtStartFootnote
\sphinxnolinkurl{https://www.jianshu.com/p/266cd3df64d5}
%
\end{footnotetext}\ignorespaces 
一款产品的起步是有个逻辑顺序的,《产品游戏化》一书里归纳出的逻辑顺序是:习惯打造、启程、发现、精通。以下,我们把“习惯打造”模块,简称为“习惯”模块。

需要注意的是,这和一个新用户使用产品的逻辑顺序并不相同,因为用户是按照“发现、启程、习惯、精通”来使用产品的。

产品起步思维是有实用场景的,更适用于正在从小量用户逐步扩展到大量用户的产品。如果你服务的是少数大客户,第一次交付的产品就需要已经相对完整才行。


\subparagraph{习惯}
\label{\detokenize{chapter_introduction/Product:id25}}
先回到“做产品”的逻辑上来,它的第一个模块是习惯。

这要求你先打造出某个
对用户有价值的闭环,用户来了,获得价值了,下一次还愿意来。这个最小的产品模块,已经可以用来做“留存假设”的验证,所以这也算是第三轮的
MVP 了,这里的 P 代表 Product。


\subparagraph{启程}
\label{\detokenize{chapter_introduction/Product:id26}}
第二个要做的模块是启程,即用户的第一次体验。

启程模块是产品的验证对象扩展开以后,做给相对的“新手用户”的,最常见的就是各种产品里的“新手上路”模块。

之所以不用最先做启程,是因为产品的早期使用者,往往是高手行家,我们也常把这群人称作种子用户、天使用户,即便没人手把手指导,他们也能用得很溜。


\subparagraph{发现}
\label{\detokenize{chapter_introduction/Product:id27}}
然后是发现模块。有了一批新人用户之后,我们算是验证完了启程与习惯模块,这时候产品应该进入推广阶段,开始做“发现”模块。

我们要发掘出用户在何时、何地会对产品产生第一印象,会通过什么渠道第一次接触产品。如果是手机
App
的话,用户在应用商店里看到的广告、搜索产品名称、下载安装,直到第一次点击打开
App 都算是发现模块。


\subparagraph{精通}
\label{\detokenize{chapter_introduction/Product:id28}}
最后要做的是精通模块。当产品运营了一段时间之后,就会有相当数量的用户对产品了如指掌,这时候才有必要给他们打造“精通”系统,让他们不断地收到新的刺激。这是高级功能,可以考虑让高级用户参与贡献,充分利用你最热情用户的深层次需求和驱动力。

比如,服务产品里,让高级用户做志愿者,论坛里让高级用户做版主,游戏里让高级玩家做分区的督导者等等,都算是产品的精通模块。

这时候,你已经在打造上一讲里提到的个体粘性、群体粘性了,这些特性的成功,会使产品拥有自己的正反馈闭环,也常常被叫做增长飞轮。

当然也有例外,有些产品,所有用户很快就精通了,基本上,这个产品也就没啥想象力了,比如手电筒
App。


\paragraph{产品服务系统 4\sphinxfootnotemark[48]}
\label{\detokenize{chapter_introduction/Product:id29}}%
\begin{footnotetext}[48]\sphinxAtStartFootnote
\sphinxnolinkurl{https://www.jianshu.com/p/75de15c9d6b3}
%
\end{footnotetext}\ignorespaces 
“产品服务系统”能以一种集成的、有针对性的方式进行产品分类,精准地满足用户需求,有助于产品的创新。

产品服务系统的核心要点是,任何广义的产品都包含有实体部分和服务部分,三大导向,从实体到服务,实体部分越来越少,服务部分越来越多,逐渐过渡。

分三大类导向的产品服务系统,即“实体导向”“使用导向”“结果导向”。


\subparagraph{实体导向}
\label{\detokenize{chapter_introduction/Product:id30}}
第一种,实体导向的产品服务系统。这种类型是以实体为主,包含有少量服务。它的服务目的是让用户可以顺利地使用产品实体,是与实体紧密相关的。比如空调和它的上门安装、保修服务。


\subparagraph{使用导向}
\label{\detokenize{chapter_introduction/Product:id31}}
使用导向的产品服务系统,它和实体导向型产品的区别在于,供给方给你的不是所有权,而是长期独占的使用权(Lease),或者是某种条件下,一段时间的使用权(Renting/Sharing),甚至是共享的使用权(Pooling)。比如摩拜单车
1 小时使用权。

因为使用导向的情况下,用户买的并不是实体,所以相关的配套服务会多一些,以确保用户使用顺利。


\subparagraph{结果导向}
\label{\detokenize{chapter_introduction/Product:id32}}
结果导向就以服务为主了,你要买的不是一个实体,而是一种“结果”,使用实体只是为了达成结果需要用的一个过程或者一个媒介而已。比如网络广告,按点击量、按成交量付费等模式。

有时在消费完结果导向的产品后,你可能甚至感知不到实体的存在,比如付费聊天、轻咨询,甚至是寺庙里求签拜佛。


\paragraph{三种导向间的演变趋势}
\label{\detokenize{chapter_introduction/Product:id33}}

\subparagraph{用户模式}
\label{\detokenize{chapter_introduction/Product:id34}}
从实体到服务的变化意味着从“成交终止”到“成交开始”。

从实体到服务,供应者与用户的关系有越来越紧密的趋势,触点越来越多,用户尝试的成本越来越低。

在这个时代,因为社会供给越来越丰富,所以各种产品的市场会越来越供过于求,这会导致需求驱动而不是生产驱动,用户变得越来越重要。所以,我们要好好思考如何更多地接触用户,给用户创造价值,从而为公司创造更多的商业价值。

比如一个做人工智能客服机器人的生意,这是一种典型的 2B
企业服务。对小客户的交付中,实体比例更多,更偏实体导向,大多数功能让客户自助完成使用。但对
VIP
大客户的交付中,就是服务比例更多,更偏结果导向,甚至会提供外包的客服人员。

因为相对来说,小客户比较容易批量获得,而大客户需要一个个地”啃“,更需要建立长期的关系。这一点,也会体现在下面的增长模式上。

所以,从这个角度来看,越是重要的用户,就越要用服务比例高的产品服务系统来完成交付。


\subparagraph{增长模式}
\label{\detokenize{chapter_introduction/Product:id35}}
增长模式下的实体到服务,是从“数量复制”到“人尽其用”。

不同的卖法,增长的方式不同。实体更容易标准化,从而可以批量地卖给更多的用户,我把这个叫作数量复制。而服务的极致体验是个性化,所以增长的模式挖掘每个用户的更多需求,这叫作人尽其用。

这个角度给我们的启发就是,随着产品供给的极大丰富,没有被开发的用户已经越来越少了,所以我们更要思考如何在已有用户身上做文章,精细化运营。

比如一个软件,是使用导向的产品,如果它卖的是软件 1
年的使用权,就没法向数据量大的用户收更多的钱。这时候如果改为结果导向,根据数据量收费,那么既可以让数据量少的用户几乎免费使用,降低他们尝试的门槛,也可以充分赚取大客户的费用,对方也更愿意为好的结果付费。


\subparagraph{财务模式}
\label{\detokenize{chapter_introduction/Product:id36}}
在财务模式下,实体到服务的变化是从“当期收入”变为“预期收入”。

从用户模式到增长模式,再到财务模式,实体比例越来越低,会造成的必然结果是短期收入减少,资产投入增加,利润减少,但预期利润增加。

比如房企不卖房,改做长租生意了,那就没有了卖房时那一大笔的即时收入,在一段时间内的资金压力就很大。

所以,偏服务的产品服务系统,不确定性更高,更需要我们掌握新的产品创新方法,更需要有长远的眼光。


\paragraph{从单一产品到产品矩阵 2\sphinxfootnotemark[49]}
\label{\detokenize{chapter_introduction/Product:id37}}%
\begin{footnotetext}[49]\sphinxAtStartFootnote
\sphinxnolinkurl{https://www.jianshu.com/p/ed738dac00e5}
%
\end{footnotetext}\ignorespaces \begin{itemize}
\item {} 
PSF,是
Problem\sphinxhyphen{}Solution\sphinxhyphen{}Fit,问题与解决方案的匹配,这是价值假设,相当于从 0
到 1;

\item {} 
PMF,是 Product\sphinxhyphen{}Market\sphinxhyphen{}Fit,产品与市场的匹配,这是增长假设,是从 1 到
N;

\item {} 
PRF,是
Positioning\sphinxhyphen{}Resource\sphinxhyphen{}Fit,定位与资源的匹配,这是长青假设,是从 N
到正无穷。

\end{itemize}


\subparagraph{价值假设:问题与解决方案的匹配}
\label{\detokenize{chapter_introduction/Product:id38}}
PSF 要验证的是价值,即问题对不对,解决方案对不对,对应着前两轮
MVP,也就是 Paperwork 和 Prototype 阶段。

这一阶段中常见的错误有三点:
\begin{enumerate}
\sphinxsetlistlabels{\arabic}{enumi}{enumii}{}{.}%
\item {} 
问题不存在,是臆想出来的。(点子过滤器来避免)

\item {} 
解决方案不存在。根本无解的事,多思无益。(询问领域专家来避免)

\item {} 
问题也有,解决方案也有,但是问题(P)和解决方案(S)不匹配。(用户测试来避免)

\end{enumerate}


\subparagraph{增长假设:产品与市场的匹配}
\label{\detokenize{chapter_introduction/Product:id39}}
如果问题(P)和解决方案(S)匹配了,达到了
PSF,我们才算有了一个产品,也就是 PMF 的
P——Product。这时候重点就变成了后两轮 MVP,Product 和 Promotion
相关的内容了。

PMF 讲的是产品与市场的匹配,要验证的是增长,也就是产品的生产 /
分销可扩展性好不好,市场是不是足够好。

产品与市场的匹配中常见的几种错误:
\begin{enumerate}
\sphinxsetlistlabels{\arabic}{enumi}{enumii}{}{.}%
\item {} 
产品有了,但本身无法规模化。(寻求模式突破来解决)

\item {} 
没有一个相对大、不断增长的市场,导致这事儿只是个小生意,不是个大事业。(当然,“做大”是一种选择,“小而美”也是一种选择,只不过你想选哪种得先想清楚。)

\item {} 
产品和市场不匹配。比如在行,产品与市场的供需关系上出现了一个逻辑问题,即“一群有时间没钱的人,花钱买一群有钱没时间的人的时间”,这是不可能有很大增长的。(需要对行业做深入的分析研究)

\item {} 
做一件事,问题与解决方案是必须匹配上的。但是如果你觉得小而美也挺好的话,追求产品与市场的匹配(PMF
和增长)就并不是必须的。

\end{enumerate}


\subparagraph{长青假设:定位与资源的匹配}
\label{\detokenize{chapter_introduction/Product:id40}}
如果你做到了产品与市场的匹配(PMF
达到),那就算找到了一个自己公司团队的定位,也就是 PRF 的
P,Positioning,下一步就是达成 PRF,完成定位与资源的匹配来扩大战果。

这一部分,就超出了单一产品的范畴,不在四轮 MVP
框架里了。这里面也有几种常犯的错误:
\begin{enumerate}
\sphinxsetlistlabels{\arabic}{enumi}{enumii}{}{.}%
\item {} 
定位不可持续。定位是公司立身之本,即“使命、愿景、价值观”,是公司早期靠着创始团队、产品、用户之间的反复互动,逐渐打磨清晰的,它给我们的后续产品指明了大方向。如果你的定位是“最好的马车公司”,那汽车时代来临时,你该怎么办?

\item {} 
资源没能积累。随着公司、产品、用户的协同发展,应该要有某种资源像雪球一样越滚越大,形成自己的增长飞轮。比如用户越来越多,成交就越来越多,对商家的议价能力就越来越强,商品价格越来越便宜,用户就越来越多,完成闭环。这就是一个典型的增长飞轮。而有不少公司,除了不断赚点钱,没能积累下什么。

\item {} 
定位和资源不匹配。这一点阿里做得不错,使命是“让天下没有难做的生意”,重要资源是不断积累的数据,数据可以帮助生意做得更好。

\item {} 
如果成功达成了定位和资源的匹配,那我们就可以说,公司有了一个很好的产品矩阵。

\end{enumerate}


\subparagraph{单一产品在矩阵中的评价}
\label{\detokenize{chapter_introduction/Product:id41}}
矩阵中的任何一个产品,做得好的话,都要考虑和其他众多产品的关系,都要求该产品满足三个条件:可复用、能积累、善生死。
\begin{enumerate}
\sphinxsetlistlabels{\arabic}{enumi}{enumii}{}{.}%
\item {} 
可复用:就是说可以复用公司的积累,比如供应链、比如数据沉淀、比如已有用户。如果不能复用的话,你推出的第二个产品和众多竞争对手相比,就没有任何优势。

\item {} 
能积累:意味着后续产品可以为公司积累将来可复用的资源,好产品应该让整体更优,而不是单纯地消耗公司的积累。

\item {} 
善生死:说的是要有合理的生命周期管理。每一个产品,都要在该进入的时候进入,该退出的时候退出。一个公司和一个生态系统一样,资源都是有限的,有时候死亡(即释放资源)可以创造巨大的价值。

\end{enumerate}


\paragraph{产品成功 6\sphinxfootnotemark[50]}
\label{\detokenize{chapter_introduction/Product:id42}}%
\begin{footnotetext}[50]\sphinxAtStartFootnote
\sphinxnolinkurl{https://www.jianshu.com/p/111d9fcc005e?utm\_campaign=maleskine\&utm\_content=note\&utm\_medium=seo\_notes\&utm\_source=recommendation}
%
\end{footnotetext}\ignorespaces 
产品设计、竞争策略、全局商战

底层思维:整体式设计是指不能单点极致,需要整体开花;用户决策思维意味着产品设计要关注用户决策,而不是一味追求用户体验;价值本位模型是指产品设计要围绕核心价值展开,流量圈养的互联网思维并不适用。

产品创新(微观层):探索——发展——成熟的三个阶段,AI+硬件的模式:硬件赋能模式和互联网管道模式(智能音箱背后的语音平台、内容服务等)。

竞争态势(中观层,竞争产品阵营当下的格局和未来的势头):对抗(巨头争霸)、割据(多品牌分散)、创新(率先进入新领域)、延伸(大生态中延伸小生态)

商战全局(宏观层):以产品为根基,表现为价值驱动(追求先进性,如大疆,激进)、认知驱动(追求差异性,如oppo,后发制人)、购买驱动(追求经济性,如小米)三种类型。


\paragraph{产品模型 7\sphinxfootnotemark[51]}
\label{\detokenize{chapter_introduction/Product:id43}}%
\begin{footnotetext}[51]\sphinxAtStartFootnote
\sphinxnolinkurl{https://weread.qq.com/web/reader/46532b707210fc4f465d044k3c5327902153c59dc0488e1}
%
\end{footnotetext}\ignorespaces 
了解各类产品的模型,比如社交产品的基本形态、电商产品的基本形态等


\paragraph{目的 8\sphinxfootnotemark[52]}
\label{\detokenize{chapter_introduction/Product:id44}}%
\begin{footnotetext}[52]\sphinxAtStartFootnote
\sphinxnolinkurl{https://weread.qq.com/web/reader/8d632bc07208ed1c8d697c4kecc32f3013eccbc87e4b62e}
%
\end{footnotetext}\ignorespaces 
产品思维是方法,而产品创新是目的。只有完成从想到做、从思维方式具体到做事方法的转变,才能使产品创新落地。


\paragraph{早期产品的三个核心问题}
\label{\detokenize{chapter_introduction/Product:id45}}\begin{itemize}
\item {} 
需求:解决什么人的什么需求 {\hyperref[\detokenize{chapter_introduction/need:need}]{\sphinxcrossref{\DUrole{std,std-ref}{需求}}}} (\autopageref*{\detokenize{chapter_introduction/need:need}})

\item {} 
具体形态:如何解决的 {\hyperref[\detokenize{chapter_knowledge/index:chap-skill}]{\sphinxcrossref{\DUrole{std,std-ref}{全流程知识}}}} (\autopageref*{\detokenize{chapter_knowledge/index:chap-skill}})

\item {} 
推广:人们怎么知道它 {\hyperref[\detokenize{chapter_idea/GTM:yunying}]{\sphinxcrossref{\DUrole{std,std-ref}{产品运营 1}}}} (\autopageref*{\detokenize{chapter_idea/GTM:yunying}})

\end{itemize}

首先是需求,产品所解决的需求是一个多大的市场规模,是大部分人都需要的,还是仅局限在一个垂直的人群,规模多大?这个需求出现的频率如何,是每天都需要的,还是每周几次,还是隔上至少个把月甚至更长时间才能想到的?这个道理很简单,那些绝大部分人都需要的而且每天都需要的,是S级的需求,比如微信解决的是沟通这样的SSS级需求,又比如搜索、支付、影音,都是大部分人经常用到的;而那些尽管小众、但经常使用,又或者虽然使用频度不高,但也是大部分人都需要的,是次一级需求,比如教育、办公,比如购物、旅游;遇到那些不知道做给谁的、不知道多久才能想起来一次的产品,基本就算了吧。

推广:销售人员把和客户接触的宝贵时间当成产品宣讲会,只把重点放在产品特性上,大讲特讲产品的技术规格和其他属性,这样的产品和服务宣传只会让客户更加迷惑,实质就是没有从客户功能需求的角度考虑问题,其实要问的问题应该是客户买电钻要在砖墙上打孔,还是要在木板上打孔,是为了挂一台电视、一幅画打孔还是为了挂衣服打孔。正是这些用户需求才让客户有了购买电钻的需求。


\paragraph{产品模型}
\label{\detokenize{chapter_introduction/Product:id46}}
产品模型=商业模式+产品架构+运营体系\sphinxhref{https://www.jianshu.com/p/39472c3f993c}{24}%
\begin{footnote}[53]\sphinxAtStartFootnote
\sphinxnolinkurl{https://www.jianshu.com/p/39472c3f993c}
%
\end{footnote}
\begin{itemize}
\item {} 
商业模式指产品的市场潜力,综合了用户与营收价值。

\item {} 
产品架构指产品设计上的框架与核心系统。

\item {} 
运营体系与产品架构类似,指运营上的组织流程与重点难点。

\end{itemize}


\paragraph{产品定位 11\sphinxfootnotemark[54]}
\label{\detokenize{chapter_introduction/Product:id47}}%
\begin{footnotetext}[54]\sphinxAtStartFootnote
\sphinxnolinkurl{https://zhuanlan.zhihu.com/p/24855458}
%
\end{footnotetext}\ignorespaces 
产品定位就是关于产品的目标、范围、特征等约束条件,包括产品定义和用户需求。

用户需求 + 产品定义 = 产品定位


\paragraph{产品层次 19\sphinxfootnotemark[55]}
\label{\detokenize{chapter_introduction/Product:id48}}%
\begin{footnotetext}[55]\sphinxAtStartFootnote
\sphinxnolinkurl{https://zhuanlan.zhihu.com/p/25772426}
%
\end{footnotetext}\ignorespaces \begin{itemize}
\item {} 
核心产品:向顾客提供的产品的基本效用或利益

\item {} 
形式产品:实现形式,品质、式样、特征、商标及包装

\item {} 
期望产品:购买产品时渴望得到与产品密切相关的一整套属性和条件

\item {} 
延伸产品:购买形式产品和期望产品时附带获得的各种利益的总和

\item {} 
潜在产品:可能发展成为未来最终产品的潜在状态的产品

\end{itemize}


\paragraph{产品计划 16\sphinxfootnotemark[56]}
\label{\detokenize{chapter_introduction/Product:id49}}%
\begin{footnotetext}[56]\sphinxAtStartFootnote
\sphinxnolinkurl{https://blog.csdn.net/liwei16611/article/details/82630078}
%
\end{footnotetext}\ignorespaces 
产品遵循1\sphinxhyphen{}3\sphinxhyphen{}6\sphinxhyphen{}9原则:一个月时间完成项目的可行性研究和市场定位、市场细化;3个月内制订出产品开发的二三级计划和产品包计划;6个月系统试运行,做市场发布准备,产品命名和定价;9个月系统商业运行,市场发布、推广和销售。

产品计划是解决开发周期过长问题的重要手段,对于开发周期短于9个月的产品,也应按照1\sphinxhyphen{}3\sphinxhyphen{}6\sphinxhyphen{}9的时间分配比例来对项目进行控制,从而达到按照暨定目标快速有效推出产品的目的。


\paragraph{性价比}
\label{\detokenize{chapter_introduction/Product:id50}}
性价比 = 价值 / 成本


\paragraph{产品价值}
\label{\detokenize{chapter_introduction/Product:id51}}\begin{itemize}
\item {} 
广度:潜在用户数*单用户价值

\item {} 
频度:需求频次*单词价值

\item {} 
强度:可替代性、紧急程度、持续时间

\end{itemize}


\paragraph{产品观念 20\sphinxfootnotemark[57]}
\label{\detokenize{chapter_introduction/Product:id52}}%
\begin{footnotetext}[57]\sphinxAtStartFootnote
\sphinxnolinkurl{http://www.crazypm.com/zixun/167422.html}
%
\end{footnotetext}\ignorespaces \begin{enumerate}
\sphinxsetlistlabels{\arabic}{enumi}{enumii}{}{.}%
\item {} 
产品的功能多还是少,不应该是开发者自己决定的,而是下游的客户和上游的组件供货商共同决定的,成熟稳定的功能可以越多越好,但不成熟或者没有把握的功能一定要尽量砍掉。这一方面是用户体验的考虑,另一方面是要节约有限的成本与宝贵的时间。

\item {} 
产品的开发应该尽可能敏捷,这种敏捷体现在软件产品上,就是从第一版成型的应用或者系统开始,都应该是可发布的。在发布之后的每个后续阶段,都保持有随时可以交付的产品,交付的产品可以做加法或者减法,但是千万不能出现说因为几个功能还未实现导致不能交付的拖延情况。

\item {} 
不要制定遥远漫长的工作计划,如果做不到高瞻远瞩,那就尽可能把最重要的功能实现,保证系统可运行,其他的,寄希望于天才的援手和用户的体谅。

\end{enumerate}

“牙刷测试”:它们得是必需品,而不是细节上的小创新。\sphinxhref{https://tech.sina.cn/csj/2019-09-04/doc-iicezueu3276088.d.html?from=wap}{25}%
\begin{footnote}[58]\sphinxAtStartFootnote
\sphinxnolinkurl{https://tech.sina.cn/csj/2019-09-04/doc-iicezueu3276088.d.html?from=wap}
%
\end{footnote}


\paragraph{更多}
\label{\detokenize{chapter_introduction/Product:id53}}
产品思维:\sphinxurl{https://www.itsiwei.com/category/rest}


\subsubsection{To Business}
\label{\detokenize{chapter_introduction/2B:to-business}}\label{\detokenize{chapter_introduction/2B::doc}}

\paragraph{如何定义B端或C端产品}
\label{\detokenize{chapter_introduction/2B:bc}}
一个产品属于B端还是C端,取决于这个产品究竟在解决什么样的问题,而不在于产品究竟会有什么样的功能。例如,IM即时通信,通常被理解为C端产品的功能,然而这个功能在某些场景下也可以被认为是B端产品的功能,如微信是典型的C端产品,但并不妨碍它发展为B端产品,企业微信便应运而生,你不能简单地说企业微信还是C端产品
\sphinxhref{https://weread.qq.com/web/reader/40632860719ad5bb4060856k9a132c802349a1158154a83}{14}%
\begin{footnote}[59]\sphinxAtStartFootnote
\sphinxnolinkurl{https://weread.qq.com/web/reader/40632860719ad5bb4060856k9a132c802349a1158154a83}
%
\end{footnote}、智能语音技术可以输出到智能客服和个人助手
\sphinxhref{https://www.pianshen.com/article/2712685407}{24}%
\begin{footnote}[60]\sphinxAtStartFootnote
\sphinxnolinkurl{https://www.pianshen.com/article/2712685407}
%
\end{footnote}


\paragraph{定义}
\label{\detokenize{chapter_introduction/2B:id1}}
To
B产品主要是面向企业的软件系统产品,例如企业的ERP系统、OA系统以及线上型业务的用户系统、订单系统等。它是一个概称,既可以指单个系统,也可以指代某一个集合体系,例如一整套的解决方案。
\sphinxhref{https://tanxianlian.com/2020/03/07/\%e6\%88\%91\%e7\%9a\%84to-b\%e4\%ba\%a7\%e5\%93\%81\%e6\%96\%b9\%e6\%b3\%95\%e8\%ae\%ba/}{1}%
\begin{footnote}[61]\sphinxAtStartFootnote
\sphinxnolinkurl{https://tanxianlian.com/2020/03/07/\%e6\%88\%91\%e7\%9a\%84to-b\%e4\%ba\%a7\%e5\%93\%81\%e6\%96\%b9\%e6\%b3\%95\%e8\%ae\%ba/}
%
\end{footnote}

B端从概念上既可以是to企业内部这个B,也可以是to企业外部(如其他企业或政府)这个B。\sphinxhref{https://mp.weixin.qq.com/s/RTEOekR8Z-0QK\_p-y2yzbQs}{20}%
\begin{footnote}[62]\sphinxAtStartFootnote
\sphinxnolinkurl{https://mp.weixin.qq.com/s/RTEOekR8Z-0QK\_p-y2yzbQs}
%
\end{footnote}

核心着力点是:在功能稳定的前提下提高企业生产、销售、组织等活动的效率。

通用性不高,难参考难复制每个企业以及企业内部的各需求部门业务都不同。

一般 To B
往往的产品往往是行业内技术或者测试出身转岗更为合适,业务的熟悉程度要求高一些。
\sphinxhref{https://m.zhipin.com/mpa/html/get/share?type=4\&contentId=8eaf00b18d9c5148tnVy2t-9GVI~\&uid=5885ce18425348b00nR73NS6E1FX\&identity=0}{3}%
\begin{footnote}[63]\sphinxAtStartFootnote
\sphinxnolinkurl{https://m.zhipin.com/mpa/html/get/share?type=4\&contentId=8eaf00b18d9c5148tnVy2t-9GVI~\&uid=5885ce18425348b00nR73NS6E1FX\&identity=0}
%
\end{footnote}

针对2B产品,部门流程改变,组织架构调整,工作流程优化等
\sphinxhref{http://www.woshipm.com/pmd/1792966.html}{4}%
\begin{footnote}[64]\sphinxAtStartFootnote
\sphinxnolinkurl{http://www.woshipm.com/pmd/1792966.html}
%
\end{footnote} 产品经理站在 IT
专业上,有更多的话语权来建议业务部门产品应该如何搭建。
\sphinxhref{https://www.yuque.com/weis/pm/wkixxq}{11}%
\begin{footnote}[65]\sphinxAtStartFootnote
\sphinxnolinkurl{https://www.yuque.com/weis/pm/wkixxq}
%
\end{footnote} 收益难以量化

其产品设计的逻辑是“重流程规则和信息流、轻体验”
\sphinxhref{https://www.aiyingli.com/74015.html}{10}%
\begin{footnote}[66]\sphinxAtStartFootnote
\sphinxnolinkurl{https://www.aiyingli.com/74015.html}
%
\end{footnote}
要解决的主要是不同生产关系的协作沟通需求,在中心化的组织架构下,B端产品需要满足不同层级,组织内外的协作沟通。在这样的背景下,B端产品会考虑比如权限设计,角色分配,运营数据展示等功能。\sphinxhref{https://coffee.pmcaff.com/article/2447262389384320/pmcaff?utm\_source=forum}{31}%
\begin{footnote}[67]\sphinxAtStartFootnote
\sphinxnolinkurl{https://coffee.pmcaff.com/article/2447262389384320/pmcaff?utm\_source=forum}
%
\end{footnote}围绕核心业务场景完善业务闭环。B端需求基本来自真实的业务场景,1个需求可能来自10个业务方的反馈,10个需求可能满足100个业务场景,因此B端的需求基本都有客户价值,也就无法像C端那样只需抓住核心功能将体验极致优化;B端需求从上到下,需要优先深入相关的业务操作场景,先满足业务场景的可用性需求,再满足业务场景下具体的操作体验
\sphinxhref{https://www.pianshen.com/article/39201625760/}{27}%
\begin{footnote}[68]\sphinxAtStartFootnote
\sphinxnolinkurl{https://www.pianshen.com/article/39201625760/}
%
\end{footnote}

ToB端会用很多的表格去承载信息,在设计架构的时候需要充分理解这些信息表格的意义是什么,要对表格反应的数据保持足够的敏感度。更多的时候,信息表格所涉及到很多选项和必填项,这个时候要区分填写表格内容的核心是什么?什么是他们最关注的信息以及收集数据的目的。
\sphinxhref{https://www.zhihu.com/question/32285554}{28}%
\begin{footnote}[69]\sphinxAtStartFootnote
\sphinxnolinkurl{https://www.zhihu.com/question/32285554}
%
\end{footnote}

其产品经理叫后端产品经理也叫后台产品经理。\sphinxhref{http://www.woshipm.com/zhichang/807191.html}{32}%
\begin{footnote}[70]\sphinxAtStartFootnote
\sphinxnolinkurl{http://www.woshipm.com/zhichang/807191.html}
%
\end{footnote}


\paragraph{用户 6\sphinxfootnotemark[71]}
\label{\detokenize{chapter_introduction/2B:id2}}%
\begin{footnotetext}[71]\sphinxAtStartFootnote
\sphinxnolinkurl{http://www.pmtalk.club/\#/article/detail/6375}
%
\end{footnotetext}\ignorespaces 
B端面向特定的群体,可以收集到明确的需求,是从1到无穷大的过程。
\sphinxhref{https://www.yinxiang.com/everhub/note/f9ab87ee-73e6-4241-9428-9507cbfd007f}{19}%
\begin{footnote}[72]\sphinxAtStartFootnote
\sphinxnolinkurl{https://www.yinxiang.com/everhub/note/f9ab87ee-73e6-4241-9428-9507cbfd007f}
%
\end{footnote}

B端产品的用户较为理性;决策者以及关键干系人,会从好用、性价比、提高效率、适配公司情况等多个维度来进行综合考虑和筛选,最后选择最为合适的产品来使用。

B端产品较为复杂,并且与日常生活相关不大,多数为垂直行业属性打造,市场上少见,所以一般需要用户花费一定时间来进行学习

角色分工:B端产品至少会分为决策者、管理者、普通员工。

G端(To
Goverment)产品,是比B端产品更B端的产品,产品经理其它能力的权重更低,找到关键決策人利益诉求的权重更高(可能是隐性的,需要高难度能力)。\sphinxhref{https://zhuanlan.zhihu.com/p/127962653}{13}%
\begin{footnote}[73]\sphinxAtStartFootnote
\sphinxnolinkurl{https://zhuanlan.zhihu.com/p/127962653}
%
\end{footnote}


\subparagraph{客户成功 22\sphinxfootnotemark[74]}
\label{\detokenize{chapter_introduction/2B:id3}}%
\begin{footnotetext}[74]\sphinxAtStartFootnote
\sphinxnolinkurl{https://www.zhihu.com/pub/reader/119980992/chapter/1284104650384265216}
%
\end{footnotetext}\ignorespaces \begin{enumerate}
\sphinxsetlistlabels{\arabic}{enumi}{enumii}{}{.}%
\item {} 
服务的客户最终获得了成功,只有这样,他们才能够继续复购企业的业务;

\item {} 
让客户实现了复购,这本质上体现了自身的贡献与价值。

\end{enumerate}


\paragraph{产品本质}
\label{\detokenize{chapter_introduction/2B:id4}}
To
B产品的本质是效率(生产力)工具,不管是服务于企业的采购、行政、人事、成品/库存管理、销售,本质上都是服务于提高企业的效率。这点在产品的策划、打磨、生产等各阶段都是考虑的重点。


\paragraph{过程 9\sphinxfootnotemark[75]}
\label{\detokenize{chapter_introduction/2B:id5}}%
\begin{footnotetext}[75]\sphinxAtStartFootnote
\sphinxnolinkurl{https://zhiya360.com/50903.html}
%
\end{footnotetext}\ignorespaces 
以B端产品为例,包含需求调研,竞品分析,产品规划,产品设计,跟进开发,测试上线,售前推广,客户部署,培训指导,售后跟踪,一次性项目/迭代优化等阶段。


\paragraph{产品分类 16\sphinxfootnotemark[76]}
\label{\detokenize{chapter_introduction/2B:id6}}%
\begin{footnotetext}[76]\sphinxAtStartFootnote
\sphinxnolinkurl{https://www.jianshu.com/p/b159b89df3f8}
%
\end{footnotetext}\ignorespaces \begin{enumerate}
\sphinxsetlistlabels{\arabic}{enumi}{enumii}{}{.}%
\item {} 
放大信息类产品
信息类平台1688、hc360等大型B2B电商平台,通过互联网的流通和放大效应,降低信息获取者的获取成本,降低信息发声者的推广成本,从而放大信息的价值;

\item {} 
提高工作协同效率产品
钉钉。OA、jira、teambition、点餐系统等皆是促进信息传递,提升企业工作协作效率。通过交互设计的手段,将不同的企业信息、任务流聚集到有效的产品之中。

\item {} 
降低固有成本类产品
智能客服系统、智能售票、财务管理软件等系统,是降低企业固有成本类产品。

\item {} 
数据挖掘类产品
随着大数据的到来,企业也越来越注重大数据对企业运营的影响,大数据相关的产品,提供准确数据统计,更全、更准确,成为企业做出正确决策的参考依据;

\end{enumerate}


\paragraph{其他}
\label{\detokenize{chapter_introduction/2B:id7}}

\subparagraph{业务渠道 18\sphinxfootnotemark[77]}
\label{\detokenize{chapter_introduction/2B:id8}}%
\begin{footnotetext}[77]\sphinxAtStartFootnote
\sphinxnolinkurl{http://reader.epubee.com/books/mobile/12/1240b863fa87878a6e1899147685e374/text00000.html}
%
\end{footnotetext}\ignorespaces 
2B业务渠道分为线下渠道和线上渠道,而线下渠道分为经销渠道、KA渠道、企业大客户渠道,线上

渠道分为1688、零售通、新通路、找钢网、找煤网、找塑料网、企业自建的B2B在线订货渠道等。


\subparagraph{收费模式}
\label{\detokenize{chapter_introduction/2B:id9}}
传统的to
B产品大多是本地部署,一次收费,后续的维护、更新等服务,按次收费。

企业外部的B端产品:平台型企业给卖家提供运营管理支持的系统

随着云计算的兴盛,Saas(Software as a
service)服务也随之兴起,云端部署+按年收费的模式开始逐渐成为主流。

经过国内外的众多实践,证明Saas云端服务+按年收费的模式有众多的优点,也是更可行有效,更能实现服务更优、利润最大化的方式。


\subparagraph{部署方式}
\label{\detokenize{chapter_introduction/2B:id10}}\begin{itemize}
\item {} 
私有化部署:软件部署在自己的IDC以及主机和存储设备中,与外网隔离

\item {} 
云部署:软件部署在第三方云服务商

\end{itemize}


\subparagraph{技术架构}
\label{\detokenize{chapter_introduction/2B:id11}}\begin{itemize}
\item {} 
B/S 更优

\item {} 
C/S

\end{itemize}


\subparagraph{业务方向:}
\label{\detokenize{chapter_introduction/2B:id12}}\begin{itemize}
\item {} 
业务支持类:企业经营管理或核心业务开展(CRM、仓配系统)

\item {} 
办公协同类:企业内部协同办公(OA office automation、HRM)

\item {} 
商家端管理:商家前台/后台/商家管理

\end{itemize}


\paragraph{迭代模式:稳定 or 常变?}
\label{\detokenize{chapter_introduction/2B:or}}
\sphinxstylestrong{对企业}:企业用户的业务在一定时间内具有连续性,因此需求也存在一定时间的延续性。在操作体验上,企业用户并不看重趣味性、更在乎便利性,因此在操作上也会形成惯性路径,即使用习惯。

因此,企业用户希望to B的产品具有一定稳定性。

但业务和需求始终都还是会有变化的,不可能始终不变,因此to
B的产品还是要保持一定的迭代节奏,只不过相比to
C产品,迭代的周期要更长,以及基于前述的原因,迭代要更多基于优化而非大改,不然就使自身丧失了当初的立身基础。

\sphinxstylestrong{对产品经理}:为大企业做内部工具,或业务支持工具。这个需求永远存在,所有企业发展越好,所有业务规模越大,这方面的需求就越强烈。大企业也不会像to
C产品一样只有一两家幸存,仅互联网就有很多企业和很多业务足够大,所以岗位容量多,选择丰富。做这些工具产品的方法和原理也比较相似,经验价值有一定可迁移性,所以如果离开一个大企业,还能去另一个大企业。所以,做to
B产品是求稳的产品经理一个不错的选择方向。
\sphinxhref{https://www.yinxiang.com/everhub/note/b60b7f01-4a91-473d-82a1-40fc5aa25734}{21}%
\begin{footnote}[78]\sphinxAtStartFootnote
\sphinxnolinkurl{https://www.yinxiang.com/everhub/note/b60b7f01-4a91-473d-82a1-40fc5aa25734}
%
\end{footnote}


\paragraph{发展路径}
\label{\detokenize{chapter_introduction/2B:id13}}
\sphinxstyleemphasis{第一阶段:内部效率工具}

该阶段是To
B产品的创生阶段,面向的用户主要是企业内部的使用者,产品的生产者是卖方,使用者是买方,产品的被使用就能直接或间接地为企业提高生产力,使产品有存续的价值和空间。

该阶段,因为面向的用户主要是企业内部的使用者,并且产品的生产者是卖方、使用者是买方的关系,因此,产品通常是免费的。

\sphinxstyleemphasis{第二阶段:内部商业化}

在很多大型企业,例如集团公司,或者是BU结构的公司,会实行内部成本核算。

内部的效率工具经由内部成本核算,实现的是内部商业化。

企业内部的中后台系统大多都属于前面的两个阶段。

这两个阶段的to B产品有两个关键词:有限内部竞争、行政+利益驱动 。

具体来说,大公司内部可能会有多个团队进行内部竞争,开发相同的产品,以及主要靠行政命令以及利益联合作为产品推广的驱动力。

\sphinxstyleemphasis{第三阶段:外部商业化}

该阶段的产品较少。

一是外部商业化的产品,因为面向外部市场,市场化对产品本身的要求会更高;

二是to
B产品的功能和架构和企业的组织结构及业务体系是适配的,因此从内部转变为外部产品的时候,在产品架构及功能体系方面,会有很大的不同;

三是因为是面向的企业增多,彼此需求并不一致,因此需要面对更高的复杂性。


\paragraph{突破点}
\label{\detokenize{chapter_introduction/2B:id14}}
宏观上,要更多地依靠生态体系,或者联盟合作,来进行市场拓展。

例如,某销售型企业需要整套的企业在线化解决方案,公司A主打产品是销售Saas系统,并且是行业最佳,但该客户还有财务、行政Saas系统的需求。

客户担心如果选用了不同服务方的不同产品,体系割裂,数据及账号权限体系不统一,并且也不便于地实现多系统的集成,所以不愿意单独选用公司A的销售Saas产品。

如果有公司B刚好能提供该客户剩余需求的财务及行政系统,公司A和公司B合作,对各自产品进行集合,打通数据及账号权限体系,打包提供给该客户,就可以提升公司A和公司B彼此的交易成功率及市场空间。


\subparagraph{权限设计 7\sphinxfootnotemark[79]}
\label{\detokenize{chapter_introduction/2B:id15}}%
\begin{footnotetext}[79]\sphinxAtStartFootnote
\sphinxnolinkurl{https://github.com/JoJoDU/Book\_Notes/issues/2}
%
\end{footnotetext}\ignorespaces 

\subparagraph{权限表}
\label{\detokenize{chapter_introduction/2B:id16}}

\begin{savenotes}\sphinxattablestart
\centering
\begin{tabulary}{\linewidth}[t]{|T|T|T|T|}
\hline
\sphinxstyletheadfamily 
一级导航
&\sphinxstyletheadfamily 
页面
&\sphinxstyletheadfamily 
页面元素
&\sphinxstyletheadfamily 
角色1
\\
\hline
客户管理
&
门店列表
&
“编辑”按钮
&
√
\\
\hline
\end{tabulary}
\par
\sphinxattableend\end{savenotes}


\subparagraph{RBAC(role based access control)权限模型}
\label{\detokenize{chapter_introduction/2B:rbac-role-based-access-control}}
ER模型:用户、角色、用户组


\subparagraph{数据权限:各个角色能看到的数据范围}
\label{\detokenize{chapter_introduction/2B:id17}}\begin{itemize}
\item {} 
机构树

\item {} 
数据范围是当前节点及其子节点

\item {} 
客户地区

\end{itemize}


\paragraph{学习难 26\sphinxfootnotemark[80]}
\label{\detokenize{chapter_introduction/2B:id18}}%
\begin{footnotetext}[80]\sphinxAtStartFootnote
\sphinxnolinkurl{https://www.36kr.com/p/1723904065537}
%
\end{footnotetext}\ignorespaces \begin{enumerate}
\sphinxsetlistlabels{\arabic}{enumi}{enumii}{}{.}%
\item {} 
因为B端多数都是企业内部系统,其业务运作和产品设计涉及商业机密,很少对外公开,更难以像C端产品那样暴露在公众面前被大家研习。

\item {} 
是因为B端产品专业性太强,比如搞供应链的,搞CRM的,搞ERP的,领域不同,业务背景不同,产品解决方案不同,除了软件设计方法论是共同的,其他部分都很难提炼出共性的方法论给B端人指导,这就不像C端产品有那么多共性的话题可以探讨。

\item {} 
是因为B端产品复杂程度高,如果想把问题聊透,必须有很深的功底。虽然业界有大量的大拿和专家,但是愿意写文章分享的毕竟是少数,因此很遗憾的造成这个领域学习资料偏少。

\item {} 
是因为很多B端产品知识被沉淀在传统软件公司,很多互联网B端从业者不理解传统软件,而传统IT人又不理解互联网,导致本该有深度融通的两者之间的割裂。

\item {} 
B端:我很难是用户

\end{enumerate}


\paragraph{深耕细作}
\label{\detokenize{chapter_introduction/2B:id19}}
在IT行业内,很多做TO
B产品的公司是可以发展很久的,比如IBM、微软等。\sphinxhref{https://www.epubit.com/onlineEbookReader?id=0dc0f81254b5455c892a7896d0f7d0ac\&pid=9821123a37484750b6317c8c1c217500\&isFalls=true}{8}%
\begin{footnote}[81]\sphinxAtStartFootnote
\sphinxnolinkurl{https://www.epubit.com/onlineEbookReader?id=0dc0f81254b5455c892a7896d0f7d0ac\&pid=9821123a37484750b6317c8c1c217500\&isFalls=true}
%
\end{footnote}

To
B产品更重要的是对商业模式的经营和核心功能的打磨。一旦占据了市场领先地位,将比较难被替代,试想一个公司的CRM系统被替代需要付出多少的代价?先要把数据转移,然后还需要适配各个系统。

在前面产品核心竞争力的章节也提到过,ToB产品提供给用户的更多的是服务,服务包含售前、售后、文档、产品功能等多个方面,建立这一套完整的体系是需要经历很长时间打磨的,所以做ToB的产品经理要耐得住性子,积累与沉淀行业\sphinxhref{http://www.woshipm.com/zhichang/4308504.html}{34}%
\begin{footnote}[82]\sphinxAtStartFootnote
\sphinxnolinkurl{http://www.woshipm.com/zhichang/4308504.html}
%
\end{footnote},打磨产品才有可能得到市场的认可。

对于 To B
来说,潜在用户一共就那么多,这里舍弃点、那里舍弃点,你还有多少用户?你还做个毛线?所以必须深耕细作,争取把行业通吃,toB
里面赢家通吃是很常见的。

深耕细作依赖行业理解。如果你没有参与过销售管理,你就很难明白为什么 CRM
里需要那么复杂的销售线索分配机制。

然而现在的互联网产品人,大多一毕业就进入互联网圈,没有接触行业一线的机会,也不愿意去了解。互联网来钱太容易,PM
都干不了脏活。不信你问问身边的,有几个敢去主动给用户打电话?

而那些在行业里经验丰富的人呢?互联网公司嫌弃他们又土又穷、不懂互联网,很少给他们转业的机会。这些人因为专业、技能、经验和学历的原因,不太容易进入互联网行业;即便进入了,也不可能担任重要角色。可以说很大一部分想法和创新都被封闭和埋没在了领域内部。

这么说肯定有点太抬高领域人才而贬低 PM
们了。事实上你让一个行业大佬来做互联网,大概率难有起色。无讼的创始人是全国顶级律师,产品一坨屎;iCourt
创始人是搞律师培训的,产品年收入破亿。toB
产品人需要把互联网和行业知识相结合,打造完整的产品研发和服务团队。有这能力的人,凤毛麟角。


\subparagraph{建立产品服务体系}
\label{\detokenize{chapter_introduction/2B:id20}}
建立产品服务体系是TOB产品与ToC产品的一大区别。在商业化服务场景下,光有孤零零的产品功能是无法跟客户需求匹配的,需要有一系列使用帮助教程。其中产品经理的主要工作是输出整个产品的功能说明文档,要细致到每个按钮。以作者参与的机器学习平台产品为例,单是功能介绍文档就有将近4万字。这些说明文档需要不断地随着产品功能的更新而更新,所以文档工作通常会占用产品经理大量的精力。另外,针对部分比较难以上手的产品,建议要录制使用视频,以视频解说的方式介绍产品的功能。视频教程也是目前人工智能ToB领域比较普遍的功能介绍方式。根据作者的工作经验,录制视频教程的效果会优于文档。

除了功能介绍文档等相关材料的开发工作,服务体系的建立依赖于许多支持团队的合作,产品经理在其中的角色是沟通和协调,将整个售前和售后链路打通。比如产品经理需要给售后团队明确的SLA准则(SLA指的是售后服务保障),并且培训售后团队,使售后团队在遇到用户索赔和追责的时候可以快速处理问题。在售前方面,产品经理也要协调各个售前工程师和销售团队,给前方团队输出与产品售卖相关的商业指导书,扫清产品售卖工作的障碍。

在产品对外服务的过程中,产品经理是整个体系的接口人,任何售前售后、开发端出现问题都会与产品经理联系,所以在各个团队之间的沟通和协调工作会占据很大的一部分精力。


\paragraph{产品路标规划:干系人关键问题拆解法(2B产品)4\sphinxfootnotemark[83]}
\label{\detokenize{chapter_introduction/2B:b-4}}%
\begin{footnotetext}[83]\sphinxAtStartFootnote
\sphinxnolinkurl{http://www.woshipm.com/pmd/1792966.html}
%
\end{footnotetext}\ignorespaces 
针对2B产品时,产品规划的核心往往是解决各干系人的问题(优先级客户 > 竞品
>
用户),围绕着产品核心价值路径,不断汇总并提出问题。沿着客户路径,不断的去分解他们的问题,同时要寻找到解决方案。2B类产品的规划就是将各种问题和解决方案进行汇总,然后按照优先级进行罗列,最终形成产品路线图。(有点像需求优先级的判断)

首先要明确产品的核心目标,在该目标的基础上,我们自己要先拆解出几个子问题,比如涉及哪些业务部门?涉及哪些职位?怎样使用产品?使用场景是什么?等。

接下来,可以在以上问题的基础上,做各部门干系人的访谈,继续获得更细节的问题,比如部门的对接人是谁?部门需要得到什么服务支持?部门需要提供什么服务?哪个部门的需求最紧急等等。

实际工作中我们可能会分解出很多的问题,在此基础上,划分好优先级,形成一个在哪个阶段使用什么方式解决哪些干系人的什么问题的产品规划方案。


\paragraph{MVP}
\label{\detokenize{chapter_introduction/2B:mvp}}
基本原则 \sphinxhref{https://www.niaogebiji.com/article-31885-1.html}{17}%
\begin{footnote}[84]\sphinxAtStartFootnote
\sphinxnolinkurl{https://www.niaogebiji.com/article-31885-1.html}
%
\end{footnote}
\begin{itemize}
\item {} 
突出优势:基于企业自身当前的能力优势

\item {} 
先易后难:从简单的功能开始

\item {} 
内外兼顾:有大局观,进行通盘考虑。

\end{itemize}

关键特征是:\sphinxhref{https://www.zhihu.com/question/417983831/answer/1777334295}{30}%
\begin{footnote}[85]\sphinxAtStartFootnote
\sphinxnolinkurl{https://www.zhihu.com/question/417983831/answer/1777334295}
%
\end{footnote}
\begin{itemize}
\item {} 
体现核心价值主张

\item {} 
可以有效的传达给早期利益相关者并得到验证

\item {} 
可以不断循环以满足更大的愿景

\end{itemize}


\paragraph{原型设计要求 5\sphinxfootnotemark[86]}
\label{\detokenize{chapter_introduction/2B:id21}}%
\begin{footnotetext}[86]\sphinxAtStartFootnote
\sphinxnolinkurl{http://www.woshipm.com/pmd/3755958.html}
%
\end{footnotetext}\ignorespaces 
对原型能力要求没那么高,基本就是一个打辅助的作用,来解释需求文档(以前我都是画个demo后直接找UI小姐姐\textasciitilde{})


\paragraph{项目管理}
\label{\detokenize{chapter_introduction/2B:id22}}
项目管理保证软件开发按计划推进、落地,保障团队的产品研发效率与质量

\begin{figure}[H]
\centering
\capstart

\noindent\sphinxincludegraphics{{project_manage}.jpg}
\caption{标准项目流程}\label{\detokenize{chapter_introduction/2B:id41}}\end{figure}


\subparagraph{工作重点}
\label{\detokenize{chapter_introduction/2B:id23}}\begin{itemize}
\item {} 
设计并优化项目管理制度:合理的规范制度可以约束产品团队行为也可以保护产品团队的权益
比如要求业务部门提交需求时提交BRD

\item {} 
负责大中型项目的立项实施

\end{itemize}


\subparagraph{如何把控项目进度}
\label{\detokenize{chapter_introduction/2B:id24}}\begin{itemize}
\item {} 
细化工作,明确交付 工作拆解,明确细化是想的负责人、交付物、时间点

\item {} 
通过机制把控进度

\end{itemize}
\begin{enumerate}
\sphinxsetlistlabels{\arabic}{enumi}{enumii}{}{.}%
\item {} 
开展定期会议:聚合项目各方人员,回顾上次会议以来的进展、遇到的苦难、下一次会议前的计划

\item {} 
每日站会

\item {} 
日报、周报:通报进展、警示风险

\end{enumerate}
\begin{itemize}
\item {} 
编写内容清晰的日报或周报
管理项目、通报进展;争取关注度和资源,解决项目中遇到的问题

\end{itemize}
\begin{enumerate}
\sphinxsetlistlabels{\arabic}{enumi}{enumii}{}{.}%
\item {} 
本周进度

\item {} 
项目风险

\item {} 
下周计划

\item {} 
整体进度

\end{enumerate}
\begin{itemize}
\item {} 
保持责任心

\end{itemize}


\paragraph{运营管理}
\label{\detokenize{chapter_introduction/2B:id25}}

\subparagraph{产品运营岗}
\label{\detokenize{chapter_introduction/2B:id26}}\begin{itemize}
\item {} 
SaaS:偏销售、BD职能

\item {} 
双边市场攻击端:商家、店铺运营,偏C端运营

\item {} 
内部业务系统(以下讨论方向)

\end{itemize}


\subparagraph{工作内容}
\label{\detokenize{chapter_introduction/2B:id27}}
工作目标:挖掘B端产品能力(现有功能推广、协助完成产品升级优化),帮助其余人解决业务问题(营收增长、风险控制)
\begin{itemize}
\item {} 
产品功能推广培训:线上推广宣传(消息推送、公告通知);现场培训(复杂升级改造)

\item {} 
问题解答处理:初上线的系统,组织试点用户群,搜集问题;解答迅速有效;总结共性问题,以便产品进行系统优化

\item {} 
需求采集过滤:收集一线业务人员的直接诉求,挖掘到真正会产生影响的需求,和PM持续优化产品

\item {} 
项目效果分析:对上线功能进行持续的数据分析和观察;作为中立方,考核项目效果和收益,给出客观分析

\item {} 
业务诊断分析:诊断业务,分析问题,提出解决方案

\end{itemize}


\subparagraph{业务运营岗}
\label{\detokenize{chapter_introduction/2B:id28}}\begin{itemize}
\item {} 
业务支持:审批、核对、检验

\item {} 
流程管理:保证分支机构管理的规范性和可靠性

\item {} 
策略制订:促销策略、定价策略、供应商返点策略、仓储排班策略

\item {} 
绩效考核制度制订:自顶向下

\item {} 
培训考核

\item {} 
项目管理

\item {} 
合规质检

\item {} 
数据分析

\end{itemize}


\paragraph{Buyer和User的区别}
\label{\detokenize{chapter_introduction/2B:buyeruser}}
产品经理在设计功能的时候一定要区分这个功能是提供给客户(Buyer)还是用户(User)的,
Buyer指的是实际为产品付费的人,User指的是产品的实际使用用户。

对于ToB产品来讲, Buyer和User往往在企业是不同的角色!且客户带来用户


\subparagraph{Buyer是决策链路的核心}
\label{\detokenize{chapter_introduction/2B:buyer}}
通常决定是否购买一款产品的人是公司的CTO或者CEO,决定购买的人是产品的客户,CTO和CEO更关注产品使用过程中的消耗以及是否能节约人力。也就是说无论是产品设计还是最终产品的营销策略,核心的问题是要提升Buyer的满意度,因为
Buyer是决定是否购买的最关键因素,User更多的是从使用层面去影响
Buyer如果想取得
Buyer的好感,首先要在售卖模式上做文章,产品的售卖是否能做到资源用量可控。比如大部分企业都是预算制,每年在某个部分的消费是提前规划好的,如果产品的售卖模式包含预付费(包年或包月)模式且包含按量付费模式,那么
Buyer在做资源预估的时候就会有更多余地。另外,CTO和CEO很关注产品在使用过程中的效果和消耗,也就是俗称的投入产出比。
很多ToB产品都会为客户设计一个看板用来观察产品的实时具体价值,这些产品的设计都是对
Buyer友好的。


\subparagraph{User决定了产品的业务深度}
\label{\detokenize{chapter_introduction/2B:user}}
既然
Buyer是决定产品购买链路最核心的因素,那么User的体验是否就不重要了?显然不是。让User体验感好,是一个产品能否在一家客户做得更深入的关键。User是产品的实际长期使用者,也是产品后期付费的推动者。
如果User验证了产品功能确实能提升自己的效率,自然会给
Buyer提供一个针对产品的正向反馈,这种反馈是产品后期能否得到续费的关键。
其实产品绝大部分的功能是要针对User设计的,提升User好感的方式也有很多种,比如在User使用产品的整个链路上,ToB产品往往会增加很多文档类的引导,目的就是提升User的好感。很多ToB产品也会把User和Buyer的使用路径通过权限做隔离,
Buyer会看到更多与产品报表相关的内容,而User则更多地看到产品功能性的内容。


\subparagraph{产品购买链路中User和Buyer之间的矛盾}
\label{\detokenize{chapter_introduction/2B:userbuyer}}
User受雇于Buyer,那么在购买决策链路中,他们之间是否也会存在矛盾呢。在许多TB产品的场景下,User和
Buyer之间是有一定矛盾的,比如人工智能算法平台这样的产品,目标客户的
Buyer一般是互联网公司的CTO,User是算法工程师。算法工程师在公司中的使命一般是开发和使用算法去解决诸如智能推荐或智能风控这样的业务问题。如果
Buyer买了算法平台这样的产品,某种意义上会替代原先算法团队的工作,这是否意味着User的工作量小了,团队价值也就没有以前那么大了。所以为了同时满足User和
Buyer的需求,产品在设计和宣传时要注意不要一味地强调替代某些人的工作,而是要把产品功能的核心放到如何去提升他人工作的效率上,这一点对于PaS层的产品尤为重要。
以上是一些针对
Buyer和User不同的产品设计理念和营销方向的分析也是ToB产品和ToC产品的主要区别之一。


\subparagraph{导致了转移成本高}
\label{\detokenize{chapter_introduction/2B:id29}}
B端产品的转移成本相比之下要高很多。消费者和直接使用者不是一批人。对于直接使用者(用户)来说,他们的使用场景是工作环境中,哪怕产品再难用,为了完成自己的工作,用户还是会使用。很多时候没得选也不允许选。对于消费者(企业)来说,打造一个系统的成本较高,费钱费时,难以做到频繁地更新迭代,加上B端产品大都不直接产生利润,企业的动力也不高,所以要求也就是“凑合凑合,能用就行”。种种原因也就造成了B端产品较高的转移成本。\sphinxhref{https://coffee.pmcaff.com/article/2447262389384320/pmcaff?utm\_source=forum}{31}%
\begin{footnote}[87]\sphinxAtStartFootnote
\sphinxnolinkurl{https://coffee.pmcaff.com/article/2447262389384320/pmcaff?utm\_source=forum}
%
\end{footnote}


\paragraph{思路}
\label{\detokenize{chapter_introduction/2B:id30}}\begin{enumerate}
\sphinxsetlistlabels{\arabic}{enumi}{enumii}{}{.}%
\item {} 
客户是谁

\item {} 
解决了客户在什么场景下的什么问题

\item {} 
解决方案是什么,利用了哪些工具/中台能力

\item {} 
在解决方案中,客户需要完成哪些操作,可以看到什么结果

\item {} 
根据这个结果,客户可以去做什么

\end{enumerate}

\begin{figure}[H]
\centering
\capstart

\noindent\sphinxincludegraphics{{2b_idea}.png}
\caption{2B思路示例}\label{\detokenize{chapter_introduction/2B:id42}}\end{figure}


\paragraph{如何挑选垂直行业}
\label{\detokenize{chapter_introduction/2B:id31}}\begin{enumerate}
\sphinxsetlistlabels{\arabic}{enumi}{enumii}{}{.}%
\item {} 
行业要发展空间巨大。

\item {} 
存在刚需且高频的普遍需求没有得到很好满足。

\item {} 
有从事过该行业的专家级用户伙伴。

\item {} 
有能够让产品快速、大规模、低成本扩张的合伙人。

\item {} 
做好吃苦3年的准备。\sphinxhref{https://www.jianshu.com/p/9c3466ec5957}{35}%
\begin{footnote}[88]\sphinxAtStartFootnote
\sphinxnolinkurl{https://www.jianshu.com/p/9c3466ec5957}
%
\end{footnote}

\end{enumerate}


\paragraph{经典模型}
\label{\detokenize{chapter_introduction/2B:id32}}
\begin{figure}[H]
\centering
\capstart

\noindent\sphinxincludegraphics{{classical_model}.png}
\caption{经典模型\sphinxhref{http://www.woshipm.com/pmd/3545074.html}{29}\sphinxfootnotemark[89]}\label{\detokenize{chapter_introduction/2B:id43}}\end{figure}
%
\begin{footnotetext}[89]\sphinxAtStartFootnote
\sphinxnolinkurl{http://www.woshipm.com/pmd/3545074.html}
%
\end{footnotetext}\ignorespaces 

\paragraph{2B2C化 23\sphinxfootnotemark[90]}
\label{\detokenize{chapter_introduction/2B:b2c-23}}%
\begin{footnotetext}[90]\sphinxAtStartFootnote
\sphinxnolinkurl{http://www.changgpm.com/thread-153-1-1.html}
%
\end{footnotetext}\ignorespaces 
原有的B端业务开始向C端进行营销,类似于阿里云、高通、蚂蚁金服等都开始打C端广告,想像当年的英特尔一样,挟消费者以令采购方。2B
VS 2C:ref:\sphinxcode{\sphinxupquote{2B\_VS\_2C}}


\paragraph{To B PM所需能力 33\sphinxfootnotemark[91]}
\label{\detokenize{chapter_introduction/2B:to-b-pm-33}}%
\begin{footnotetext}[91]\sphinxAtStartFootnote
\sphinxnolinkurl{https://mp.weixin.qq.com/s/AVxGdQ0UPj11bMDdF8dPJw}
%
\end{footnotetext}\ignorespaces 

\subparagraph{行业理解能力}
\label{\detokenize{chapter_introduction/2B:id33}}
在To
B产品领域,每个行业都有自己的规则甚至是“潜规则”,要想做好行业类产品,必须对行业知识具有深入的理解。行业知识的学习需要长时间的积累,一旦掌握这些行业知识与“潜规则”之后,就会形成自己的能力壁垒,这也就是为什么To
B产品经理被认为是少有的越老越吃香的职业。


\subparagraph{商业化能力}
\label{\detokenize{chapter_introduction/2B:id34}}
我们对To C产品的盈利模式已经比较清楚。比如一款To
C产品几乎可以在初期不考虑商业模式的情况下,便可以走出一条利用人口和流量红利进行广告获利或者增值服务获利的商业化路径。但这对于To
B产品而言是走不通的,一来各行各业的商业化路径各不相同,二来To
B产品的商业化之路要比To C产品艰难得多。因此,对于To
B产品而言,做好商业化是成功的关键。


\subparagraph{客户导向}
\label{\detokenize{chapter_introduction/2B:id35}}
客户导向作为To B产品经理的核心底层能力,在工作中尤为重要。To
B产品归根结底是面向企业客户的服务,无论从前期的客户需求调研、业务逻辑分析、产品功能建设,再到后期的产品试用、产品交付、客户关系建设、售后服务支持,都要求产品经理具备客户导向的核心能力。尤其在越来越多To
B企业建立客户成功团队后,更是需要以客户为核心,采取各种方法帮助客户取得成功。


\subparagraph{业务知识}
\label{\detokenize{chapter_introduction/2B:id36}}
对于To
B产品经理而言,业务流程和底层逻辑是整个产品的基石。对于产品经理而言,必须对业务流程足够熟悉,才能基于清晰的底层逻辑搭建起上层业务系统,才能够真正解决客户的核心业务问题。同时,在运营过程中,与客户的交流内容也应该围绕核心的业务内容展开,这就要求产品经理对自己所负责的业务知识足够熟悉。具备一定的技术知识,对产品的底层逻辑有足够清晰的理解,就能够在产品设计与客户对接过程中更加从容。


\subparagraph{产品交付}
\label{\detokenize{chapter_introduction/2B:id37}}
与To C产品做完之后就可以直接推给用户不同,To
B产品上线之后还需要与客户进行一系列的对接,才能成功让客户使用产品。To
B产品上线后,需要对销售和商务同事进行针对性的培训,需要制作宣传PPT把产品具备的功能介绍给客户,需要在客户感兴趣时引导客户试用,并在必要时上门进行产品部署。To
B产品的建设只是基础,后续的产品交付环节才是产品成功的关键。这就要求产品经理具备产品交付的能力。


\subparagraph{定价、成本和资源管理}
\label{\detokenize{chapter_introduction/2B:id38}}
定价体系决定了一个To
B产品的收益走向。定价太低可能赚不到钱,定价太高客户就会选择其他竞品。如何结合自己的产品核心优势,采用灵活的计费方式,制定合适的定价策略,对产品的成败同样至关重要。


\subparagraph{新“技术产品”能力模型}
\label{\detokenize{chapter_introduction/2B:id39}}
伴随着产业互联网领域的快速发展,To
B产品经理的能力模型并非一成不变(其他互联网职位也同样如此)。

例如,腾讯在2020年就对产品经理通道的职位及能力模型做了大幅度的调整。针对To
B产品经理,除了已有的“行业应用”职位外,新增了“技术产品”职位,两者的区别在于:“行业应用”更偏向于上文所述的To
B产品经理能力模型,“技术产品”在此基础上更深一层,要求产品经理具备在某个领域的专业技术能力,有能力策划供行业开发者使用的产品(例如:代码编译检测工具)。没有技术背景的产品经理很难承担“技术产品”的职责。


\paragraph{常见AI产品 24\sphinxfootnotemark[92]}
\label{\detokenize{chapter_introduction/2B:ai-24}}%
\begin{footnotetext}[92]\sphinxAtStartFootnote
\sphinxnolinkurl{https://www.pianshen.com/article/2712685407}
%
\end{footnotetext}\ignorespaces \begin{enumerate}
\sphinxsetlistlabels{\arabic}{enumi}{enumii}{}{.}%
\item {} 
智能家居:智能家庭机器人、智能音箱、智能手表等等,成为智能家居控制中心(小米、360、京东)

\item {} 
智能语音助手:Siri、Cortana、Google Assistant、度秘、Bixby

\item {} 
其他:各类Bot、AR/VR、无人机

\end{enumerate}


\paragraph{AI PM}
\label{\detokenize{chapter_introduction/2B:ai-pm}}
关注人工智能产品周期的第一和最后一英里。B2B公司为一小部分消费者解决非常复杂的问题。以安全为例:许多支持AI/
ml的安全公司只专注于应用威胁和异常检测。尽管它们服务的公司可能非常多样化,但提供这些人工智能产品的公司明确关注\sphinxstylestrong{一到两种产品类型}——这是消费者人工智能产品很少拥有的优势。

就商业模式而言,市面上传统toB的AI科技公司,大多倾向采用SaaS订阅模式提供AI服务,如书中所言,对甲方客户公司来说降低了采购门槛,同时也降低了乙方AI服务公司的签单难度,但增加了乙方的运营压力,服务标准化,继而规模化显得生死攸关。在国内市场环境下,服务标准化很理想,现实很骨感,每家甲方公司(尤其传统大公司)都有自己的管理特色和业务特色,若需要深入到甲方客户业务中,就做不到自己的产品标准化,更别说通过标品规模化降低单位成本。既要初心、又要资金,所以选择AI应用场景几乎决定了一家toB的AI科技公司的规模,也决定了个人未来职业发展的高度和宽度。

对企业而言,人工智能产品的目标就是提高企业生产力。人工智能技术通过替代企业中的劳动力提高劳动效率和延伸劳动资料这两种方式,提升企业的生产力。\sphinxhref{https://weread.qq.com/web/reader/0c032c9071dbddbc0c06459k70e32fb021170efdf2eca12}{15}%
\begin{footnote}[93]\sphinxAtStartFootnote
\sphinxnolinkurl{https://weread.qq.com/web/reader/0c032c9071dbddbc0c06459k70e32fb021170efdf2eca12}
%
\end{footnote}


\subparagraph{语音2B产品的困境 25\sphinxfootnotemark[94]}
\label{\detokenize{chapter_introduction/2B:b-25}}%
\begin{footnotetext}[94]\sphinxAtStartFootnote
\sphinxnolinkurl{https://zhuanlan.zhihu.com/p/80824253}
%
\end{footnotetext}\ignorespaces 
\sphinxstylestrong{1. 项目周期长。}
迭代慢,一年以上是很正常的时间,非常不利于个人的成长,稍有不慎就有可能被后起之秀超越。

\sphinxstylestrong{2. 沟通事项多。}
对外,会花费大把的时间去了解甲方的需求,功能实现方案完成后,还会花很多时间和甲方们决定最终方案。对内,还要多和算法岗的伙伴们进行沟通,很多训练的语句都有可能自己花时间帮助清洗。Bug不仅来自功能逻辑,软件开发人员,还会来自nlp。同时,对于nlp中模型输入的语句泛化能力弱,语言、口音形式多样,完全覆盖所有输入可能性非常小;对于模型的输出,给与的是一个概率输出,识别错误是一定会存在的。所以bug次数增多,debug时间成本增加。

\sphinxstylestrong{3. 前人经验不足。}
一般一个行业的发展,有三个阶段:技术优先于产品,产品优先于技术,运营优先于产品。因为现在行业处于起步阶段,没有过多的过来人的指导,很多人都是摸着石头过河,试错过程漫长。

\sphinxstylestrong{4. 技术主导话语权。}
现阶段,技术不成熟,语音识别和理解准确率和反应时间都还有很大的优化空间,语音对于去噪的定位能力不完善,使得语音产品的应用场景大大受限,对于要实现何种功能,完成的结果能达到何种标准,算法工程师可能比产品经理更加明白。

\sphinxstylestrong{5. 甲方提供需求。}
甲方本身处于行业之中,有足够多的经验,他们知道自己要的需求是什么,用户是什么(随着时代发展,这些需求和用户可能会改变,不一定对),只会要求产品按照他的思路拆解和实现功能,提供实现方案,不会在乎你的新颖产品方案,只要听话就行。

\sphinxstylestrong{6. 语音边界不明确。}
对于一般有型产品,用户的操作路线和产品的执行路线是既定好的,但是对于语音产品,没有操作标准,用户能够想问什么就问什么,产品一旦回答不上来,就会降低用户的满意度。

\sphinxstylestrong{7. 人体本身的限制。}
用户用眼去获取信息,能够短时间内获取大量信息内容。对于语音,用户只能记住短时间内的信息,并且语音播报本身就是一个过程,需要一个延迟。

\sphinxstylestrong{注意}

\sphinxstylestrong{1. 明确产品边界。}
在一个或少数几个自己擅长的领域内深耕,能够不断满足用户的需求,并尽量让用户的表现在自己的可控范围之内。在不能实现的领域内,给与明确的边界,直接表明自己做不到,不要给与用户带来过高的不切实际的期望。对于时刻满足用户要求所耗费的精力非常大,实现的可能性也是非常小,所有需求也是不可能穷举完的。

\sphinxstylestrong{2. 明确产品定位。}
新品类和旧品类的判断标准是这个产品解决的主要需求是已有的需求还是新的需求(非智能手机解决的需求主要是沟通联系,不论是不是触屏。只有在苹果一代之后,手机解决的主要需求增加为娱乐才是新品类智能手机,包括之后的线上支付也是一个革新)。语音在技术没有成熟之前,只是一个伪新品类,并不能够满足新的需求,解决的只有提高交互效率,缩短完成任务的路径,更远一步就是给用户带来一些尝鲜感。所以不要改变核心需求,在保证现阶段旧产品的效率和准确率的情况下,提高实现核心需求的便携性,同时在非核心需求上增加用户的新鲜体验感,不断带来小惊喜。

\sphinxstylestrong{3. 提供个性化和情感联系。}
语音最终的优势就是社交和情感联系。可以提供个性化的语音播报,尽可能的识别用户的情感,语音回复和用户保持在一个情感频道。


\paragraph{AI To B}
\label{\detokenize{chapter_introduction/2B:ai-to-b}}
To
B生意获客和客户关系才是壁垒。AI企业在AI行业早期获取融资主要靠创始团队背景、囤积人才数量、获得各种比赛名次,因此在早期军备竞赛催生了AI行业获取人才的高溢价。但到了获客和项目实施阶段,才会发现AI行业人才实在太贵了,导致成本高企,在某些场景和客户群体中,无法与灵活的外包企业低成本抗衡,订单也就拿不下来,影响获客,其次人力成本居高不大也导致难以取得盈亏平衡。因此AI产品实际上是起到两方面价值:创造客户价值和降低实施成本。了解客户需求,不断将需求转化为标准产品方式,将产品经理的劳动凝结到产品中,降低实施成本,进而提升获客竞争力和企业利润。\sphinxhref{https://zhuanlan.zhihu.com/p/55315606}{36}%
\begin{footnote}[95]\sphinxAtStartFootnote
\sphinxnolinkurl{https://zhuanlan.zhihu.com/p/55315606}
%
\end{footnote}


\paragraph{更多}
\label{\detokenize{chapter_introduction/2B:id40}}\begin{itemize}
\item {} 
\sphinxhref{https://www.yuque.com/linyecx/abusg2/occ7pr}{B 类产品文案指南}%
\begin{footnote}[96]\sphinxAtStartFootnote
\sphinxnolinkurl{https://www.yuque.com/linyecx/abusg2/occ7pr}
%
\end{footnote}

\item {} 
\sphinxhref{https://financesonline.com/}{寻找最好的B2B的软件}%
\begin{footnote}[97]\sphinxAtStartFootnote
\sphinxnolinkurl{https://financesonline.com/}
%
\end{footnote}

\item {} 
\sphinxhref{https://www.yuque.com/weis/tob}{B端产品目录}%
\begin{footnote}[98]\sphinxAtStartFootnote
\sphinxnolinkurl{https://www.yuque.com/weis/tob}
%
\end{footnote}

\end{itemize}


\subsubsection{To Customer}
\label{\detokenize{chapter_introduction/2C:to-customer}}\label{\detokenize{chapter_introduction/2C::doc}}

\paragraph{定义}
\label{\detokenize{chapter_introduction/2C:id1}}
C端产品为满足个人服务,满足民生生活,以提供便捷、满足兴趣、欲望、社交、工具的需求为主,不直接为产品带来利益,基本都是免费的。但是C端产品真正所销售的是其用户群体,用户注意力,用户时间、买单方式客户或广告主;

其产品经理叫用户端产品经理,也叫前端产品经理。\sphinxhref{http://www.woshipm.com/zhichang/807191.html}{25}%
\begin{footnote}[99]\sphinxAtStartFootnote
\sphinxnolinkurl{http://www.woshipm.com/zhichang/807191.html}
%
\end{footnote}


\paragraph{方法论}
\label{\detokenize{chapter_introduction/2C:id2}}
To
C产品有众多方法论,从PC互联网到移动互联网也经历了一些演变,但大体都围绕着“用户体验”(User
Experience,简称UE/UX)阐述,百度百科上对这个词的解释是——用户在使用产品过程中建立起来的一种纯主观感受。

围绕“用户体验”,聚焦的维度通常包括内容、功能、交互、视觉等。数据驱动设计,收益可量化(UV,PV,日活,转化率),运营很重要。\sphinxhref{https://github.com/JoJoDU/Book\_Notes/issues/2}{8}%
\begin{footnote}[100]\sphinxAtStartFootnote
\sphinxnolinkurl{https://github.com/JoJoDU/Book\_Notes/issues/2}
%
\end{footnote}

但我觉得这样的方法论是比较单薄的。在产品的设计和开发之外,还应该基于“产品生命周期”的视角,将产品的运营容纳进整体的方法论之中。通用性高可参考可复制。
\sphinxhref{https://m.zhipin.com/mpa/html/get/share?type=4\&contentId=8eaf00b18d9c5148tnVy2t-9GVI~\&uid=5885ce18425348b00nR73NS6E1FX\&identity=0}{3}%
\begin{footnote}[101]\sphinxAtStartFootnote
\sphinxnolinkurl{https://m.zhipin.com/mpa/html/get/share?type=4\&contentId=8eaf00b18d9c5148tnVy2t-9GVI~\&uid=5885ce18425348b00nR73NS6E1FX\&identity=0}
%
\end{footnote}


\paragraph{按功能分类 8\sphinxfootnotemark[102]}
\label{\detokenize{chapter_introduction/2C:id3}}%
\begin{footnotetext}[102]\sphinxAtStartFootnote
\sphinxnolinkurl{https://github.com/JoJoDU/Book\_Notes/issues/2}
%
\end{footnotetext}\ignorespaces \begin{itemize}
\item {} 
工具类:独立功能解决具体需求

\item {} 
内容类:OGC(occupationally职业生产内容)、PGC(professionally专业)、UGC(user用户)

\item {} 
贡献:全新的销售渠道

\item {} 
社交类

\item {} 
平台类

\end{itemize}


\paragraph{业务渠道 14\sphinxfootnotemark[103]}
\label{\detokenize{chapter_introduction/2C:id4}}%
\begin{footnotetext}[103]\sphinxAtStartFootnote
\sphinxnolinkurl{http://reader.epubee.com/books/mobile/12/1240b863fa87878a6e1899147685e374/text00000.html}
%
\end{footnotetext}\ignorespaces 
2C业务渠道分为线下渠道、电商渠道、创新渠道。
\begin{itemize}
\item {} 
线下渠道:按经营主体可分为直营门店、加盟门店,按门店位置可分为商场专柜、社区店、奥莱店、工厂折扣店、街边专卖店。

\item {} 
电商渠道:按平台可分为天猫、淘宝、京东、苏宁、唯品会、亚马逊、拼多多、有赞、微盟,及自建官方商城微商城、小程序、App等。

\item {} 
创新渠道:分为社交电商和内容电商。社交电商如拼多多、抖音、今日头条、环球、微信、微博等,内容电商如小红书、礼物说等。

\end{itemize}


\paragraph{价值 13\sphinxfootnotemark[104]}
\label{\detokenize{chapter_introduction/2C:id5}}%
\begin{footnotetext}[104]\sphinxAtStartFootnote
\sphinxnolinkurl{https://www.jianshu.com/p/b159b89df3f8}
%
\end{footnotetext}\ignorespaces 
通过获取更多用户的注意力,故C端产品的用户行为涉及核心在于\sphinxstylestrong{塑造用户行为},C端产品设计师通过交互设计、视觉设计的方式影响用户注意力,终极目标是使用户的使用行为尽在掌控之中,而不是产品去辅助用户行为;


\paragraph{用户 6\sphinxfootnotemark[105]}
\label{\detokenize{chapter_introduction/2C:id6}}%
\begin{footnotetext}[105]\sphinxAtStartFootnote
\sphinxnolinkurl{http://www.pmtalk.club/\#/article/detail/6375}
%
\end{footnotetext}\ignorespaces 
C端产品的用户比较感性;往往一款产品可以满足某个需求,或者让自己兴奋,则会直接使用。

C端产品比较重视用户体验,往往以一两个核心功能来撬动用户,希望用户能够以最短路径和时间达到“兴奋”点;

角色分工:角色单一,用户角色高度集中

C端:我可以是用户


\paragraph{Buyer和User是同一个人}
\label{\detokenize{chapter_introduction/2C:buyeruser}}
对于ToC产品来讲,
Buyer和User是同一个人,例如手游App,实际参与使用App的用户也是最终充值消费的用户,所以在设计ToC产品的时候,不太需要把每个功能和付费途径做拆解,只要保证用户在具体使用产品的过程中可以顺畅地进行消费即可。且用户带来客户


\paragraph{C端产品思维 11\sphinxfootnotemark[106]}
\label{\detokenize{chapter_introduction/2C:c-11}}%
\begin{footnotetext}[106]\sphinxAtStartFootnote
\sphinxnolinkurl{https://weread.qq.com/web/reader/40632860719ad5bb4060856k9a132c802349a1158154a83}
%
\end{footnotetext}\ignorespaces \begin{itemize}
\item {} 
用户思维:以用户为中心,认识产品的核心用户,明白运用人工智能是为了提升用户体验。

\item {} 
简约思维:不要在核心功能外画蛇添足,并不是所有功能加上人工智能就能成为好的功能。

\item {} 
极致思维:超越用户预期,人工智能可以赋予产品超越用户预期的功能。

\item {} 
迭代思维:小步快跑,人工智能有时不是那么完美,企业可以在小步快跑中达到目的。

\item {} 
社会化思维:用网络的方式完成分工与合作。许多需要模型学习的标注数据,同样可使用该方法来实现。

\item {} 
平台思维:建设开放、共享、共赢的平台,同时人工智能的能力输出实际上也能够为模型的不断进化提供支持。

\item {} 
跨界思维:要有大眼光,用多角度、多视野看待问题和提出解决方案,并结合人工智能的多项手段,提出综合解决方案。

\end{itemize}


\paragraph{C端产品的设计原则}
\label{\detokenize{chapter_introduction/2C:c}}
C端产品的设计十分注重细节,产品在方案设计过程中,在初期主要思考的是主体框架和流程,在后期主要注重产品细节的设计,以下是10大经典的产品设计原则,可供AI产品经理参考。
\begin{enumerate}
\sphinxsetlistlabels{\arabic}{enumi}{enumii}{}{.}%
\item {} 
状态可见或可知原则用户的任何操作,单击、滑动、按下按钮、语音唤醒等,产品应即时给出反馈。“即时”是指响应时间小于用户能忍受的等待时间,这个反馈可以是页面形式也可以是语音提示。

\item {} 
环境贴切原则产品的一切表现或表述,应该尽可能贴近用户所处的环境(年龄、学历、文化、时代背景)。系统所使用的词、短语应该是用户熟悉的概念,而不是系统术语。如个人助理中,机器与人的交谈会非常注重人性化的回复。

\item {} 
用户控制性与自由度原则不要替用户做决定,为了避免用户的误用、误碰,产品应该可以撤销或重复操作,在人工智能产品中,则可以用语音提示的方式告知用户。

\item {} 
一致性原则一致性不仅指产品中的用语、功能、操作、界面的一致,还包括产品应遵循行业规则,如智能音箱的唤醒就是一个通用操作。

\item {} 
防错原则在用户选择动作发生之前,就要防止用户有容易混淆或者错误的选择,比出现错误信息提示更好的是用更好的设计来防止此类问题发生,在语音交互系统中回声的消除、误唤起就要遵守防错原则。

\item {} 
易获取原则尽量减少用户对操作目标的记忆负荷,无论是操作动作还是选项都应该是可见的,而系统的使用说明应该是可见的或者是容易获取的。

\item {} 
灵活高效原则中级用户的数量远高于初级和高级用户的数量,这意味着企业需要为大多数用户设计,不要低估、也不可轻视,要保持灵活高效的产品设计原则。

\item {} 
审美与简约设计如果用户使用产品的习惯是浏览产品,一般动作不是读、不是看,而是浏览,那么界面就需要突出重点,弱化和剔除无关信息。如果用户是在一个无法看只能听的环境中使用产品,则需要设计简单的语音沟通功能。

\item {} 
容错原则容错指的是允许用户犯错,错误信息应该用语音表达,较准确地指出问题所在,并且提出一个用户可进行实际操作的解决方案。

\item {} 
人性化帮助原则系统帮助性提示包含①无须提示;②一次性提示;③常驻提示;④帮助文档;提示的方式可以是文本、图片和语音,如果系统不使用文档是最好的。系统提供的任何信息应当是容易去搜索的,并且专注于用户的任务,列出了具体的步骤。

\end{enumerate}


\paragraph{增长}
\label{\detokenize{chapter_introduction/2C:id7}}
根据具体的立足点,还可以把增长大概分为这样几类:用户增长、流量增长、交易增长。

用户增长,已经发展出了完整的方法论,例如AARRR模型:获取(Acquisition)\sphinxhyphen{}
激活(Activation)\sphinxhyphen{}(留存(Retention)\sphinxhyphen{} 收入(Revenue)\sphinxhyphen{}推荐(Refer)。


\paragraph{风口论}
\label{\detokenize{chapter_introduction/2C:id8}}
针对2C产品,遇到了市场新趋势,市场上竞品服务模式需要统一更改等
\sphinxhref{http://www.woshipm.com/pmd/1792966.html}{4}%
\begin{footnote}[107]\sphinxAtStartFootnote
\sphinxnolinkurl{http://www.woshipm.com/pmd/1792966.html}
%
\end{footnote}

大家平时讨论最多的都是 to C 互联网,听到最多的一个词是「风口」。为什么
to C 那么在意风口?因为 to C 强调创新和需求体量。

to C
爆发通常靠两点:更好地解决需求、创造新需求。这个过程需要不断试错,费时费力费钱。相比之下更聪明的做法肯定是抄作业、抢风口。

既然是抢风口,比的就是谁快。什么鸡巴精益创业、敏捷开发、弹性架构、人月神话,只要业务能跑起来、让运营去做增长,管你是
PHP、Python 还是易语言写出来的代码,能 Run
就行。而且初期系统挂的越多越好,挂的多说明你业务增长快,说明你火爆。越挂越有人想注册,去投资人那这理由还能加钱。

在这种氛围的长期熏陶下,to C
产品人越发重视细节、重视核心想法的表达、越发去抓大放小、越发忽略系统的顶层架构和长远战略。

另外由于 to C
病毒传播的可行性强,产品人会觉得只要发点优惠券烧钱、广告轰炸烧钱、做足微信传播,用户自然就能指数增长。当他们涉足
to B 领域时,发现这些套路根本不 Work。


\subparagraph{心态 15\sphinxfootnotemark[108]}
\label{\detokenize{chapter_introduction/2C:id9}}%
\begin{footnotetext}[108]\sphinxAtStartFootnote
\sphinxnolinkurl{https://www.zhihu.com/pub/reader/119583028/chapter/1057335985485971456}
%
\end{footnotetext}\ignorespaces 
记住,风口上能飞的从来不是猪,猪即便在风口上飞起来了,也会在后面的寒冬中冻死。寒冬虽然会过去,但春天不一定会暖,即便暖流来临,甚至成为热潮,谁能担保,下一轮寒冬,不会在资本游戏里来得更快呢?

资本带来的,除了红海,还有血海。切莫为利红了眼,为益杀红眼,在海中迷了眼。


\paragraph{竞争激烈后的马太效应}
\label{\detokenize{chapter_introduction/2C:id10}}
C端的平台产品显著区别于B端的是:“任一用户对平台的议价能力都很弱,单个用户的离开对业务的影响很小,并且C端平台规模有较强的马太效应(供需趋向于集中化),规模本身就是护城河。”\sphinxhref{http://dadaghp.com/index/index/article\_detail/id/658.html}{26}%
\begin{footnote}[109]\sphinxAtStartFootnote
\sphinxnolinkurl{http://dadaghp.com/index/index/article\_detail/id/658.html}
%
\end{footnote}

相对于以业务为主线的B端产品,C端产品的价值竞争更激烈,因为市面上可选择的产品太多了。例如,用微信支付还是用支付宝支付,本质上解决的都是支付问题,而在线支付就是产品的价值。

最后的胜出者坐拥千万或上亿用户,自然是产品经理的成功典范,但是,我们不能只看到幸存者的风光,当初跟滴滴竞争做打车业务的团队有三百个曾经的团购市场也被称为“百团大战”,其他几百个团队现在到哪儿去了?选择做to
C
产品加入生死战场,跟团队一荣俱荣一损俱损,大部分产品经理注定失败,不论你的专业能力优秀与否。\sphinxhref{https://www.yinxiang.com/everhub/note/b60b7f01-4a91-473d-82a1-40fc5aa25734}{16}%
\begin{footnote}[110]\sphinxAtStartFootnote
\sphinxnolinkurl{https://www.yinxiang.com/everhub/note/b60b7f01-4a91-473d-82a1-40fc5aa25734}
%
\end{footnote}


\paragraph{2C思路}
\label{\detokenize{chapter_introduction/2C:id11}}\begin{enumerate}
\sphinxsetlistlabels{\arabic}{enumi}{enumii}{}{.}%
\item {} 
做这个产品的目标是什么

\item {} 
要实现这个目的,需要有哪些模块

\item {} 
要做好每个模块,需要做好那些事

\item {} 
要做好这些事,应关注哪些指标

\item {} 
要完成这些指标,产品及运营的重点是什么\sphinxhref{https://www.bilibili.com/video/BV19a4y1s7aN?from=search\&seid=11977543152973696126}{23}%
\begin{footnote}[111]\sphinxAtStartFootnote
\sphinxnolinkurl{https://www.bilibili.com/video/BV19a4y1s7aN?from=search\&seid=11977543152973696126}
%
\end{footnote}

\end{enumerate}

\begin{figure}[H]
\centering
\capstart

\noindent\sphinxincludegraphics{{2c_idea}.png}
\caption{2C思路示例}\label{\detokenize{chapter_introduction/2C:id28}}\end{figure}


\paragraph{过程 9\sphinxfootnotemark[112]}
\label{\detokenize{chapter_introduction/2C:id12}}%
\begin{footnotetext}[112]\sphinxAtStartFootnote
\sphinxnolinkurl{https://zhiya360.com/50903.html}
%
\end{footnotetext}\ignorespaces 
C端产品生命周期通常包含:需求调研、竞品分析,产品规划,产品设计、跟进开发、测试上线、冷启动期、运营推广,迭代优化等阶段。


\paragraph{产品路标规划:产品生命周期法(2C产品)4\sphinxfootnotemark[113]}
\label{\detokenize{chapter_introduction/2C:c-4}}%
\begin{footnotetext}[113]\sphinxAtStartFootnote
\sphinxnolinkurl{http://www.woshipm.com/pmd/1792966.html}
%
\end{footnotetext}\ignorespaces 
\sphinxstylestrong{产品生命周期}:2C产品一般会经过引入期、成长期、成熟期和衰退期。
\begin{center}\sphinxincludegraphics{{product_life}.png}\end{center}

\sphinxstylestrong{产品生命周期法则}是按照产品不同的生命周期目标来制定产品规划。


\subparagraph{引入期}
\label{\detokenize{chapter_introduction/2C:id13}}
引入期,即产品MVP阶段,此阶段要用最小成本快速验证产品思路在目标用户群中的接收度,减少产品走错路的风险。在这个阶段产品规划时,就要考虑如何让目标用户快速了解和使用我们的产品,如何找到种子用户,如何快速获得用户的反馈,得到反馈后如何快速根据这些反馈进行产品迭代,如何初步的推广产品等,结合这些问题去规划该阶段的产品规划。


\subparagraph{找种子用户}
\label{\detokenize{chapter_introduction/2C:id14}}

\subparagraph{冷启动}
\label{\detokenize{chapter_introduction/2C:id15}}
\sphinxhref{https://zhuanlan.zhihu.com/lengqidong}{冷启动}%
\begin{footnote}[114]\sphinxAtStartFootnote
\sphinxnolinkurl{https://zhuanlan.zhihu.com/lengqidong}
%
\end{footnote}通俗地说是指不通过大规模的市场推广,而是通过优质的内容或者熟人口碑传播进行产品启动的方法。冷启动可以有效地降低项目风险,但是启动速度比较慢。

冷启动的典型例子是知乎。知乎最开始就是周源凭借自己在互联网行业的人脉,以向专业人士发\sphinxhref{https://www.zhihu.com/topic/19613081/hot}{邀请码}%
\begin{footnote}[115]\sphinxAtStartFootnote
\sphinxnolinkurl{https://www.zhihu.com/topic/19613081/hot}
%
\end{footnote}的形式邀请用户进行注册的,如李开复、徐小平、周鸿祎等人都是知乎的早期用户。这些人在知识的广博性及专业性上都远胜于普通用户,这与知乎“高质量知识分享社区”的定位吻合。反过来,这些人的站台,也为知乎后续长远的发展奠定了基础。知乎在2013
年才开放用户注册。


\subparagraph{热启动}
\label{\detokenize{chapter_introduction/2C:id16}}
热启动,顾名思义,就是公司通过大量的资源(包含人力、资金等)投入让产品迅速启动,实现用户的爆发式增长,一般被大型公司采用。

热启动的典型例子是QQ 系的产品,如QQ 空间、QQ 邮箱等都是以QQ
为土壤迅速发展起来的。\sphinxhref{https://weread.qq.com/web/reader/8d232b60721a488e8d21e54k65132ca01b6512bd43d90e3}{7}%
\begin{footnote}[116]\sphinxAtStartFootnote
\sphinxnolinkurl{https://weread.qq.com/web/reader/8d232b60721a488e8d21e54k65132ca01b6512bd43d90e3}
%
\end{footnote}


\subparagraph{成长期}
\label{\detokenize{chapter_introduction/2C:id17}}
在产品的成长期,我们更关心的会是拉新(补贴、活动、邀请)和促进活跃,核心用户群要快速稳定地增长,同时也要保证一定的留存率。此时,产品规划的目标就是要考虑通过怎样的策略去实现上述目标,比如产品功能的优化、运营推广、产品性能优化(技术手段)等。产品优化:关注用户在每个核心页面的访问时长、核心页面的转化率及用户使用路径,不断提升产品用户体验;用户拉新和留存:每日新增用户数、次日留存率、7
日留存率、DAU(Daily Active User,日活跃用户数)、MAU(Monthly Active
User,月活跃用户数);推广:推广渠道数据,筛选出投入产出比最高的推广渠道并持续投入


\subparagraph{成熟期}
\label{\detokenize{chapter_introduction/2C:id18}}
在成熟期,产品活跃用户的增长会很缓慢,因为此阶段出现有大量的竞争对手,目标用户已被市场覆盖,或者是产品的模式等原因。在此阶段,产品规划应该去关注如何提升用户的转化率、如何提高产品的盈利能力,如果是产品自身的模式原因,就要去改善现有产品的服务、模式以及运营策略等,进一步提升产品的活跃用户。重点观测的数据指标:老用户留存率、老用户流失速度、每日新增用户数、新用户增长速度。


\subparagraph{衰退期}
\label{\detokenize{chapter_introduction/2C:id19}}
处于衰退期的产品,其实能够起死回生的几率不大,除非它的产品经理是卓越的领袖。重点观测的数据指标:每日用户流失数、用户流失速度、挽回效果数据。此阶段的规划可以尝试去发掘产品的第二春或颠覆式创新。此时如果要放弃产品,就要做好产品退出市场的相关工作
\sphinxhref{https://www.zhihu.com/pub/reader/119967224/chapter/1284014013069127680}{17}%
\begin{footnote}[117]\sphinxAtStartFootnote
\sphinxnolinkurl{https://www.zhihu.com/pub/reader/119967224/chapter/1284014013069127680}
%
\end{footnote}。

\begin{center}\sphinxincludegraphics{{product_diedai}.png}\end{center}
{\color{red}\bfseries{}|}各阶段应该做的{\color{red}\bfseries{}|}\sphinxhref{https://www.zhihu.com/question/267914037/answer/741359690}{24}%
\begin{footnote}[118]\sphinxAtStartFootnote
\sphinxnolinkurl{https://www.zhihu.com/question/267914037/answer/741359690}
%
\end{footnote}

17年后C端流量红利的消退,获客成本越来越高,流量获取越来越难,C端绝大多数流量都集中在腾讯、头条这种大厂,中小公司在C端的生存其实越来越难。\sphinxhref{http://www.woshipm.com/zhichang/4308504.html}{27}%
\begin{footnote}[119]\sphinxAtStartFootnote
\sphinxnolinkurl{http://www.woshipm.com/zhichang/4308504.html}
%
\end{footnote}


\subparagraph{创新}
\label{\detokenize{chapter_introduction/2C:id24}}
\begin{figure}[H]
\centering
\capstart

\noindent\sphinxincludegraphics{{innovation_life_cycle}.jpg}
\caption{创新\sphinxhref{https://www.linkedin.com/in/adnanboz?miniProfileUrn=urn\%3Ali\%3Afs\_miniProfile\%3AACoAAAFuSMgBFXwiCwojgL9ZKcAXUtKU0gz43I4\&lipi=urn\%3Ali\%3Apage\%3Ad\_flagship3\_feed\%3B0WkZWz7WSMSnTAYrp3M2Pw\%3D\%3D\&licu=urn\%3Ali\%3Acontrol\%3Ad\_flagship3\_feed-actor\_container\&lici=WZ8FA0AvHxl3fvZi6nTkaw\%3D\%3D}{22}\sphinxfootnotemark[120]}\label{\detokenize{chapter_introduction/2C:id29}}\end{figure}
%
\begin{footnotetext}[120]\sphinxAtStartFootnote
\sphinxnolinkurl{https://www.linkedin.com/in/adnanboz?miniProfileUrn=urn\%3Ali\%3Afs\_miniProfile\%3AACoAAAFuSMgBFXwiCwojgL9ZKcAXUtKU0gz43I4\&lipi=urn\%3Ali\%3Apage\%3Ad\_flagship3\_feed\%3B0WkZWz7WSMSnTAYrp3M2Pw\%3D\%3D\&licu=urn\%3Ali\%3Acontrol\%3Ad\_flagship3\_feed-actor\_container\&lici=WZ8FA0AvHxl3fvZi6nTkaw\%3D\%3D}
%
\end{footnotetext}\ignorespaces 

\paragraph{原型能力 5\sphinxfootnotemark[121]}
\label{\detokenize{chapter_introduction/2C:id25}}%
\begin{footnotetext}[121]\sphinxAtStartFootnote
\sphinxnolinkurl{http://www.woshipm.com/pmd/3755958.html}
%
\end{footnotetext}\ignorespaces 
C端的产品更重交互,所有对原型能力要求高一些,有的公司会要求产品画高保真设计图。


\paragraph{与产品价值相矛盾}
\label{\detokenize{chapter_introduction/2C:id26}}
C
端产品时常会遇到与产品价值相矛盾的情况,例如视频产品最核心的用户体验就是让用户不间断地看视频,但往往碍于公司生存压力,不得不在视频播放时插入广告金主的广告内容。


\paragraph{2C2B化 18\sphinxfootnotemark[122]}
\label{\detokenize{chapter_introduction/2C:c2b-18}}%
\begin{footnotetext}[122]\sphinxAtStartFootnote
\sphinxnolinkurl{http://www.changgpm.com/thread-153-1-1.html}
%
\end{footnotetext}\ignorespaces 
本来是针对C端的产品和服务,开始不断先通过B端进行推广,比如很多人已经发现,有些O2O服务与其不断打广告,还不如\sphinxstylestrong{搞定一个数万人大公司的HR},变成员工福利,瞬间得到推广。


\paragraph{2B VS 2C}
\label{\detokenize{chapter_introduction/2C:b-vs-2c}}\label{\detokenize{chapter_introduction/2C:id27}}
\begin{figure}[H]
\centering
\capstart

\noindent\sphinxincludegraphics{{2B_VS_2C}.png}
\caption{2B VS 2C}\label{\detokenize{chapter_introduction/2C:id30}}\end{figure}


\paragraph{AI PM直接合作 10\sphinxfootnotemark[123]}
\label{\detokenize{chapter_introduction/2C:ai-pm-10}}%
\begin{footnotetext}[123]\sphinxAtStartFootnote
\sphinxnolinkurl{https://www.oreilly.com/radar/practical-skills-for-the-ai-product-manager/}
%
\end{footnotetext}\ignorespaces 
产品经理更有可能\sphinxstylestrong{直接}与功能团队合作,做更多客户驱动的工作。因为他们正在打造一款将被大众消费的人工智能产品,所以有可能(甚至是可取的)优化以实现快速实验和迭代的准确性


\paragraph{AI作用 12\sphinxfootnotemark[124]}
\label{\detokenize{chapter_introduction/2C:ai-12}}%
\begin{footnotetext}[124]\sphinxAtStartFootnote
\sphinxnolinkurl{https://weread.qq.com/web/reader/0c032c9071dbddbc0c06459k70e32fb021170efdf2eca12}
%
\end{footnotetext}\ignorespaces 
对个人端消费级产品而言,人工智能的意义在于将人类本身的感官和技能进行了技术形态的延伸。消费级产品的人工智能应用点一般集中在三个方面:一是信息采集;二是协助判断;三是协助处理。

人工智能产品是通过信息采集获取大量的数据,然后通过对数据训练得出适用性模型,接着将个人的信息数据作为输入并通过模型给出答案,因此人工智能产品必须拥有数据采集的能力,以便于进行智能化的判断。从智能穿戴设备到智能家居,从推荐引擎到预测系统,均需要通过各种传感器及输入设备获取数据。

消费级产品可以更好地帮助人类感知外部和自身的信息,能够帮助人们更为高效、快捷地处理信息。这种处理表现为两个方向:一是对个体的信息量化;二是对信息的处理进化。人们在使用一些人工智能产品的时候会发现,这些人工智能产品表现得越来越了解自己:你爱看什么类型的新闻,客户端就会给你推什么类型的新闻;你爱吃什么类型的菜,客户端就会给你推送什么类型的餐馆。原因就是这些人工智能产品已经通过采集及算法模型得出了一个量化数据的“你”。不仅如此,一些医疗保健和运动健身领域的产品通过心率、步频、身高、体重、速度、血压等信息的检测将个人信息更细致地量化。除自身信息的量化外,人们通过这些人工智能产品能够更好地感知和处理更多的数据,令信息处理能力大幅度提升。


\paragraph{常见AI产品 19\sphinxfootnotemark[125]}
\label{\detokenize{chapter_introduction/2C:ai-19}}%
\begin{footnotetext}[125]\sphinxAtStartFootnote
\sphinxnolinkurl{https://www.pianshen.com/article/2712685407/}
%
\end{footnotetext}\ignorespaces 
依托AI技术为主业务产品的风控、推广、用户运营等常规业务部门赋能,提高企业效率。\sphinxhref{https://www.zhihu.com/people/zhu-guan-jin-ming/answers/by\_votes}{21}%
\begin{footnote}[126]\sphinxAtStartFootnote
\sphinxnolinkurl{https://www.zhihu.com/people/zhu-guan-jin-ming/answers/by\_votes}
%
\end{footnote}
\begin{enumerate}
\sphinxsetlistlabels{\arabic}{enumi}{enumii}{}{.}%
\item {} 
智能家居:智能家庭机器人、智能音箱、智能手表等等,成为智能家居控制中心(小米、360、京东)

\item {} 
智能语音助手:Siri、Cortana、Google Assistant、度秘、Bixby

\item {} 
其他:各类Bot、AR/VR、无人机

\item {} 
贴近厨师的AI软体厨师产品!\sphinxhref{https://zhuanlan.zhihu.com/p/294947630}{20}%
\begin{footnote}[127]\sphinxAtStartFootnote
\sphinxnolinkurl{https://zhuanlan.zhihu.com/p/294947630}
%
\end{footnote}\sphinxurl{https://bridge.ai/kitchen/}

\item {} 
用户无感身份安全验证!\sphinxurl{https://unify.id/}

\item {} 
帮用户撰写、修改、提升简历,增加面试机会,提升面试成功率!\sphinxurl{https://mosaic.ai/}

\item {} 
用AI增强决策能力!\sphinxurl{https://notion.ai/}

\item {} 
任何时间任何场景找到自己喜欢的活动!\sphinxurl{https://www.robby.ai/about/}

\end{enumerate}


\subsubsection{平台型}
\label{\detokenize{chapter_introduction/platform:id1}}\label{\detokenize{chapter_introduction/platform::doc}}

\paragraph{PM的职能 1\sphinxfootnotemark[128]}
\label{\detokenize{chapter_introduction/platform:pm-1}}%
\begin{footnotetext}[128]\sphinxAtStartFootnote
\sphinxnolinkurl{https://www.iamxiarui.com/?p=1369}
%
\end{footnotetext}\ignorespaces \begin{itemize}
\item {} 
收集需求确定方向,联想真实业务场景

\item {} 
定义并验证产品中各个业务或非业务逻辑

\item {} 
抽象产品的业务场景,不仅满足当下需求,更要预见未来

\item {} 
规划并验证具体方案,保证产品的实际落地

\end{itemize}


\subparagraph{抽象}
\label{\detokenize{chapter_introduction/platform:id2}}
定期复盘,复盘的一个很重要的事情就是把自己做的事情、做的业务向上抽象一个层次。你要看一看能否总结出某种规律、能否看清楚自己到底在干一件什么事、到底满足的是人类的哪一个非常基本的需求、现在提供的方案是不是最好的、要想满足这个需求还能找到哪些方案、是不是该对曾经的好方案进行迭代。

滴滴要解决的是用户出行的问题,所以才会有不同层次的解决方案,有顺风车、专车、出租车。对于出差或旅游的人来说,“住酒店”并不是一个本质的需求,而“在外面的时候有地方住”才是。所以,爱彼迎(Airbnb)就去做了共享房间的生意,更好地解决了外出人的“住”的痛点,所以成功了。

再举个例子,美国网飞公司(Netflix)的创始人哈斯廷斯的思路非常清楚,在原来出租录像带的时候他就知道用户的需求不是要租录像带,真正的需求是看电影,其实不在意是出租录像带、DVD,还是使用在线视频的模式,所以网飞公司马上就转型做了视频内容点播业务。


\paragraph{互联网上半场的平台}
\label{\detokenize{chapter_introduction/platform:id3}}
互联网上半场的商业模式,主要是以满足C端用户需求为核心的平台产品,包括:社交平台、电商平台、团购平台、出行平台、内容平台。平台通过链接人与人,内容与人,商家与人,聚合大规模的流量。

\begin{figure}[H]
\centering
\capstart

\noindent\sphinxincludegraphics{{now_platform}.png}
\caption{平台产品示例}\label{\detokenize{chapter_introduction/platform:id12}}\end{figure}


\paragraph{平台型产品经理}
\label{\detokenize{chapter_introduction/platform:id4}}
对于需要产出平台化解决方案的平台型产品经理而言,通常需要从C端用户型产品经理及B端商业化产品经理处得到已经经过初步加工的用户或客户需求,明确平台设计的目的和意义。当然,部分公司中也存在平台型产品经理兼任用户型或商业型产品经理角色的情况,由平台型产品经理亲自分析用户需求或调研客户需求,并最终给出问题的平台化解决方案。平台型产品经理需要根据平台设计的的与意义,将得到的用户或客户需求进一步对需求目标进行拆解,明确这套平台化解决方案中需要求解哪些问题,并最终将具体问题提交给各个模块相应的技术人员


\subparagraph{不全面的需求}
\label{\detokenize{chapter_introduction/platform:id5}}
不全面的需求在交付技术人员开发时通常不存在问题,具体而明确。但当开发完成准备上线,甚至上线之后,才发现有些业务场景\sphinxstylestrong{根本未考虑,出现逻辑覆盖不全、缺少逻辑甚至缺少模块}的情况。对于平台型产品经理来说,平台部分有更多的极端情况和隐藏逻辑需要处理,且平台的设计需要有足够的预见性,需要在具体产品需求还不存在的情况下,提前做出适应性设计。如果出现一个产品需求就要对平台做一次修改,那么平台就失去了其解决通用性问题的功能,没有为业务发展提供长期支撑的意义于平台竞品相较于2C产品的竞品要少得多,且很多公司的后台并不开放,因此更需要平台型产品经理在信息有限的情况下做出正确的决策,要比普通用户想得更多,更全面地思考问题。举例子,机器学习模型的优秀表现离不开大量高质量的训练数据,与之相对的,离线标注模块也是任何一个基于机器学习的业务平台所不可或缺的重要组成部分,但对终端用户而言,薮据标注这个过程并不可见。如果平台型产品经理只具备从普通用户角度看问题的能力,停留在问题的表面,那么势必无法给出合格的平台设计方案。
\sphinxhref{https://cread.jd.com/read/startRead.action?bookId=30457741\&readType=1}{6}%
\begin{footnote}[129]\sphinxAtStartFootnote
\sphinxnolinkurl{https://cread.jd.com/read/startRead.action?bookId=30457741\&readType=1}
%
\end{footnote}


\subparagraph{AI平台}
\label{\detokenize{chapter_introduction/platform:ai}}
AI平台=AI SAAS+(PAAS)+(IAAS)

AI和数据可同为一个中台、AI平台可视为业务中台一部分、AI平台整合进技术中台或后台等。规模较小、资源有限的企业通常会选择使用第三方AI平台对业务进行服务,而非自建AI平台。

\begin{figure}[H]
\centering
\capstart

\noindent\sphinxincludegraphics{{AI_train_platform}.png}
\caption{AI商品模型训练平台}\label{\detokenize{chapter_introduction/platform:id13}}\end{figure}

常见问题:\sphinxhref{http://www.cet.com.cn/wzsy/cyzx/2741907.shtml}{13}%
\begin{footnote}[130]\sphinxAtStartFootnote
\sphinxnolinkurl{http://www.cet.com.cn/wzsy/cyzx/2741907.shtml}
%
\end{footnote}
\begin{enumerate}
\sphinxsetlistlabels{\arabic}{enumi}{enumii}{}{.}%
\item {} 
模型环境构建难,从模型到服务转化慢。

\item {} 
多个模型共用一套环境,但模型需要独立更新,对环境的维护需格外谨慎,变更风险巨大。

\item {} 
模型服务交付周期长,工程环节耗时长,人力投入大。

\item {} 
模型服务回退慢,模型发布后,如发现有问题尝试退回到上一个版本需要数分钟,乃至数小时。

\item {} 
模型服务扩容慢,面对突发流量时,响应延迟。

\end{enumerate}


\subparagraph{AI平台型产品经理 3\sphinxfootnotemark[131]}
\label{\detokenize{chapter_introduction/platform:ai-3}}%
\begin{footnotetext}[131]\sphinxAtStartFootnote
\sphinxnolinkurl{https://www.zhihu.com/question/57815929}
%
\end{footnotetext}\ignorespaces 
单独将AI平台产品经理列出来,是因为AI平台产品经理不局限于某一个AI研究领域,更专注于打造一个通用的机器学习平台,这里也可以细分为深度学习平台、强化学习平台等。这个平台可以用来各种AI应用场景的建模,提升科学家们建模的效率以及一定程度上降低建模的门槛,使得普通的产品\&运营等都可以在平台上进行简单地建模。国内以百度的Paddle\sphinxhyphen{}Paddle,第四范式的先知,阿里的PAI平台为主流的机器学习平台。AI平台产品经理可以说是AI产品经理中对技术能力要求最高的职位了

需要对整个机器学习建模流程十分熟悉,比如数据处理、模型构建、特征工程、效果评估等。同时还要对工程化、大数据处理、算力资源管理等有一定了解。如果没有一定的机器学习背景是很难成为一名合格的AI平台产品经理。

\begin{figure}[H]
\centering
\capstart

\noindent\sphinxincludegraphics{{AI_platform_upgrade}.png}
\caption{AI平台的进化}\label{\detokenize{chapter_introduction/platform:id14}}\end{figure}


\paragraph{AI中台}
\label{\detokenize{chapter_introduction/platform:id6}}
对AI能力重视的公司会独立建设AI中台部门,做好对公司级的AI能力打磨建设好平台型的产品,梳理好对应的算法运营/数据运营,能快速承接业务。但是最担心的是中台部门不中台,急于出成绩,会独立外探业务,与业务部门形成竞争关系,最终其实还是烟囱式的发展模式,“形实不一致”,基础建设与业务发展会逐渐脱节。\sphinxhref{https://www.zhihu.com/question/346379206}{4}%
\begin{footnote}[132]\sphinxAtStartFootnote
\sphinxnolinkurl{https://www.zhihu.com/question/346379206}
%
\end{footnote}

\sphinxurl{https://medium.com/@unbug/how-to-build-an-ai-platform-2493d57c9e8}

要做一个AI中台,当我们的产品经理过AI中台的需求,上面的4行我们产品AI产品经理需要定义哪一行?需要说明哪一行?需要撰写的PRD文档又是哪一行?需要跨团队沟通的又是哪一行?\sphinxhref{https://www.shangyexinzhi.com/article/2251387.html}{8}%
\begin{footnote}[133]\sphinxAtStartFootnote
\sphinxnolinkurl{https://www.shangyexinzhi.com/article/2251387.html}
%
\end{footnote}

\sphinxurl{http://www.woshipm.com/pmd/4146522.html}


\paragraph{用例}
\label{\detokenize{chapter_introduction/platform:id7}}
\begin{figure}[H]
\centering
\capstart

\noindent\sphinxincludegraphics{{platform_flow_chart}.jpg}
\caption{几个核心平台系统之间的用例图\sphinxhref{https://zhuanlan.zhihu.com/p/269732570}{5}\sphinxfootnotemark[134]}\label{\detokenize{chapter_introduction/platform:id15}}\end{figure}
%
\begin{footnotetext}[134]\sphinxAtStartFootnote
\sphinxnolinkurl{https://zhuanlan.zhihu.com/p/269732570}
%
\end{footnotetext}\ignorespaces 
\begin{figure}[H]
\centering
\capstart

\noindent\sphinxincludegraphics{{platform_function}.jpg}
\caption{几个核心平台系统之间的产品功能架构}\label{\detokenize{chapter_introduction/platform:id16}}\end{figure}

\sphinxstylestrong{必需的功能点:}

产品服务中心、解决方案中心、系统配置管理、管理控制台、账户管理、权限管理、计量计费管理、离线数据集成、脚本清洗、数据质量管理、运维信息统计、元数据管理、数据检索、任务运维管理、数据预标注、Web
IDE、自定义算法、模型评估管理、资源管理、算法模板管理、代码构建、CPU
容器部署、部署环境管理、模型部署、发布流程管理、GPU 容器服务、CPU
容器服务、数据卷管理、对外资源管理服务、资源配额管理、资源调度策略管理。

\sphinxstylestrong{附加功能点:}

人工智能服务市场、文档支撑中心、工单中心、消息中心、内容发布管理、实时数据集成、SDK
清洗、数据血缘管理、计费策略管理、结算管理、数据类目管理、模型压缩优先、公共模型库、训练任务管理、私有模型库、模型评估管理、数据接入管理、系统配置管理、模型版本管理、GPU
容器部署、服务灰度发布、服务效果跟踪。

\sphinxstylestrong{较有难度的功能点:}

图像标注、任务管理、数据集管理、文本标注、质检管理、源数据管理、语音标注、团队管理、视频标注、实验管理、机器学习算法、数据接入管理、画布管理、私有镜像仓库、组件管理、运维管理、服务上下线。


\paragraph{平台设计通用工作流程}
\label{\detokenize{chapter_introduction/platform:id8}}
从更具体的角度描述一个结合了平台设计的业务需求从最初想法到最终落地的通用化流程。
\sphinxhref{http://reader.epubee.com/books/mobile/41/41f170eb06525e985bbddd6eae13589d/text00006.html}{7}%
\begin{footnote}[135]\sphinxAtStartFootnote
\sphinxnolinkurl{http://reader.epubee.com/books/mobile/41/41f170eb06525e985bbddd6eae13589d/text00006.html}
%
\end{footnote}

这个过程通常包含8个阶段,分别为:方向阶段、联想阶段、定义阶段、推演阶段、抽象阶段、规划阶段、验证阶段和落地阶段。

其中,方向阶段和联想阶段通常由用户型产品经理及商业型产品经理完成。在方向阶段,用户型产品经理及商业型产品经理分别通过对用户行为数据和与客户商谈结果的分析,初步抽象得到用户需求与客户需求,然后将原始的用户需求、客户需求转换为具体的平台设计目标并交付给平台型产品经理。联想阶段一般与方向阶段同步进行,指用户型产品经理及商业型产品经理不仅需要交付平台设计目标,还需要将与该设计目标相对应的典型业务场景一并提供给平台型产品经理,联想出的真实业务场景越多,提供的信息越充分,后续阶段就越不容易出现解决方案与原设计目标不符的情况。这些真实业务场景将成为后续推演阶段进行系统基本可行性验证的素材。

在定义阶段、推演阶段通常需要平台型产品经理与技术人员通力合作。定义阶段的主要任务是结合前两个阶段所得到的平台设计目标与具体业务场景,明确一个怎样的系统能够承担这样的任务。另外,需要回答前文提到的一些系统设计中的基本问题,比如系统的模块构成、各个功能模块的职责、每个模块的输入和输出、模块间的上下位关系及系统调用顺序。同时,为了保证通用性,还需要对业务逻辑和非业务逻辑进行划分,对业务逻辑做可配置化设计,并将非业务逻辑固化在平台内部,对平台使用者不可见,降低其理解成本。推演阶段紧跟在定义阶段之后,其主要任务是利用联想阶段所得到的具体业务场景,结合定义阶段产出的系统设计方案进行推演,假定系统开发已经完成,应当如何对业务逻辑进行配置,及其与固化的非业务逻辑相结合之后,是否能够满足业务场景的需求。另外,还需要对整个系统的信息通路完整、各模块职责履行的一致性等进行检查。推演阶段完成后,意味着系统框架已经确定。

抽象阶段大多需要平台型产品经理独立完成。系统框架的确定只是平台设计方案确定的第一步,虽然业务逻辑与非业务逻辑的划分已经完成,但业务逻辑究竟如何以平台产品的形式呈现出来是平台型产品经理必须解决的关键问题。结合了平台设计的业务需求要解决的通常不是一个问题,而是一类问题,甚至是多类问题相互交织。平台型产品经理需要对这些相互交织的问题抽丝剥茧,抽象出一类或多类问题中的基本影响因素。2.1节提到,产品经理需要从业务发展的角度协助技术人员保证系统的可拓展性,以避免未来业务需求发生变化时,或者随业务发展出现了需要解决的新问题时,不得不对已有的系统设计进行修改,从而带来大量不必要的工作。利用基本影响因素的排列组合来列举业务场景所有可能的方式能够有效避免遗漏,使得产品经理能够更好地协助技术人员设计出拓展性较好的系统设计方案。同时,平台型产品经理也需要以业务需求的基本影响因素为基础完成平台原型设计,这个平台原型不仅需要解决当下的业务需求,还需要为未来可预见的业务需求提前做好准备。

规划阶段需要平台型产品经理与用户型产品经理或商业型产品经理协作完成。抽象阶段完成后,得到的实际上并不是一个可以马上执行的实施方案,而是一个长期发展方向。规划阶段要求平台型产品经理与用户型产品经理及商业型产品经理沟通,根据各类业务需求的优先级,确定平台的一期落地方案,并根据业务发展趋势对后续平台的迭代方案进行初步规划。规划阶段实际上就是在做需求的整理与细化,平衡技术实现与业务需求,结合业务未来的发展趋势,形成一套长期的迭代优化计划。

验证阶段与落地阶段是平台化解决方案实施的最后两个阶段,主要需要平台型产品经理与技术人员参与。在验证阶段,平台型产品经理需要就规划阶段产出的一期落地方案及后续的平台迭代优化计划向技术人员核对,保证方案的可行性。落地阶段与传统的产品需求落地的过程类似,需要进行任务拆解、任务分配、时间评估、确定上线计划等。

结合了平台设计的业务需求解决方案设计,其核心在于定义与抽象,这两个阶段是其他类型的需求解决方案中相对较少涉及的。定义阶段明确了系统框架,决定了系统未来发展的可能性;而抽象阶段是决定系统能否如预期般发挥价值的最重要的阶段,也是将平台设计目标这个抽象实体转换为可行方案过程中的重要转折点。


\subparagraph{国内外相关的AI平台有:}
\label{\detokenize{chapter_introduction/platform:id9}}
国内:
\begin{enumerate}
\sphinxsetlistlabels{\arabic}{enumi}{enumii}{}{.}%
\item {} 
华为ModelArts

\item {} 
阿里云 PAI

\item {} 
百度 Paddle Paddle

\item {} 
腾讯 DI\sphinxhyphen{}X深度学习平台

\item {} 
金山云 人工智能平台

\item {} 
qingcloud 人工智能平台

\item {} 
京东 JDAINeuFoundry

\item {} 
小米Cloud\sphinxhyphen{}ml平台

\end{enumerate}

国外:
\begin{enumerate}
\sphinxsetlistlabels{\arabic}{enumi}{enumii}{}{.}%
\item {} 
Microsoft Azure Machine Learning

\item {} 
AWS Machine Learning

\item {} 
Google Cloud Platform

\end{enumerate}

\begin{figure}[H]
\centering
\capstart

\noindent\sphinxincludegraphics{{AI_platform_k8s}.png}
\caption{机器学习平台架构\sphinxhyphen{}K8S\sphinxhref{https://segmentfault.com/a/1190000019215007}{10}\sphinxfootnotemark[136]}\label{\detokenize{chapter_introduction/platform:id17}}\end{figure}
%
\begin{footnotetext}[136]\sphinxAtStartFootnote
\sphinxnolinkurl{https://segmentfault.com/a/1190000019215007}
%
\end{footnotetext}\ignorespaces 
Gartner 最新发布了两份 AI 魔力象限《Magic Quadrant for Data Science and
Machine Learning Platforms(数据科学与机器学习平台)》(下称 「DSML
报告」)和《Magic Quadrant for Cloud AI Developer Services(云 AI
开发者服务)》(下称「CAIDS」
报告),对全球厂商进行了年度评估。这两份报告可以看作是 AI
工程化领域的盘点,给予希望选择正确的 DSML 和 CAIDS 解决方案,并提升 AI
生产力的企业以权威的参考。


\paragraph{平台云原生化}
\label{\detokenize{chapter_introduction/platform:id10}}
AI 工程化落地的首个基础能力就是平台云原生化。其实 AI
平台的构建有很多实现方法,但云原生是目前最普适的方法。因为云原生门槛不高,没有具体限制技术选型,尤其是它所倡导的开放、弹性和生态等原则可以迅速拉低
AI 平台的实现门槛。

开放意味着需要把 OpenAPI
放到产品的第一优先级来考虑,支持用户和其他云产品通过 OpenAPI
访问产品所有功能,可以被第二方和第三方厂商工具自由集成;同时能够擅于调用其他云上产品来构建自己的服务,比如云上数据库
RDS、云日志服务 SLS 等。

弹性是在设计之初就要设想产品的规模大小,物理资源尽量不要自建,充分利用云的弹性。

生态有两层含义,一是与业界开源社区保持合作,尽量不要重复造轮子和发明新规范,二是开放的内容生态,能够吸引个人开发者和企业共同建设
AI 平台,繁荣生态。


\paragraph{用户分类}
\label{\detokenize{chapter_introduction/platform:id11}}
建模人群大致可以分为两类:

1.数据科学家。计算机或人工智能专业的科班生,他们的特点是算法和工程能力较强,业务能力较弱;这类同学更喜欢编程式建模,因为编程式建模的自由度可以让他们的专业能力充分发挥出来。

2.行业建模人员。其他领域的同学,如金融领域的同学,他们的特点是对数据和业务的理解深,但对算法和工程的理解浅,这类用户比较排斥编程,更喜欢页面化建模产品,希望产品可以给予他们更多的AI能力的帮助,让其发挥出\sphinxstylestrong{自身的业务能力}。\sphinxhref{https://tech.antfin.com/community/articles/98}{12}%
\begin{footnote}[137]\sphinxAtStartFootnote
\sphinxnolinkurl{https://tech.antfin.com/community/articles/98}
%
\end{footnote}


\subsubsection{财务规划}
\label{\detokenize{chapter_introduction/money:id1}}\label{\detokenize{chapter_introduction/money::doc}}

\paragraph{产品财务规划 5\sphinxfootnotemark[138]}
\label{\detokenize{chapter_introduction/money:id2}}%
\begin{footnotetext}[138]\sphinxAtStartFootnote
\sphinxnolinkurl{http://www.woshipm.com/pmd/1792966.html}
%
\end{footnotetext}\ignorespaces 
产品财务规划是产品规划和商业模式的进一步落地,在进行产品的财务规划时,我们可以请求公司财务部门的协助,做好产品财务规划落地。

在财务规划阶段,我们要回答好如下问题,产品投入多大?成本如何?后续盈利模式如何?需要多久能够开始盈利?是否有进一步的可持续的盈利模式等。

在成本方面,要将资源分阶段投入,产品也要分迭代版本逐步退出,以降低企业的现金流压力。在产品规划中,加入产品各版本以及涉及到的资源投入,能够更好的说明产品成本信息。

在收益方面,除上述盈利模式等问题的考虑外,如果要将产品推出市场,还要考虑产品的定价策略。产品定价策略需要考虑市场本身的成熟度,客户关系积累程度以及产品本身的发展等多方面信息。


\paragraph{赚钱}
\label{\detokenize{chapter_introduction/money:id3}}
所谓“赚钱”,本质上即是:抓住他人欲望,满足他人需求,使之心甘情愿的为这份渴望而支出。

特斯联副总裁谢超告诉极客公园,在他看来,一旦研发开始投入,再想要掉转船头就很难了,「你一定要想好,未来的产出是什么」。\sphinxhref{https://tech.sina.com.cn/roll/2020-07-12/doc-iivhuipn2598506.shtml}{1}%
\begin{footnote}[139]\sphinxAtStartFootnote
\sphinxnolinkurl{https://tech.sina.com.cn/roll/2020-07-12/doc-iivhuipn2598506.shtml}
%
\end{footnote}

摆在特斯联面前的有两层选择,是做单品还是场景化解决方案,是做消费市场还是政府主导的公共事业市场。二乘二,一共有四条路可以走。做单品,可以做智能门锁等智能硬件;做场景化解决方案,则需要有对场景的深刻理解及部署能力。消费级市场和政府级市场的需求不同,企业的组织构建也就不同。

「我的原则是要用产品证明价值,让别人买单,而不是免费给别人用,这也是做为一个企业的本分」,服务金融客户的科技公司氪信的
CEO 朱明杰这样说。

金融机构一面服务广大 C
端用户,一面服务投资机构,一旦出错,是真金白银的损失,试错成本非常高。因此在金融行业,金融机构与科技公司正式签合同之前,一般双方要在实际业务场景中磨合一段时间。这段时间非常长,有的甚至长达一年,目的是为了充分验证竞标公司的技术实力,减少出错的可能。


\paragraph{赚钱几种人2\sphinxfootnotemark[140]}
\label{\detokenize{chapter_introduction/money:id4}}%
\begin{footnotetext}[140]\sphinxAtStartFootnote
\sphinxnolinkurl{https://www.sohu.com/a/409718794\_312708}
%
\end{footnotetext}\ignorespaces 
为什么2020年底还在神化AI的主力军基本上只剩培训机构和自媒体了呢?因为多数还在搞AI产品的公司都意识到了一个问题:靠AI本身是赚不到钱,只有把产品/服务卖出去才能赚到钱。

真正由 AI
驱动的产品并不多,理性认识你负责的那个模型对项目到底有多重要,可以更合理的调配工作时间与精力,也能在和外部对接时省去很多不必要的口舌。

AI赚了钱或得了利的主要是三种人:第一种是赚了风投的钱,吐血的是大大小小的孙正义。第二种是搭了巨无霸的顺风车,那些IT大厂不惜巨资做AI,不是因为AI给他们做出了赚钱的产品,而是想靠炒作AI提升股价,最终是让股民买单。大厂无一例外不敢不上,不能不鼓吹AI,无论其创始人对AI是真了解还是门外汉。他输不起,泡沫起处,你不冲浪冒险,你连游戏都玩不了,入不了局。第三种才是真正找到了市场切入点,把AI落地做成了规模化产品,占住了某个领域市场,也彰显了
AI
的威力。可惜,这第三类跟大熊猫似的,非常珍稀,而且多是九死一生侥幸生存下来的。包括特斯拉的自动驾驶,也是大难不死,现在才见到了曙光。


\paragraph{分析师3\sphinxfootnotemark[141]}
\label{\detokenize{chapter_introduction/money:id5}}%
\begin{footnotetext}[141]\sphinxAtStartFootnote
\sphinxnolinkurl{https://blogs.nvidia.cn/2017/09/30/ai-how-to-speed-up-the-analysis-of-financial-markets/}
%
\end{footnotetext}\ignorespaces 
Triumph
的数据科学家将专用数据库中的新闻传输到深度学习系统中。机器经过训练,每三毫秒即可分析一篇文章,一天能处理数十万篇,这个原来我们认为无法实现的结果,近期得以突破。

系统可以识别文章中数百个关键字。称为 GloVe
的无指导性学习算法可为每个关键字赋予一个数值,然后系统的其他模型可以理解并使用该数值。

深度学习系统最终会产生三个结果:它将文章关联至适当的股票和公司;它为每篇文章得出一个影响力得分(正面、中立和负面);它可以评估新闻影响市场的可能性。

在一段时期内,“虚假新闻”充斥着传统新闻领域,公司的数据科学家使用特定关键字和声誉好的新闻来源来提高可靠性。


\subparagraph{外包}
\label{\detokenize{chapter_introduction/money:id6}}\begin{itemize}
\item {} 
\sphinxurl{https://www.fiverr.com/?source=top\_nav}

\item {} 
\sphinxurl{https://www.duiyou360.com/}

\end{itemize}


\paragraph{成本 13\sphinxfootnotemark[142]}
\label{\detokenize{chapter_introduction/money:id7}}%
\begin{footnotetext}[142]\sphinxAtStartFootnote
\sphinxnolinkurl{https://www.zhihu.com/people/zhu-guan-jin-ming/answers/by\_votes}
%
\end{footnotetext}\ignorespaces \begin{enumerate}
\sphinxsetlistlabels{\arabic}{enumi}{enumii}{}{.}%
\item {} 
数据,每年需要花大量的钱买标注数据,这个价钱是很贵的,楼主可以去搜索标注公司的价钱,比如语音标注好点的大概200元/小时,买个1万小时就是200万人命币了,而且这种都是需要持续去买,持续给模型喂数据的,不然模型的效果会退化。

\item {} 
算力,训练模型和推理都需要用到硬件,其中最贵的还是CPU和GPU,尤其是GPU,差一点的1、2千一张卡,好一点的1、2万一张卡,租的话便宜的也要1、2千一个月了,一般做SaaS的采用租的模式,数量根据业务规模来,做私有化的都要采购服务器,成本更高。

\item {} 
算法,这块除了训练模型需要采购或者租用的服务器硬件以外,最大的开支还是人力开支,现在的算法人员的薪资水平在互联网绝对也算是拔尖的。去年我曾当面听着一个互联网公司的算法负责人吐槽招聘成本太高,招不起。

\end{enumerate}


\paragraph{盈利模式 4\sphinxfootnotemark[143]}
\label{\detokenize{chapter_introduction/money:id8}}%
\begin{footnotetext}[143]\sphinxAtStartFootnote
\sphinxnolinkurl{https://www.zhihu.com/question/20781934}
%
\end{footnotetext}\ignorespaces 
强调如何获取利润

每个企业能长期活下去,早期可以靠融资,可以烧钱,但一个企业不可能永远都烧钱,最终都还是要靠自己赚钱活下去,以及老股东投资了也要获得回报。

工具类:坐拥数十甚至上千万用户,却不知如何有效的将流量转化为收入——这是典型的“商业模式没想明白后遗症”!

想清楚:
\begin{itemize}
\item {} 
预计多久后,产品可以带来第一笔收入?

\item {} 
一年以后,你的产品盈利模式是怎样的?

\item {} 
在后面发展的每个阶段,项目的\sphinxstylestrong{支出预算、人员规模是多少?}要实现长远的发展目标,估计需要多久的时间和多少资金?

\end{itemize}


\paragraph{互联网的盈利模式}
\label{\detokenize{chapter_introduction/money:id9}}
\sphinxhref{https://www.zhihu.com/question/20304614/answer/1608253955}{6}%
\begin{footnote}[144]\sphinxAtStartFootnote
\sphinxnolinkurl{https://www.zhihu.com/question/20304614/answer/1608253955}
%
\end{footnote}
\begin{enumerate}
\sphinxsetlistlabels{\arabic}{enumi}{enumii}{}{.}%
\item {} 
提供一个公平的、纯粹的平台,把卖家和买家,或者把创作者(creator)和用户(viewer)双方联系起来,抽取佣金或者赚点广告费。

\item {} 
提供一个平台,但是把平台本身当作商品。这种公司出售的“商品”就是在这个平台“作弊”的机会。

\item {} 
“杀猪盘”。就是既提供平台,也下场和卖家/创作者竞争。把卖家/创作者当作猪,吸引他们入场,养肥了之后再杀。

\end{enumerate}

形式:广告、平台佣金、实物/虚拟商品售卖、增值服务及金融服务


\subparagraph{流量变现:广告费、抽取佣金or利润/利息差}
\label{\detokenize{chapter_introduction/money:or}}
贩卖流量较多使用kill time产品

以创作、社交为主的平台来说,流量就是一切。所以最重要的就是吸引优秀创作者。为此,很多公司还会和创作者分享利润(revenue
sharing)。比如在播放YouTube优秀创作者视频的时候投放的广告收入,YouTube都会分一部分给创作者本身。YouTube顶流李子柒,每年光YouTube广告收费分成就有七百万元左右。而在直播平台Twitch,如果你有几万粉丝,同时在线人数保持在100以上,就可以申请成为Twitch伙伴(partner)。除了打赏、广告、带货收入之外,只要你签约不去其他平台直播,Twitch每年还会另外给你40万元左右的收入。打赏自由知识分享的

电商直播:通过品类垄断的强议价能力,获取渠道利润;通过高效的仓储物流体系压缩流通成本,进一步拓展利润空间;通过龙头效应带来的厂家营销资源投入(代理、广告、商品运营等与厂家的营销合作);明星营销,给商品的大量曝光渠道。


\subparagraph{广告 7\sphinxfootnotemark[145]}
\label{\detokenize{chapter_introduction/money:id10}}%
\begin{footnotetext}[145]\sphinxAtStartFootnote
\sphinxnolinkurl{https://weread.qq.com/web/reader/8d232b60721a488e8d21e54kc20321001cc20ad4d76f5ae}
%
\end{footnotetext}\ignorespaces 
广告售卖形式主要有两种:品牌广告和效果广告。

其中品牌广告单价高,品质好。获得品牌类广告投放需要满足:内容和媒体平台为主,内容品质较高,用户质量较好,有媒体影响力;用户数量超过百万,才值得大型广告商青睐。

效果广告是整个互联网广告的主体。没有投放条件,只要有用户量即可,不过利润较低,没有大量用户无法产生规模收入。\sphinxhref{https://www.jianshu.com/p/60a253d06e03}{14}%
\begin{footnote}[146]\sphinxAtStartFootnote
\sphinxnolinkurl{https://www.jianshu.com/p/60a253d06e03}
%
\end{footnote}

形式:
\begin{itemize}
\item {} 
搜索广告典型的搜索广告是百度的竞价广告,展示在搜索结果列表中。商家竞价购买关键词,当用户搜索的内容触发关键词时,出价最高的广告就会被优先展示。

\item {} 
展示类广告展示类广告一般出现在信息类网站中的Banner、竖边、通栏等位置,而且展示类广告以品牌广告居多,即它更注重品牌曝光。

\item {} 
开屏广告部分App
的启动过程中会显示一副全屏广告,形式可能是图片,也可能是视频,展示时间为3~5秒不等,这就是开屏广告。知乎、豆瓣、小红书上等都会有这种类型的广告。

\item {} 
信息流广告信息流广告是指以文章、图片、视频等形式插入信息列表的广告,常见于内容类的产品,如百度、知乎、今日头条等App
中经常出现标记了“广告”字眼的信息。

\item {} 
视频广告在爱奇艺、优酷、腾讯视频等视频网站,如果用户没有购买会员,就会在视频播放前、播放过程中及暂停过程中看到广告,这些广告就是视频广告。

\end{itemize}

计费模式:
\begin{enumerate}
\sphinxsetlistlabels{\arabic}{enumi}{enumii}{}{.}%
\item {} 
CPM(Cost per
Mile,按千次展示收费):只要曝光就收费,不管点击、下载或注册等后续流程。这种模式适合想扩大知名度做品牌广告的广告主。早期门户网站的展示类广告基本都采用这种模式,少数开屏广告也采用这种模式。

\item {} 
CPC(Cost per
Click,按点击收费):在这种模式下,不管展示了多少次,只要用户不点击,广告主都不需要付费,只有用户点击了,广告主才需要付费。这种模式对广告主比较友好,因为首先它加大了平台作弊的难度;其次,它可以检测每个平台的流量质量,点击率高的就意味着质量高、用户精准,广告主以后可以多在这个平台投放广告。目前,这种模式常见于信息流广告和开屏广告。

\item {} 
CPA(Cost per Action,按用户行动收费):A
代表Action(行动),具体的用户行动是多种多样的,可以是下载、安装、购买等,具体是指哪种行动,需要在广告洽谈的时候,广告主和平台协商好,只有用户产生了协商好的行动,广告主才付费。

\end{enumerate}

可以看到,CPD(Cost per Download,按下载量收费)、CPI(Cost per
Install,按安装量收费)、CPS(Cost per Sales,按销售量收费)适合App
下载、增加新用户等需要明确转化行动的广告主。相对来说,CPM
适合以宣传品牌为主的广告主,而CPC 和CPA 倾向于保护广告主的利益。


\subparagraph{平台佣金}
\label{\detokenize{chapter_introduction/money:id11}}\begin{itemize}
\item {} 
平台模式:只提供交易平台(佣金、管理费用),卖家处理商品管理、仓储、配送、售后服务、开具发票

\item {} 
自营模式:买断货物,企业提供商品管理、仓储、配送、售后服务、开具发票服务;优势:服务体验优,利润率高;缺点:运营成本高

\end{itemize}

大部分平台型产品本身不拥有资产,但是通过整合资源提升服务效率获利。比较重要的盈利模式是收取平台佣金,但是收取对象不同。例如,淘宝平台和美团平台佣金的收取对象是商家,滴滴出行平台佣金的收取对象是司机,腾讯课堂平台佣金的收取对象是教师、教育机构,同花顺平台佣金的收取对象是股票投资者,映客直播平台佣金的收取对象是主播,直卖网平台佣金的收取对象是生产厂家等,这些平台的主要盈利模式就是平台佣金。平台上内容的所有权不归平台,所以这种盈利模式与实物/虚拟商品售卖的盈利模式有本质的区别。


\subparagraph{赚取中间利润}
\label{\detokenize{chapter_introduction/money:id12}}\begin{enumerate}
\sphinxsetlistlabels{\arabic}{enumi}{enumii}{}{.}%
\item {} 
实物商品。例如,京东的自营商品。京东采销人员向供应商采购商品在京东上售卖,商品的所有权属于京东,京东通过售卖商品赚取中间利润。又如,网易严选在代工厂贴牌之后直接售卖商品,这也是自营实物商品的模式。还有众筹/团购/电商/O2O\sphinxhref{https://t.qidianla.com/1156537.html}{17}%
\begin{footnote}[147]\sphinxAtStartFootnote
\sphinxnolinkurl{https://t.qidianla.com/1156537.html}
%
\end{footnote}

\item {} 
虚拟商品。例如,内容付费之一猿辅导的K12
网课,所有课程都是猿辅导的在职教师录制的,属于自营模式,而且课程属于虚拟商品。又如,粉笔网的网课等也属于虚拟商品。还有分答,值乎等。之后还可以用过IP延伸,周边产品\sphinxhref{https://t.qidianla.com/1156537.html}{17}%
\begin{footnote}[148]\sphinxAtStartFootnote
\sphinxnolinkurl{https://t.qidianla.com/1156537.html}
%
\end{footnote}来进一步内容变现。

\item {} 
虚拟服务虚拟服务与虚拟商品不同,它不是商品,而是一种服务,如阿里云服务(宽带、云存储等服务)、百度地图(WebAPI
服务)等。

\end{enumerate}


\subparagraph{赚取利息差}
\label{\detokenize{chapter_introduction/money:id13}}
金融借贷根据服务对象的不同,金融借贷可分为消费金融和供应链金融。消费金融是指2C
业务,如京东的白条及线下婚庆公司、教育公司的分期服务等;供应链金融是指2B
业务,如京东的京小贷、京保贝等业务都是为京东商家提供贷款的业务。

沉淀资金沉淀资金的金融服务模式是指利用沉淀在平台上的资金投资或者开展其他业务而产生收益。例如,用户在京东平台购物需要实时支付,但是京东平台跟第三方商家的结算是有一定的账期的,如30
天。那么,在账期内,这些资金就会沉淀在京东平台上,京东平台就可以利用这些资金投资或开展其他业务。


\subparagraph{“作弊”的机会——增值服务}
\label{\detokenize{chapter_introduction/money:id14}}
直接收费较多使用在save time产品

Facebook的“商品”,便是在Facebook的平台上“作弊”。也就是交钱让Facebook推广你的广告帖、广告视频。Facebook对优秀创作者并不支持,反而是打压。如果你不给Facebook交推广费,即使你一直在Facebook创作优秀的内容,很多人关注,很多人转发你的内容,Facebook也会故意通过算法打压你的内容,让别人无法看到。所以在Facebook能不出推广费用而获得流量是非常非常困难的。

虽然这种盈利方式来钱快,但是也有弊端,也就是因为失去优质内容,容易丢失用户。Facebook的应对方式就是通过主打和真实世界认识的亲朋好友的联系,推广和Facebook账号绑定的聊天软件Messenger来锁住用户。很多人为了了解亲朋好友的动态,不得不捏着鼻子看着Facebook上面充斥的广告贴。即使这样,最近Facebook的北美月活跃用户(monthly
active users)也一直在下降,有成为北美人人网的趋势。

基础功能免费吸引用户,增值服务收费实现盈利,这就是增值服务这种盈利模式的拆解。例如,百度网盘基础版的上传、下载等功能都可以免费使用,百度网盘也会免费为用户提供一部分存储空间,但是用户想获得更大的存储空间、更快的下载速度等,就要购买产品会员,这就是增值服务。又如,QQ
超级会员,很多年轻人喜欢的装扮特权(如挂件头像、气泡等),以及一些热门功能特权(如消息记录漫游、3000
人超大群)都属于增值服务。再如,CCtalk
的基础营销工具及授课是可以免费使用的,但是用户想要获取短信通知、多群直播、高清授课、录制下载等高级功能,就要付费购买。

形式:
\begin{enumerate}
\sphinxsetlistlabels{\arabic}{enumi}{enumii}{}{.}%
\item {} 
按需付费。如购买付费电影

\item {} 
按时间付费。如购买一段时间的服务

\end{enumerate}


\subparagraph{“杀猪盘”}
\label{\detokenize{chapter_introduction/money:id15}}
亚马逊这个网站,既提供一个平台给小商家在上面卖东西,但是他们自己也有自营的网店业务。他们先让第三方卖家入场卖东西,让他们赚点小钱。但是这些第三方卖家的价格、浏览量、销售额全部被亚马逊平台所掌握。之后亚马逊便会通过这些数据进行“严选”,找出一些比较好赚钱的商品,直接下场与第三方卖家竞争,通过价格战消灭第三方卖家。有时候甚至会随意封禁第三方卖家。

但是如果一个公司市场占有率还很低,公司官方提供的商品/内容质量远远低于第三方卖家/创作者,还学亚马逊搞杀猪盘,竭泽而渔的话,很难说是一种明智的行为。


\paragraph{数据的盈利模式 12\sphinxfootnotemark[149]}
\label{\detokenize{chapter_introduction/money:id16}}%
\begin{footnotetext}[149]\sphinxAtStartFootnote
\sphinxnolinkurl{https://www.pmcaff.com/discuss/1224707763133504?newwindow=1s}
%
\end{footnotetext}\ignorespaces 

\subparagraph{卖数据}
\label{\detokenize{chapter_introduction/money:id17}}
简单粗暴,一手交钱,一手给数据,行走于法律边缘,来钱最快。但是,买数据的人也不傻,他们只会挑业界标杆来合作,用有限的资金,换取高质量的数据。这就导致了只有行业标杆的几个企业能够通过这个方式挣得一定的资金。当然啦,这个资金不少的,毕竟数据很贵的。


\subparagraph{卖产品}
\label{\detokenize{chapter_introduction/money:id18}}
这方面的产品有:
\begin{itemize}
\item {} 
人群画像产品

\item {} 
爬虫舆论产品

\item {} 
APP分析产品

\item {} 
营销分析产品

\item {} 
等等。。。

\end{itemize}

但是,看到这些产品,不知大家是否发现?他们都是一些辅助性产品,并不直接产生价值——这就直接决定了其市场规模的高度,总不会高于创造价值的产品的。


\subparagraph{卖咨询服务}
\label{\detokenize{chapter_introduction/money:id19}}
这个就比较显然来,虽然各种咨询服务都十分昂贵,但是,客户少啊,所以你去看看世界500强,除了埃森哲一家,没有任何一家咨询公司入选了;而且埃森哲的纯咨询业务也是囊中羞涩的。


\subparagraph{基于数据的解决方案}
\label{\detokenize{chapter_introduction/money:id20}}
例如,一个独角兽的新能源车企,最大的卖点肯定就是车内的个性化的智能服务。但作为一个初创企业来说,它必定没有数据积累,哪怕能够拿到顾客的手机号码,但肯定不可能立刻拿到这个用户的过往数据的(经常出入的商圈,触媒习惯,兴趣爱好等),这时候数据服务商就可以通过其自身数据积累和技术能力,为初创企业提供这种基于具体业务需求的解决方案了,这也是很多大型数据服务商开始涉足的变现方式。


\paragraph{人工智能的盈利模式 9\sphinxfootnotemark[150]}
\label{\detokenize{chapter_introduction/money:id21}}%
\begin{footnotetext}[150]\sphinxAtStartFootnote
\sphinxnolinkurl{https://weread.qq.com/web/reader/0c032c9071dbddbc0c06459k65132ca01b6512bd43d90e3}
%
\end{footnotetext}\ignorespaces 
未来的人工智能有哪些商业模式?:\sphinxurl{https://www.zhihu.com/question/41848628}


\subparagraph{通过专利技术的授权、转让或置换实现盈利}
\label{\detokenize{chapter_introduction/money:id22}}
随着经济的发展,中国已成为全球消费品生产、消费和贸易大国,中国的人工智能产品也越来越多,而支撑人工智能产品的基础是中国在人工智能技术方面的不断创新。全球人工智能领域的专利数量自2011年开始逐渐呈现爆发式增长,每年的复合增长率达到30\%以上。中国在AI方面的专利技术布局程度已经位居世界第一。正因为专利技术在人工智能产品中的重要性,在人工智能市场上,通过技术创新申请专利,并将专利技术转让已经成了一种非常重要的盈利模式。

对比一下中国和美国在人工智能领域的六个企业——腾讯、百度、阿里巴巴、IBM、微软、谷歌可知,这些企业都非常注重整体的专利布局。而且,通过比较这些企业申请的专利可知,美国企业热衷于机器学习、语音识别、语言合成处理等领域,中国企业则倾向于支付、交互技术、视频图像信息处理、智能搜索等领域。另外,六家企业都比较感兴趣的领域有无人驾驶、数据文本聚类、指纹识别等。

IBM专利布局比较全面,其中算法优化、自然语言处理、自主驾驶领域布局优势明显;微软的专利布局主要在机器学习、神经网络、音视频识别等领域;谷歌主要是在无人驾驶、语音识别、自然语言处理领域有较多专利;腾讯主要在即时通信、数据处理、支付平台、数据交易等人工智能领域展开布局;百度比较热衷在搜索业务、无人驾驶、语音识别、图像识别等领域布局;阿里巴巴则在支付平台、信息交互、广告投放等领域布局明显。

企业在申请人工智能相关专利技术时,应充分体现出理论层次性、技术创新性、工程复杂性。在撰写时应注意以下几个方面:
\begin{enumerate}
\sphinxsetlistlabels{\arabic}{enumi}{enumii}{}{.}%
\item {} 
建议突出其成果专利的创新能力;

\item {} 
建议突出技术细节,并描述技术深度;

\item {} 
建议合理运用应用场景结合技术创新。

\end{enumerate}


\subparagraph{通过输出人工智能技术实现盈利}
\label{\detokenize{chapter_introduction/money:id23}}
除通过专利技术的授权、转让或置换实现盈利外,拥有人工智能技术的公司也可以通过对外提供技术服务实现盈利。目前人工智能的技术服务体系包括了基础级技术服务、技术级技术服务和行业级技术服务三个层面。

基础级技术服务是指企业通过提供框架平台或算法平台来提供的技术服务。例如,百度AI开放平台,阿里云ET大脑、腾讯AI平台、讯飞开放平台等。提供基础级技术的公司通过推出这些平台接口,吸引更多的用户,从而进一步活跃其产品的应用,并逐渐打造起一个开发者生态,并通过生态的活跃,提供其产品在行业中的应用阿里云ET大脑提供了多项人工智能技术服务,包括ET行业大脑、人工智能解决方案、人工智能接口、算法平台、ET大脑生态等内容。

上述公司在人工智能基础技术领域内有一定的优势,因此这类公司会在具体的技术领域进行拓展延伸。当然,如果这类公司仅仅提供技术,则会有竞争力弱的困扰,所以这些仅提供技术的公司往往通过“人工智能+行业”的模式形成具体的解决方案从而实现持续的盈利。

如果人工智能类的公司既拥有技术,同时又拥有大量的数据积累,则可以通过提供人工智能产品应用实现盈利。例如,格灵深瞳将人工智能和视频监控进行结合,开发了威目视图,实现了图像识别、人车定位识别;旷视科技的Face++平台,已经是我国领先的人脸识别的服务平台。


\subparagraph{通过销售产品实现盈利}
\label{\detokenize{chapter_introduction/money:id24}}
人工智能专利技术的转让及人工智能技术服务的输出,一般都是面向企业的,而人工智能产品则是面向大众的,易形成影响力。随着人工智能的发展,人工智能的产品类型也越来越多,如人工智能机器人、智能音箱、实时翻译工具、电子商务推荐助手、医疗影像检查、智能手环、游戏等。

人工智能产品销售方向有两个:一是面向企业,即2B;二是面向个人,即2C。2B方向的产品主要以提高生产力为目标,为企业降本增效,如智能分拣机器人、智能服务员、智能客服等。2C方向的产品主要是作为人体的延伸进行辅助判断及辅助操作,如智能音箱、智能推荐系统等。天猫精灵是阿里巴巴人工智能实验室于2017年7月5日发布的人工智能产品。天猫精灵内置智能语音助手AliGenie,能听懂普通话语音指令,并实现智能家居控制、语音购物、手机充值、音乐播放等功能。2018年5月27日,阿里巴巴公布了天猫精灵的销量,销售总量超过了300万台,业绩非常优秀。


\paragraph{为何免费}
\label{\detokenize{chapter_introduction/money:id25}}
免费商业模式的本质,即交叉补贴。

前提:
\begin{enumerate}
\sphinxsetlistlabels{\arabic}{enumi}{enumii}{}{.}%
\item {} 
能不能找到补贴方?

\item {} 
从补贴方获得的收益能否覆盖免费的成本?

\item {} 
在找到能覆盖免费陈本的补贴方之前,这个时间成本是否可承受?你总不能死在找到补贴方之前吧。

\end{enumerate}

◆ 直接交叉补贴:产品之间的交叉补贴,用免费吸引你掏腰包买其他的产品。

腾讯公司马化腾就是利用免费核心产品QQ,绑架近几亿用户,从而向这些用户销售一些增值产品来赚取利润,比如QQ秀,钻等。电信运营商依靠赠送手机或话费来吸引用户,赚取流量和话费等等。

◆ 三方市场:利益主体之间的交叉补贴

媒体行业是三方市场模式的典型,广播、报纸、电视和杂志等等,用户不用付费可以免费的到信息、内容或软件。由广告商买单。即媒体将用户卖给广告商。内容消费者得到了免费,但有广告主来买单。

◆ 现在和未来之间的交叉补贴

用好的产品免费让用户使用,用户为了得到更好,就愿意付费享受更好的服务,让被动掏钱变为主动付费,只要这样用户的付费意愿就会更强烈,免费也能赚钱,就是这个逻辑。\sphinxhref{https://www.zhihu.com/question/38281398}{15}%
\begin{footnote}[151]\sphinxAtStartFootnote
\sphinxnolinkurl{https://www.zhihu.com/question/38281398}
%
\end{footnote}

比如滴滴打车,在推广的时候,很多人享受过免费打车的。但这个钱最终会在习惯被养成之后赚回来。

◆ 货币市场和非货币市场的交叉补贴:

任何人都可以免费得到其他人赠送的产品或服务,且不需要得到金钱回报,获得的是关注度和声誉。撰写博客,发布微博、微信等,并非出于谋取利益,而是与人分享喜怒哀乐,期待结识朋友;公益捐助,获得慈善相关的名声等等。


\paragraph{ROI}
\label{\detokenize{chapter_introduction/money:roi}}
Open source models, data and transfer learning are also enabling
businesses to more easily move models into production and to achieve an
ROI.


\paragraph{融资}
\label{\detokenize{chapter_introduction/money:id26}}
企业融资,说白了就是企业如何获得正向的现金流。因为有了钱,你就可以当个土豪“买买买”,买装备、买土地、买资源、买人才、买用户甚至买竞争对手。但是,市场上真正缺钱的都是中小企业/民营企业/初创企业这样的企业,出身差、没钱、没人才、没资源,在债权融资要不来钱的情况下,这些企业就选择股权融资。

一般来说,融资轮次的划分为种子轮、天使轮、A轮、B轮、C轮、D轮、E轮等,但根据实际情况,有些项目也会进行preA轮、A+轮、C+轮融资,不管是什么轮,其核心无非是投资人投的多少钱的问题,
\begin{itemize}
\item {} 
种子轮:种子阶段的融资人,通常只有idea和团队,但没具体产品的初始形态,投资人一般多是亲朋好友、或者创业者自掏腰包,当然现在也涌现不少种子时期投资人;倘若你的融资项目团队,有idea,马上进入最终的落地,那么就可以进行种子轮融资,一般项目融资都在100万左右,根据不同的赛道,可能从几十万到200万不等。

\item {} 
天使轮:天使阶段的项目通常是团队ready,有产品雏形,有产品初步的商业规划,却也陷入找人——做产品——没人了——找人——做产品的循环之中,如果融资项目已经起步,产品初具模样,有种子数据显示出增长趋势、留存、复购等证明。同时积累了一些核心用户,商业处于待验证的阶段,那么找天使投资人或机构,开始天使轮融资便是最为合适的,融资金额大概在300万到500万左右;

\item {} 
PreA轮:是一个夹层轮,融资人根据自身项目的成熟度,再决定是否要融资,倘若项目前期整体数据已经具有一定规模,只是未占据市场前列,那么可以进行PreA轮融资;

\item {} 
A轮:对于拥有成熟产品,完整详细的商业及盈利模式,同时在行业内拥有一定地位与口碑的项目,哪怕现阶段处于亏损状态,也可以选择专业的风险投资机构进行A轮融资,这一阶段融资人已经不可能只凭借idea融资,而是要有用户,包括月活、日活、要有自己的商业模式,有能与竞品抗衡的成熟产品,有一定的市场位置;

\item {} 
B轮:经过一轮的烧钱后,项目有较大的发展,商业模式与盈利模式均已得到很好的验证,有的已经开始盈利。此时,融资人可能需要资金支持推出新的业务、拓展新领域,那么就适合说服上一轮风险投资跟投,或寻找新的风投机构的加入,又或是吸引私募股权投资机构加入的形式,开始新的一轮的B轮融资。

\item {} 
C轮:如果此时融资人的项目十分成熟,在行业内基本可以稳坐前三把交椅,正在为上市做准备,那么就适合进行C轮融资,此时除了可以进一步拓展新业务,也可以补全商业闭环,准备上市打好基础。

\end{itemize}

当公司逐渐成为行业要角,进入Pre\sphinxhyphen{}IPO阶段,这时投资银行(Investment
Bank)便会出现,来协助公司顺利「上市柜」(IPO)。直到成功上市,熬过了股票「闭锁期」后,先前的投资人才有机会出脱持股,顺利出场。\sphinxhref{https://dahetalk.com/2018/12/02/\%E3\%80\%90\%E6\%96\%B0\%E5\%89\%B5\%E8\%9E\%8D\%E8\%B3\%87\%E3\%80\%91\%E7\%A8\%AE\%E5\%AD\%90\%E8\%BC\%AA\%E3\%80\%81\%E5\%A4\%A9\%E4\%BD\%BF\%E8\%BC\%AA\%E3\%80\%81a\%E8\%BC\%AA\%E3\%80\%81b\%E8\%BC\%AA\%E3\%80\%81c\%E8\%BC\%AA\%EF\%BC\%8C\%E4\%BD\%A0/}{16}%
\begin{footnote}[152]\sphinxAtStartFootnote
\sphinxnolinkurl{https://dahetalk.com/2018/12/02/\%E3\%80\%90\%E6\%96\%B0\%E5\%89\%B5\%E8\%9E\%8D\%E8\%B3\%87\%E3\%80\%91\%E7\%A8\%AE\%E5\%AD\%90\%E8\%BC\%AA\%E3\%80\%81\%E5\%A4\%A9\%E4\%BD\%BF\%E8\%BC\%AA\%E3\%80\%81a\%E8\%BC\%AA\%E3\%80\%81b\%E8\%BC\%AA\%E3\%80\%81c\%E8\%BC\%AA\%EF\%BC\%8C\%E4\%BD\%A0/}
%
\end{footnote}

首次公开募股(Initial Public
Offering)是指一家企业第一次将它的股份向公众出售。

“IPO本质是为了实现更大规模的目标,需要更大规模的资本。”二级市场的资金量远超一级市场,可以真正帮助公司实现商业化。\sphinxhref{https://www.cyzone.cn/article/628604.html}{20}%
\begin{footnote}[153]\sphinxAtStartFootnote
\sphinxnolinkurl{https://www.cyzone.cn/article/628604.html}
%
\end{footnote}

{\color{red}\bfseries{}|}融资过程{\color{red}\bfseries{}|}\sphinxhref{https://dahetalk.com/2018/12/02/\%E3\%80\%90\%E6\%96\%B0\%E5\%89\%B5\%E8\%9E\%8D\%E8\%B3\%87\%E3\%80\%91\%E7\%A8\%AE\%E5\%AD\%90\%E8\%BC\%AA\%E3\%80\%81\%E5\%A4\%A9\%E4\%BD\%BF\%E8\%BC\%AA\%E3\%80\%81a\%E8\%BC\%AA\%E3\%80\%81b\%E8\%BC\%AA\%E3\%80\%81c\%E8\%BC\%AA\%EF\%BC\%8C\%E4\%BD\%A0/}{16}%
\begin{footnote}[154]\sphinxAtStartFootnote
\sphinxnolinkurl{https://dahetalk.com/2018/12/02/\%E3\%80\%90\%E6\%96\%B0\%E5\%89\%B5\%E8\%9E\%8D\%E8\%B3\%87\%E3\%80\%91\%E7\%A8\%AE\%E5\%AD\%90\%E8\%BC\%AA\%E3\%80\%81\%E5\%A4\%A9\%E4\%BD\%BF\%E8\%BC\%AA\%E3\%80\%81a\%E8\%BC\%AA\%E3\%80\%81b\%E8\%BC\%AA\%E3\%80\%81c\%E8\%BC\%AA\%EF\%BC\%8C\%E4\%BD\%A0/}
%
\end{footnote}
{\color{red}\bfseries{}|}AI创业基金工作{\color{red}\bfseries{}|}\sphinxhref{https://www.zhihu.com/question/19658921/answer/52438369}{19}%
\begin{footnote}[155]\sphinxAtStartFootnote
\sphinxnolinkurl{https://www.zhihu.com/question/19658921/answer/52438369}
%
\end{footnote}


\subparagraph{Pre\sphinxhyphen{}A轮、B+轮、B++轮、D轮、E轮,又是什麽?}
\label{\detokenize{chapter_introduction/money:pre-ab-b-de}}
一般来说,融资募到C轮就差不多了。D轮指的是你把C轮的钱烧完了,但还没进入上市柜阶段;同理,E轮指的是你把D轮的钱烧完,但依旧还是没上市柜。

而大家常听到的Pre\sphinxhyphen{}A轮,指的是天使轮的钱花完了,但产品还不够成熟,尚未到A轮阶段,进退两难下,只好又募一个round,我们就称它为Pre\sphinxhyphen{}A轮。至于B+轮、B++轮都是相同意思,因为还没到下一round的水准,所以只好一直无限+++++。

在规划里,B轮融资需要证明技术实力,因而团队专攻技术竞赛成绩;C轮投资人会比较业务优势,团队需要在此之前让原型车落地;D轮公司已经开始试运营,商业化推进自然会成为投资人眼中的亮点。\sphinxhref{https://www.cyzone.cn/article/628604.html}{20}%
\begin{footnote}[156]\sphinxAtStartFootnote
\sphinxnolinkurl{https://www.cyzone.cn/article/628604.html}
%
\end{footnote}


\subparagraph{AI 创业}
\label{\detokenize{chapter_introduction/money:ai}}
\begin{figure}[H]
\centering
\capstart

\noindent\sphinxincludegraphics{{AI_entrepreneur}.png}
\caption{AI 创业}\label{\detokenize{chapter_introduction/money:id36}}\end{figure}

所有AI企业,作为前沿、高端的研发与技术密集型行业,收入规模在相当长一段时间内,有可能无法支撑巨额的、持续的、大规模研发投入、场景探索及市场开拓等,持续亏损的风险与压力,会长期存在。\sphinxhref{https://www.weiyangx.com/382066.html}{18}%
\begin{footnote}[157]\sphinxAtStartFootnote
\sphinxnolinkurl{https://www.weiyangx.com/382066.html}
%
\end{footnote}


\subparagraph{AI 融资}
\label{\detokenize{chapter_introduction/money:id35}}
人工智能基金创业工作室从零开始创建新的人工智能公司。这些公司将AI技术和应用连接起来,专注于推动世界前进的行业和问题。\sphinxurl{https://aifund.ai/}

\sphinxhref{https://www.lieyunwang.com/archives/472130}{2020「年度最佳人工智能领域投资机构TOP10」}%
\begin{footnote}[158]\sphinxAtStartFootnote
\sphinxnolinkurl{https://www.lieyunwang.com/archives/472130}
%
\end{footnote}


\subsubsection{Product Mananger}
\label{\detokenize{chapter_introduction/PM:product-mananger}}\label{\detokenize{chapter_introduction/PM::doc}}
互联网上半场,资金为王,流量为王,营销为王,其典型特征是“短、平、快”;互联网下半场,存量博弈,品牌为王时代到来,能否留住用户、获得市场口碑,靠的是过硬的产品和精耕细作的产品人。
——王超辉 58到家高级总监
\sphinxhref{https://weread.qq.com/web/reader/77532110721ea34a7751c9ake4d32d5015e4da3b7fbb1fas}{22}%
\begin{footnote}[159]\sphinxAtStartFootnote
\sphinxnolinkurl{https://weread.qq.com/web/reader/77532110721ea34a7751c9ake4d32d5015e4da3b7fbb1fas}
%
\end{footnote}


\paragraph{定义}
\label{\detokenize{chapter_introduction/PM:id1}}
真正的产品经理懂得搭建产品矩阵,认清行业趋势,发现需求,用数据、专业度和案例事实去协调团队其他成员,保障项目准时完成,并且能让产品盈利的。
\sphinxhref{https://www.zhihu.com/pub/reader/119583028/chapter/1057335985074978816s}{24}%
\begin{footnote}[160]\sphinxAtStartFootnote
\sphinxnolinkurl{https://www.zhihu.com/pub/reader/119583028/chapter/1057335985074978816s}
%
\end{footnote}是《启示录》里说的“发现有价值的、可用的、可行的产品的岗位”。其站在商业、技术和用户体验(BTC原则)三者的交点。商业指的是专注于实现产品的最大商业价值,技术指的是在产品如何被创造出来所涉及的过程,而用户体验指的是以产品去跟用户交流。\sphinxhref{https://coffee.pmcaff.com/article/2447262389384320/pmcaff?utm\_source=forum}{47}%
\begin{footnote}[161]\sphinxAtStartFootnote
\sphinxnolinkurl{https://coffee.pmcaff.com/article/2447262389384320/pmcaff?utm\_source=forum}
%
\end{footnote}

能力要求(自己总结的):用户观察、同理心、市场判断(发现机会)–》逻辑沟通来串联商业与技术(整合、争取资源)–》不断项目管理来推进执行–》推向市场来不断反馈(运营、营销、销售)

眼(发现)\sphinxhyphen{}》说并行\sphinxhyphen{}》眼(修正)。


\subparagraph{心态}
\label{\detokenize{chapter_introduction/PM:id2}}
只要你能够发现问题并描述清楚,转化为一个需求,进而转化为一个任务,争取到支持,发动起一批人,将这个任务完成,并创造了价值,并持续不断以主人翁的心态去跟踪、维护这个产物,那么,你就是产品经理。


\subparagraph{市场满足}
\label{\detokenize{chapter_introduction/PM:id3}}
产品经理的价值是创造一款满足产品\sphinxhyphen{}市场匹配(Product\sphinxhyphen{}market
fit,PMF)「在一个好的市场里,能用一个产品去满足这个市场」的产品
\sphinxhref{http://www.ramywu.com/work/2018/05/31/AI-PM-Interview/}{9}%
\begin{footnote}[162]\sphinxAtStartFootnote
\sphinxnolinkurl{http://www.ramywu.com/work/2018/05/31/AI-PM-Interview/}
%
\end{footnote}

为企业获得利益回报而创造顾客价值,并与顾客建立稳定的关系;为产品的市场结果负责。
\begin{itemize}
\item {} 
顾客价值,即为顾客创造的产品(包含服务,我将其归到产品里,以后所有提到产品的地方都包含服务);

\item {} 
商业价值:利益回报,简单讲就是买产品赚取金钱收益,当然也可能是合理降低回报以获取品牌收益,但最终还是为了赚钱;

\item {} 
建立顾客关系,即持续赚钱,尽可能占领更多的市场。

\end{itemize}


\subparagraph{协作}
\label{\detokenize{chapter_introduction/PM:id4}}
产品经理是“将不同语言的公司的所有各种功能和角色结合在一起的粘合剂” – Ken
Norton,GV \sphinxhref{https://easyai.tech/author/xiaoqiang/page/5/}{7}%
\begin{footnote}[163]\sphinxAtStartFootnote
\sphinxnolinkurl{https://easyai.tech/author/xiaoqiang/page/5/}
%
\end{footnote}

产品经理还是一个项目的信息汇集中心,对于公司的战略方向,产品经理比其他人更早地知道。产品经理也会经常跟运营人员、销售人员开会(“贩卖”自己的产品理念),所以对于公司的运营节奏、销售数据,产品经理都可能提前知悉。这些都是可以帮助产品经理快速成长的点,也是产品经理不断积累人脉的基础。
\sphinxhref{https://weread.qq.com/web/reader/8d232b60721a488e8d21e54k8f132430178f14e45fce0f7}{15}%
\begin{footnote}[164]\sphinxAtStartFootnote
\sphinxnolinkurl{https://weread.qq.com/web/reader/8d232b60721a488e8d21e54k8f132430178f14e45fce0f7}
%
\end{footnote}

\begin{figure}[H]
\centering
\capstart

\noindent\sphinxincludegraphics{{business_value}.png}
\caption{为了商业价值}\label{\detokenize{chapter_introduction/PM:id67}}\end{figure}

传统产品经理泛指传统互联网产品经理,区别于宝洁(产品经理这个职能的开创者)时期的实体工业产品经理,互联网产品经理标准化是在移动互联网的商业模式成型过程中,也就是2010s这个时期,在这个时期随着移动互联网的发展,各个公司特别是BAT为代表的巨头公司细化了产品经理所需要具备的能力模型,并且基于这个能力模型去对产品人员进行量化考核
\sphinxhref{https://www.jianshu.com/p/fd466ed1bda6}{1}%
\begin{footnote}[165]\sphinxAtStartFootnote
\sphinxnolinkurl{https://www.jianshu.com/p/fd466ed1bda6}
%
\end{footnote}


\subparagraph{核心能力}
\label{\detokenize{chapter_introduction/PM:id5}}
产品经理的核心能力是最大化的\sphinxstylestrong{资源整合管理能力},包括资源发现,资源获取,资源管理,资源再生,资源裁剪等。
\sphinxhref{https://www.zhihu.com/question/57815929/answer/981667560}{34}%
\begin{footnote}[166]\sphinxAtStartFootnote
\sphinxnolinkurl{https://www.zhihu.com/question/57815929/answer/981667560}
%
\end{footnote}


\subparagraph{能力来源}
\label{\detokenize{chapter_introduction/PM:id6}}
产品经理的能力主要来源于三部分:业务经验、平台适配、认知能力。

业务经验,是指在特定业务场景下,产品经理对用户模型的认知程度;平台适配,是指产品经理对所处平台组织结构、决策方式等的适应程度;而认知能力更多的是,明白如何做正确的事情,以及如何正确的做事情。

业务经验和平台适配不具备通用性,且最多三到五年就会到达天花板,而认知能力无上限,且具备极强的通用性。\sphinxhref{https://www.jianshu.com/p/ea942a96a668}{41}%
\begin{footnote}[167]\sphinxAtStartFootnote
\sphinxnolinkurl{https://www.jianshu.com/p/ea942a96a668}
%
\end{footnote}


\paragraph{VS Project Mananger}
\label{\detokenize{chapter_introduction/PM:vs-project-mananger}}
\sphinxstylestrong{产品经理(Product
Manager)}:更复合型人才,负责挖掘用户需求,并提出能够同时满足用户需求和公司利益的产品方案。对产品本身负责,对商业计划负责,对用户负责,或者说对产品发布后是否受用户认可负责。与产品运营人员共背用户数、活跃度、留存率、营收等KPI指标。\sphinxhref{https://blog.csdn.net/zcl050505/article/details/111772891}{42}%
\begin{footnote}[168]\sphinxAtStartFootnote
\sphinxnolinkurl{https://blog.csdn.net/zcl050505/article/details/111772891}
%
\end{footnote}

多做需求快速试错:
对于大多数产品经理来说,用户是否喜欢新功能其实是未知的,不然人人都是产品经理了,免不了试错的过程。做5个新功能可能没有1个用户喜欢的,做50个呢?总能撞上几个让用户喜欢的吧。所以很多不成熟的产品经理喜欢通过增加功能点来“碰运气”。

\sphinxstylestrong{项目经理(Project
Manager)}:更专才,负责管理整个产品开发过程的项目,负责协调产品开发中的一切资源,包含人、时间、资源、成本等。是整个项目的牵头人,对项目的开发过程和按时完成预期结果负责。也就是说,项目经理的KPI应该是项目的按时完成与完成质量。用户数、活跃度等指标一概与项目经理无关,更别说干涉需求。

按时按质完成:做得越多就错得越多,很容易影响项目按时验收,以及项目结果质量。比如APP产品的开发,大家都知道修复完一个bug后,正常情况下会出9个新bug。所以项目经理是喜欢需求越少越好,这样项目复杂程度会低,工作量与风险程度都很容易控制在安全范围内。

由目标的差异推导,项目经理更强调执行,是接到一个任务,\sphinxstylestrong{“多快好省”地做事};产品经理更强调创新,是设定一个目标,\sphinxstylestrong{做符合商业目标的事}。

如果由一个人同时担任产品经理和项目经理,就很容易左右互搏。需求是自己的,当然希望都做。但开发过程中遇到难题怎么办。砍需求?改方案?自己预期中最完美的方案要舍弃还是挺心痛的。坚持把需求按原方案做完?项目延期,影响正常上线,责任谁担?

如果产品经理是大写的PM,那么项目经理就是小写的pm,\sphinxstylestrong{项目经理}可以理解成是产品经理工作\sphinxstylestrong{在项目管理上的细分岗位}职能。一般产品经理如果在工作中忙得过来就会负责项目管理层面的工作,但是如果产品上线,产品层面工作颗粒度变细,这部分工作产品经理就需要拆分出去给其Ta小伙伴负责。\sphinxhref{https://zhuanlan.zhihu.com/p/25796796}{56}%
\begin{footnote}[169]\sphinxAtStartFootnote
\sphinxnolinkurl{https://zhuanlan.zhihu.com/p/25796796}
%
\end{footnote}


\subparagraph{时代背景}
\label{\detokenize{chapter_introduction/PM:id7}}
过去需要的是项目经理式的工作能力,而现在更强调产品经理式的工作能力。

时代背景——产品的供给,都在从短缺走向丰饶。选择的丰富导致选择的困难。市场从从生产驱动变成了需求驱动。


\paragraph{你配PM吗}
\label{\detokenize{chapter_introduction/PM:pm}}

\subparagraph{不是人人都是PM}
\label{\detokenize{chapter_introduction/PM:id8}}
他们所理解的产品经理,无非是写文档、画原型、催进度,保持一颗满足用户需求的心,再喊几句「用户永远不知道自己想要什么」便可以为所欲为的一个岗位。

既不用懂代码,也不用会设计,更不用跑业务,还有着「经理」称号,并且平均薪水一度在招聘网站的薪资报告中显示排行第一,好像是一个绝佳的工作岗位。\sphinxhref{https://www.zhihu.com/pub/reader/119583028/chapter/1057335985074978816}{23}%
\begin{footnote}[170]\sphinxAtStartFootnote
\sphinxnolinkurl{https://www.zhihu.com/pub/reader/119583028/chapter/1057335985074978816}
%
\end{footnote}

对产品经理有洁癖的腾讯,把12级(原P4\sphinxhyphen{}1)以下( 总共4 级\sphinxhyphen{}14
级)的产品从业者都改名叫什么产品策划、产品运营,只有综合能力达标的那一小撮人,才配继续叫产品经理。\sphinxhref{https://m.k.sohu.com/d/495625828?channelId=1\&page=1}{8}%
\begin{footnote}[171]\sphinxAtStartFootnote
\sphinxnolinkurl{https://m.k.sohu.com/d/495625828?channelId=1\&page=1}
%
\end{footnote}

\begin{center}\sphinxincludegraphics{{design_GTM}.jpg}\end{center} \sphinxincludegraphics{{tencent_PM}.png}


\subparagraph{画原型 != 产品经理 26\sphinxfootnotemark[172]}
\label{\detokenize{chapter_introduction/PM:id9}}%
\begin{footnotetext}[172]\sphinxAtStartFootnote
\sphinxnolinkurl{https://www.zhihu.com/pub/reader/119980992/chapter/1284104609385250816}
%
\end{footnotetext}\ignorespaces 
你无时无刻不在提醒自己,产品经理就是要画原型,因为有了原型才能去和研发部同事开需求评审会,有了原型才能第一时间和领导沟通需求,有了原型才有了能够写进周报里的工作事项。

这使得产品经理似乎成了互联网行业里最不需要专业技能的岗位,站在大街上随便叫一个人,定向培训一个月的
Axure 就可以直接上岗了。


\subparagraph{沉溺于细节 27\sphinxfootnotemark[173]}
\label{\detokenize{chapter_introduction/PM:id10}}%
\begin{footnotetext}[173]\sphinxAtStartFootnote
\sphinxnolinkurl{https://www.zhihu.com/pub/reader/119980992/chapter/1284104608756113408}
%
\end{footnotetext}\ignorespaces 
很多类似按钮放左或放右的问题,如果你真的做了 A/B
测试,那么你会发现两者的数据是完全一致的,无论是第一视觉、操作的难易程度还是数据的转化等,都是一样的。


\subparagraph{只看表面 28\sphinxfootnotemark[174]}
\label{\detokenize{chapter_introduction/PM:id11}}%
\begin{footnotetext}[174]\sphinxAtStartFootnote
\sphinxnolinkurl{https://www.zhihu.com/pub/reader/119980992/chapter/1284104609385250816}
%
\end{footnotetext}\ignorespaces 
产品经理要谨记自己的第一要务是基于目标用户的某个问题,提供优于市场的解决方案,目标用户需要的是解决方案,而不是简单的视觉冲击(界面设计那是设计师的活)。视觉能够在某个瞬间刺激用户,但如果没有长期吸引用户的价值点,那么他们终将会离开。

产品经理解决目标用户问题的最好办法不是花费精力去研究视觉,而是要先把视觉的外衣褪去,抓住核心的问题点去寻找更好的解决方案。


\subparagraph{需求评审会上自说自话 29\sphinxfootnotemark[175]}
\label{\detokenize{chapter_introduction/PM:id12}}%
\begin{footnotetext}[175]\sphinxAtStartFootnote
\sphinxnolinkurl{https://www.zhihu.com/pub/reader/119980992/chapter/1284104611201466368}
%
\end{footnotetext}\ignorespaces 
产品经理在需求评审会上“专注而又认真”地讲解,压根儿不管下面的听众。

执行时才发现,有很多不明白和无法实现的地方,“这里的逻辑到底是怎样的?异常情况要如何处理?这个方式实现不了怎么办?”

就这样,进度又一次被耽搁。研发人员认为产品经理的需求不靠谱,有很多功能需要花费很长时间才能实现;产品经理认为研发人员根本不用心,而且也不怎么加班;而测试则一方面认为产品经理的需求本身不够严谨,另一方面又认为研发人员写的代码漏洞百出。

或是一次性甩17个小功能需求参加需求评审会,是“自杀式”的打法,不利于产品迭代,同时更改的功能点过多,不够聚焦。动机和功能的缘由必须总结清楚,否则无法说服开发。\sphinxhref{https://t.qidianla.com/1156501.html}{54}%
\begin{footnote}[176]\sphinxAtStartFootnote
\sphinxnolinkurl{https://t.qidianla.com/1156501.html}
%
\end{footnote}


\subparagraph{对测试不闻不问 30\sphinxfootnotemark[177]}
\label{\detokenize{chapter_introduction/PM:id13}}%
\begin{footnotetext}[177]\sphinxAtStartFootnote
\sphinxnolinkurl{https://www.zhihu.com/pub/reader/119980992/chapter/1284104611813195776}
%
\end{footnotetext}\ignorespaces 
测试人员本身更关注流程、压力等方面的测试,不会过多地考虑背景、目的、核心功能,因此测试人员做的测试可以理解为走流程的测试,而产品经理做的测试则可以理解为战略性的测试。

产品经理要做的测试基于页面、流程的核心功能的还原程度,如今日头条类产品的算法推荐逻辑、淘宝类购物平台的千人千面、金融类产品背后的风控模型等。也可以说,这个核心功能等同于这个版本,如果这个核心功能没有达到预期的效果,那么这个版本本身就应该直接被废弃。

因此,产品经理不能把测试、上线这些事情全部交给研发人员和测试人员,而应参与整个研发过程,第一时间了解研发人员、测试人员对于需求本身的理解,以及现阶段实现的情况、难点和需要的支援等,从而更好地达到上线的效果。


\subparagraph{上线后彻底不管 31\sphinxfootnotemark[178]}
\label{\detokenize{chapter_introduction/PM:id14}}%
\begin{footnotetext}[178]\sphinxAtStartFootnote
\sphinxnolinkurl{https://www.zhihu.com/pub/reader/119980992/chapter/1284104612782419968}
%
\end{footnotetext}\ignorespaces 
工作要产品策划和产品运营融合起来。产品经理要对结果负责,而数据能最好展示结果。

产品经理每天都要看自己负责的产品功能的数据并对其了如指掌,能够基本判断阶段性的数据起伏背后的原因和预设各种数据的埋点,这些本身就属于产品经理的基本职责。

而上线后彻底不管,就像只生不养。研发人员和测试人员只是帮助产品功能上线,而市场运营的同事才能够让产品的价值最大化。

产品上线之前,产品经理需要提前做好产品上线的市场运营工作,而不是单纯地等着自己负责的产品功能上线,然后看着它悄无声息地躺在
App 的某个角落,无人理睬,最终又悄无声息地下线。


\subparagraph{产品没落很轻松 32\sphinxfootnotemark[179]}
\label{\detokenize{chapter_introduction/PM:id15}}%
\begin{footnotetext}[179]\sphinxAtStartFootnote
\sphinxnolinkurl{https://www.zhihu.com/pub/reader/119980992/chapter/1284104613399535616}
%
\end{footnotetext}\ignorespaces 
很多产品不到 3
年便走向没落了,而很多产品经理只能算自己产品的月活用户(而且一个月只登录一次),很多时候线上产品出现了漏洞,自己却是最后一个才发现的,更不用说主动去探索更多的商业化空间和优化空间了。

产品没落了,产品经理虽然不一定非要悲痛欲绝、情绪失控,但是至少要懂得抓住这样的机会快速地总结和复盘\sphinxhref{https://www.zhihu.com/pub/reader/119980992/chapter/1284104613692768256}{33}%
\begin{footnote}[180]\sphinxAtStartFootnote
\sphinxnolinkurl{https://www.zhihu.com/pub/reader/119980992/chapter/1284104613692768256}
%
\end{footnote},思考产品没落的原因是什么。


\paragraph{岗位稀缺?}
\label{\detokenize{chapter_introduction/PM:id16}}
由于产品经理负责的产品也许是一个公司的核心,产品经理的好坏直接影响一个公司的运作,所以这样的职位一般是不招聘新人的,甚至少于三年产品类工作经验都直接不考虑。并且很多公司或者团队也是很少培养这方面的新人,而大公司会通过校招等方式招聘一些有潜质的人才进行岗位培养,但是往往都是几千几万人竞争1个名额,也并非所有人都有机会。
\sphinxhref{https://tangjie.me/blog/129.html}{43}%
\begin{footnote}[181]\sphinxAtStartFootnote
\sphinxnolinkurl{https://tangjie.me/blog/129.html}
%
\end{footnote}


\paragraph{分类维度 38\sphinxfootnotemark[182]}
\label{\detokenize{chapter_introduction/PM:id17}}%
\begin{footnotetext}[182]\sphinxAtStartFootnote
\sphinxnolinkurl{https://www.zhihu.com/question/26679255/answer/1446764998}
%
\end{footnotetext}\ignorespaces \begin{enumerate}
\sphinxsetlistlabels{\arabic}{enumi}{enumii}{}{.}%
\item {} 
按照行业分:金融产品经理、医疗产品经理、教育产品经理、电商产品经理等等;

\item {} 
按照产品形态分:移动产品经理、PC产品经理、小程序产品经理、M站产品经理、后台产品经理等等;

\item {} 
按照工作内容及方法分:功能产品经理、数据产品经理、策略产品经理、商业产品经理等等,

\end{enumerate}


\subparagraph{职能分类 44\sphinxfootnotemark[183]}
\label{\detokenize{chapter_introduction/PM:id18}}%
\begin{footnotetext}[183]\sphinxAtStartFootnote
\sphinxnolinkurl{https://tangjie.me/blog/183.html}
%
\end{footnotetext}\ignorespaces 

\subparagraph{功能型产品经理}
\label{\detokenize{chapter_introduction/PM:id19}}
功能型产品经理主要就是设计功能,通常都是刚入门或入门不久的产品经理,像产品助理(专员)也是属于功能型。功能型产品经理一般只需要懂工作中的各类常用工具软件的使用,以及各种常见的产品模式的用户角色和功能结构,懂得这两大项就完全可以胜任功能型产品经理的工作了。常用的工具软件有Office办公软件、思维导图软件、原型设计软件等;常见的产品模式有B2C、O2O、SNS等。

功能型产品经理常见的工作情况就是执行公司的产品基础规划和设计,比如公司需要做一个B2C模式的电子商务网站(或App),功能型产品经理就只需要将B2C模式的用户角色和产品功能规划并设计出来就可以了,一般不用全局性考虑产品的运作策略,或者说功能型产品经理还不具备全局性规划产品的能力,所以功能型产品经理更像是一个工匠。

但是这类产品经理也不是轻易胜任的,需要了解各类产品模式的用户角色权限与产品功能结构是什么样的和怎么实现的。当遇到公司提出需求的时候,能够第一时间就对产品形态和功能结构有一个初步的思路,一旦明确了产品需求就能够清晰的知道如何展开工作。


\subparagraph{运营型产品经理}
\label{\detokenize{chapter_introduction/PM:id20}}
运营型产品经理就需要对产品进行全局性思考,负责产品的整体规划和设计,并且能够独立完成产品的一系列策划工作,同时还需要考虑产品的后续运营和拓展。所以运营型产品经理不仅仅需要考虑产品实现,还要考虑产品市场以及运营。正所谓产品和运营不分家,产品决定运营的宽度,运营决定产品的深度。

运营型产品经理就需要我们有很多行业知识的积累和思考,不仅要懂产品、懂用户体验,还要懂市场、懂运营、懂商务等。充分了解产品的市场和运营,可以帮助产品经理规划和设计出更符合实际需求的产品,避免了闭门造车。


\subparagraph{管理型产品经理}
\label{\detokenize{chapter_introduction/PM:id21}}
管理型产品经理就偏向于行政意义上的管理者了,比如产品部经理或者产品总监。管理型产品经理会对公司的产品线进行管理,沟通和协调公司资源,对接产品和业务,所以管理型产品经理有很强的战略思维和决断能力。通常这种职位会在大公司或者有多个产品经理的公司里出现,凡是公司里有很多产品经理,就会有管理型产品经理负责整体管理,担任产品部门的经理或总监,因此管理型产品经理不仅仅要具备功能型和运营型产品经理的职能,还要具备很强的团队和项目管理能力。


\paragraph{能力要求}
\label{\detokenize{chapter_introduction/PM:id22}}
软能力包括了最常提到的学习能力、执行能力、沟通能力、责任感、沟通表达能力、市场洞察能力、创新能力、影响力等等,这些能力是比较难以量化,需要通过具体项目推进去观察,带有一定的主观性。

硬能力包括了产品规划、需求调研、需求拟定(原型、需求文档等)项目管理、商务沟通、运营数据分析、市场营销等

\begin{figure}[H]
\centering
\capstart

\noindent\sphinxincludegraphics{{PM}.jpg}
\caption{PM能力模型}\label{\detokenize{chapter_introduction/PM:id68}}\end{figure}


\paragraph{工作主线}
\label{\detokenize{chapter_introduction/PM:id23}}
主线是围绕产品从0\sphinxhyphen{}1\sphinxhyphen{}N全周期的具体推进。


\subparagraph{产品工作框架}
\label{\detokenize{chapter_introduction/PM:id24}}
Cobit框架: 规划\sphinxhyphen{}》设计\sphinxhyphen{}》研发\sphinxhyphen{}》发布\sphinxhyphen{}》监控

\begin{center}\sphinxincludegraphics{{product_process}.png}\end{center} \sphinxincludegraphics{{PM_process}.jpg} \sphinxincludegraphics{{PM_process_mindmap}.png}


\subparagraph{工作内容 2\sphinxfootnotemark[184]}
\label{\detokenize{chapter_introduction/PM:id25}}%
\begin{footnotetext}[184]\sphinxAtStartFootnote
\sphinxnolinkurl{https://www.zhihu.com/question/343743405/answer/1237754321s}
%
\end{footnotetext}\ignorespaces \begin{enumerate}
\sphinxsetlistlabels{\arabic}{enumi}{enumii}{}{.}%
\item {} 
做行业洞察和市场调研,分析行业和产品的发展趋势,友商的竞品分析和客户的需求分析等,输出MRD,需求用例评审。

\item {} 
根据MRD结合公司现有的技术积累、公司战略方向、客户痛点需求和市场销售预期写PRD。
\begin{enumerate}
\sphinxsetlistlabels{\arabic}{enumii}{enumiii}{}{.}%
\item {} 
先分析业务,整理出需求用例文档,需求用例评审通过\sphinxhref{https://www.zhihu.com/question/36913495/answer/252737063}{6}%
\begin{footnote}[185]\sphinxAtStartFootnote
\sphinxnolinkurl{https://www.zhihu.com/question/36913495/answer/252737063}
%
\end{footnote}

\item {} 
用 Axure 制作原型图,原型图评审通过

\item {} 
用 PhotoShop 做出效果图,效果评审通过

\item {} 
切图出素材,再然后开始做软件架构设计,架构评审通过

\end{enumerate}

\item {} 
推动研发的开发和资源投入,项目管理(制定计划并跟踪、确定资源投入、把控质量,写周报等汇报),产品生命周期管理等

\item {} 
负责产品的推广策略、要写一堆的产品推广资料

\item {} 
负责产品经营性工作,要负责产品营销策略和产品销售业绩,所以经常要做产品经营性数据分析

\item {} 
培训、拜访客户、挖坑、填坑balabala…..等其他非核心内容工作。

\end{enumerate}


\paragraph{产品经理的角色理解 5\sphinxfootnotemark[186]}
\label{\detokenize{chapter_introduction/PM:id26}}%
\begin{footnotetext}[186]\sphinxAtStartFootnote
\sphinxnolinkurl{https://www.zhihu.com/question/31636227/answer/1251352264}
%
\end{footnotetext}\ignorespaces 
产品经理不做具体的开发工作,只是规划产品的功能和发展方向,然后去协调UI、UE、前端、开发、测试等部门,一起协同完成产品的开发。从这个意义上讲,产品经理是做协调工作的

首先我们要明确的一件事是:虽然称为产品经理,但产品经理是没有管理权限的,也就是说产品经理在公司几乎不能要求别人做什么事情,而只能是协调他人做什么事。

弄清楚了这一点,我们再来看产品经理在公司的角色,就可以归结为协调者。所谓协调者,可以从以下几个方面来理解:


\subparagraph{信息的协调者}
\label{\detokenize{chapter_introduction/PM:id27}}
在前面介绍产品经理做什么的时候,也说到产品经理会接触公司大部分的部门,因此产品经理就会收集到这些部门与自身产品相关的信息。例如产品经理可以从公司领导那里获得产品战略发展的信息;可以从UI那里那里获得LOGO含义的信息;可以从开发那里获得产品底层框架的信息,等等。当这些信息达到产品经理手里时,并不是信息的终结,而是信息分析与传递的开始。产品经理需要将这些信息转化,转化成大家需要且易懂的信息,进而再传递给需要的成员。从这个意义上讲,产品经理在公司更多扮演了信息收集者和传递者的角色。


\subparagraph{资源的协调者}
\label{\detokenize{chapter_introduction/PM:id28}}
虽然说产品经理手里没有管理权,但却在很大程度上决定产品的发展,因此产品经理可以发挥影响力来协调广泛的资源。我们都知道,产品经理需要和公司领导、UI、前段、开发、测试、客服等部门进行协调,而这些部门同事的工作基本上也都是围绕着产品经理展开的,所以两者之间是一种相互依存的关系。

在这种情况下,产品经理就可以根据产品计划来协调资源。不过,这里非常考验产品经理协调资源的能力,尤其是在产品经理手里有若干项目,或者有若干个产品经理要共享有限的资源的情况下,这时候协调的好与坏,直接决定了项目的进度与效率。

再上升一个层次看产品经理的角色,其手里可能握有产品的生杀大权。也就是说,产品经理可能会决定一个产品的成与败,一个优秀的产品经理可以化腐朽为神奇,成为人们心中的大咖,而不好的产品经理却可能化神奇为腐朽,将产品和团队带入迷茫之中。

对于很多产品小白而言,可能做的更多还是领导指派的具体事务,不过只要保持进步,终有一天会成为中流砥柱,而如果你已经小有成就,对产品也需要抱有敬畏之心,因为世界变化太快,成败往往就一瞬之间的事情。


\paragraph{产品经理接触的人}
\label{\detokenize{chapter_introduction/PM:id29}}
分两部分来说:产品规划与产品开发。


\subparagraph{就产品规划而言,产品经理接触到的人包括但不限于:}
\label{\detokenize{chapter_introduction/PM:prod-people}}\label{\detokenize{chapter_introduction/PM:id30}}
\sphinxstylestrong{互联网公司职位分为这几种:}
\sphinxhref{https://www.zhihu.com/question/26043439/answer/873138501}{39}%
\begin{footnote}[187]\sphinxAtStartFootnote
\sphinxnolinkurl{https://www.zhihu.com/question/26043439/answer/873138501}
%
\end{footnote}
\begin{itemize}
\item {} 
三大必备职位:技术、运营、产品。

\item {} 
三大辅助性职位:UI、测试、市场。

\item {} 
三大支持性职位:客服、行政、总经办。

\end{itemize}

1)直线领导:

当我们做产品规划时,必然要和直线领导就方案达成共识,才能进一步向外沟通确认,因此在产品规划阶段,你需要频繁地与直线领导沟通或汇报(有时候直线领导可能不参与具体讨论,但需要知道进度)。

2)公司领导

有时候,公司领导可能是某个需求的提出者。这种情况下,产品经理(或直线领导)需要向公司领导汇报相关解决方案。

3)业务人员

如果你负责的产品有业务人员的话,那他们也是产品重要的需求方,同时他们在与客户接触中,会出现种种问题。这个时候,都需要产品经理参与解决。

4)客服人员

针对产品规划,客服人员反馈的用户数据尤为重要,因此产品经理需要频繁地与客服人员进行沟通,搜集数据,整理并转化为需求。

5)用户

用户研究是产品规划阶段的核心工作之一,也是产品经理难得的接触真正用户的机会。在这个阶段中,产品经理可以采用用户访谈、调查问卷、可用性测试等方式,多多与用户进行接触。


\subparagraph{就产品开发而言,产品经理接触的人包括但不限于:}
\label{\detokenize{chapter_introduction/PM:id31}}
1)UI/UE

当产品原型最终确定,就可以进入UI设计(多为GUI)阶段,这个时候产品经理就需要和UI探讨原型细节,进入设计阶段。用户界面是系统和用户之间进行交互和信息交换的媒介,它实现息的内部形式与人类可以接受形式之间的转换。体验其实也就是一系列感官的综合。

\begin{figure}[H]
\centering
\capstart

\noindent\sphinxincludegraphics{{UX}.png}
\caption{UX}\label{\detokenize{chapter_introduction/PM:id69}}\end{figure}

2)前端

UI设计完成后,就开始转入前端工作。对于前端而言,会更加关注细节,每一个按钮的状态变化,每一个交互细节,都需要详细说明。这块一般是由产品经理和UI共同提供的。

不过如果是移动端产品,前端基本上就不太会参与,页面切图和标注工作主要是由UI完成。

3)开发

开发的工作主要是参照需求文档来展开的,因此产品经理需要就需求文档细节与开发进行充分沟通,以保证开发工作的有效性。
\begin{itemize}
\item {} 
研发经理:研发经理是技术研发管理职位,负责了解项目的需求,系统分析,做相关的技术选型,制定开发计划与开发规范。

\item {} 
架构师:架构师是软件系统和网络系统的设计师,负责确认和评估产品需求、搭建软件研发和网络系统的核心构架、并扫清主要难点。架构师着眼于“技术实现”,能对常见场景快速给出最恰当的技术解决方案,并能评估团队实现功能需求的代价。架构师分为软件架构师和系统架构师两类,分别专注于软件开发和系统运维两个阶段的系统设计。

\item {} 
Web前端工程师:Web前端工程师是界面研发职位,负责根据架构设计文档和界面设计稿,使用Web技术(HTML/CSS/JavaScript等)进行Web产品界面开发,并调用Server端接口实现Web应用。

\item {} 
APP开发工程师:APP开发工程师是APP界面研发职位,负责根据需求文档和界面设计稿开发出APP客户端界面,并调用Server端接口实现APP应用

\end{itemize}

4)测试

开发完成了项目工作,就进入了测试阶段。一般情况下,测试人员会在开始之前召开测试用例评审,然后才进入具体的测试阶段。无论是测试用例编写阶段,还是测试阶段,执行测试任务、提交测试Bug、跟进Bug修正,产品经理都是要与测试充分沟通的。

如果把产品经理比作“爸爸”,开发比作“妈妈”,那么测试就是“产检医生”,产品能不能健康出生、茁壮成长,关键看大夫的能力和责任心。当然,还有一个重要因素是“妈妈”不能太不负责任,在怀孕期间太任性,大吃大喝、喝酒抽烟、剧烈运动,完全不顾孩子死活,即使大夫再牛逼也无回天之力。\sphinxhref{http://dadaghp.com/index/index/article\_detail/mikuai/wenzhang/id/314.html}{55}%
\begin{footnote}[188]\sphinxAtStartFootnote
\sphinxnolinkurl{http://dadaghp.com/index/index/article\_detail/mikuai/wenzhang/id/314.html}
%
\end{footnote}

事实上,项目开发的工作是阶段性的,但产品经理与团队的接触则是全程的。从需求的发生,到项目的上线,产品经理都需要与UI、前端、开发、测试等人员充分接触,对产品需求进行沟通评估。


\subparagraph{在生活中锻炼产品规划 21\sphinxfootnotemark[189]}
\label{\detokenize{chapter_introduction/PM:id32}}%
\begin{footnotetext}[189]\sphinxAtStartFootnote
\sphinxnolinkurl{https://weread.qq.com/web/reader/46532b707210fc4f465d044k4e73277021a4e732ced3b55}
%
\end{footnotetext}\ignorespaces 
例子:小曹在北京的互联网中心上班,每到下班的时候,就会有大量的人从各个写字楼里“喷涌而出”,场面非常壮观。这些人有两个比较大的流量集散点,一个是公交站,另一个是地铁站,小曹就是在公交站等车群众中的一员。小曹边等车边思考,如何能够做一个产品来解决这个片区的人流拥堵问题呢。小曹想过公交信息查询产品,想过共享巴士产品,想过商圈引流产品,这些产品要么属于伪需求产品,要么产品路径冗长,要么没有清晰的商业模式。
\begin{enumerate}
\sphinxsetlistlabels{\arabic}{enumi}{enumii}{}{.}%
\item {} 
非常多的行业帮你建立“场景思维”。你可以通过不同场景的串联、不同行业的特点,看到用户的很多需求是如何被满足的。

\item {} 
真的用户:大多数产品经理都是在办公室里做产品规划的,或者做一些竞品的功能截图。这都不如来到用户身边感受得更深刻。

\item {} 
丰富真实的用户体验。在银行、医院排队的焦虑、很多线下场景的烦琐流程、很多设备的交互体验做得不够完美、很多客服对待用户不友好。

\end{enumerate}


\subparagraph{产品经理的交流}
\label{\detokenize{chapter_introduction/PM:id33}}
当产品经理与运设技一对一两个团队交流时,其实是六个方面在交流:
\begin{enumerate}
\sphinxsetlistlabels{\arabic}{enumi}{enumii}{}{.}%
\item {} 
产品经理以为的产品

\item {} 
产品经理以为的运设技(运营、设计、技术)

\item {} 
真正的产品

\item {} 
运设技以为的产品

\item {} 
运设技以为的产品经理

\item {} 
真正的运设技

\end{enumerate}

主观与客观、产品、产品经理、运设技


\paragraph{0\sphinxhyphen{}1/1\sphinxhyphen{}∞分类 25\sphinxfootnotemark[190]}
\label{\detokenize{chapter_introduction/PM:id34}}%
\begin{footnotetext}[190]\sphinxAtStartFootnote
\sphinxnolinkurl{https://www.yinxiang.com/everhub/note/96c994d6-c748-419e-8d3e-eeef2c929f4d}
%
\end{footnotetext}\ignorespaces 
\begin{figure}[H]
\centering
\capstart

\noindent\sphinxincludegraphics{{0_1_∞}.png}
\caption{时代与产品经理}\label{\detokenize{chapter_introduction/PM:id70}}\end{figure}

\begin{figure}[H]
\centering
\capstart

\noindent\sphinxincludegraphics{{PM_naotu}.png}
\caption{\sphinxhref{https://naotu.baidu.com/file/20572456d256fb1718cfeb645cf41b5f}{产品经理脑图实战}\sphinxfootnotemark[191]}\label{\detokenize{chapter_introduction/PM:id71}}\end{figure}
%
\begin{footnotetext}[191]\sphinxAtStartFootnote
\sphinxnolinkurl{https://naotu.baidu.com/file/20572456d256fb1718cfeb645cf41b5f}
%
\end{footnotetext}\ignorespaces 

\subparagraph{需求挖掘 25\sphinxfootnotemark[192]}
\label{\detokenize{chapter_introduction/PM:id35}}%
\begin{footnotetext}[192]\sphinxAtStartFootnote
\sphinxnolinkurl{https://www.yinxiang.com/everhub/note/96c994d6-c748-419e-8d3e-eeef2c929f4d}
%
\end{footnotetext}\ignorespaces 
需求挖掘,也可以称作产品定义、从 0 到
1、模式创新等等,这类是在新要素到来时寻找巨大体验差空间的角色

在三个要素接踵而至的创业红利期,第一代产品经理做的大多是需求挖掘,而且一旦挖准,这些人也大都成为了成功创业者甚至巨头企业老板。

真正定义产品的,其实是早期产品经理或创始团队。甚至像美团的战略思路,产品模型都是要找现成的,产品研发和业务团队的职责,就是让成本和效率做到极致,逼死竞争对手。

创业红利期,产品经理可以通过印证用户需求来证明自己能力,边际成本是很低的,比如要基于
QQ 做 QQ
秀,或要基于百度搜索做百度贴吧,是产品经理可以驱动的事情。一旦成功就能奠定地位。


\subparagraph{关注效率成本}
\label{\detokenize{chapter_introduction/PM:id36}}
关注效率成本,从体验、从交互、从增长、从问题拆解、从项目推进等维度,去实现产品效率成本的优化,不改变产品模型和业务模式。

而红利消失后,绝大多数产品经理就成为了螺丝钉,哪怕在某些公司称为 CPO 或
VP
的产品经理,也是带领产品团队做效率成本优化的角色,而非定义产品的角色。

在螺丝钉时代,产品经理往往不是定义而是优化,那就变成跟运营一样追求业务指标的角色,更多是用
KPI
证明自己的能力,且要在项目中跟运营、市场等争取自己的决策权和话语权。或者换个视角说,没有运营和业务的配合协同,螺丝钉产品经理很难独立达成业务目标。

这两年常被半开玩笑说起的,古典产品经理的时代结束了,其实就是指“做定义”的角色消失,全部褪去光环,成为跟运营一样“做经营”的角色(甚至有时不如运营)。

哪怕经常被人讲说唱衰行业制造焦虑,我还是要坚持这么讲。未来相当长期的一段时间里,各行各业需要的,更多就是\sphinxstylestrong{运营}一样的螺丝钉角色来制作产品,甚至有的公司就叫产品运营或者业务经理,title
已然不重要了。


\paragraph{偏技能/管理分类}
\label{\detokenize{chapter_introduction/PM:id37}}

\subparagraph{技能型产品经理}
\label{\detokenize{chapter_introduction/PM:id38}}\label{\detokenize{chapter_introduction/PM:id39}}
所谓技能型产品经理,就是对某个特定领域有很深的研究,具有较高的专业门槛。为了更直观地了解技能型产品经理,我们来看一则招聘广告:

职位描述:
\begin{itemize}
\item {} 
负责京准通(京东广告平台)创意审核系统,AI方向的优化升级相关工作;

\item {} 
从AI审核、人工审核、创意自动化等多个方向出发,提出优化改进方案,
最终实现审核时效及审核通过率的提升;

\item {} 
AI在广告投放平台的其他应用试验:包含效果优化,预算控制等。

\item {} 
了解行业整体发展趋势,定期对相关竞品进行跟踪和分析;
关注产品运营数据和用户反馈,深入发掘用户的需求,持续改进产品。

\end{itemize}

任职要求:
\begin{itemize}
\item {} 
熟悉互联网精准广告的投放流程,具备互联网商业变现或者广告行业工作经验者优先;有AI相关工作经验的优先

\item {} 
良好的需求分析、数据分析、产品设计能力,熟悉产品设计工作流程;

\item {} 
优秀的沟通协调能力,整合各相关团队资源,推动跨团队合作。
以上是京东商城招聘AI广告产品经理的招聘信息。从信息中,我们可以看到,对产品经理的要求几乎都是关于AI方面。对于此类工作,如果没有深厚的专业知识和行业经验,是很难胜任的。

\end{itemize}


\subparagraph{管理型产品经理}
\label{\detokenize{chapter_introduction/PM:id40}}
相比较技能型产品经理,管理型产品经理的要求更多偏向于规划、协调等方面。同样,我们来看下面招聘信息:

职位描述:
\begin{itemize}
\item {} 
负责规划、设计、运营管理产品,架构专车B:raw\sphinxhyphen{}latex:\sphinxtitleref{C端产品系统};

\item {} 
根据每个阶段的业务目标,确立需求的优先级,满足业务每个阶段的人员效率要求,支持业务快速发展;

\item {} 
负责具体系统项目的计划、需求和产品文档撰写,详细阐述产品功能和操作流程;

\item {} 
跟进协调与支持产品相关的技术团队完成产品开发任务,保证按时上线。

\end{itemize}

任职要求:
\begin{itemize}
\item {} 
5年以上互联网产品设计经验,有丰富的系统设计或独立业务经验的产品架构师优先;

\item {} 
良好的逻辑思维能力、系统思维和广阔的业务视野;

\item {} 
良好的表达能力、沟通能力、抗压能力和团队管理能力;

\item {} 
富有激情和强烈的创新意识和团队合作。

\end{itemize}


\paragraph{大厂VS咨询VS创业 11\sphinxfootnotemark[193]}
\label{\detokenize{chapter_introduction/PM:vsvs-11}}%
\begin{footnotetext}[193]\sphinxAtStartFootnote
\sphinxnolinkurl{https://www.bilibili.com/read/cv4579443/}
%
\end{footnotetext}\ignorespaces 

\subparagraph{大厂产品经理}
\label{\detokenize{chapter_introduction/PM:id41}}
以腾讯(商户管理)产品经理的工作职责,我们可以看到大厂的产品经理需要具备的关键技能体现在
4 方面:
\begin{enumerate}
\sphinxsetlistlabels{\arabic}{enumi}{enumii}{}{.}%
\item {} 
产品设计和运营能力

\item {} 
持续优化和运营能力

\item {} 
组织协调和跨部门协作能力

\item {} 
长期规划能力

\end{enumerate}

大厂产品经理需要具备的技能中,有 2 个关键技能非常值得大家注意:
\begin{enumerate}
\sphinxsetlistlabels{\arabic}{enumi}{enumii}{}{.}%
\item {} 
软技能

\end{enumerate}

在大公司,需要产品经理具备软技能,比如书写邮件能力、组织开会能力、整理会议纪要能力、协调资源能力。
\begin{enumerate}
\sphinxsetlistlabels{\arabic}{enumi}{enumii}{}{.}%
\setcounter{enumi}{1}
\item {} 
跨部门协作

\end{enumerate}

在大公司,各部门的职能划分非常细,比如市场、销售、运营推广、用户调研、市场调研都是由不同的部门来承接,所以大厂的产品经理在工作中,需要跟多个部门进行跨部门协作和协调,才能把产品顺利上线。


\paragraph{咨询公司产品经理 12\sphinxfootnotemark[194]}
\label{\detokenize{chapter_introduction/PM:id42}}%
\begin{footnotetext}[194]\sphinxAtStartFootnote
\sphinxnolinkurl{https://zhuanlan.zhihu.com/p/347994504}
%
\end{footnotetext}\ignorespaces \begin{enumerate}
\sphinxsetlistlabels{\arabic}{enumi}{enumii}{}{.}%
\item {} 
研究并理解客户的战略、商业模式,挖掘并揭示客户的痛点和诉求

\item {} 
帮助客户识别商业机会并建议实施方案

\item {} 
引导需求探寻和创新思考工作坊,产出客户认可的解决方案

\item {} 
创建并清楚展示方案蓝图,确保客户和交付团队理解并达成共识

\item {} 
定义关键目标、成功要素,识别风险、挑战、依赖和约束

\item {} 
有效引导和促进 Product
Owner、客户出资人、行业专家、技术团队、最终用户间的沟通和协作,保证产品从概念、到原型、到上线及运营的端到端交付

\end{enumerate}


\subparagraph{创业公司}
\label{\detokenize{chapter_introduction/PM:id43}}
创业公司的产品经理需要具备的关键技能

与大厂不同的是,创业公司产品经理的关键技能主要体现在 3 方面:
\begin{enumerate}
\sphinxsetlistlabels{\arabic}{enumi}{enumii}{}{.}%
\item {} 
领导力

\item {} 
魄力

\item {} 
凝聚力

\end{enumerate}

创业公司产品经理的工作职责有 4 个关键点:
\begin{enumerate}
\sphinxsetlistlabels{\arabic}{enumi}{enumii}{}{.}%
\item {} 
制定方向和策略

\end{enumerate}

在产品的初期,产品经理需要参与公司和产品愿景和规划的过程,从制定产品方向和策略开始,而不仅仅是考虑产品功能的设计。
\begin{enumerate}
\sphinxsetlistlabels{\arabic}{enumi}{enumii}{}{.}%
\setcounter{enumi}{1}
\item {} 
全流程参与

\end{enumerate}

创业公司的产品经理需要参与到产品的所有环节,比如从产品远景、规划、原型设计、交互设计、视觉设计、开发上线的每一个环节。
\begin{enumerate}
\sphinxsetlistlabels{\arabic}{enumi}{enumii}{}{.}%
\setcounter{enumi}{2}
\item {} 
发挥空间大

\end{enumerate}

创业公司的产品经理需要主动承担和负责产品的整个生命周期,凝聚团队成员协作,发挥空间较大。
\begin{enumerate}
\sphinxsetlistlabels{\arabic}{enumi}{enumii}{}{.}%
\setcounter{enumi}{3}
\item {} 
高风险

\end{enumerate}

大厂的产品可能是已经成型、上线、有一定数量的客户,但是创业公司的产品需要试错,并不知道产品推向市场以后的反应是怎样的,所以具有相对较大的风险。

模拟AI创业:\sphinxurl{https://blog.csdn.net/weixin\_45036344/article/details/95051856}


\paragraph{小白学习}
\label{\detokenize{chapter_introduction/PM:id44}}

\subparagraph{重心 50\sphinxfootnotemark[195]}
\label{\detokenize{chapter_introduction/PM:id45}}%
\begin{footnotetext}[195]\sphinxAtStartFootnote
\sphinxnolinkurl{https://www.bilibili.com/video/BV1it41137Xg?p=3}
%
\end{footnotetext}\ignorespaces \begin{enumerate}
\sphinxsetlistlabels{\arabic}{enumi}{enumii}{}{.}%
\item {} 
实操去落地:原型、文档、竞品分析、架构

\item {} 
把产品认知:从用户直观的好用好玩,来解构产品。京东、淘宝等产品的组成,数据流动关系、如何管理前台等。

\item {} 
学项目流程:了解团队的各个角色,如何配合,在不同阶段中重点把握,哪里有难点,哪里容易出现问题。区分开项目经理

\end{enumerate}


\subparagraph{基础}
\label{\detokenize{chapter_introduction/PM:id46}}

\subparagraph{视野}
\label{\detokenize{chapter_introduction/PM:id47}}\begin{itemize}
\item {} 
对各个行业的产品要了解。(比如:垂直电商也要了解电商平台。)

\item {} 
公司业务不止一种形态,加边缘业务。(电商、O2O、教育、咨询。。。)

\item {} 
现象级产品(比如:开心网很多用户又消亡的背后成败原因)

\item {} 
查资料(搜索引擎:谷歌,学习国外,像素级抄袭商业模式和产品形态,反copy\sphinxhref{https://www.zhihu.com/question/61037384}{52}%
\begin{footnote}[196]\sphinxAtStartFootnote
\sphinxnolinkurl{https://www.zhihu.com/question/61037384}
%
\end{footnote}:共享单车LimeBike、移动支付\sphinxhref{https://www.wsj.com/video/china/F83E17D1-0B64-4E43-AE68-A4A6F0B1D20E.html?mod=rss}{51}%
\begin{footnote}[197]\sphinxAtStartFootnote
\sphinxnolinkurl{https://www.wsj.com/video/china/F83E17D1-0B64-4E43-AE68-A4A6F0B1D20E.html?mod=rss}
%
\end{footnote};天涯为何不死;)

\item {} 
虎嗅App看新闻(回顾、分析、研究、扩展)

\end{itemize}


\subparagraph{表达能力}
\label{\detokenize{chapter_introduction/PM:id48}}\begin{itemize}
\item {} 
作为点子、观念的阐述者

\item {} 
活在聚光灯下,作为表演者,之后给掌声或臭鸡蛋。优缺点会放大,条理不清晰。

\item {} 
原型的评审:你站在前面讲产品,下面的指出我的问题。情绪化的人还是成熟的人?

\item {} 
台下有很多人,节奏不自然。。每天都要演讲,调整好心态和情绪。

\end{itemize}


\subparagraph{技术开发}
\label{\detokenize{chapter_introduction/PM:id49}}\begin{itemize}
\item {} 
代码上的区别:前端(浏览器中解析呈现:HTML, CSS,
JS等标记语言)、后端(在服务器中运行:jsp、javaBean
、dao层、controller层和service层等业务逻辑代码,还有数据库)

\item {} 
展现形式的区别:前台(只用户不能进行登录就可以看到的内容、页面,就像百度首页或者其他站点、博客、企业官网等等一样,是呈现给用户的视觉和基本的操作。)、后台(指程序的使用人员,管理人员经过密码或其他验证手段之后才可看到的内容,往往可以进行一些操作,比如发布文章,填写工作日报,数据的增删查改操作等等。
用户浏览网页时,我们看不见的后台数据跑动。后台包括前端,后端。)

\item {} 
训练模型的区别:动态训练(采用在线训练方式。也就是说,数据会不断进入系统,我们通过不断地更新系统将这些数据整合到模型中。)、静态训练(采用离线训练方式。也就是说,我们只训练模型一次,然后使用训练后的模型一段时间。)

\item {} 
模型推理的区别:静态推理(离线推理,是使用 MapReduce
或类似方法批量进行所有可能的预测。然后,将预测记录到 SSTable 或
Bigtable
中,并将它们提供给一个缓存/查询表。)动态推理(在线推理,是使用服务器根据需要进行预测。)\sphinxhref{https://developers.google.com/machine-learning/crash-course/static-vs-dynamic-inference/video-lecture?hl=zh-cn}{53}%
\begin{footnote}[198]\sphinxAtStartFootnote
\sphinxnolinkurl{https://developers.google.com/machine-learning/crash-course/static-vs-dynamic-inference/video-lecture?hl=zh-cn}
%
\end{footnote}

\item {} 
网站、域名、服务器、IP

\item {} 
缓存、接口、数据库、H5,JAVA, PHP

\end{itemize}


\subparagraph{更多技能及实践}
\label{\detokenize{chapter_introduction/PM:id50}}\begin{itemize}
\item {} 
逻辑思维:可用性、易用性:逻辑正确去保证解决问题。

\item {} 
基础的电脑操作

\item {} 
多学竞品分析,少学需求分析。

\item {} 
多学数据分析,少学人性分析。

\item {} 
多学布局设计,少学交互设计。

\item {} 
多学项目管理,少学用户体验。

\item {} 
多看发展历史,少看热门案例。

\item {} 
多画流程图,少画脑图。

\item {} 
多自己思考,少听专家。

\item {} 
多做练习,少看书。

\end{itemize}


\paragraph{分成三个层次:}
\label{\detokenize{chapter_introduction/PM:id51}}\begin{enumerate}
\sphinxsetlistlabels{\arabic}{enumi}{enumii}{}{.}%
\item {} 
对功能负责,就所谓做feature:根据业务方的需求主导项目,做出某个产品的功能,达到满足需求上线。

\item {} 
对产品负责。需要负责整个产品生命周期,从需求收集、需求调研、理解用户、洞察用户,到产品实现,验证发现新的问题去反馈,最终打造出一款非常好的产品。

\item {} 
对目标负责。目标导向,更好地去利用资源服务目标(资源并不一定是产品或者研发,也可能包括新的技术,新的资源新的商业模式,最终是服务于业务目标的)。

\end{enumerate}


\subparagraph{结果 3\sphinxfootnotemark[199]}
\label{\detokenize{chapter_introduction/PM:id52}}%
\begin{footnotetext}[199]\sphinxAtStartFootnote
\sphinxnolinkurl{http://www.woshipm.com/pmd/3945349.html}
%
\end{footnotetext}\ignorespaces \begin{enumerate}
\sphinxsetlistlabels{\arabic}{enumi}{enumii}{}{.}%
\item {} 
产品设计结果:高效快速的将需求产品化,面对同样问题或需求,更好的解决方案、更少的开发量、更快的上线。举例,用半年做出来的和用2个月做出来的同功能、扩展性、结果的东西,投资收益后者是前者的3倍,这之间的差值,是产品经理之间的差值。这里更多的强调是“把事情做对”,即事情分给你,可以以最高性价比的方式做出来,做好。

\item {} 
数据结果:用户对产品的使用情况,更准确、更多、更系统的挖掘用户的场景,系统性的解决场景背后的问题,并使得上线之后的产品得到更多用户的认可和使用。同样是花了2个月优化了某模块,有的产品经理可以让模块使用人数增2倍,有的产品经理只可以让模块使用人数提升20\%,有的甚至优化之后使用量还下降。这些数据之间的差值是产品经理之间的差值。

\item {} 
商业结果:一方面是短期带来的收入,B端的新签价值、续约价值,C端广告费,文章阅读费用等。另一方面是长期带来的战略布局价值,如产品矩阵的构建,产品架构支撑大客户的扩展,支撑在某个领域的布局等。

\end{enumerate}


\paragraph{产品思维与技术思维的区别 4\sphinxfootnotemark[200]}
\label{\detokenize{chapter_introduction/PM:id53}}%
\begin{footnotetext}[200]\sphinxAtStartFootnote
\sphinxnolinkurl{http://www.woshipm.com/pmd/1629952.html}
%
\end{footnotetext}\ignorespaces 
\begin{figure}[H]
\centering
\capstart

\noindent\sphinxincludegraphics{{tech_product0}.jpeg}
\caption{技术VS产品}\label{\detokenize{chapter_introduction/PM:id72}}\end{figure}
\begin{itemize}
\item {} 
\sphinxstylestrong{产品经理}思考的是产品的\sphinxstylestrong{用户价值和使用场景},同时还需要考虑产品所承载的\sphinxstylestrong{业务闭环及商业价值}

\item {} 
\sphinxstylestrong{工程师}看到产品设计后,在脑海里构建的是拆解后的技术实现要点,好比一栋房子的内部结构。对于一个产品,工程师需要先构建产品的技术架构,然后评估产品功能的技术成本。

\end{itemize}

\begin{figure}[H]
\centering
\capstart

\noindent\sphinxincludegraphics{{tech_product}.jpeg}
\caption{技术VS产品的分工}\label{\detokenize{chapter_introduction/PM:id73}}\end{figure}

\begin{figure}[H]
\centering
\capstart

\noindent\sphinxincludegraphics{{PM_vs_Engineer}.png}
\caption{技术VS产品的区别\sphinxhref{http://www.woshipm.com/pmd/3024508.html}{48}\sphinxfootnotemark[201]}\label{\detokenize{chapter_introduction/PM:id74}}\end{figure}
%
\begin{footnotetext}[201]\sphinxAtStartFootnote
\sphinxnolinkurl{http://www.woshipm.com/pmd/3024508.html}
%
\end{footnotetext}\ignorespaces 
产品经理是发现需求后做产品策略做对的产品,例如:当快手2011年开始上市场运营,而今日头条系从2016年才开始做抖音,那么如果你是技术思维的话,你准备研究比快手更好的AI模型?然后超越快手吗?

那我们看抖音的产品负责人士怎么运用产品思维做产品策略的。

首先AI技术模型全世界都是公开的,这一点从产品角度看没有门槛。

另外抖音的产品一下子发三款,分别是:
\begin{enumerate}
\sphinxsetlistlabels{\arabic}{enumi}{enumii}{}{.}%
\item {} 
跟快手一模一样的纯粹类UGC平台火山小视频;

\item {} 
较长视频西瓜视频平台;

\item {} 
做一款又类PGC优质内容的平台抖音,在同时从市场收购一款。2017年11月10日头条以10亿美元购北美音乐短视频社交平台Musical.ly,与抖音合并。

\end{enumerate}

如果头条是技术思维的话,通过技术逆向看Musical.ly源码,会不出意外发现我们也能做呀,我们技术比Musical.ly还好。

笔者建议以上思想想在AI时代做产品经理一定要买本《AI+时代产品经理的思维方法》一书,好好读读产品经理的本质是啥。

例如:上面的例子再分析,如果头条是技术思维抖音早就被2018年腾讯系的微视干死了,还哪里会等你慢慢开发一个Musical.ly。


\paragraph{我适合当产品经理吗 10\sphinxfootnotemark[202]}
\label{\detokenize{chapter_introduction/PM:id54}}%
\begin{footnotetext}[202]\sphinxAtStartFootnote
\sphinxnolinkurl{https://www.bilibili.com/video/BV1qv411B7J1}
%
\end{footnotetext}\ignorespaces \begin{enumerate}
\sphinxsetlistlabels{\arabic}{enumi}{enumii}{}{.}%
\item {} 
你要想上班

\item {} 
不轻松躺着赚钱

\item {} 
发展比稳定更重要

\item {} 
学历是影响因素

\item {} 
轻松还赚大钱不存在

\item {} 
想创业,产品是关键

\item {} 
性格偏中性些

\end{enumerate}


\paragraph{天赋 17\sphinxfootnotemark[203]}
\label{\detokenize{chapter_introduction/PM:id55}}%
\begin{footnotetext}[203]\sphinxAtStartFootnote
\sphinxnolinkurl{https://www.zhihu.com/question/22113339/answer/1418832617}
%
\end{footnotetext}\ignorespaces 
A 类:有深度思考能力或\sphinxstylestrong{超常同理心}

对产品经理来说,深度思考是指习惯思考事物背后的本质,且在同等条件下,对事物的洞察更深或更快。能深度思考的人很少见,但只有借助于深度思考,在微观场景和宏观背景下发现并理解事物的共性、差异性和各种因果关系,才能在这个现实世界中不断总结出规律和特点,提高未来决策和行为的成功率。

知人知面不知心,科学方法只能高效处理客观行为,行为背后的心理动机却无法确定和验证,这就需要产品经理带着同理心来工作。同理心是指能够站在别人的角度去思考,并准确地察觉和判断别人的感受。同理心是天赋本能,每个人多少都会有,后天也能通过刻意训练适度提高。当然,有超常同理心的人也很少见,但一旦有,做产品经理就极具优势。

世界上永远不会有两场相同的战争,产品经理也面临相同情况,永远要在变化的环境中去发现和解决新问题,这其实是一个要永远保持创造性的工作,如果产品经理的先天天赋占优,同等条件下的创造性和输出能力也会占优。

A
类产品经理很少见,这跟智商、经验、级别都不一定有关,更多是跟特殊天赋和潜力有关。事后分析一个产品或行业的得失和规律相对容易,很多人都能做得不错,但当产品和行业还处于结局不确定的发展过程中,就能更早更深地察觉到市场需求和行业方向的特质是企业最希望产品经理拥有的,这也是我们总在努力寻找
A 类产品经理的原因。

A 类人才里面当然也会有强弱之分,但是,只要符合 A
类标准就够了,甚至只符合 B
类标准,掌握了科学方法又经过充分实践历练,也够了。因为,对于大多数产品经理来说,创造成功产品的主要瓶颈还是机遇,如果能够抓住好的时代机遇,时代会推着你走。

潜力和优势来源: \sphinxhref{https://github.com/JoJoDU/Book\_Notes/issues/3}{18}%
\begin{footnote}[204]\sphinxAtStartFootnote
\sphinxnolinkurl{https://github.com/JoJoDU/Book\_Notes/issues/3}
%
\end{footnote}
\begin{itemize}
\item {} 
感兴趣的领域做到勤奋和自省

\item {} 
利他,替众人着想和想众人所想——市场导向型PM

\item {} 
产品实践经历

\end{itemize}


\paragraph{未来能成为优秀的产品经理}
\label{\detokenize{chapter_introduction/PM:id56}}\begin{enumerate}
\sphinxsetlistlabels{\arabic}{enumi}{enumii}{}{.}%
\item {} 
10\textasciitilde{}20w。目标不清晰,行动能力弱。

\item {} 
20\textasciitilde{}50w。目标清晰,行动能力强。

\item {} 
50w+。目标清晰,有干劲、胆量。

\end{enumerate}


\paragraph{“抄”,“超”,“钞” 19\sphinxfootnotemark[205]}
\label{\detokenize{chapter_introduction/PM:id57}}%
\begin{footnotetext}[205]\sphinxAtStartFootnote
\sphinxnolinkurl{https://wen.woshipm.com/question/detail/c5toar.html?sf=wipm}
%
\end{footnotetext}\ignorespaces \begin{itemize}
\item {} 
“抄”:就是抄袭,只有你研究的竞品和你现在做的业务差不多,那就直接抄,最起码人家做的这些在市场上已经验证了,用户也接收了,只要你理解了他的逻辑直接拿过来没什么问题。

\item {} 
“超”:既然抄袭了,总不能一辈子跟着后面走,产品上线后接收反馈就要有超越和优化的想法,有些地方确实用户不适合的话就需要懂脑子进行优化,超越你所抄袭的竞品。

\item {} 
“钞”:顾名思义就是钱了,只要产品做得好,肯定就可以给公司带来效益和价值,自然而然你也会得到更多的资源和奖励。

\end{itemize}


\paragraph{技术落地的周期 20\sphinxfootnotemark[206]}
\label{\detokenize{chapter_introduction/PM:id58}}%
\begin{footnotetext}[206]\sphinxAtStartFootnote
\sphinxnolinkurl{https://blog.csdn.net/pA2elX78qaJTADH/article/details/79989230?spm=1001.2014.3001.5502}
%
\end{footnotetext}\ignorespaces 
技术落地的一个必然周期,第一波是谁能造出锤子,第二波是谁能用好有限的几把锤子,第三波才是当锤子足够多的时候(工具完备),弄清楚用户需要什么,再去想用那把锤子能搞定这个需求。


\paragraph{阶段}
\label{\detokenize{chapter_introduction/PM:id59}}\begin{itemize}
\item {} 
产品经理阶段:我自己在做这个岗位,也会服务产品经理同行。

\item {} 
产品思维阶段:我去服务泛产品经理,抽象出背后相对通用的思维方式,去影响更多人。

\item {} 
产品创新阶段:我认识到产品思维是方法,而产品创新是目的,更直接地,从想到做,从思维方式到做事方法,更落地。

\end{itemize}


\paragraph{PM 十问 35\sphinxfootnotemark[207]}
\label{\detokenize{chapter_introduction/PM:pm-35}}%
\begin{footnotetext}[207]\sphinxAtStartFootnote
\sphinxnolinkurl{https://coffee.pmcaff.com/article/2628979102597248/pmcaff?utm\_source=forum}
%
\end{footnotetext}\ignorespaces \begin{enumerate}
\sphinxsetlistlabels{\arabic}{enumi}{enumii}{}{.}%
\item {} 
产品要解决什么问题?(产品价值)

\item {} 
为谁解决这个问题?(目标市场)

\item {} 
成功的机会有多大?(市场规模)

\item {} 
怎样判断产品成功与否?(度量指标或收益指标)

\item {} 
有哪些同类产品?(竞争格局)

\item {} 
为什么我们最适合做这个产品?(竞争优势)

\item {} 
时机合适吗?(市场时机)

\item {} 
如何把产品推向市场?(营销组合策略)

\item {} 
成功的必要条件是什么?(解决方案要满足的条件)

\item {} 
根据以上问题,给出评估结论。(继续或放弃)

\end{enumerate}


\paragraph{职业病37\sphinxfootnotemark[208]}
\label{\detokenize{chapter_introduction/PM:id60}}%
\begin{footnotetext}[208]\sphinxAtStartFootnote
\sphinxnolinkurl{https://www.zhihu.com/question/19657029/answer/1699164788}
%
\end{footnotetext}\ignorespaces \begin{enumerate}
\sphinxsetlistlabels{\arabic}{enumi}{enumii}{}{.}%
\item {} 
对陌生人天然的跪舔:平时舔客户太多了,人人都是爷,平时遇到陌生人自然地伸出手自我介绍,赔笑,跪舔

\item {} 
对钱的绝对敏感:啥事都爱问“how
much”,女朋友买了件衣服问我好不好看,我永远回答“多少钱买的”

\item {} 
能动嘴绝不动手:产品经理一般都是下达指令的,在家也一样。。。基本上不自己干活,喜欢致使别人干活

\item {} 
对于deadline的绝对执着:无论啥事,最后都会问一句“什么时候搞定”,因为在工作中背负太多压力,不确定交付日期的事不做

\item {} 
缺乏安全感:工作中被开发坑惯了。。。啥事都爱问“确定能做么”,连物业帮忙修马桶,也要反复确认“能修好么”

\item {} 
对钱的绝对敏感:啥事都爱问“how
much”,女朋友买了件衣服问我好不好看,我永远回答“多少钱买的”

\item {} 
爱热闹爱协同:平时工作的时候各种协同各种共创,平时生活里也爱热闹,愿意组织大型聚会

\end{enumerate}


\paragraph{转行 39\sphinxfootnotemark[209]}
\label{\detokenize{chapter_introduction/PM:id61}}%
\begin{footnotetext}[209]\sphinxAtStartFootnote
\sphinxnolinkurl{https://www.zhihu.com/question/26043439/answer/873138501}
%
\end{footnotetext}\ignorespaces 
培训机构像产品手记、黑马程序员。\sphinxhref{https://zhuanlan.zhihu.com/p/213734104}{产品经理培训的坑在哪里?}%
\begin{footnote}[210]\sphinxAtStartFootnote
\sphinxnolinkurl{https://zhuanlan.zhihu.com/p/213734104}
%
\end{footnote}

互联网行业也在转行:

在互联网行业内转行情况有两种,要么是遇上了职业瓶颈,要么这个职位实在太累了。

技术,测试,UI的职业瓶颈期在28岁。到了这个年龄就已经是高龄。找工作很难找了。尤其是到了三十岁更没有公司要了。因为这种职位加班严重,到三十岁后精力很难保证加班。多数人精力已经不足。跳槽基本不可能拿高薪。这类职业转产品的关键在于累。对于不想从事这么累的同学来说,转行的两个方向只能是产品和运营,但多数人无一例外选择了产品。因为产品相比较而言要比运营薪资更高些。

运营转产品也很多,关键原因并非运营简单,而是薪资提不上去。运营提薪资比产品难得多,但干活却比产品还要多。


\paragraph{能力模型}
\label{\detokenize{chapter_introduction/PM:id62}}
\begin{figure}[H]
\centering
\capstart

\noindent\sphinxincludegraphics{{PM_ability}.png}
\caption{产品经理能力模型}\label{\detokenize{chapter_introduction/PM:id75}}\end{figure}


\subparagraph{Baidu}
\label{\detokenize{chapter_introduction/PM:baidu}}
\begin{figure}[H]
\centering
\capstart

\noindent\sphinxincludegraphics{{baidu_PM}.png}
\caption{百度\sphinxhyphen{}产品经理能力模型\sphinxhref{https://g.yuque.com/zhongguodianxinyanjiuyuan/bgso10/xawnsb}{45}\sphinxfootnotemark[211]}\label{\detokenize{chapter_introduction/PM:id76}}\end{figure}
%
\begin{footnotetext}[211]\sphinxAtStartFootnote
\sphinxnolinkurl{https://g.yuque.com/zhongguodianxinyanjiuyuan/bgso10/xawnsb}
%
\end{footnotetext}\ignorespaces 
百度产品经理的职级从P3开始,至P8+。相比于鹅厂(工作能力、专业知识、专业技能、组织影响力)衡量产品经理的维度,百度衡量产品经理的维度变成了三个:软能力、硬能力、公共基础。看起来比较简洁,更接近我们平时对产品经理的认知。


\paragraph{产品经理成熟的标准是什么? 16\sphinxfootnotemark[212]}
\label{\detokenize{chapter_introduction/PM:id63}}%
\begin{footnotetext}[212]\sphinxAtStartFootnote
\sphinxnolinkurl{https://zhuanlan.zhihu.com/p/38392075}
%
\end{footnotetext}\ignorespaces 
即便团队对他们没要求,他们依然会懂技术、懂设计、懂营销、懂商业、懂管理、懂业务、懂心理。

PM最终拼的是人文素养和灵魂境界,而不是职位名称、从业年数、名校背景。

看他做一款创新型产品时,更依赖竞品调研还是独立判断。站在巨人的肩膀上是没错,但前瞻性的方案更依赖人性洞察和市场嗅觉。


\paragraph{生存报告}
\label{\detokenize{chapter_introduction/PM:id64}}
2020年产品经理生存报告:
\sphinxurl{https://coffee.pmcaff.com/article/KDLE41yRkx?rts=201105225049\_nch}


\paragraph{知识宇宙}
\label{\detokenize{chapter_introduction/PM:id65}}
\begin{figure}[H]
\centering
\capstart

\noindent\sphinxincludegraphics{{PM_knowledge}.png}
\caption{PM知识宇宙\sphinxhref{https://g.yuque.com/zhongguodianxinyanjiuyuan/bgso10/ab5ucf}{46}\sphinxfootnotemark[213]}\label{\detokenize{chapter_introduction/PM:id77}}\end{figure}
%
\begin{footnotetext}[213]\sphinxAtStartFootnote
\sphinxnolinkurl{https://g.yuque.com/zhongguodianxinyanjiuyuan/bgso10/ab5ucf}
%
\end{footnotetext}\ignorespaces 

\paragraph{更多}
\label{\detokenize{chapter_introduction/PM:id66}}
社区:
\begin{itemize}
\item {} 
UCD大社区: www.ucdchina.co

\item {} 
腾讯CDC: \sphinxurl{http://cdc.tencent.com}

\item {} 
淘宝UED: \sphinxurl{http://ued.taobao.com}

\item {} 
百度UED: \sphinxurl{http://ued.baidu.com/}

\item {} 
\sphinxurl{http://www.pmtalk.club/}

\item {} 
\sphinxurl{https://www.pmcaff.com/}

\item {} 
\sphinxurl{https://www.woshipm.com/}

\item {} 
\sphinxurl{https://dh.woshipm.com/\#section-16}

\item {} 
\sphinxurl{http://www.crazypm.com/}

\item {} 
\sphinxurl{https://pm-ren.com/}

\item {} 
\sphinxurl{http://beforweb.com/product}

\item {} 
\sphinxurl{http://masterchat.cn/}

\item {} 
\sphinxurl{http://tech2ipo.com/}

\item {} 
\sphinxurl{http://86pm.com/}

\item {} 
\sphinxurl{https://www.producthunt.com/}

\item {} 
\sphinxurl{http://ued.pm/}

\item {} 
\sphinxurl{http://www.masterchat.cn/}

\end{itemize}

导航:

\sphinxurl{http://www.pm265.com/}

信息:
\begin{itemize}
\item {} 
\sphinxurl{http://www.aihot.net/}

\item {} 
\sphinxurl{https://www.aiaor.com/}

\item {} 
\sphinxurl{http://wiki.jikexueyuan.com/list/product}

\item {} 
\sphinxurl{https://www.chanpingo.com/}

\item {} 
\sphinxurl{http://www.wordpm.com/}

\item {} 
\sphinxurl{http://www.pmtoo.com/}

\item {} 
\sphinxurl{http://www.chanpin100.com/}

\item {} 
\sphinxurl{https://www.qidianla.com/}

\item {} 
\sphinxurl{http://www.dengta360.cn/index.html\%22\%20\%5Ct\%20\%22\_blank}

\item {} 
\sphinxurl{https://www.mockplus.cn/}

\item {} 
\sphinxurl{http://www.managershare.com/}

\item {} 
\sphinxurl{http://www.geekpark.net/}

\item {} 
\sphinxurl{http://www.ipmtalk.com/}

\item {} 
\sphinxurl{https://t.qidianla.com/}

\item {} 
\sphinxurl{http://www.51pmexp.com/}

\item {} 
\sphinxurl{https://www.yuque.com/books/share/2325abf6-ed56-4941-bf99-94edeb122076}?\#\%20\%E3\%80\%8A\%E4\%BA\%A7\%E5\%93\%81API:\%E8\%BF\%9B\%E9\%98\%B6\%E5\%85\%A8\%E6\%A0\%88PM\%E6\%89\%8B\%E5\%86\%8C\%E3\%80\%8B

\item {} 
\sphinxurl{http://dadaghp.com/}

\item {} 
\sphinxurl{https://www.jianshu.com/u/c22ccc510fb9}

\end{itemize}

真题:柠檬two公众号

书籍:\sphinxhref{https://zhuanlan.zhihu.com/p/127373717}{产品经理必读商业思维与视野格局类书目(产品手记推荐) \sphinxhyphen{}
「已注销」的文章 \sphinxhyphen{} 知乎}%
\begin{footnote}[214]\sphinxAtStartFootnote
\sphinxnolinkurl{https://zhuanlan.zhihu.com/p/127373717}
%
\end{footnote}

心态:Stay Hungry ,Stay Foolish
\sphinxhref{https://zhuanlan.zhihu.com/p/268180702}{36}%
\begin{footnote}[215]\sphinxAtStartFootnote
\sphinxnolinkurl{https://zhuanlan.zhihu.com/p/268180702}
%
\end{footnote}


\subsubsection{AI}
\label{\detokenize{chapter_introduction/AI:ai}}\label{\detokenize{chapter_introduction/AI::doc}}
“软件正在吞噬世界,但AI会吃软件”,Jensen Huang(Nvidia首席执行官)

\begin{figure}[H]
\centering
\capstart

\noindent\sphinxincludegraphics{{AI}.jpg}
\caption{AI\sphinxhref{https://www.jiqizhixin.com/articles/2017-12-27-5}{14}\sphinxfootnotemark[216]}\label{\detokenize{chapter_introduction/AI:id33}}\end{figure}
%
\begin{footnotetext}[216]\sphinxAtStartFootnote
\sphinxnolinkurl{https://www.jiqizhixin.com/articles/2017-12-27-5}
%
\end{footnotetext}\ignorespaces 

\paragraph{历史}
\label{\detokenize{chapter_introduction/AI:id1}}
{\color{red}\bfseries{}|}AI历史{\color{red}\bfseries{}|}\sphinxhref{http://www.changgpm.com/thread-248-1-1.html}{12}%
\begin{footnote}[217]\sphinxAtStartFootnote
\sphinxnolinkurl{http://www.changgpm.com/thread-248-1-1.html}
%
\end{footnote}
{\color{red}\bfseries{}|}AI学派{\color{red}\bfseries{}|}\sphinxhref{http://ai.itheima.com/news/20191105/143608.html}{20}%
\begin{footnote}[218]\sphinxAtStartFootnote
\sphinxnolinkurl{http://ai.itheima.com/news/20191105/143608.html}
%
\end{footnote}


\paragraph{细分}
\label{\detokenize{chapter_introduction/AI:id10}}
\begin{figure}[H]
\centering
\capstart

\noindent\sphinxincludegraphics{{AI_class}.jpg}
\caption{AI类别}\label{\detokenize{chapter_introduction/AI:id34}}\end{figure}


\paragraph{Map}
\label{\detokenize{chapter_introduction/AI:map}}
\begin{figure}[H]
\centering
\capstart

\noindent\sphinxincludegraphics{{AI_map}.jpeg}
\caption{AI
map\sphinxhref{https://medium.com/swlh/the-map-of-artificial-intelligence-2020-2c4f446f4e43}{16}\sphinxfootnotemark[219]}\label{\detokenize{chapter_introduction/AI:id35}}\end{figure}
%
\begin{footnotetext}[219]\sphinxAtStartFootnote
\sphinxnolinkurl{https://medium.com/swlh/the-map-of-artificial-intelligence-2020-2c4f446f4e43}
%
\end{footnotetext}\ignorespaces 

\paragraph{分层}
\label{\detokenize{chapter_introduction/AI:id11}}
\begin{figure}[H]
\centering
\capstart

\noindent\sphinxincludegraphics{{AI_tech}.png}
\caption{AI科技\sphinxhref{https://cloud.tencent.com/edu/learning/live-2877}{22}\sphinxfootnotemark[220]}\label{\detokenize{chapter_introduction/AI:id36}}\end{figure}
%
\begin{footnotetext}[220]\sphinxAtStartFootnote
\sphinxnolinkurl{https://cloud.tencent.com/edu/learning/live-2877}
%
\end{footnotetext}\ignorespaces 
\begin{figure}[H]
\centering
\capstart

\noindent\sphinxincludegraphics{{AI_business}.png}
\caption{AI商业\sphinxhref{https://cloud.tencent.com/edu/learning/live-2877}{22}\sphinxfootnotemark[221]}\label{\detokenize{chapter_introduction/AI:id37}}\end{figure}
%
\begin{footnotetext}[221]\sphinxAtStartFootnote
\sphinxnolinkurl{https://cloud.tencent.com/edu/learning/live-2877}
%
\end{footnotetext}\ignorespaces 

\paragraph{优先级}
\label{\detokenize{chapter_introduction/AI:id12}}
\begin{figure}[H]
\centering
\capstart

\noindent\sphinxincludegraphics{{Tech_Priorty}.png}
\caption{技术优先级\sphinxhref{https://gw.alipayobjects.com/os/bmw-prod/6f1e0b5c-e068-49a6-bc0a-90d5e9131a72.pdf}{25}\sphinxfootnotemark[222]}\label{\detokenize{chapter_introduction/AI:id38}}\end{figure}
%
\begin{footnotetext}[222]\sphinxAtStartFootnote
\sphinxnolinkurl{https://gw.alipayobjects.com/os/bmw-prod/6f1e0b5c-e068-49a6-bc0a-90d5e9131a72.pdf}
%
\end{footnotetext}\ignorespaces 

\paragraph{类型 13\sphinxfootnotemark[223]}
\label{\detokenize{chapter_introduction/AI:id13}}%
\begin{footnotetext}[223]\sphinxAtStartFootnote
\sphinxnolinkurl{https://easyai.tech/blog/test-ai-with-hbi/}
%
\end{footnotetext}\ignorespaces \begin{itemize}
\item {} 
“肌肉T恤”——忽悠

\item {} 
“建美男”——AlphaGo作秀

\item {} 
“搏击手”——怎么实用怎么来

\end{itemize}


\paragraph{HBI原则}
\label{\detokenize{chapter_introduction/AI:hbi}}\begin{itemize}
\item {} 
高频(High frequency)

\item {} 
大数据(Big data)

\item {} 
没有规律(Irregular)

\end{itemize}


\paragraph{AI在其中扮演什么角色 6\sphinxfootnotemark[224]}
\label{\detokenize{chapter_introduction/AI:ai-6}}%
\begin{footnotetext}[224]\sphinxAtStartFootnote
\sphinxnolinkurl{https://www.zhihu.com/people/hanniman-2/posts?page=2}
%
\end{footnotetext}\ignorespaces 
AI革命可以看作是生产力的革命,从生产力的角度讲,第一是将人类从现实世界的非创造性劳动当中解放出来,从而更快速的向虚拟世界迁移;第二是赋予创造性劳动更低的门槛,以建设更丰富的虚拟世界。

这两点我总结为叫做对人类的“去工具化”,就是说,人之所以为人,是有人固有的价值,比如“想象力创造力”,“理解另一个人类需求的共情能力”。这些很难被机器替代。而人类完成自我实现,却需要掌握大量复杂工具,逐渐将自己培养成工具。比如,你有很好的想象力,却不会有画笔,也难以完成一幅画作。掌握画笔本身就是“工具化”。

但是我认为AI可以帮助人类实现“去工具化”,真正“身随意动”的发挥人之所以为人的价值,具体就是依靠上述两点。


\paragraph{AI思维}
\label{\detokenize{chapter_introduction/AI:id14}}
AI思维是一种全部数据化、全部用人工智能的方法实时的反映出来,并且根据这些数据进行调整的方式。\sphinxhref{https://www.chenpe.com/news/215513.html}{23}%
\begin{footnote}[225]\sphinxAtStartFootnote
\sphinxnolinkurl{https://www.chenpe.com/news/215513.html}
%
\end{footnote}

将弱人机+好流程胜过强人机+差流程的规律称为“卡斯帕罗夫定律”\sphinxhref{http://www.woshipm.com/ai/4416771.html}{24}%
\begin{footnote}[226]\sphinxAtStartFootnote
\sphinxnolinkurl{http://www.woshipm.com/ai/4416771.html}
%
\end{footnote}


\paragraph{可能优势 9\sphinxfootnotemark[227]}
\label{\detokenize{chapter_introduction/AI:id15}}%
\begin{footnotetext}[227]\sphinxAtStartFootnote
\sphinxnolinkurl{https://pair.withgoogle.com/chapter/user-needs/}
%
\end{footnotetext}\ignorespaces \begin{itemize}
\item {} 
向不同的用户推荐不同的内容。比如为电影提供个性化的建议。

\item {} 
对未来事件的预测。例如,显示11月下旬飞往丹佛的机票价格。

\item {} 
个性化改善了用户体验。随着时间的推移,个性化的自动家用恒温器使家庭更舒适,恒温器更高效。

\item {} 
自然语言理解。听写软件要求人工智能能够很好地适应不同的语言和说话风格。

\item {} 
对一整类实体的识别。把每一张脸都编程进照片标签应用程序是不可能的,它使用人工智能来识别同一个人的两张照片。

\item {} 
检测随时间变化的低发生事件。信用卡诈骗不断演变,很少发生在个人身上,但却经常发生在一大群人身上。人工智能可以学习这些不断演变的模式,并在出现新的欺诈类型时发现它们。

\item {} 
特定领域的代理或机器人体验。对于大量用户来说,酒店预订遵循类似的模式,并且可以实现自动化,以加快过程。

\item {} 
显示动态内容比显示可预测的界面更有效。来自流媒体服务的人工智能建议会显示用户几乎不可能找到的新内容。

\end{itemize}


\paragraph{可能劣势}
\label{\detokenize{chapter_introduction/AI:id16}}\begin{itemize}
\item {} 
保持可预测性。有时候,核心体验中最有价值的部分是其可预测性,而不考虑上下文或额外的用户输入。例如,当“Home”或“Cancel”按钮停留在相同的位置时,它更容易作为逃生通道使用。

\item {} 
提供静态或有限的信息。例如,信用卡输入表单是简单的、标准的,并且对于不同的用户没有非常不同的信息需求。

\item {} 
最小化代价高昂的错误。如果错误的代价非常高,超过了成功率的小幅提高带来的好处,比如导航指南建议一条越野路线,以节省几秒钟的旅行时间。

\item {} 
完整的透明度。如果用户、客户或开发人员需要准确地理解代码中发生的一切,就像开源软件一样。人工智能并不总是能提供那种程度的解释。

\item {} 
高速和低成本的优化。对于该业务来说,开发速度和率先进入市场是否比其他任何事情都重要,包括添加人工智能将带来的价值。

\item {} 
自动化的高价值的任务。如果有人明确告诉你,他们不想要一项由人工智能自动完成或增强的任务,这是一个不要试图破坏的好任务。我们将在下面更多地讨论人们如何评价某些类型的任务。

\end{itemize}


\paragraph{人工智能行业吐槽}
\label{\detokenize{chapter_introduction/AI:id17}}

\subparagraph{鱼龙混杂}
\label{\detokenize{chapter_introduction/AI:id18}}
伴随着行业持续火热,资金流不断涌入,现状却是整个行业内对技术、业务、商务都精通的产品大咖非常之少,滥竽充数的人很多。能够对行业的技术边界了然于胸,又对这个行业的产业链、利益链有深入理解的人才不可多得,大厂哄抢。有人戏称目前很多人工智能产品都是“人工智障”,可见该行业要实现真正的产业化、产品化,还有很大的空间。


\subparagraph{概念空洞}
\label{\detokenize{chapter_introduction/AI:id19}}
我曾笑称,进入这个行业真是感觉到中华文字的博大精深,把很多早就出现的技术名词玩文字游戏包装一下,突然就变得高大上起来了。天天张口闭口“动态时空库”、“计算引擎”、“一人一档”、“端到端解决方案”、“AI赋能”等等,其实稍微了解一下就发现,“动态时空库”不就是摄像头抓拍,“计算引擎”不就是服务器、“一人一档”不就是数据分组、“端到端解决方案”不就是软硬件都有、“AI赋能”不就是算法能力。但是这个行业就是这样的现状,只有包装了才有爆点,包装了才能融资,融资了才更需要噱头去营销,你也很难说这是良性循环,还是恶性循环。AI行业的最核心还是算法,传统研发人员会在算法这个盒子外面加一层包装,用所谓的云平台、互联网接口去封装,产品设计会在研发的基础上再加一层包装,解决方案会在产品基础上再加一层包装,当用户通过品宣与销售之口了解AI时已经在怀疑AI是不是快要取代人类了。所以才导致了大众认知和现实能力之间有巨大的鸿沟,目前的行业才不断的强调AI决胜在落地。只有有开创性的产品落地,才能弥补公众认知与现实能力的缺口。


\subparagraph{方案同质化}
\label{\detokenize{chapter_introduction/AI:id20}}
如果你稍微深入的了解过这个行业,你大概会与我有同样的想法,如果从非算法人员的角度来讲,这个行业的技术类别并没有那么复杂,相比于已经发展成熟的电力行业、电子行业、通信行业,其实它的知识宽度还算单纯,相对比较容易梳理清楚。再加上行业产品同质化严重,基本上这个行业的方案就是你抄我,我抄你,谁都说自己是首创,谁都从不同的角度去宣传自己是第一。很多概念也不知道是谁第一个提出,反正渐渐的就发现行业内各家都这么说。目前整个CV领域,基本上to
B和to G就集中在安防领域,to
C就集中在手机端的图像软件处理上以及金融认证比对上了,除此之外真的很难找到什么可圈可点的应用亮点。


\subparagraph{企图一蹴而就}
\label{\detokenize{chapter_introduction/AI:id21}}
他说:“每个组织都在关注的机会是拥有适应性系统的能力。”“这是一次旅行。这不是你能买到的东西,然后突然按下开关。按照人工智能的定义,它需要时间去学习。”


\paragraph{作用}
\label{\detokenize{chapter_introduction/AI:id22}}
韩国生物技术公司Seegene最近将人工智能技术用于开发新型冠状病毒的检测试剂盒。该公司报告称,人工智能将开发时间从几个月缩短到几周。据美国有线电视新闻网(CNN)报道,在疫情快速蔓延期间,韩国迅速部署急需的检测试剂盒,使其能够为本国公民提供免费检测,帮助遏制病毒的传播。\sphinxhref{https://www.productplan.com/ai-product-management/}{15}%
\begin{footnote}[228]\sphinxAtStartFootnote
\sphinxnolinkurl{https://www.productplan.com/ai-product-management/}
%
\end{footnote}


\paragraph{竞争优势}
\label{\detokenize{chapter_introduction/AI:id23}}
竞争优势来自于将AI应用到你的数据中,并创新你的商业模式。
\sphinxhref{https://www.productplan.com/ai-product-management/}{15}%
\begin{footnote}[229]\sphinxAtStartFootnote
\sphinxnolinkurl{https://www.productplan.com/ai-product-management/}
%
\end{footnote}


\paragraph{人工智能层次2\sphinxfootnotemark[230]}
\label{\detokenize{chapter_introduction/AI:id24}}%
\begin{footnotetext}[230]\sphinxAtStartFootnote
\sphinxnolinkurl{https://easyai.tech/blog/ai-pm-knowledge/}
%
\end{footnotetext}\ignorespaces 
\begin{figure}[H]
\centering
\capstart

\noindent\sphinxincludegraphics{{ceng}.jpg}
\caption{AI应用层、技术层、基础层}\label{\detokenize{chapter_introduction/AI:id39}}\end{figure}


\paragraph{人工智能几问3\sphinxfootnotemark[231]}
\label{\detokenize{chapter_introduction/AI:id25}}%
\begin{footnotetext}[231]\sphinxAtStartFootnote
\sphinxnolinkurl{https://www.sohu.com/a/364264851\_114819}
%
\end{footnotetext}\ignorespaces \begin{enumerate}
\sphinxsetlistlabels{\arabic}{enumi}{enumii}{}{.}%
\item {} 
人工智能和互联网时代的不同是什么?

\end{enumerate}

互联网主要是重构生产要素(即重构商业模式),人工智能则是升级生产要素。

比如在出行领域,出行平台直接连接了司机和乘客,重构了线上、线下的出行流程;但是人工智能则是从自动驾驶技术切入,重构了车辆和司机本身。
\begin{enumerate}
\sphinxsetlistlabels{\arabic}{enumi}{enumii}{}{.}%
\setcounter{enumi}{1}
\item {} 
人工智能没有普及的原因是什么?

\end{enumerate}

医疗领域、自动驾驶等,容错度低\sphinxhref{http://www.ramywu.com/work/2017/08/20/Product-Orientation/}{5}%
\begin{footnote}[232]\sphinxAtStartFootnote
\sphinxnolinkurl{http://www.ramywu.com/work/2017/08/20/Product-Orientation/}
%
\end{footnote},在准确率不够或样本不够多,满足不了安全需求,不敢普及。

计算特斯拉的事故率时,样本是很少的,对比基于整个社会上的车辆数和里程数。

只有等到特斯拉自动驾驶的车辆数和里程数积累到一定量级,样本足够大后,才能和人工驾驶的事故率进行比较,也才能真正证明自动驾驶是否更优于人工驾驶。
\begin{enumerate}
\sphinxsetlistlabels{\arabic}{enumi}{enumii}{}{.}%
\setcounter{enumi}{2}
\item {} 
AI
在什么场景下才能发挥出最大的作用?\sphinxhref{http://www.ramywu.com/work/2017/08/20/Product-Orientation/}{5}%
\begin{footnote}[233]\sphinxAtStartFootnote
\sphinxnolinkurl{http://www.ramywu.com/work/2017/08/20/Product-Orientation/}
%
\end{footnote}

\end{enumerate}

人工的优势是:可以解决创造性质的问题,复杂判断的问题。而 AI
的优势有哪些呢?在什么场景下才能发挥出最大的作用?

(1)数据量规模庞大,人工速度拼不过的时候,比如:
\begin{itemize}
\item {} 
在机场安防监控,肉眼一个个识别 拼不过 AI 人脸1:N快速识别;

\item {} 
快递行业尤其是双十一,每天都几百万的数据量,在做分拣时候,工业拍照扫描分拣和肉眼\sphinxhyphen{}
分拣都经常出错,10\%\sphinxhyphen{}20\%的出错率都会造成巨大的损失;

\item {} 
出版社、公众号编辑每天会处理大批量文字;

\end{itemize}

(2)简单且重复、精细的,人肉无法快速识别时,比如:

简单+重复:
\begin{itemize}
\item {} 
快递员每天都要发快递和联系收件人,而输入快递单里的手机号会很辛苦,内置系统通过快\sphinxhyphen{}
递单 OCR 识别能快速发送到联系人;

\item {} 
微信编辑写完文章还要人工做枯燥重复的文字检查,速度很慢,出错率高,急切需要提升文字的发布速度;

\end{itemize}

精细:
\begin{itemize}
\item {} 
检测人脸中两只眼睛的距离,机器是可以计算的,而肉眼做不到;

\item {} 
处理初级的错误,如形近字,肉眼也看不见如此微妙的错误;

\end{itemize}


\paragraph{在To B产品中可以替代人工劳动力的例子: 8\sphinxfootnotemark[234]}
\label{\detokenize{chapter_introduction/AI:to-b-8}}%
\begin{footnotetext}[234]\sphinxAtStartFootnote
\sphinxnolinkurl{http://www.crazypm.com/zixun/102296.html}
%
\end{footnotetext}\ignorespaces \begin{itemize}
\item {} 
腾讯觅影(\sphinxurl{http://t.cn/RYRDSmI} ):替代医生的部分职责;

\item {} 
百度Apollo(\sphinxurl{http://apollo.auto/} ):完全替代汽车驾驶员的职责;

\item {} 
商汤\sphinxhyphen{}公安人脸识别智能(\sphinxurl{http://t.cn/RYRD0zo}
):替代公安人员的部分职责;

\item {} 
网易七鱼\sphinxhyphen{}智能客服(\sphinxurl{http://t.cn/RYRDYwY} ):替代客服人员的部分职责;

\item {} 
UIzard(\sphinxurl{http://t.cn/RYRD89b} ):替代前端工程师的部分职责;

\item {} 
鲁班设计AI(\sphinxurl{http://t.cn/RYRD3y1} ):替代UI设计师的部分职责;

\item {} 
.Boomtrain的智能营销平台(\sphinxurl{http://t.cn/RYRDdYk}
):替代营销人员的部分职责;

\item {} 
京东仓库机器人(\sphinxurl{http://t.cn/RYRDsfH}
):完全替代仓库库管、分拣员、包装员等各种角色;

\item {} 
阿里巴巴天巡(\sphinxurl{http://t.cn/RYRkhsC} ):替代服务器运维人员30\%的工作;

\item {} 
Abyss Creations娃娃(\sphinxurl{http://t.cn/RCi65Q7} ):替代….(自己去看吧)

\end{itemize}

产品经理只有先除掉PC时代的上亿PV,移动互联网时代的数亿DAU,在产品经理眼中的障碍,才能看得清AI时代并解决PC和移动互联网时解决不了的痛点。


\paragraph{物联网、大数据、人工智能的融合 10\sphinxfootnotemark[235]}
\label{\detokenize{chapter_introduction/AI:id26}}%
\begin{footnotetext}[235]\sphinxAtStartFootnote
\sphinxnolinkurl{https://www.zhihu.com/people/muzimuhua/answers/by\_votes}
%
\end{footnotetext}\ignorespaces 
\begin{figure}[H]
\centering
\capstart

\noindent\sphinxincludegraphics{{AI_mix}.jpg}
\caption{融合\sphinxhref{http://www.changgpm.com/thread-350-1-1.html}{11}\sphinxfootnotemark[236]}\label{\detokenize{chapter_introduction/AI:id40}}\end{figure}
%
\begin{footnotetext}[236]\sphinxAtStartFootnote
\sphinxnolinkurl{http://www.changgpm.com/thread-350-1-1.html}
%
\end{footnotetext}\ignorespaces 
从整体闭环的角度考虑,从感知层、数据处理和传输层、决策层来看,

物联网是将终端、将感知器接入到网络中,使数据可用,他起到了感知数据的作用,在这个层面上,人工智能的感知能力也可以起到数据结构化的作用,比如语音机器人、图像识别等,能够获取到非结构化数据中的结构化信息。

大数据能够汇总所有的结构化、非结构化数据,做为数据湖泊,将各类数据做整合、做计算、做处理、做层次传输。

最终数据给到人工智能去做最终的数据计算、挖掘、预测、归类等等,给出决策再传递到物联网层面去做具体的执行。


\paragraph{场景}
\label{\detokenize{chapter_introduction/AI:id27}}
过去几年,AI的浪潮一波波袭来,而在过去一年,AI的风口慢慢小了,甚至之前疯狂追捧的资本也趋于冷静。从AI本身看,有两个原因:
\sphinxhref{http://www.woshipm.com/ai/3330480.html}{17}%
\begin{footnote}[237]\sphinxAtStartFootnote
\sphinxnolinkurl{http://www.woshipm.com/ai/3330480.html}
%
\end{footnote}
\begin{enumerate}
\sphinxsetlistlabels{\arabic}{enumi}{enumii}{}{.}%
\item {} 
目前AI的技术发展已经到了瓶颈期,除非有突破性的技术

\item {} 
AI落地难度大,各类场景还在探索中

\end{enumerate}

基于第二点,简单来说,如果把AI比作一把锤子工具,真正需要这个锤子的钉子不多,甚至很多看起来是钉子,其实都是螺丝,我只需要一把轻盈的螺丝刀就可以解决问题了。面的闭环:物联网\sphinxhyphen{}>大数据\sphinxhyphen{}>人工智能\sphinxhyphen{}>大数据\sphinxhyphen{}>物联网


\paragraph{AI任务}
\label{\detokenize{chapter_introduction/AI:id28}}
所有的AI任务都可以划分成为两类:\sphinxhref{http://www.uml.org.cn/ai/201912183.asp}{21}%
\begin{footnote}[238]\sphinxAtStartFootnote
\sphinxnolinkurl{http://www.uml.org.cn/ai/201912183.asp}
%
\end{footnote}

一种是针对某个业务领域内特定类型数据,提供对此类数据的基础AI学习、预测、分析能力的“横向”任务,例如计算机视觉、自然语言处理任务等;

另一种则是面向业务具体需求的、相对特殊化与个性化的“纵向”任务,例如金融领域的智能风控、电商领域的产品推荐以及比较常见的用户画像构建等。

就这两类AI任务来说,无论哪类任务都可以独立对外服务,也可以混合起来相互之间集成、组合,形成AI解决方案来支持更复杂的业务场景。我们构建智能化业务应用的核心就是将智能化需求分解、映射为具体的AI任务并一一实现,最后再进行合理地编排组合,实现任务目标。

但另一方面,在两类任务的实施过程中,其敏捷化需求存在着不同,对AI中台应该提供的服务需求也不同。相对而言,横向任务的敏捷化比较容易实现。

对于横向任务,除部分场景外,很多时候其本身并不直接解决业务需求,常作为基础模型对数据进行初步加工,再由一些纵向任务来对接需求。这也给算法实施团队充足的时间对横向任务模型进行充分的雕琢,对其敏捷性进行完善。


\paragraph{著名AI风投、学术机构和公司}
\label{\detokenize{chapter_introduction/AI:id29}}
\begin{figure}[H]
\centering
\capstart

\noindent\sphinxincludegraphics{{AI_related}.jpg}
\caption{著名AI风投、学术机构和公司\sphinxhref{https://www.zhihu.com/question/282715644s}{18}\sphinxfootnotemark[239]}\label{\detokenize{chapter_introduction/AI:id41}}\end{figure}
%
\begin{footnotetext}[239]\sphinxAtStartFootnote
\sphinxnolinkurl{https://www.zhihu.com/question/282715644s}
%
\end{footnotetext}\ignorespaces 

\paragraph{AI国家}
\label{\detokenize{chapter_introduction/AI:id30}}
\begin{figure}[H]
\centering
\capstart

\noindent\sphinxincludegraphics{{AI_country}.png}
\caption{AI国家对比\sphinxhref{https://ciraa.zju.edu.cn/report/report20200323.pdf}{19}\sphinxfootnotemark[240]}\label{\detokenize{chapter_introduction/AI:id42}}\end{figure}
%
\begin{footnotetext}[240]\sphinxAtStartFootnote
\sphinxnolinkurl{https://ciraa.zju.edu.cn/report/report20200323.pdf}
%
\end{footnotetext}\ignorespaces 

\paragraph{课程推荐}
\label{\detokenize{chapter_introduction/AI:id31}}
CS 188 | Introduction to Artificial
Intelligence:\sphinxurl{https://inst.eecs.berkeley.edu/~cs188/sp21/}

\sphinxurl{http://aima.cs.berkeley.edu/}


\paragraph{大众对AI的认知}
\label{\detokenize{chapter_introduction/AI:id32}}
\sphinxurl{https://www.yuque.com/weis/paper/cbgbkg}


\subsubsection{AI 产品}
\label{\detokenize{chapter_introduction/AI_Product:ai}}\label{\detokenize{chapter_introduction/AI_Product::doc}}

\paragraph{定义 6\sphinxfootnotemark[241]}
\label{\detokenize{chapter_introduction/AI_Product:id1}}%
\begin{footnotetext}[241]\sphinxAtStartFootnote
\sphinxnolinkurl{https://www.zhihu.com/question/346379206/answer/1756356249}
%
\end{footnotetext}\ignorespaces 
与AI三大基石(数据、算法、算力)全部有关的产品就是AI产品。它需要满足:
\begin{enumerate}
\sphinxsetlistlabels{\arabic}{enumi}{enumii}{}{.}%
\item {} 
经过数据处理(数据采集、清洗、标注、训练等),能反馈类人脑的结果;

\item {} 
有算法运行,能随数据训练的增多,算法越发精准而强大;

\item {} 
系统运行于算力底座上,计算形式可以多样(端、边、云),提供单元可以是CPU、GPU或NPU。

\end{enumerate}


\paragraph{AI产品与传统产品的区别:}
\label{\detokenize{chapter_introduction/AI_Product:id2}}\begin{enumerate}
\sphinxsetlistlabels{\arabic}{enumi}{enumii}{}{.}%
\item {} 
AI产品诞生的市场背景是甚至一个垂直的细分领域均有一个APP产品的市场环境,这个时候需要AI产品做到比原来的产品好上10倍的体验或者比原来的产品快10倍以上才能赢得市场的环境。

\item {} 
在做纯APP的时候是不需要考虑供应链的,但是由于广义范畴上的AI产品是\sphinxstylestrong{从数据获取到数据分析再到数据应用上,少不了硬件等外设}的采用,例如:用深度摄像头采集更多的数据,采用NB\sphinxhyphen{}IoT采集人和物体的行为数据,均需要硬件的融合。AI产品是更加考验产品经理综合素质的,除了设计管理好传统的软件上下游之外,还融入了供应链产业的深挖,例如:当你的摄像头与AI主体硬件产品出现BUG的时候,你需要联系的事摄像头生产厂商,而不像APP时代仅仅需要再成熟的手机上研发即可,这个时候需要产品调动的是摄像头整个研发甚至一个工厂来配合你。

\item {} 
需求的变化有:

\end{enumerate}
\begin{itemize}
\item {} 
例如:新零售,用户需要货来匹配人,这里需要LBS和更多智能传感器的数据来服务人。

\item {} 
例如:线下商铺原来是不知道哪个用户来逛街,哪个潜在消费者在哪个商品前停留的更久,节假日购买热销商铺结账需要排队等等需求正好使得AI产品得以展身手的时刻。

\end{itemize}


\paragraph{形态}
\label{\detokenize{chapter_introduction/AI_Product:id3}}
AI产品需要打破传统GUI的局限,AI产品对外提供的产品形态不仅可以包括前后端GUI的完整系统,还可以是\sphinxstylestrong{API接口和SDK的形式};


\subparagraph{API、SDK类}
\label{\detokenize{chapter_introduction/AI_Product:apisdk}}
在之前的文章《如何做一款SDK产品》中对SDK产品做了些简单描述,本文从API的形式去描述,产品需求文档中需要包括接口的输入、输出、算法准确率、误检率、漏检率、接口耗时性能和算法约束规则。

以图片识别为例,产品需要定义好的核心字段包括:
\begin{itemize}
\item {} 
输入:图片格式—jpg、jpeg、png等;图片传输格式—base64或url;ROI区域—数量,默认可以是整张图片,最多支持多少个;ROI的画法—矩形(左顶点+长宽),多边形(所有坐标点);识别类型—如果接口支持识别多个内容,这个字段可以加上指定需要算法识别的内容;其他的鉴权、时间戳之类的字段信息可以让开发定义;

\item {} 
输出:核心信息同算法需求,但是需要落实到接口字段上,如总目标数量、每一个目标的坐标信息、置信度,其他分析目标的特殊字段信息;

\item {} 
算法准确率、误检率、漏检率,这点笔者建议最好是以业务指标分析,以目标检测为例,通常算法是以mAP来衡量的,它是从目标维度来评估,但是用户通常是从图片的维度来衡量,一个图片中有误检或漏检的,用户可能就会认为这张图片识别出错;因此,需要产品定义明确好;

\item {} 
耗时性能,这里的耗时性能是指接口的\sphinxstylestrong{整体耗时},即用户传入图片到返回结果的耗时,需要算法和开发一起评估,产品只需要定义产品需求。

\end{itemize}


\subparagraph{GUI类}
\label{\detokenize{chapter_introduction/AI_Product:gui}}
如果是GUI类的形态,产品重点关注的应该是\sphinxstylestrong{产品原型}如何设计,用户体验,用户使用流程等,这个与传统的产品设计并没有什么区别,只需要将算法需求单独拆分给算法小伙伴。同时约定好算法的规则即可,所谓的算法规则即接受在当前的算法能力下对用户使用上的约束规则,比如要求用户上传的文件格式、命名有什么要求,算法返回给用户的识别结果有什么限制。

作为产品其实很无奈,原则上应该以用户为中心挖掘用户最自然的用户习惯,但是在AI技术不成熟的情况下,需要牺牲些用户体验。


\subparagraph{对话式UI}
\label{\detokenize{chapter_introduction/AI_Product:ui}}
GUI的复杂性体现在功能越多,菜单层级越多,交互的控件也越来越复杂,对于用户的学习成本是非常高的。导航与层级是GUI结构所决定的,这一点与对话式UI完全不同,对话式UI通过机器学习之后,可以无限地消灭层级关系,这样可以减少用户操作路径,直达目标。\sphinxhref{http://www.woshipm.com/it/581011.html}{18}%
\begin{footnote}[242]\sphinxAtStartFootnote
\sphinxnolinkurl{http://www.woshipm.com/it/581011.html}
%
\end{footnote}


\paragraph{AI 技术实际应用情况}
\label{\detokenize{chapter_introduction/AI_Product:id4}}
\begin{figure}[H]
\centering
\capstart

\noindent\sphinxincludegraphics{{AI_use}.jpg}
\caption{2018年AI
技术实际应用情况\sphinxhref{https://zhuanlan.zhihu.com/p/37333774}{9}\sphinxfootnotemark[243]}\label{\detokenize{chapter_introduction/AI_Product:id17}}\end{figure}
%
\begin{footnotetext}[243]\sphinxAtStartFootnote
\sphinxnolinkurl{https://zhuanlan.zhihu.com/p/37333774}
%
\end{footnotetext}\ignorespaces 

\paragraph{常见产品}
\label{\detokenize{chapter_introduction/AI_Product:id5}}
\begin{figure}[H]
\centering
\capstart

\noindent\sphinxincludegraphics{{AI_product_often}.png}
\caption{常见产品\sphinxhref{https://coffee.pmcaff.com/article/2258532879227008/pmcaff?utm\_source=forum}{17}\sphinxfootnotemark[244]}\label{\detokenize{chapter_introduction/AI_Product:id18}}\end{figure}
%
\begin{footnotetext}[244]\sphinxAtStartFootnote
\sphinxnolinkurl{https://coffee.pmcaff.com/article/2258532879227008/pmcaff?utm\_source=forum}
%
\end{footnotetext}\ignorespaces 

\subparagraph{应用领域}
\label{\detokenize{chapter_introduction/AI_Product:id6}}
\sphinxurl{http://www.woshipm.com/ai/4020213.html}


\paragraph{AI产品的趋势}
\label{\detokenize{chapter_introduction/AI_Product:id7}}
\begin{figure}[H]
\centering
\capstart

\noindent\sphinxincludegraphics{{AI_product_trend}.jpg}
\caption{AI产品的趋势\sphinxhref{http://www.xmamiga.com/3573/}{7}\sphinxfootnotemark[245]}\label{\detokenize{chapter_introduction/AI_Product:id19}}\end{figure}
%
\begin{footnotetext}[245]\sphinxAtStartFootnote
\sphinxnolinkurl{http://www.xmamiga.com/3573/}
%
\end{footnotetext}\ignorespaces \begin{itemize}
\item {} 
产品逻辑化简为繁,用户学习成本降低(PM :
尝试用“颠覆式思维”设计产品);

\item {} 
从用户角度考虑投入产出比(PM :
选择用户最“痛”的点或者直接和利益挂钩的点作为需求切入点);

\item {} 
算法可解释性差,产品需要逐渐获得用户的信任(PM:技术经过验证之后上市,树立了与专业形象,赢得用户信任);

\item {} 
技术的飞速发展,带来了多元化的交互行为(PM:学会合理利用多种床设备,创造更多交互方式来满足用户需求);

\item {} 
产品的需求不一定来源于确定的因果关系(PM:输出的未必是确定的页面内容,可能是一对规则和策略)

\item {} 
PM在开始需求定义前应充分了解目前技术水平和资源的局限性,避免定义一些研发很难实现的需求

\end{itemize}


\paragraph{AI产品价值 2\sphinxfootnotemark[246]}
\label{\detokenize{chapter_introduction/AI_Product:ai-2}}%
\begin{footnotetext}[246]\sphinxAtStartFootnote
\sphinxnolinkurl{http://www.woshipm.com/pmd/3657472.html}
%
\end{footnotetext}\ignorespaces 
首先为了验证产品是否对业务产生了价值,用一个粗略的公式表示AI产品的业务价值,其次是为了分析产品的哪些品功能存在优化空间,最后还可以驱动业务决策

AI产品价值=(提高的时效*时效成本+GMV提升)\sphinxhyphen{}(AI硬件资源成本+研发成本)


\paragraph{关注产品}
\label{\detokenize{chapter_introduction/AI_Product:id8}}
简单地说你使用了机器学习并不足以让你从其他科技产品中脱颖而出;他们也在使用它。让你从竞争对手中脱颖而出的是你如何使用这些数据。
\sphinxhref{https://www.appcues.com/blog/product-managers-and-artificial-intelligence}{10}%
\begin{footnote}[247]\sphinxAtStartFootnote
\sphinxnolinkurl{https://www.appcues.com/blog/product-managers-and-artificial-intelligence}
%
\end{footnote}


\subparagraph{辅助视障人士的天使眼智能眼镜}
\label{\detokenize{chapter_introduction/AI_Product:id9}}
世界上首款辅助视障人士感知世界和出行的智能眼镜。最为重要的功能就是,通过人工智能的图像识别技术识别出使用者环境中的信息,并转化为听觉信号。
\sphinxhref{http://www.woshipm.com/ai/967258.html}{4}%
\begin{footnote}[248]\sphinxAtStartFootnote
\sphinxnolinkurl{http://www.woshipm.com/ai/967258.html}
%
\end{footnote}


\paragraph{管理AI产品}
\label{\detokenize{chapter_introduction/AI_Product:id10}}
\begin{figure}[H]
\centering
\capstart

\noindent\sphinxincludegraphics{{REL}.png}
\caption{REL}\label{\detokenize{chapter_introduction/AI_Product:id20}}\end{figure}


\paragraph{AI 蓝图}
\label{\detokenize{chapter_introduction/AI_Product:id11}}
\begin{figure}[H]
\centering
\capstart

\noindent\sphinxincludegraphics{{AI_blueprint}.jpg}
\caption{AI
蓝图\sphinxhref{https://www.slideshare.net/Happy.Prototyper/mix2018ai-ai-vp}{13}\sphinxfootnotemark[249]}\label{\detokenize{chapter_introduction/AI_Product:id21}}\end{figure}
%
\begin{footnotetext}[249]\sphinxAtStartFootnote
\sphinxnolinkurl{https://www.slideshare.net/Happy.Prototyper/mix2018ai-ai-vp}
%
\end{footnotetext}\ignorespaces 

\paragraph{AI 专利}
\label{\detokenize{chapter_introduction/AI_Product:id12}}
\sphinxurl{https://www.darts-ip.com/de/2019\%E4\%BA\%BA\%E5\%B7\%A5\%E6\%99\%BA\%E8\%83\%BD\%E7\%9A\%84\%E4\%B8\%93\%E5\%88\%A9\%E7\%94\%B3\%E8\%AF\%B7\%E8\%B6\%8B\%E5\%8A\%BF/}


\paragraph{AI能力与场景的匹配}
\label{\detokenize{chapter_introduction/AI_Product:id13}}
\begin{figure}[H]
\centering
\capstart

\noindent\sphinxincludegraphics{{AI_scene}.png}
\caption{AI能力与场景的匹配}\label{\detokenize{chapter_introduction/AI_Product:id22}}\end{figure}

\begin{figure}[H]
\centering
\capstart

\noindent\sphinxincludegraphics{{AI_solve}.png}
\caption{解决到什么程度}\label{\detokenize{chapter_introduction/AI_Product:id23}}\end{figure}

场景需求被解决到什么程度才正好?可以从三个方面进行展开:适度匹配,SOTA极限,团队资源。
\begin{itemize}
\item {} 
\sphinxstylestrong{适度匹配}的意义更多在于性能的合理利用,90\%的人脸识别准确率显然不能用于金融支付场景,但是90\%的人体检测准确率在人流量监测场景下却能够适用。硬要将人流量监测场景下的人体检测准确率提升到99\%,暂且不谈能不能实现,单是研发投入就会极大提升,实际效用差却没有多大。金融支付场景下的人脸识别准确率对于实际效用影响巨大,是硬着头皮也要进行研发投入的点。将性能合理利用,适度匹配,根据实际调整解决程度。

\item {} 
\sphinxstylestrong{SOTA极限}其实决定了解决程度的上限,SOTA是State\sphinxhyphen{}of\sphinxhyphen{}the\sphinxhyphen{}Art的缩写,有趣的是,它的意思指向“在一些benchmark的数据集上跑分非常高的模型”,代表了最优秀的一批算法。
很容易理解,如果目前行业的极限算法的能力是这样,那对于场景需求的解决程度也只能以此为上限。如果说SOTA极限来自于外部,那团队资源则是内部的上限。学界的有些算法方案在发布的时候,会切实考虑到工业应用,会将源码及部署方案一并发布到github等托管平台;有些算法方案则只有paper放出,没有相关实现资源。

\item {} 
如果最合适的算法恰好没有放出实现资源,而团队人力和时间又不足,则只能在放出过实现资源的算法里面找相对合适的,相应的解决程度也会受限。如果\sphinxstylestrong{团队资源充足},就有机会通过paper进行算法复现,将需求解决得更好。

\end{itemize}


\paragraph{AI产品研发生命周期}
\label{\detokenize{chapter_introduction/AI_Product:id14}}
\begin{figure}[H]
\centering
\capstart

\noindent\sphinxincludegraphics{{AI_product_life}.jpg}
\caption{AI产品研发生命周期\sphinxhref{http://www.uml.org.cn/ai/201912183.asp}{14}\sphinxfootnotemark[250]}\label{\detokenize{chapter_introduction/AI_Product:id24}}\end{figure}
%
\begin{footnotetext}[250]\sphinxAtStartFootnote
\sphinxnolinkurl{http://www.uml.org.cn/ai/201912183.asp}
%
\end{footnotetext}\ignorespaces 

\paragraph{造成人工智能产品设计失败的常见原因}
\label{\detokenize{chapter_introduction/AI_Product:id15}}\begin{itemize}
\item {} 
技术可行性:基于数据统计学习构建的模型,在实际使用中有很多限制。

\item {} 
组织变革:在技术和产品层面外,对于流程,组织都有一定要求,执行推进难度较大。\sphinxhref{https://zhuanlan.zhihu.com/p/218468169}{16}%
\begin{footnote}[251]\sphinxAtStartFootnote
\sphinxnolinkurl{https://zhuanlan.zhihu.com/p/218468169}
%
\end{footnote}

\item {} 
技术驱动产品设计。产品应该从需求出发而非从技术出发。

\item {} 
忽略用户期望管理,华而不实的产品使用户失望。

\item {} 
单点突破带来的价值有限,产品价格与用户付出代价不成正比。

\item {} 
一味追求技术,忽略用户体验。 \sphinxhref{http://www.xmamiga.com/3573/}{12}%
\begin{footnote}[252]\sphinxAtStartFootnote
\sphinxnolinkurl{http://www.xmamiga.com/3573/}
%
\end{footnote}

\end{itemize}


\paragraph{核心价值}
\label{\detokenize{chapter_introduction/AI_Product:id16}}
企业的核心价值,特别是工业ML产品,如预测性维护软件,往往来自其预测的功能性能(如准确性)。\sphinxhref{https://radiant-brushlands-42789.herokuapp.com/towardsdatascience.com/how-to-manage-machine-learning-products-part-1-386e7011258a}{15}%
\begin{footnote}[253]\sphinxAtStartFootnote
\sphinxnolinkurl{https://radiant-brushlands-42789.herokuapp.com/towardsdatascience.com/how-to-manage-machine-learning-products-part-1-386e7011258a}
%
\end{footnote}


\subsubsection{产品机会}
\label{\detokenize{chapter_introduction/opportunity:id1}}\label{\detokenize{chapter_introduction/opportunity::doc}}
一切以用户价值为依归。——产品经理价值观

机会是指关于开发新产品的任何想法,它可以是一个产品最初的描述、一种新的需求、一种新发现的技术,或者是一个初步需求与可能解决方案的联系;

\begin{figure}[H]
\centering
\capstart

\noindent\sphinxincludegraphics{{where_opportunity}.png}
\caption{机会是怎么来的}\label{\detokenize{chapter_introduction/opportunity:id11}}\end{figure}
\begin{enumerate}
\sphinxsetlistlabels{\arabic}{enumi}{enumii}{}{.}%
\item {} 
抽象出用户需求:情感\&逻辑

\item {} 
需求实例化:搜索框/搜索按钮实现快速匹配

\item {} 
强化产品需求:搜索框的大小、颜色、风格、位置\sphinxhref{https://coffee.pmcaff.com/article/2447262389384320/pmcaff?utm\_source=forum}{6}%
\begin{footnote}[254]\sphinxAtStartFootnote
\sphinxnolinkurl{https://coffee.pmcaff.com/article/2447262389384320/pmcaff?utm\_source=forum}
%
\end{footnote}

\end{enumerate}


\paragraph{层次}
\label{\detokenize{chapter_introduction/opportunity:id2}}
\begin{figure}[H]
\centering
\capstart

\noindent\sphinxincludegraphics{{solve}.png}
\caption{各层次的机会\sphinxhref{https://blog.csdn.net/weixin\_45036344/article/details/102699064}{7}\sphinxfootnotemark[255]}\label{\detokenize{chapter_introduction/opportunity:id12}}\end{figure}
%
\begin{footnotetext}[255]\sphinxAtStartFootnote
\sphinxnolinkurl{https://blog.csdn.net/weixin\_45036344/article/details/102699064}
%
\end{footnotetext}\ignorespaces \begin{itemize}
\item {} 
层次1:市场发展快、扩大快、变化快且已有产品降价快,风险相对最低;

\item {} 
层次2:机会推向市场、技术或两方面均鲜为人知的细分领域,风险适中;

\item {} 
层次3:从0到1,颠覆性的创造,拥有最大的不确定性。

\end{itemize}


\paragraph{创造价值 5\sphinxfootnotemark[256]}
\label{\detokenize{chapter_introduction/opportunity:id3}}%
\begin{footnotetext}[256]\sphinxAtStartFootnote
\sphinxnolinkurl{https://weread.qq.com/web/reader/46532b707210fc4f465d044kaab325601eaab3238922e53}
%
\end{footnotetext}\ignorespaces 
聚焦创造价值,而非痛点。

在没有触摸屏的智能手机时代,没有人认为按键手机在使用方面有痛点。再比如,对于网上购物,没有人认为在线下买东西很烦躁;对于移动支付,十年前没有人觉得使用现金很痛苦。所以,不要动不动就把痛点挂在嘴边,而要把给用户创造价值牢记在心。也许我们在帮用户节省时间,也许在帮用户节约金钱,也许在帮用户简单出行,也许在帮用户创造在家里就可以安心购物的“线上商城”。


\paragraph{痛点}
\label{\detokenize{chapter_introduction/opportunity:id4}}
痛点是恐惧


\subparagraph{分类 1\sphinxfootnotemark[257]}
\label{\detokenize{chapter_introduction/opportunity:id5}}%
\begin{footnotetext}[257]\sphinxAtStartFootnote
\sphinxnolinkurl{https://www.zhihu.com/question/21155472/answer/1580037628}
%
\end{footnotetext}\ignorespaces 
所谓“痛点”可分为三类:

第一类是人类普遍有所体会的某种心理上的难受,或者某些蠢蠢欲动的欲望没有得到满足的难受,这种难受常常经过外界刺激而有所强化。例如,思乡、恐高、怕抽血、窥探隐私、八卦欲等等。
第二类是体验过某种产品后,如果不买会难受,会有不满足感,可谓欲罢不能。比如喝惯可乐、玩过网络游戏为什么难以摆脱。
第三类是在购买过程中小小地难受一下,如此使得顾客最终获得产品时,强烈地对比出愉悦感来。例如,苹果新品发布时总是长龙排队,甚至有顾客不得不凌晨开始在店铺外占位。

与第一类用于化解痛点的产品不同,第三类痛点是有意设置的,它有时是因为企业稀缺的资源所造就的,企业为了将有限的资源聚焦在最具竞争力的产品或服务上,因而剔除了某些附加服务。如果顾客体验到的痛感远远小于获得产品和服务时的满足感,那么顾客就不再计较这其中的痛点。成功击中这类痛点的产品,要么消解了痛点,要么弱化了痛点。


\subparagraph{如何找到痛点 2\sphinxfootnotemark[258]}
\label{\detokenize{chapter_introduction/opportunity:id6}}%
\begin{footnotetext}[258]\sphinxAtStartFootnote
\sphinxnolinkurl{https://wiki.mbalib.com/wiki/\%E7\%97\%9B\%E7\%82\%B9\%E8\%90\%A5\%E9\%94\%80}
%
\end{footnotetext}\ignorespaces 
  在客户营销学中,消费者的痛点是指消费者在体验产品或服务过程中原本的期望没有得到满足而造成的心理落差或不满,这种不满最终在消费者心智模式中形成负面情绪爆发,让消费者感觉到痛。当设计客户体验的时候,痛点是一个必须存在的“天使”,不仅可以用来对比体验中的愉悦,还可以节省资源,摆脱束缚。原因有如下三点:
\begin{enumerate}
\sphinxsetlistlabels{\arabic}{enumi}{enumii}{}{.}%
\item {} 
痛点是基于心理感受对比的体验营销的重要手段。在体验营销中,企业需要构建让消费者足够满意和愉悦的痒点和兴奋点。对于消费者而言,在鱼和熊掌不可得兼的前提下,只能抱以遗憾,选择对其更为重要的痒点或兴奋点因素,忽视这些痛点带来的不满。这是心理学中的“残缺美”现象,因为有所期望,所以才会不满,倘若没有了期望,不满从何而来?
另外,痛点也是一个相对的概念,是基于同行业的竞争而做出的对比后形成的判断,这是和其对立的词痒点比较而来的。在销售过程中,甲企业的产品除具有相同的功能特点外,还具有一些其他的附加服务,而这些恰恰是乙企业所不具备的,这对倾向于乙品牌的消费者来说,就形成了大脑认知中的痛点现象。

\item {} 
痛点是企业聚焦战略选择的必然结果。如果说兴奋点是战略选择的话,那么痛点是基于战略上的放弃。这个观点是从企业战略角度分析的,是战略的取舍问题,是典型的聚焦特点。
在销售过程中,为了实现客户价值的转移,很多企业通过丰富的体验营销手段来吸引消费者。但是随之带来的问题就是消费者的欲望是无穷的,而企业的资源又是稀缺的,全方位无条件满足消费者的欲望是不现实的。怎么样才能实现客户让渡价值的最大化,企业必然的选择就是聚焦战略,即把所提供的产品或服务形成最闪亮的亮点,其余的附加服务都取消掉。必然的结果是消费者会感受到不满意,但倘若最为满意所带来的价值远大于这种痛点所带来的不满意,那么客户最终会淡化或忽视这种痛点。

\item {} 
痛点是一种引领企业创新的过程。在竞争中,能够在消费者心智中占有一席之地,乃是所有营销人孜孜不倦的追求。而痛点就是营销策划活动中最常用和最有效的手段之一。在动态的竞争过程中,每一个企业为了参与竞争都试图通过差异化来占领自己的区域,所以“反定位”就成了有效手段,于是竞争对手的体验营销中的痛点就成了其他企业营销创新的痒点。
对于痛点寻找需要第一,对自己的产品和服务有充分的了解,还有就是对竞争对手的产品或服务有充分的了解。第二,是对消费者消费心理有充分的解读。对于自己产品或服务和竞争对手的产品或服务的了解是用来做差异化定位的,通过细分市场区找痛点。对消费者的了解是非常重要的,因为购物的主题就是他们这些人,你只有知道他们的真正需求,然后满足他们那么你的产品或服务就是成功的,否则失败。痛点是一个长期观察挖掘的过程不可能一触而就的,这些都是细节的问题,都是消费者最关注的细节,做好这些,结果就可想而知了。

\end{enumerate}


\paragraph{爽点 3\sphinxfootnotemark[259]}
\label{\detokenize{chapter_introduction/opportunity:id7}}%
\begin{footnotetext}[259]\sphinxAtStartFootnote
\sphinxnolinkurl{https://www.jianshu.com/p/fa5e2c1f3930}
%
\end{footnotetext}\ignorespaces 
有需求,还能被即时满足,这就是爽。

外卖产品就是满足了痒点。


\paragraph{痒点}
\label{\detokenize{chapter_introduction/opportunity:id8}}
痒点满足的是人的虚拟自我。什么是虚拟自我?就是想象中那个理想的自己。

网红产品靠的就是痒点。


\paragraph{需要掌握的技能/方法:}
\label{\detokenize{chapter_introduction/opportunity:id9}}
心情看板:比如说你去北京三甲医院挂号,因为三甲医院资源紧张,你在挂号排队的时候,心情会或许会非常的低落。心情看板主要为了识别出用户在完成整个流程中心情的最低点和最高点,便于根据用户情绪起伏中发现设计机会点,给问题提供针对性的假设。

问题假设:给用户亟待解决的问题提出假设,比如说怎么才能在最短时间内挂到专家门诊号。

设计挑战:主要是为了拓展解决问题的不同思路,可以在团队内部通过头脑风暴的形式做设计挑战。


\paragraph{战略示意图}
\label{\detokenize{chapter_introduction/opportunity:id10}}
Jeroen
Kraaijenbrink的战略示意图适合制定企业的总体战略,有利于在新的产品出现时,根据新的发展战略确认其是否可行。

\begin{figure}[H]
\centering
\capstart

\noindent\sphinxincludegraphics{{stratgy_pic}.png}
\caption{战略示意图}\label{\detokenize{chapter_introduction/opportunity:id13}}\end{figure}
\begin{enumerate}
\sphinxsetlistlabels{\arabic}{enumi}{enumii}{}{.}%
\item {} 
核心价值:核心价值是指企业通过某种方式提供的产品和服务,企业需思考产品对于消费者的价值是什么,产品能够解决的问题是什么。

\item {} 
核心资源:企业的产品拥有什么资源,使得产品在市场竞争环境中具有优势,表现为产品的的核心资源和核心活动,人工智能类产品一般是核心的算法、数据等。

\item {} 
盈利模式:根据企业的核心资源,产品能够如何收取费用?从谁身上收取、怎样收取、何时收取?

\item {} 
用户需求:产品服务于什么组织与群体,产品应该满足什么需要呢?

\item {} 
价值和目标:企业要明确产品要达到什么目标,产品解决的最重要的问题是什么。需要注意的是,不要将收益或者股东利益当作产品的主要目标。

\item {} 
合作伙伴:重要的合作伙伴,能够保证企业的产品和服务更有价值。

\item {} 
风险和成本:商业模式运作过程中承担什么样的财务、社会或其他风险及企业如何管理这些风险。

\item {} 
竞争对手:客户会用比较的方式来决定是否购买你的产品和服务,你有什么样的竞争优势。

\item {} 
团队氛围:团队文化和结构是什么样的,工作的组织结构和文化环境是制定决策并完美执行的重要因素。

\item {} 
机会与威胁:市场中有什么因素影响着组织,这些因素是机会还是威胁?对未来的商业模式有着怎样的影响?

\end{enumerate}


\paragraph{AI的机会}
\label{\detokenize{chapter_introduction/opportunity:ai}}
\begin{figure}[H]
\centering
\capstart

\noindent\sphinxincludegraphics{{AI_opportunity}.png}
\caption{AI opportunity}\label{\detokenize{chapter_introduction/opportunity:id14}}\end{figure}


\subsubsection{AI产品经理}
\label{\detokenize{chapter_introduction/AI_PM:ai}}\label{\detokenize{chapter_introduction/AI_PM::doc}}
AI产品经理的本质,是产品经理。

\begin{figure}[H]
\centering
\capstart

\noindent\sphinxincludegraphics[width=400\sphinxpxdimen]{{AI_PM}.png}
\caption{AI\_PM}\label{\detokenize{chapter_introduction/AI_PM:id46}}\end{figure}

就像互联网产品经理刚出现的时候也没有真正的一套知识和技能体系供大家去参考,目前的互联网产品经理的知识和技能体系也是随着各大公司招聘要求相互碰撞和产品研发过程中不断摸索所得到的一个共同体系,现在的AI产品经理,由于没有大面积成熟商业产品落地,现在所面临的知识空窗期只能说更为严重。


\paragraph{产品经理历史与技术迭代}
\label{\detokenize{chapter_introduction/AI_PM:id1}}

\subparagraph{FMCG PM}
\label{\detokenize{chapter_introduction/AI_PM:fmcg-pm}}
诞生于快速消费品时代,以宝洁公司品牌经理岗位的设置及制度的成型为标志。产品经理其实是品牌经理,他的主要职责是负责这个品牌的市场定位、营销推广、渠道建设,协调推进各部门共同为新产品提供全方位的服务。产品本身是高度同质化的,一个产品究竟是一亿量级还是十亿量级,取决于产品经理的\sphinxstylestrong{品牌营销能力}。

宝洁公司采用多品牌战略:洗发水领域常见的飘柔、潘婷、海飞丝都是宝洁旗下产品,消费者不管选哪个产品,对于宝洁公司而言都是赚的。除了日常用品,顶级的SK\sphinxhyphen{}Ⅱ也是宝洁旗下的\sphinxhref{https://weread.qq.com/web/reader/77532110721ea34a7751c9ak8f132430178f14e45fce0f7}{36}%
\begin{footnote}[260]\sphinxAtStartFootnote
\sphinxnolinkurl{https://weread.qq.com/web/reader/77532110721ea34a7751c9ak8f132430178f14e45fce0f7}
%
\end{footnote}

对于企业来说,品牌即定位,一个好的产品往往会将其品牌与消费者的认知固定在一起。很多时候,一家企业要做不同的用户分层,既有物美价廉的大众产品,也提供高端价位的特级产品。如果都用一个共同品牌,难免造成消费者的认知混乱。

例如,伊利和蒙牛是大众经常选择的平价牛奶,如果高端定位牛奶依旧沿用自有品牌,那么从价格方面就很难让消费者接受。因此,两家在高端定位的牛奶品牌方面分别推出了金典和特仑苏。

除了快消品行业,如今很多大公司选择孵化产品或者创立子公司,一定程度上与多品牌战略有异曲同工之处。


\subparagraph{IPM}
\label{\detokenize{chapter_introduction/AI_PM:ipm}}
IPM (Internet Product
Manager),互联网产品经理是互联网公司中的一种职能,负责互联网产品的计划和推广,以及互联网产品生命周期的演化。根据所负责的互联网产品是用户产品还是商业产品,可以分为互联网用户产品经理和互联网商业产品经理。用户产品经理最关心的是互联网用户产品的用户体验,商业产品经理最关心的是互联网商业产品的\sphinxstylestrong{流量变现能力}。

互联网产品经理还可以有如下2个分类维度:
\begin{itemize}
\item {} 
“产品形态”维度:PC客户端产品经理、网站产品经理(Web/H5)、移动端产品经理(iOS/Android)、Server产品经理……

\item {} 
“行业领域”维度:工具产品经理、社交产品经理、电商产品经理、O2O产品经理……

\end{itemize}


\subparagraph{“互联网产品经理演进史”}
\label{\detokenize{chapter_introduction/AI_PM:id2}}
消费品时代,产品经理(Product
Manager)的本质是“营销产品经理”。因为需求相对明确、产品同质化、生产标准化。
软件时代,PM本质是“项目产品经理”。因为需求相对容易明确,用户对产品体验要求不高(选择少,必须用)——PM“在管理生产上更能创造价值,沟通协调,版本控制,按时交付。”
互联网时代,PM本质是“需求产品经理”。因为需求+体验,能产生更大价值。


\subparagraph{互联网时代特性 17\sphinxfootnotemark[261]}
\label{\detokenize{chapter_introduction/AI_PM:id3}}%
\begin{footnotetext}[261]\sphinxAtStartFootnote
\sphinxnolinkurl{https://github.com/JoJoDU/Book\_Notes/issues/3}
%
\end{footnotetext}\ignorespaces 
一、
\begin{itemize}
\item {} 
信息复制分发的边际成本降低 \sphinxstyleemphasis{分析} 关键变化:连接让信息快速交互传递 –>
全新的产品类别:在线信息产品 –> 互联网PM 信息分发历史:信件 –>
报纸、杂志 –> 软件(光盘、软盘)

\item {} 
信息过载
\sphinxhref{https://weread.qq.com/web/reader/0c032c9071dbddbc0c06459k70e32fb021170efdf2eca12}{31}%
\begin{footnote}[262]\sphinxAtStartFootnote
\sphinxnolinkurl{https://weread.qq.com/web/reader/0c032c9071dbddbc0c06459k70e32fb021170efdf2eca12}
%
\end{footnote}
\sphinxstyleemphasis{分析}由于互联网的信息的快速传输、数据的海量共享,使得大量冗余的信息充斥着人们的眼球。

\item {} 
用户量巨大 多生产、多供给一份信息的边际成本降低 –>
信息传播极快;免费信息 –> 转化用户群极大扩展; \sphinxstylestrong{导致}
在线信息产品极易爆发增长
互联网使信息传播和沟通更加高效,用户更易接触、度量、比较不同的产品;
\sphinxstylestrong{导致}市场竞争激烈,出 现马太效应

\end{itemize}

二、
\begin{itemize}
\item {} 
快速迭代:市场反馈快,产品生产交付快,分发快;降低试错成本

\item {} 
数据(反馈和信息)、AB测试(实现更大规模和更快速的迭代)、千人千面(基于用户反馈数据快速验证和迭代)
\sphinxstylestrong{PM上线能力:理解用户、理解交易}

\end{itemize}

三、
\begin{itemize}
\item {} 
体验设计价值增大

\item {} 
需求 + 体验:以用户为核心,\sphinxstylestrong{横向组织资源,按需求生产和销售}

\end{itemize}

互联网思维:“专注、口碑、极致、快”。互联网资讯推荐36氪、虎嗅、钛媒体、鞭牛士公众号


\subparagraph{移动互联网时代}
\label{\detokenize{chapter_introduction/AI_PM:id4}}
尤其是2012年移动互联网爆发,市场上迫切需要大量的产品经理来做App,而以之前的行业储备,专业人员远远不够。很多应届生甚至其他岗位的人纷纷转行做移动端产品经理。这个时期的产品多为面向个人用户的(to
C)产品,没有软件时代那么多专业限制,因此降低了产品经理的准入门槛,也催生了很多产品经理培训班。


\subparagraph{大数据、人工智能、5G时代}
\label{\detokenize{chapter_introduction/AI_PM:g}}
从2018年开始,产品经理的发展进入一个新时代。大数据、人工智能等技术日趋成熟,人口和流量红利相对骤减,消费互联网度过了快速发展期,以阿里巴巴、腾讯等为代表的巨头公司纷纷转战产业互联网。与此同时,产品经理这个岗位变得更加细分,能力要求也更加综合。

2018年被很多业内人士戏称为“寒冬”的开始,而产品经理也迫切需要从红利躺赢到实力打拼的转变。岗位更加\sphinxstylestrong{细分},其中数据产品经理、策略产品经理、商业化产品经理等需求旺盛,各个公司都需要专业人员提升用户转化率和盈利收入。同时,对于产品经理的能力要求也更加综合。


\subparagraph{互联网 VS AI}
\label{\detokenize{chapter_introduction/AI_PM:vs-ai}}
\begin{figure}[H]
\centering
\capstart

\noindent\sphinxincludegraphics{{Internet_VS_AI}.png}
\caption{互联网时代 VS AI时代\sphinxhref{https://zhuanlan.zhihu.com/p/43888627}{27}\sphinxfootnotemark[263]}\label{\detokenize{chapter_introduction/AI_PM:id47}}\end{figure}
%
\begin{footnotetext}[263]\sphinxAtStartFootnote
\sphinxnolinkurl{https://zhuanlan.zhihu.com/p/43888627}
%
\end{footnotetext}\ignorespaces 

\subparagraph{AI产品经理与之区别 8\sphinxfootnotemark[264]}
\label{\detokenize{chapter_introduction/AI_PM:ai-8}}%
\begin{footnotetext}[264]\sphinxAtStartFootnote
\sphinxnolinkurl{https://easyai.tech/blog/ai-pm-knowledge/}
%
\end{footnotetext}\ignorespaces 
\begin{figure}[H]
\centering
\capstart

\noindent\sphinxincludegraphics{{Product_center}.png}
\caption{产品}\label{\detokenize{chapter_introduction/AI_PM:id48}}\end{figure}


\subparagraph{数据集优先}
\label{\detokenize{chapter_introduction/AI_PM:id5}}
AI
PM首先创建/收集一个代表问题空间的数据集。只有这样,他们才会要求工程师对问题进行迭代,以提供需要解决的问题90\%的准确性。而不是画UX。

\sphinxurl{https://appen.com/}
\sphinxhref{https://medium.com/@fabian.kutschera/udacitys-ai-product-manager-a-review-2faba9ba3669}{26}%
\begin{footnote}[265]\sphinxAtStartFootnote
\sphinxnolinkurl{https://medium.com/@fabian.kutschera/udacitys-ai-product-manager-a-review-2faba9ba3669}
%
\end{footnote}

法律法规和国家政策:
\sphinxhref{https://www.bilibili.com/video/av800293586/}{28}%
\begin{footnote}[266]\sphinxAtStartFootnote
\sphinxnolinkurl{https://www.bilibili.com/video/av800293586/}
%
\end{footnote}
\begin{itemize}
\item {} 
数据与隐私保护(egGeneral Data ProtectionRegulation)

\item {} 
各种协议(开源软件、公开数据、公共数据集)

\item {} 
大数据、人工智能战略(eg教育领域)

\end{itemize}


\subparagraph{广泛涉猎 18\sphinxfootnotemark[267]}
\label{\detokenize{chapter_introduction/AI_PM:id6}}%
\begin{footnotetext}[267]\sphinxAtStartFootnote
\sphinxnolinkurl{http://www.woshipm.com/pd/2209024.html}
%
\end{footnotetext}\ignorespaces 
与互联网产品不同,AI产品经理需要广泛涉猎不同行业不同地域的操作习惯,借鉴硬件、软件行业的优秀的交互,不断总结,思考更为轻松、自然、平顺的产品体验。


\subparagraph{真正的“需求产品经理” 16\sphinxfootnotemark[268]}
\label{\detokenize{chapter_introduction/AI_PM:id7}}%
\begin{footnotetext}[268]\sphinxAtStartFootnote
\sphinxnolinkurl{https://mp.weixin.qq.com/s?\_\_biz=MjM5NzA5OTAwMA==\&mid=2650005725\&idx=1\&sn=75d33e7ae76805c9bd2db9d90147e27b\&chksm=bed8644a89afed5c69f23c01a86a601399269c9d604862305e94c89ba10890e951c550df0386\&scene=21\#wechat\_redirect}
%
\end{footnotetext}\ignorespaces 
AI行业(产品/市场)变化太快,而且是大调整。最多6个月,如果不去接触一线的情况,就会突然发现自己不熟悉市场了。

客户希望你解决他们的问题;他们不在乎你用的是哪种神经网络。你可能会发现自己根本不需要AI,这也没什么。

总之,机会多、难度大、变化又快又大,导致老板得承认自己的知识背景和精力有限,可能无法兼顾所有可能方向,必须让AI产品经理成为细分领域的小CEO,来做决策和承担更大压力。


\subparagraph{需求验证更为重要}
\label{\detokenize{chapter_introduction/AI_PM:id8}}
对于AI产品经理来说,产品能够达到60分的可用及格线往往比开发一款完美产品更为重要,只有我们能够通过技术验证场景,该一款产品才有可能进行商业落地,往往AI产品经理会遇到一个窘况,自己好不容易发现了一个场景需求,在个人看来通过人工智能手段能够大大提高效率,但是在经过半年甚至一年的技术验证之后发现并不能


\subparagraph{技术迭代更为快速}
\label{\detokenize{chapter_introduction/AI_PM:id9}}
研究人工智能翻译的人写的一段话,大致的含义就是他看了谷歌的一篇关于智能翻译的论文,发现自己之前所有的技术积累都已经落后了。关注前沿的行业技术更新就显得更为重要了,往往去年的某一个需求验证没有通过,但是随着时间推移又能够通过技术手段进行解决了


\subparagraph{与传统替代品博弈 39\sphinxfootnotemark[269]}
\label{\detokenize{chapter_introduction/AI_PM:id10}}%
\begin{footnotetext}[269]\sphinxAtStartFootnote
\sphinxnolinkurl{http://www.changgpm.com/thread-28-1-1.html}
%
\end{footnotetext}\ignorespaces 
概率论是AI产品的逻辑基础,所有的AI产品不像传统的互联网产品解决绝对的数据运算问题,而是某个需求在一定概率上得到解决,所以交付给用户的AI产品所能达到的解决概率,用户或客户是否能够接受。另外AI产品需要借助大量数据进行算法模型训练,所花费的成本,相比传统产品要高出很多。在产品定价上,如何与传统的替代品进行博弈,平衡投入产出比。


\subparagraph{确认自己是AI PM吗? 25\sphinxfootnotemark[270]}
\label{\detokenize{chapter_introduction/AI_PM:ai-pm-25}}%
\begin{footnotetext}[270]\sphinxAtStartFootnote
\sphinxnolinkurl{https://medium.com/@donnabella/what-does-it-mean-to-be-an-ai-product-manager-d67dc97da2e1}
%
\end{footnotetext}\ignorespaces \begin{itemize}
\item {} 
你是否首先查看了你的问题空间并收集了唯一的数据源?然后让开发者努力达到市场所要求的解决问题的准确度?

\item {} 
你了解你的AI性能/基准测试数据吗?

\item {} 
你的训练数据集是独特的和有区别的吗?

\item {} 
你的产品是否包含了所有合法的数据?

\item {} 
你是否一直在寻找额外的数据来源来继续改进你的产品?

\end{itemize}


\paragraph{AI产品经理的工作特点及发展方向:1\sphinxfootnotemark[271]}
\label{\detokenize{chapter_introduction/AI_PM:ai-1}}%
\begin{footnotetext}[271]\sphinxAtStartFootnote
\sphinxnolinkurl{https://www.boxuegu.com/news/4368.html}
%
\end{footnotetext}\ignorespaces 
把AI产品经理分为四个象限,分别是:
\begin{enumerate}
\sphinxsetlistlabels{\arabic}{enumi}{enumii}{}{.}%
\item {} 
突破算法产品经理,在大企业的研究部门、实验室,主要面向基础算法的更新迭代,强调对底层算法的逻辑理解、项目协调能力、竞品收集分析能力。在国内主要分布于BAT等一线互联网企业,或者讯飞、商汤等AI为主的企业;这类产品经理日常工作以研究为主,失败大于成功,不过没有苛刻的KPI,多为学术型人才。

\item {} 
创新产品经理,多为技术出身,在某个技术领域是个专家型人才。投入到初创公司,利用所掌握的技术能力,设计创新型产品,担任主要产品的设计工作,可以说是公司的关键人物,多是应用最新的前沿技术,结合垂直场景或领域,设计出创造型产品。

\item {} 
交付产品经理,多为产品出身,AI技术能力不是长项,但产品能力扎实,熟悉成熟AI技术,主要面向时间业务场景的AI落地,强调对行业的业务理解,典型场景分析,制定产品落地方案。

\item {} 
普及行业产品经理,多为非技术出身,熟悉成熟的AI技术能力,熟悉市场上成熟的AI产品,且具备深刻行业理解力,分析AI行业落地方向,能够很好的完成相关AI产品的拆解、分析、改造,从而制定产品整体规划方向,面向算法需求提出。

\end{enumerate}

\begin{figure}[H]
\centering
\capstart

\noindent\sphinxincludegraphics{{+AI+}.png}
\caption{+AI+\sphinxhref{https://www.bilibili.com/video/BV1MK411u7SH?p=10}{14}\sphinxfootnotemark[272]}\label{\detokenize{chapter_introduction/AI_PM:id49}}\end{figure}
%
\begin{footnotetext}[272]\sphinxAtStartFootnote
\sphinxnolinkurl{https://www.bilibili.com/video/BV1MK411u7SH?p=10}
%
\end{footnotetext}\ignorespaces 
\begin{figure}[H]
\centering
\capstart

\noindent\sphinxincludegraphics{{AIPM_ability}.png}
\caption{AI PM 能力模型}\label{\detokenize{chapter_introduction/AI_PM:id50}}\end{figure}


\paragraph{负责}
\label{\detokenize{chapter_introduction/AI_PM:id11}}\begin{itemize}
\item {} 
决定AI产品的核心功能、受众和预期用途(竞争优势)

\item {} 
评估输入数据管道,并确保它们在整个AI产品生命周期中得到维护

\item {} 
协调跨职能团队(数据工程、研究科学、数据科学、机器学习工程和软件工程)

\item {} 
决定关键界面和设计:用户界面和体验(UI/UX)和功能工程

\item {} 
将模型和服务器基础设施与现有软件产品集成

\item {} 
与ML工程师和数据科学家一起进行技术堆栈的设计和决策

\item {} 
交付AI产品并在发布后进行管理

\item {} 
与工程、基础设施和站点可靠性团队协调,以确保所有发布的特性都能得到大规模支持

\item {} 
负责人脸识别硬件产品的产品设计及客户对接工作。以客户为中心,持续进行产品迭代、项目周期管理及商业变现探索

\item {} 
负责设备端算法的场景化提升工作,协同算法及测试同学,持续优化数据闭环、效果评估、模型迭代等统筹及推进事宜

\item {} 
负责协助销售、运营、市场等部门或角色,进行定价设计、推广策略、线上线下活动等事宜的推进与落地
\sphinxhref{https://www.zhipin.com/job\_detail/579ddf7917fda82d0Xx909-1GFA~.html}{46}%
\begin{footnote}[273]\sphinxAtStartFootnote
\sphinxnolinkurl{https://www.zhipin.com/job\_detail/579ddf7917fda82d0Xx909-1GFA~.html}
%
\end{footnote}

\end{itemize}


\paragraph{AI 产品三要素 34\sphinxfootnotemark[274]}
\label{\detokenize{chapter_introduction/AI_PM:ai-34}}%
\begin{footnotetext}[274]\sphinxAtStartFootnote
\sphinxnolinkurl{http://reader.epubee.com/books/mobile/f4/f4c52db61d39acb835e2709cbed1585e/text00004.html?fromPre=last}
%
\end{footnotetext}\ignorespaces 
核心技术、产品化、商业化三要素
\begin{itemize}
\item {} 
核心技术:警惕“零感知”、领跑

\item {} 
产品化:快速了解、传递价值、融入生活

\item {} 
商业化:商业化则决定了产品将价值变现的能力

\end{itemize}


\paragraph{AI PM 主要能力}
\label{\detokenize{chapter_introduction/AI_PM:ai-pm}}
\begin{figure}[H]
\centering
\capstart

\noindent\sphinxincludegraphics{{PM_ability2}.png}
\caption{PM能力模型}\label{\detokenize{chapter_introduction/AI_PM:id51}}\end{figure}

一个合格的AI产品经理在具备传统产品经理能力模型的基础上,应该熟悉AI技术的效能与边界,对AI产业的三驾马车算法、算力、数据有一定的理解(对PM来说不要求上手coding,但是要对实现原理要理解),并且对部分垂直场景业务逻辑深耕,甚至达到跨领域协作的产品境界。展开来谈,这部分,我围绕算法、算力、数据、硬件、系统架构、业务六个维度来解释AI产品经理所需要的能力加成。\sphinxhref{https://www.jianshu.com/p/fd466ed1bda6}{2}%
\begin{footnote}[275]\sphinxAtStartFootnote
\sphinxnolinkurl{https://www.jianshu.com/p/fd466ed1bda6}
%
\end{footnote}

\begin{figure}[H]
\centering
\capstart

\noindent\sphinxincludegraphics{{AI_PM_knowledge}.jpg}
\caption{人工智能产品经理的知识体系\sphinxhref{http://www.xmamiga.com/3573/}{47}\sphinxfootnotemark[276]}\label{\detokenize{chapter_introduction/AI_PM:id52}}\end{figure}
%
\begin{footnotetext}[276]\sphinxAtStartFootnote
\sphinxnolinkurl{http://www.xmamiga.com/3573/}
%
\end{footnotetext}\ignorespaces 
1、
技术能力,AI产品常常要深入算法逻辑,产品经理不要求具备编码能力,但需要理解各团队的工作流程及模式特点,尤其是基础算法的业务模式。为了提需求写MRD和PRD、产品卖点、产品竞争优势、产品销售打法。

\sphinxhref{https://www.bilibili.com/video/BV1m4411W7yh?from=search\&seid=8482961800886970351}{PyTorch PM 回忆 1.0
版本诞生记:从科研到落地}%
\begin{footnote}[277]\sphinxAtStartFootnote
\sphinxnolinkurl{https://www.bilibili.com/video/BV1m4411W7yh?from=search\&seid=8482961800886970351}
%
\end{footnote}
\begin{itemize}
\item {} 
算法:算法就是计算或者解决问题的步骤。想用计算机解决特定的问题,就要遵循相应的算法。

\item {} 
算力:算力简单来说就是实现算法功能的资源要求,可以按照云、边、端来分类,云即云计算、边即嵌入式、端泛指服务器类资源,而这三者的背后核心都是集成电路,也就是芯片。嵌入式硬件:包含嵌入式微处理器、存储器(SDRAM、ROM、Flash等)、通用设备接口和I/O接口(A/D、D/A、I/O等)。实现一款产品往往还包含了外围元器件,比如
GPS、气压计、超声波、PIR
等等。作为PM的你就开始想了,需要多大的算力、运行内存预计需要多少、人脸库以及视频/照片存储需要多大的空间。是手动唤醒设备还是无感的呢?分别用什么元器件可满足需求呢?满足需求的情况下,预计硬件成本是多少?性能是否足够?

\item {} 
数据:数据包括这两层理解。第一层意思通用数据分析能力,这里不仅包括针对PV/UV/访问时长/新用户人数等运营指标的数据分析,更深一点的还会包括以分布式数据处理为核心的大数据技术(hadoop/spark/Hbase/Kafka等等),当然产品经理去了解大数据知识不是为了开发,而是为了在产品设计之初就协同研发一起评估中长期的技术需求和能力边界。
监督学习:不断地用标注后的数据去训练模型,不断调整模型参数,得到指标数值更高的模型。除此之外延伸的数据清晰、数据入库、数据验证、数据可视化等工作。数据是新的用户界面、用户体验。\sphinxhref{http://www.woshipm.com/ai/3379253.html}{50}%
\begin{footnote}[278]\sphinxAtStartFootnote
\sphinxnolinkurl{http://www.woshipm.com/ai/3379253.html}
%
\end{footnote}

\item {} 
硬件:AI硬件产品经理来说,还需要关注摄像头、门禁设备、传感器等硬件知识。

\item {} 
系统架构:了解Hadoop、Spark、Hive、ES、Flink、Kafka等组件,清楚各个组件的作用以及如何进行连接,更多的是对工程化有一定了解。\sphinxhref{https://www.zhihu.com/question/57815929/answer/1730120649}{38}%
\begin{footnote}[279]\sphinxAtStartFootnote
\sphinxnolinkurl{https://www.zhihu.com/question/57815929/answer/1730120649}
%
\end{footnote}

\end{itemize}

\sphinxstylestrong{以无人机产品为例}

硬件方面:
\begin{itemize}
\item {} 
处理器:计算平台,需要满足飞控、视觉算法(目标跟踪、Vslam等)算力要求。

\item {} 
传感器:GPS、激光雷达、视觉传感器、陀螺仪等数据采集器件,需要满足功能效果。

\end{itemize}

算法方面:
\begin{itemize}
\item {} 
飞控算法:飞控的性能要求,PID 控制、目标跟踪、手势识别等

\item {} 
定位导航:因为飞机不是在单一环境下运行,有可能GPS信号没有/不佳,光线环境佳等。需要多传感器融合,比如视觉+气压计+超声波+GPS
。

\item {} 
虽然算法产品经理有深度,但需要更高的广度去完成一个产品设计。

\end{itemize}

数据方面:
\begin{itemize}
\item {} 
各传感器的效果决定了数据的质量。

\end{itemize}

2、
分析及沟通能力,因为AI常常会涉及底层AI算法、工程化SDK开发、业务PASS中台开发、前端业务开发、智能硬件等多个团队,因此产品经理需要具备整体思维,对端到端的整体架构及相互模式有总分认知,能清晰定位问题点,具备与业务端及技术端的翻译能力,这样才能有效定位问题点,将各团队业务有效协同。

3、业务能力,随着AI技术的日益成熟,AI的行业落地正成为重点关注的方向,因此产品经理需要有行业知识及业务落地分析能力,不用了解细化业务流程,但需要清晰典型业务场景。

对于AI产品经理来说,思考的核心可以有两个走向(开源节流),第一个走向是传统问题能否利用AI技术更低成本的解决(节流),第二个走向是是否能利用AI技术创造需求并创造付费模式(开源)。

同时,我认为术业有专攻,优秀的产品一定是在特定行业垂直领域反复打磨的,因此一个合格的AI产品经理对某个业务领域一定要有深耕的,才会厚积薄发创造出更有沉淀的产品,AI产品经理本质上还是一个产品经理,一定是以业务为导向的。


\subparagraph{数据驱动 36\sphinxfootnotemark[280] 数据分析}
\label{\detokenize{chapter_introduction/AI_PM:data-analysis}}%
\begin{footnotetext}[280]\sphinxAtStartFootnote
\sphinxnolinkurl{https://weread.qq.com/web/reader/77532110721ea34a7751c9ak8f132430178f14e45fce0f7}
%
\end{footnotetext}\ignorespaces 
“数据驱动”的思维和能力也是AI产品经理的特需。为啥?因为AI跟数据是强关联的,所有对AI的训练和迭代都离不开数据,这也导致产品经理日常需要更关注数据这个闭环,不仅要依赖数据功能上线后的评估和优化,还要运用数据进行策略设计。

将数据视作企业生产经营的原材料,“数据驱动”的过程可理解为DIPOA模型,即Data—Input—Process—Output—Action
Point,数据作为生产要素,输入到数据应用的程序化链路中,经过一定步骤的加工与处理,形成相应的输出,再将这些输出作用到对应的作用点上产生价值,完成“数据驱动”的一次作业链条。DIPOA模型是参照经典的SIPOC模型来进行设计,能以传输链条的形式生动的解释“数据驱动”的作用过程与机制。

\begin{figure}[H]
\centering
\capstart

\noindent\sphinxincludegraphics{{DIPOA}.png}
\caption{DIPOA}\label{\detokenize{chapter_introduction/AI_PM:id53}}\end{figure}


\subparagraph{怎么衡量“懂技术”7\sphinxfootnotemark[281]}
\label{\detokenize{chapter_introduction/AI_PM:id12}}%
\begin{footnotetext}[281]\sphinxAtStartFootnote
\sphinxnolinkurl{https://zhuanlan.zhihu.com/p/33524676}
%
\end{footnotetext}\ignorespaces 
无论你是三个阵营中的哪个,你的技术知识,应该帮助你回答下面几个问题:
\begin{enumerate}
\sphinxsetlistlabels{\arabic}{enumi}{enumii}{}{.}%
\item {} 
人工智能技术可能会给你的产品带来多大价值?因为产品永远是需求驱动,而非技术驱动。别忘了,再前沿的技术,从理论到产品落地是有巨大投入的。

\item {} 
从技术角度,将人工智能技术应用到你的产品中需要哪些资源或准备?例如需要更多的数据,更完善的算法模型?尽管很难量化这样的需求,你还是要尽可能的掌握更多信息去做判断。

\item {} 
从技术角度识别人工智能领域中的哪些理论已经有了最佳实践,即需要判断技术的成熟度。

\end{enumerate}

当你在将AI技术应用到产品中时,你应该能够给出答案:
\begin{enumerate}
\sphinxsetlistlabels{\arabic}{enumi}{enumii}{}{.}%
\item {} 
识别人工智能带来的价值是否真的被客户认可?这样的技术真的比传统技术更好吗?你需要多长时间或多少样例数据来验证你的人工智能产品已经站住脚了?

\item {} 
一旦产品上线后的效果没有预期好,你是否有备用计划?

\item {} 
任何一个机器学习功能的上线都需要占用研发80\%或更多的时间来完成对数据的准备(机器学习对数据的准备更占用时间),你是否已经和研发部门充分沟通并达成一致?

\end{enumerate}


\subparagraph{技术瓶颈 9\sphinxfootnotemark[282]–可解释性}
\label{\detokenize{chapter_introduction/AI_PM:id13}}%
\begin{footnotetext}[282]\sphinxAtStartFootnote
\sphinxnolinkurl{http://www.woshipm.com/pmd/798007.html}
%
\end{footnotetext}\ignorespaces \begin{enumerate}
\sphinxsetlistlabels{\arabic}{enumi}{enumii}{}{.}%
\item {} 
深度学习对于技术人员的经验依赖性依然很强,调参、收集数据、架构设计等没有通识的普遍规律,黑盒下的操作还是占很大比例。

\item {} 
对于每个技术背后的原理,知识体系往往存在着断层,很多过程我们是无法用语言或图像描述出来的。

\item {} 
算法可视化很苦恼,可能连设计者都无法用任何方式将内在的原理可视化给用户看。

\end{enumerate}


\subparagraph{Google总结了可解释性原则如下10\sphinxfootnotemark[283]}
\label{\detokenize{chapter_introduction/AI_PM:google10}}%
\begin{footnotetext}[283]\sphinxAtStartFootnote
\sphinxnolinkurl{https://easyai.tech/author/xiaoqiang/page/4/}
%
\end{footnotetext}\ignorespaces \begin{itemize}
\item {} 
了解隐藏层的作用:深层学习模型中的大部分知识都是在隐藏层中形成的。在宏观层面理解不同隐藏层的功能对于解释深度学习模型至关重要。

\item {} 
了解节点的激活方式:可解释性的关键不在于理解网络中各个神经元的功能,而是在同一空间位置一起激发的互连神经元群。通过互连神经元组对网络进行分段将提供更简单的抽象级别来理解其功能。

\item {} 
理解概念是如何形成的:了解神经网络形成的深度,然后可以组合成最终输出的个体概念是可解释性的另一个关键构建块。

\end{itemize}


\subparagraph{AI 团队}
\label{\detokenize{chapter_introduction/AI_PM:id14}}
\begin{figure}[H]
\centering
\capstart

\noindent\sphinxincludegraphics{{AI_team}.png}
\caption{常见 AI 团队架构图}\label{\detokenize{chapter_introduction/AI_PM:id54}}\end{figure}

\begin{figure}[H]
\centering
\capstart

\noindent\sphinxincludegraphics{{AI_Position}.png}
\caption{示例:ultralytics}\label{\detokenize{chapter_introduction/AI_PM:id55}}\end{figure}

数据组负责整理和分析数据。AI
算法组是公司的工程基础。随着公司团队的成熟,可衍生出 AI
研究组。平台组负责部署、维护和扩展关键基础架构。有些公司会成立硬件组专注物理产品的开发。
AI
应用和产品组在产品的生命周期中开发和管理产品。商业化团队由销售、市场营销和法律等专业人才组成,以确保产品能够成功发布并被市场接受。
\begin{itemize}
\item {} 
ML产品经理:制定任务优先级,推进项目进展

\item {} 
DevOps工程师:部署和运维线上系统

\item {} 
数据工程师:构建data pipeline,数据存储基础,相关监控等

\item {} 
机器学习工程师:训练和部署模型

\item {} 
机器学习研究员:更面向未来的算法技术调研和前沿探索

\item {} 
数据科学家:一个非常广义的职位,总体来说会更偏向算法,数据分析与业务连接部分

\end{itemize}

\begin{figure}[H]
\centering
\capstart

\noindent\sphinxincludegraphics{{ML_roles}.png}
\caption{ML角色}\label{\detokenize{chapter_introduction/AI_PM:id56}}\end{figure}


\subparagraph{整体内的角色}
\label{\detokenize{chapter_introduction/AI_PM:id15}}
\begin{figure}[H]
\centering
\capstart

\noindent\sphinxincludegraphics{{talents_structure}.png}
\caption{人才架构\sphinxhref{https://www.zhihu.com/question/53498066/answer/488592129}{52}\sphinxfootnotemark[284]}\label{\detokenize{chapter_introduction/AI_PM:id57}}\end{figure}
%
\begin{footnotetext}[284]\sphinxAtStartFootnote
\sphinxnolinkurl{https://www.zhihu.com/question/53498066/answer/488592129}
%
\end{footnotetext}\ignorespaces 

\subparagraph{行业解决方案经理}
\label{\detokenize{chapter_introduction/AI_PM:id16}}
行业方案经理主要是企业从单产品或产品组合的销售模式转型向行业项目销售模式中产生的一个PM型岗位。卖的不再是某几个产品,而是整套解决方法方案。

垂直行业方案也分不同发展阶段,这里不多赘述。就讲一下大概AI+行业的价值所在:

对企业来说:AI+行业方案能有效的帮助企业从产品组合式销售中脱离出来,变成一个通过帮客户解决业务痛点的端到端solution,对客户来说吸引力更高。而AI由于带来数据结构化提取能力上的提升,海量的结构化数据能帮助企业向大数据公司和云服务公司转型,毛利更高,销售额更高,能有效提高经营利润,缓解价格战的压力。和客户双赢。

对客户来说:通过大数据、云服务和AI技术的业务融合解决方案能解放很大一部分运营效率和提升经营能力,能为客户解决痛点需求,会有非常大的方案吸引力。

而行业方案经理就要求必须懂垂直行业,懂客户需求,想尽办法给客户创造价值,所以好的行业方案经理不光要懂传统产品,也要懂软件平台,懂云服务,懂AI,懂数据分析,懂业务模型…等等。这样才能规划开发行业产品和有价值的行业解决方案。


\subparagraph{AI产品经理与编码技术人员的关系(区别于算法技术人员)4\sphinxfootnotemark[285]}
\label{\detokenize{chapter_introduction/AI_PM:ai-4}}%
\begin{footnotetext}[285]\sphinxAtStartFootnote
\sphinxnolinkurl{http://www.woshipm.com/pmd/1629952.html}
%
\end{footnotetext}\ignorespaces 
张小龙、雷军、雷军认为其在金山的优点是勤劳,缺点是没有顺势而为,说白了什么叫顺势而为。笔者理解顺势而为就是产品思维,以用户为中心的思维再来看张小龙。

张小龙做微信的时候他指出:一个亿级用户的产品经理,无需做到透彻思考人性和产品的所有方面,但需要在极端现实主义和极端理想主义之间取得平衡。做产品力求简单美,要满足用户“贪嗔痴”。

关心的是用户!!!


\subparagraph{AI产品经理与算法技术人员的关系4\sphinxfootnotemark[286]}
\label{\detokenize{chapter_introduction/AI_PM:ai4}}%
\begin{footnotetext}[286]\sphinxAtStartFootnote
\sphinxnolinkurl{http://www.woshipm.com/pmd/1629952.html}
%
\end{footnotetext}\ignorespaces 
\begin{figure}[H]
\centering
\capstart

\noindent\sphinxincludegraphics{{7c24fb51abddcf2e46282d021974942cca1dc3ea}.html}
\caption{20 万、50 万、100 万年薪的算法工程师在能力素质模型上有哪些差距?}\label{\detokenize{chapter_introduction/AI_PM:id58}}\end{figure}

『你的边界在哪?技术型产品和技术的边界在哪?』
\sphinxstylestrong{技术负责给出问题的解,而技术型产品负责给出要求解的问题。}

像灵犬的产品经理还要写灵犬反低俗助手产品的产品介绍、产品Q\&A。产品用户调研、产品推广,产品策略制定例如通过灵犬小程序产品可以收集数据来优化今日头条的本体反低俗模型产品。

什么需求?为了解决今日头条本身平台上鉴定低俗内容的

行为:检测其阅读内容的健康指数,输出对应的分数、评级和结论。不同于色情信息,处理低俗信息的一个难点在于,人们对于低俗的判断标准具有一定的主观性,合理筛选难度大。团队根据测试员的意见反馈
\begin{enumerate}
\sphinxsetlistlabels{\arabic}{enumi}{enumii}{}{.}%
\item {} 
灵犬的色彩以及对搜索框居中设计的布局可见灵犬是个单独的产品模块即可称之为独立的产品。

\item {} 
命名实体(NER)技术用来识别里面人名、地名、事物的名称等关键名词

\item {} 
灵犬反低俗助手产品的产品介绍、产品Q\&A、产品用户调研、产品推广、产品策略制定。

\end{enumerate}

AI产品经理应该能够使用设计专家使用的快速创新工具,包括用户体验模型、线框和用户调查。在这个阶段,确定产品要解决的问题或机会也是至关重要的。在他的文章“产品经理的机器学习”中,Neal
Lathia将ML问题类型分为六个类别:排名、推荐、分类、回归、聚类和异常检测。AI
PM只有在尽可能准确地确定他们想要解决的问题,并将问题归入其中一个类别之后,才能进入功能开发和实验阶段。\sphinxhref{https://www.oreilly.com/radar/practical-skills-for-the-ai-product-manager/}{22}%
\begin{footnote}[287]\sphinxAtStartFootnote
\sphinxnolinkurl{https://www.oreilly.com/radar/practical-skills-for-the-ai-product-manager/}
%
\end{footnote}

AI产品经理在需求评审【由项目经理(有单独项目经理的公司)组织产品经理、研发人员、测试人员、UI
设计人员听产品经理讲解需求的过程\sphinxhref{https://weread.qq.com/web/reader/8d232b60721a488e8d21e54kc51323901dc51ce410c121b}{12}%
\begin{footnote}[288]\sphinxAtStartFootnote
\sphinxnolinkurl{https://weread.qq.com/web/reader/8d232b60721a488e8d21e54kc51323901dc51ce410c121b}
%
\end{footnote}】的阶段,需要与算法共同明确的主要有以下几点:\sphinxhref{https://m.k.sohu.com/d/495625828?channelId=1\&page=1}{11}%
\begin{footnote}[289]\sphinxAtStartFootnote
\sphinxnolinkurl{https://m.k.sohu.com/d/495625828?channelId=1\&page=1}
%
\end{footnote}
\begin{itemize}
\item {} 
模型目标

\item {} 
特征选择

\item {} 
数据采集

\item {} 
验收标准

\end{itemize}

至于数据预处理、模型选取、特征工程、调参等等部分,如果你有精力和能力去理解那自然是好的,但如果不能,只需要理解算法运作的基本原理即可。


\subparagraph{编码技术人员与算法技术人员的关系}
\label{\detokenize{chapter_introduction/AI_PM:id17}}
在软件开发中,无论是多么匪夷所思的
BUG,大都能查出具体的原因并给出修复方案,这个问题是确定的。

但在视觉模型这边,无论是多么合情合理的 bad
case,大都只能给出合理的推测:缺特定场景的数据?超参数不合适?没收敛好?那补数据、调参、重新训练之后一定能解决这个问题吗?会不会按下葫芦浮起瓢?不知道。这个问题是不确定的。


\subparagraph{VS 数据产品经理 19\sphinxfootnotemark[290]}
\label{\detokenize{chapter_introduction/AI_PM:vs-19}}%
\begin{footnotetext}[290]\sphinxAtStartFootnote
\sphinxnolinkurl{https://www.sohu.com/a/397318209\_114819}
%
\end{footnotetext}\ignorespaces 
四分五裂的\sphinxhref{https://g.yuque.com/amir/pm/ux6yun}{数据产品经理}%
\begin{footnote}[291]\sphinxAtStartFootnote
\sphinxnolinkurl{https://g.yuque.com/amir/pm/ux6yun}
%
\end{footnote}:
\begin{enumerate}
\sphinxsetlistlabels{\arabic}{enumi}{enumii}{}{.}%
\item {} 
设计数据仓库、统筹数据治理的

\item {} 
做B报表搞可视化分析的

\item {} 
规划画像标签体系、数据中台的

\item {} 
设计DMP、运营策略平台的

\end{enumerate}

总之,凡是核心依赖于数据,产岀物是可交互操作的实体,就都可以算成数据产品,而做这些东西的非研发人员,也自然就都是数据产品经理。


\subparagraph{AI 产品 VS 数据产品}
\label{\detokenize{chapter_introduction/AI_PM:ai-vs}}
\begin{figure}[H]
\centering
\capstart

\noindent\sphinxincludegraphics{{data_vs_AI_product}.jpg}
\caption{AI 产品 VS
数据产品\sphinxhref{https://www.zhihu.com/question/425088404/answer/1613313769}{5}\sphinxfootnotemark[292]}\label{\detokenize{chapter_introduction/AI_PM:id59}}\end{figure}
%
\begin{footnotetext}[292]\sphinxAtStartFootnote
\sphinxnolinkurl{https://www.zhihu.com/question/425088404/answer/1613313769}
%
\end{footnotetext}\ignorespaces 
第一层,\sphinxstylestrong{数据治理层面},数据产品经理在数据治理层面会将数据整体做治理,纵向上梳理业务来源、到数据仓库再到数据集市的数据处理方式,横向上梳理数据主题、数据规范和存储模式等数据模型,在这中间再孵化出如数据开发平台、质量监控平台、数据资产管理、任务调度平台等等。

\begin{figure}[H]
\centering
\capstart

\noindent\sphinxincludegraphics{{AI_data_PM}.jpg}
\caption{一体两翼}\label{\detokenize{chapter_introduction/AI_PM:id60}}\end{figure}

第二层,\sphinxstylestrong{数据/算法平台层面},数据产品经理基于已有的数据情况,建设对接业务方的通用能力,包括建设通用型用户标签和人/货/场的画像体系,建设通用的广告营销平台,建设通用的查询分析平台等。

AI平台型产品经理,这里的平台可以是数据众包平台、AI能力开放平台、AI训练平台,都是将数据标注能力和算法模块能力做上层延展,对内或者对外形成通用能力的输出。以数据众包平台为例,可以是数据标注的交易撮合平台,也可以是将内部的标注能力直接对外输出,这样一方面可以不但可以获取直接的人力收入,也可以间接获得数据收入。而AI自助训练平台,则是将基础模型(如基础图像分类模型)包装成可流程化、可视化的训练过程,让非算法同学可以参与到训练过程中,自己完成模型的训练和优化。

第一层和第二层往往会有交叉,公司内部如果愿意搭建AI中台,那么第一层和第二层上的内容就应该是整合打通的,从数据流入、标注到模型训练过程,整体做到可视的一体化产品。

第三层,\sphinxstylestrong{业务应用层面},数据产品典型的像字节跳动典型的推荐系统,整体可以划分如客户营销体系、客户风控体系、决策分析体系(如管理驾驶舱等)。AI产品则是在可以明确边界和可收敛识别规则的业务场景进行应用,包括语音、计算机视觉、自然语言处理的场景,典型的如内容审核(判断内容是否违规)、内容理解(判断图像中是否存在的目标和场景)、写诗机器人(将诗的固定模式提取后进行诗句输出)等。


\subparagraph{产品目标不同}
\label{\detokenize{chapter_introduction/AI_PM:id18}}
数据产品经理的产品目标是用数据确认确定性的需求;AI产品经理的产品目标是创造性的解决不确定性的产品需求。

当增长遇瓶颈;当产品不能精准的推荐给用户;当生产效率变低;当产品经理不能预测新的产品需求和新的服务需求;当人力成本变高,当有些固定流程的工作可以被机器人代替;

前类主要是数据产品经理要解决的问题,通过数据来验证产品提出的产品需求的正确性,通过上线后的数据来发现产品需要迭代改进甚至创新的点,通过数据分析,数据挖掘发现原本发现不了的产品问题,改进问题。

后类主要是AI产品经理的产品目标,AI一方面能帮人节省时间,另外能预测原本发现不了的产品和服务需求,还有AI能够解决不确定性的产品服务需求。


\subparagraph{产品过程步骤不同}
\label{\detokenize{chapter_introduction/AI_PM:id19}}
数据产品经理的数据分析的步骤一般可以分为如下6个步骤:
\begin{enumerate}
\sphinxsetlistlabels{\arabic}{enumi}{enumii}{}{.}%
\item {} 
明确分析的目的

\item {} 
数据准备

\item {} 
数据清洗

\item {} 
数据分析

\item {} 
数据可视化

\item {} 
分析报告

\end{enumerate}

AI产品经理案例:训练神经网络经典案例拆解:
\begin{enumerate}
\sphinxsetlistlabels{\arabic}{enumi}{enumii}{}{.}%
\item {} 
选定一个基础模型

\item {} 
设定初始化参数代入模型

\item {} 
用训练集对模型进行训练

\item {} 
通过一些数量指标,评估训练误差

\item {} 
如果训练误差不满足要求,继续调整参数

\item {} 
重复7–8次

\item {} 
采集新的数据,生成新的数据集。

\end{enumerate}


\subparagraph{懂不懂AI技术?懂的程度?}
\label{\detokenize{chapter_introduction/AI_PM:id20}}
产品经理是发现需求并在不确定的需求里面确定需求,而Kaggle一类的思维模式是辅助识别需求。

从算法工程师的角度切入做产品经理的人比比皆是,但是非算法工程师出身的产品负责人人数更多。

柔宇的刘博士懂柔性传感器技术,但是需要帮这种技术产品化,故此需要用柔性显示+做出柔记(柔记:一种写在真纸上,记载在AI芯片里的智能手写本),抢在三星之前发布柔性可折叠手机做出柔派(柔性技术+AI+新交互的手机和PAD同体款),早日做出柔派是要抢占用户心智。


\subparagraph{思维VS技术}
\label{\detokenize{chapter_introduction/AI_PM:vs}}
如果方向和方法错了你越执着于执行和操作,你错的越深你的产品越没有用户和客户。

我们常说:需求是洞,产品是钉子,技术选型是锤子,即AI产品经理本质核心工作是持续从用户需求出发,满足用户需求。洞察、分析、不断的满足用户需求。

真正由 AI
驱动的产品并不多,理性认识你负责的那个模型对项目到底有多重要,可以更合理的调配工作时间与精力,也能在和外部对接时省去很多不必要的口舌。


\subparagraph{分饰角色}
\label{\detokenize{chapter_introduction/AI_PM:id21}}
如何平衡炼丹、工程和业务?这得看你在实际工作中分饰几个角色。

如果三个角色都是你自己来,那么这就只是一个时间和精力分配的问题。

如果只是模型的训练和部署由你负责,那么这就是时间和精力分配+与产品有效沟通两个问题。

以上两种情况都有大量的先进经验可以借鉴,建议直接站内搜索。

如果算法工程师只负责训练,问题会相对复杂一些。


\subparagraph{缺乏角色 22\sphinxfootnotemark[293]}
\label{\detokenize{chapter_introduction/AI_PM:id22}}%
\begin{footnotetext}[293]\sphinxAtStartFootnote
\sphinxnolinkurl{https://www.oreilly.com/radar/practical-skills-for-the-ai-product-manager/}
%
\end{footnotetext}\ignorespaces 
缺乏特定的角色定义并不会阻止成功,但它确实引入了随着业务规模的扩大而积累技术债务的风险。重要的是,一个组织的整体数据战略包括路标(可能是产品管道中的阶段),标志着升级AI资源、技术和领导力的适当时间和条件。这一责任落在行政领导身上。如果没有高管的支持,强大的人工智能产品管理和工程领导力就无法蓬勃发展。


\subparagraph{AI各层次}
\label{\detokenize{chapter_introduction/AI_PM:id23}}
美团外卖:实时智能调度

\begin{figure}[H]
\centering
\capstart

\noindent\sphinxincludegraphics{{AI_cengci}.png}
\caption{层次 \sphinxhref{https://www.bilibili.com/video/BV1MK411u7SH?p=8}{15}\sphinxfootnotemark[294]}\label{\detokenize{chapter_introduction/AI_PM:id61}}\end{figure}
%
\begin{footnotetext}[294]\sphinxAtStartFootnote
\sphinxnolinkurl{https://www.bilibili.com/video/BV1MK411u7SH?p=8}
%
\end{footnotetext}\ignorespaces 

\subparagraph{流程}
\label{\detokenize{chapter_introduction/AI_PM:id24}}
泳道流程图:

\begin{figure}[H]
\centering
\capstart

\noindent\sphinxincludegraphics{{all_process}.png}
\caption{全流程}\label{\detokenize{chapter_introduction/AI_PM:id62}}\end{figure}

下图提供了与这些角色(在纵轴上)的生命周期的每个阶段(在横轴上)相关联的任务(在蓝色中)和工件(在绿色中)的网格视图:

\begin{figure}[H]
\centering
\capstart

\noindent\sphinxincludegraphics{{ms_flow_chart}.png}
\caption{微软(TDSP)}\label{\detokenize{chapter_introduction/AI_PM:id63}}\end{figure}


\subparagraph{成长路径}
\label{\detokenize{chapter_introduction/AI_PM:id25}}
\begin{figure}[H]
\centering
\capstart

\noindent\sphinxincludegraphics{{AIPM_env}.png}
\caption{不同企业背景下AI产品经理的成长环境\sphinxhref{https://weread.qq.com/web/reader/40632860719ad5bb4060856k9f6326602389f61408e3715}{30}\sphinxfootnotemark[295]}\label{\detokenize{chapter_introduction/AI_PM:id64}}\end{figure}
%
\begin{footnotetext}[295]\sphinxAtStartFootnote
\sphinxnolinkurl{https://weread.qq.com/web/reader/40632860719ad5bb4060856k9f6326602389f61408e3715}
%
\end{footnotetext}\ignorespaces 

\subparagraph{学习路线}
\label{\detokenize{chapter_introduction/AI_PM:id26}}
AI是一个技能型的职业,其主要的机会在于细分领域和交叉领域,AI产品经理所面临的最大的难度其实就是在于怎么去基于场景去定义需求。
主要的学习路线我个人认为可以分为三部走:
\begin{enumerate}
\sphinxsetlistlabels{\arabic}{enumi}{enumii}{}{.}%
\item {} 
第一步,找到自己的兴趣点和特长,最好和自己之前有技能重合的领域,分技术大类的话其实也就是人机交互/计算机视觉/自然语言处理/生物特征识别等几个大类,这些大类又相对应的分出很多小类。

\item {} 
第二步就是要选择好自己的方向是基于平台类,还是聊天类亦或者是基于场景类。

\item {} 
最后一步其实就是实施转型了,这是最为艰难的一步,当然也是最终要的一步,这里我们重点聊聊实施转型这一步。

\end{enumerate}


\paragraph{误区}
\label{\detokenize{chapter_introduction/AI_PM:id27}}
两个主要的非技术问题可能会阻碍AI的发展:要么你的团队过于学术化,要么你的所有项目都是长期性的。\sphinxhref{https://medium.com/@vlad.mysla/my-first-year-as-a-product-manager-for-artificial-intelligence-ai-f8485c644713s}{45}%
\begin{footnote}[296]\sphinxAtStartFootnote
\sphinxnolinkurl{https://medium.com/@vlad.mysla/my-first-year-as-a-product-manager-for-artificial-intelligence-ai-f8485c644713s}
%
\end{footnote}


\subparagraph{不是学术研究 5\sphinxfootnotemark[297]}
\label{\detokenize{chapter_introduction/AI_PM:id28}}%
\begin{footnotetext}[297]\sphinxAtStartFootnote
\sphinxnolinkurl{https://www.zhihu.com/question/425088404/answer/1613313769}
%
\end{footnotetext}\ignorespaces 
如何确定哪些任务可以用人工智能完成?如何划分数据集?模型性能不佳时如何判断问题所在?如何判断某个改进思路的可行性?深度学习项目通常需要消耗大量的资源,与其投入一两个月的精力实现某思路结果发现性能并不尽如人意,“谋定而后动”是十分必要的。目前,直接关注深度学习在实际项目实践中经验心得的中文资料还十分匮乏,本文力图对深度学习项目实践从项目选型、数据准备、训练、模型分析、模型部署全阶段的注意事项和技巧进行梳理。

学术研究的目标。学术研究的目的是为人类认识世界提供新的知识,因此很看重独创性(novelty)和
方法可复现性。创新是学术研究的核心,同样的研究问题,无论以何种形式,已经有人发表,那么该工作的创新性将大打折扣。

项目实践的目标。而项目实践的目标通常是得到一个高性能的模型,对这个模型是否独创或可复现并不看重。这并不是说项目实践比学术研究更简单,不客气地讲,有的学术研究就是在想方设法过拟合几个基准(benchmark)数据集的测试集以在论文中展示出漂亮的数字。而项目实践对模型的泛化性能十分看重,如果过拟合测试集,只会让你的老板短暂的高兴一下之后将发现线上指标十分糟糕。此外,要想做好项目实践,除了需要读论文、复现结果、产生自己idea的能力外,还需要下载处理数据、调试代码等脏活累活,这两部分同等重要。


\subparagraph{沉溺论文 6\sphinxfootnotemark[298]}
\label{\detokenize{chapter_introduction/AI_PM:id29}}%
\begin{footnotetext}[298]\sphinxAtStartFootnote
\sphinxnolinkurl{https://zhuanlan.zhihu.com/p/73661738}
%
\end{footnotetext}\ignorespaces 
不要沉溺在论文的海洋。现在人工智能正值热潮,每年新发表的论文非常多。而机器学习是实践科学,尤其是当你不是该领域专家时,事先很难知道哪种方案在实际中效果最好,通常需要尝试很多的思路。机器学习实战的过程是思路、代码实现、实验结果的迭代循环。迭代循环的越快,取得的进展越大。不要在开始前想的过多,尤其不要一开始就想着设计和构建一个完美的系统,最初的方案要越早构建和训练越好,之后根据偏差/方差分析和误差分析确定下一步的工作方向,并进行迭代。

一流论文大杀四方,灌水论文没什么价值,发论文是实习生的工作,full\sphinxhyphen{}time
是要给公司赚钱的,不要心存侥幸,大清已经亡了。如果没有什么厉害的成果,而你又志在
industry,不要浪费时间灌水了,好好准备面试。
\sphinxhref{https://www.deepshare.net/index.php?c=article\&id=212}{48}%
\begin{footnote}[299]\sphinxAtStartFootnote
\sphinxnolinkurl{https://www.deepshare.net/index.php?c=article\&id=212}
%
\end{footnote}


\subparagraph{学生思维}
\label{\detokenize{chapter_introduction/AI_PM:id30}}
读书时,我们主要是面向论文炼丹的:需求就是SOTA,数据主要靠公开数据集,我们要做的主要就是训练、迭代、刷分——这是典型的Hard
\& Clean Problem。

工业界不是这样的。公司是花钱请人来在用\sphinxstylestrong{有限的资源}解决实际问题的。

实际情况往往是:需求常生变、采集有周期、标注有成本、清洗要策略、训练要时间、测试卡标准。和算法工程师最相关的训练迭代,往往既不是最关键的,也不是最复杂的,甚至都未必是最贵的环节。

需求分析的重要性是显而易见的,需求都歪了后面再努力也是白给。

要对全流程有所理解,然后根据实际情况坚持以理服人,对产品划清能力边界,严防脑洞大开;对工程尽量提供更有确定性的消息,以及,文档例程不要过于放飞自我。


\subparagraph{唯SOTA论}
\label{\detokenize{chapter_introduction/AI_PM:sota}}
因为选择SOTA != 选择未经验证的解决方案 != 选择了高风险。

除非你在做 概念验证 (PoC) 开发
或者预研性质的项目,否则一个项目的目标、成本、排期、风险点一定是在实施之前就基本确认的。用尽可能低的预算解决问题的方法有很多,但未经广泛验证的
SOTA 绝非上策。

在实践中,如果不能加数据,那么通常是 simple, solid work
更好用一点:花样百出 Attention
模块没有几个在多数任务中都能即插即用即涨点的;ArcFace
的核心就那么几行,用过都说好。


\subparagraph{别研究难题过长}
\label{\detokenize{chapter_introduction/AI_PM:id31}}
你应该是科学的,你应该研究一些可能会花费你几个季度甚至几年时间的难题。但如果你长时间不提供业务成果,你可能会损害团队的声誉和对人工智能的信心。因此,试着在良好的理论和交付可测试的产品特性之间取得平衡,让它们成为现实。一开始,并不是每件事都能成功,这没关系。


\subparagraph{形势简单化}
\label{\detokenize{chapter_introduction/AI_PM:id32}}
除非你即将或已经入职的公司基础设施完备、支持团队给力,不然在实际工作中难免碰到各种一地鸡毛的事情,比如:
\begin{itemize}
\item {} 
数据清洗、预处理没有合适的工具

\item {} 
实验管理没有合适的工具(MLFlow、 Weights \& Biases)

\item {} 
结果可视化没有合适的工具

\end{itemize}

Serving Infrastructure: This includes tools for model development (such
as the Cloudera Data Science Workbench, Domino Data Lab, Data Robot, and
Dataiku) and production serving infrastructure (such as Seldon,
Sagemaker, and TFX).
\sphinxhref{https://www.oreilly.com/radar/practical-skills-for-the-ai-product-manager/}{24}%
\begin{footnote}[300]\sphinxAtStartFootnote
\sphinxnolinkurl{https://www.oreilly.com/radar/practical-skills-for-the-ai-product-manager/}
%
\end{footnote}


\paragraph{难题 29\sphinxfootnotemark[301]}
\label{\detokenize{chapter_introduction/AI_PM:id33}}%
\begin{footnotetext}[301]\sphinxAtStartFootnote
\sphinxnolinkurl{https://www.bilibili.com/video/BV1xK4y1j7ep}
%
\end{footnotetext}\ignorespaces \begin{itemize}
\item {} 
通用产品vs定制方案

\item {} 
完全没有数据vs没有标注数据

\item {} 
炫酷的技术vs可怜的性能指标

\item {} 
被宣传鼓舞的期待vs低可用性的产品

\item {} 
产品整体指标vs个别 bad case

\item {} 
沟通难题:与技术人员、与客户

\end{itemize}


\subparagraph{如何跟算法工程师沟通的}
\label{\detokenize{chapter_introduction/AI_PM:id34}}
\begin{figure}[H]
\centering
\capstart

\noindent\sphinxincludegraphics{{communicate2Engineer}.png}
\caption{与算法工程师沟通\sphinxhref{http://shujuren.club/a/AI0102.html}{51}\sphinxfootnotemark[302]}\label{\detokenize{chapter_introduction/AI_PM:id65}}\end{figure}
%
\begin{footnotetext}[302]\sphinxAtStartFootnote
\sphinxnolinkurl{http://shujuren.club/a/AI0102.html}
%
\end{footnotetext}\ignorespaces \begin{itemize}
\item {} 
比如在真实场景中用户投诉来对齐需求,并和算法工程师沟通产品的行业背景

\item {} 
定义边界:为了达到客户预期,必须提高xx准确率的问题。

\end{itemize}
\begin{enumerate}
\sphinxsetlistlabels{\arabic}{enumi}{enumii}{}{.}%
\item {} 
先根据用户反馈,定位问题;

\item {} 
然后线上看数据,分析问题;

\item {} 
列出问题可能存在的根源,哪些是算法问题,哪些是工程问题等;

\item {} 
做了一个问题分析表,附带数据图片样本;

\item {} 
根据问题分析表中发现的具体问题,和 RD 沟通哪些能解决;

\end{enumerate}
\begin{itemize}
\item {} 
再看自己可以做哪些事去推进,如果是新需求可能要\sphinxstylestrong{收集更多数据},并申请标注团队资源等…

\end{itemize}


\paragraph{Awesome}
\label{\detokenize{chapter_introduction/AI_PM:awesome}}
\sphinxurl{https://www.yuque.com/gdbwhd/mesmph/uuuuu\#LsKnF}


\paragraph{To B/C 13\sphinxfootnotemark[303]}
\label{\detokenize{chapter_introduction/AI_PM:to-b-c-13}}%
\begin{footnotetext}[303]\sphinxAtStartFootnote
\sphinxnolinkurl{https://www.bilibili.com/video/BV1MK411u7SH?p=14}
%
\end{footnotetext}\ignorespaces 
以科大讯飞为例
\begin{enumerate}
\sphinxsetlistlabels{\arabic}{enumi}{enumii}{}{.}%
\item {} 
原有语音识别技术,要做商业化

\item {} 
需要结合行业找应用场景,做解决方案

\end{enumerate}


\subparagraph{TO B(汽车行业)}
\label{\detokenize{chapter_introduction/AI_PM:to-b}}
核心:调硏客户需求、出解决方案
\begin{enumerate}
\sphinxsetlistlabels{\arabic}{enumi}{enumii}{}{.}%
\item {} 
熟悉行业,要求对行业必须深度理解

\item {} 
找客户,调研客户需求、出解决方案、投标/比标

\end{enumerate}


\subparagraph{TO C(小蛋机器人)}
\label{\detokenize{chapter_introduction/AI_PM:to-c}}
核心:为其他行业树立标杆
\begin{enumerate}
\sphinxsetlistlabels{\arabic}{enumi}{enumii}{}{.}%
\item {} 
超强创新能力

\item {} 
注重性能的稳定性

\end{enumerate}


\subparagraph{度量标准}
\label{\detokenize{chapter_introduction/AI_PM:id35}}
对于没有成熟数据或机器学习实践的企业来说,定义和同意度量标准通常是困难的。

最坏的情况是企业没有任何指标。如果业务缺乏度量标准,它可能也缺乏数据基础设施、收集、治理等方面的规程。

AI
产品应该如何衡量?\sphinxhref{https://coffee.pmcaff.com/article/1329730610781312/pmcaff?utm\_source=forum}{41}%
\begin{footnote}[304]\sphinxAtStartFootnote
\sphinxnolinkurl{https://coffee.pmcaff.com/article/1329730610781312/pmcaff?utm\_source=forum}
%
\end{footnote}
\begin{enumerate}
\sphinxsetlistlabels{\arabic}{enumi}{enumii}{}{.}%
\item {} 
首先是 AI
产品的用户体验如何测量?与传统的衡量方法有什么区别?设定了哪些衡量指标?

\item {} 
根据什么标准来设定的这些指标?

\item {} 
哪些指标与 AI 相关,哪些不相关?

\item {} 
应用了 AI 技术的指标,产品指标是如何分解到 AI 技术指标的?

\item {} 
AI
产品经理如何完成这些指标?(进一步提问技术细节,考察产品经理对数据和技术的理解)

\end{enumerate}


\paragraph{红线 42\sphinxfootnotemark[305]}
\label{\detokenize{chapter_introduction/AI_PM:id36}}%
\begin{footnotetext}[305]\sphinxAtStartFootnote
\sphinxnolinkurl{http://www.xmamiga.com/3573/}
%
\end{footnotetext}\ignorespaces 

\subparagraph{安全}
\label{\detokenize{chapter_introduction/AI_PM:id37}}
人工智能产品认为可控;人工智能产品不会影响公共安全。


\subparagraph{隐私}
\label{\detokenize{chapter_introduction/AI_PM:id38}}
出台的法律法规,比如2018年生效的欧盟《通用数据保护条例》(GDPR),2020年1月生效的《2018年加州消费者隐私法案》(CCPA),这些立法正给商业运营和人工智能带来巨大影响。

2020年7月,全国人大常委会法制工作委员会在中国人大网公布了《中华人民共和国数据安全法(草案)》,并向社会大众征求意见,这会进一步规范行业内的行为和竞争。
\sphinxhref{https://www.productplan.com/ai-product-management/}{44}%
\begin{footnote}[306]\sphinxAtStartFootnote
\sphinxnolinkurl{https://www.productplan.com/ai-product-management/}
%
\end{footnote}

人工智能产品经理至少要评估一下四项:
\begin{enumerate}
\sphinxsetlistlabels{\arabic}{enumi}{enumii}{}{.}%
\item {} 
评估所有产品流程中涉及用户权利的风险。

\item {} 
评估产品在设计或运行过程中的系统描述。

\item {} 
基于产品设计或运行的目的,评估过程是否是必要的。

\item {} 
针对识别出的风险,给出针对风险的管理措施。

\end{enumerate}

减少对训练数据量的需求:
\begin{itemize}
\item {} 
生成对抗网络(GAN):通过轮流训练判别器和生成器,令其互相对抗,从复杂概率分布中取样,生成文字、图片、语音等。

\item {} 
联合学习(Federal
Learning):部分训练过程放到用户手机,将模型传回服务器,不涉及用户敏感数据。

\item {} 
迁移学习(Transfer
Learning):把一个场景学习到的模型举一反三迁移到类似的场景中的方法。

\end{itemize}

在不减少数据的基础上保护隐私:
\begin{itemize}
\item {} 
差分隐私技术(Different
Privacy):在数据库检索时,加入满足某种分布的噪声,使查询结果随机化。

\item {} 
同态加密技术(Homomorphic
Encryption):在密文上进行计算,生成加密结果,解密后的结果与对明文进行相同操作产生的结果一致。核心在于,支持在加密的数据上进行查询操作,解决数据委托给第三方如云计算公司时的安全问题。

\item {} 
提高算法可解释性,避免黑盒子事件的发生。

\end{itemize}


\subparagraph{道德}
\label{\detokenize{chapter_introduction/AI_PM:id39}}\begin{enumerate}
\sphinxsetlistlabels{\arabic}{enumi}{enumii}{}{.}%
\item {} 
这是需要解决的问题吗?这个解决方案怎么可能被滥用?

\item {} 
人工智能产品算法的“可解释性差”、“不透明”,使得一旦发生伦理道德事故无法评判。

\item {} 
人工智能代替人履行社会职能的时候,产品的“不可预见性”有可能导致伦理道德争议。

\item {} 
人工智能产品的道德地位值得思考。

\end{enumerate}

\sphinxurl{https://learning.oreilly.com/library/view/ethics-and-data/9781492043898/}


\paragraph{高风险}
\label{\detokenize{chapter_introduction/AI_PM:id40}}
基于深度学习的产品很难(甚至不可能)开发;这是一个典型的“高回报与高风险”的情况,在这种情况下,计算投资回报本来就很困难。


\paragraph{优秀的AI PM 32\sphinxfootnotemark[307]}
\label{\detokenize{chapter_introduction/AI_PM:ai-pm-32}}%
\begin{footnotetext}[307]\sphinxAtStartFootnote
\sphinxnolinkurl{https://weread.qq.com/web/reader/0c032c9071dbddbc0c06459kb6d32b90216b6d767d2f0dc}
%
\end{footnotetext}\ignorespaces 
人工智能产品经理的工作是分析出有价值的商业场景,并对其进行评估,评估其可行性、必要性、社会及商业价值、道德及法律框架。除分析有价值的商业场景外,人工智能产品经理还需要评估技术的可行性,评估技术能够达到的最优度,并根据内外部资源评估出产品价值与技术实现的平衡点,以及商业价值与技术成本之间的平衡点。最后,人工智能产品经理应能够通过交互体系设计出完整的产品。如果一个人能够做到以上几点,那么他会是一个非常优秀的人工智能产品经理。

思考的问题:\sphinxhref{https://www.shangyexinzhi.com/article/1682418.html}{33}%
\begin{footnote}[308]\sphinxAtStartFootnote
\sphinxnolinkurl{https://www.shangyexinzhi.com/article/1682418.html}
%
\end{footnote}
\begin{enumerate}
\sphinxsetlistlabels{\arabic}{enumi}{enumii}{}{.}%
\item {} 
算法的实现,我们需要多少训练数据?这些数据从哪里来?

\item {} 
模型是否需要更新?是需要定期更新呢还是实时更新?

\item {} 
如果更新,如何更新?在线还是离线?

\item {} 
推理能力是部署在终端上还是云端上?部署在终端和云端带来的推理时延能否接受?

\item {} 
终端计算和存储能力是否有限制?是否需要对模型进行压缩?

\item {} 
识别结果和操作日志如何保存并用于后续的算法优化?

\end{enumerate}

成功的人工智能产品管理就是要发现正确的数据,然后弄清楚如何利用这些数据来设计一款创新的产品,让顾客满意,并让他们继续光顾。\sphinxhref{https://www.productplan.com/ai-product-management/}{44}%
\begin{footnote}[309]\sphinxAtStartFootnote
\sphinxnolinkurl{https://www.productplan.com/ai-product-management/}
%
\end{footnote}


\subparagraph{最好的人工智能产品 22\sphinxfootnotemark[310]}
\label{\detokenize{chapter_introduction/AI_PM:id41}}%
\begin{footnotetext}[310]\sphinxAtStartFootnote
\sphinxnolinkurl{https://www.oreilly.com/radar/practical-skills-for-the-ai-product-manager/}
%
\end{footnotetext}\ignorespaces 
在最好的人工智能产品中,用户无法分辨底层模型如何影响他们的体验。他们既不知道也不关心应用程序中是否存在人工智能。以Stitch
Fix为例,它使用了多种算法方法来提供定制风格的建议。当Stitch
Fix用户与人工智能产品交互时,他们会与预测和推荐引擎交互。他们在体验过程中与之互动的信息是一种人工智能产品——但他们既不知道,也不关心,他们所看到的一切背后是人工智能。如果算法做出了完美的预测,但用户无法想象佩戴所展示的物品,该产品仍然是一个失败的产品。在现实中,ML模型远非完美,因此更有必要确定用户体验。


\subparagraph{失败的人工智能产品 23\sphinxfootnotemark[311]}
\label{\detokenize{chapter_introduction/AI_PM:id42}}%
\begin{footnotetext}[311]\sphinxAtStartFootnote
\sphinxnolinkurl{https://neal-lathia.medium.com/machine-learning-for-product-managers-ba9cf8724e57}
%
\end{footnotetext}\ignorespaces \begin{enumerate}
\sphinxsetlistlabels{\arabic}{enumi}{enumii}{}{.}%
\item {} 
由于ML本质上是通过例子来训练算法,产品失败的方式呈现出各种各样的新维度(这里的图片是微软的Tay机器人,它在被放到网上后变成了种族主义者)。

\item {} 
一个计算一个人怀孕概率的算法,他们用它来发送优惠券和折扣————这个系统向一个十几岁的女孩发送了婴儿服装的优惠券,这让她的父亲非常愤怒:为什么他的女儿会被婴儿服装的优惠券盯上?但不久之后,塔吉特收到了他的道歉:女孩确实怀孕了。

\item {} 
产品的使用或应用方式与产品设计师的设想不同。无论是基于敏感推断的目标客户,还是人们刺激机器人,还是用于训练人脸检测算法的有偏见的数据集

\end{enumerate}


\paragraph{转行}
\label{\detokenize{chapter_introduction/AI_PM:id43}}
“别拿你的业余去挑战别人的专业”——你只看到了别人成功的一面,却没有看到别人背后的努力付出和专业的积累。\sphinxhref{https://weread.qq.com/web/reader/46532b707210fc4f465d044k6ea321b021d6ea9ab1ba605}{35}%
\begin{footnote}[312]\sphinxAtStartFootnote
\sphinxnolinkurl{https://weread.qq.com/web/reader/46532b707210fc4f465d044k6ea321b021d6ea9ab1ba605}
%
\end{footnote}

转型AI产品经理的三部曲:\sphinxstylestrong{输入、输出、实施。}其核心思路是,先有大量的input(AI相关信息摄入),再有干货文章输出,进而“自己给自己背书”,以此争取一线AI公司的面试和offer机会。

从大学生到AI产品经理:\sphinxurl{https://mp.weixin.qq.com/s/tvPBiuNeyjZwvL94DWw3Dw}

TODO:https://medium.com/@liwdai/ai\sphinxhyphen{}pm\sphinxhyphen{}\%E5\%92\%8C\%E5\%B8\%B8\%E8\%A7\%84\sphinxhyphen{}pm\sphinxhyphen{}\%E7\%9A\%84\%E5\%8C\%BA\%E5\%88\%AB\%E6\%98\%AF\%E4\%BB\%80\%E4\%B9\%88\sphinxhyphen{}ecdcbe25ef40


\paragraph{相关问题}
\label{\detokenize{chapter_introduction/AI_PM:awesome-1}}\label{\detokenize{chapter_introduction/AI_PM:id44}}
由于ML产品的性质,我们想给ML科学家更多的时间和空间去探索和实验。但我们需要帮助团队专注于客户的需求和愿意为之付费的东西。这就是为什么我们需要确定探索的时间,并鼓励团队尽早并频繁地从头到尾测试模型。

例如,定义数据策略并不是数据科学家的责任。这是一个战略决策,甚至在建立ML产品之前,项目经理和高管就需要达成一致。您的公司在获得训练模型对抗竞争对手所需的专有数据方面是否具有可辩护的优势?还是行业巨头已经控制了大部分数据?

仅仅让工程和产品团队了解问题是不够的。你的公关或营销了解ML产品的性质吗?利弊呢?那么我们正在做的权衡呢?做出错误预测的后果和代价是什么?公司准备好回答所有这些问题了吗?
\sphinxhref{https://radiant-brushlands-42789.herokuapp.com/towardsdatascience.com/three-questions-every-ml-product-manager-must-answer-35d73127cd5d}{49}%
\begin{footnote}[313]\sphinxAtStartFootnote
\sphinxnolinkurl{https://radiant-brushlands-42789.herokuapp.com/towardsdatascience.com/three-questions-every-ml-product-manager-must-answer-35d73127cd5d}
%
\end{footnote}


\paragraph{Awesome}
\label{\detokenize{chapter_introduction/AI_PM:id45}}
\sphinxurl{http://www.changgpm.com/}

\begin{figure}[H]
\centering
\capstart

\noindent\sphinxincludegraphics{{AI_product_test}.png}
\caption{成为AI产品经理测试题\sphinxhref{https://time.geekbang.org/quiz/?act\_id=373\&exam\_id=1092\&source=1830990}{53}\sphinxfootnotemark[314]}\label{\detokenize{chapter_introduction/AI_PM:id66}}\end{figure}
%
\begin{footnotetext}[314]\sphinxAtStartFootnote
\sphinxnolinkurl{https://time.geekbang.org/quiz/?act\_id=373\&exam\_id=1092\&source=1830990}
%
\end{footnotetext}\ignorespaces 

\subsubsection{AI产品经理的完整能力矩阵 1\sphinxfootnotemark[315]}
\label{\detokenize{chapter_introduction/ability:ai-1}}\label{\detokenize{chapter_introduction/ability::doc}}%
\begin{footnotetext}[315]\sphinxAtStartFootnote
\sphinxnolinkurl{https://www.jianshu.com/p/fd466ed1bda6}
%
\end{footnotetext}\ignorespaces 

\paragraph{对比产品经理能力模型}
\label{\detokenize{chapter_introduction/ability:id1}}
\begin{figure}[H]
\centering
\capstart

\noindent\sphinxincludegraphics{{AIPM_vs_PM}.png}
\caption{AIPM\_vs\_PM}\label{\detokenize{chapter_introduction/ability:id19}}\end{figure}


\paragraph{门槛 11\sphinxfootnotemark[316]}
\label{\detokenize{chapter_introduction/ability:id2}}%
\begin{footnotetext}[316]\sphinxAtStartFootnote
\sphinxnolinkurl{https://weread.qq.com/web/reader/46532b707210fc4f465d044kc20321001cc20ad4d76f5ae}
%
\end{footnotetext}\ignorespaces 
凡是大家认为好的地方,一定会有条件,好地方一定是资源和位置都稀缺的,不可能随便让所有人都去,这叫入门门槛。

你要想进入大公司找到好工作,就要展示出你的能力,而不要过分在乎过往的经历。没有大学因为你的高中是重点高中就直接录取你,同样也不会有公司因为你过去在大公司工作就直接录用你。


\paragraph{总结}
\label{\detokenize{chapter_introduction/ability:id3}}
势+道+法+术+器+核心价值观来总结,“势”即君主的统治权力,“道”就是通用软技能,“法”就是团队管理方法,“术”就是具体的实现方法,“器”就是高效的实现工具,而核心价值观是作为AI产品经理的自我修养,是内功!


\paragraph{势}
\label{\detokenize{chapter_introduction/ability:id4}}
张小龙–“产品经理是站在上帝身边的人”。同样优秀的产品,发布实时机早了可能成先烈,发布晚了可能冲入红海没法突出重围。
\begin{enumerate}
\sphinxsetlistlabels{\arabic}{enumi}{enumii}{}{.}%
\item {} 
社会变革:改朝换代,颠覆式。

\item {} 
人口结构变革:每一代人都有其个性化需求,带来大量的增量市场。

\item {} 
技术变革:触屏技术的出现,直接将手机的从联系工具变成了信息获取和处理平台

\item {} 
资本市场环境改变:共享单车在红包大战时很难进入投资人的视野,产品需要资本为其成长加速,需要持续关注资本市场。

\item {} 
产品规模变革:豆瓣和知乎,小而美,大量用户涌入,原有的打分和评价系统不适新的场景

\end{enumerate}


\paragraph{道}
\label{\detokenize{chapter_introduction/ability:id5}}\begin{itemize}
\item {} 
用户/客户敏感即同理心(感知情绪、理解情绪、接纳欲望、真实去理想化、透过现象看本质、善于观察
\sphinxhref{https://weread.qq.com/web/reader/77532110721ea34a7751c9ak8e232ec02198e296a067180}{14}%
\begin{footnote}[317]\sphinxAtStartFootnote
\sphinxnolinkurl{https://weread.qq.com/web/reader/77532110721ea34a7751c9ak8e232ec02198e296a067180}
%
\end{footnote};妨碍同理心发展、进步的最大敌人就是——自以为是、固执\sphinxhref{https://blog.csdn.net/Dylan\_zhijing/article/details/108334435?spm=1001.2014.3001.5502}{23}%
\begin{footnote}[318]\sphinxAtStartFootnote
\sphinxnolinkurl{https://blog.csdn.net/Dylan\_zhijing/article/details/108334435?spm=1001.2014.3001.5502}
%
\end{footnote})

\item {} 
自我驱动(主动收集需求,并把它转化成产品需求
\sphinxhref{http://www.woshipm.com/zhichang/459131.html}{8}%
\begin{footnote}[319]\sphinxAtStartFootnote
\sphinxnolinkurl{http://www.woshipm.com/zhichang/459131.html}
%
\end{footnote})/学习能力(产品经理需要懂营销,懂技术、懂运营、懂设计
\sphinxhref{http://www.woshipm.com/zhichang/459131.html}{8}%
\begin{footnote}[320]\sphinxAtStartFootnote
\sphinxnolinkurl{http://www.woshipm.com/zhichang/459131.html}
%
\end{footnote})

\item {} 
沟通表达能力(一坨屎去撕逼:回响、star、输出感悟、用心
\sphinxhref{http://www.woshipm.com/pmd/4256992.html}{5}%
\begin{footnote}[321]\sphinxAtStartFootnote
\sphinxnolinkurl{http://www.woshipm.com/pmd/4256992.html}
%
\end{footnote}、Case包装
\sphinxhref{http://www.woshipm.com/pmd/3945349.html}{7}%
\begin{footnote}[322]\sphinxAtStartFootnote
\sphinxnolinkurl{http://www.woshipm.com/pmd/3945349.html}
%
\end{footnote};主动汇报、做好记录、尽早铺路
\sphinxhref{https://weread.qq.com/web/reader/77532110721ea34a7751c9ak6ea321b021d6ea9ab1ba605}{16}%
\begin{footnote}[323]\sphinxAtStartFootnote
\sphinxnolinkurl{https://weread.qq.com/web/reader/77532110721ea34a7751c9ak6ea321b021d6ea9ab1ba605}
%
\end{footnote};不要盛气凌人、主动去理解他人的困难,并且尽力帮忙解决;开会前有目的的腹稿、有结论和实际方案建议、有记录过程和结论\sphinxhref{https://www.zhihu.com/question/29342383/answer/46616997}{21}%
\begin{footnote}[324]\sphinxAtStartFootnote
\sphinxnolinkurl{https://www.zhihu.com/question/29342383/answer/46616997}
%
\end{footnote};参考书籍《关键对话》《非暴力沟通》\sphinxhref{http://www.woshipm.com/pmd/3024508.html}{22}%
\begin{footnote}[325]\sphinxAtStartFootnote
\sphinxnolinkurl{http://www.woshipm.com/pmd/3024508.html}
%
\end{footnote})

\item {} 
市场洞察能力(视野/前沿)(研究市场以了解用户需求、竞争状态、市场规模和盈利模式,发现创新或者改进产品的潜在机会;通过与用户和潜在用户进行沟通交流,明确符合该机会中的目标用户群体与特征;与直接面对用户/客户的一线同事/同行交流,获取、分析、评估用户的需求。)

\item {} 
高效执行能力
(决策力背后体现的是对过去信息的有效收集,对当下形势的清晰判断和对未来趋势的准确把握。好的产品经理总能在关键时刻,踩对点,比如我最敬佩的产品大师张一鸣,在今日头条成立初期,就定下了头条内容+算法的战略方针。
\sphinxhref{https://www.zhihu.com/question/31636227/answer/1162506705}{4}%
\begin{footnote}[326]\sphinxAtStartFootnote
\sphinxnolinkurl{https://www.zhihu.com/question/31636227/answer/1162506705}
%
\end{footnote})

\item {} 
判断趋势能力(趋势预判表现为细致的观察、信息的敏感度、成熟的推演等)

\item {} 
创新能力(“渐进式创新”者更偏向于沟通、管理,对某个领域有长期积累;“颠覆式创新”者更全能,更需要快速学习和独立思考的能力,因为他需要在复杂的市场里摸索出一个可行的商业模式。
\sphinxhref{https://weread.qq.com/web/reader/8d632bc07208ed1c8d697c4ka5732aa0226a5771bce9dc4}{13}%
\begin{footnote}[327]\sphinxAtStartFootnote
\sphinxnolinkurl{https://weread.qq.com/web/reader/8d632bc07208ed1c8d697c4ka5732aa0226a5771bce9dc4}
%
\end{footnote};更重要的能力不在于创新,而在于什么时候,为谁,做何种创新。\sphinxhref{https://www.zhihu.com/question/22613861/answer/118658853}{25}%
\begin{footnote}[328]\sphinxAtStartFootnote
\sphinxnolinkurl{https://www.zhihu.com/question/22613861/answer/118658853}
%
\end{footnote})

\item {} 
情绪控制能力(做产品可能比做其他工作更需要主人翁心态和强大的抗压能力。产品工作在大多数情况下都带有探索前进的性质,因此,当短时间内成果不明显时难免会遇到各种非议;产品经理需要强大的内心装下委屈,把压力化作前进路上的动力,专注产品,坚守初心。淡定的前提是你对与产品有关的所有事情都能有一种掌控感,即使遇到未知的风险,也能够泰然处之。
\sphinxhref{https://weread.qq.com/web/reader/77532110721ea34a7751c9ak341323f021e34173cb3824c}{15}%
\begin{footnote}[329]\sphinxAtStartFootnote
\sphinxnolinkurl{https://weread.qq.com/web/reader/77532110721ea34a7751c9ak341323f021e34173cb3824c}
%
\end{footnote}
可以不会,但你要有自信接受挑战,并且当前方一片黑暗的时候,如何去安全稳定的走过这片黑暗
\sphinxhref{https://medium.com/@liwdai/\%E8\%AF\%B7\%E8\%AE\%A4\%E7\%9C\%9F\%E9\%9D\%A2\%E8\%AF\%95-\%E4\%B9\%9F\%E8\%AF\%B7\%E8\%AE\%A4\%E7\%9C\%9F\%E5\%87\%86\%E5\%A4\%87\%E9\%9D\%A2\%E8\%AF\%95-36a2aa6344c1}{17}%
\begin{footnote}[330]\sphinxAtStartFootnote
\sphinxnolinkurl{https://medium.com/@liwdai/\%E8\%AF\%B7\%E8\%AE\%A4\%E7\%9C\%9F\%E9\%9D\%A2\%E8\%AF\%95-\%E4\%B9\%9F\%E8\%AF\%B7\%E8\%AE\%A4\%E7\%9C\%9F\%E5\%87\%86\%E5\%A4\%87\%E9\%9D\%A2\%E8\%AF\%95-36a2aa6344c1}
%
\end{footnote})

\item {} 
时间管理能力

\item {} 
多观点整合能力(阴阳历结合,只是为了春节等传统节日,那只显示那些就够了\sphinxhref{http://www.woshipm.com/pmd/3024508.html}{22}%
\begin{footnote}[331]\sphinxAtStartFootnote
\sphinxnolinkurl{http://www.woshipm.com/pmd/3024508.html}
%
\end{footnote})

\item {} 
资源协同能力(多资源协助)

\item {} 
影响他人能力(销售方案和销售激励
\sphinxhref{http://www.woshipm.com/pmd/3945349.html}{7}%
\begin{footnote}[332]\sphinxAtStartFootnote
\sphinxnolinkurl{http://www.woshipm.com/pmd/3945349.html}
%
\end{footnote})

\item {} 
团队建设能力(跨部门协作的能力,产品价值在公司内的传递,并协作宣传方案。
\sphinxhref{http://www.woshipm.com/pmd/3945349.html}{7}%
\begin{footnote}[333]\sphinxAtStartFootnote
\sphinxnolinkurl{http://www.woshipm.com/pmd/3945349.html}
%
\end{footnote})

\item {} 
外语能力

\item {} 
产品定义和原型设计能力(产品需求文档(PRD)来进行描述,形成包含交互设计(UX)、用户界面设计(UI)的具体方案:在方案形成过程中,通常会通过原型设计是把自己的想法或者团队讨论确定的方案以具象的形式呈现出来,便于团队成员的理解或者用于用户测试,最终达成共识,形成确定的方案。)

\item {} 
系统能力(将所有资源有效打包成产品、服务提供给消费者。
\sphinxhref{http://www.woshipm.com/pmd/3130419.html}{10}%
\begin{footnote}[334]\sphinxAtStartFootnote
\sphinxnolinkurl{http://www.woshipm.com/pmd/3130419.html}
%
\end{footnote}不止从自身,而是以行业和市场的高标准要求。\sphinxhref{https://zhuanlan.zhihu.com/p/24410557}{18}%
\begin{footnote}[335]\sphinxAtStartFootnote
\sphinxnolinkurl{https://zhuanlan.zhihu.com/p/24410557}
%
\end{footnote}目光长远,看这件事的半衰期\sphinxhref{https://xueqiu.com/6003295262/136559377}{19}%
\begin{footnote}[336]\sphinxAtStartFootnote
\sphinxnolinkurl{https://xueqiu.com/6003295262/136559377}
%
\end{footnote}
发展一定会手忙脚乱的失控。\sphinxhref{https://news.mbalib.com/story/248017}{20}%
\begin{footnote}[337]\sphinxAtStartFootnote
\sphinxnolinkurl{https://news.mbalib.com/story/248017}
%
\end{footnote})

\item {} 
思考能力(我的老师是小孩,他们会问:为什么男女厕所要分开?为什么纸边会剌我的手指头呀?为什么会流血?为什么地面这么硬硬的?\sphinxhref{https://medium.com/@liwdai/\%E8\%AF\%B7\%E8\%AE\%A4\%E7\%9C\%9F\%E9\%9D\%A2\%E8\%AF\%95-\%E4\%B9\%9F\%E8\%AF\%B7\%E8\%AE\%A4\%E7\%9C\%9F\%E5\%87\%86\%E5\%A4\%87\%E9\%9D\%A2\%E8\%AF\%95-36a2aa6344c1}{17}%
\begin{footnote}[338]\sphinxAtStartFootnote
\sphinxnolinkurl{https://medium.com/@liwdai/\%E8\%AF\%B7\%E8\%AE\%A4\%E7\%9C\%9F\%E9\%9D\%A2\%E8\%AF\%95-\%E4\%B9\%9F\%E8\%AF\%B7\%E8\%AE\%A4\%E7\%9C\%9F\%E5\%87\%86\%E5\%A4\%87\%E9\%9D\%A2\%E8\%AF\%95-36a2aa6344c1}
%
\end{footnote})

\item {} 
责任心(要考虑设计、开发未来几周,或者未来几个月要做哪些事情\sphinxhref{https://www.zhihu.com/question/21015379/answer/1365070268}{24}%
\begin{footnote}[339]\sphinxAtStartFootnote
\sphinxnolinkurl{https://www.zhihu.com/question/21015379/answer/1365070268}
%
\end{footnote};任何阻碍产品正常上线的事情都应该由你解决,任何能对产品产生帮助的事情都应该给予关心。\sphinxhref{https://www.zhihu.com/question/29342383/answer/46616997}{21}%
\begin{footnote}[340]\sphinxAtStartFootnote
\sphinxnolinkurl{https://www.zhihu.com/question/29342383/answer/46616997}
%
\end{footnote};为整个团队负责、要为公司业绩负责)

\end{itemize}


\paragraph{法6\sphinxfootnotemark[341]}
\label{\detokenize{chapter_introduction/ability:id6}}%
\begin{footnotetext}[341]\sphinxAtStartFootnote
\sphinxnolinkurl{http://www.woshipm.com/pmd/693904.html}
%
\end{footnotetext}\ignorespaces 
优秀的产品需要优秀的产品经理,我们来看看这些大公司都是如何创造产品文化的:

腾讯公司对于产品经理的要求是10/100/1000,即产品经理每个月必须做10个用户调查,关注100个用户博客,收集反馈1000个用户体验。

阿里巴巴会把新人产品经理扔到客服去处理三个月投诉,这样产品经理就非常清楚应该做什么了。

笔者认为产品经理都应该是终生学习者,对未知充满好奇,所以分享文化是必须,这既加强了产品的沟通和汇报能力,也能让团队形成探索氛围,促进大家大开脑洞。


\paragraph{术}
\label{\detokenize{chapter_introduction/ability:id7}}
\begin{figure}[H]
\centering
\capstart

\noindent\sphinxincludegraphics{{AI_knowledge_map}.jpg}
\caption{术:产品经理知识地图}\label{\detokenize{chapter_introduction/ability:id20}}\end{figure}

文本撰写、工具使用、沟通交流等能力像是产品经理的右手,创新思维、发散思维等能力像是产品经理的左手,左手的能力将决定一个产品经理能走多高。右手一般人经过一段时间的训练都能训练出来,而左手却很难,只有接触过足够多的人见过足够事,有着丰富的阅历也许才能培养出来。
\sphinxhref{http://www.woshipm.com/zhichang/315041.html}{9}%
\begin{footnote}[342]\sphinxAtStartFootnote
\sphinxnolinkurl{http://www.woshipm.com/zhichang/315041.html}
%
\end{footnote}


\paragraph{能力 12\sphinxfootnotemark[343]}
\label{\detokenize{chapter_introduction/ability:id8}}%
\begin{footnotetext}[343]\sphinxAtStartFootnote
\sphinxnolinkurl{https://weread.qq.com/web/reader/46532b707210fc4f465d044k02e32f0021b02e74f10ece8}
%
\end{footnotetext}\ignorespaces \begin{itemize}
\item {} 
广度:公司的规模小、人员少,分工不细,更好交流合作。全局视图:为了完成共同目标要做的模块,能够总结出每个模块的重点工作。

\item {} 
深度:有的小公司在特定的规模上是可以提供完整服务的,或者说有些问题不会爆发,可是如果换成大公司就不行了。对于想去大公司工作的产品经理来说,你在现在的工作中就要思考大公司的产品经理该怎么做。

\item {} 
高度:站在50层高楼,就能理解为什么拥堵、外沿的路况如何、整个片区的交通规划应该怎么调整。

\item {} 
速度:市场不会给你慢慢思考的时间。需要迅速放大规模。

\end{itemize}


\paragraph{对产品的理解}
\label{\detokenize{chapter_introduction/ability:id9}}\begin{itemize}
\item {} 
对公司定位的理解,跟老板和投资方有关

\item {} 
对用户定位的理解,跟公司的定位和市场的状况有关

\item {} 
对产品定位的理解,跟用户的定位和推出产品的初衷有关

\item {} 
对公司研发能力的理解,包括设计能力、开发能力和运营能力

\item {} 
对其他部门状况的理解,包括各部门在做的事情、大家进行的状态

\end{itemize}

总之,你做出的每个判断必须基于对产品多方面的理解,而不是对竞品的理解、对市场的理解这零散的因素。

只有这样,当设计师做出你不满意的稿子时你可以说“你的这个风格可能适合年轻人,但我们的目标用户是商务人士”,而不是说“你这个不够大气,没有
feel 啊”;当你要求工程师改一个看起来不重要的 BUG
时可以说“后续运营部门计划有好几次大型活动,到时流量会瞬间暴涨,你这个
BUG 会放大 10
倍,所以很危险”,而不是说“重要不重要不是你工程师定的,是产品经理定的,你改就行啦”;当你跟老板讨论要不要加一个功能时可以说“我觉得我们这样的功能是一个重要的补充,跟下个版本要做的事情不谋而合”,而不是“竞品做了咱们不能落下啊,管它有没有用”。\sphinxhref{https://www.zhihu.com/question/29342383/answer/46616997}{21}%
\begin{footnote}[344]\sphinxAtStartFootnote
\sphinxnolinkurl{https://www.zhihu.com/question/29342383/answer/46616997}
%
\end{footnote}


\subparagraph{产品生命周期}
\label{\detokenize{chapter_introduction/ability:id10}}

\subparagraph{产品战略与产品创新阶段}
\label{\detokenize{chapter_introduction/ability:id11}}\begin{itemize}
\item {} 
市场分析:PEST分析、APPEALS方法、战略定位分析(SPAN)、麦肯锡市场细分八法;

\item {} 
竞争力分析:波士顿矩阵(BCG矩阵)、GE分析、麦肯锡三层面理论等;

\item {} 
机会判断;竞品分析画布、MRD撰写;

\item {} 
用户研究:A/B test、问卷调研、可用性测试、干系人地图、用户洋葱模型等

\end{itemize}


\subparagraph{产品规划与商业模式阶段}
\label{\detokenize{chapter_introduction/ability:id12}}\begin{itemize}
\item {} 
需求分析:马斯诺需求层次理论、3W2H方法、5WHY分析法、PSPS模型等

\item {} 
商业分析:SWOT分析、波特五力分析、精益商业画布、BRD文档;

\item {} 
优先级评估:火车模型、Kano模型、COD评分表方法、四象限方法、MoSCoW方法等;

\item {} 
数据分析:数据获取、SQL数据库、Python、统计学、数据分析核心模块、可视化、报告撰写。

\item {} 
产品规划:产品架构图、产品路线图、计划扑克工作量评估法、六西格玛、TRIZ、盈利模式设计、MVP定义、突出重点(避免认知失调);

\end{itemize}


\subparagraph{产品运营与营销阶段}
\label{\detokenize{chapter_introduction/ability:id13}}
产品运营:AARRR产品运营模型、OGSM工具、运营数据分析、灰度测试、同期群分析、网络推广优化、市场维护等;

持续了解和收集基本数据,追踪产品投放到市场上的效果和反馈,以便不断迭代优化。
\sphinxhref{https://www.zhihu.com/question/31636227}{2}%
\begin{footnote}[345]\sphinxAtStartFootnote
\sphinxnolinkurl{https://www.zhihu.com/question/31636227}
%
\end{footnote} 工具:Google
Analytics、百度统计、TalkingData、友盟、GrowingIO 等等。

产品营销:FABE法则、电梯演讲、产品路演等;


\subparagraph{产品生命周期管理}
\label{\detokenize{chapter_introduction/ability:id14}}\begin{itemize}
\item {} 
产品方法框架:IPD、门径管理流程、抄超钞等;

\item {} 
产品宏观思维:波士顿矩阵、多产品组合战略等;

\item {} 
团队建设:团队文化定义、组织架构建设等;

\end{itemize}


\subparagraph{AI产品方法}
\label{\detokenize{chapter_introduction/ability:ai}}\begin{itemize}
\item {} 
算法

\item {} 
算力

\item {} 
数据

\item {} 
硬件

\item {} 
业务

\end{itemize}


\paragraph{器}
\label{\detokenize{chapter_introduction/ability:id15}}\begin{itemize}
\item {} 
通用办公工具:office三件套、Xmind类思维导图(\sphinxurl{https://mm.edrawsoft.cn/})、think\sphinxhyphen{}cell麦客–信息收集等;

\item {} 
产品流程设计:Visio、Processon、亿图等;

\item {} 
产品原型设计:Axure、Sketch、墨刀等;

\item {} 
数据分析工具:SQL、python、powerBI、SPSS、百度指数、talkingdata、ASO100、艾瑞指数、微博数据中心、\sphinxurl{https://www.dydata.io}/等;

\item {} 
项目管理工具:Teambition、Trello–任务管理、Demoo\sphinxhyphen{}原型展示、石墨文档、禅道–项目管理、leangoo、CORNERSTONE等;

\item {} 
AI工具:Python、Tensorflow、PyTorch、MXNet等

\item {} 
主要文档:MRD、BRD、PRD;

\end{itemize}


\paragraph{核心价值观}
\label{\detokenize{chapter_introduction/ability:id16}}
这里我要援引经典的产品设计五要素图来解释AI产品经理的核心价值观。

\begin{figure}[H]
\centering
\capstart

\noindent\sphinxincludegraphics{{产品设计五要素}.png}
\caption{产品设计五要素}\label{\detokenize{chapter_introduction/ability:id21}}\end{figure}


\subparagraph{初心}
\label{\detokenize{chapter_introduction/ability:id17}}
作为AI产品经理要时刻记住自己做产品的初心,也就是最底层的战略层,一方面是这个产品的初衷是什么,想清楚了它才能走的长远,如果只是未来表层和框架的浅显需求而做设计,那这个产品设计是站不住脚的,只有从战略层进行思考,产品整体设计才经得起推敲,那时即使在部分表层有缺陷,也瑕不掩瑜,这就好像哲学终的“本我”。


\subparagraph{自我定位}
\label{\detokenize{chapter_introduction/ability:id18}}
AI产品经理的自我定位也非常重要,在我的工作经历中,看过很多产品经理,因为主观或客观的产品立场不坚定,有时候把自己做成了商务、解决方案,有时候在一些技术架构方面与研发团队钻牛角尖,但往往丢失了一个产品经理的初心,最终产品走向也不是很理想。作为产品经理,我们需要把握的是整个产品的生命线,而很多细枝末节的事情,有细分领域更专业的人去做。


\subsubsection{产品经理的一天}
\label{\detokenize{chapter_introduction/1Day:id1}}\label{\detokenize{chapter_introduction/1Day::doc}}
结合产品经理基本工作流程来看这个问题,会更容易理解一些。虽然具体的产品开发工作不用产品经理做,但产品经理也绝对做不了甩手掌柜。在有产品开发时,他需要时刻关注产品的进度,进行问题确认,必要的时候协调资源;在没有产品开发时,他需要进行规划,同时还要关注市场及竞品的变化,以能够及时洞察产品的发展趋势。

把以上的这段文字转换成场景,基本上产品经理的一天就能呈现在我们面前。


\paragraph{高薪的真相 3\sphinxfootnotemark[346]}
\label{\detokenize{chapter_introduction/1Day:id2}}%
\begin{footnotetext}[346]\sphinxAtStartFootnote
\sphinxnolinkurl{https://weread.qq.com/web/reader/46532b707210fc4f465d044kc9f326d018c9f0f895fb5e4}
%
\end{footnotetext}\ignorespaces 
互联网公司给出的产品经理的工资大多是20 000元以上。

产品经理的工作是非常辛苦的,他们经常要加班,甚至在很多竞争激烈的行业,产品经理的工作实行“996”工作制,这样的工作节奏你能接受吗?如果把工资按照每小时来计算

这些产品经理需要24小时开机,随传随到,每天有干不完的工作,回不完的微信消息,经常需要到凌晨才能睡觉,早上8点左右还要起床,如果工作在北京5环以内,住在通州,那么上班的通勤时间可能要一个半小时,这样的生活状态是你想要的吗?这样的高工资是你追求的吗?


\paragraph{场景}
\label{\detokenize{chapter_introduction/1Day:id3}}
早上在上班通勤的路上,产品经理可能会打开新闻客户端,关注自己感兴趣或与工作相关的内容,必要的话会把重要信息或链接记在备忘录里。

到公司以后,规划自己一天的工作,打开电脑首先查看一下邮箱,邮箱里有四五封邮件,其中两封邮件是测试发的bug信息,需要沟通确认;有一封邮件是协同部门发来的,内容主要是得到了一些用户反馈,需要满足新的需求,需要进行评估;还有一封会议通知邮件,下午三点要开某产品需求沟通会议。

然后产品经理的一天也就围绕这几封邮件开始了。

开产品经理组内晨会:每天的产品经理组内晨会都是产品经理跟领导沟通的好机会,如果产品经理在工作中遇到了自己解决不了的问题,要学会寻求领导的帮助。

上午他可能会先去和测试沟通确认一下两个bug该如何修改,沟通的过程中又发现了新的问题,所以后来和测试的沟通就变成了和测试、开发、前端等同事的沟通。问题解决了,时间也过去了半个多小时。

解决了测试bug的问题,产品经理需要好好想想协同部门提的需求。对产品经理而言,需要很慎重地对待需求,有的需求不一定要满足,而有的则必须快速响应。经过初步分析,这个需求是需要满足的,但如何做还需要和领导沟通一下。因此,产品经理就去找领导沟通,沟通后最终形成了一个初步的方案,产品经理以此回复了邮件。而此时已经上班两个多小时了。

\begin{figure}[H]
\centering
\capstart

\noindent\sphinxincludegraphics{{email_structure}.png}
\caption{项目邮件基本结构\sphinxhref{https://g.yuque.com/zhongguodianxinyanjiuyuan/bgso10/wbglgs}{5}\sphinxfootnotemark[347]}\label{\detokenize{chapter_introduction/1Day:id8}}\end{figure}
%
\begin{footnotetext}[347]\sphinxAtStartFootnote
\sphinxnolinkurl{https://g.yuque.com/zhongguodianxinyanjiuyuan/bgso10/wbglgs}
%
\end{footnotetext}\ignorespaces 
产品经理刚发完邮件,着手开始准备下午开会资料时,电话响了。电话是客服同事打来的,说用户使用出现了问题,需要马上解决。产品经理只好先暂时放下手里的活,去解决用户问题。用户问题解决了,时间基本上也就到中午了。

下午的工作内容相对比较整,简单说就是准备开会、开会。不过,在这个过程中,还是时不时需要跟项目成员确认信息;收到其他同事的微信或电话。下午的会开得还算成功,不过有些需求的细节还是需要调整。会议开完,产品经理就开始着手修改工作了。等修改完了,差不多也就到了下班的时间。

其实上面说了那么多,总结起来讲产品经理的一天就是由\sphinxstylestrong{洞察趋势、内部沟通、整理信息、产品思考}四大部分组成,其中沟通会占大部分时间,形式有面对面、电话、会议等等。所以沟通能力对产品经理来讲,尤为重要。


\subparagraph{小结}
\label{\detokenize{chapter_introduction/1Day:id4}}
关于产品经理工作流程,我们可以归纳为想、写、说、做、改五个字。任何一个阶段,都由人、物、信息三种元素组成,产品经理的工作也都以此展开。


\subparagraph{不同场景的产品经理的一天}
\label{\detokenize{chapter_introduction/1Day:id5}}

\subparagraph{BAT 公司(腾讯商户)产品经理}
\label{\detokenize{chapter_introduction/1Day:bat}}\begin{enumerate}
\sphinxsetlistlabels{\arabic}{enumi}{enumii}{}{.}%
\item {} 
负责设计及规划商户管理及运营管理的系统需求

\item {} 
结合商户管理的实际情况,梳理平台管理规则并进行持续优化

\item {} 
协助业务管理、风控管理等需求,制定合理产品及业务流程

\item {} 
协调资源,规划产品解决方案并持续运营,推进产品方案落地

\item {} 
制定合理产品路线图,对所负责的产品进行中长期规划

\end{enumerate}

\begin{figure}[H]
\centering
\capstart

\noindent\sphinxincludegraphics{{tencent_business_PM}.jpg}
\caption{BAT 公司(腾讯商户)产品经理的一天}\label{\detokenize{chapter_introduction/1Day:id9}}\end{figure}


\subparagraph{咨询公司产品经理}
\label{\detokenize{chapter_introduction/1Day:id6}}\begin{enumerate}
\sphinxsetlistlabels{\arabic}{enumi}{enumii}{}{.}%
\item {} 
研究并理解客户的战略、商业模式,挖掘并揭示客户的痛点和诉求

\item {} 
帮助客户识别商业机会并建议实施方案

\item {} 
引导需求探寻和创新思考工作坊,产出客户认可的解决方案

\item {} 
创建并清楚展示方案蓝图,确保客户和交付团队理解并达成共识

\item {} 
定义关键目标、成功要素,识别风险、挑战、依赖和约束

\item {} 
有效引导和促进 Product
Owner、客户出资人、行业专家、技术团队、最终用户间的沟通

\item {} 
和协作,保证产品从概念、到原型、到上线及运营的端到端交付

\end{enumerate}

\begin{figure}[H]
\centering
\capstart

\noindent\sphinxincludegraphics{{Consult_PM}.jpg}
\caption{咨询公司产品经理的一天}\label{\detokenize{chapter_introduction/1Day:id10}}\end{figure}


\subparagraph{某未融资创业公司产品经理}
\label{\detokenize{chapter_introduction/1Day:id7}}\begin{itemize}
\item {} 
领导创新产品的构思、技术开发和发布

\item {} 
通过达成共识,在整个公司内建立共同的愿景和产品发展方向

\item {} 
通过与工程师和设计师团队合作,共同推动产品开发

\item {} 
将可用性研究和市场分析整合到产品需求中,以进一步加强用户粘性、
用户参与度和用户满意度

\item {} 
定义、分析及监控产品各个成功的指标

\item {} 
了解公司的战略,竞争地位,帮助产品进一步发展并引领业界发展方向

\item {} 
在不断变化的环境中最大限度地提高产品效率,领导团队高效执行有创造性的解决方案

\end{itemize}

\begin{figure}[H]
\centering
\capstart

\noindent\sphinxincludegraphics{{startup_PM}.jpg}
\caption{某未融资创业公司产品经理的一天}\label{\detokenize{chapter_introduction/1Day:id11}}\end{figure}


\subsubsection{职业发展路径}
\label{\detokenize{chapter_introduction/career_path:id1}}\label{\detokenize{chapter_introduction/career_path::doc}}
职业生涯规划是指人和工作两个部分的结合,首先要对这个人和这个公司进行主观、客观条件的评估分析,然后再根据这个人的兴趣爱好、能力特点进行利弊分析,结合当前社会发展,以及未来几年的趋势,确定一个最佳的职业奋斗目标,并为实现这个目标做出行之有效的安排。职业生涯规划最终应该落地到目标和行之有效的安排。


\paragraph{择业}
\label{\detokenize{chapter_introduction/career_path:id2}}
每一次择业都是一种选择。产品经理应该慎重对待自己的每一次选择,每一次赛道的变化都可能吞噬之前的积累,每一次公司的变更都可能让自己退步。


\paragraph{产品经理}
\label{\detokenize{chapter_introduction/career_path:id3}}
\sphinxstylestrong{懂用户、会权衡}

因为产品经理入门确实没有门槛,但一路上却有很多看不见的门槛;你想到的位置不同,门槛也不同。\sphinxhref{https://weread.qq.com/web/reader/77532110721ea34a7751c9ak1c3321802231c383cd30bb3}{10}%
\begin{footnote}[348]\sphinxAtStartFootnote
\sphinxnolinkurl{https://weread.qq.com/web/reader/77532110721ea34a7751c9ak1c3321802231c383cd30bb3}
%
\end{footnote}


\paragraph{什么样的人适合做产品经理?}
\label{\detokenize{chapter_introduction/career_path:id4}}\begin{enumerate}
\sphinxsetlistlabels{\arabic}{enumi}{enumii}{}{.}%
\item {} 
喜欢体验各种新鲜事情,遇到不喜欢的设计有自己独特的想法;

\item {} 
逻辑清晰,能用简短的语言描述复杂的事物;

\item {} 
喜欢与人沟通,有很强的的同理心去揣测身边人的心理活动;(销售自己的理念,销售自己对这个产品的想法)

\item {} 
对新奇的事物有新鲜感,喜欢追根刨底的问问题;

\item {} 
喜欢发起辩论,用自己的思考和逻辑说服别人;

\item {} 
注重细节体验,能从细节中发现问题,并对细节瑕疵不能忍。

\end{enumerate}


\paragraph{三线 16\sphinxfootnotemark[349]}
\label{\detokenize{chapter_introduction/career_path:id5}}%
\begin{footnotetext}[349]\sphinxAtStartFootnote
\sphinxnolinkurl{https://www.zhihu.com/question/20791021/answer/640398686}
%
\end{footnotetext}\ignorespaces \begin{enumerate}
\sphinxsetlistlabels{\arabic}{enumi}{enumii}{}{.}%
\item {} 
技能线

\end{enumerate}
\begin{itemize}
\item {} 
月收入:10\sphinxhyphen{}20k

\item {} 
阶段:1\sphinxhyphen{}2年

\item {} 
关键词:掌握PM的全套方法论

\item {} 
核心竞争力:掌握产品经理工作的完整流程,并对每个环节有一定的实操,能产出出色的成果;

\end{itemize}
\begin{enumerate}
\sphinxsetlistlabels{\arabic}{enumi}{enumii}{}{.}%
\setcounter{enumi}{1}
\item {} 
业务线

\end{enumerate}
\begin{itemize}
\item {} 
月收入:20\sphinxhyphen{}30K

\item {} 
阶段:3\sphinxhyphen{}4年

\item {} 
关键词:了解业务

\item {} 
核心竞争力:对于公司所从事的业务有深刻的理解,对于供给双方的需求和服务有深入的认识,能够从业务的场景出发进行产品设计;

\end{itemize}
\begin{enumerate}
\sphinxsetlistlabels{\arabic}{enumi}{enumii}{}{.}%
\setcounter{enumi}{2}
\item {} 
商业线

\end{enumerate}
\begin{itemize}
\item {} 
月收入:30\sphinxhyphen{}40+K

\item {} 
阶段:5\sphinxhyphen{}6年

\item {} 
关键词:产品经理思维,判勳力

\item {} 
核心竞争力:构建出自己的产品思维,具备出色的判断力,形成自己的商业分析逻辑,能够帮公司赚钱(或具备这个意识),有获取新用户的意识;

\end{itemize}

\begin{figure}[H]
\centering
\capstart

\noindent\sphinxincludegraphics{{PM_career_path}.png}
\caption{组织架构与晋升发展}\label{\detokenize{chapter_introduction/career_path:id27}}\end{figure}

\begin{figure}[H]
\centering
\capstart

\noindent\sphinxincludegraphics{{PM_upgrade}.png}
\caption{PM的升级蜕变\sphinxhref{https://mp.weixin.qq.com/s?\_\_biz=MjM5MzE3MDQ3Mw==\&mid=2650404998\&idx=3\&sn=e4bf27058ac6a697bfb1ae3cbb319e14\&chksm=be964dc089e1c4d613d4dcf763e01fbc65dee8b08136e34ebf62c1d22cbc7d83c58502416f2a\&scene=21\#wechat\_redirect}{18}\sphinxfootnotemark[350]}\label{\detokenize{chapter_introduction/career_path:id28}}\end{figure}
%
\begin{footnotetext}[350]\sphinxAtStartFootnote
\sphinxnolinkurl{https://mp.weixin.qq.com/s?\_\_biz=MjM5MzE3MDQ3Mw==\&mid=2650404998\&idx=3\&sn=e4bf27058ac6a697bfb1ae3cbb319e14\&chksm=be964dc089e1c4d613d4dcf763e01fbc65dee8b08136e34ebf62c1d22cbc7d83c58502416f2a\&scene=21\#wechat\_redirect}
%
\end{footnotetext}\ignorespaces 
\begin{figure}[H]
\centering
\capstart

\noindent\sphinxincludegraphics{{PM_ability_upgrade}.png}
\caption{PM技能树\sphinxhref{https://www.zhihu.com/question/323588594/answer/677650489}{20}\sphinxfootnotemark[351]}\label{\detokenize{chapter_introduction/career_path:id29}}\end{figure}
%
\begin{footnotetext}[351]\sphinxAtStartFootnote
\sphinxnolinkurl{https://www.zhihu.com/question/323588594/answer/677650489}
%
\end{footnotetext}\ignorespaces 

\paragraph{衡量价值}
\label{\detokenize{chapter_introduction/career_path:id6}}
产品经理在市场上值多少钱 = 基石能力 x 专业知识 x 行业知识 x 资历履历 x
圈子\sphinxhref{https://www.zhihu.com/question/20791021/answer/640398686}{16}%
\begin{footnote}[352]\sphinxAtStartFootnote
\sphinxnolinkurl{https://www.zhihu.com/question/20791021/answer/640398686}
%
\end{footnote}


\paragraph{职级}
\label{\detokenize{chapter_introduction/career_path:id7}}

\subparagraph{影响因素}
\label{\detokenize{chapter_introduction/career_path:id8}}
产品经理职级,其背后影响因素众多,如:\sphinxhref{https://www.zhihu.com/question/19565317}{21}%
\begin{footnote}[353]\sphinxAtStartFootnote
\sphinxnolinkurl{https://www.zhihu.com/question/19565317}
%
\end{footnote}
\begin{itemize}
\item {} 
能力:专业能力,业务能力,管理能力。 • 素质:品性,潜力。 •

\end{itemize}

岗位:业务规模,团队规模,边缘程度。 • 绩效:战功,绩优,业务增长。 •
资历:领域经验,工龄,履历背景。 •
博弈:稀缺性,谈判能力,跳槽升级,高薪倒推,离职挽留,让位补偿。 •
评审问题:随机性,临场表现,评委水平,作弊,领导强推。


\subparagraph{实习生}
\label{\detokenize{chapter_introduction/career_path:id9}}\label{\detokenize{chapter_introduction/career_path:id10}}
理解产品可以从理解基础技术开始。

这里提到的基础技术包括了客户端技术分类,例如Android、iOS、H5或者微信小程序:它们各自的技术特点是什么。比如为什么开发Android和iOS应用的是两个不同的技术职能,而开发H5的是一个技术职能。

另外,产品实习生需要对产品底层的技术通信原理进行理解。(比例大约是20比1,来负责这些正式员工不愿意干的脏活杂活)

比如当我们使用客户端产品发送一条消息时,这条消息经过了哪些环节后被另一个客户端收到——这里就涉及了什么是服务端,客户端和服务端的主要职能和通信机制是什么。先从大局观上对互联网产品的技术框架有一个基本认知。

建立互联网产品技术认知,划分清楚技术职能,了解各技术的特点和应用场景,就能胜任最基本的产品工作了。

产品助理的本职: \sphinxhref{http://www.woshipm.com/pmd/415296.html}{4}%
\begin{footnote}[354]\sphinxAtStartFootnote
\sphinxnolinkurl{http://www.woshipm.com/pmd/415296.html}
%
\end{footnote}
\begin{enumerate}
\sphinxsetlistlabels{\arabic}{enumi}{enumii}{}{.}%
\item {} 
需求文档:将需求整理成规范的文档

\item {} 
\sphinxstylestrong{框线图的完善}:将草图,概念图,补充成为整套的框线文件

\item {} 
材料收集:收集市场,运营,案例等产品经理所需要的一些材料

\item {} 
基础的用户调研:收集用户的反馈信息,观察用户基于产品,基于市场产生的言行,并及时向产品经理反馈

\item {} 
反馈技巧:助理必须要学会的是如何向产品经理良好沟通反馈,

\end{enumerate}

晋升能力则是以下四个技能
\begin{enumerate}
\sphinxsetlistlabels{\arabic}{enumi}{enumii}{}{.}%
\item {} 
团队沟通:与设计,研发,QA,运营等团队中的其他角色能进行针对项目的良好沟通,形成讨论氛围

\item {} 
进度跟进:能清楚掌握到项目的执行进度,已经完成情况,剩余时间,延期风险评估。

\item {} 
产品发布:着手了解各个渠道,平台的产品发布规则,准备相应的发布资料

\item {} 
小模块的独立策划:这个环节需要的不仅仅是创新思维,更重要的是在一个小模块的环境下,去探索产品思维,要知道产品思维是个很全很宽的面性思维

\end{enumerate}

阶段建议:\sphinxhref{https://www.iamxiarui.com/?p=1369}{6}%
\begin{footnote}[355]\sphinxAtStartFootnote
\sphinxnolinkurl{https://www.iamxiarui.com/?p=1369}
%
\end{footnote}
\begin{enumerate}
\sphinxsetlistlabels{\arabic}{enumi}{enumii}{}{.}%
\item {} 
建立自己的知识库/资源库/模板库

\item {} 
拥有自己的工作/设计/文档规范

\item {} 
按照最高标准要求自己

\item {} 
密集归纳法学习,在效率降低时开设其他学习曲线

\item {} 
升维打击算法思考问题

\end{enumerate}

\sphinxstylestrong{完善自身的知识体系,优化现存问题的体验。}


\subparagraph{产品经理}
\label{\detokenize{chapter_introduction/career_path:id11}}
过年的时候,大家会在微信收发红包,微信红包就是一个具体的功能模块,如果你在微信做产品经理,那或许就要从负责一个功能模块开始历练了。

要能建立完整的技术基础概念认知,能从技术角度对产品方案进行初步评估和判断。

面试考核的重点:
\begin{itemize}
\item {} 
执行力:初级产品经理最重要的就是执行力,因为大部分的情况下,产品的大方向不由他控制,只负责局部的数据,用户需求往往比较明显,所以对于需求的把握能力要求并不高,能深度的做好用户调研和反馈,快速的迭代并提升数据就可以了,而以上的这些,就要求应聘者有强大的内驱力,可以有力的推动项目内成员达成目标。

\item {} 
综合能力:以逻辑能力、沟通表达能力为主,逻辑能力是PM安家立命之本,对于初级产品经理来说,能不能理清楚\sphinxstylestrong{功能模块、架构和整个产品的关系}非常重要,除此之外,功能的设计和迭代的节奏,也非常考验产品经理的逻辑能力,一个页面会遇到几种使用场景?不同场景之间的关系是什么?如何让一个页面同时满足多种入口和多种需求?没有优秀的逻辑,处理这些问题的时候,就会有纰漏。

\item {} 
交互设计:国内很多的一线互联网企业都有专业的交互设计师(更多地考虑用户(目标、场景)),相处过很多tx的PM,都会在入司后问到交互设计师在哪?但个人认为,PM应该兼顾交互设计师的工作,特别是初创型企业,大部分都没有专职的交互设计师。对于初级产品经理来说,可以把单个模块的交互做完整,输出整洁、清晰的产品需求交付物就算合格了,面试官可以让面试者带一些相关的设计产出,并当面提问,面试的效果就比较好。

\end{itemize}

阶段建议:
\begin{itemize}
\item {} 
批量化输出能力

\item {} 
产品研发标准化

\item {} 
思考问题模型化

\end{itemize}

要在沟通中,把上交的方案落地,并按照时间节点以及实际情况(例如人员、预算等不可控因素)把任务合理细化,一一拆解,下达到各个部门,且不断跟进,每日整理问题,每日复盘,以不变应万变,化解问题,达成需求,最终让你的产品从一纸原型变为设想形态。\sphinxhref{https://www.zhihu.com/pub/reader/119583028/chapter/1057335985192501248}{12}%
\begin{footnote}[356]\sphinxAtStartFootnote
\sphinxnolinkurl{https://www.zhihu.com/pub/reader/119583028/chapter/1057335985192501248}
%
\end{footnote}


\subparagraph{高级产品经理}
\label{\detokenize{chapter_introduction/career_path:id12}}
如果你从产品经理提升为高级产品经理,将会负责微信整个支付功能,也就是一条产品线,除了微信红包,还有涉及到支付的其他功能,比如钱包、收付款等模块。

面试考核的重点:
\begin{itemize}
\item {} 
需求把控能力:这个阶段的产品经理,往往是企业招聘回来之后负责新产品的,那么对于需求的把控能力就非常的重要,把控不单单是指理解,还要包括控制,好的产品是有节奏的,特别是涉及多个部门的资源和排期,很有一种带着镣铐跳舞的感觉。
如果是我面试这部分的产品经理,我会直接问他的产品经历,重点推敲几个核心逻辑
他的产品经历,重点推敲几个核心逻辑
1、“为什么要做这个产品,需求是什么?” 2、“用户的核心场景是怎样的?”
3、“做起来之后,对业务线有什么价值?”

\item {} 
\sphinxstylestrong{资源协调、项目推动能力}:带独立的产品,和做模块是不一样的,做一个小模块,评审通过,点对点找开发沟通就可以了,但是独立的产品包含的是一整个打包的功能List,其中涉及的开发量也往往不是一个开发可以完成的,而前后端的对接,各种语言的通讯等细节都决定了排期和节奏,这些对于一个产品经理的资源协调能力要求很高,定什么里程碑,开发之间要什么时候对接,测试什么时候进行,版本回滚的机制和风险方案,这些都是考验一个产品经理资源协调,项目推动能力的地方。

\end{itemize}

高级产品经理与普通产品经理的差异:
\begin{itemize}
\item {} 
需要以产品为核心驱动与其他部门形成协作体

\item {} 
需要考虑产品的未来需求演进(做长半衰期的事情)

\item {} 
需要能更好的进行换位思考,进一步挖掘运营需求

\item {} 
需要优先考虑低成本的实现方案(用低成本实现伟大创新)

\item {} 
需要有既简练又高效的沟通方式

\item {} 
需要有清晰的项目管理流程

\item {} 
需要有高质量的文档及原型

\end{itemize}

\begin{figure}[H]
\centering
\capstart

\noindent\sphinxincludegraphics{{career_path_vs}.png}
\caption{产品专员\sphinxhyphen{}>产品经理\sphinxhyphen{}>高级产品经理}\label{\detokenize{chapter_introduction/career_path:id30}}\end{figure}


\subparagraph{产品总监}
\label{\detokenize{chapter_introduction/career_path:id13}}
当你从高级产品经理晋升为产品总监,你就不只需要负责微信支付产品线,还要肩负微信涉及到移动支付领域的整体工作。微信支付涉及移动支付领域的工作不只是微信内部的产品上线和协调工作,还涉及到外部协调和对接,比如说与金融机构的协调。(根据百度百科的定义:移动支付是指移动客户端利用手机等电子产品来进行电子货币支付,移动支付将互联网、终端设备、金融机构有效地联合起来,形成了一个新型的支付体系。)

对于高阶产品经理,能从业务角度和产品发展角度对技术架构进行预判,能掌握新技术的基本原理并加以运用到产品和业务中,是产品综合实力的一种体现,能做出在时间、资源、效率上最优的产品决策。


\subparagraph{事业部负责人}
\label{\detokenize{chapter_introduction/career_path:id14}}
除了要具备产品总监的能力还要懂运营和渠道、资金和财务,对业务业绩负责;

商业产品经理(为整个商业目标负责的角色):在毕业后的前两年做技术工程师,后来转型做了三年的产品经理,现在开始做用户增长方面的运营工作,开始带团队,培养自己的领导力,锻炼自己的战略规划能力、总结复盘能力、汇报能力等。\sphinxhref{https://weread.qq.com/web/reader/46532b707210fc4f465d044k33e3289021c33e75ff09694}{8}%
\begin{footnote}[357]\sphinxAtStartFootnote
\sphinxnolinkurl{https://weread.qq.com/web/reader/46532b707210fc4f465d044k33e3289021c33e75ff09694}
%
\end{footnote}


\subparagraph{产品副总裁}
\label{\detokenize{chapter_introduction/career_path:id15}}
如果你从产品总监,升为产品副总裁,那就需要负责微信产品部门的整体工作,不只包括微信支付,还有小程序、微信公众平台、微信广告等。

这一阶段的产品经理需要协调战略、配置资源。资源是永远不够,再大的企业,你会有更大的野心和雄心,而且永远会出现误判的情况,所以对于战略层的最大要求就是心力,要心硬如铁,该对不起你的时候,只能对不起,该牺牲你的时候,就只能牺牲你。如果还好你没有牺牲掉,你还活过来,我会再温暖地拥抱你,然后你再到那儿再去牺牲一次。

高阶的产品经理(VP)要做的是把CEO的战略进行落地,设计组织、人才的结构,制定
KPI考核制度,配置好资源都是这个阶段的产品经理应该关心的事。\sphinxhref{http://m.74cms.com/m\_view/id/1106.html}{19}%
\begin{footnote}[358]\sphinxAtStartFootnote
\sphinxnolinkurl{http://m.74cms.com/m\_view/id/1106.html}
%
\end{footnote}


\subparagraph{产品CEO}
\label{\detokenize{chapter_introduction/career_path:ceo}}
在整个产品经理职业发展路径中,如果你最后担任产品CEO角色,就像张小龙,不仅负责整个微信产品部门,还会负责腾讯的其他产品或业务,比如说FoxMail(QQ邮箱)。

这个层次需要的是资源整合能力、管理能力以及对商业的精准判断。

对产品之外的事情应该主动关注,不管是市场营销还是渠道管理,甚至也要关注财务、人力资源。如果你要想成为CEO,那么这些都是要了解、要精通的。
\sphinxhref{https://weread.qq.com/web/reader/46532b707210fc4f465d044k70e32fb021170efdf2eca12}{7}%
\begin{footnote}[359]\sphinxAtStartFootnote
\sphinxnolinkurl{https://weread.qq.com/web/reader/46532b707210fc4f465d044k70e32fb021170efdf2eca12}
%
\end{footnote}

\begin{center}\sphinxincludegraphics{{path}.jpg}\end{center} \sphinxincludegraphics{{PM_top}.jpg}


\paragraph{「急流勇退」}
\label{\detokenize{chapter_introduction/career_path:id16}}
老人们的态度更值得玩味:他们之中朝着这一条路「走到黑」的人是少数,有些产品经理,往上游而去,职场路变为供应方,有人则游向下游,改做渠道。又因为产品经理是什么都要懂一点,不少老人,改做运营或设计,甚至成为程序员。\sphinxhref{https://www.zhihu.com/pub/reader/119583028/chapter/1057335985192501248}{12}%
\begin{footnote}[360]\sphinxAtStartFootnote
\sphinxnolinkurl{https://www.zhihu.com/pub/reader/119583028/chapter/1057335985192501248}
%
\end{footnote}


\paragraph{分类}
\label{\detokenize{chapter_introduction/career_path:id17}}\begin{itemize}
\item {} 
执行类产品经理:指只掌握需求生产能力的产品经理;

\item {} 
筹划类产品经理:指开始参与市场工作的产品经理。

\end{itemize}

\begin{figure}[H]
\centering
\capstart

\noindent\sphinxincludegraphics[width=400\sphinxpxdimen]{{PM_class}.png}
\caption{产品经理能力\sphinxhref{http://www.woshipm.com/pmd/2466877.html}{5}\sphinxfootnotemark[361]}\label{\detokenize{chapter_introduction/career_path:id31}}\end{figure}
%
\begin{footnotetext}[361]\sphinxAtStartFootnote
\sphinxnolinkurl{http://www.woshipm.com/pmd/2466877.html}
%
\end{footnotetext}\ignorespaces 
\sphinxstylestrong{对比程序员的成长路径}

几乎所有高薪架构师,都懂得多门主流编程语言,如 C++、Java、Python
等,以确保在架构系统时局限性更小,此外,他们还可以使用如 MySQL、SQL
Server、sybase、jracle、infomix 等多种数据库,他们还了解文件系统特性,如
NFS、GFS、NTDFS、XFS 等,甚至做过几年 Windows
开发。正是这些经历,才造就了一名优秀的架构师或 CTO。

\begin{center}\sphinxincludegraphics{{engineer_ability}.jpg}\end{center} \sphinxincludegraphics{{coder_path}.png}

\begin{figure}[H]
\centering
\capstart

\noindent\sphinxincludegraphics{{all_path}.jpg}
\caption{职位路径}\label{\detokenize{chapter_introduction/career_path:id32}}\end{figure}


\paragraph{了解产品流程 2\sphinxfootnotemark[362]}
\label{\detokenize{chapter_introduction/career_path:id18}}%
\begin{footnotetext}[362]\sphinxAtStartFootnote
\sphinxnolinkurl{http://www.woshipm.com/zhichang/906380.html}
%
\end{footnotetext}\ignorespaces 
对于一年以下产品经验的应届生,我会让他开始独立做运营类的需求,一般这样的需求比较简单,涉及的关联系统也会单一,对核心业务的要求也没那么高,逻辑思维上也比较简洁,这也是他了解产品流程,业务流程最快的方式,而且运营类活动活动周期短,反馈快,他能快速知道自己的不足之处,快速提升产品思维,数据意识和沟通效率,快速高效的反馈,是其快速成长的关键。

产品管理流程分为:产品定义、产品设计、UI
设计、开发、测试、预发布、实验局、发布、持续运营这 9 个环节;
\sphinxhref{http://www.xmamiga.com/3573/}{15}%
\begin{footnote}[363]\sphinxAtStartFootnote
\sphinxnolinkurl{http://www.xmamiga.com/3573/}
%
\end{footnote}


\paragraph{当导师提升自己的领导力 13\sphinxfootnotemark[364]}
\label{\detokenize{chapter_introduction/career_path:id19}}%
\begin{footnotetext}[364]\sphinxAtStartFootnote
\sphinxnolinkurl{https://www.zhihu.com/pub/reader/119980992/chapter/1284104650384265216}
%
\end{footnotetext}\ignorespaces 
产品经理要通过自身方法论的沉淀主动地寻求知识传承的机会,同时也要抓住给应届毕业生当导师的机会,快速地扩大自己领导力的地盘,从而不断地提升自己的领导力。如果有一天机会来了,那么管理岗位自然就是自己的了。


\paragraph{理解青春饭}
\label{\detokenize{chapter_introduction/career_path:id20}}
体力:在行业尚有大量新市场可开拓时,企业由于想快速争夺用户,不可避免地会导致员工的工作强度增大。
脑力:要不断快速地学习大量的新知识。\sphinxhref{https://www.zhihu.com/question/20791021/answer/86421255}{17}%
\begin{footnote}[365]\sphinxAtStartFootnote
\sphinxnolinkurl{https://www.zhihu.com/question/20791021/answer/86421255}
%
\end{footnote}


\paragraph{职级晋升 3\sphinxfootnotemark[366]}
\label{\detokenize{chapter_introduction/career_path:id21}}%
\begin{footnotetext}[366]\sphinxAtStartFootnote
\sphinxnolinkurl{https://www.yuque.com/weis/pm/lto95c}
%
\end{footnotetext}\ignorespaces 
晋升和职级标准制定的理性目标应该是为公司发展服务。

最合理的标准需要考虑公司内部业务和人才的现状、未来发展预期,来决定公司未来一段时间应该侧重激励什么。比如侧重短期绩效,则人人争先,短期内公司会有较强的战斗力;如果注重潜力,优先选拔高潜年轻人,则对公司的长期竞争力有利;如果注重专业能力,则公司的产品质量或技术含量会领先;如果注重协调沟通和文化价值观,则公司的组织能力和大规模作战能力会有优势。

公司制定晋升和职级标准,还要考虑内部的文化历史惯性和理解能力,以及外部大众的接受度,考虑在相关人才市场上的稀缺性和企业的竞争力。兼顾了上述约束条件,还最有利于公司短、中、长期发展目标的,才是理性的晋升和职级标准。

产品经理绩效的定义可以差别很大,体验、收入、增长、创新、进度、效率、产品架构设计、组织建设、业务方满意度等均可作为判断标准,收入还可以分为侧重短期数字指标和长期总收入最大化。对产品经理能力的定义也可以差别很大,专业能力、业务能力、管理能力就是三种完全不同的发展方向,但它们都可能创造巨大价值,所以要把合适的人放在合适的岗位上。

资深产品经理的级别升高,在企业里越来越重要,他的素质、潜力、品性的重要性(相对专业能力)会越来越高,这是因为高阶产品经理通常是一个中枢岗位,要协调很多团队间的工作,要权衡很多员工和很多用户间的利益分配。
有些人的职级高,可能是因为他负责产品的业务规模大,或者团队规模大,或者给边缘业务的优待(边缘业务难吸引优秀人才,需要额外福利)。这样的晋升明规则或潜规则本身没有错,是符合企业利益的,但总会有聪明人会钻漏洞,比如拼命地招人以扩大团队规模,或者拼命做大业务规模以追求不健康的增长(一般是不计
ROI 的高额营销资源投入,或透支公司整体的品牌口碑)。

职级晋升看重领域经验、工龄、履历背景的企业也是有的,如果追求业务稳定发展,这也没什么错。还有些情况是因为稀缺性,某些人才很稀缺,就容易获得更高的薪酬和级别。还有些情况是,员工被猎头或朋友诱惑得到了好的工作机会,想离职,那么企业为了挽留他而给他加薪升级是很常见的。也有些公司的薪酬级别对应关系较严格,有的部门要招进某个高薪人才,就会给他申报更高职级。也有些人因为项目烂尾(不是他的过错)补偿晋级,或者被调去边缘岗位而补偿晋级。还有一些职级错配的原因,可能是评审有随机性,或者某人是擅长做
PPT
的演讲型选手,或做出把他人的业绩说成是自己业绩的作弊行为,或者领导强推特批帮助晋升等。


\paragraph{空降}
\label{\detokenize{chapter_introduction/career_path:id22}}
空降高阶产品经理,成功率天然就是低的。这是因为,产品经理这个职业既需要纵向深入理解业务,又需要横向跟很多团队深度协作,所以空降高阶人员天然就要付出很高的熟悉成本和磨合成本。产品经理做决策还无法都用数据和事实说话,必须依赖知识和数据背后的判断和理念,而空降新人不可能与原有团队总是达成共识,这也使得基层产品经理遇到上级换人和技术运营搭档换人时,如同跳槽一样难以适应。于是,空降高阶产品经理的常见结果就是走一批原来的下属产品经理。只有在这几种情况下,空降高阶产品经理的成功率会高一些:任务是复制一个产品;开始一个新产品;灾后重建,原产品出了大问题,人心思变;有巨大新要素成熟,给产品带来创造巨大新价值的机会。


\paragraph{理解上级}
\label{\detokenize{chapter_introduction/career_path:id23}}
产品经理不能只盯着产品功能思考问题,不能一直按照自己的产品情怀去工作,要能够理解公司的战略,要能够站在上级领导的角度思考问题,这样才能够知道到底哪个环节有问题,才知道如何提升对应的能力。比如,现在新用户的注册转化率比较低,你不能单纯地认为这是市场推广做得不够、流量下降导致的结果。你作为产品经理要能够知道当前的数据,理解市场推广的渠道效率、匹配度,然后再回到产品流程中找原因,想办法优化调整,千万不要觉得自己做的产品功能非常好、用户的交互体验非常顺畅等,一定要站在上级的角度看是否已经达到了公司的商业目标。这才是为什么产品经理要成为全栈产品经理的原因。


\paragraph{择行 11\sphinxfootnotemark[367]}
\label{\detokenize{chapter_introduction/career_path:id24}}%
\begin{footnotetext}[367]\sphinxAtStartFootnote
\sphinxnolinkurl{https://weread.qq.com/web/reader/77532110721ea34a7751c9akc1632f5021fc16a5320f3dc}
%
\end{footnotetext}\ignorespaces 
消费互联网红利递减,产业互联网异军突起,产品的受众人群可能是有专属业务技能和业务知识的用户。因此,产品设计会与业务有更多的关联。此时对于产品经理来说,行业经验和业务知识的积累尤为重要。


\paragraph{产品经理的发展建议 14\sphinxfootnotemark[368]}
\label{\detokenize{chapter_introduction/career_path:id25}}%
\begin{footnotetext}[368]\sphinxAtStartFootnote
\sphinxnolinkurl{https://www.zhihu.com/pub/reader/119980992/chapter/1284104631514009600}
%
\end{footnotetext}\ignorespaces \begin{enumerate}
\sphinxsetlistlabels{\arabic}{enumi}{enumii}{}{.}%
\item {} 
产业互联网。将具体的业务与互联网相结合,打造自身竞争力。举一个车险领域的例子,常规的互联网产品经理会停留在
App
产品策划、用户体验上,缺少对车险业务的关注。产品经理只有深入车险的具体业务中,才能成为这个产业的专家。

\item {} 
综合发展。除了产品方向,产品经理可以培养运营、项目管理、商业分析等方面的能力,让自己成为一个综合型人才。我见过一些产品经理转型运营、转型投资,他们都非常成功,综合能力强也意味着发展的机会比较多。

\end{enumerate}


\paragraph{为什么大家现在选择产品经理、设计师这些职位呢? 16\sphinxfootnotemark[369]}
\label{\detokenize{chapter_introduction/career_path:id26}}%
\begin{footnotetext}[369]\sphinxAtStartFootnote
\sphinxnolinkurl{https://www.zhihu.com/question/20791021/answer/640398686}
%
\end{footnotetext}\ignorespaces 
你会发现当你大学毕业,在找工作的时候,或者说你在转行的时候,有些工作的这种壁垒是很低的,比如说一般性的销售工作,它的门槛是非常低的,没有什么不可替代性。

但是产品经理也好,设计师也好,这个行业里面做得越久,在这个领域里面的一些垂直的领域扎根越深,那么你的领域知识,你的专业技能会越来越强,这些东西都会成为你的壁垒。

在很多传统行业里面,就算你一个人再牛,我有十个人,甚至我有一百个人,是能够比你一个人做的事情要更多的;但是在互联网领域里面,它的特点就是一个资深的/一个真正做的好的专家,产品经理
or 设计师,一个人的创造力很可能会大于几十个人甚至一百个人的创造力。


\subsection{思维/软实力}
\label{\detokenize{chapter_idea/index:chap-idea}}\label{\detokenize{chapter_idea/index:id1}}\label{\detokenize{chapter_idea/index::doc}}

\subsubsection{思维}
\label{\detokenize{chapter_idea/idea:id1}}\label{\detokenize{chapter_idea/idea::doc}}

\paragraph{规划思维应用职业规划}
\label{\detokenize{chapter_idea/idea:id2}}
小白需要实践的是如何通过自己的能力从资源生产者变成资源管理者,最后变成资源支配者。
\sphinxhref{http://www.woshipm.com/zhichang/4114367.html}{1}%
\begin{footnote}[370]\sphinxAtStartFootnote
\sphinxnolinkurl{http://www.woshipm.com/zhichang/4114367.html}
%
\end{footnote}

处于成长期阶段的人通过自己的技能输出得到回报,属于资源生产者;
处于领导者阶段的人除了技能输出,还是有整合资源的能力,属于资源管理者;
处于自我实现阶段产品经理通过参与创业的方式触碰行业天花板,属于资源支配者。

一个人属于哪个层面,与处在该层面的时间没有直接的联系,很多在一个行业做了N年的人,如果他们一直在成长期打转,在职业生涯中从来都没有支配过资源,那么就很难获得超额的价值回报。

其实有产品思维的牛人,不一定最开始是产品经理,比如王峰最开始的时候金山是负责营销工作,因为业绩出色,后来才成为了金山词霸和金山毒霸的产品市场负责人,最后做到金山副总裁。

在金山他跨越了成长期和领导者期,财务自由之后最后离职创业就很顺其自然,后面他成立了游戏公司上市,以及现在做的区块链媒体都属于自我实现阶段。


\paragraph{逻辑思维来撕逼和思考}
\label{\detokenize{chapter_idea/idea:id3}}
逻辑思维能力被简单地理解为“任何一件事情都可以归纳出一个中心论点,而这个中心论点可以由3到7个论据进行支撑,每一个论据本身又可以成为一个论点。”

在逻辑思维中,由论据到论点的推理过程就叫作推理,推理是最根本的逻辑关系。

论点的成立与否是逻辑思维的主题,而论点的成立与否又直接源自论据的真假,如果论据是假的,那么得到的论点也是假的。

提升交流质量很重要一步就是用逻辑思考对方的观点是否正确,否者只能淹没在信息中。

还有的朋友在吵架撕逼的时候,用逻辑思维找出对方的论据错误和论据到论点的推理错误,往往能在撕逼中脱颖而出。


\paragraph{敏捷思维对抗不确定性}
\label{\detokenize{chapter_idea/idea:id4}}
有一次一个研发同学和我聊起买房的事情,他手里有些积蓄,想在杭州买一套大户型的房子,但是还差很多钱,于是他想着把钱攒够了在去买房。

但是我们知道杭州房价过去几年一直处于上升通道,用敏捷思维思考这个事情,我给他的建议是用自己手里的钱可以先买一个小户型的房子,有机会再换成大户型。

先解决从无到有,再去解决从小到大,如果继续等下去付出的成本只会越来越高。在面对重大不确定的重大选择时,最稳妥的方式就是用敏捷思维先解决核心需求。

利用敏捷思维要求每次只做最重要的事情,满足用户和环境不断的变化,并且努力把每一件小事情做好,期待未来回报奇点的到来。

身边大多数牛人拥有迭代思维还意味着他们很乐观,因为纵使对当前的产品不满意,但是他们会想各种办法,哪怕每天进步一点点,去不断迭代自己,而不是把时间浪费在对未知的恐惧。


\paragraph{运营思维做销售}
\label{\detokenize{chapter_idea/idea:id5}}
一个小米同学和我讲,在小米初创期间,黎万强一手打造了小米的社区模式,但是初期的阿黎在互联网圈中并不知名,小米初期也没有很强的品牌影响力。

所以阿黎只能组织团队,注册了很多小号,在各个论坛发布了很多MIUI的帖子,刚好用户也有需求,才慢慢积累的种子用户。

初期做法是不是和做微商有点类似?


\paragraph{有限理性}
\label{\detokenize{chapter_idea/idea:id6}}

\subparagraph{原因}
\label{\detokenize{chapter_idea/idea:id7}}\begin{itemize}
\item {} 
信息获取能力有限

\item {} 
信息处理能力有限

\item {} 
禀赋偏好导致的个体差异

\item {} 
环境的不确定因素

\end{itemize}


\subparagraph{理性决策三要素}
\label{\detokenize{chapter_idea/idea:id8}}\begin{itemize}
\item {} 
理性的信念:与真实世界一致的信念,对自我认知的认知;人类最大的理性,就是理解自身认知能力的局限性

\item {} 
理性的目标:约束条件下的价值(总效用)最大化

\item {} 
理性的行动:给定目标下,寻找最优的解决方案

\end{itemize}


\paragraph{质疑}
\label{\detokenize{chapter_idea/idea:id9}}
当你有一个想法的时候,你不要抱着它是对的去调研和分析,而是站在对立面,本着「这事情不靠谱」的思路,去做调研和分析。如果分析调研到最后,发现这个事情,真的确实可以,那么就是真的可以。\sphinxhref{https://www.zhihu.com/question/19638354/answer/377605037}{2}%
\begin{footnote}[371]\sphinxAtStartFootnote
\sphinxnolinkurl{https://www.zhihu.com/question/19638354/answer/377605037}
%
\end{footnote}


\subsubsection{同理心}
\label{\detokenize{chapter_idea/empathy:id1}}\label{\detokenize{chapter_idea/empathy::doc}}

\paragraph{「小白理论」}
\label{\detokenize{chapter_idea/empathy:id2}}
在看待产品的时候,不能以我们创造者的专业身份来看,而需要用同理心,将自己转变为一个产品的典型用户,才能准确挖掘到用户心底最真实的诉求。

乔布斯只需要 1 秒钟就能变成「白痴」,这是他最厉害的地方,马化腾大概需要
5 秒,而小龙自己差不多需要 10 秒钟。

TODO:

面对一个海量用户产品的时候,产品经理必须是一部分人的「白痴」,同时又是另一部分人「最精明的保护者」。


\paragraph{心理测试}
\label{\detokenize{chapter_idea/empathy:id3}}
观察用户不起眼的微小动作,喜欢揣测他人的动机,并验证自己是否猜对了,养成这样的习惯,可以提高自己对用户心理诉求的敏锐度。


\subsubsection{创意}
\label{\detokenize{chapter_idea/create:id1}}\label{\detokenize{chapter_idea/create::doc}}

\paragraph{一个产品的起点不是调研而是创意}
\label{\detokenize{chapter_idea/create:id2}}
很多其它产品经理教程也好、攻略、笔记也好,貌似一个新产品的起点都是各种调研,市场调研、商业调研、用户调研啊等等…不过假如没有创意就没有调研的目标和目的!所以咱们先从创意收集聊起。\sphinxhref{https://www.zhihu.com/column/i-mpm}{1}%
\begin{footnote}[372]\sphinxAtStartFootnote
\sphinxnolinkurl{https://www.zhihu.com/column/i-mpm}
%
\end{footnote}


\paragraph{创意有可能由谁提出,由谁记录?}
\label{\detokenize{chapter_idea/create:id3}}
产品的创意几乎所有人都可能提出,而这个记录产品创意的人应该是产品经理。

每个参与者都有其不同的视角和认知层次,产品或项目初期的参与者可能有产品经理、UI设计、研发工程师、运营、市场推广、以及各种高管。


\paragraph{创意如何分类和记录?}
\label{\detokenize{chapter_idea/create:id4}}
创意的分类维度有很多,如按照职责来记录或按照「谁说的」来分类记录。

记录方式也简单,最简单可以用纸和笔或者白板,但是这种方式不易维护也容易丢失。


\paragraph{记录和整理完成之后呢?}
\label{\detokenize{chapter_idea/create:id5}}
产品经理需要把创意,通过邮件发送给所有参与者,这里强调一下,最好不要用qq或者微信,要用邮件。

\sphinxstylestrong{因为电子邮件是可以追溯的,万一自己的电脑出问题了,弄丢了也可以通过邮箱找回。}

这样优秀的习惯一定会被大家所认可和喜欢,不要觉得做会议纪要LOW,因为这样才能真正做到了解你所做的产品,才能做到了解每个人的想法,为今后的工作开个好头。


\paragraph{产品经理在此过程中的角色和态度应该是什么样的?}
\label{\detokenize{chapter_idea/create:id6}}\begin{itemize}
\item {} 
更多时候应该做一名倾听者

\item {} 
在恰当的时候发表确切的言论

\item {} 
做好记录,确保不遗漏任何一个好点子

\end{itemize}


\paragraph{下面小结一下创意收集阶段}
\label{\detokenize{chapter_idea/create:id7}}\begin{itemize}
\item {} 
产品启动前头脑风暴

\item {} 
讨论产品的目标行业和市场以及商业模式

\item {} 
有可能会涉及到产品的实现和推广运营策略

\item {} 
作为产品经理,做好记录尤为重要

\end{itemize}


\subsubsection{数据 1\sphinxfootnotemark[373]}
\label{\detokenize{chapter_idea/data:id1}}\label{\detokenize{chapter_idea/data::doc}}%
\begin{footnotetext}[373]\sphinxAtStartFootnote
\sphinxnolinkurl{http://www.woshipm.com/data-analysis/2696737.html}
%
\end{footnotetext}\ignorespaces 
数据通常来讲能够反应出某项业务或某类业务。


\paragraph{作用}
\label{\detokenize{chapter_idea/data:id2}}
一方面,在后续的运营计划中,数据能有效地帮助运营的小伙伴进行拉新、留存以及促活,另一方面,数据也能帮助产品经理发现用户在使用过程中出现的问题,从而进行流程的优化与布局的提升,提升用户体验。\sphinxhref{http://www.woshipm.com/pmd/707412.html}{4}%
\begin{footnote}[374]\sphinxAtStartFootnote
\sphinxnolinkurl{http://www.woshipm.com/pmd/707412.html}
%
\end{footnote}

比如像很多互联网公司都成立了大数据团队,收集用户的社交、电商、搜索行为等数据,通过所搜集的大数据来制定商业决策依据,以及通过数据挖掘形式,找到创新产品的机会。

大的互联网公司在满足自己内部决策需求的同时,也成了了大数据部门给其它公司进行赋能,比如蚂蚁金服的\sphinxstylestrong{数据产品芝麻信用},不仅能够成为蚂蚁内部各种金融产品的信用审核依据,也开放给了很多行业如出行、金融、共享服务公司等,极大提高了基于信用服务的门槛和便捷性。

通过数据采集处理分析驱动产品的价值验证、功能优化和业务决策


\paragraph{数据至上}
\label{\detokenize{chapter_idea/data:id3}}
数据至上包含两个方面:(1)能够量化客户所关心的问题,做一个数字化的支持者;(2)能够建立用于构建高质量
AI
模型的综合数据集。此外,还需要获取准确反映用户工作、行为、交互模式和痛点的数据。数据形式可以是像素、字符、数字或者比特。

如果能够对数据提取、数据处理以及数据可视化有基本的了解,则有助于创建更具客户价值的
AI 解决方案。\sphinxhref{http://www.uml.org.cn/devprocess/201910163.asps}{12}%
\begin{footnote}[375]\sphinxAtStartFootnote
\sphinxnolinkurl{http://www.uml.org.cn/devprocess/201910163.asps}
%
\end{footnote}

\sphinxstylestrong{如何数据化考核设计结果?}\sphinxhref{https://www.yuque.com/linyecx/abusg2/gsyrft}{14}%
\begin{footnote}[376]\sphinxAtStartFootnote
\sphinxnolinkurl{https://www.yuque.com/linyecx/abusg2/gsyrft}
%
\end{footnote}


\paragraph{伪数据}
\label{\detokenize{chapter_idea/data:id4}}
只蹭热度,没有做好产品核心功能的引流,导致最终拉新效果很差,活动很快就叫停了。如果我们只看活动的曝光量、参与量,会觉得很兴奋,但其实离产品的核心目标很远。

由于标题党导致的高点击率的数据而沾沾自喜,殊不知这种勤奋的打扰,长期来看是在损伤用户的利益,透支产品的信用\sphinxhref{https://www.zhihu.com/market/paid\_column/1312360599620358144/section/1332369605311516672}{8}%
\begin{footnote}[377]\sphinxAtStartFootnote
\sphinxnolinkurl{https://www.zhihu.com/market/paid\_column/1312360599620358144/section/1332369605311516672}
%
\end{footnote}


\paragraph{数据分类 {[}20{]}}
\label{\detokenize{chapter_idea/data:id5}}

\subparagraph{设备类:}
\label{\detokenize{chapter_idea/data:id6}}
设备类数据主要指用户客户端(如手机、平板电脑、笔记本、PC等
)等各类参数,主要通过页面、APP内嵌入各类sdk,js脚本等方式进行采集和获取。


\subparagraph{环境类:}
\label{\detokenize{chapter_idea/data:id7}}
环境类数据是指用户发起操作请求时所处环境的相关数据,可以分为虚拟环境和物理环境两大类。
\begin{itemize}
\item {} 
虚拟环境数据,主要指用户所的IP、WiFi等网络环境相关数据。

\item {} 
物理环境数据,主要指用户的手机定位、基站位置等相关数据。

\end{itemize}


\subparagraph{行为类:}
\label{\detokenize{chapter_idea/data:id8}}
行为类数据是指用户在网页或APP上进行各种操作时的各类数据,如用户页面停留时长、文本输入时长、键盘敲击频次等。


\subparagraph{第三方数据:}
\label{\detokenize{chapter_idea/data:id9}}
第三方数据指通过从公开途径或第三方数据服务商处获取的各类数据,包括但不限于用户的运营商数据、电商消费数据、银行数据、司法数据等各类数据。


\paragraph{解决对策}
\label{\detokenize{chapter_idea/data:id10}}
把预估的数据代入到决策模型中,进行模拟仿真,来评估执行决策结果的成本以及决策风险;并相互沟通这种有依据的成本。


\paragraph{业务数据 2\sphinxfootnotemark[378]}
\label{\detokenize{chapter_idea/data:id11}}%
\begin{footnotetext}[378]\sphinxAtStartFootnote
\sphinxnolinkurl{http://www.woshipm.com/pmd/3657472.html}
%
\end{footnotetext}\ignorespaces 
AI产品也需要采用类似数据埋点的方式去收集产品投放前后的业务指标差异,比如:GMV差异、点击率差异、转化率差异。首先为了验证产品是否对业务产生了价值,用一个粗略的公式表示AI产品的业务价值,其次是为了分析产品的哪些品功能存在优化空间,最后还可以驱动业务决策,例如例如推荐系统在电商商品推荐和广告推荐中的应用。

AI产品价值=(提高的时效*时效成本+GMV提升)\sphinxhyphen{}(AI硬件资源成本+研发成本)


\paragraph{数据采集}
\label{\detokenize{chapter_idea/data:id12}}
\begin{figure}[H]
\centering
\capstart

\noindent\sphinxincludegraphics{{data_collect}.png}
\caption{数据采集{[}21{]}}\label{\detokenize{chapter_idea/data:id28}}\end{figure}

数据的采集可以通过线下构建对应业务场景需要的环境进行拍摄采集,也可以通过平台内已有数据(线上数据、旧数据)、第三方数据(通过开源、付费购买、爬虫爬取多多种形式)获取。
\begin{enumerate}
\sphinxsetlistlabels{\arabic}{enumi}{enumii}{}{.}%
\item {} 
线上数据集的处理:多采用badcase,重新标注、增强。

\item {} 
爬取数据集:爬取公开渠道如百度图片的对应label数据集,并区分可用不可用。

\end{enumerate}

因数据集都为图片数据,并且模型是基于深度学习技术构建,故涉及到数据ETL、特征工程等一些处理暂时不需要,后续可根据业务场景和应用技术的拓展,在技术架构和平台架构补充上对应的能力。

数据集获取完成后,可以将数据按照不同的类型存放,通过数据集管理页面进管理。数据类型可以按照不同的维度区分,例如:
\begin{enumerate}
\sphinxsetlistlabels{\arabic}{enumi}{enumii}{}{.}%
\item {} 
以标品和非标品区分。

\end{enumerate}
\begin{itemize}
\item {} 
标品数据:标品静态状态数据、标品动态状态数据

\item {} 
非标品数据:标品多角度数据、标品静态状数据、标品动态状态数据、其他异常情况数据

\end{itemize}
\begin{enumerate}
\sphinxsetlistlabels{\arabic}{enumi}{enumii}{}{.}%
\setcounter{enumi}{1}
\item {} 
以数据来源渠道区分。

\end{enumerate}
\begin{itemize}
\item {} 
线下:构建不同的场景(静态动态)进行拍摄采集

\item {} 
线上:平台内已有数据(线上数据、旧数据)、第三方数据(开源数据集、付费数据集、爬取数据集)

\end{itemize}
\begin{enumerate}
\sphinxsetlistlabels{\arabic}{enumi}{enumii}{}{.}%
\setcounter{enumi}{2}
\item {} 
以数据格式区分。图片、视频、其他格式(2d、3d)。

\item {} 
以数据使用性区分。基本数据集、训练数据集(含标注)、验证数据集、异常数据集、自定义数据集。

\end{enumerate}

数据集应当有生命周期的管理和备注信息,以免在运营一段时间后数据量杂乱冗余。{[}18{]}


\subparagraph{问题 6\sphinxfootnotemark[379] 7\sphinxfootnotemark[380]}
\label{\detokenize{chapter_idea/data:id13}}%
\begin{footnotetext}[379]\sphinxAtStartFootnote
\sphinxnolinkurl{http://www.xmamiga.com/3573/}
%
\end{footnotetext}\ignorespaces %
\begin{footnotetext}[380]\sphinxAtStartFootnote
\sphinxnolinkurl{https://www.bilibili.com/video/BV1Zp4y1Q7ub?from=search\&seid=1470711389248919578}
%
\end{footnotetext}\ignorespaces \begin{itemize}
\item {} 
采集难:实际场景的数据很难采集完全,导致模型泛化能力有限

\item {} 
标注成本高:需要标注大量的数据,时间周期长

\item {} 
清洗成本高:难以快速获取高质量的数据

\end{itemize}


\subparagraph{数据质量}
\label{\detokenize{chapter_idea/data:id14}}\begin{itemize}
\item {} 
关联度;

\item {} 
时效性;

\item {} 
范围;

\item {} 
可信性。

\end{itemize}


\subparagraph{数据来源}
\label{\detokenize{chapter_idea/data:id15}}
稳定的数据来源渠道能够持续的提供深度学习“粮食”(深度学习理论上样本是多多益善,且个类别需要均衡),因此一方面可以将线上的业务样本进行沉淀,被动积累,另一方面可以针对业务样本类型,针对性的从目标网站进行爬取。当然,不能忽略的一个大的数据来源是,开源的数据集,或者竞赛的样本集,这些数据一般有较高的标注质量,可以直接拿来训练或者经过极小成本的人工审核即可以达到训练要求。


\subparagraph{数据沉淀}
\label{\detokenize{chapter_idea/data:id16}}
可以将线上的业务样本进行沉淀,被动积累

AI产品除了收集业务指标数据指导产品是否需要优化,还需要进一步做好训练数据沉淀工作。AI技术在投入试点到成熟推广,训练数据一直都是必不可少的,尤其是真实场景的数据对算法迭代更是起到“致命”的作用。

因此,如果能够源源不断的回收实际场景数据并且清洗标注,就可以提升算法准确率指标,最终提高产品使用效果,例如:可以考虑通过以下流程来实现。

\begin{figure}[H]
\centering
\capstart

\noindent\sphinxincludegraphics{{data_flow_chart}.png}
\caption{数据沉淀流程图}\label{\detokenize{chapter_idea/data:id29}}\end{figure}

其他来源:直接购买行业数据和免费的数据源;自行采集和爬取;第三方合作。


\subparagraph{数据标注}
\label{\detokenize{chapter_idea/data:id17}}
真实数据集(ground\sphinxhyphen{}truth dataset)是常规数据集,但已添加了注释。
注释可以是在图像上绘制的框,表示样本的书面文本,电子表格的新列或机器学习算法应学习输出的其他任何内容。\sphinxhref{https://wao.ai/blog/dataset-vs-ground-truth-dataset\#:~:text=A\%20ground\%2Dtruth\%20dataset\%20is,algorithm\%20should\%20learn\%20to\%20output.}{10}%
\begin{footnote}[381]\sphinxAtStartFootnote
\sphinxnolinkurl{https://wao.ai/blog/dataset-vs-ground-truth-dataset\#:~:text=A\%20ground\%2Dtruth\%20dataset\%20is,algorithm\%20should\%20learn\%20to\%20output.}
%
\end{footnote}

真实数据集分为以下几类:\sphinxhref{https://link.springer.com/chapter/10.1007/978-1-4302-5930-5\_7}{11}%
\begin{footnote}[382]\sphinxAtStartFootnote
\sphinxnolinkurl{https://link.springer.com/chapter/10.1007/978-1-4302-5930-5\_7}
%
\end{footnote}
\begin{enumerate}
\sphinxsetlistlabels{\arabic}{enumi}{enumii}{}{.}%
\item {} 
合成制作:图像由计算机模型或效果图生成。

\item {} 
真实制作:一个视频或图像序列的设计和制作。

\item {} 
真实选择:真实的图像从现有的源中选择。

\item {} 
机器自动标注:采用特征分析和学习的方法从数据中提取特征。

\item {} 
人工标注:专家定义特征和对象的位置。

\item {} 
组合式:上述任何一种混合物。

\end{enumerate}

\begin{figure}[H]
\centering
\capstart

\noindent\sphinxincludegraphics{{biaozhu}.jpg}
\caption{标注流程图}\label{\detokenize{chapter_idea/data:id30}}\end{figure}

问题:数据遮挡现象严重、数据多样性不足,例如光线差异、数据样本不均衡

图像智能标注、文本智能标注、难例识别、多人标注

更多见\sphinxhref{https://www.cnwebe.com/articles/43675.html}{3}%
\begin{footnote}[383]\sphinxAtStartFootnote
\sphinxnolinkurl{https://www.cnwebe.com/articles/43675.html}
%
\end{footnote}
\sphinxhref{https://www.zhihu.com/market/paid\_column/1312360599620358144/section/1332369605311516672}{8}%
\begin{footnote}[384]\sphinxAtStartFootnote
\sphinxnolinkurl{https://www.zhihu.com/market/paid\_column/1312360599620358144/section/1332369605311516672}
%
\end{footnote}

\sphinxurl{https://wao.ai/}


\subparagraph{数据清洗}
\label{\detokenize{chapter_idea/data:id18}}
问题:数据质量不佳、数据分布不均、大量干扰数据、大量重复数据

数据清洗可使数据获得用于分析的正确结构(Shape)和质量(Quality)。

相似度去重、去模糊、裁剪/旋转等、自定义
\begin{itemize}
\item {} 
单变量探索

\item {} 
多变量探索

\item {} 
采样 —— 平衡(Balanced)、分层(Stratified)…

\item {} 
数据分配 —— 创建训练+验证+测试数据集 …

\item {} 
数据替换 —— 剪切(Cutting)、分割(Splitting)、合并

\item {} 
缺失值处理:删除或填充(Imputation)

\item {} 
构造特征变量

\item {} 
特征工程:从已有的数据中构造出对目标变量有强影响力的特征变量

\item {} 
标准化和归一化:无量纲化

\item {} 
独热编码:将类别变量进行拆分

\item {} 
加权与选择 —— 属性加权、自动优化 …

\item {} 
属性生成 —— ID生成
…\sphinxhref{https://www.080910t.com/research/data-preprocessing-and-data-cleaning/}{16}%
\begin{footnote}[385]\sphinxAtStartFootnote
\sphinxnolinkurl{https://www.080910t.com/research/data-preprocessing-and-data-cleaning/}
%
\end{footnote}

\item {} 
分箱(Binning):数值型变量转为类别变量,或连续性变量变为离散型变量

\item {} 
数值变量和类别变量
\sphinxhref{http://www.followmedoitbbs.com/forum.php?mod=viewthread\&tid=8312\&extra=page\%3D1}{13}%
\begin{footnote}[386]\sphinxAtStartFootnote
\sphinxnolinkurl{http://www.followmedoitbbs.com/forum.php?mod=viewthread\&tid=8312\&extra=page\%3D1}
%
\end{footnote}

\end{itemize}


\subparagraph{数据扩充}
\label{\detokenize{chapter_idea/data:id19}}
增强、合成、生成、数据集市


\subparagraph{数据读取}
\label{\detokenize{chapter_idea/data:id20}}\begin{enumerate}
\sphinxsetlistlabels{\arabic}{enumi}{enumii}{}{.}%
\item {} 
获取数据:爬虫、数据库导出

\item {} 
存储数据:csv、excel、json、mysql
\sphinxhref{http://www.followmedoitbbs.com/forum.php?mod=viewthread\&tid=8312\&extra=page\%3D1}{13}%
\begin{footnote}[387]\sphinxAtStartFootnote
\sphinxnolinkurl{http://www.followmedoitbbs.com/forum.php?mod=viewthread\&tid=8312\&extra=page\%3D1}
%
\end{footnote}

\end{enumerate}


\subparagraph{管理分析}
\label{\detokenize{chapter_idea/data:id21}}
数据集管理、版本管理、数据挖掘、数据可视化


\paragraph{放入真实商业环境 3\sphinxfootnotemark[388]}
\label{\detokenize{chapter_idea/data:id22}}%
\begin{footnotetext}[388]\sphinxAtStartFootnote
\sphinxnolinkurl{https://www.cnwebe.com/articles/43675.html}
%
\end{footnotetext}\ignorespaces 
不止GMV=DAU\sphinxstyleemphasis{转化率}客单价
\begin{enumerate}
\sphinxsetlistlabels{\arabic}{enumi}{enumii}{}{.}%
\item {} 
剔除虚假证据

\item {} 
深入发现问题

\item {} 
挖掘潜在因素

\item {} 
观察长期趋势

\end{enumerate}


\paragraph{工具:}
\label{\detokenize{chapter_idea/data:id23}}
Excel、SQL

神策分析、GrowingIO、友盟、TalkingData、诸葛IO、\sphinxurl{http://www.51.la/}、\sphinxurl{http://www.google.cn/analytics/}
这种工具平台


\paragraph{数据问题 5\sphinxfootnotemark[389]}
\label{\detokenize{chapter_idea/data:id24}}%
\begin{footnotetext}[389]\sphinxAtStartFootnote
\sphinxnolinkurl{http://www.changgpm.com/thread-350-1-1.html}
%
\end{footnotetext}\ignorespaces \begin{enumerate}
\sphinxsetlistlabels{\arabic}{enumi}{enumii}{}{.}%
\item {} 
存不下

\item {} 
流不动

\item {} 
用不好

\end{enumerate}


\subparagraph{存不下——数字化浪潮下的海量数据存储挑战}
\label{\detokenize{chapter_idea/data:id25}}
数据量从PB级向EB级迈进,数据量将从2018年的32.5ZB快速增长到2025年的180ZB。

存储扩展性不足:传统存储由独立的控制器与硬盘框组成,当容量不足时可增加新的硬盘框进行级联,但由于控制器的处理能力受限,存储的扩展能力非常有限。

存储协议类型单一:非结构化数据逐步成为企业数据的主体。随着电商、物联网等业务扩张,80\%的新增数据由各类音视频、日志等非结构化数据构成。然而传统存储协议类型单一,无法同时满足块、对象、文件、大数据等多样性数据的存取需求,企业不得不为每一种新的数据类型新增一种存储设备,增加了高效利用存储资源的难度。存储成本依然高昂:越来越多的企业选择将数据长期保存。2017年起,移动运营商因合规性要求,将其设备日志的保存周期从2个月增加至6个月。

这意味着其数据存储服务器的设备规模将增加至少2倍。传统的架构中,服务器因存储需求不断扩容,但CPU的使用率却始终处于较低的状态,资源得不到合理利用,无疑会对采购成本和维护成本造成更大的压力。企业不得不因为存储成本而放弃大量宝贵数据。


\subparagraph{流不动——由来已久的数据孤岛难题}
\label{\detokenize{chapter_idea/data:id26}}
孤立的数据价值并不显著,只有当数据像水一样流动起来,才能打破“数据壁垒”,最大化释放其价值。

数据的“三类孤岛”:应用孤岛:不同应用产生的数据分别存放在不同的存储系统中,而且这些数据由于各自的特征,彼此之间是无法共享使用的,即形成“应用孤岛”问题;管理孤岛:为对生产数据加以保护和使用,会将生产数据的一个副本,拷贝到各个系统(如备份、容灾、归档、开发测试和分析系统)中进行管理和使用。即便是同一份数据,为实现不同目的,还需分别存储、管理和使用,即形成“管理孤岛”问题;地理孤岛:由于企业的更新换代,将存在多套存储设备,比如生产环境、非生产环境、云环境和边缘环境,企业的数据将存放在不同的地方,形成“地理孤岛”问题。


\subparagraph{用不好——数据供应不足造成应用复杂低效}
\label{\detokenize{chapter_idea/data:id27}}
海量的数据孕育了前所未有的机遇,也带来了巨大的挑战。甚至有人说,从来不缺数据,数据多了反而成为一种负担。也有人说,数据只是资源,而不是资产,很难产生价值。其根本原因是没有用好数据,数据没有释放价值。而影响数据价值释放的主要原因是数据供应不足,无法反馈业务本质,支持业务决策:大量数据未存储。

企业每天会产生大量数据,但传统的数据录入需要预先的人工规划,这导致大量非结构化数据以及一些新型的数据无法进入系统(例如IoT数据、视频数据、图片数据等)。数据的缺失会削弱对业务的感知,无法真实及时地反映出业务本质。

找不到数据传统企业通常通过数据表来管理和分析数据,规模较大的公司数据表甚至可以达到数百万张,而且分散在各个业务系统中。如果没有统一数据目录和全局数据视图,要在上百万张报表中找到特定的数据,好比大海捞针,无法应对灵活多变的业务需求。

{[}17{]}: {[}18{]}:
\sphinxurl{https://coffee.pmcaff.com/article/2162967852132480/pmcaff?utm\_source=forum\&newwindow=1}
{[}19{]}: \sphinxurl{https://www.yuque.com/linyecx/abusg2/gsyrft} {[}20{]}:
\sphinxurl{https://zhuanlan.zhihu.com/p/59042022} {[}21{]}:
\sphinxurl{http://sjrzld.com/a/AI0273.html}


\subsubsection{沟通}
\label{\detokenize{chapter_idea/communicate:id1}}\label{\detokenize{chapter_idea/communicate::doc}}
产品经理的沟通能力非常重要,以后要面对大大小小各种跨部门沟通,包括对研发、运营、市场、客服等等,良好的沟通能力主要体现在语言表达清晰度、流畅度、断句顿挫合理等等,让人舒服的沟通能力是面试官考量的重点。尤其面试官如果作为未来的上级,顺畅的沟通方式是非常核心的考察点。

沟通包括沟通内容和沟通技巧,技巧属于剑宗,内容属于气宗。懂沟通技巧的人,即使说的内容没有任何营养,但是别人爱听。而博学多才的人,如一些老教授,即使性格怪癖,很难沟通,但是他的话也很容易让人信服。在我们综合实力和个人魅力还达不到一定层次时,我们可以通过一些沟通技巧来提升沟通效果。\sphinxhref{http://dadaghp.com/index/index/article\_detail/mikuai/wenzhang/id/314.html}{1}%
\begin{footnote}[390]\sphinxAtStartFootnote
\sphinxnolinkurl{http://dadaghp.com/index/index/article\_detail/mikuai/wenzhang/id/314.html}
%
\end{footnote}


\subsubsection{谈判}
\label{\detokenize{chapter_idea/negotiation:id1}}\label{\detokenize{chapter_idea/negotiation::doc}}
产品经理需要谈判的地方也很多,比如涉及跨部门合作需要对接资源的时候,涉及部门利益的交换,比如与上下游合作的供应商有业务往来需要谈判。


\paragraph{本质}
\label{\detokenize{chapter_idea/negotiation:id2}}
产品经理在跨部门沟通需要拿资源的时候,这个主动权在资源方,要做到不卑不吭,这里的双赢心态第一要素是要突出谈判结果对谈判方是有利的,然后对公司是有利的,最后才说对自己是有利的。

产品经理在与供应商谈判的时候,这时主动权在自己这方,这个时候也需要可以同时与多家同级别的供应商进行谈判,突出迂回策略。


\paragraph{过程}
\label{\detokenize{chapter_idea/negotiation:id3}}
主动:不迫不及待 最好先引导对方先说出自己的条件
语言上的不接受,还要再情绪表达上,适当的沉默
借鉴行业标准,行业标准是法律法规、先前的惯例、市场行业价格。


\paragraph{谈判技巧}
\label{\detokenize{chapter_idea/negotiation:id4}}\begin{itemize}
\item {} 
黑脸白脸

\item {} 
虚构领导

\item {} 
最后恭喜

\end{itemize}


\paragraph{协议2\sphinxfootnotemark[391]}
\label{\detokenize{chapter_idea/negotiation:id5}}%
\begin{footnotetext}[391]\sphinxAtStartFootnote
\sphinxnolinkurl{http://terms.aliyun.com/legal-agreement/terms/suit\_bu1\_ali\_cloud/suit\_bu1\_ali\_cloud201802261104\_19214.html?spm=a2c4g.11186623.2.11.58bf5c39z5GukP}
%
\end{footnotetext}\ignorespaces 
免费:不排除日后收取费用的可能,将提前10个自然日通过在网站内合适版面发布公告或发送站内通知等方式公布收费政策及规范。仍使用阿里云服务的,应按届时有效的收费政策付费并应遵守届时公布的有效的服务条款。拒绝支付服务费的,有权不再向您提供服务,并有权利不再继续保留您的业务数据。

完整性和保密性:您对的数据以及进入和管理口令、密码的完整性和保密性负责。因您维护不当或保密不当致使上述数据、口令、密码等丢失或泄漏所引起的损失和后果均由您承担。

数据:当服务期届满、服务提前终止(包括双方协商一致提前终止,其他原因导致的提前终止等)或您发生欠费时,除法律法规明确规定、主管部门要求或双方另有约定外,仅在一定的缓冲期(以您所订购的服务适用的专有条款、产品文档、服务说明等所载明的时限为准)内继续存储您的用户业务数据(如有),缓冲期届满将删除所有用户业务数据,包括所有缓存或者备份的副本,不再保留您的任何用户业务数据。用户业务数据一经删除,即不可恢复;您应承担数据因此被删除所引发的后果和责任,您理解并同意,阿里云没有继续保留、导出或者返还用户业务数据的义务。


\subsubsection{品牌}
\label{\detokenize{chapter_idea/brand:id1}}\label{\detokenize{chapter_idea/brand::doc}}

\paragraph{品牌管理}
\label{\detokenize{chapter_idea/brand:id2}}
随着生产力的丰富,产品的竞争焦点已经从功能、价格等基础要素,发展到产品的品牌、文化、价值观认同等高级要素了。


\subparagraph{定义}
\label{\detokenize{chapter_idea/brand:id3}}
针对企业产品和服务的品牌,综合地运用企业资源,通过计划、组织、实施、控制来实现企业品牌战略目标的经营管理过程。

品牌是一种错综复杂的象征。它是品牌属性、名称、包装、价格、历史、信誉,广告方式的无形总称。品牌同时也是消费者对其使用者的印象,以其自身的经验而有所界定。产品是工厂生产的东西;品牌是消费者所购买的东西。产品可以被竞争者模仿,但\sphinxstylestrong{品牌则是独一无二}的,产品极易迅速过时落伍,但成功的品牌却能持久不坠,品牌的价值将长期影响企业。


\subparagraph{金三角}
\label{\detokenize{chapter_idea/brand:id4}}
金三角=领导能力+分析能力+沟通与协调能力
\begin{enumerate}
\sphinxsetlistlabels{\arabic}{enumi}{enumii}{}{.}%
\item {} 
领导能力:品牌经理必须是一个能驱动事情发生的领导者,虽然他领导的对象可能根本就不是他的下属,他需要驱动多个部门、甚至老板一起完成品牌目标。

\item {} 
分析能力:品牌经理必须是个发现机会的高手,分析中找到品牌的成长机会。

\item {} 
沟通与协调能力:品牌经理必须在口头和书面沟通上都具备突出的资质,以支持其行动。

\end{enumerate}


\subparagraph{银三角}
\label{\detokenize{chapter_idea/brand:id5}}
银三角=执行能力+创新能力+掌握专业技能的能力
\begin{enumerate}
\sphinxsetlistlabels{\arabic}{enumi}{enumii}{}{.}%
\item {} 
执行能力:想到还要做到,品牌经理要在关键时刻推动事情发生,自己动手+请别人动手,关键是看到结果。

\item {} 
创新能力:品牌经理不能因循守旧,要能够突破过去,不断创新。

\item {} 
掌握专业技能的能力:这个不用多说,品牌经理必须是专家+杂家的组合,专业范畴的市场研究、广告管理、促销管理、产品管理、品牌战略规划都是基本功,还需要汲取其它专业部门的经验和专业知识,才能全面发展。

\end{enumerate}


\paragraph{品牌资产 1\sphinxfootnotemark[392]}
\label{\detokenize{chapter_idea/brand:id6}}%
\begin{footnotetext}[392]\sphinxAtStartFootnote
\sphinxnolinkurl{http://reader.epubee.com/books/mobile/e2/e22be26cde02a62274cac6fa3d3c6fb5/text00006.html?fromPre=last}
%
\end{footnotetext}\ignorespaces 
品牌意识越强,消费者在选购产品时越会想到该品牌,最终购买该品牌的可能性也越大。
同时,品牌意识也会影响品牌联想与品牌形象的形成及其强度。

品牌资产的主要项目有:
\begin{enumerate}
\sphinxsetlistlabels{\arabic}{enumi}{enumii}{}{.}%
\item {} 
品牌意识(brand name awareness)

\item {} 
品牌忠诚(brand loyalty)

\item {} 
感知质量(perceived quality)

\item {} 
品牌联想(brand associations)

\end{enumerate}

\begin{figure}[H]
\centering
\capstart

\noindent\sphinxincludegraphics{{brand_asset}.jpg}
\caption{品牌资产}\label{\detokenize{chapter_idea/brand:id8}}\end{figure}


\subparagraph{品牌意识(brand name awareness) 2\sphinxfootnotemark[393]}
\label{\detokenize{chapter_idea/brand:brand-name-awareness-2}}%
\begin{footnotetext}[393]\sphinxAtStartFootnote
\sphinxnolinkurl{http://reader.epubee.com/books/mobile/e2/e22be26cde02a62274cac6fa3d3c6fb5/text00007.html}
%
\end{footnotetext}\ignorespaces 
指的是一个品牌在消费者心中的强度。假如在消费者心中布满了心理看板,每一个看板对应一个品牌的话,品牌在消费者心中意识的强弱就是对应看板的大小。品牌意识是根据消费者对一个品牌的不同的记忆方式进行测量的,从再认(以前曾见过这一品牌吗)到回忆(这类产品你能记起哪些品牌),再到“第一回忆”(第一个回忆出的品牌),最后到支配(唯一回忆出的品牌)。然而,心理学家和经济学家认为,再认和回忆不止是记得一个品牌的信号。

品牌再认(recognition)反映了从过去的接触中获得的熟悉度。再认不必记得在哪里见过该品牌,该品牌为什么与其他品牌有所不同,甚至不需要知道该品牌所属的产品类别。它只需要消费者记得曾经见过该品牌即可。


\paragraph{品牌定位 3\sphinxfootnotemark[394]}
\label{\detokenize{chapter_idea/brand:id7}}%
\begin{footnotetext}[394]\sphinxAtStartFootnote
\sphinxnolinkurl{https://www.zhihu.com/pub/reader/119980992/chapter/1284104652725256192}
%
\end{footnotetext}\ignorespaces 
「定位之父」特劳特提出「品牌定位是企业在市场定位和产品定位的基础上,对特定的品牌在文化取向及个性差异上的商业性决策,它是建立一个与目标市场有关的品牌形象的过程和结果。换言之,即为某个特定品牌确定一个适当的市场位置,使商品在消费者的心中占据一个特殊的位置,当某种需要突然产生时,比如在炎热的夏天突然口渴时,人们会立刻想到『可口可乐』的清凉爽口」。


\subsubsection{费米推理}
\label{\detokenize{chapter_idea/decompose:id1}}\label{\detokenize{chapter_idea/decompose::doc}}
费米解答问题的方式是推理思维,估算只是方法,对问题分解是推理的必然过程。这个思维过程就是——哪些真实存在的因素导致这样的事情发生?

就是在信息不完整的情况下,凭借对对象事物的深刻理解和洞察,科学地作出一些假设使得问题得以简化,复杂的程度得以降低,从而得到符合或接近实际的估计。

费米推论是将复杂、困难的问题分解成小的、可以解决的部分,从而以最直接的方法迅速解决问题。

我们将问题分解为可知和不可知的部分,对尚未确定的答案的部分一概视为未知,说出自己的假设并加以验证。大胆猜想,用于试错。宁愿迅速发现错误,也不要用模糊的措辞隐藏错误。对于“不可知的因素”,凭直觉和经验给出结果,最后,得出结论。


\subsubsection{转换成本1\sphinxfootnotemark[395]}
\label{\detokenize{chapter_idea/convert:id1}}\label{\detokenize{chapter_idea/convert::doc}}%
\begin{footnotetext}[395]\sphinxAtStartFootnote
\sphinxnolinkurl{http://www.woshipm.com/pmd/3431686.html}
%
\end{footnotetext}\ignorespaces 

\paragraph{社交货币}
\label{\detokenize{chapter_idea/convert:id2}}
社交货币也有分类,分为内部和外部,外部比较典型的就是利用物质与用户产生连接或者增加其社交价值。相对而言,用户喜欢贪一些小便宜,他的关系网里面就会聚集一些同类爱好的用户,这种情况的传播其实是在增加他的价值。

你的关系网里面没有这样的人,则需要谨慎分享。比如你的人脉圈里都是一些比较清高、爱面子的人,这时候外物货币就不太适合对这群人起作用。

这时候,你要做社交货币,就要从内部出发,比如打造荣耀、归属感、等级制度等等来自精神层面的内部货币。


\paragraph{核心存量}
\label{\detokenize{chapter_idea/convert:id3}}
每个产品都必然有自己的核心存量,也就是你给用户提供的核心价值,成本就是围绕核心价值来展开。

核心存量:当你的产品失去这个存量后,用户将会迅速流失。


\paragraph{连接度}
\label{\detokenize{chapter_idea/convert:id4}}
你加了多少个人\textasciicircum{}即社交关系的链接度,这里的连接度分为深度和宽度。

深度:你的通讯录里面有没有人跟你关系很好。
宽度:你的通讯录里面有多少个好友。

很多人都想着再出一个社交的app来打败微信,你去看看那些想要打败微信的社交软件,它们活的怎么样了?

腾讯很清楚,微信最大的威胁绝对不是同类社交app,而是其他领域的潜力巨无霸,比如:抖音,就是腾讯完全没有预料到的,让腾讯措手不及。

很多老板就是喜欢盯着对手的动作来展开行动,产品经理也很难办,这种情况的老板陷入了知识的诅咒,他们的焦点必然会盯着对手。他们甚至不懂产品,不懂运营,但是你懂啊,你要给他讲清楚这些道理,你甚至可以直接说“我比你专业,请听我的”,老板为啥花钱请你?因为你专业。


\subsubsection{商业思维}
\label{\detokenize{chapter_idea/business:id1}}\label{\detokenize{chapter_idea/business::doc}}

\paragraph{项目主人}
\label{\detokenize{chapter_idea/business:id2}}
产品经理的确有很多时候需要驱动设计、研发、测试、市场、营销等同事才能够完成整个产品的上线运营,但这应该占产品经理的40\%精力。产品经理还需要研究行业的变化,了解商业模式的变化,通过数据去看公司产品的\sphinxstylestrong{赢利}情况,找到可以改进的点。产品经理不仅是项目经理的角色,还是整个项目的\sphinxstylestrong{主人}。\sphinxhref{https://weread.qq.com/web/reader/46532b707210fc4f465d044k6ea321b021d6ea9ab1ba605}{12}%
\begin{footnote}[396]\sphinxAtStartFootnote
\sphinxnolinkurl{https://weread.qq.com/web/reader/46532b707210fc4f465d044k6ea321b021d6ea9ab1ba605}
%
\end{footnote}

利润 = 收入\sphinxhyphen{}成本\sphinxhyphen{}费用;


\subparagraph{目的}
\label{\detokenize{chapter_idea/business:id3}}
有些技术人天生有player的属性,所谓player,就是不满足于只做pawn的人。

技术只是手段,那手段用上了必然要有其目的。盈利?赚吆喝?布局?完成投资人的任务?Whatever
you name it.

所以不论是承担项目研发,技术负责,产品负责还是自己创业,有商业思维的人,总会比没有的人看问题深一层,跟他们聊天往往能感受到犀利,不同于纯技术思维的那种“愣”劲儿。
\sphinxhref{https://www.zhihu.com/question/348474416/answer/841775222}{3}%
\begin{footnote}[397]\sphinxAtStartFootnote
\sphinxnolinkurl{https://www.zhihu.com/question/348474416/answer/841775222}
%
\end{footnote}


\subparagraph{AI to B 要理解商业}
\label{\detokenize{chapter_idea/business:ai-to-b}}
\begin{figure}[H]
\centering
\capstart

\noindent\sphinxincludegraphics{{scientist_vs_businessman}.jpg}
\caption{科学家创业难以跨越的“死亡谷”难题}\label{\detokenize{chapter_idea/business:id29}}\end{figure}


\subparagraph{精益思想}
\label{\detokenize{chapter_idea/business:id4}}
用更少资源来做更多事情的方法
\begin{enumerate}
\sphinxsetlistlabels{\arabic}{enumi}{enumii}{}{.}%
\item {} 
快

\item {} 
流动产生价值,定期回顾需求价值

\item {} 
kiss原则。keep it simple,stupid。采取简单的方案

\item {} 
处在联系中的事物,才能被简化。纯粹地减少,不是简化。要将被简化的部分进行转移。

\item {} 
不害人的需求,不是完整的需求。veblen原理:无论多坏的改变,都会有一些人受益;无论多好的改变,都会有人受损。

\item {} 
化散乱为规律,化应急为预测。没有预测,就会疲于奔命,四处救火。

\item {} 
只可图示,不可言传。

\item {} 
让公路排满车,就是堵车。要将工作的焦点转移到重要不紧急的事情上去。

\item {} 
目标明确的战士,即使身陷重围,也会向着胜利而战斗。资源总是稀缺的,聚焦才能高效。

\item {} 
持续改进,不忘初心。小心“目标侵蚀”。

\item {} 
细节体现专业。对事物的不断细分,才能体现专业。深入到场景,最容易发现细分。

\item {} 
不要造永动机。不要只优化细节而忽略整体,永动机就是如此。

\item {} 
先准确,后精确。先把握需求的准确方向,再探寻正确的细节。

\end{enumerate}


\paragraph{商业}
\label{\detokenize{chapter_idea/business:id5}}
商业是一种有组织的提供顾客所需产品与服务的一种行为,以货币(包括电子货币)为媒介进行交换,\sphinxhref{https://zhuanlan.zhihu.com/p/25965712}{21}%
\begin{footnote}[398]\sphinxAtStartFootnote
\sphinxnolinkurl{https://zhuanlan.zhihu.com/p/25965712}
%
\end{footnote}它最本质的内容就是通过产品与服务交换实现盈利。


\subparagraph{商业切入口}
\label{\detokenize{chapter_idea/business:id6}}
最初小米公司只是做手机的,当手机做的不错的时候,他们发现可以基于自己IOT的研发能力,做其它智能产品;其它产品基于小米的IOT平台,能够实现研发,品牌,供应链等联动优势,最后形成一个基于智能硬件的智能网络生态,最大化的实现了产品的商业价值。


\paragraph{商业化}
\label{\detokenize{chapter_idea/business:id7}}
创业活动,不断创新的商业模式、线上线下的商业营销活动、除了教育等几个少数行业,大部分产业成熟的标志是实现商业化等等

光做出好产品是不够的,还要在市场上真的有价值,并且能持续保有竞争力。任何新技术都会随着时间的推移而扩散,一般所拥有的时间窗口最多也就是一年多的时间。

在这一段时间内,如何看待当前所面临的场景?在这个场景中技术到底占据多大的地位?
是非关键性的应用还是关键性的应用?技术上的突破和分配,是否产生根本性的问题?在技术的壁垒期,我们能否利用这一段时间构建起技术以外的壁垒?

只有壁垒构建出来,利用时间窗口期把技术优势转化成其他的竞争性壁垒,这样的行业才值得去做。\sphinxhref{https://www.infoq.cn/article/2017/12/Ground-practice-visual-AI}{15}%
\begin{footnote}[399]\sphinxAtStartFootnote
\sphinxnolinkurl{https://www.infoq.cn/article/2017/12/Ground-practice-visual-AI}
%
\end{footnote}


\paragraph{商业化idea}
\label{\detokenize{chapter_idea/business:idea}}
那些值得为之探索商业模式的idea应该源于创始人对某些事物长期思考和体会得到的一些不同寻常的见解。长期来看,
新的创业机会一定是技术创新引起的,
而商业化idea往往拼的是如何理解新技术给社会带来的变化。


\paragraph{商业模式 7\sphinxfootnotemark[400]}
\label{\detokenize{chapter_idea/business:id8}}%
\begin{footnotetext}[400]\sphinxAtStartFootnote
\sphinxnolinkurl{https://36kr.com/p/1721542885377}
%
\end{footnotetext}\ignorespaces 
商业模式就是指企业设计完整的商业逻辑,从而实现企业的生存价值。(《商业模式新生代》\sphinxhref{http://www.woshipm.com/pmd/3024508.html}{18}%
\begin{footnote}[401]\sphinxAtStartFootnote
\sphinxnolinkurl{http://www.woshipm.com/pmd/3024508.html}
%
\end{footnote})

设计合理的商业模式,就要充分考虑和运用企业运行的内外要素(包括商业前景,融资和资本运作\sphinxhref{https://coffee.pmcaff.com/article/2447262389384320/pmcaff?utm\_source=forum}{17}%
\begin{footnote}[402]\sphinxAtStartFootnote
\sphinxnolinkurl{https://coffee.pmcaff.com/article/2447262389384320/pmcaff?utm\_source=forum}
%
\end{footnote}),从而形成一个完整的高效率商业化运行系统,它可以保持产品独特的核心竞争力,并通过最优形式满足客户需求、实现自身价值,与此同时达成持续盈利的目标。
\sphinxhref{https://weread.qq.com/web/reader/40632860719ad5bb4060856kc0c320a0232c0c7c76d365a}{10}%
\begin{footnote}[403]\sphinxAtStartFootnote
\sphinxnolinkurl{https://weread.qq.com/web/reader/40632860719ad5bb4060856kc0c320a0232c0c7c76d365a}
%
\end{footnote}

至少包含了四个方面:产品模式、用户模式、推广模式,最后才是盈利模式。一句话,商业模式是你能提供一个什么样的产品,给什么样的用户创造什么样的价值,在创造用户价值的过程中,用什么样的方法获得商业价值。


\subparagraph{产品模式}
\label{\detokenize{chapter_idea/business:id9}}
所有的商业模式都要建立在产品模式的基础之上,没有了对产品和用户的思考,公司是不可能做大的,这样的公司注定也走不了多远。


\subparagraph{用户模式}
\label{\detokenize{chapter_idea/business:id10}}
你一定要找到对你的产品需求最强烈的目标用户。如果你说自己的产品是普世的产品,是放之四海而皆准的产品,这就说明你没有认真思考过。
\begin{itemize}
\item {} 
YY语音:帮助这些游戏工会在游戏对战中多对多沟通

\item {} 
UC手机浏览器:解决省流量的问题,因为当时手机流量很贵,网速慢且资费高

\end{itemize}


\subparagraph{推广模式}
\label{\detokenize{chapter_idea/business:id11}}
即使产品做得再好,如果只靠自然的口碑,只要还没接触到大多数目标用户,就有可能先被互联网巨头盯上。人家一模仿一捆绑,你多年的心血就算白费了。

真正的推广模式是要根据你的用户群,根据你的产品,去设计相应的推广方法。而不是拿钱去刷地铁、刷公交、刷路牌广告。

真正的推广是对产品的不断完善和提升。在推广的过程中,你要研究市场,和目标用户打交道,了解用户真正的需求,了解用户使用产品时遇到的困惑和问题,然后再反馈到产品上进行改进,由此不断帮助产品调整和完善。

\sphinxstylestrong{BD/销售}一个个地推式拉客户的方式能够简单粗暴地提升客户量,但这种方式自身最大的瓶颈便是时间长和边际成本线性增加。一个BD一个月拉5个客户,10个客户一个月最佳情况也就拉50个客户,同时商务的成本也扩大了10倍,在AI行业本来就人力成本高企的前提下,进一步提升了固定成本。

\sphinxstylestrong{采用与合作伙伴合作}的方式将AI落地,则是另外一种状态。基于共同利益,选择深耕落地行业多年的合作伙伴,一起服务好这个行业的客户。借助合作伙伴已有的资源和客户,相对自身开拓市场来讲,更为更为高效。同时也能有效抢占市场,取得竞争上的相对优势。所需要付出的,只是与合作伙伴共享收益。譬如视觉领域AI巨头拿一些政府的单子,并非自身去投标,而是与运营商背景的合作伙伴一起,共同准备投标,拿下单子,共同实施和分享收益。

在相对分散的中小型企业或组织细分领域,采用渠道合作相比BD/销售方式是一种投入产出比更为合理的方式。BD/销售搞定头部客户,中长尾市场便可以采用渠道合作的方式迅速占领。\sphinxhref{https://coffee.pmcaff.com/article/1593027702113408/pmcaff?utm\_source=forum}{16}%
\begin{footnote}[404]\sphinxAtStartFootnote
\sphinxnolinkurl{https://coffee.pmcaff.com/article/1593027702113408/pmcaff?utm\_source=forum}
%
\end{footnote}


\subparagraph{盈利模式}
\label{\detokenize{chapter_idea/business:id12}}
用户增长是一件很酷的事情,但是如果只追求数据的好看而不思索如何盈利,那么,寒冬之中倒下的很有可能就是这家企业。

Google的两个天才创始人做搜索引擎,好几年找不到赚钱的方法,只能是给雅虎这类的门户网站提供搜索技术服务来赚点糊口的钱。

Overture创造的付费点击模式,确实为广告客户创造了商业价值,但作为寄生于搜索引擎的企业,Overture却并没有为用户创造价值。反而是Google将搜索引擎的用户价值和Overture的付费点击模式完美地结合在了一起。


\subparagraph{AI VS 互联网创业20\sphinxfootnotemark[405]}
\label{\detokenize{chapter_idea/business:ai-vs-20}}%
\begin{footnotetext}[405]\sphinxAtStartFootnote
\sphinxnolinkurl{https://www.weiyangx.com/382066.html}
%
\end{footnotetext}\ignorespaces 
要知道,在这一波人工智能的大浪潮之前,2000年前后的那一拨互联网的浪潮中,我国的诸多互联网创业公司,包括百度、腾讯、阿里等都是参考硅谷等国外相对成熟的技术与商业模式创新,所以规模化、盈利的时间相对更快一些,商业价值的实现在比较短的时间就可以体现。

但这一次的AI浪潮明显不同,前沿理论、专用芯片、算法框架都需要从底层原创,还需要与行业和数据结合,而不同行业错综复杂,需求高度定制化,造成AI商业价值的落地的周期,一定是相对较长。


\paragraph{价值层面}
\label{\detokenize{chapter_idea/business:id13}}
BCG的价值3层面把商业模式分成了价值定位和价值传导2个大的层面,每个层面又包括3个小的具体模块,需要分别设计和规划。
\sphinxhref{https://weread.qq.com/web/reader/40632860719ad5bb4060856kc0c320a0232c0c7c76d365a}{10}%
\begin{footnote}[406]\sphinxAtStartFootnote
\sphinxnolinkurl{https://weread.qq.com/web/reader/40632860719ad5bb4060856kc0c320a0232c0c7c76d365a}
%
\end{footnote}


\subparagraph{商业模式画布 1\sphinxfootnotemark[407]}
\label{\detokenize{chapter_idea/business:id14}}%
\begin{footnotetext}[407]\sphinxAtStartFootnote
\sphinxnolinkurl{http://www.woshipm.com/pmd/2180363.html}
%
\end{footnotetext}\ignorespaces 
商业模式画布(BMC)是著名商业模式创新作家、商业顾问亚历山大·奥斯特瓦德在2008年提出的概念。

商业画布是一种能够帮助创业者催生创意、降低猜测、确保他们找对了目标用户合理解决问题的工具。

商业画布不仅能够提供更多灵活多变的计划,还更容易满足用户的需求。更重要的是它可以将商业模式中的元素标准化井强调元素间的相互作用。

\begin{center}\sphinxincludegraphics{{business_draw}.png}\end{center} \sphinxincludegraphics{{business_closed_loop}.png}
\begin{enumerate}
\sphinxsetlistlabels{\arabic}{enumi}{enumii}{}{.}%
\item {} 
客户细分(Customer
Segments):为谁服务?谁来买单?大众/小众市场、利基市场、区隔化市场、多元化市场、多边平台市场。

\item {} 
价值主张(Value
Propositions):服务或产品有什么价值?颠覆式创新、更快更好、个性定制、专注把事情做好、优秀的设计、价格优势、削减成本、抑制风险、连接、方便易用等特点。

\item {} 
渠道通路(Channels):认知、评估、购买、传递、售后;通路有:搜索引擎、公众平台、应用商店、线下资源等。

\item {} 
客户关系(Customer
Relationships):借助客户口碑传播获客从而维持持续收入

\item {} 
核心资源(Key
Resources):实体资产用户基数、知识产权、人力资源、金融资产、经营资质、用户基数

\item {} 
关键业务(Key
Activities):具体如何服务客户(驱动你做出产品、需求变化)

\item {} 
重要合作(Key
Partnerships):非竞争者之间的战略联盟关系、与竞争者之间的战略合作关系、为开发新业务而构建的合资关系、以及买卖关系。

\item {} 
收入来源(Revenue
Streams):售卖实体产品、使用权收费、租凭收费、“中介”收费、广告收费。

\item {} 
成本结构(Cost Structure):成本驱动型是越少越好

\end{enumerate}

不用纠结商业模式画布是不是最好的商业模式模型,只要将其作为商业模式设计入门的初步理解材料即可
\sphinxhref{https://www.zhihu.com/question/21472586s}{2}%
\begin{footnote}[408]\sphinxAtStartFootnote
\sphinxnolinkurl{https://www.zhihu.com/question/21472586s}
%
\end{footnote}

\begin{figure}[H]
\centering
\capstart

\noindent\sphinxincludegraphics{{weread_business_draw}.png}
\caption{微信读书的商业画布\sphinxhref{https://vickydyy.github.io/2019/05/26/Data-Business-Thought/}{22}\sphinxfootnotemark[409]}\label{\detokenize{chapter_idea/business:id30}}\end{figure}
%
\begin{footnotetext}[409]\sphinxAtStartFootnote
\sphinxnolinkurl{https://vickydyy.github.io/2019/05/26/Data-Business-Thought/}
%
\end{footnotetext}\ignorespaces 
工具:\sphinxhref{https://bms.your01.com/}{BMS(商业模式沙盘:Business Mode
Sandboxie)}%
\begin{footnote}[410]\sphinxAtStartFootnote
\sphinxnolinkurl{https://bms.your01.com/}
%
\end{footnote}


\subparagraph{多层次}
\label{\detokenize{chapter_idea/business:id15}}
将人工智能产品的规划、设计、实践与商业模式画布相结合,提出在产品、市场和效益3个层面进行人工智能产品设计,并列出了设计过程中涉及的14项指标。在这3个层面上,通过14项指标,产品经理可以轻松构建人工智能产品画布,如下图所示。人工智能产品画布可以帮助产品经理高效地确定产品规划、厘清产品脉络、确定产品结构,从而提升人工智能产品的设计效率。
\sphinxhref{https://weread.qq.com/web/reader/0c032c9071dbddbc0c06459k1c3321802231c383cd30bb3}{11}%
\begin{footnote}[411]\sphinxAtStartFootnote
\sphinxnolinkurl{https://weread.qq.com/web/reader/0c032c9071dbddbc0c06459k1c3321802231c383cd30bb3}
%
\end{footnote}

\begin{figure}[H]
\centering
\capstart

\noindent\sphinxincludegraphics{{business_cengci}.png}
\caption{多层次分析}\label{\detokenize{chapter_idea/business:id31}}\end{figure}
\begin{enumerate}
\sphinxsetlistlabels{\arabic}{enumi}{enumii}{}{.}%
\item {} 
\sphinxstylestrong{产品层面}:产品层面主要包括一些产品实现的细节:一是人工智能产品的实现方案,包括产品目标、范围、可行性及关键功能架构;二是人工智能产品应用的具体行业和场景,并确定该产品在该场景下实现的效能指标和价值指标;三是实现人工智能产品的技术选型、算法分析和技术指标设计等。

\item {} 
\sphinxstylestrong{市场层面}:人工智能产品是否成功关键在于产品是否可以获得市场的认可。即使产品非常优秀,如果没有被市场和客户认可,一切投入也都将化为乌有。产品经理在产品规划过程中应从市场层面完成针对产品使用者、购买者、影响者、决策者等的客群分析,完成竞争对手分析,完成产品定价策略规划,以及完成渠道规划。市场层面产品规划最关键的内容是确定产品价值主张。产品价值主张不仅指明了产品方向,而且关系到产品的成败。产品价值主张包括产品带来的社会价值、生产力价值等。

\item {} 
\sphinxstylestrong{效益层面}:在一个产品概念创立之初,需要建立人工智能产品的效益目标,效益目标可以从经济效益、社会效益等不同层面进行考量,作为产品经理要着重对产品的成本和收入进行分析。产品经理应对效益目标进行拆解,计算投入产出比,如果投入产出比不够理想,则研发该产品没有意义。

\end{enumerate}


\paragraph{交易模型}
\label{\detokenize{chapter_idea/business:id16}}
以交易为基本单元来研究产品,目标是建立可持续交易的互惠模型


\subparagraph{企业、用户、产品关系}
\label{\detokenize{chapter_idea/business:id17}}
用户选择产品:效用\sphinxhyphen{}成本>0:
\begin{itemize}
\item {} 
直接成本:付出的时间、金钱、数据、态度等

\item {} 
间接成本:为了促成交易,付出的搜寻成本

\end{itemize}

企业生产产品:收益\sphinxhyphen{}成本>0
\begin{itemize}
\item {} 
收益:现金收入、增加未来收益的各方信任、品牌声誉等

\end{itemize}


\subparagraph{效用(欲望的满足程度)的三个属性}
\label{\detokenize{chapter_idea/business:id18}}\begin{itemize}
\item {} 
多样性:时间、欲望、心里感觉、情绪、认知

\item {} 
无限性:需求永远无法被完全满足,因为需求是会变得越来越大的

\item {} 
个体性:人会受到情境、禀赋、偏好、认知等影响,所以同一个产品带来的效用,对于不同的人来说差距很大,信息的完全性及原有的思维框架会影响每个人对效用的判断

\end{itemize}


\subparagraph{交易成本}
\label{\detokenize{chapter_idea/business:id19}}
交易成本:完成一笔交易时,交易双方在买卖前后所产生的各种与此交易相关的成本。也可以理解为”所有买方(卖方)付出但是卖方(买方)没有收到的成本。


\subparagraph{分类 19\sphinxfootnotemark[412]}
\label{\detokenize{chapter_idea/business:id20}}%
\begin{footnotetext}[412]\sphinxAtStartFootnote
\sphinxnolinkurl{http://www.woshipm.com/pmd/3402762.html}
%
\end{footnotetext}\ignorespaces \begin{enumerate}
\sphinxsetlistlabels{\arabic}{enumi}{enumii}{}{.}%
\item {} 
搜寻(商品和交易对象)成本和度量(交易对象和商品的属性)成本;

\item {} 
寻价(议价比价)成本和决策(决策和订立契约)成本;

\item {} 
实施成本和保障(权利、违约、意外、监督等)成本。

\end{enumerate}


\subparagraph{搜寻成本与度量成本}
\label{\detokenize{chapter_idea/business:id21}}\begin{itemize}
\item {} 
认知困难:要找的SKU到底在哪个导航里呢?

\item {} 
负面预期:这个ICON可以点吗,点了会不会出问题?

\item {} 
度量困难:这家餐厅装修看着不错,到底好不好吃?大众点评的评分可信吗?

\end{itemize}


\subparagraph{寻价成本和决策成本}
\label{\detokenize{chapter_idea/business:id22}}\begin{itemize}
\item {} 
供给不足导致的排队等待:软件园高峰期,在不加价的情况下,需要排队近1个小时才能打到车。

\item {} 
线下的议价流程:在线上租房信息平台达成意向后,仍需要线下与房东再面对面议价,签署合同。

\item {} 
商业化带来的决策延迟:每次打开APP,弹窗广告都要强制展示5min以上。

\end{itemize}


\subparagraph{实施成本和保障成本}
\label{\detokenize{chapter_idea/business:id23}}\begin{itemize}
\item {} 
冗余的操作:每次微信AA账单需要打开“钱包\sphinxhyphen{}收付款\sphinxhyphen{}群收款”然后发到群里,而不能直接在群聊里发起。

\item {} 
中断与重复:每次公众号文章看到一半退出回消息,要重新看需要打开公众号再把文章搜索出来(浮窗功能出来之前..)。

\item {} 
信任危机:害怕低价促销的产品质量没有保障,不敢买(在“7天无理由退货”推出前的困境)。

\end{itemize}

降低交易成本
\begin{itemize}
\item {} 
标准化:把供给品尽量变成标准品,降低了度量成本,降低了不确定性带来的决策成本和保障成本

\item {} 
线上化:降低了企业与用户触发、服务、维护等成本

\end{itemize}


\paragraph{三级火箭 9\sphinxfootnotemark[413]}
\label{\detokenize{chapter_idea/business:id24}}%
\begin{footnotetext}[413]\sphinxAtStartFootnote
\sphinxnolinkurl{https://www.jianshu.com/p/ff38ced05cbd}
%
\end{footnotetext}\ignorespaces 
互联网商业就是产品、流量、转化率三个词。

第一级:搭建高频头部流量 第二级:沉淀某类用户的商业场景
第三级:完成商业闭环


\subparagraph{例子}
\label{\detokenize{chapter_idea/business:id25}}
360的一级火箭是免费杀毒工具;二级火箭是从免费杀毒工具变为网络安全平台(360安全浏览器、360安全网址导航);三级火箭就是它最终承载的商业闭环(从安全浏览器和网址导航的广告收入)。

搜狗现在的一级火箭是来自腾讯的头部流量;二级火箭是内置搜索,通过庞大的使用场景去释放更多搜索的需求。三级火箭即商业变现。

逻辑思维第一级火箭是罗振宇坚持了多年的免费脱口秀;第二级火箭是得到APP,沉淀用户的商业场景;第三级火箭,得到APP里面的付费课程。

小米的一级火箭是手机;二级火箭是一系列的零售场景(小米商城、小米之家、小米小店);三级火箭是一个高利润的产品。

你要赚利润的东西,并非是他人要赚钱的地方。面对这样的竞争者,传统的生意套路会失效。你以为亚朵在做酒店,其实亚朵在做社群共创的实景电商。


\subparagraph{必要条件}
\label{\detokenize{chapter_idea/business:id26}}\begin{enumerate}
\sphinxsetlistlabels{\arabic}{enumi}{enumii}{}{.}%
\item {} 
三级火箭递推一定是高频推低频。

\item {} 
通过一级火箭获得大量用户之后,要快速开展一个能够沉淀用户的商业场景。

\item {} 
操控三级火箭的人,一定是个势能积累到一定程度的人。(首先要有强大的融资能力;其次在头部流量铺开的时候要有短时间聚拢资源的能力)

\item {} 
操盘三级火箭的人一定是个狠人。(一级火箭就是抢别人流量,要能够承受他人指责)

\end{enumerate}


\subparagraph{原理}
\label{\detokenize{chapter_idea/business:id27}}
火箭级数越多,需要的燃料越少。但每增加一级,不可控程度越高。就像做商业,模型过于复杂,变现链条过长,就容易玩脱。

所以,三级火箭是一个成本和可控性平衡后的选择。


\subparagraph{AI产品的商业化}
\label{\detokenize{chapter_idea/business:ai}}\begin{itemize}
\item {} 
基于企业服务费的商业路径:参照行业内对手的收费模式,是按单收费,还是按配置收费
(朝头部客户去做,有大订单,投入产出比高,eg:金融领域千万订单)

\item {} 
基于互联网玩法的商业路径:小度音箱的模式,先近于免费抢占市场,后割韭菜。(SAAS服务一视同仁,找代理铺量)

\end{itemize}

TODO:\sphinxhref{https://radiant-brushlands-42789.herokuapp.com/medium.com/predict/choosing-the-right-ai-business-model-df5d81420d74}{14}%
\begin{footnote}[414]\sphinxAtStartFootnote
\sphinxnolinkurl{https://radiant-brushlands-42789.herokuapp.com/medium.com/predict/choosing-the-right-ai-business-model-df5d81420d74}
%
\end{footnote}

\sphinxhref{https://github.com/scutan90/DeepLearning-500-questions/blob/master/ch19\_\%E8\%BD\%AF\%E4\%BB\%B6\%E4\%B8\%93\%E5\%88\%A9\%E7\%94\%B3\%E8\%AF\%B7\%E5\%8F\%8A\%E6\%9D\%83\%E5\%88\%A9\%E4\%BF\%9D\%E6\%8A\%A4/\%E7\%AC\%AC\%E5\%8D\%81\%E4\%B9\%9D\%E7\%AB\%A0\_\%E8\%BD\%AF\%E4\%BB\%B6\%E4\%B8\%93\%E5\%88\%A9\%E7\%94\%B3\%E8\%AF\%B7\%E5\%8F\%8A\%E6\%9D\%83\%E5\%88\%A9\%E4\%BF\%9D\%E6\%8A\%A4.md}{软件(算法)专利保护}%
\begin{footnote}[415]\sphinxAtStartFootnote
\sphinxnolinkurl{https://github.com/scutan90/DeepLearning-500-questions/blob/master/ch19\_\%E8\%BD\%AF\%E4\%BB\%B6\%E4\%B8\%93\%E5\%88\%A9\%E7\%94\%B3\%E8\%AF\%B7\%E5\%8F\%8A\%E6\%9D\%83\%E5\%88\%A9\%E4\%BF\%9D\%E6\%8A\%A4/\%E7\%AC\%AC\%E5\%8D\%81\%E4\%B9\%9D\%E7\%AB\%A0\_\%E8\%BD\%AF\%E4\%BB\%B6\%E4\%B8\%93\%E5\%88\%A9\%E7\%94\%B3\%E8\%AF\%B7\%E5\%8F\%8A\%E6\%9D\%83\%E5\%88\%A9\%E4\%BF\%9D\%E6\%8A\%A4.md}
%
\end{footnote}


\paragraph{阿里云视觉智能开放平台 4\sphinxfootnotemark[416] 5\sphinxfootnotemark[417] 6\sphinxfootnotemark[418]}
\label{\detokenize{chapter_idea/business:id28}}%
\begin{footnotetext}[416]\sphinxAtStartFootnote
\sphinxnolinkurl{https://help.aliyun.com/document\_detail/143096.html?spm=a2c4g.11186623.6.548.1a4a53cblCY4Zg}
%
\end{footnotetext}\ignorespaces %
\begin{footnotetext}[417]\sphinxAtStartFootnote
\sphinxnolinkurl{https://developer.aliyun.com/article/778839?spm=a2c6h.12873581.0.dArticle778839.5de439932BzTaX\&groupCode=viapi}
%
\end{footnotetext}\ignorespaces %
\begin{footnotetext}[418]\sphinxAtStartFootnote
\sphinxnolinkurl{https://help.aliyun.com/document\_detail/182962.html?spm=a211p3.14020179.J\_7524944390.13.738f4b58g1fD6Y}
%
\end{footnotetext}\ignorespaces 
商业化提供了预付费QPS、后付费、预付费资源包、按量付费四种收费模式。

离线SDK介绍:阿里云视觉智能开放平台的离线SDK可以为终端设备提供AI能力,目前支持提供OCR、美颜、分割等常用AI能力的离线SDK。阿里云视觉智能开放平台通过license授权方式管理离线SDK。

准备工作:在安装和使用阿里云SDK前,确保您已经注册阿里云账号并生成访问密钥(AccessKey)。详情请参见创建AccessKey。


\subsubsection{批判性思维1\sphinxfootnotemark[419]}
\label{\detokenize{chapter_idea/critical:id1}}\label{\detokenize{chapter_idea/critical::doc}}%
\begin{footnotetext}[419]\sphinxAtStartFootnote
\sphinxnolinkurl{http://www.woshipm.com/pmd/284339.html}
%
\end{footnotetext}\ignorespaces 
《学会提问:批判性思维指南》(尼尔·布朗)

信息大爆炸时代,我们每天都会接收到很多的观点,批判性的去阅读变得异常重要。本书提供了一个批判思维的完整结构

什么是论题?什么是结论? 理由是什么? 哪些词语有歧义?
什么是价值观冲突?什么是价值观假设? 什么是描述性假设?
推理中存在谬误吗? 这些证据的可信度有多大? 你发型干扰性原因了吗?
统计数据是否具有欺骗性? 哪些重要数据被遗漏了? 什么结论可能是合理的?


\paragraph{经验和方法 3\sphinxfootnotemark[420]}
\label{\detokenize{chapter_idea/critical:id2}}%
\begin{footnotetext}[420]\sphinxAtStartFootnote
\sphinxnolinkurl{https://www.iamxiarui.com/?p=1416}
%
\end{footnotetext}\ignorespaces 

\subparagraph{经验和方法的区别}
\label{\detokenize{chapter_idea/critical:id3}}\begin{itemize}
\item {} 
经验看似复杂实则简单,学习成本低,易学易用见效快,但是会随着环境的变化而很快过时

\item {} 
方法看似简单实则复杂,极其抽象不易理解和学习,但灵活,泛用性广,不易过时

\end{itemize}


\subparagraph{经验和标准的关系}
\label{\detokenize{chapter_idea/critical:id4}}\begin{itemize}
\item {} 
AI 领域没有经验和标准,\sphinxstylestrong{谁活下来了,谁就是经验和标准}

\item {} 
掌握方法,可以利用自身优势创造经验

\item {} 
分辨经验和方法:看是否能跨场景使用

\item {} 
标准是成功经验的产物

\end{itemize}


\subparagraph{用具有方向的思考方式取代猜测}
\label{\detokenize{chapter_idea/critical:id5}}\begin{itemize}
\item {} 
依靠方法,并不是依靠经验

\item {} 
结合之前的经验划定一个思考的范围

\item {} 
在这个范围内猜测,并通过猜测不断缩小范围

\end{itemize}


\paragraph{知道与学会 2\sphinxfootnotemark[421]}
\label{\detokenize{chapter_idea/critical:id6}}%
\begin{footnotetext}[421]\sphinxAtStartFootnote
\sphinxnolinkurl{https://www.iamxiarui.com/?p=1416}
%
\end{footnotetext}\ignorespaces \begin{itemize}
\item {} 
半对半错比全错还可怕

\item {} 
碎片化学习需要有将所有碎片整合的能力,如果没有反而影响认知

\item {} 
信息利用的两要素:获取成本和鉴别成本

\end{itemize}

产品经理一定要具备的能力:
\begin{itemize}
\item {} 
精确定义身边所有问题

\item {} 
定义的标准是可以在任何一个场景可以复现

\item {} 
知道是了解,学会是可复制

\end{itemize}


\subsubsection{注意力}
\label{\detokenize{chapter_idea/attention:id1}}\label{\detokenize{chapter_idea/attention::doc}}
小便池里假苍蝇,溅出率降低60\%。
\begin{itemize}
\item {} 
微博的小红点、发布(橙色)

\item {} 
淘宝里的立即购买(红色)、加入购物车(橙色)

\item {} 
分类导航上下差5\sphinxhyphen{}10\%

\end{itemize}


\paragraph{AIDA模型}
\label{\detokenize{chapter_idea/attention:aida}}
AIDA模式也称“爱达”公式,是艾尔莫·李维斯Elmo Lewis
在1898年首次提出总结的推销模式,是西方推销学中一个重要的公式,它的具体函义是指一个成功的推销员必须把顾客的注意力吸引或转变到产品上,使顾客对推销人员所推销的产品产生兴趣,这样顾客欲望也就随之产生,尔后再促使采取购买行为,达成交易。

AIDA是四个英文单词的首字母:A为Attention,即引起注意;I为Interest,即诱发兴趣;D为Desire,即刺激欲望;最后一个字母A为Action,即促成购买。

AIDA模型可以看作是用户做购买决策的整个行为过程,从引起注意到最终的购买行为的过程。

尽管AIDA模型是为线下推销场景提出的,但放在线上电商鼎盛的今天,依旧有很大的参考价值,下面,我们就来分析一下现有的线上产品是如何运用AIDA模型来辅助用户购买决策的。


\subsubsection{Goal}
\label{\detokenize{chapter_idea/goal:goal}}\label{\detokenize{chapter_idea/goal::doc}}
目标的意义:
\begin{itemize}
\item {} 
可以将模糊的授意或导向具体成可执行、可拆解甚至是可衡量的目标;

\item {} 
评估项目的可行性,标识出关键环节和风险环节;

\item {} 
统一项目所有成员对工作方向或价值的认知,保证全体成员同时朝一个方向努力;

\item {} 
将项目成员之间的工作进行串联,形成协同效应。

\end{itemize}


\paragraph{OKR目标管理法 6\sphinxfootnotemark[422]}
\label{\detokenize{chapter_idea/goal:okr-6}}%
\begin{footnotetext}[422]\sphinxAtStartFootnote
\sphinxnolinkurl{https://www.toutiao.com/a6643946609216324109/s}
%
\end{footnotetext}\ignorespaces 
目前,国内外常用的目标管理方法是由英特尔发明,在Uber、谷歌、LinkedIn等公司成功推广的OKR,即目标与关键成果法。

OKR是一套严密的思考框架和持续的纪律要求,旨在确保员工紧密协作,把精力聚焦在能促进组织成长的、可衡量的贡献上。

使用OKR做目标管理主要分为三个阶段:目标制定、落地执行、结果评估。
\begin{itemize}
\item {} 
目标(O)回答的是“我们想做什么”的问题,是定性的;

\item {} 
关键结果(KR)回答的是“我们如何知道自己是否达成了目标要求”,是定量的。

\end{itemize}


\subparagraph{目标制定}
\label{\detokenize{chapter_idea/goal:id1}}
企业可参考长远目标,将整体经营过程,按照工作性质的不同、时间安排的先后次序,分解成若干相互关联的阶段,分阶段逐步实现长期目标。

目标先行,然后将目标自上而下拆解,将项目与目标对应,不同级别的目标对应不同的项目(一个目标可对应多个项目),最后确保每个项目都有明确的目标。


\subparagraph{满足SMART原则}
\label{\detokenize{chapter_idea/goal:smart}}
Specific(明确性):目标必须是明确的,不能是模棱两可或含糊不清的,比如“优化客户服务意识”就不是一个明确的目标。
Measurable(可衡量):关键结果必须是可衡量的,可用于衡量的方法有:基线法、里程碑法、正向度量法、负向度量法等,如“用户留存时间从60分钟提升为80分钟”就是一个可衡量的关键结果。
Attainable(可实现):OKR鼓励在设定目标时具有一定的野心,但也要考虑可实现的,不能天马行空设定一个无法实现的无意义目标。
Relevant(相关性):公司级目标要跟公司的战略对齐,部门级目标要跟公司目标对齐,个人目标要跟部门目标对齐,这样才能确保全员目标聚焦。
Time\sphinxhyphen{}bound(时限性):没有时间限制,目标的制定就失去了意义,在OKR实施中,时限性体现在周期的设定上。


\subparagraph{DUMB}
\label{\detokenize{chapter_idea/goal:dumb}}\begin{itemize}
\item {} 
切实可行 (Doable)

\item {} 
易于理解 (Understandable)

\item {} 
可干预可管理 (Manageable)

\item {} 
正向的有益的 (Beneficial)

\end{itemize}


\subparagraph{More}
\label{\detokenize{chapter_idea/goal:more}}
可优化性,并能迅速解决现有的问题,并且无论是短期需求还是长期发展,此方案都可应对。


\subparagraph{落地执行}
\label{\detokenize{chapter_idea/goal:id2}}
目标的具体执行需要落地为一个个具体的项目,每个项目都会包含若干任务,因此,目标的最后执行是在任务上,只是我们将具有相同目标的任务聚合在一起,放在同一个项目中进行管理。


\subparagraph{结果评估}
\label{\detokenize{chapter_idea/goal:id3}}
每个周期末,全员需要对本周期的目标完成情况总结、评估并打分。
\begin{itemize}
\item {} 
周例会:每周例会评估本周目标的进展情况,以及关键结果的风险状态。

\item {} 
季度中期审视:要确保目标在季度结束时完成,建议在季度中期对目标进度进行审视与评估,以便今早的找到可能存在的风险及解决方案。

\item {} 
季度末评估:在季度末的评估会议上,需要回顾这一季度目标的完成情况及最终的评分情况,最佳的得分应该在0.7分上下。在这个评估会议上需要回答好两个问题“做到什么程度”和“如何做到这个程度的”。

\end{itemize}

负责人对目标完成情况打分可能只需要几分钟,但是更多的工作应该是在员工大会上:
\begin{itemize}
\item {} 
集体讨论以及回顾本周期目标的执行情况;

\item {} 
激励有野心的,优秀的成员;

\item {} 
对执行过程中的问题进行汇总;

\item {} 
为新一周期的目标制定提供参考。

\end{itemize}


\paragraph{产品经理的工作目标 3\sphinxfootnotemark[423]}
\label{\detokenize{chapter_idea/goal:id4}}%
\begin{footnotetext}[423]\sphinxAtStartFootnote
\sphinxnolinkurl{https://weread.qq.com/web/reader/46532b707210fc4f465d044k341323f021e34173cb3824c}
%
\end{footnotetext}\ignorespaces 
产品经理的工作目标是帮助用户创造价值,帮助公司创造商业价值,而不是把工作定位成编写产品文档,画产品交互设计图。


\subparagraph{太关注产出}
\label{\detokenize{chapter_idea/goal:id5}}
把产品带来的产出排在第一位,比如做这个产品功能能带来多少新用户,做那个产品功能能带来多少消费用户的转化等。


\subparagraph{感知用户能力弱}
\label{\detokenize{chapter_idea/goal:id6}}
如果一个产品经理只是在工作的时候讨论需求,在回到家后只是在产品有故障时才关注一下,那么肯定没有大的发展,即便已经做了组长。因为他缺少对用户的感知,缺少对产品做到极致的追求,缺少发现产品亮点的眼睛,把太多时间放到了生活、娱乐,但是又没有从中锻炼出用户同理心、对用户需求的敏感度,只能谈一些似是而非的产品理念,不能够深究。


\subparagraph{沟通能力差}
\label{\detokenize{chapter_idea/goal:id7}}
如果一个产品组是用表格写产品需求的,那么说明这个产品组的层次非常低,这并不是文档格式的问题,而是一个团队的产品文化有问题。产品经理的工作绝对不只是把粗糙的交互设计图放在表格的左边,随便写几个产品流程的判断条件就可以交给开发同事了。

很多语句读起来不通顺,只有在产品需求评审会上逐一讲解,其他人才能理解,大大增加了沟通成本。产品经理要能够讲清楚技术方面的前置条件和后置条件,还要能够讲清楚这个页面的主要功能诉求、最希望用户怎么操作、是怎样通过交互设计和文案设计引导用户的(包括如何引导用户的下一步动作,比如图1\sphinxhyphen{}9显示在用户获得奖品后,只有一个“确认”按钮,并且让用户去分享)。在产品文档中还需要加入更多的异常流设计。


\paragraph{清晰的学习的目标}
\label{\detokenize{chapter_idea/goal:id8}}\begin{itemize}
\item {} 
可行性:这个级别的产品经理要能对一个需求或问题给出高可行性的解决方案,大约是市场上的P6级,小领域的熟练执行者

\item {} 
创造:要能为一个需求或问题找到最优解。这需要不断洞察环境、用户的持续变化和趋势

\item {} 
权衡:能跳出单个需求,从全局角度考虑权衡取舍。即使一个需求为真、可行、有最优解,最重要的还是决定当下要不要做,分配多少资源做

\item {} 
变迁:能跳出当下,对世事变迁敏感,对需求或问题做决策时,形成习惯性地预判(并反思自己的判断是否正确)能力

\item {} 
方法论:这个级别要求有成体系的优质方法论输出。这个层级主要是追求“影响力”和“确定性”

\end{itemize}


\paragraph{AI 产品的目标}
\label{\detokenize{chapter_idea/goal:ai}}
目标不应该是与领先的互联网公司竞争,而是为你的垂直行业部门赋能人工智能。
\sphinxhref{https://hbr.org/2019/02/how-to-choose-your-first-ai-project}{5}%
\begin{footnote}[424]\sphinxAtStartFootnote
\sphinxnolinkurl{https://hbr.org/2019/02/how-to-choose-your-first-ai-project}
%
\end{footnote}


\paragraph{no code1\sphinxfootnotemark[425]}
\label{\detokenize{chapter_idea/goal:no-code1}}%
\begin{footnotetext}[425]\sphinxAtStartFootnote
\sphinxnolinkurl{https://github.com/cedrickchee/knowledge/blob/master/courses/fast.ai/deep-learning-part-1/2019-edition/lesson-7-resnet-unet-gan-rnn.md}
%
\end{footnotetext}\ignorespaces 
So as long we need to code, we failed that because 99.8\% of the world
can’t code. The main goal would be to get to a point where it’s not a
library but a piece of software that doesn’t required code. It certainly
shouldn’t require a lenghty hardworking course like this one. I want to
get rid of the course, get rid of the code. I want to make it so you can
do usual stuff quickly and easily. That’s may be 5 years, may be longer.


\subsubsection{技术理解力}
\label{\detokenize{chapter_idea/understand_tech:id1}}\label{\detokenize{chapter_idea/understand_tech::doc}}

\paragraph{为何需要}
\label{\detokenize{chapter_idea/understand_tech:id2}}
在传统意义上,产品经理根据需求特性,抽象产品特性,思考产品逻辑,制定产品目标、愿景、实施计划,拟定详细的文档,按照交互\sphinxhyphen{}设计\sphinxhyphen{}重构\sphinxhyphen{}前后台开发\sphinxhyphen{}测试验收上线的流程,一步步推进即可。但看似合理且被大多数产品经理认为是理所当然的流程中,却隐藏着技术理解层面严重的bug。


\paragraph{技术理解的作用 14\sphinxfootnotemark[426]}
\label{\detokenize{chapter_idea/understand_tech:id3}}%
\begin{footnotetext}[426]\sphinxAtStartFootnote
\sphinxnolinkurl{https://www.zhihu.com/question/57815929/answer/1338813523}
%
\end{footnotetext}\ignorespaces 

\subparagraph{了解层次}
\label{\detokenize{chapter_idea/understand_tech:id4}}\begin{enumerate}
\sphinxsetlistlabels{\arabic}{enumi}{enumii}{}{.}%
\item {} 
形成缜密的逻辑思维:PRD方案的完备要求,如果不完备,可能开发结构大大不同,造成工期延误返工的现象。
\sphinxhref{https://www.zhihu.com/search?type=content\&q=\%E4\%BA\%A7\%E5\%93\%81+\%E7\%AC\%94\%E8\%AF\%95}{22}%
\begin{footnote}[427]\sphinxAtStartFootnote
\sphinxnolinkurl{https://www.zhihu.com/search?type=content\&q=\%E4\%BA\%A7\%E5\%93\%81+\%E7\%AC\%94\%E8\%AF\%95}
%
\end{footnote}
也要信任研发经理评估出来的工期
\sphinxhref{https://www.zhihu.com/question/19554113/answer/308056760}{25}%
\begin{footnote}[428]\sphinxAtStartFootnote
\sphinxnolinkurl{https://www.zhihu.com/question/19554113/answer/308056760}
%
\end{footnote}

\item {} 
算法的适用场景:知道要做什么,风险是什么\sphinxhyphen{}走哪条路能到。技术给产品带来多少价值\sphinxhref{https://zhuanlan.zhihu.com/p/33524676}{19}%
\begin{footnote}[429]\sphinxAtStartFootnote
\sphinxnolinkurl{https://zhuanlan.zhihu.com/p/33524676}
%
\end{footnote};解决方案转化到技术的资源需求(需要更多的数据,更完善的算法模型
\sphinxhref{https://zhuanlan.zhihu.com/p/33524676}{19}%
\begin{footnote}[430]\sphinxAtStartFootnote
\sphinxnolinkurl{https://zhuanlan.zhihu.com/p/33524676}
%
\end{footnote}),缩短研发工程师找到最佳技术方案的时间
\sphinxhref{http://www.changgpm.com/thread-356-1-1.html}{16}%
\begin{footnote}[431]\sphinxAtStartFootnote
\sphinxnolinkurl{http://www.changgpm.com/thread-356-1-1.html}
%
\end{footnote};考虑风险,不完全依赖AI,而是AI辅助\sphinxhref{https://medium.com/@liwdai/how-machine-learning-influences-responsibilities-of-product-managers-bf63c3bf57b5}{20}%
\begin{footnote}[432]\sphinxAtStartFootnote
\sphinxnolinkurl{https://medium.com/@liwdai/how-machine-learning-influences-responsibilities-of-product-managers-bf63c3bf57b5}
%
\end{footnote},如人脸识别可能被攻击,用“异步审核”策略,用人工检测的方式保证通过率和准确率,保证用户体验,降低业务风险。

\item {} 
技术边界、与技术团队的能力边界、技术成熟度:知道什么能做,什么不能做,技术最好的情况是如何的。——路可能有障碍但能通。开发部门可接受,使得业务可落地。哪些理论已经有了最佳实践。做法:和ML
Engineers每周一起讨论AI/ML论文,研究应用到我们产品上的可行性。\sphinxhref{https://medium.com/3pm-lab/3-major-differences-of-being-a-product-manager-in-big-companies-vs-startups-36861e35c5e3}{23}%
\begin{footnote}[433]\sphinxAtStartFootnote
\sphinxnolinkurl{https://medium.com/3pm-lab/3-major-differences-of-being-a-product-manager-in-big-companies-vs-startups-36861e35c5e3}
%
\end{footnote}

\item {} 
权衡技术方案(了解整个 AI
端到端产品的工具和技术\sphinxhref{http://www.uml.org.cn/devprocess/201910163.asps}{24}%
\begin{footnote}[434]\sphinxAtStartFootnote
\sphinxnolinkurl{http://www.uml.org.cn/devprocess/201910163.asps}
%
\end{footnote}):知道什么好做,什么不好做。——路上的障碍物好不好处理。使得业务能按时完成。也即,你知道系统中的每个模块的定位和意义(数据摄入工具(比如
Kafka)、数据处理系统(比如 Spark)以及 处理大数据的 NoSQL DBMS(比如
Cassandra)、云服务(利用 AWS、GCP、IBM 和
Azure)),并有能力\sphinxstylestrong{以业务需求为导向}协助技术人员、甚至引导技术人员完成对系统架构的优化与改造,使其在未来能够更好的满足业务发展对于技术的要求。

\item {} 
业务长时间可用性:知道什么该做,什么不该做。——这些障碍物究竟该怎样处理,才能让它们在最长的时间范围内不会成为干扰。当你与技术人员合作设计方案的时候,应该从业务发展的角度看待问题,帮助技术人员明确各个模块的定位,使得我们的系统能够在尽可能长的时间保证可用性,能够随着业务的发展\sphinxstylestrong{一同成长,而不是频繁重构}。

\end{enumerate}


\subparagraph{懂算法的程度 20\sphinxfootnotemark[435]}
\label{\detokenize{chapter_idea/understand_tech:id5}}%
\begin{footnotetext}[435]\sphinxAtStartFootnote
\sphinxnolinkurl{https://medium.com/@liwdai/how-machine-learning-influences-responsibilities-of-product-managers-bf63c3bf57b5}
%
\end{footnotetext}\ignorespaces 
以监督式支持向量机SVM为例,AI产品经理懂的SVM内容建议如下:

\sphinxstylestrong{首先:明白SVM模型成熟的用途。}

例如:
\begin{itemize}
\item {} 
其一用于文本和超文本的分类,在归纳和直推方法中都可以显著减少所需要的有类标的样本数。

\item {} 
其二用于图像分类。支持向量机能够获取明显更高的搜索准确度。

\item {} 
其三用于手写字体识别。

\item {} 
其四用于医学中分类蛋白质,超过90\%的化合物能够被正确分类。

\item {} 
其它则靠AI产品经理配合SVM算法模型专家共同探讨研究。

\end{itemize}

\sphinxstylestrong{其次:知晓SVM的定义及核函数表达式}

支持向量机在高维或无限维空间中构造超平面或超平面集合,其可以用于分类、回归或其他任务。直观来说,分类边界距离最近的训练数据点越远越好,因为这样可以缩小分类器的泛化误差。

SVM的原始问题是在有限维空间中陈述的,但用于区分的集合在该空间中往往线性不可分。为此,有人提出将原有限维空间映射到维数高得多的空间中,在该空间中进行分离可能会更容易。为了保持计算负荷合理,人们选择适合该问题的核函数
k(x,y)
来定义SVM方案使用的映射,以确保用原始空间中的变量可以很容易计算点积。高维空间中的超平面定义为与该空间中的某向量的点积是常数的点的集合。

定义超平面的向量可以选择在数据基中出现的特征向量Xi的图像的参数ai的线性组合。通过选择超平面,被映射到超平面上的特征空间中的点集
x 由以下关系定义:
\begin{equation}\label{equation:chapter_idea/understand_tech:chapter_idea/understand_tech:0}
\begin{split}\sum_{i} \alpha_{i} k\left(x_{i}, x\right)=\text { constant }\end{split}
\end{equation}
如果随着 y 逐渐远离 x,k(x,y) 变小,则求和中的每一项都是在衡量测试点 x
与对应的数据基点
Xi的接近程度。这样,上述内核的总和可以用于衡量每个测试点相对于待分离的集合中的数据点的相对接近度。

第三、至于线性SVM的间隔计算可以由SVM模型算法工程专家来操作。AI产品经理明确SVM的用途、定义、和在遇到问题的时候知道这个问题是由SVM引起的或者可以找SVM专家协作解决即可。


\subparagraph{融入开发过程}
\label{\detokenize{chapter_idea/understand_tech:id6}}\begin{itemize}
\item {} 
能利用自已的行业知识帮助算法工程师进行数据预处理,获得高质量数据集。

\item {} 
能对解决方案所需的开发量进行进行初步估计。

\item {} 
能将产品需求拆分成系统模块并进行简单旳任务分解,能找到相关开发人员推动落地实施。

\item {} 
数据采集、数据标注、数据清洗、数据扩充、数据特征。

\end{itemize}


\subparagraph{协助系统架构师搭建合理的系统架构}
\label{\detokenize{chapter_idea/understand_tech:id7}}\begin{itemize}
\item {} 
明确系统中的每个模块的定位和意义

\item {} 
从业务发展的角度指导技术人员进行系统架构的设计

\end{itemize}


\subparagraph{与公司领导或客户沟通解释}
\label{\detokenize{chapter_idea/understand_tech:id8}}
例如,在机器学习模型训练过程中,由于技术的复杂性会导致出现很多计划外的工作量和效果,当老板提出质疑时,需要产品经理主动解释当前的状况,并结合其对市场竞争状况、用户需求的理解说服老板投入更多资源,为研发获得更多的支持。当某些预测模型的精准度不是特别高时,产品经理还应学会与客户进行技巧性的沟通,为产品争取更多的优化时间。\sphinxhref{http://www.changgpm.com/thread-356-1-1.html}{16}%
\begin{footnote}[436]\sphinxAtStartFootnote
\sphinxnolinkurl{http://www.changgpm.com/thread-356-1-1.html}
%
\end{footnote}


\paragraph{产品形式}
\label{\detokenize{chapter_idea/understand_tech:id9}}
主流的产品有客户端、Web端(PC网页端)、移动端(WebApp、原生App、HybridApp)


\paragraph{对软件设计的理解问题}
\label{\detokenize{chapter_idea/understand_tech:id10}}\begin{itemize}
\item {} 
面向过程,是指以任务/事件为中心,进行软件设计。

\item {} 
面向对象,是指以事物为中心的软件设计。

\end{itemize}

搭乘地铁从T站到F站的简单例子来说明:

如果用面向过程的设计方式,会将地铁启动、运行、到站等一系列的动作设定为过程,也许还要设定地铁维修的过程。然后将这所有的\sphinxstylestrong{过程按照逻辑}串在一起,完成一个任务。

如果用面向对象的方式设计,那直接将地铁定义为\sphinxstylestrong{对象},地铁的颜色、形状等则为\sphinxstylestrong{属性},地铁的运行和到站就是地铁的\sphinxstylestrong{方法},也即地铁的行为,而不是过程。


\paragraph{对需求实施的理解问题}
\label{\detokenize{chapter_idea/understand_tech:id11}}
曾因为一个简单页面的图文修改,对技术人员嗤之以鼻,但当了解内情后才发现,不仅仅是修改html的问题,还涉及到css、json、数据库读取修改以及数据字段的调整。所以对于需求的理解,从产品经理和技术人员角度而言,所看到的大小和范围,也许就像冰山一样,水面和水底有很大的区别。

在这种技术层面的改动要大于产品预期的情况,难免就会产生分歧。为了尽量使需求的实施理解,也能保持同步,可以参考以下要素:
\begin{enumerate}
\sphinxsetlistlabels{\arabic}{enumi}{enumii}{}{.}%
\item {} 
参加技术人员的概要设计评审:当产品需求提到技术层面时,一般技术人员会对需求进行概要设计、评审、详细设计及评审、开发实施等环节。当然产品经理一般不会在技术层面介入太深,但为了尽量使需求不偏离目标,参加技术层面的概要设计评审,是很好的一个选择,虽然对于多数产品经理而言,不一定能全听懂技术在概要设计层面的讨论。参加概要设计评审可以了解需求在启动技术设计时,涉及到的相关系统、干系人、内外部团队等,大致了解技术实施层面的困难、瓶颈和资源需求。以减少用户类型、路径等环节的偏差。

\item {} 
提前向技术同步产品的远期愿景:同步产品愿景和长期版本目标,可以是在需求刚出现时,也可以是在交互设计时,但个人感觉最晚不能晚于技术的概要设计。提前同步产品愿景,可以在技术人员做技术设计时,能确定数据、架构、迭代以及预留字段,更能确定技术实现方式,是按照较大的系统实施,还是按照简单的逻辑实施,因为很多时候,技术的实现方式有多种选择。以免产品的期望是宏伟大厦,因为没有提前同步给技术,导致技术在打地基时,按照普通的平房实施了。

\item {} 
了解需求中的关键点:这一点需要在每一次技术沟通中进行确认,但尽量在技术概要设计前了解清楚,这也就是参加技术概要设计评审的重要性所在。了解需求的关键点,了解了相关困难、瓶颈、资源需求等,对于需求实施的排期、时间节点评估则会掌握的比较清晰。

\end{enumerate}


\paragraph{系统设计中需要明确的问题 12\sphinxfootnotemark[437]}
\label{\detokenize{chapter_idea/understand_tech:id12}}%
\begin{footnotetext}[437]\sphinxAtStartFootnote
\sphinxnolinkurl{https://mp.weixin.qq.com/s?\_\_biz=MjM5NzA5OTAwMA==\&mid=2650005955\&idx=1\&sn=f59d7983001064f21cb15ee57da13fa8\&chksm=bed8655489afec42924c355061fa03d1d43ce943cf8aadc0103705690d72346f18e0dbb1cf00\&scene=21\#wechat\_redirect}
%
\end{footnotetext}\ignorespaces 
在系统设计中,至少需要明确以下问题:
\begin{enumerate}
\sphinxsetlistlabels{\arabic}{enumi}{enumii}{}{.}%
\item {} 
该系统涉及到的模块有哪些?哪些模块是已有的,哪些模块是新增的?

\item {} 
每个模块的定位,或者说定义是什么?在系统中扮演什么样的角色,起到什么样的作用?旧有模块的定义是否满足我们的要求,新模块的定义是否清晰明确?

\item {} 
每个模块的输入输出是什么?每个模块所获得的输入是否刚好满足其能完成任务的需求,既不缺乏信息,也不存在会导致依赖的信息冗余?

\item {} 
模块间的上下位关系是否明确,是否与该模块的原有定位相契合?

\item {} 
系统整体的模块的调用顺序是什么?是否拥有合理的信息通路?是否保证了模块上下位关系的一致性?是否存在下位模块僭越上位模块进行/被进行跨层级调用的情况?

\end{enumerate}


\paragraph{项目进度推进}
\label{\detokenize{chapter_idea/understand_tech:id13}}
产品和技术都转换思维,首先是了解对方的想法,然后是从对方角度思考,共同发掘问题和困难所在,再去解决。这样提前预估、制定时间节点、共同督促的推进方式,才能使项目推进更顺利。
\begin{enumerate}
\sphinxsetlistlabels{\arabic}{enumi}{enumii}{}{.}%
\item {} 
了解实时进度,根据需求的关键点,把控项目进度:前文提到,了解需在技术实施环节的关键节点,目的就是为了整体把控需求,防止在关键节点掉链子。有时是需要产品协助,或是督促技术打通关键节点的问题,有时则是因为前期的评估和了解,提前将实施中关键节点可能存在的问题消化掉。

\item {} 
需求实施的“时间最小单元”不能太久:需求实施的“时间最小单元”,我把它定义为,需求实施过程中,可以标识为里程碑或是有明确交付物的最短时间。例如一个H5的登录注册功能的开发,判断每个输入框信息输入格式是否准确,将信息提交至数据库,数据库写入数据并返回是否正确写入,给用户对应的反馈,这些每个环节的开发所需时间,都可以理解为一个时间最小单元。按照正常的逻辑,这样的时间最小单元,建议是0.5天至3天,最好不超过3天。

\item {} 
时不时的讨论推进的困难和进度、调整变更需求\sphinxhref{http://www.woshipm.com/pmd/4288664.html}{13}%
\begin{footnote}[438]\sphinxAtStartFootnote
\sphinxnolinkurl{http://www.woshipm.com/pmd/4288664.html}
%
\end{footnote}:对于推进实施中的需求,不能当成一个完全交出去的任务,更不能当“甩手掌柜”,而是应该参照时间最小单元,不时的讨论推进中是否存在困难,应如何解决困难,询问时间最小单元中的推进进度,如有没有进度,则可能需要调整计划了。

\end{enumerate}


\paragraph{什么是技术架构?}
\label{\detokenize{chapter_idea/understand_tech:id14}}
架构就是对系统中的实体以及实体之间的关系所进行的抽象描述,是一系列的决策,架构也是产品的结构和愿景。

系统架构是概念的体现,是对物/信息的功能与形式元素之间的对应情况所做的分配,是对元素之间的关系以及元素同周边环境之间的关系所做的定义。

做好架构是个复杂的任务,也是个很大的话题,本篇就不做深入了。有了架构之后,就需要让干系人理解、遵循相关决策。


\paragraph{单体应用和微服务}
\label{\detokenize{chapter_idea/understand_tech:id15}}
同样的,在早期大部分应用不会考虑到技术架构,但随着用户增加和未来性能要求则需要重构,这就需要到技术资深的架构师。而市面上的架构主要分为下面三类

单体应用程序:应用程序的全部功能被一起打包作为单个单元或应用程序.这个单元可以是JAR、WAR、EAR,或其他一些归档格式,但其全部集成在一个单一的单元.
微服务:微服务是一个新兴的软件架构,就是把一个大型的单个应用程序和服务拆分为数十个的支持微服务。一个微服务的策略可以让工作变得更为简便,它可扩展单个组件而不是整个的应用程序堆栈,从而满足服务等级协议。


\subparagraph{单体应用}
\label{\detokenize{chapter_idea/understand_tech:id16}}

\subparagraph{优点}
\label{\detokenize{chapter_idea/understand_tech:id17}}\begin{enumerate}
\sphinxsetlistlabels{\arabic}{enumi}{enumii}{}{.}%
\item {} 
方便调试,代码都在一起;

\item {} 
没有分布式开销,所有服务都在本地容器内;

\item {} 
中小型项目可以快速迭代,不需要太多资源。单体应用缺点:

\end{enumerate}


\subparagraph{缺点}
\label{\detokenize{chapter_idea/understand_tech:id18}}\begin{enumerate}
\sphinxsetlistlabels{\arabic}{enumi}{enumii}{}{.}%
\item {} 
可复用性差:服务被打包在应用中,功能不易复用;

\item {} 
系统启动慢,一个进程包含了所有的业务逻辑,涉及到的启动模块过多,导致系统的启动、重启时间周期过长。

\item {} 
线上问题修复周期长;任何一个线上问题修复需要对整个应用系统进行全面升级。

\end{enumerate}


\subparagraph{面向服务架构(SOA)}
\label{\detokenize{chapter_idea/understand_tech:soa}}\label{\detokenize{chapter_idea/understand_tech:id20}}\label{\detokenize{chapter_idea/understand_tech:id19}}

\subparagraph{企业服务总线(ESB)}
\label{\detokenize{chapter_idea/understand_tech:esb}}
ESB是面向服务架构(SOA)的核心构成部分,指传统数据连接技术(web、xml、中间件技术)结合的产物,简单来说,就是一根管道,用来连接各个服务节点,为了集成不同系统,不同协议的服务,服务总线做了消息的转化解释和路由工作,让不同的服务互联互通;是一个具有标准接口、实现了互连、通信、服务路由。


\subparagraph{特点}
\label{\detokenize{chapter_idea/understand_tech:id21}}\begin{enumerate}
\sphinxsetlistlabels{\arabic}{enumi}{enumii}{}{.}%
\item {} 
系统集成:从系统角度讲,解决了企业系统与系统间通信问题,把原来散乱、无规划的系统间的网状结构梳理成规整,可治理的系统。在梳理时则需要引用一些产品,常用的是企业服务总线(ESB)、技术规范、服务管理规范。主要解决核心问题,无序变有序。

\item {} 
系统的服务化:从功能角度讲,把业务转换成可复用、可组装的服务,通过服务的编排实现业务的快速复制。目的是把原先固有的业务功能转变为通用的业务服务,实现快速复用。主要解决的核心问题,原来固有业务可复用。

\item {} 
业务的服务化:从企业的角度讲,把原来职能化的企业架构转变为服务化的企业架构,进一步提升企业的对外服务能力。把一个业务单元封装成一项服务。主要解决的核心问题是高效。

\end{enumerate}


\subparagraph{优点}
\label{\detokenize{chapter_idea/understand_tech:id22}}\begin{enumerate}
\sphinxsetlistlabels{\arabic}{enumi}{enumii}{}{.}%
\item {} 
数据统一,共享数据库,使服务接口使用同一的数据模型的数据,确保数据一致性

\item {} 
灵活性较高,缩短产品和服务的上线时间,降低了开发与改变流程的成本系统

\item {} 
由子系统组成,系统易于重构

\end{enumerate}


\subparagraph{缺点}
\label{\detokenize{chapter_idea/understand_tech:id23}}\begin{enumerate}
\sphinxsetlistlabels{\arabic}{enumi}{enumii}{}{.}%
\item {} 
技术不匹配,在某些情况并不能轻松对操作平台进行重新打包,原因是业务功能结构需求不匹配

\item {} 
系统间交互需要使用远程通讯 ,一定程度上降低了响应速度

\end{enumerate}


\subparagraph{微服务架构}
\label{\detokenize{chapter_idea/understand_tech:id25}}\label{\detokenize{chapter_idea/understand_tech:id24}}\label{\detokenize{chapter_idea/understand_tech:id26}}

\subparagraph{优点}
\label{\detokenize{chapter_idea/understand_tech:id27}}\begin{enumerate}
\sphinxsetlistlabels{\arabic}{enumi}{enumii}{}{.}%
\item {} 
分而治之;单个服务功能内聚,复杂性低;方便团队的拆分和管理;

\item {} 
单独部署,独立开发;

\item {} 
易于扩展,某一项服务的性能达到瓶颈,只需增加该服务的节点数即可,其它服务不改变

\item {} 
易于维护,每个微服务的职责单一,复杂性降低,不会牵一发而动全身

\end{enumerate}


\subparagraph{缺点}
\label{\detokenize{chapter_idea/understand_tech:id28}}\begin{enumerate}
\sphinxsetlistlabels{\arabic}{enumi}{enumii}{}{.}%
\item {} 
开发难度大,前期服务的定义和拆分需要较大工作量,每个服务都需要单独部署,运维、测试成本增加;

\item {} 
跨服务的调用通常是不同的机器,甚至是不同的机房,开发人员需要处理超时、网络异常等问题,原来的函数调用改为服务调用。

\item {} 
效率相对低,团队依赖强,一个服务的版本延迟会拖慢整个应用的开发周期。

\item {} 
需要分布式事务的支持。

\end{enumerate}


\subparagraph{中台 10\sphinxfootnotemark[439]}
\label{\detokenize{chapter_idea/understand_tech:id29}}%
\begin{footnotetext}[439]\sphinxAtStartFootnote
\sphinxnolinkurl{https://www.jianshu.com/p/a5894e8ba3f3}
%
\end{footnotetext}\ignorespaces 
中台是随着公司业务高速发展,组织不断膨胀的过程中暴露的问题需要解决。将企业的核心能力随着业务不断发展以数字化形式沉淀到平台,形成以服务为中心,由业务中台和数据中台构建起数据闭环运转的运营体系,供企业更高效的进行业务探索和创新。
中台做到前后分离,后台统一提供数据接口,前台实现业务流转。
\begin{enumerate}
\sphinxsetlistlabels{\arabic}{enumi}{enumii}{}{.}%
\item {} 
中台与微服务的区别:

\end{enumerate}
\begin{itemize}
\item {} 
中台是提升企业的能力的复用,一种方法论/思想。

\item {} 
微服务是独立开发、维护、部署的小型业务组件,一种技术架构。

\end{itemize}
\begin{enumerate}
\sphinxsetlistlabels{\arabic}{enumi}{enumii}{}{.}%
\setcounter{enumi}{1}
\item {} 
中台与微服务的关系:

\end{enumerate}
\begin{itemize}
\item {} 
微服务架构,是实现中台思想的落地的重要手段。

\end{itemize}
\begin{enumerate}
\sphinxsetlistlabels{\arabic}{enumi}{enumii}{}{.}%
\setcounter{enumi}{2}
\item {} 
中台解决的核心问题:

\end{enumerate}
\begin{itemize}
\item {} 
为减少重复业务系统开发及实现系统数据共享一个技术平台底座,将多年技术沉淀的价值最大化,统一各个业务部门或系统重复使用、重复建设的功能和系统统一规划和管理。

\end{itemize}
\begin{enumerate}
\sphinxsetlistlabels{\arabic}{enumi}{enumii}{}{.}%
\setcounter{enumi}{3}
\item {} 
什么时候需要中台:

\end{enumerate}
\begin{itemize}
\item {} 
如阿里:淘宝,有订单、库存、评价、积分、物流等业务系统。天猫也有订单、库存、评价、积分、物流等业务系统。1688,也有类似业务系统。多个系统有重复业务系统需要建设,且系统间数据不能完全共享,系统各自运行。此时使用技术中台以及业务中台,来实现业务重用及数据共享,把技术沉淀价值最大化。

\end{itemize}

AI中台,更多见:\sphinxurl{https://aieye-top.github.io/d2cl/chapter\_deploy/AI-zhongtai.html}


\subparagraph{SaaS}
\label{\detokenize{chapter_idea/understand_tech:saas}}
SaaS是一个服务需求方的完整解决方案产品产品,如它为顾客提供了完整的端到端解决方案,如计算后台到客户操作终端;\sphinxhref{http://www.woshipm.com/pd/4090455.html}{11}%
\begin{footnote}[440]\sphinxAtStartFootnote
\sphinxnolinkurl{http://www.woshipm.com/pd/4090455.html}
%
\end{footnote}


\paragraph{与测试相关的专业名词}
\label{\detokenize{chapter_idea/understand_tech:id30}}
有自动化测试,接口自动化测试,功能测试,性能测试,UI自动化测试,压力测试;自动化测试完成不了的我们测试人员会编写测试用例进行测试\sphinxhref{https://coffee.pmcaff.com/article/2447262389384320/pmcaff?utm\_source=forum}{26}%
\begin{footnote}[441]\sphinxAtStartFootnote
\sphinxnolinkurl{https://coffee.pmcaff.com/article/2447262389384320/pmcaff?utm\_source=forum}
%
\end{footnote}常见的测试工具,如jmeter,这些会些基本的测试就可以了。\sphinxhref{https://www.zhihu.com/question/22613861}{27}%
\begin{footnote}[442]\sphinxAtStartFootnote
\sphinxnolinkurl{https://www.zhihu.com/question/22613861}
%
\end{footnote}
\begin{enumerate}
\sphinxsetlistlabels{\arabic}{enumi}{enumii}{}{.}%
\item {} 
提测研发人员在开发完某个功能之后,把代码打包并提交给测试人员开始测试,就叫提测。

\item {} 
复现之前测试发现的Bug
再次出现,就叫复现。能否复现对于研发人员排查Bug非常重要。

\item {} 
测试用例测试用例是指测试人员根据PRD
撰写的测试流程及事项。例如,知乎要上线一个收藏文章的功能,点击“收藏”按钮,该文章就出现在收藏列表中,并且“收藏”按钮变为“已收藏”按钮,这就是一条测试用例;点击“已收藏”按钮,“已收藏”按钮就变为“收藏”按钮,同时该文章从收藏列表中消失,这就是另外一条测试用例。

\item {} 
功能测试功能测试是单一功能的测试,如某次迭代要做一个分享功能,功能测试就是测试分享这个功能是否符合PRD
的要求。

\item {} 
回归测试可以将回归测试理解为整体测试。例如,某次迭代要上线一个分享功能,需要测试一下这个功能是否会影响其他功能的正常使用,所以回归测试要测试的就是整个产品的所有功能。(回归测试是指修改了旧代码后,重新进行测试以确认修改没有引入新的错误或导致其他代码产生错误。自动回归测试将大幅降低系统测试、维护升级等阶段的成本。\sphinxhref{https://baike.baidu.com/item/\%E5\%9B\%9E\%E5\%BD\%92\%E6\%B5\%8B\%E8\%AF\%95}{28}%
\begin{footnote}[443]\sphinxAtStartFootnote
\sphinxnolinkurl{https://baike.baidu.com/item/\%E5\%9B\%9E\%E5\%BD\%92\%E6\%B5\%8B\%E8\%AF\%95}
%
\end{footnote})

\item {} 
测试报告测试报告是指在测试完成之后,由测试人员撰写的说明Bug
均已修复,可以上线的邮件或报告。

\end{enumerate}


\paragraph{产业生态 15\sphinxfootnotemark[444]}
\label{\detokenize{chapter_idea/understand_tech:id31}}%
\begin{footnotetext}[444]\sphinxAtStartFootnote
\sphinxnolinkurl{https://www.infoq.cn/article/oup6iotjzb9zrdg500ks}
%
\end{footnotetext}\ignorespaces \begin{itemize}
\item {} 
基于 Wintel 体系的计算机产业生态:以“Intel+Windows+软件”生态

\item {} 
基于 Android/iOS
体系的智能设备产业生态:以“CPU(ARM)+操作系统+开发工具+应用商店+各类应用”为核心

\item {} 
基于云原生(Cloud
Native)体系的云计算产业生态:构建起以“云厂商+异构软硬件+云边端+Serverless
化+软件全生命周期+开发者+企业客户”为核心

\end{itemize}


\subparagraph{云原生}
\label{\detokenize{chapter_idea/understand_tech:id32}}
容器+Kubernetes 技术的逐步成熟与发展,以“云原生(Cloud
Native)”为关键词的技术生态雏形基本确立。
\begin{itemize}
\item {} 
云原生技术:让系统更加弹性可靠容错、松耦合、易管理、可观察;代表技术是容器、微服务、服务网格、不可变基础设施和声明式
API。

\item {} 
云原生产品:云计算平台提供的数据库、大数据、中间件、函数技术、容器服务等开放标准的原生产品服务。

\item {} 
云原生架构:生于云长于云,最大化运用云的能力,依赖云产品构建的 IT
架构,让开发者聚焦于业务而不是底层技术。

\end{itemize}

以容器、Kubernetes
技术为主,向下封装底层基础设施差异性,如异构环境,异构硬件,向上支撑多样性的工作负载,如新型计算等,覆盖云、边、端,赋能无边界计算、分布式云,云原生逐步成为云计算的新界面,新一代的操作系统。

从
ISV(独立软件提供商)的软件全生命周期,到硬件厂商、云厂商、ISV、企业客户之间的新一轮的软硬件的供需体系,再到云计算技术、社区、ISV、开发者之间的技术互动体系中,云原生技术作为新一代云技术操作系统

当前云原生技术发展趋势是,以容器、Kubernetes
为核心的云原生技术逐渐稳定与成熟,后期将发展为以服务治理、云边端一体化、Serverless
等上层技术栈为创新发展的核心。


\paragraph{懂技术,更得懂AI的局限 4\sphinxfootnotemark[445]}
\label{\detokenize{chapter_idea/understand_tech:ai-4}}%
\begin{footnotetext}[445]\sphinxAtStartFootnote
\sphinxnolinkurl{https://www.chinaventure.com.cn/news/114-20191210-350906.html}
%
\end{footnotetext}\ignorespaces 
除了基本的产品技能还要掌握AI基础技术知识,如NLP自然语言、DL深度学习、ML机器学习、大数据等

AI公司的产品里一类是应用AI技术的垂直业务产品,另外一类是AI服务的平台产品。前者负责AI能力在细分领域的应用;后者则是对AI能力的汇总和包装,例如各种AI开放平台、各种云计算平台,这就要求产品经理必须熟知公司内部的AI技术能力,还要有能力作为售前支持,为使用方提供技术咨询。

当实现一款功能的设计的时候,最基础的认知就是要首先确定什么能做什么不能做,对于可见的一些服务,比方说手机APP中的用户使用用链路来讲,一个功能能否实现是比较容易确定的。但是如果是AI类产品的设计,需要涉及到对算法以及数据的理解,只有当产品经理真正了解每种算法的玩法以及数据的使用链路,才可以将功能做活,保留高鲁棒性。

大部分的AI产品的服务对象是to B端的企业用户,
B端用户和C端用户的使用行为习惯是截然不同的,所以就有很多C端的产品转向B端出现的水土不服。


\paragraph{机器学习增加了不确定性 7\sphinxfootnotemark[446]}
\label{\detokenize{chapter_idea/understand_tech:id33}}%
\begin{footnotetext}[446]\sphinxAtStartFootnote
\sphinxnolinkurl{https://www.oreilly.com/radar/what-you-need-to-know-about-product-management-for-ai/}
%
\end{footnotetext}\ignorespaces 
有了机器学习,我们通常会得到一个在统计上比简单技术更准确的系统,但也有一个缺点,那就是一小部分模型预测总是错误的,有时会以难以理解的方式出现。

这种转变需要在软件工程实践中进行根本性的改变。用看似相似的输入输出对数据集训练的相同神经网络代码可以给出完全不同的结果。相同代码生成的模型输出将随着训练数据的大小(标记示例的数量)、网络训练参数和训练运行时等内容的变化而变化。这对软件测试、版本控制、部署和其他核心开发过程有严重的影响。

对于任何给定的输入,相同的程序不一定会产生相同的输出;输出完全取决于模型是如何训练的。对训练数据进行更改,用相同的代码重复训练过程,您将从模型中得到不同的输出预测。也许差别很细微,也许差别很大,但它们是不同的。

在这种不确定性之下,开发过程本身还存在着进一步的不确定性。很难预测一个人工智能项目需要多长时间。对传统软件来说,预测开发时间已经够困难的了,但至少我们可以根据过去的经验做出一些一般性的猜测。我们知道“进步”是什么意思。使用人工智能,你通常不知道会发生什么,直到你尝试它。花上几周甚至几个月的时间才能找到可行的方法,将模型的准确率从70\%提高到74\%,这种情况并不少见。很难说最大的模型改进是来自更好的神经网络设计、输入特征还是训练数据。你经常不能告诉经理模型将在下周或下个月完成;你的下一次尝试可能会成功,或者你可能会受挫好几个星期。你常常不知道某件事是否可行,直到你做了实验。


\paragraph{技术预研 21\sphinxfootnotemark[447]}
\label{\detokenize{chapter_idea/understand_tech:id34}}%
\begin{footnotetext}[447]\sphinxAtStartFootnote
\sphinxnolinkurl{http://www.xmamiga.com/3573/}
%
\end{footnotetext}\ignorespaces 
当产品目标从宏观到微观都有明确的定义后,产品经理就可以开始:技术预研。人工智能产品经理要理解技术的实现过程,这就要求产品经理在关注用户体验的同时要关注这些体验的实现方式和过程。如果不懂技术原理,产品经理可能无法提出创造性和颠覆性产品创意,同时产品经理需要给研发团队提供研发阶段的帮助也需要懂技术。


\subparagraph{领域技术基本现状和趋势}
\label{\detokenize{chapter_idea/understand_tech:id35}}
用人脸识别来举例:

计算机视觉的整体发展趋势:
\begin{itemize}
\item {} 
从“让机器看”到“让机器看懂、理解、执行”

\item {} 
从看图片到看视频

\item {} 
从分类到识别,再到理解

\end{itemize}

最终就是 图像分割 —\sphinxhyphen{}> 特征提取 —\sphinxhyphen{}> 行为识别 的整个过程。

常见的人脸识别应用:人脸图像预处理、人脸图像检测、人脸图像采集、人脸特征提取、人脸特征识别、表情识别、3D人脸重建、人脸变形。

一般的人脸识别主要有五部分:
\begin{enumerate}
\sphinxsetlistlabels{\arabic}{enumi}{enumii}{}{.}%
\item {} 
图像采集:使用被检测物体的重要特征显现,同时过滤掉不重要特征

\item {} 
人脸检测

\item {} 
人脸图像预处理

\item {} 
人脸图像特征提取

\item {} 
人脸匹配与识别

\end{enumerate}


\subparagraph{领域前沿技术}
\label{\detokenize{chapter_idea/understand_tech:id36}}
在深度学习、传感器技术、芯片的发展的当下,深度摄像头(3D传感器)成为近来机器视觉方面的投资和创业热点。通过深度相机就可以构建人脸的三位信息,在人体跟踪,人机交互,AR等领域运用广泛。
目前,比较成熟的深度方案有:
\begin{itemize}
\item {} 
结构光:通过发射特定图形的散斑或者点阵激光红外图案,摄像头捕捉反射回来的图案,比较散斑和原始的大小测算物体和摄像头之间的距离。多用于近距离场景。

\item {} 
双目视觉:通过两个摄像头的视差来获得深度信息,运算量大、实时性差。适用于手势识别。

\item {} 
飞行时间法3D成像:通过红外光反射回来的时间差或相位差来获得深度信息。

\end{itemize}


\subparagraph{常见技术逻辑}
\label{\detokenize{chapter_idea/understand_tech:id37}}
以人脸识别在安防中的逻辑为例。
人脸图像采集:图像体积、图像分辨率、图像外部采集环境

人脸检测:人脸检测的目的是从图像中确定人脸的位置和大小。常见的算法有:Viola\sphinxhyphen{}Jones、Haar+AdaBoost、CascadeCNN等,产品经理需要有一套量化标准。
\begin{itemize}
\item {} 
检测率:存在人脸且被检测出的图像在所有人脸图像中的比例。

\item {} 
漏检率:存在人脸且没有被检测出来的图像在所有存在人脸的图像中的比例。

\item {} 
误检率:不存在人脸但是检测存在人脸的图像在所有不存在人脸图像中的比例。

\end{itemize}

产品经理需要了解行业内对产品质量的衡量标准,在产品需求阶段衡量产品需求描述,量化产品目标。项目验收中用数据量化产品质量。

图像预处理:图像预处理的目的是提高图片质量,去噪,使得图像特征表现出来。主要技术手段有:人脸图像的几何校正,光照补偿、尺寸归一化、灰度变换、去噪、边界增强、提高对比度、直方图均衡化、中值滤波以及锐化。产品经理需要了解行业中特有的数据治理技术,包括不同类型数据的治理周期、需要投入的成本、数据治理过程中的阻碍等。

人脸图像特征提取:特征提取的目的是针对数据的原始特征的缺陷,降低特征维度,提高分类器的设计和性能。人工智能产品经理需要理解不同框架的逻辑以及区别,对前沿技术保持敏感度,不断优化功能和产品体验。

人脸匹配与识别:对提取的人脸数据与数据库中的特征模板进行匹配,设定一个阈值,超过该阈值即可判定为某一个人。
\begin{itemize}
\item {} 
人脸识别:计算两张脸的相似度。主要有身份验证等。

\item {} 
人脸检索:给定一张脸,找出同一张脸的图片。活体检测检索,通过眨眼等动作。主要用于签到考勤、门禁闸机、安防监控。

\end{itemize}


\subparagraph{判断技术切入点}
\label{\detokenize{chapter_idea/understand_tech:id38}}
在充足的产品预研后,接下来是选择合理的技术方向。目前主要有软件为切入点和自研“软件+硬件”切入点。

产品经理的技术预研和研发人员不同,重点关注技术的趋势、领先性、主流算法框架的优劣,横向对比竞争对手之间的技术实现手段和重点商品的参数,从中提炼自身产品的优势。
产品经理需要将产品的技术底层实现的方式,作为量化产品需求的依据和前提。


\paragraph{技术可行性 9\sphinxfootnotemark[448]}
\label{\detokenize{chapter_idea/understand_tech:id39}}%
\begin{footnotetext}[448]\sphinxAtStartFootnote
\sphinxnolinkurl{https://wiki.mbalib.com/wiki/\%E6\%8A\%80\%E6\%9C\%AF\%E5\%8F\%AF\%E8\%A1\%8C\%E6\%80\%A7}
%
\end{footnotetext}\ignorespaces 
技术可行性是指决策的技术和决策方案的技术不能突破组织所拥有的或有关人员所掌握的技术资源条件的边界。

做技术可行性分析时需注意全面考虑系统开发过程所涉及的所有技术问题,尽可能采用成熟技术,慎重引入先进技术,着眼于具体的开发环境和开发人员,技术可行性评价等问题。


\subparagraph{精确定义}
\label{\detokenize{chapter_idea/understand_tech:id40}}
“可行”的一个重要部分是精确定义。正如杰里米·乔丹所说:“一个明确定义的问题已经解决了一半。”如果你能非常精确地说出你想要完成的事情,并把它分解成更简单的问题,你就有了一个良好的开端。Jordan有一些很好的建议:从自己动手解决问题开始。如果你想帮助客户整理他们手机上的图片,花点时间整理你的手机上的图片。与真正的客户面谈,看看他们想要什么。建立一个他们可以用真实数据尝试的原型。最重要的是,不要认为“我们想帮助客户组织图片”是一个充分的问题陈述。它不是;你必须更详细地了解你的客户是谁,他们想如何组织他们的图片,他们可能有什么样的图片,他们想如何搜索,等等。


\paragraph{数据标记}
\label{\detokenize{chapter_idea/understand_tech:id41}}
看看您可以多快地为ML算法构建一个带有标记的基准数据集以及明确、狭窄定义的精度目标。数据标记的便捷性是机器学习是否具有成本效益的一个很好的代理。如果您可以在产品的正常用户活动中构建数据标记(例如,标记垃圾邮件),那么您就有机会收集足够多的输入\sphinxhyphen{}输出对来训练您的模型。否则,您将为标记数据的外部服务烧钱,而且在您进行第一次演示之前的前期成本很容易成为项目中最昂贵的部分。没有大量的原始数据和标注的训练数据,解决大多数人工智能问题是不可能的。


\paragraph{“BUG”一词}
\label{\detokenize{chapter_idea/understand_tech:bug}}
“BUG”一词在工程师与产品经理的角度中可能也存在偏差,在工程师的角度看,它是因为代码或者逻辑出错而导致的功能性错误,因为不影响产品功能的优化,所以不是BUG。而产品经理的角度看,认为是影响用户体验的产品BUG,本质上是交互设计问题,在加载过程中需要对用户有所提示使得产品体验更好。两种角度,两种观点,而书中告诉了我们解决方法。


\paragraph{系统设计中需要明确的问题}
\label{\detokenize{chapter_idea/understand_tech:id42}}\begin{enumerate}
\sphinxsetlistlabels{\arabic}{enumi}{enumii}{}{.}%
\item {} 
在系统设计中,至少需要明确以下问题:

\item {} 
该系统涉及到的模块有哪些?哪些模块是已有的,哪些模块是新增的?

\item {} 
每个模块的定位,或者说定义是什么?在系统中扮演什么样的角色,起到什么样的作用?旧有模块的定义是否满足我们的要求,新模块的定义是否清晰明确?

\item {} 
每个模块的输入输出是什么?每个模块所获得的输入是否刚好满足其能完成任务的需求,既不缺乏信息,也不存在会导致依赖的信息冗余?

\item {} 
模块间的上下位关系是否明确,是否与该模块的原有定位相契合?

\item {} 
系统整体的模块的调用顺序是什么?是否拥有合理的信息通路?是否保证了模块上下位关系的一致性?是否存在下位模块僭越上位模块进行/被进行跨层级调用的情况?

\end{enumerate}

做个形象点的类比,设计系统就像拼拼图。第一个问题,就是看我们手上有哪些拼图;第二个问题,就是看拼图上的画是什么;第三个问题就是看拼图的边缘是什么样的;第四个问题,就是看哪些拼图的边缘是相互契合的;第五个问题,就是拼好后,看整幅拼图是否存在不一致错误。
\sphinxhref{https://www.sohu.com/a/203392612\_744545}{18}%
\begin{footnote}[449]\sphinxAtStartFootnote
\sphinxnolinkurl{https://www.sohu.com/a/203392612\_744545}
%
\end{footnote}


\paragraph{技术专利}
\label{\detokenize{chapter_idea/understand_tech:id43}}
\sphinxurl{http://www.xmamiga.com/177/}


\paragraph{更多}
\label{\detokenize{chapter_idea/understand_tech:id44}}
9个B端产品经理需要懂的技术 \sphinxhyphen{} 起点学院的文章 \sphinxhyphen{} 知乎
\sphinxurl{https://zhuanlan.zhihu.com/p/144314827}


\subsubsection{设计思维 1\sphinxfootnotemark[450]}
\label{\detokenize{chapter_idea/design:id1}}\label{\detokenize{chapter_idea/design::doc}}%
\begin{footnotetext}[450]\sphinxAtStartFootnote
\sphinxnolinkurl{https://weread.qq.com/web/reader/8d232b60721a488e8d21e54kaab325601eaab3238922e53}
%
\end{footnotetext}\ignorespaces 
能够服务于体验的设计才是出色的设计。——苹果官网


\paragraph{产品设计}
\label{\detokenize{chapter_idea/design:id2}}
产品经理的核心工作是产品设计。产品设计的重点在于提升产品的用户体验,让用户喜欢并且坚持使用我们的产品。

提升产品的用户体验分两个维度。第一个维度是提升产品的可用性,即让产品确实能解决用户的需求“痛点”,这部分内容我会在第4
章中详细讲解。第二个维度是提升产品的易用性,即让产品对用户来说易于学习和使用、记忆负担小,且产品的使用满意度高。要想提升产品的易用性,需要仔细打磨产品交互,而交互设计七大定律和尼尔森十大原则就是在解决产品易用性的问题。

TODO:https://tangjie.me/blog/118.html

\begin{figure}[H]
\centering
\capstart

\noindent\sphinxincludegraphics{{design_a_new_feature}.png}
\caption{设计新功能\sphinxhref{http://www.woshipm.com/zhichang/4180017.html}{14}\sphinxfootnotemark[451]}\label{\detokenize{chapter_idea/design:id20}}\end{figure}
%
\begin{footnotetext}[451]\sphinxAtStartFootnote
\sphinxnolinkurl{http://www.woshipm.com/zhichang/4180017.html}
%
\end{footnotetext}\ignorespaces 

\paragraph{设计模式 9\sphinxfootnotemark[452]}
\label{\detokenize{chapter_idea/design:id3}}%
\begin{footnotetext}[452]\sphinxAtStartFootnote
\sphinxnolinkurl{https://www.yinxiang.com/everhub/note/f9ab87ee-73e6-4241-9428-9507cbfd007f}
%
\end{footnotetext}\ignorespaces 
模式思维源于《建筑模式语言》。模式,是指可以重复使用的方式和方法。

总结属于自己的设计模式:模式名称、概念和价值、适用范围、模式描述、相关模式。
\begin{itemize}
\item {} 
流程:站点地图\sphinxhyphen{}页面类型\sphinxhyphen{}思考页面对应模式\sphinxhyphen{}用组件搭建模式

\item {} 
更多:《界面设计模式》《网站设计模式:有效地交互设计框架和模式》

\end{itemize}


\paragraph{交互设计七大定律}
\label{\detokenize{chapter_idea/design:id4}}\begin{enumerate}
\sphinxsetlistlabels{\arabic}{enumi}{enumii}{}{.}%
\item {} 
费茨定律

\item {} 
7±2 法则

\item {} 
奥卡姆剃刀原理

\item {} 
接近法则

\item {} 
希克定律

\item {} 
特斯勒定律

\item {} 
新乡重夫:防错原则

\end{enumerate}


\paragraph{尼尔森十大原则 2\sphinxfootnotemark[453]}
\label{\detokenize{chapter_idea/design:nielsen}}\label{\detokenize{chapter_idea/design:id5}}%
\begin{footnotetext}[453]\sphinxAtStartFootnote
\sphinxnolinkurl{https://weread.qq.com/web/reader/0c032c9071dbddbc0c06459k37632cd021737693cfc7149}
%
\end{footnotetext}\ignorespaces 
\sphinxstylestrong{尼尔森十大可用性原则}是人机交互学博士尼尔森(Nielsen)在分析了200多个可用性问题后提炼出的十项通用型原则,是产品设计与用户体验设计的重要参考标准,值得深入研究与运用。
\begin{enumerate}
\sphinxsetlistlabels{\arabic}{enumi}{enumii}{}{.}%
\item {} 
\sphinxstylestrong{状态可见原则(Visibility of System
Status)}:在产品使用过程中应该让用户知道发生了什么,并在合适的时间做出合适的反馈。

\item {} 
\sphinxstylestrong{环境贴切原则(Match between System and the Real
World)}:在进行产品设计时应使用用户熟悉的语言体系、操作模式,尽量遵循现实世界中符合逻辑的交互过程。

\item {} 
\sphinxstylestrong{用户可控原则(User Control and
Freedom)}:对于用户的一些误操作提供二次确认和错误修正功能,这样可以提高产品的可控性,否则对产品进行一些关键性操作可能带来毁灭性的打击。例如,数据删除操作,如果没有二次确认,可能会造成严重的影响。

\item {} 
\sphinxstylestrong{一致性原则(Consistency and
Standards)}:在进行产品设计时采用的语言体系、操作模式、视觉风格、组件样式等应保持统一。

\item {} 
\sphinxstylestrong{防错原则(Error
Prevention)}:在进行产品设计时,应提供防止误操作的机制,减少用户犯错的可能。例如,当关键数据没有填写时,保存按钮置灰等。

\item {} 
\sphinxstylestrong{无回忆原则(Recognition Rather
thanRecall)}:在产品使用过程中,应尽量减少用户对操作过程的记忆负荷,所有的动作和处理都应该是可见的。用户在一个页面进行操作和处理时,无须记忆上一个页面的内容。

\item {} 
\sphinxstylestrong{灵活高效原则(Flexibility and Efficiency of
Use)}:在产品设计过程中,应充分考虑操作的灵活性和信息传递的高效性,简化操作过程,减少信息传递过程,为用户提供更便捷的操作方式。就像在电子商务系统中寻找商品一样,如果商品种类过多,用户需要逐级查找才可以寻找到目标商品,为了避免出现分类过多的问题,电子商务系统提供了虚拟分类、分类导航等多种方式让用户可以便捷地找到目标商品。

\item {} 
\sphinxstylestrong{易扫原则(Aesthetic and Minimalist
Design)}:保留主要信息展示,尽量避免无关信息影响主要信息,保证主要信息的简洁和美观。互联网用户浏览界面的动作不是读,也不是看,而是扫。易扫,意味着突出重点,弱化和剔除无关信息。

\item {} 
\sphinxstylestrong{容错原则(Help Users Recognize,Diagnose,and Recover From
Errors)}:产品设计时应充分考虑用户可能出现的错误,并设计功能帮助用户从错误中恢复,将损失降到最低。如果无法自动挽回,则应提供详尽的说明文字和指导方向。

\item {} 
\sphinxstylestrong{人性化帮助原则(Help and
Documentation)}:优秀的产品不需要帮助文档,如果必须要提供帮助文档,则以流程化、图形化的形式提供。

\end{enumerate}

{\color{red}\bfseries{}|}口碑分析1{\color{red}\bfseries{}|}\sphinxhref{https://vickydyy.github.io/2019/05/26/Product-User-Thought/}{15}%
\begin{footnote}[454]\sphinxAtStartFootnote
\sphinxnolinkurl{https://vickydyy.github.io/2019/05/26/Product-User-Thought/}
%
\end{footnote}
{\color{red}\bfseries{}|}口碑分析2{\color{red}\bfseries{}|}\sphinxhref{https://vickydyy.github.io/2019/05/26/Product-User-Thought/}{15}%
\begin{footnote}[455]\sphinxAtStartFootnote
\sphinxnolinkurl{https://vickydyy.github.io/2019/05/26/Product-User-Thought/}
%
\end{footnote}


\subparagraph{违反原则}
\label{\detokenize{chapter_idea/design:id14}}
之前版本,支付宝支付在美团付款页面都处于折叠状态。————阿里巴巴就撤离了美团,便再次将目标投向了“饿了么”,之后美团就一直是依靠腾讯这颗“大树”生存。
\sphinxhref{https://new.qq.com/omn/20200801/20200801A0CZYF00.html}{4}%
\begin{footnote}[456]\sphinxAtStartFootnote
\sphinxnolinkurl{https://new.qq.com/omn/20200801/20200801A0CZYF00.html}
%
\end{footnote}


\paragraph{设计原则}
\label{\detokenize{chapter_idea/design:id15}}\begin{itemize}
\item {} 
好的产品是有创意的

\item {} 
好的产品必须对人有用

\item {} 
好的产品是优美的

\item {} 
好的产品是容易使用的

\item {} 
好的产品是含蓄的,不招摇的

\item {} 
好的产品是诚实的

\item {} 
好的产品会经久不衰

\item {} 
好的产品不会放过任何一个细节

\item {} 
好的产品是环保的,不浪费太多资源的

\item {} 
好的产品尽可能少地体现设计(少即是多:尽量简化产品的功能模块、交互流程、界面元素、配色字体等{[}18{]})
\#\# 美术设计
\sphinxhref{https://weread.qq.com/web/reader/77532110721ea34a7751c9ak02e32f0021b02e74f10ece8}{7}%
\begin{footnote}[457]\sphinxAtStartFootnote
\sphinxnolinkurl{https://weread.qq.com/web/reader/77532110721ea34a7751c9ak02e32f0021b02e74f10ece8}
%
\end{footnote}

\end{itemize}

看到一款应用时,可以先用上面介绍的「黄金圈法则」去思考,然后从结构层,框架层,表现层去一步一步拆解分析。比如,微信7.0大改版,从层级结构上有哪些变化?意图是什么?视觉上为什么开屏有一朵花?想传达什么信号?等等……只有多思考,你的审美才会提升,产品感也会越来越好。
\sphinxhref{http://www.woshipm.com/pmd/1774122.html}{11}%
\begin{footnote}[458]\sphinxAtStartFootnote
\sphinxnolinkurl{http://www.woshipm.com/pmd/1774122.html}
%
\end{footnote}

作为产品经理,在日常工作中对产品质量的把控也是一项很重要的工作。一是防止需求在执行过程中变形或走样,二是提升产品最终交付到用户手中时的质量。

对产品UI设计图的把控是重要的一环,因为这直接决定了产品的“脸面”,也决定了用户看见产品时产生的第一印象。每个设计师的水平及其对需求的理解程度不同,最后设计出来的页面效果也各不相同。产品经理作为全局把控者,需要具备分辨美丑的能力,并对设计图把关,而不是设计出来什么样就是什么样。

对于UI图,产品经理一般需要关注哪些方面呢?

第一是整体背景。一些活动类的产品设计,要看背景颜色是否与产品的主题搭配。例如,一个抽奖的活动采用了冷色调,会让用户一看就没有任何参与的欲望;或者背景元素占据太多视觉空间,导致核心主题不突出等。

第二是整个页面结构。有时设计师会调整产品原型的布局,因此产品经理要关注这些调整是否合理,是否会影响用户使用产品时的体验。例如,设计师将右下角悬浮的发布按钮移到了右上角,对于单手操作的用户来说就很不方便。

第三是元素排布。产品经理需要关注元素是否按照突出的优先级进行展示、元素之间是否对齐、元素的间距是否得当、同一元素的字体大小是否统一等。

第四是配色。按钮和字体的颜色亮度是否适中、是否与产品色调协调等。一般成熟的产品都会有自己的配色规范,而规范内会有很多类别可选择。因此,产品经理要关注设计师选择的规范类别是否与产品的风格一致。

第五是字体字号。文字是构成页面的核心内容,如果字号太大,会让整个页面看起来不够精致,但是字号太小又不易阅读。因此,产品经理要根据描述内容的重要性调整文字的字体和字号。例如,标题类内容一般需要突出,字号相对较大;而对标题的描述仅仅是作为一个补充,字号就应该设置得小一些。

除此之外,还有一些其他因素也会影响页面的整体美观程度,产品经理需要在日常工作中不断学习、训练,提升自己的审美能力。另外,关注页面的美观性并不是一味地只追求好看,更不能单纯地为了好看而牺牲用户体验。那么,在日常工作中,产品经理应该如何提升自己的审美能力呢?

第一,要多看优秀的产品设计。产品经理经常看一些国内外优秀的产品设计,潜移默化中会影响自身的审美水平,当回过头再看自己的产品设计时,就很容易判断页面设计得好看与否。常见的设计平台如国内的花瓣、站酷、最美应用等,以及国外的Behance、dribbble等,都是查看优秀设计作品的优质渠道。除了在工作中多看多思考之外,产品经理在生活中还要做一个有心人,多观察日常环境中的设计,如地铁指示牌、饭店的招牌、活动海报等,看它们的配色、排版及突出传达的信息,通过不断地进行“头脑体操”提升自己的审美能力。

第二,分析优秀产品设计背后的原因。产品经理在能分辨产品设计的好坏之后,还需要知道好在哪里、坏在哪里(什么东西吸引了你?为什么是这样的配色?你是怎么样和身边的设计进行交互的?),只有这样才能针对设计问题提出改进意见或方案。这也就要求大家在看一些好的作品时多思考,分析背后的原因。例如,是因为配色用得好、布局排列好,还是字体或字号选得好等。通过不断地分析思考,时间久了,产品经理也就能大体总结一个优秀设计作品的构成要素,从而更好地进行日常工作中的产品设计。

第三,在设计产品原型时,如果条件允许,可以尝试出一些高保真的原型图。当看过优秀的设计作品后,很多方法、思路都会停留在脑海中。而当真正落地到实践时,产品经理会对其理解得更加深刻,审美能力也会进一步得到提升。

在日常设计中,产品经理要尽量让UI设计师模拟真实的产品使用场景,使用真实的封面图和文案元素,这样就可以结合真实场景给出优化建议,建立反馈闭环,从而提升产品设计的合理性、美观性。


\paragraph{增长设计 6\sphinxfootnotemark[459]}
\label{\detokenize{chapter_idea/design:id16}}\label{\detokenize{chapter_idea/design:id17}}%
\begin{footnotetext}[459]\sphinxAtStartFootnote
\sphinxnolinkurl{https://weread.qq.com/web/reader/8d632bc07208ed1c8d697c4kd67323c0227d67d8ab4fb04}
%
\end{footnotetext}\ignorespaces \begin{itemize}
\item {} 
扩展分析

\item {} 
生命周期

\item {} 
传播设计

\item {} 
黏性设计

\item {} 
演进设计

\item {} 
产品矩阵

\item {} 
增强回路

\item {} 
增长飞轮

\item {} 
杠杆没计

\item {} 
成功指标

\item {} 
模式设计

\end{itemize}


\paragraph{产品设计}
\label{\detokenize{chapter_idea/design:id18}}\begin{itemize}
\item {} 
产品画布

\item {} 
产品系统

\item {} 
头条报道

\item {} 
服务蓝图

\item {} 
价值没计

\item {} 
触发设计

\item {} 
体验没计

\item {} 
峰终设计

\item {} 
节奏设计

\item {} 
用户测试

\item {} 
原型验证

\end{itemize}


\paragraph{产品架构设计}
\label{\detokenize{chapter_idea/design:id19}}
\begin{figure}[H]
\centering
\capstart

\noindent\sphinxincludegraphics{{product_arch_design}.png}
\caption{产品架构设计\sphinxhref{https://tangjie.me/blog/171.html}{12}\sphinxfootnotemark[460]}\label{\detokenize{chapter_idea/design:id21}}\end{figure}
%
\begin{footnotetext}[460]\sphinxAtStartFootnote
\sphinxnolinkurl{https://tangjie.me/blog/171.html}
%
\end{footnotetext}\ignorespaces 

\paragraph{AI产品设计原则 10\sphinxfootnotemark[461]}
\label{\detokenize{chapter_idea/design:ai-10}}%
\begin{footnotetext}[461]\sphinxAtStartFootnote
\sphinxnolinkurl{https://zhuanlan.zhihu.com/p/80134682}
%
\end{footnotetext}\ignorespaces \begin{enumerate}
\sphinxsetlistlabels{\arabic}{enumi}{enumii}{}{.}%
\item {} 
数据驱动:用户行为日志是用户需求挖掘的宝地,用日志分析当前提供服务的满足度,以及挖掘潜在用户需求,迭代验证产品方案,闭环跑通。

\item {} 
容错设计:由于推理的概率性错误,加上不能一蹴而就。就需要有容错设计来解决。(为了应对算法失灵的情况,形色还配备了专业的人士进行人工鉴定及解答。\sphinxhref{http://www.woshipm.com/ai/2296413.htmls}{13}%
\begin{footnote}[462]\sphinxAtStartFootnote
\sphinxnolinkurl{http://www.woshipm.com/ai/2296413.htmls}
%
\end{footnote})

\item {} 
上新的机制:为了节省研发和设计AI模型的时间成本,要在初期就定义好产品的
信息架构:ref:\sphinxcode{\sphinxupquote{information\_infra}},并尽量复用已有的产品样式

\end{enumerate}


\subsubsection{GTM(go to marketing)}
\label{\detokenize{chapter_idea/GTM:gtm-go-to-marketing}}\label{\detokenize{chapter_idea/GTM::doc}}
市场推广、培训、成本核算。


\paragraph{业绩相关}
\label{\detokenize{chapter_idea/GTM:id1}}
业绩体现包括:用户增长+核心路径转化提升+营收增长等。\sphinxhref{http://dadaghp.com/index/index/article\_detail/id/670.html}{13}%
\begin{footnote}[463]\sphinxAtStartFootnote
\sphinxnolinkurl{http://dadaghp.com/index/index/article\_detail/id/670.html}
%
\end{footnote}

业绩体现一定要量化,用具体的数字一是有说服力,尤其是应聘行业内的岗位,二是熟记数据可以体现对产品的关注程度,第三精准的数据阐述逻辑可以体现产品经理的数据分析能力。

通常来讲,产品营收=用户规模x转化率x单笔毛利\sphinxhyphen{}成本,我们再进行拆解
产品营收=(存量用户+新增用户x留存率)x转化率x单笔毛利\sphinxhyphen{}成本,由上面公式可见,要想提高产品营收那重要的就是提升新增用户数、用户留存率、转化率以及降低成本。


\paragraph{产品运营 1\sphinxfootnotemark[464]}
\label{\detokenize{chapter_idea/GTM:yunying}}\label{\detokenize{chapter_idea/GTM:id2}}%
\begin{footnotetext}[464]\sphinxAtStartFootnote
\sphinxnolinkurl{https://baike.baidu.com/item/\%E4\%BA\%A7\%E5\%93\%81\%E8\%BF\%90\%E8\%90\%A5/1978562}
%
\end{footnotetext}\ignorespaces 

\subparagraph{了解运营的目的 3\sphinxfootnotemark[465]}
\label{\detokenize{chapter_idea/GTM:id3}}%
\begin{footnotetext}[465]\sphinxAtStartFootnote
\sphinxnolinkurl{https://www.zhihu.com/pub/reader/119980992/chapter/1284104607329615872}
%
\end{footnotetext}\ignorespaces 
为了让产品经理除了单纯地进行产品功能优化,还要思考如何让更多的用户来使用产品,只有让更多的用户使用了产品,产品才能实现更好的自我优化

产品的本质是解决用户的痛点,而运营的本质是通过各种\sphinxstylestrong{方法和渠道把产品的价值传递给用户}。


\subparagraph{在生命周期中的作用}
\label{\detokenize{chapter_idea/GTM:id4}}
\begin{figure}[H]
\centering
\capstart

\noindent\sphinxincludegraphics{{operation_in_life_cycle}.png}
\caption{运营在生命周期中的作用\sphinxhref{https://mp.weixin.qq.com/s?\_\_biz=MjM5MzE3MDQ3Mw==\&mid=2650404998\&idx=3\&sn=e4bf27058ac6a697bfb1ae3cbb319e14\&chksm=be964dc089e1c4d613d4dcf763e01fbc65dee8b08136e34ebf62c1d22cbc7d83c58502416f2a\&scene=21\#wechat\_redirect}{14}\sphinxfootnotemark[466]}\label{\detokenize{chapter_idea/GTM:id21}}\end{figure}
%
\begin{footnotetext}[466]\sphinxAtStartFootnote
\sphinxnolinkurl{https://mp.weixin.qq.com/s?\_\_biz=MjM5MzE3MDQ3Mw==\&mid=2650404998\&idx=3\&sn=e4bf27058ac6a697bfb1ae3cbb319e14\&chksm=be964dc089e1c4d613d4dcf763e01fbc65dee8b08136e34ebf62c1d22cbc7d83c58502416f2a\&scene=21\#wechat\_redirect}
%
\end{footnotetext}\ignorespaces 

\subparagraph{产品与运营的关系}
\label{\detokenize{chapter_idea/GTM:id5}}
产品经理主要是负责产品设计阶段的工作,运营是负责产品推广的工作,把产品当成一个孩子来看,产品设计阶段就是“生孩子”,产品推广阶段就是“把孩子养大,而且还必须养好”。

产品设计阶段主要是界定并为用户提供长期价值,在运营阶段主要是完善产品长期价值,并提供给用户短期价值。

很多产品的长期价值往往用户一时半会儿感知不到,需要运营创造一些短期价值去刺激用户使用和体验,并根据用户的持续反馈调整、迭代、优化来完善长期价值。

三驾马车:内容、用户、活动


\subparagraph{作用}
\label{\detokenize{chapter_idea/GTM:id6}}
运营相当于一个保姆,是一项从内容建设,用户维护,活动策划三个层面来管理产品内容和用户的职业。
\begin{enumerate}
\sphinxsetlistlabels{\arabic}{enumi}{enumii}{}{.}%
\item {} 
内容建设:建立标准,挽留、裂变:防止劣质的内容驱逐优质的用户

\item {} 
用户维护:挽留:建立完善Q\&A机制,解决用户投诉和困难;拉新:主动邀请有价值的用户来使用产品。

\item {} 
活动策划:煽动用户互动,加强产品品牌。产品运营最接近用户,需求质量高。

\end{enumerate}


\subparagraph{工作内容 2\sphinxfootnotemark[467]}
\label{\detokenize{chapter_idea/GTM:id7}}%
\begin{footnotetext}[467]\sphinxAtStartFootnote
\sphinxnolinkurl{https://www.zhihu.com/pub/reader/119911878/chapter/1283841129226715136}
%
\end{footnotetext}\ignorespaces 
从内容编辑、SEO 优化、产品推广、SEM、活动运营、社群运营、BD
合作等基本版块到整体统筹独立负责 APP 社区体系搭建


\subparagraph{种子期:关注留存率}
\label{\detokenize{chapter_idea/GTM:id8}}
关注自然留存率:
\begin{itemize}
\item {} 
留存率好——核心功能被认可用户群体是否有偏差

\item {} 
留存率不好——产品核心功能不明确

\end{itemize}


\subparagraph{种子用户特点}
\label{\detokenize{chapter_idea/GTM:id9}}\begin{itemize}
\item {} 
强需求

\item {} 
核心功能敏感

\item {} 
愿意尝试使用并传播

\end{itemize}


\subparagraph{启动法}
\label{\detokenize{chapter_idea/GTM:id10}}\begin{itemize}
\item {} 
事件法:试用为内容型产品地推法:

\item {} 
020产品:坚持坚决的执行地推

\item {} 
马甲法:社交产品,运营横仿用户发

\item {} 
传染法:基于特定的用户群兴趣点进行关系链传播

\end{itemize}


\subparagraph{爆发期:提高来源量}
\label{\detokenize{chapter_idea/GTM:id11}}
产品:完善产品核心功能,暂时不要新功能

外延做深种子群体扩大范围推广传播

运营目标:拉来更多的用户,来源量(一个产品定位越精确,推广的难度就越低。)
\begin{itemize}
\item {} 
做深种子群体扩大范围推广传插

\item {} 
做广跨领域或行业展开推广

\item {} 
买:raw\sphinxhyphen{}latex:\sphinxtitleref{培养流量}:培养网络红人,打广告

\item {} 
傍大款:与大平台合作,例如从酷6看网络运营度贴吧合作\sphinxhref{https://www.jianshu.com/p/b62409f10470}{9}%
\begin{footnote}[468]\sphinxAtStartFootnote
\sphinxnolinkurl{https://www.jianshu.com/p/b62409f10470}
%
\end{footnote}

\item {} 
靠关系:通过关系链网络化传播

\item {} 
装有钱:运营资金雄厚(补贴,激励方式),目标是拉来更多的用户

\item {} 
共同富裕:与上传视频者共享广告利益

\item {} 
导流:游戏依靠QQ微信的周边。

\end{itemize}


\subparagraph{平台期活跃度}
\label{\detokenize{chapter_idea/GTM:id12}}
平台期是为了下个爆发期而做的运营准备

产品:提高\sphinxhref{https://blog.csdn.net/lanxingfeifei/article/details/89843332}{用户体验}%
\begin{footnote}[469]\sphinxAtStartFootnote
\sphinxnolinkurl{https://blog.csdn.net/lanxingfeifei/article/details/89843332}
%
\end{footnote},考虑系统架构是否能够继续支持下一轮爆发性增长
\begin{itemize}
\item {} 
举办活动

\item {} 
用户等级

\item {} 
新功能拉动
\sphinxhref{https://pic4.zhimg.com/v2-670698cb727b90e20895360b2fe85ea8\_r.jpg?source=1940ef5c}{8}%
\begin{footnote}[470]\sphinxAtStartFootnote
\sphinxnolinkurl{https://pic4.zhimg.com/v2-670698cb727b90e20895360b2fe85ea8\_r.jpg?source=1940ef5c}
%
\end{footnote}

\end{itemize}


\subparagraph{好文案}
\label{\detokenize{chapter_idea/GTM:id13}}
终极文案不是突出产品优势和品牌,而是唤起消费者的消费冲动

更多:
\begin{itemize}
\item {} 
\sphinxurl{https://zhuanlan.zhihu.com/p/93474666}

\end{itemize}


\subparagraph{NEXT方法 5\sphinxfootnotemark[471]}
\label{\detokenize{chapter_idea/GTM:next-5}}%
\begin{footnotetext}[471]\sphinxAtStartFootnote
\sphinxnolinkurl{http://www.changgpm.com/}
%
\end{footnotetext}\ignorespaces \begin{itemize}
\item {} 
N(NEW):做出了 (New)新的产品价值(WHY)

\item {} 
E(Expolsive):无法脱颖而出的产品,其实是因为做得还不够极致。

\item {} 
X(XENOGENEIC):必须走出MRD,多做田野调查,多与用户面对面交流。

\item {} 
T(TALENT):越用这些产品用户越觉得这些产品懂自己。

\end{itemize}


\subparagraph{体系的建立}
\label{\detokenize{chapter_idea/GTM:id14}}
比如天猫京东等电商平台为了让商家在618和双11中更好的准备打折促销活动,会在大促之前发布商家作战地图,商家作战地图纵向涵盖了商家所有的活动,比如推广、视觉、商品、社交、内容、用户等,时间跨度包含了筹备期、蓄水期、预热期
、售卖期 、爆发期、返场期、总结复盘等。

运营模型的整理的关键来自于用户的关键行为,比如上图中我们思考的用户从哪里来(线上和线下),用户如何认知我们(认知),如何让用户进店(进店),用户进店之后怎么转化(购买),用户怎么沉淀下来(社群),以及最后如何让用户给我们的产品做自传播(传播)。


\subparagraph{指标确认}
\label{\detokenize{chapter_idea/GTM:id15}}
在互联网中,运营指标指的是需要成功完成活动的数据指标,如电商运营中关于流量性的指标独立访客数(UV)、页面访问数(PV)、成交金额(GMV)、销售金额等。

而工具类产品的核心指标是用户体验。算法工程师其实存在一个比较大的困惑,如何用数据去度量用户体验这个比较虚的目标。最常用的一个度量指标就是用户净推荐值,具体来讲就是去问用户“你是否愿意将这个产品推荐给你的朋友或者同事”。

更进一步,我们可以设置一个北极星指标,例如产品日活,然后结合每个模块进行细致拆分。我们可能需要关注每日新增用户,核心用户在今天的贡献数值,打开率,使用频次,目标达成率,分享率等等。当我们把这些核心目标真正拆解清楚的时候,我们就有了主要指标,就是我们应该怎么样去把这个产品做好。

\begin{figure}[H]
\centering
\capstart

\noindent\sphinxincludegraphics{{goal_fenjie}.jpg}
\caption{核心目标的拆解}\label{\detokenize{chapter_idea/GTM:id22}}\end{figure}


\subparagraph{模块划分}
\label{\detokenize{chapter_idea/GTM:id16}}
为了达到运营活动的运营效果需要把整个运营活动按职能进行拆分。比如常见的运营职能有内容运营、数据运营、活动运营、用户运营、渠道运营、市场运营、会员运营、社群运营、商家运营等。


\subparagraph{ASO}
\label{\detokenize{chapter_idea/GTM:aso}}
应用商店优化(ASO)是指提高应用或游戏在应用商店中的曝光度,以提高应用的自然下载量为目标的过程。当应用在各种搜索条件中排名靠前,在排行榜中保持较高的位置,或在应用商店中获得推荐时,它们就更容易被发现。


\subparagraph{AI产品的运营 6\sphinxfootnotemark[472]}
\label{\detokenize{chapter_idea/GTM:ai-6}}%
\begin{footnotetext}[472]\sphinxAtStartFootnote
\sphinxnolinkurl{http://www.xmamiga.com/3573/}
%
\end{footnotetext}\ignorespaces 
上线、包装、宣传,产品经理尽量评估产品的商业化和产品化效果,动态调整算法模型的研发投入量。


\paragraph{营销}
\label{\detokenize{chapter_idea/GTM:id17}}
根据目标用户、产品特点及品牌塑造需要,进行营销及公关策略的制定和执行,以实现有效传播、危机化解、产品目标达成。\sphinxhref{https://t.qidianla.com/1175149.html}{12}%
\begin{footnote}[473]\sphinxAtStartFootnote
\sphinxnolinkurl{https://t.qidianla.com/1175149.html}
%
\end{footnote}

推荐书籍《定位》《流量池》《运营之光》《增长黑客》《参与感》《消费者行为学》)《爆款文案》\sphinxhref{http://www.woshipm.com/pmd/3024508.html}{10}%
\begin{footnote}[474]\sphinxAtStartFootnote
\sphinxnolinkurl{http://www.woshipm.com/pmd/3024508.html}
%
\end{footnote}


\subparagraph{营销很好,没有盈利}
\label{\detokenize{chapter_idea/GTM:id18}}\begin{enumerate}
\sphinxsetlistlabels{\arabic}{enumi}{enumii}{}{.}%
\item {} 
提高用户量;

\item {} 
客单价(用户价值);

\item {} 
成本管理;

\item {} 
增加资产的周转率;

\item {} 
寻找“增长杠杆”。

\end{enumerate}


\subparagraph{饥饿营销}
\label{\detokenize{chapter_idea/GTM:id19}}
真正目的不是为了利润,而是为了品牌附加值。

前提:
\begin{enumerate}
\sphinxsetlistlabels{\arabic}{enumi}{enumii}{}{.}%
\item {} 
产品具备不可替代性

\item {} 
消费者心智不成熟

\item {} 
市场竞争不激烈。

\end{enumerate}

副作用:
\begin{enumerate}
\sphinxsetlistlabels{\arabic}{enumi}{enumii}{}{.}%
\item {} 
客户流失。过度饥饿营销,就是将客户“送”给竞争对手。

\item {} 
顾客反感。过度饥饿营销,会让消费者饿到冷静,觉得被愚弄,对品牌产生厌恶。

\end{enumerate}


\subparagraph{互动营销MIND方法论}
\label{\detokenize{chapter_idea/GTM:mind}}\begin{enumerate}
\sphinxsetlistlabels{\arabic}{enumi}{enumii}{}{.}%
\item {} 
M(Measurability):用可衡量的效果体现在线营销的有效性、可持续性以及科学性。

\item {} 
I(Interactive
Experience):用互动式的体验提供高质量的创新体验和妙趣横生的网络生活感受。

\item {} 
N(Navigation):用精确化的导航保障目标用户的精准选择和在线营销体验的效果。

\item {} 
D(Differentiation):用差异化的定位创造在线营销的不同,满足客户独特性的需求。

\end{enumerate}


\subparagraph{More}
\label{\detokenize{chapter_idea/GTM:more}}
\sphinxurl{https://www.niaogebiji.com/}


\paragraph{销售}
\label{\detokenize{chapter_idea/GTM:id20}}
永远把客户的利益放在第一位,尽全力帮助客户成功。

如果你是客户,你想要什么样的方案


\subsubsection{时间}
\label{\detokenize{chapter_idea/time:id1}}\label{\detokenize{chapter_idea/time::doc}}

\paragraph{长期主义 1\sphinxfootnotemark[475]}
\label{\detokenize{chapter_idea/time:id2}}%
\begin{footnotetext}[475]\sphinxAtStartFootnote
\sphinxnolinkurl{https://weread.qq.com/web/reader/8d632bc07208ed1c8d697c4k1c3321802231c383cd30bb3}
%
\end{footnotetext}\ignorespaces 
长期主义提醒着我们:做任何事情,不但要想着提供用户价值、赚到今天的钱,还要考虑如何给将来做铺垫,让公司可以顺利地完成项目产品化、产品平台化、平台生态化,并且在业务以外沉淀组织能力,在不同的行业之间适时切换,建立跨周期的产品创新型公司。


\subsubsection{风险}
\label{\detokenize{chapter_idea/risk:id1}}\label{\detokenize{chapter_idea/risk::doc}}

\subsection{全流程知识}
\label{\detokenize{chapter_knowledge/index:chap-skill}}\label{\detokenize{chapter_knowledge/index:id1}}\label{\detokenize{chapter_knowledge/index::doc}}
\begin{figure}[H]
\centering
\capstart

\noindent\sphinxincludegraphics{{whole_process}.png}
\caption{整个产品实现的流程和生命周期{[}1{]}}\label{\detokenize{chapter_knowledge/index:id2}}\end{figure}

\begin{figure}[H]
\centering
\capstart

\noindent\sphinxincludegraphics{{whole_project}.png}
\caption{完整项目的角色}\label{\detokenize{chapter_knowledge/index:id3}}\end{figure}

\begin{figure}[H]
\centering
\capstart

\noindent\sphinxincludegraphics{{PM_work}.png}
\caption{产品经理的工作职能{[}6{]}}\label{\detokenize{chapter_knowledge/index:id4}}\end{figure}
\begin{itemize}
\item {} 
领域专家:用无人机看书,农业专家对二月份的树该有大致的样子是清楚的。甲方翻译成任务交给数据科学家(乙方)去提高。

\item {} 
数据科学家:关心实际业务问题。不断开发新领域。

\item {} 
AI专家:不仅能用,精度性能。领域做深。{[}4{]}

\end{itemize}

自然语言处理不如图片应用做的好。

\sphinxurl{https://i.am.ai/roadmap}

AI产品经理的进阶地图 \sphinxhyphen{} 知乎
\sphinxurl{https://www.zhihu.com/club/1266018382773624832/post/1266047295663403008}

AI产品经理是“产品经理”这个职业的一个分支,在成为AI产品经理之前,你需要先成为一名合格的产品经理(拥有产品思维、产品经理相关的知识体系)

第0阶段:产品经理基础知识 ————————————————
产品经理知识架构(C端产品经理、B端产品经理、策略产品经理、数据产品经理)
\begin{enumerate}
\sphinxsetlistlabels{\arabic}{enumi}{enumii}{}{.}%
\item {} 
商业分析、市场分析、需求分析;

\item {} 
产品设计:整体设计、详细设计;

\item {} 
项目管理:需求池管理、版本迭代管理;

\item {} 
产品运营:市场运营、用户运营、内容运营;

\end{enumerate}

第1阶段:AI产品经理的基础知识

————————————————
1.AI产品经理基础:人工智能概论、AI产品经理职责、资源管理(数据集、算法模型、策略、软硬件资源、需求管理)、工作流程、核心技术、行业应用
2.机器学习:数据采集、数据探索、数据预处理、ML模型、模型训练、场景应用
3.深度学习:基于深度神经网络的模型应用(CNN、RNN、GAN)
4.强化学习:基于智能体/环境接口的马尔科夫过程模型;

第2阶段:AI领域的纵向深度挖掘

———————————————— 1.模式识别:图像识别、人脸识别、语音识别、视频识别等;
2.自然语言处理(NLP):自然语言理解(NLU)、自然语言生成(NLG)、知识图谱{[}2{]}等
3.智能语音:语音识别、语音合成、声纹识别
4.智能软硬件:芯片{[}5{]}、智能软硬件项目流程;
5.专家系统:基于专家知识架构的AI综合应用 6.机器人:基于生物智能的AI应用
7.其他领域的纵向深度挖掘


\subsubsection{商业化产品“七步设计法” 1\sphinxfootnotemark[476]}
\label{\detokenize{chapter_knowledge/steps:id1}}\label{\detokenize{chapter_knowledge/steps::doc}}%
\begin{footnotetext}[476]\sphinxAtStartFootnote
\sphinxnolinkurl{http://www.woshipm.com/pd/3784247.html}
%
\end{footnotetext}\ignorespaces 
方法论试图提炼出从0到1设计商业化产品的通用思路,重点介绍宏观设计逻辑,即具备普适性的全链路商业化产品设计流程,而不倾向于介绍某款单品类产品的设计细节。但也会在各个流程环节会特别说明适用的产品,比如适合to
C产品还是to B产品,或者适合广告产品、会员产品还是其他商业化产品。


\paragraph{如何设计一款专业的、具备竞争力的商业化产品呢?}
\label{\detokenize{chapter_knowledge/steps:id2}}
你们可能会从不同的视角给出不同的答案,这可能是岗位差异造成的。然而,当我们从公司的视角考虑商业化产品设计时,就必须具备全局思维,即面向商业化全链路来设计可闭环变现的产品。

在商业流通市场中,无论商业化产品属于哪种形态或者哪个种类,产品所遵循的商业化底层逻辑是相通的,即无论你的变现模式是卖流量、卖软件、卖服务,还是其他的变现模式,最终你卖的都是“商品”,都需要遵循商业化规律和一些普适的商业操作标准。

\begin{figure}[H]
\centering
\capstart

\noindent\sphinxincludegraphics{{steps}.png}
\caption{打造一份优秀作品的步骤}\label{\detokenize{chapter_knowledge/steps:id5}}\end{figure}


\paragraph{“七步设计法”}
\label{\detokenize{chapter_knowledge/steps:id3}}
\begin{figure}[H]
\centering
\capstart

\noindent\sphinxincludegraphics{{PM_process}.png}
\caption{一个完整的产品案例\sphinxhref{https://www.zhihu.com/search?type=content\&q=AI\%E4\%BA\%A7\%E5\%93\%81\%E7\%BB\%8F\%E7\%90\%86}{3}\sphinxfootnotemark[477]}\label{\detokenize{chapter_knowledge/steps:id6}}\end{figure}
%
\begin{footnotetext}[477]\sphinxAtStartFootnote
\sphinxnolinkurl{https://www.zhihu.com/search?type=content\&q=AI\%E4\%BA\%A7\%E5\%93\%81\%E7\%BB\%8F\%E7\%90\%86}
%
\end{footnotetext}\ignorespaces \begin{enumerate}
\sphinxsetlistlabels{\arabic}{enumi}{enumii}{}{.}%
\item {} 
输出BRD、资源评估报告等。

\item {} 
输出产品设计规划、解决方案设计规划等。

\item {} 
输出PRD(产品需求文档)、产品交付清单等。

\item {} 
输出产品报价单、SKU目录、财务模型等。

\item {} 
输出售卖渠道策略、售卖政策、销售协议等。

\item {} 
输出各类商业化产品包装资料、商业化产品官网等。

\item {} 
输出产品发布管理文档、各类产品培训资料等。

\end{enumerate}

\begin{figure}[H]
\centering
\capstart

\noindent\sphinxincludegraphics{{work_output}.png}
\caption{工作输出\sphinxhref{https://shimo.im/docs/vyCrK3rQQ6KC9Ryp/read}{4}\sphinxfootnotemark[478]}\label{\detokenize{chapter_knowledge/steps:id7}}\end{figure}
%
\begin{footnotetext}[478]\sphinxAtStartFootnote
\sphinxnolinkurl{https://shimo.im/docs/vyCrK3rQQ6KC9Ryp/read}
%
\end{footnotetext}\ignorespaces 

\paragraph{工作}
\label{\detokenize{chapter_knowledge/steps:id4}}\begin{enumerate}
\sphinxsetlistlabels{\arabic}{enumi}{enumii}{}{.}%
\item {} 
计划:使用计划系列卡来定义产品的关键要素并调整团队成员在一个战线节奏和目标上。

\item {} 
调研:使用调研系列卡来收集有关市场和用户的数据并分析他们的可用性。

\item {} 
理解:使用理解系列卡来合成相关数据并发现有价值的见解。

\item {} 
思考:使用思考系列卡来提升你的洞察力并创造有价值的新想法。

\item {} 
评估:使用评估系列卡来检测你的构思并针对现实情况收集反馈意见。\sphinxhref{https://www.zhihu.com/search?type=content\&q=AI\%E4\%BA\%A7\%E5\%93\%81\%E7\%BB\%8F\%E7\%90\%86}{3}%
\begin{footnote}[479]\sphinxAtStartFootnote
\sphinxnolinkurl{https://www.zhihu.com/search?type=content\&q=AI\%E4\%BA\%A7\%E5\%93\%81\%E7\%BB\%8F\%E7\%90\%86}
%
\end{footnote}

\end{enumerate}

\begin{figure}[H]
\centering
\capstart

\noindent\sphinxincludegraphics{{PM_work}.png}
\caption{PM work}\label{\detokenize{chapter_knowledge/steps:id8}}\end{figure}


\subsubsection{BRD}
\label{\detokenize{chapter_knowledge/BRD:brd}}\label{\detokenize{chapter_knowledge/BRD::doc}}

\paragraph{BRD —> MRD —> PRD}
\label{\detokenize{chapter_knowledge/BRD:brd-mrd-prd}}
BRD —> MRD —> PRD是一个从高到底的逐层递进的关系
\sphinxhref{https://blog.csdn.net/liwei16611/article/details/106638921}{12}%
\begin{footnote}[480]\sphinxAtStartFootnote
\sphinxnolinkurl{https://blog.csdn.net/liwei16611/article/details/106638921}
%
\end{footnote}
\begin{itemize}
\item {} 
BRD从战略高度告诉我们做什么产品;

\item {} 
MRD从战术的角度告诉我们怎么做;

\item {} 
PRD非常细化的告诉我们做成什么样。

\end{itemize}

BRD决定了产品的商业价值、PRD决定了项目质量水平、MRD在中间起到一个承上启下的作用,质量好坏直接影响到产品项目的开展,并直接影响到公司产品战略意图的实现。

PRD、BRD和MRD,一起被认为是从市场到产品需要建立的文档规范。


\paragraph{定义}
\label{\detokenize{chapter_knowledge/BRD:id1}}
BRD指的是商业需求文档(Business Requirement
Document)。在这篇的文档当中不会有详细的产品规划,只会有基于市场调查和用户需求调查的产品构思。
\sphinxhref{http://www.woshipm.com/pmd/178527.html}{1}%
\begin{footnote}[481]\sphinxAtStartFootnote
\sphinxnolinkurl{http://www.woshipm.com/pmd/178527.html}
%
\end{footnote}

BRD
是给谁看的呢?老板、投资人、股东,目的是让他们知道这款产品如何给公司盈利。BRD
的撰写侧重点是需求描述、盈利模式。产品总监或者产品VP(Vice
President,副总监)才需要写BRD,初级产品经理基本接触不到BRD

这篇文档常以PPT等形式,内容涉及市场分析、销售策略、盈利预测等\sphinxhref{https://quizlet.com/129588206/\%E4\%BA\%BA\%E4\%BA\%BA\%E9\%83\%BD\%E6\%98\%AF\%E4\%BA\%A7\%E5\%93\%81\%E7\%BB\%8F\%E7\%90\%86-\%E7\%AC\%94\%E8\%AE\%B0-flash-cards/}{8}%
\begin{footnote}[482]\sphinxAtStartFootnote
\sphinxnolinkurl{https://quizlet.com/129588206/\%E4\%BA\%BA\%E4\%BA\%BA\%E9\%83\%BD\%E6\%98\%AF\%E4\%BA\%A7\%E5\%93\%81\%E7\%BB\%8F\%E7\%90\%86-\%E7\%AC\%94\%E8\%AE\%B0-flash-cards/}
%
\end{footnote},以数据说服他们,发掘现有产品改进的可能性或帮产品的立项,来获得公司资源支持。

\begin{figure}[H]
\centering
\capstart

\noindent\sphinxincludegraphics{{BRD_report}.png}
\caption{汇报对象}\label{\detokenize{chapter_knowledge/BRD:id18}}\end{figure}


\paragraph{顺势而为 7\sphinxfootnotemark[483]}
\label{\detokenize{chapter_knowledge/BRD:id2}}%
\begin{footnotetext}[483]\sphinxAtStartFootnote
\sphinxnolinkurl{https://www.jianshu.com/p/a4b1fd94b49a}
%
\end{footnotetext}\ignorespaces 
\begin{figure}[H]
\centering
\capstart

\noindent\sphinxincludegraphics{{point_filter}.png}
\caption{点子过滤}\label{\detokenize{chapter_knowledge/BRD:id19}}\end{figure}


\paragraph{产品立项流程 4\sphinxfootnotemark[484]}
\label{\detokenize{chapter_knowledge/BRD:id3}}%
\begin{footnotetext}[484]\sphinxAtStartFootnote
\sphinxnolinkurl{https://www.bilibili.com/video/BV1254y1D7Ht?from=search\&seid=14167562900175777805}
%
\end{footnotetext}\ignorespaces 
项目概述 商业价值 项目的目标 项目风险 项目干系人、组织和其他产品


\paragraph{前期调研 3\sphinxfootnotemark[485]}
\label{\detokenize{chapter_knowledge/BRD:id4}}%
\begin{footnotetext}[485]\sphinxAtStartFootnote
\sphinxnolinkurl{https://www.bilibili.com/video/BV1wz4y1y7sg}
%
\end{footnotetext}\ignorespaces 
人群特征 需求特点 用户价值 体量规模 竞争优势 产品定位 资源能力 成本收益


\paragraph{产品目的 2\sphinxfootnotemark[486]}
\label{\detokenize{chapter_knowledge/BRD:id5}}%
\begin{footnotetext}[486]\sphinxAtStartFootnote
\sphinxnolinkurl{http://www.woshipm.com/pmd/21446.html}
%
\end{footnotetext}\ignorespaces 
提出一个清晰、简明的价值主张,让它很容易被接受,要让产品团队、管理人员、用户、市场人员清楚的明白这个产品到底是什么意图。

考虑“velevator pitch”(电梯间演讲、电梯行销)测试。


\paragraph{一个中心}
\label{\detokenize{chapter_knowledge/BRD:id6}}
收益!来生存


\paragraph{多个基本点}
\label{\detokenize{chapter_knowledge/BRD:id7}}\begin{enumerate}
\sphinxsetlistlabels{\arabic}{enumi}{enumii}{}{.}%
\item {} 
市场调查报告;

\item {} 
竞争对手报告;

\item {} 
用户需求调研报告;

\item {} 
产品功能构思;

\item {} 
产品运营构思;

\item {} 
收益分析;

\item {} 
风险分析;

\item {} 
其他。

\end{enumerate}


\paragraph{战略}
\label{\detokenize{chapter_knowledge/BRD:id8}}
方法:GE矩阵,SPAN战略定位分析,BCG矩阵,价值链分析法,kano模型
\sphinxhref{https://www.jianshu.com/p/1d733b336f76s}{10}%
\begin{footnote}[487]\sphinxAtStartFootnote
\sphinxnolinkurl{https://www.jianshu.com/p/1d733b336f76s}
%
\end{footnote}
\begin{itemize}
\item {} 
生命周期产品战略中引入期=创新扩散理论中尝鲜者、早期使用者+AARRR模型中获取用户阶段(KOL)

\item {} 
竞争产品战略中领先者=Ansoff矩阵+波特五力(竞争者门槛提高,替代品转型升级)+GE矩阵+BGC矩阵(产品组合)

\item {} 
梅奥主义产品战略=GE矩阵+价值链方法(内部)+帕累托改进(总资源不变的情况下,优化分配资源)

\end{itemize}


\paragraph{例子}
\label{\detokenize{chapter_knowledge/BRD:id9}}
\begin{figure}[H]
\centering
\capstart

\noindent\sphinxincludegraphics{{BRD}.jpg}
\caption{支付宝用户事业部产品提案模板}\label{\detokenize{chapter_knowledge/BRD:id20}}\end{figure}


\paragraph{可行性评估 3\sphinxfootnotemark[488]}
\label{\detokenize{chapter_knowledge/BRD:id10}}%
\begin{footnotetext}[488]\sphinxAtStartFootnote
\sphinxnolinkurl{https://www.bilibili.com/video/BV1wz4y1y7sg}
%
\end{footnotetext}\ignorespaces 
MVP:核心是试错,有反馈渠道。
PMF:Dropbox用视频介绍产品的功能,来测试反馈。核心功能内测

最短时间 最小成本 可行性验证


\paragraph{可用性测试}
\label{\detokenize{chapter_knowledge/BRD:id11}}

\subparagraph{确定测试目标:}
\label{\detokenize{chapter_knowledge/BRD:id12}}\begin{itemize}
\item {} 
产品设计方案

\item {} 
测试功能点

\item {} 
A/B测试

\item {} 
产品改进方案测试

\end{itemize}


\subparagraph{定义用户:}
\label{\detokenize{chapter_knowledge/BRD:id13}}\begin{itemize}
\item {} 
可从用户访谈和问卷调查中选择

\item {} 
存量用户或者新用户

\end{itemize}


\subparagraph{测试过程记录:}
\label{\detokenize{chapter_knowledge/BRD:id14}}\begin{itemize}
\item {} 
录屏、录音和摄像

\item {} 
记录A/B选项结果

\item {} 
页面埋点追踪

\item {} 
过程中的疑惑点,改进点及时其他特殊情况

\end{itemize}


\subparagraph{结果分析根据过程记录总结、修改方案,如:}
\label{\detokenize{chapter_knowledge/BRD:id15}}\begin{itemize}
\item {} 
通过统计分析追踪结果

\item {} 
AB测试结果得出改进方案

\end{itemize}


\paragraph{四轮 MVP 框架}
\label{\detokenize{chapter_knowledge/BRD:mvp}}
VUCA 的中文含义分别对应着易变性、不确定性、复杂性和模糊性。V:Volatility
易变性U:Uncertainty 不确定性C:Complexity 复杂性A:Ambiguity 模糊性

如今VUCA时代信息无时无刻不在变化,用户的需求无时无刻不在变化。

\begin{figure}[H]
\centering
\capstart

\noindent\sphinxincludegraphics{{MVP}.png}
\caption{MVP框架}\label{\detokenize{chapter_knowledge/BRD:id21}}\end{figure}
\begin{enumerate}
\sphinxsetlistlabels{\arabic}{enumi}{enumii}{}{.}%
\item {} 
Paperwork:产出物是纸面研究的结论,用的方法是 Discovery
Sprint,探索冲刺。

\item {} 
Prototype:在方案层面“先发散,后收敛”,做出原型,获得反馈后,不断修正原型,用的方法叫
Design Sprint,设计冲刺。

\item {} 
Product:验证的重点是真实产品是否可以培养出用户习惯,用户愿意用,能更高效地解决用户需求、创造价值,并且让用户愿意反复使用。这时候,我们会关注某些和用户留存有关的指标。

\item {} 
Promotion:做小规模推广尝试,测试渠道,逐步确定优选渠道,降低分销成本。对应的方法论是
Distribution Sprint,分销冲刺。

\end{enumerate}

注意:
\begin{enumerate}
\sphinxsetlistlabels{\arabic}{enumi}{enumii}{}{.}%
\item {} 
用户参与都是必须的

\item {} 
过滤器的开口应该越来越小

\item {} 
在每一轮停留的时间、投入的资源也不尽相同

\item {} 
这四轮走完,产品也才刚刚上路

\end{enumerate}

设计冲刺分为“理解领域、聚焦方向、发散解法、选择解法、制作原型、用户测试”六步。


\paragraph{项目风险 RAID 4\sphinxfootnotemark[489]}
\label{\detokenize{chapter_knowledge/BRD:raid-4}}%
\begin{footnotetext}[489]\sphinxAtStartFootnote
\sphinxnolinkurl{https://www.bilibili.com/video/BV1254y1D7Ht?from=search\&seid=14167562900175777805}
%
\end{footnotetext}\ignorespaces \begin{itemize}
\item {} 
Risk风险:会对项目产生负面影响的事件,事件可能发生的概率和随之对项目带来的影响

\item {} 
Assumption假设:知群可以预想到的因素,一旦发生就会促进项目成功(但不发生就没有促进效果)

\item {} 
Issues问题:在项目中任何不怡当的,需要管理和解决的事情,这些事情需要持续跟踪并记录

\item {} 
Dependence依赖:任何项目所依赖的或者依赖该项目的事件和工作,需要记录依赖实现的时间

\end{itemize}


\paragraph{PEST分析 5\sphinxfootnotemark[490]}
\label{\detokenize{chapter_knowledge/BRD:pest-5}}%
\begin{footnotetext}[490]\sphinxAtStartFootnote
\sphinxnolinkurl{https://zh.wikipedia.org/wiki/PEST\%E5\%88\%86\%E6\%9E\%90}
%
\end{footnotetext}\ignorespaces \begin{itemize}
\item {} 
政治因素包含了国家制度、国际关系、国家财政政策、国家福利及保障政策、国家货币政策、★行业的准入门槛、★行业的监管政策\sphinxhref{http://www.woshipm.com/pmd/2751064.html}{13}%
\begin{footnote}[491]\sphinxAtStartFootnote
\sphinxnolinkurl{http://www.woshipm.com/pmd/2751064.html}
%
\end{footnote}、租税政策、劳工法律、环境管制、贸易限制、关税与政治稳定。数据源:见国务院、行业监管网站。

\item {} 
经济因素有经济增长水平及增速、利率、汇率和通货膨胀率、各产业收入占比及增幅、人均可支配收入水平、人均纯收入水平、用户消费偏好及增幅\sphinxhref{http://www.woshipm.com/pmd/2751064.html}{13}%
\begin{footnote}[492]\sphinxAtStartFootnote
\sphinxnolinkurl{http://www.woshipm.com/pmd/2751064.html}
%
\end{footnote}。数据源:国家统计局来获取实时、准确、免费的数据

\item {} 
社会因素通常着重在文化观点,维度有人口环境(人口规模、人口成长率、年龄结构、人口分布、种族结构、婚姻状况、职业分布、资产水平)与文化背景(受教育程度、消费观念、价值观念、审美观点、风俗习惯、宗教信仰、健康意识、工作态度及安全需求。)数据源:艾瑞咨询、易观智库、企鹅智库、199IT。

\item {} 
科技因素包含生态与环境方面,决定进入障碍和最低有效生产水准,影响委外购买决策。科技因素着重在研发活动、自动化、技术诱因和科技发展的速度。

\end{itemize}

PEST分析与外部总体环境的因素互相结合就可归纳出SWOT分析中的机会与威胁。

\begin{figure}[H]
\centering
\capstart

\noindent\sphinxincludegraphics{{PEST_eg}.jpg}
\caption{直播行业的PEST\sphinxhref{https://www.zhihu.com/question/19749199/answer/1497421911}{11}\sphinxfootnotemark[493]}\label{\detokenize{chapter_knowledge/BRD:id22}}\end{figure}
%
\begin{footnotetext}[493]\sphinxAtStartFootnote
\sphinxnolinkurl{https://www.zhihu.com/question/19749199/answer/1497421911}
%
\end{footnotetext}\ignorespaces 

\paragraph{波特五力分析 6\sphinxfootnotemark[494]}
\label{\detokenize{chapter_knowledge/BRD:id16}}%
\begin{footnotetext}[494]\sphinxAtStartFootnote
\sphinxnolinkurl{https://zh.wikipedia.org/wiki/PEST\%E5\%88\%86\%E6\%9E\%90}
%
\end{footnotetext}\ignorespaces 
波特五力分析来定义出一个市场吸引力高低程度。

来自买方的议价能力、来自供应商的议价能力、来自潜在进入者的威胁、来自替代品的威胁和潜在竞争者的威胁
—
共同组合而演变出影响公司的第五种力量:来自现有竞争者的威胁。而每一种力量都由数项指标决定:
\begin{enumerate}
\sphinxsetlistlabels{\arabic}{enumi}{enumii}{}{.}%
\item {} 
来自买方的议价能力(Bargaining power of customers)

\item {} 
来自供应商的议价能力(Bargaining power of suppliers)

\item {} 
来自潜在进入者的威胁(Threat of new entrants)

\item {} 
来自替代品的威胁(Threat of substitutes)

\item {} 
来自现有竞争者的威胁(Competitive rivalry)

\end{enumerate}

\sphinxstylestrong{案例:}摩拜单车
\begin{itemize}
\item {} 
供应商:单车制作商,智能硬件提供商

\item {} 
购买者:骑行爱好者,短距离出行

\item {} 
潜在进入者:各大单车品牌,骑行爱好者社群

\item {} 
替代品:滴滴,电动车,汽车

\item {} 
行业竞争者:ofo,小蓝,小鸣\sphinxhref{https://t.qidianla.com/1156537.html}{14}%
\begin{footnote}[495]\sphinxAtStartFootnote
\sphinxnolinkurl{https://t.qidianla.com/1156537.html}
%
\end{footnote}

\end{itemize}


\subparagraph{来自买方的议价能力(Bargaining power of customers)}
\label{\detokenize{chapter_knowledge/BRD:bargaining-power-of-customers}}\begin{itemize}
\item {} 
买方集中度(buyer concentration to firm concentration ratio)

\item {} 
谈判杠杆(bargaining leverage)

\item {} 
买方购买数量(total buyer volume)

\item {} 
买方相对于厂商的转换成本(buyer switching costs relative to firm
switching costs)

\item {} 
买方获取资讯的能力(buyer information availability)

\item {} 
买方垂直整合(bargaining leverage,backward vertical
integration)的程度或可能性

\item {} 
现存替代品(availability of existing substitute products or
services)

\item {} 
消费者价格敏感度(buyer price sensitivity)

\item {} 
总消费金额(price of total purchase)

\end{itemize}


\subparagraph{来自供应商的议价能力(Bargaining power of suppliers)}
\label{\detokenize{chapter_knowledge/BRD:bargaining-power-of-suppliers}}\begin{itemize}
\item {} 
供应商相对于厂商的转换成本 (switching costs of firms in the
industry)

\item {} 
投入原料的差异化程度

\item {} 
现存的替代原料(presence of substitute inputs)

\item {} 
供应商集中度 (supplier concentration)

\item {} 
供应商垂直整合(bargaining leverage,forward vertical
integration)的程度或可能性

\item {} 
原料价格占产品售价的比例

\end{itemize}


\subparagraph{来自潜在进入者的威胁(Threat of new entrants)}
\label{\detokenize{chapter_knowledge/BRD:threat-of-new-entrants}}\begin{itemize}
\item {} 
消费者对替代品的偏好倾向

\item {} 
替代品相对的价格效用比

\item {} 
消费者的转换成本

\item {} 
消费者认知的品牌差异

\end{itemize}


\subparagraph{来自现有竞争者的威胁(Competitive rivalry)}
\label{\detokenize{chapter_knowledge/BRD:competitive-rivalry}}\begin{itemize}
\item {} 
现有竞争者的数目

\item {} 
产业成长率(industry growth)

\item {} 
产业存在超额产能的情况

\item {} 
退出障碍 (exit barrier)

\item {} 
竞争者的多样性 (diversity of rivals)

\item {} 
资讯的复杂度和不对称

\item {} 
品牌权益 (brand equity)

\item {} 
每单位附加价值摊提到的固定资产

\item {} 
大量的广告需求

\item {} 
不同的产品 (product differences)

\end{itemize}


\subparagraph{\$APPEALS方法}
\label{\detokenize{chapter_knowledge/BRD:appeals}}
是IBM在IPD总结和分析出来的客户需求分析的一种方法。它从8个方面对产品进行客户需求定义和产品定位。具体如下:
\sphinxhref{https://www.jianshu.com/p/1d733b336f76s}{10}%
\begin{footnote}[496]\sphinxAtStartFootnote
\sphinxnolinkurl{https://www.jianshu.com/p/1d733b336f76s}
%
\end{footnote}
\begin{enumerate}
\sphinxsetlistlabels{\arabic}{enumi}{enumii}{}{.}%
\item {} 
\$\sphinxhyphen{}产品价格(Price);

\item {} 
A\sphinxhyphen{}可获得性(Availability);

\item {} 
P\sphinxhyphen{}包装(Packaging);

\item {} 
P\sphinxhyphen{}性能(Performance);

\item {} 
E\sphinxhyphen{}易用性(Easy to use);

\item {} 
A\sphinxhyphen{}保证程度(Assurances);

\item {} 
L\sphinxhyphen{}生命周期成本(Life cycle of cost);

\item {} 
S\sphinxhyphen{}社会接受程度(Social acceptance)。

\end{enumerate}


\paragraph{战略定位分析 SPAN}
\label{\detokenize{chapter_knowledge/BRD:span}}
SPAN方法(Strategy positioning
Analysis\sphinxhref{http://reader.epubee.com/books/mobile/a0/a0bcbc34f65fcce9147c5238fb6d210b/text00021.html}{9}%
\begin{footnote}[497]\sphinxAtStartFootnote
\sphinxnolinkurl{http://reader.epubee.com/books/mobile/a0/a0bcbc34f65fcce9147c5238fb6d210b/text00021.html}
%
\end{footnote})从分析细分市场的吸引力和公司的竞争力出发对各个细分市场进行深入分析,为公司最终选定细分市场并在此基础上进行产品规划提供决策依据细分市场(Segmenting
Marketing)是市场管理和产品规划流程的重要步骤。在这个步骤,首先要根据一定标准对根据公司总体战略要进入的市场进行细分,并做初步的定性选择。主要从以下5个方面进行考虑:独特性、重要性、可衡量性、持久性和可识别性。
\begin{enumerate}
\sphinxsetlistlabels{\arabic}{enumi}{enumii}{}{.}%
\item {} 
独特性:该细分市场是否要求成本优势、高的资本投入、满足客户独特的需要、或者提供的产品要有足够的差异化。并且为了满足这些独特性是否需要一定的进入门槛;

\item {} 
重要性:这个细分市场要能达到一定的规模,这个规模能产生足够的利润来进行产品差异化、从事大型市场活动或提供售后服务;

\item {} 
可衡量性:能够衡量这个细分市场的市场销量与增长率;

\item {} 
持久性:最基本的要求是细分市场的存在至少要能够持续到公司产生利润;

\item {} 
可识别性:能够通过在这个细分市场中目标明确的销售与宣传,高效覆盖各个独特的客户群体。\sphinxhref{https://www.jianshu.com/p/1d733b336f76s}{10}%
\begin{footnote}[498]\sphinxAtStartFootnote
\sphinxnolinkurl{https://www.jianshu.com/p/1d733b336f76s}
%
\end{footnote}

\end{enumerate}


\paragraph{价值链方法}
\label{\detokenize{chapter_knowledge/BRD:id17}}\begin{enumerate}
\sphinxsetlistlabels{\arabic}{enumi}{enumii}{}{.}%
\item {} 
内部价值链分析.

\item {} 
纵向价值链分析

\item {} 
横向价值链分析

\end{enumerate}


\subsubsection{行业分析}
\label{\detokenize{chapter_knowledge/industry_analysis:industry-analysis}}\label{\detokenize{chapter_knowledge/industry_analysis:id1}}\label{\detokenize{chapter_knowledge/industry_analysis::doc}}
行业(industry)
:指一组提供同类相互密切替代商品或服务的公司。\sphinxhref{https://baike.baidu.com/item/\%E8\%A1\%8C\%E4\%B8\%9A}{16}%
\begin{footnote}[499]\sphinxAtStartFootnote
\sphinxnolinkurl{https://baike.baidu.com/item/\%E8\%A1\%8C\%E4\%B8\%9A}
%
\end{footnote}


\paragraph{目的}
\label{\detokenize{chapter_knowledge/industry_analysis:id2}}
产品经理需要随时了解市场和行业的变化,\sphinxstylestrong{提前}做好对行业变化的预判工作,找到机会点。这样,才能保证我们做出的产品能够更好的顺应市场。
\begin{enumerate}
\sphinxsetlistlabels{\arabic}{enumi}{enumii}{}{.}%
\item {} 
在研发面前就可以用有前瞻性的行业发展视角讲解需求,取得技术人员的认同和全力配合。

\item {} 
在老板面前要用行业内技术的发展趋势和如何将技术商业化角度阐释产品的设计方向和目标,取得资源上的支持。

\item {} 
在客户面前可以从行业标准和历史角度阐释今天的产品定位,并将产品置于竞争环境中进行客观的横向比对突出产品优势和价值。\sphinxhref{https://zhuanlan.zhihu.com/p/36869482}{14}%
\begin{footnote}[500]\sphinxAtStartFootnote
\sphinxnolinkurl{https://zhuanlan.zhihu.com/p/36869482}
%
\end{footnote}

\end{enumerate}

使自己具备无需通过隶属关系或公司内部权利就能够成功的影响的技术团队和其他跨职能团队,完成任务的能力。


\paragraph{如何了解一个行业? 15\sphinxfootnotemark[501]}
\label{\detokenize{chapter_knowledge/industry_analysis:id3}}%
\begin{footnotetext}[501]\sphinxAtStartFootnote
\sphinxnolinkurl{http://shujuren.club/a/AI0102.html}
%
\end{footnotetext}\ignorespaces 
第一步:行业认知
\begin{enumerate}
\sphinxsetlistlabels{\arabic}{enumi}{enumii}{}{.}%
\item {} 
行业分解认知

\item {} 
行业组合认知

\end{enumerate}

第二步:行业分析
\begin{enumerate}
\sphinxsetlistlabels{\arabic}{enumi}{enumii}{}{.}%
\item {} 
业务流程分析

\item {} 
产业链分析

\item {} 
商业模式分析

\end{enumerate}

第三步:行业常识
\begin{enumerate}
\sphinxsetlistlabels{\arabic}{enumi}{enumii}{}{.}%
\item {} 
业内知名企业

\item {} 
行业领导者

\end{enumerate}


\paragraph{分析出机会点}
\label{\detokenize{chapter_knowledge/industry_analysis:id4}}
产品小白如何快速做行业分析,找到机会点,最后能够做出分析结论:我们在哪些产品链条上的市场规模如何?存在哪些机会点?我们会面临上下游及竞品等哪些挑战?从中可以得到哪些商业价值?(需要日常练习,积累经验)

最后,针对我们的目标用户的典型需求,进行MVP。


\paragraph{套路}
\label{\detokenize{chapter_knowledge/industry_analysis:id5}}
\begin{figure}[H]
\centering
\capstart

\noindent\sphinxincludegraphics{{industry_module}.png}
\caption{industry\_module}\label{\detokenize{chapter_knowledge/industry_analysis:id27}}\end{figure}


\paragraph{行业分析的思路如下:}
\label{\detokenize{chapter_knowledge/industry_analysis:id6}}\begin{enumerate}
\sphinxsetlistlabels{\arabic}{enumi}{enumii}{}{.}%
\item {} 
了解市场规模(这个市场有多大?市场变化趋势如何?目前处于生命周期的哪个阶段?这个市场的中头尾部企业有哪些?他们的商业模式?细分市场趋势如何?这个产品会影响什么?技术壁垒、竞争壁垒是什么?指标:市场销售总量,年复合增长率,标志性现象看前瞻性)见
{\hyperref[\detokenize{chapter_knowledge/industry_analysis:market-size}]{\sphinxcrossref{\DUrole{std,std-ref}{了解市场规模}}}} (\autopageref*{\detokenize{chapter_knowledge/industry_analysis:market-size}})

\item {} 
产业地图分析(这个市场现状如何?行业如何运转?行业如何拆分?上下游关系如何?所在的竞争环境如何?细分市场的产品如何?指标:细分市场比例,增长率)见
\DUrole{xref,std,std-ref}{industry\_map}

\item {} 
行业典型产品分析(针对我们要做的产品,来了解竞品怎么做?竞品核心业务逻辑如何?功能架构如何?运营动作如何?用户评价如何?优缺点如何?):ref:\sphinxcode{\sphinxupquote{goods\_analysis}}

\item {} 
用户分析(针对我们要做的产品,来分析用户画像,从中挖掘需求,用户行为如何?机会点是什么?):ref:\sphinxcode{\sphinxupquote{users\_analysis}}

\item {} 
总结(分析结论:我们在哪些产品链条上的市场规模如何?存在哪些机会点?我们会面临上下游及竞品等哪些挑战?从中可以得到哪些商业价值?)

\end{enumerate}


\paragraph{了解市场规模}
\label{\detokenize{chapter_knowledge/industry_analysis:market-size}}\label{\detokenize{chapter_knowledge/industry_analysis:id7}}

\subparagraph{估算市场规模(增长空间)}
\label{\detokenize{chapter_knowledge/industry_analysis:id8}}
主要研究目标产品或行业的整体规模,可能包括目标产品一定时间内的行业产量,产值等
\sphinxhref{https://t.qidianla.com/1156537.html}{17}%
\begin{footnote}[502]\sphinxAtStartFootnote
\sphinxnolinkurl{https://t.qidianla.com/1156537.html}
%
\end{footnote}

\sphinxstylestrong{案例讲解}:全中国共享巴士的市场规模

先收集到以下的基本数据:
\begin{enumerate}
\sphinxsetlistlabels{\arabic}{enumi}{enumii}{}{.}%
\item {} 
中国人口基数:M人

\item {} 
每天出行用户规模:M*60\%人

\item {} 
其中的私家车用户:N人

\item {} 
公共交通每日运力:X人

\item {} 
人均城市内交通消费:R元

\end{enumerate}

然后进行计算得,总规模预计:(0.6M\sphinxhyphen{}2N\sphinxhyphen{}X\sphinxstyleemphasis{R)} 365


\subparagraph{趋势 10\sphinxfootnotemark[503]}
\label{\detokenize{chapter_knowledge/industry_analysis:id9}}%
\begin{footnotetext}[503]\sphinxAtStartFootnote
\sphinxnolinkurl{https://www.zhihu.com/pub/reader/119980992/chapter/1284104614602792960}
%
\end{footnotetext}\ignorespaces 
趋势代表着风向。顺风or逆风?

例如:抖音是 2016 年才成立的,为什么能够快速杀出重围,占据重要地位呢?

除了自身推荐算法的优势,还有趋势这个核心关键词,在它的背后又隐藏着两个核心的数据指标:
\begin{enumerate}
\sphinxsetlistlabels{\arabic}{enumi}{enumii}{}{.}%
\item {} 
智能手机出货量,2011 年中国智能手机的出货量为 1.18 亿部,而 2016
年中国智能手机的出货量为 4.65 亿部;

\item {} 
运营商流量,2011 年流量主要还是以 3G 为主且流量费约为 50 元 500MB,而
2016
年以后三大运营商逐步推出了不限流量套餐包,基本上套餐包的费用为每月
100 元左右。

\end{enumerate}

因此,抖音推出的时间无论是在硬件覆盖,还是软件的各种指标和要求方面,都实现了近乎完美的匹配。因此,抖音在强大的自身推荐算法基础上,结合趋势这一关键指标,取得阶段性的胜利也是正常的。


\subparagraph{宏观趋势}
\label{\detokenize{chapter_knowledge/industry_analysis:id10}}\begin{itemize}
\item {} 
我国:中国政府网、国家统计局、各部委及地方网站

\item {} 
世界:「互联网女皇」玛丽·米克尔每年都会发布的《互联网趋势报告》

\end{itemize}


\subparagraph{技术成熟度 Hype Cycle}
\label{\detokenize{chapter_knowledge/industry_analysis:hype-cycle}}
我们倾向于高估技术短期内的影响,并低估长期效应。——罗伊·阿马拉

Hype Cycle, 直译为炒作周期, 又称为技术成熟度曲线。
名为炒作,实是为了表示技术的受关注程度。这个模型由著名咨询公司Gartner发布,包含了Gartner对技术发展周期的预测。
Hype Cycle提供给我们Gartner公司对各种技术所处的发展阶段和趋势的预测。

Hype Cycle曲线的横轴表示技术的成熟度, 纵轴表示技术受关注程度。


\subparagraph{关键阶段 6\sphinxfootnotemark[504]}
\label{\detokenize{chapter_knowledge/industry_analysis:id11}}%
\begin{footnotetext}[504]\sphinxAtStartFootnote
\sphinxnolinkurl{https://www.shangyexinzhi.com/article/1924707.html}
%
\end{footnotetext}\ignorespaces 
\begin{figure}[H]
\centering
\capstart

\noindent\sphinxincludegraphics{{Dunning_Effect}.jpg}
\caption{Dunning\sphinxhyphen{}Kruger Effect}\label{\detokenize{chapter_knowledge/industry_analysis:id28}}\end{figure}

\begin{center}\sphinxincludegraphics{{hype_cycle}.jpg}\end{center} \sphinxincludegraphics{{hype_cycle_detail}.jpg}

每个技术成熟度曲线都将技术的生命周期划分为五个关键阶段:
\begin{itemize}
\item {} 
技术萌芽期:潜在的技术突破即将开始。早期的概念验证报道和媒体关注引发广泛宣传。通常不存在可用的产品,商业可行性未得到证明。

\item {} 
期望膨胀期:早期宣传产生了许多成功案例 —
通常也伴随着多次失败。某些公司会采取行动,但大多数不会。

\item {} 
泡沫破裂谷底期:随着实验和实施失败,人们的兴趣逐渐减弱。技术创造者被抛弃或失败。只有幸存的提供商改进产品,使早期采用者满意,投资才会继续。

\item {} 
稳步爬升复苏期:有关该技术如何使企业受益的更多实例开始具体化,并获得更广泛的认识。技术提供商推出第二代和第三代产品。更多企业投资试验;保守的公司依然很谨慎。

\item {} 
生产成熟期:主流采用开始激增。评估提供商生存能力的标准更加明确。该技术的广泛市场适用性和相关性明显得到回报。

\end{itemize}


\subparagraph{工具}
\label{\detokenize{chapter_knowledge/industry_analysis:id12}}\begin{itemize}
\item {} 
百度的一系列工具,如百度舆情、百度司南、百度指数等工具,这些工具可以帮助你全方位分析互联网舆论

\item {} 
Think with Google,该产品提供了丰富的营销工具和行业趋势分析报告。
\sphinxhref{https://www.zhihu.com/pub/reader/119919151/chapter/1283860049233436672}{13}%
\begin{footnote}[505]\sphinxAtStartFootnote
\sphinxnolinkurl{https://www.zhihu.com/pub/reader/119919151/chapter/1283860049233436672}
%
\end{footnote}

\end{itemize}


\paragraph{产业地图分析}
\label{\detokenize{chapter_knowledge/industry_analysis:id13}}

\subparagraph{市场调查 3\sphinxfootnotemark[506]}
\label{\detokenize{chapter_knowledge/industry_analysis:id14}}%
\begin{footnotetext}[506]\sphinxAtStartFootnote
\sphinxnolinkurl{https://baike.baidu.com/item/\%E5\%B8\%82\%E5\%9C\%BA\%E8\%B0\%83\%E6\%9F\%A5/170622\#:~:text=\%E5\%B8\%82\%E5\%9C\%BA\%E8\%B0\%83\%E6\%9F\%A5\%E6\%98\%AF\%E6\%8C\%87\%E7\%94\%A8,\%E6\%8F\%90\%E4\%BE\%9B\%E5\%AE\%A2\%E8\%A7\%82\%E3\%80\%81\%E6\%AD\%A3\%E7\%A1\%AE\%E7\%9A\%84\%E4\%BE\%9D\%E6\%8D\%AE\%E3\%80\%82}
%
\end{footnotetext}\ignorespaces 
市场调查是指用科学的方法,有目的、系统地搜集、记录、整理和分析市场情况,了解市场的现状及其发展趋势,为企业的决策者制定政策、进行市场预测、做出经营决策、制定计划提供客观、正确的依据。

专业知名机构:
\sphinxhref{https://www.shangyexinzhi.com/article/1924707.html}{6}%
\begin{footnote}[507]\sphinxAtStartFootnote
\sphinxnolinkurl{https://www.shangyexinzhi.com/article/1924707.html}
%
\end{footnote}
\begin{itemize}
\item {} 
China Ceidea Market Research 策点市场调研公司

\item {} 
Acorn Marketing \& Research Consultants 毅群市场研究咨询股份有限公司

\item {} 
上海伊霍珀信息科技股份有限公司

\item {} 
北京新数易博(EBMRS)信息咨询有限公司

\item {} 
华通明略(MillwardBrown ACSR)信息咨询有限公司
中机系(北京)信息技术研究院

\item {} 
中国商业数据中心

\item {} 
尼尔森市场研究中心

\item {} 
数字100市场研究公司

\item {} 
益普索(中国)市场研究咨询有限公司

\item {} 
凯度(中国)购物者指数(Kantar Worldpanel Chin\sphinxhyphen{} a)

\item {} 
上海AC尼尔森市场研究公司

\item {} 
盖洛普(中国)咨询有限公司

\item {} 
华南国际市场研究公司

\item {} 
百维数元信息科技(北京)有限公司

\item {} 
艾斯艾(北京)市场调查有限公司(SSI China)

\item {} 
欧睿(Euromonitor)市场调查机构

\end{itemize}


\subparagraph{关注以下几个重要指标}
\label{\detokenize{chapter_knowledge/industry_analysis:id15}}\begin{itemize}
\item {} 
TAM:即Total Available
Market,总有效市场或者市场规模,这是行业空间的天花板。然而,这是一个庞大的、基本没用的数字;

\item {} 
SAM:即Serviceable Available
Market,可服务市场,在基于公司内外部资源的客观条件下,所能服务到的市场范围。这个数字小了很多,基本有点用了;

\item {} 
SOM:即Serviceable Obtainable
Market,可获得市场,在能服务到的市场范围内,有能力拿下来的市场范围。这个数字进一步缩小,可以作为业务目标了;

\item {} 
Market Share:市场占有率,关注该产品在TAM中所占有份额;

\item {} 
Market Growth:市场成长性,关注整个行业TAM的增长或下降趋势;

\item {} 
Market Net
Value:公司实际收入,基于SOM所推断出来的公司实际收入(非流水,流水有可能只是过账户一道手,不一定是收入)。

\end{itemize}


\subparagraph{三四规则分析竞争地位 5\sphinxfootnotemark[508]}
\label{\detokenize{chapter_knowledge/industry_analysis:id16}}%
\begin{footnotetext}[508]\sphinxAtStartFootnote
\sphinxnolinkurl{https://weread.qq.com/web/reader/40632860719ad5bb4060856k283328802332838023a7529}
%
\end{footnotetext}\ignorespaces 
三四规则可用于分析企业在一个成熟市场中的竞争地位,它将参与市场竞争的企业分为三类,分别是领先者、参与者和生存者。三四规则描述了这样一个市场规律:在有影响力的领先者之中,企业的数量绝对不会超过三个,而在这三个企业之中,最有实力的竞争者的市场份额又不会超过最小者的四倍。

一般来说,领先者是指市场占有率在15\%以上、可以对市场变化产生重大影响的企业,体现在价格、产量等方面;参与者一般是指市场占有率为5\%~15\%的企业,这些企业虽然不能对市场产生重大的影响,但是它们是市场竞争的有效参与者;生存者一般是局部细分市场的填补者,这些企业的市场占有率都非常低,通常小于5\%。

三四规则的成立也有两个假定条件。
\begin{enumerate}
\sphinxsetlistlabels{\arabic}{enumi}{enumii}{}{.}%
\item {} 
在任何两个竞争者之间保持2∶1的市场份额均衡点时,无论哪个竞争者要增加或减少市场份额,都显得不切实际而且得不偿失。

\item {} 
当市场份额小于最大竞争者的1/4时,就不可能有效参与竞争。

\end{enumerate}

我们通过“三四规则”可以了解一些市场规律,倘若两个竞争者拥有几乎相同的市场份额,在竞争时谁能提高相对市场份额,谁就能同时取得在产量和成本两个方面的增长,而在任何主要竞争者的激烈争夺情况下,最有可能受到伤害的却是市场中最弱的生存者。


\subparagraph{行业标准 7\sphinxfootnotemark[509]}
\label{\detokenize{chapter_knowledge/industry_analysis:id17}}%
\begin{footnotetext}[509]\sphinxAtStartFootnote
\sphinxnolinkurl{http://reader.epubee.com/books/mobile/f4/f4c52db61d39acb835e2709cbed1585e/text00005.html}
%
\end{footnotetext}\ignorespaces 
了解行业中规范的设计和开发标准不仅能缩短产品设计中的基础探索周期,而且能避免未来由于产品不符合行业标准或规则而带来的风险。如果公司能参与行业标准的制定最好,即使不能,熟悉行业标准和规范对于产品定义、设计都非常关键。

例如《机器人性能规范》《机器人安全要求》可以指导产品非功能需求设计,《机器人系统与集成标准》可以保证产品的集成设计方案是合理且通用的,《服务机器人模块化设计总则》可以指导机器人的模块化设计等。


\subparagraph{行业数据分析}
\label{\detokenize{chapter_knowledge/industry_analysis:id18}}
和企业内部数据(特别是财务数据)相比,完全不是一个量级的准。

\begin{figure}[H]
\centering
\capstart

\noindent\sphinxincludegraphics{{industry_data}.png}
\caption{行业分析数据}\label{\detokenize{chapter_knowledge/industry_analysis:id29}}\end{figure}


\subparagraph{渠道获取}
\label{\detokenize{chapter_knowledge/industry_analysis:id19}}\begin{enumerate}
\sphinxsetlistlabels{\arabic}{enumi}{enumii}{}{.}%
\item {} 
内部市场、运营部门、管理层等信息收集

\item {} 
艾瑞咨询、DCCI互联网数据中心、Alexa、Appstore、企鹅智库、猎豹智库、易观、比达咨询、IT
桔子、199IT
互联网数据中心\sphinxhref{https://www.zhihu.com/pub/reader/119919151/chapter/1283860049233436672}{13}%
\begin{footnote}[510]\sphinxAtStartFootnote
\sphinxnolinkurl{https://www.zhihu.com/pub/reader/119919151/chapter/1283860049233436672}
%
\end{footnote}、\sphinxurl{https://www.yanbaoke.com/index}
、\sphinxurl{https://www.qianzhan.com/}、 \sphinxurl{https://www.cyzone.cn}/等平台
\sphinxhref{http://www.woshipm.com/pmd/1792207.html}{12}%
\begin{footnote}[511]\sphinxAtStartFootnote
\sphinxnolinkurl{http://www.woshipm.com/pmd/1792207.html}
%
\end{footnote}

\item {} 
竞争对手网站、交流互动平台、产品历史更新版本、促销活动、最新调整、招聘信息等

\item {} 
竞争对手的季度/年度财报

\item {} 
行业媒体平台新闻、论坛、QQ群等

\item {} 
调查核心用户、活跃用户、普通用户不同需求弥补和代替的产品

\item {} 
使用对方的产品、客服咨询、技术问答等等

\item {} 
搜索国外同行业的官网及行业信息订阅(市场竞争可能不大,但盈利模式和功能定义用户群体具有一定前瞻性和市场趋势导向性)

\end{enumerate}


\paragraph{成为一个行业专家}
\label{\detokenize{chapter_knowledge/industry_analysis:id20}}
如何快速深入一个行业,笔者基于自身经验,罗列了如下6个维度:行业特点、行业运行趋势、商业模式、竞争力因素分析、行业整合、政府管制。以个人/家庭服务机器人为例。

\begin{figure}[H]
\centering
\capstart

\noindent\sphinxincludegraphics{{PM_industry}.jpg}
\caption{如何快速深入一个行业}\label{\detokenize{chapter_knowledge/industry_analysis:id30}}\end{figure}

\begin{figure}[H]
\centering
\capstart

\noindent\sphinxincludegraphics{{dive_industry}.jpg}
\caption{深入了解行业:点线面}\label{\detokenize{chapter_knowledge/industry_analysis:id31}}\end{figure}

现代管理学之父彼得·德鲁克(Peter
F.Drucker)曾经说过“企业的目的是创造和留住顾客。”在人工智能行业初期,一定是由技术驱动形成单个的场景应用和创新,随着市场同质化竞争日趋严重,企业一旦在某个“点”建立起竞争优势后,就需要快速转向“线”,即为客户创造更丰富的产品和服务,让客户不断看到新的价值和惊喜,最终积累更多的忠诚客户。

拥有对“点”的把控力,只是产品经理修炼成为行业专家的第一步。人工智能产品经理还要通过深挖场景价值,完善产品链条,即形成从“点”到“线”的变化。如果说互联网时代的主流价值观是“流量为王”,那么人工智能时代的产品就是“获得更多的超级用户”,这些超级用户创造了绝大部分的企业利润。企业靠“点”的创新只会保证其在第一阶段获取更多的初始客户,如果想要保住这些用户,而且要让他们变为忠诚的“超级用户”,就需要定制化、一站式的完整解决方案。

人工智能产品经理可以从下面几个方面进行从“点”到“线”的积累。

(1)深挖用户在场景中的需求,为用户提供解决方案而不仅仅是产品。举个例子:作为电商平台,给用户提供完美的网购体验并没有完,有些平台还会自建物流体系,延长服务链条,这么做在赚取额外利润的同时,还积累了大量会员。

当发现用户口袋里的钱不够时,没关系,用户还可以赊账,这样又衍生出了金融服务。按照这种逻辑规划出来的产品本质上就是解决方案,因为用户永远都会不停地挑剔、比较,只有产品的链条足够长,才能保持品牌持久的竞争力。而找到这样的“线”,就是产品经理尤其是人工智能产品经理重要的使命。

(2)挖掘用户数据中的价值,为用户创造惊喜。例如,如果你是做线上房屋租赁平台产品的,可以通过分析每个用户线上的行为和习惯数据建立个人的消费和信用模型,筛选出一些优质用户,以免租金和免租房押金的方式,定期提供福利和惊喜,甚至可以通过用户所在行业、兴趣爱好帮助用户匹配最适合的房东,当用户想换个地方住的时候还可以提供个性化建议。这就是一种典型的从“点”到“线”的思路。人工智能产品经理要通过人工智能技术挖掘那些从量变到质变的潜在机会,最终实现产品服务链条化,积累更多的忠诚用户。

当人工智能产品完成从“点”到“线”的变化后,需要进一步巩固自身优势,让产品变成“面”。“面”包括两方面的含义,一是通过引入外部资源建立紧密的协同关系并构建更宽广的产品覆盖度,与用户产生更多的联系;二是指通过整合公司内部资源打通各产品线的数据和基础服务,形成公司内部的产品生态。

人工智能产品经理可以从两个方面进行从“线”到“面”的整合。

(1)整合外部资源,实现多元化协作:由于人工智能产品的架构复杂,数据、算法、计算能力想要实现快速积累并整合,在某种程度上可以通过对外协作和资源整合的方式实现。因此需要人工智能产品经理做好整合资源的准备并提出解决方案。例如高质量有效数据的共享及交易;和传统行业解决方案公司或业内具有影响力的客户从数据、行业资源等方面进行深入合作、优势互补;如果是做软件的公司,就和一些硬件供应商进行软硬技术的融合,通过整合上下游资源形成利益结盟。

人工智能行业的产业链协作还处于初期,未来越来越多的公司在研发自己的人工智能产品时会主动选择协作,人工智能产品经理在市场竞争中应保持和外部资源的密切关系,这不仅对公司来说是一种积累和扩大优势的方式,而且对于产品经理个人来说也是一种扩大个人在行业中影响力的途径。

(2)布局内部产品生态化:当公司的产品线变得丰富后,产品经理应通过构建人工智能统一平台,实现各条产品线的优势联合与价值共享。比如公司有三条产品线,每条产品线有大量的交叉用户,而且都包含搜索引擎、推荐引擎、智能售后机器人等通用功能,这个时候就可以考虑整合三条产品线的用户数据和算法(例如智能交互、语义搜索、智能匹配等),统一研发公司级别的搜索平台、个性化推荐引擎和知识图谱等。这样的公司级别的平台反过来为三条产品线的用户提供全方位的个性化决策服务。随着各平台对基础服务的优化,会增强各条产品线的竞争力,进而产生更多有价值的数据,最终形成良性循环。另外,当公司有新的产品线成立时,可以在公司现有平台基础上快速建立自身优势,快速融入公司的产品生态。

主要包括阅读行业新闻、行业分析报告,关注行业意见领袖的公众号等,每天至少需要30分钟,要保证质量。


\paragraph{行研框架}
\label{\detokenize{chapter_knowledge/industry_analysis:id21}}
腾讯5G生态计划负责人 余一列出了一个做行研的基本框架:
\begin{enumerate}
\sphinxsetlistlabels{\arabic}{enumi}{enumii}{}{.}%
\item {} 
确定研究目标;

\item {} 
圈定已有资料的概览范围,上市公司财报及分析报告、咨询公司报告、数据机构资料、行业专业网站、政府网站、招聘网站、媒体网站等;

\item {} 
需要圈定时间和目标,不要迷失在资料中。

\item {} 
输出初步框架,行业现状(规模、结构、阶段)、行业趋势(发展推动要素、推动力分析)、竞争格局、其他。

\item {} 
业内访谈 ,产业链、公司、专家、技术。

\item {} 
输出。

\end{enumerate}

Envolve Group Co\sphinxhyphen{}founder
刘嘉培Alex详细拆解了查阅报告材料和思考的六个步骤,即要Top\sphinxhyphen{}down地思考一个行业:第一,先看整体市场规模,再看CAGR年复合增速,并思考:a)
容得下几家巨头公司? b) 增长的驱动因素是什么?
第二,了解最新资本市场活动:投资总额、IPO数量、兼并收购数量,思考:a)
行业受资本青睐吗?为什么?b)
大家是想靠估值倍数、分红、增长、并购重组挣钱?第三,利用MECE的方式把市场分割成多个不同的赛道:a)
关注不同赛道的行业规模、增速、市场活动、趋势、龙头、商业模式b)
考虑行业上下游之间的关系:整合还是分散?竞争还是合作?会一家独大还是百花齐放?c)
不同赛道里面最容易出现商业模式成熟、盈利模式清晰的公司的是哪个?
第四,关注最新行业动态、趋势和“催化剂”:a)
趋势是利好还是利空?对巨头有利还是对挑战者有利?b) 看行业垂直媒体c)
看公司研报。第五,研究行业巨头3\sphinxhyphen{}5家,新兴挑战者企业8\sphinxhyphen{}15家,做总结:a)
总结领先产品、品牌策略、用户positioning,b)
总结商业模式、盈利模式、经营模型、竞争策略,c)
二级市场估值倍数和市值变化规律,d) Where they started and how they got
here。第六,从创始人、投资人、客户/用户、投行咨询四个角度问:a)如果现在进入市场的话,会怎么做?b)如果投股票、收购公司、天使投资、债券的话,分别怎么投?为什么?在赌什么?c)
作为一个用户,最希望看到的是什么?为什么?

在实操过程中,Red Tripod captial Investment Director Vivian
Young特别强调了供需分析的重要性:一般的行业分析员大部分的时间就是在做供需分析,分析时要注意当前的供求结构关系,区分国内还是全球,存量还是增量。另外,需要重点关注:需求周期,产能周期,需求传导的逻辑,传导的节奏等。除了供需,还要研究行业未来发展趋势,其中,政策影响很关键。

除了具体的方法与步骤,阿尔法公社投资经理Gang
Liu还提醒大家,在做行研的时候不能求快,要以慢为快,在有限的时间段里,花更多的时间在研究上,方式方法重要,但执行同样重要。同时,要敬畏专业性,尽可能的找到这个领域的一线从业者或者专家,多跟他们交流。交叉验证,保持思辨性很重要。


\paragraph{如何才能加强行业理解?11\sphinxfootnotemark[512]}
\label{\detokenize{chapter_knowledge/industry_analysis:id22}}%
\begin{footnotetext}[512]\sphinxAtStartFootnote
\sphinxnolinkurl{https://www.zhihu.com/pub/reader/119980992/chapter/1284104632479215616}
%
\end{footnotetext}\ignorespaces 
只有一个办法就是\sphinxstylestrong{进入这个行业}。

以我个人为例,我在做某互联网金融平台之前没有专门接触过金融。因此,我所做的事情本质上就是按照他人的想法原原本本地实现,没有任何自己的想法,唯一有的就是互联网人一直所提倡的用户体验。实际上,在产业层面,核心的体验远远不是界面上所展示的,更多的还是底层的业务本身。

于是,我兼任了受理部管理的工作。这个岗位可以理解为前端销售和后台系统的中间层,工作主要是与前端进行业务对接并进行后续的实际操作,同时完成系统侧的录入、审核等。

这样的经历让我明白了业务本身到底是如何流转的,也让我了解了销售人员、合作渠道和受理的同事的利益诉求分别是什么。除此之外,我还知道了风控、财务、合作的资金方和中间的保险公司考虑的都是什么。

此后,我又兼任了分公司的负责人,能够更好地走到市场的前线,了解市场的状况、合作渠道现阶段的情形和现阶段市场上的缺失,考虑自己是否能够抓住其中一些机会。

总之,产品经理要想真的加强对某个行业的理解,一定要沉浸其中。


\paragraph{形成行业认知 9\sphinxfootnotemark[513]}
\label{\detokenize{chapter_knowledge/industry_analysis:id23}}%
\begin{footnotetext}[513]\sphinxAtStartFootnote
\sphinxnolinkurl{https://weread.qq.com/web/reader/8d632bc07208ed1c8d697c4ka5732aa0226a5771bce9dc4}
%
\end{footnotetext}\ignorespaces 
产品经理在一个行业待久了,大量的行业研究、竞品分析、用户访谈、数据分析、一轮又一轮的项目试错经验,最后都会沉淀为这个产品经理的行业认知,而这都是公司花大量真金白银买来的经验。比如,特定用户在打车的时候,究竟是对等候时间更敏感,还是对价格更敏感,抑或是对安全更敏感?如果这一敏感要素是等候时间,那么多久是用户的心理底线?如何通过产品策略来提升这个忍受值?这些问题的答案可能是产品经理花了大量时间、不断做实验摸索出来的认知。而如果能把这样的产品经理招到自己的团队,那么我们将瞬间获得这些宝贵的行业认知和经验。


\subparagraph{广度 14\sphinxfootnotemark[514]}
\label{\detokenize{chapter_knowledge/industry_analysis:id24}}%
\begin{footnotetext}[514]\sphinxAtStartFootnote
\sphinxnolinkurl{https://zhuanlan.zhihu.com/p/36869482}
%
\end{footnotetext}\ignorespaces \begin{enumerate}
\sphinxsetlistlabels{\arabic}{enumi}{enumii}{}{.}%
\item {} 
行业特点:增长能力、与宏观经济周期的关系、固有风险等。

\item {} 
商业模式:挣钱的手段,产业链逻辑是怎样的,价值链是如何构成的等。

\item {} 
行业运行趋势:国内外的行业发展趋势和方向,供应商谈判能力,购买者谈判能力,现有同行竞争的局面,龙头企业(不应该只限定一家)目前面临的主要问题,有哪些陈宫的管理和技术经验,这些经验是否可以借鉴或复制,新进入者威胁,替代产品和服务威胁等。

\item {} 
竞争力因素分析:价格、品质、质量、分销能力、上游资源、成本、产品差异、技术壁垒、管理水平、地理位置等。

\item {} 
行业整合:行业集中度、国家法规、外资进入、收购兼并等。

\item {} 
政府管制:准入门槛、价格、税收、进出口等。

\end{enumerate}


\subparagraph{深度}
\label{\detokenize{chapter_knowledge/industry_analysis:id25}}
对行业理解的深度通常会针对具体用户的痛点和使用场景进行分析,我相信这部分工作大部分产品经理都会做,因此这方面的具体做法不在我本文中做过多描述。我仅提出几个总结下加强学习和制造创新思维的方式:
\begin{enumerate}
\sphinxsetlistlabels{\arabic}{enumi}{enumii}{}{.}%
\item {} 
不要拘泥于现存的技术手段去完成某种客户的需求,可能解决同一个需求的方式有很多,只是你所在行业对于前沿技术的使用比较慢,可以从其他行业将技术历年移植过来进行解决。

\item {} 
不要被用户传统的解决问题的手段所局限,往往所谓外行的办法解决了传统方式无法解决的问题。

\item {} 
做用户的助手,不要做用户的老师。如果你没有对行业有特别深入的理解,请不要轻易创造某种需求。因为产品经理是要对整个产品生命负责,而不只是创造哪些看起来好看,一味追求技术的时髦,到头来发现其实用户真实需求没被解决,而你却创造出一堆你认为的需求,而你却乐在其中。

\end{enumerate}

获得以上所有信息的方式,你可以通过与行业专家面对面沟通,可以通过直接购买数据报告,还可以每天与你的用户泡在一起,多与其他行业的人交流。无论哪种办法,把你的精力多分配到对于行业的学习和信息获取方面,你就拥有越多的信息帮助你进行判断和产品设计。


\paragraph{更多}
\label{\detokenize{chapter_knowledge/industry_analysis:id26}}
在咨询公司工作的人、金融业做投研、行业研究的人,这些人在写报告的时候,通过什么渠道来获取资料和数据?:
\sphinxurl{https://www.zhihu.com/question/278003245/answer/397609947}

在哪里能找到各行业的分析研究报告? \sphinxhyphen{} 曹婷婷的回答 \sphinxhyphen{} 知乎
\sphinxurl{https://www.zhihu.com/question/19766160/answer/92693568}


\subsubsection{市场分析}
\label{\detokenize{chapter_knowledge/market_analysis:market-analysis}}\label{\detokenize{chapter_knowledge/market_analysis:id1}}\label{\detokenize{chapter_knowledge/market_analysis::doc}}\begin{itemize}
\item {} 
市场:商品服务交易的地方

\item {} 
三要素:购买者、购买力、购买欲市场分析·

\item {} 
定义:就是对市场容量、市场规模\sphinxhref{http://www.51pmexp.com/?p=872}{1}%
\begin{footnote}[515]\sphinxAtStartFootnote
\sphinxnolinkurl{http://www.51pmexp.com/?p=872}
%
\end{footnote}、市场特性、供需变化的各种因素及其动态、趋势等相关内容进行的实事求是的经济分析及预测。

\item {} 
要求:客观真实、系统严密、信息加工、决策导向。

\item {} 
是项目启动的重要前提之一

\item {} 
根本目的:了解是否有利可图\sphinxhref{https://t.qidianla.com/1166273.html}{2}%
\begin{footnote}[516]\sphinxAtStartFootnote
\sphinxnolinkurl{https://t.qidianla.com/1166273.html}
%
\end{footnote}

\item {} 
市场分析是行业分析的一部分,行业分析更具有全局性、战略性,市场分析一般情况下,是行业内的市场分析。

\end{itemize}


\paragraph{层次}
\label{\detokenize{chapter_knowledge/market_analysis:id2}}\begin{enumerate}
\sphinxsetlistlabels{\arabic}{enumi}{enumii}{}{.}%
\item {} 
宏观经济分析。指的是分析一般经济环境及影响未来供需平衡的因素,如产业范围、经济增长率、产业政策及发展方向、行业设施利用率、货币汇率及利率、税收政策与税率、政府体制结构与政治环境、关税政策与进出口限制、人工成本、通货膨胀、消费价格指数、订购状况等因素。

\item {} 
中观经济分析。它集中于研究特定的工业部门,并且在这个层次,很多信息都可以从国家的中央统计部门和工业机构中获得。它们有关于营利性、技术发展的劳动成本、间接成本、资本利用、订购状况、能源消耗等具体信息。这个层次主要包括以下信息:供求分析、行业效率、行业增长状态、行业生产与库存量、市场供应结构、供应商的数量与分布等。

\item {} 
微观经济分析。它集中于评估个别产业供应和产品的优势与劣势,如供应商财务审计、组织架构、质量体系与水平、产品开发能力、工艺水平、生产能力与产量、交货周期及准时率、服务质量、成本结构与价格水平,以及作为供应商认证程序一部分的质量审计等。它的目标是对于供应商的特定能力和其长期市场地位进行透彻地理解。

\end{enumerate}


\paragraph{主要任务}
\label{\detokenize{chapter_knowledge/market_analysis:id3}}
市场分析的主要任务是:分析预测全社会对项目产品的需求量;分析同类产品的市场供给量及竞争对手情况;初步确定生产规模;初步测算项目的经济效益。
市场分析是工业发展与工业布局研究的组成部分之一。


\subparagraph{分类}
\label{\detokenize{chapter_knowledge/market_analysis:id4}}\begin{enumerate}
\sphinxsetlistlabels{\arabic}{enumi}{enumii}{}{.}%
\item {} 
市场需求预测分析。包括现在市场需求量估计和预测未来市场容量及产品竞争能力。通常采用调查分析法、统计分析法和相关分析预测法;

\item {} 
市场需求层次和各类地区市场需求量分析。即根据各市场特点、人口分布、经济收入、消费习惯、行政区划、畅销牌号、生产性消费等,确定不同地区、不同消费者及用户的需要量以及运输和销售费用。一般可采用产销区划、市场区划、市场占有率及调查分析的方法进行;

\item {} 
估计产品生命周期及可销售时间。即预测市场需要的时间,使生产及分配等活动与市场需要量作最适当的配合。通过市场分析可确定产品的未来需求量、品种及持续时间;产品销路及竞争能力;产品规格品种变化及更新;产品需求量的地区分布等。

\end{enumerate}


\paragraph{作用}
\label{\detokenize{chapter_knowledge/market_analysis:id5}}
在工业发展与布局研究中,市场分析有助于确定地区工业部门或企业的发展水平和发展规模,及时调整产业结构;有助于调整产品结构,提高竞争能力;有助于在运输和生产成本最小的原则下,合理布置工业企业。


\subsubsection{竞品分析 1\sphinxfootnotemark[517]}
\label{\detokenize{chapter_knowledge/goods_analysis:id1}}\label{\detokenize{chapter_knowledge/goods_analysis::doc}}%
\begin{footnotetext}[517]\sphinxAtStartFootnote
\sphinxnolinkurl{http://www.woshipm.com/pmd/1842636.html}
%
\end{footnotetext}\ignorespaces 

\paragraph{竞品本质}
\label{\detokenize{chapter_knowledge/goods_analysis:id2}}
目标用户是同一群人或者组织。\sphinxhref{https://weread.qq.com/web/reader/8d632bc07208ed1c8d697c4k9bf32f301f9bf31c7ff0a60}{5}%
\begin{footnote}[518]\sphinxAtStartFootnote
\sphinxnolinkurl{https://weread.qq.com/web/reader/8d632bc07208ed1c8d697c4k9bf32f301f9bf31c7ff0a60}
%
\end{footnote}


\paragraph{目的}
\label{\detokenize{chapter_knowledge/goods_analysis:id3}}
竞品分析的目标包括:\sphinxstylestrong{学习借鉴、优化自身、跟随竞争、反制对手。}最终\sphinxstylestrong{为了从竞争者那里抢用户!}。\sphinxhref{https://blog.csdn.net/weixin\_45036344/article/details/103200505}{11}%
\begin{footnote}[519]\sphinxAtStartFootnote
\sphinxnolinkurl{https://blog.csdn.net/weixin\_45036344/article/details/103200505}
%
\end{footnote}
工作目标的不同,需要解决的问题不同。竞品分析的方式方法也会不同,最终\sphinxstylestrong{把问题解决了}才是有效的竞品分析!

竞品分析的目的是获取有价值的信息,基于信息的指导,让本产品在不同目标上做得更好;
\begin{itemize}
\item {} 
用户分级来提升粘性活跃度和内容质量。

\item {} 
年龄大看不清分析竞品的字号大小,但目标群体年轻人就不重要了。

\end{itemize}


\paragraph{文档结构及内容:17\sphinxfootnotemark[520]}
\label{\detokenize{chapter_knowledge/goods_analysis:id4}}%
\begin{footnotetext}[520]\sphinxAtStartFootnote
\sphinxnolinkurl{https://t.qidianla.com/1156575.html}
%
\end{footnotetext}\ignorespaces \begin{enumerate}
\sphinxsetlistlabels{\arabic}{enumi}{enumii}{}{.}%
\item {} 
行业分析及趋势分析

\item {} 
竞品背景对比分析(确定分析对象,目标等)

\item {} 
竞品定位对比分析(产品定位,用户对比)

\item {} 
竞品产品对比分析(功能对比,体验对比)

\item {} 
竞品策略对比分析(产品策略,运营策略)

\item {} 
竞品用户对比分析(用户群体,使用场景)

\item {} 
结论

\end{enumerate}


\subparagraph{背景分析18\sphinxfootnotemark[521]}
\label{\detokenize{chapter_knowledge/goods_analysis:id5}}%
\begin{footnotetext}[521]\sphinxAtStartFootnote
\sphinxnolinkurl{https://t.qidianla.com/1156575.html}
%
\end{footnotetext}\ignorespaces \begin{itemize}
\item {} 
公司资源/背景 :对手公司的投资人有哪些?有没有什么独特的资源?

\item {} 
团队背景:团队人员主要来自哪里?创始人团队什么背景?

\item {} 
人员构成:这家公司的组织架构图是怎样的?哪一块比较薄弱?

\item {} 
公司愿景:公司的主要目的是干嘛的?

\item {} 
成本构成:有多少钱?分布来自哪里?要用到哪里去?

\end{itemize}

背景资料收集:
\begin{itemize}
\item {} 
国外:CrunchBase

\item {} 
国内: IT桔子

\end{itemize}


\paragraph{分类 4\sphinxfootnotemark[522]}
\label{\detokenize{chapter_knowledge/goods_analysis:id6}}%
\begin{footnotetext}[522]\sphinxAtStartFootnote
\sphinxnolinkurl{https://www.zhihu.com/question/39005837/answer/167081923}
%
\end{footnotetext}\ignorespaces \begin{itemize}
\item {} 
功能分析:主要是列出自己产品的功能和同级别产品的功能,从中发现成本差异,研发能力差异。

\item {} 
交互分析:交互其实已经包含工业设计中,但是有必要单独拿出来分析,因为决定了用户体验。是用已经发展多年的按键,触屏,手机控制还是用目前最火的语音交互?为什么?

\item {} 
设计分析:在工业设计上,设计语言不仅是公司战略级的事务,也是产品极为重要的元素之一。设计领域还是有规可循的,把该类产品历史发展的设计风格和现在市场的风格研究一番,就能略知一二。还应该讨论,设计元素的变更有什么原因,目前的技术是否能够支持。

\item {} 
战略(产品定位与用户需求)

\item {} 
业务(主要功能)

\item {} 
结构(产品结构与信息结构)

\item {} 
体验(交互与视觉设计)

\item {} 
模式(商业模式)\sphinxhref{http://www.woshipm.com/pmd/1642415.html}{12}%
\begin{footnote}[523]\sphinxAtStartFootnote
\sphinxnolinkurl{http://www.woshipm.com/pmd/1642415.html}
%
\end{footnote}

\end{itemize}


\paragraph{原则}
\label{\detokenize{chapter_knowledge/goods_analysis:id7}}\begin{itemize}
\item {} 
海盗指标

\item {} 
场景分析

\item {} 
关键流程分析

\item {} 
尼尔森法则

\end{itemize}


\paragraph{步骤14\sphinxfootnotemark[524]}
\label{\detokenize{chapter_knowledge/goods_analysis:id8}}%
\begin{footnotetext}[524]\sphinxAtStartFootnote
\sphinxnolinkurl{https://t.qidianla.com/1149667.html}
%
\end{footnotetext}\ignorespaces \begin{enumerate}
\sphinxsetlistlabels{\arabic}{enumi}{enumii}{}{.}%
\item {} 
要选择好竞品:明确你要调研的产品的方向或功能优化的方向,找应用市场上排名高、垂直领域做的比较好、市场份额高\sphinxhref{https://t.qidianla.com/1175640.html}{16}%
\begin{footnote}[525]\sphinxAtStartFootnote
\sphinxnolinkurl{https://t.qidianla.com/1175640.html}
%
\end{footnote}的产品,2\sphinxhyphen{}3个(推荐七麦数据);

\item {} 
有了目标之后,我主要针对大而细两点进行分析。大的方面是找数据分析市场现状、市场阶段、产品阶段、用户人群、用户场景、产品定位等,这些都是为后面的产品需求设计而服务的;

\item {} 
有了前面的基础细的方面就是针对性地深入挖掘,首先了解竞品的核心流程和功能结构,然后看他在产品设计上是怎么体现的,主流程的操作需要多少步完成,导航栏的分类都是什么,首页都引导了哪些内容,版本变更情况,运营策略情况…这样才算基本了解一个竞品;

\item {} 
最后再得出结论如果我想实现某一个功能的话是否可行,怎样实现。

\end{enumerate}


\subparagraph{竞品定义}
\label{\detokenize{chapter_knowledge/goods_analysis:id9}}
蛋糕 = 需求量*ARPU,僧多粥少?

竞品一词来源于经济学领域,是指对竞争对手产品的优劣势进行比较分析。随着互联网时代的到来,现在的竞品用处更加宽泛。追根溯源竞品分析更像是工业时代企业管理里面的标杆管理,都是一种见贤思齐的人性体现,我更愿意接受竞品分析是战略规划工具这一说法。

竞品理念最早的成功案例要说富士施乐(FUJI
XEROX)公司,施乐公司通过不断的同当时的日本企业进行对标,最后在复印机市场成功逆袭日本佳能。施乐公司成功后,这个方法风靡美国很多大型公司。


\subparagraph{对什么?}
\label{\detokenize{chapter_knowledge/goods_analysis:id10}}
出发点不对的强制竞品分析任务只会适得其反,带着目的与初衷愿景去做竞品分析,做出来的结果才有意义。


\subparagraph{和谁对?}
\label{\detokenize{chapter_knowledge/goods_analysis:id11}}
这五力分别指,同行业内现有竞争者的竞争能力、潜在竞争者进入的能力、替代品的替代能力、供应商的讨价还价能力、购买者的讨价还价能力。我们可能对直接竞争对手关注颇多,但殊不知竞品无处不在。

重新定义用户视角下的竞品分类。

做竞品分析,是\sphinxstylestrong{为了从竞争者那里抢用户!}

用户认为我们是什么,把我们归到哪一类,我们就应该在这个范围内去找自己的竞品,这之后,才谈得上抢用户。

分析“用户”!
\begin{itemize}
\item {} 
知识付费类的APP:喜马拉雅APP,也可以是混沌大学APP

\item {} 
消遣无聊时光:喜马拉雅了,还可能是花椒、蜻蜓FM。

\item {} 
学习:选择线下大学、培训班、学习类图书、音像制品等等

\item {} 
没有明确的需求方向:所有可能抢占用户钱包和时间的产品都可以作为彼此的竞品。“得到”的竞品可能就是“王者荣耀”

\end{itemize}


\paragraph{信息收集与整理}
\label{\detokenize{chapter_knowledge/goods_analysis:id12}}

\subparagraph{网络}
\label{\detokenize{chapter_knowledge/goods_analysis:id13}}\begin{itemize}
\item {} 
公司的季报、年报。公司的官网(版本更新情况、人才招聘、最近新闻等),很多官网都会详细介绍拳头产品,甚至提供完整的产品手册(新手说明、常见问题、技术文档等),你可以和自己的产品比对,快速找出优缺点。、产品论坛(用户的反馈、客服的回答等)

\item {} 
行业媒体的新闻、论坛或分析文章等。

\item {} 
公信力的第三方咨询或报告公司,比如:应用宝、aso100、艾瑞咨询、Talkingdata、IDC、麦肯锡、易观智库、企鹅智库、猎豹智库、CNNIC、百度指数、微信指数、极光数据、\sphinxhref{https://qianfan.analysys.cn/}{易观千帆}%
\begin{footnote}[526]\sphinxAtStartFootnote
\sphinxnolinkurl{https://qianfan.analysys.cn/}
%
\end{footnote}、友盟数据、App
Annie等。

\item {} 
行业协会,一般是行业内的自律组织,很多行业协会都会定期在网站上发布业内活动、重大新闻、大型会晤、科技进步以及行业数据。虽然可能有水分,但是相对比较直观量化。

\end{itemize}


\subparagraph{卧底}
\label{\detokenize{chapter_knowledge/goods_analysis:id14}}\begin{itemize}
\item {} 
从对方公司、相关渠道、市场、运营等部门获得对方内部的信息等。

\item {} 
寻找竞品的使用用户(核心用户、普通用户等)进行骚扰。

\item {} 
对竞品公司客服或技术进行骚扰等,获取想了解的信息(特别是产品实现规则等方面)\sphinxhref{http://www.51pmexp.com/?p=62}{15}%
\begin{footnote}[527]\sphinxAtStartFootnote
\sphinxnolinkurl{http://www.51pmexp.com/?p=62}
%
\end{footnote}

\end{itemize}


\paragraph{竞品分类}
\label{\detokenize{chapter_knowledge/goods_analysis:id15}}
\begin{figure}[H]
\centering
\capstart

\noindent\sphinxincludegraphics{{goods_env}.png}
\caption{竞品生态的组成}\label{\detokenize{chapter_knowledge/goods_analysis:id32}}\end{figure}

对于问题和方案的异同,我们可以用象限概念来帮助理解,把与自己产品有关系的潜在竞争对手分为四大类:问题同方案同、问题同方案异、问题异方案同、问题异方案异。
\begin{itemize}
\item {} 
问题同方案同:直接竞品的厮杀通常是渐进式的创新,此消彼长

\item {} 
问题同方案异:用不同方案解决相似问题的产品,往往会成为行业里颠覆巨头的下一代产品。要特别关注!

\item {} 
问题异方案异:产品存在跨行业迁移,京东原来只卖 3C
数码,积累了用户和基础设施之后,卖起书来一点都不比当当差、星巴克原来只卖咖啡,2019
年也推出了茶饮料。

\item {} 
问题异方案异:占用了相似的不可再生资源,比如时间、金钱、人才等。产品产业链条中的上下游。任何行业里的某个角色,如果做大做强了,都很可能忍不住要占据产业链条里更多的位置

\end{itemize}

\begin{figure}[H]
\centering
\capstart

\noindent\sphinxincludegraphics{{goods_env_eg}.png}
\caption{例子}\label{\detokenize{chapter_knowledge/goods_analysis:id33}}\end{figure}


\subparagraph{竞品分类举例 2\sphinxfootnotemark[528]}
\label{\detokenize{chapter_knowledge/goods_analysis:id16}}%
\begin{footnotetext}[528]\sphinxAtStartFootnote
\sphinxnolinkurl{https://www.bilibili.com/video/BV1wz4y1y7sg?p=4}
%
\end{footnotetext}\ignorespaces 
烧饼:工艺、口味、为啥好?
\begin{enumerate}
\sphinxsetlistlabels{\arabic}{enumi}{enumii}{}{.}%
\item {} 
直接竞品:与产品定位(目标方向、目标用户需求、产品功能等)相似,这一类是最容易找到也最需要时常关注的竞品。边上家的烧饼

\item {} 
间接竞品:产品的目标人群可能相似,但是功能需求方面不太相同;或者产品的商业模式不同,但其他内容相似。卖麻花

\item {} 
潜在竞品:行业相近、业务相近的企业产品。卖臭豆腐的也想卖烧饼

\end{enumerate}
\begin{itemize}
\item {} 
一共有多少人跟我抢这块蛋糕?(竞争形势)

\item {} 
最好的几个是谁?(用户规模、融資、口碑)

\item {} 
他们用什么方法抢的?(产品模式)

\item {} 
他们产品有多少功能?(需求分析)

\item {} 
他们盈利模式是如何设计的?

\item {} 
运营转化策略是什如何推广的?

\item {} 
他们抢到了多少?

\item {} 
他们的发展曲线

\item {} 
竞品的优点和缺点

\item {} 
参考他们的转化漏斗模型:用户量\sphinxhyphen{}》活跃\sphinxhyphen{}》转化

\end{itemize}

观察角度:
\begin{itemize}
\item {} 
定位规划

\item {} 
功能设计

\item {} 
优化迭代

\end{itemize}

目的:为我所用!


\subparagraph{竞品概述}
\label{\detokenize{chapter_knowledge/goods_analysis:id17}}
如果是大项目,当然就不需要写对方的体量(下载量等数据),就直接写就可以了;如果你自己做的项目是比较小众的,那么就需要列一下对方的体量,体量是很好反映一个公司,一个产品受欢迎程度的指标;这个指标决定了看报告的人,会对你的报告投入多少的信任度,倒不是说你的报告,而是对于竞品的功能和要做功能的本身质疑;\sphinxhref{https://www.zhihu.com/question/23601989/answer/317794141}{9}%
\begin{footnote}[529]\sphinxAtStartFootnote
\sphinxnolinkurl{https://www.zhihu.com/question/23601989/answer/317794141}
%
\end{footnote}


\subparagraph{分析维度}
\label{\detokenize{chapter_knowledge/goods_analysis:id18}}
竞品的商业模式、竞品目标用户、竞品的运营推广营销策略、技术分析、市场份额,从这几个维度进行分析。\sphinxhref{https://www.pianshen.com/article/89602055805/}{7}%
\begin{footnote}[530]\sphinxAtStartFootnote
\sphinxnolinkurl{https://www.pianshen.com/article/89602055805/}
%
\end{footnote}
\begin{enumerate}
\sphinxsetlistlabels{\arabic}{enumi}{enumii}{}{.}%
\item {} 
竞品的商业模式。就是直接竞争产品如何盈利,如何赚钱的,对直接竞品内容的详细展开。

\item {} 
竞品目标用户。各个竞品根据产品定位的差异,或者推广方式和覆盖地区的不同,目标用户不一样。人人贷在北京,则它的目标用户主要是30\sphinxhyphen{}39岁的男性用户,且由于公司本身在北京,则北京用户居多。点融网和人人贷的用户年龄层次分布和性别比例差不多,主要是地域不一样,以上海用户居多。

\item {} 
竞品运营、营销、推广策略。从运营、营销、推广等维度分析产品迭代策略。

\item {} 
技术分析。包括项目研发可能遇到的技术壁垒,如人工智能、语音图像识别等。

\item {} 
市场份额。从不同角度了解竞品的市场情况,例如:可以通过Alexa网站了解流量排名,以及了解各大应用市场的安装量、活跃用户、地区分布、用户增长率等。

\end{enumerate}


\paragraph{无竞品}
\label{\detokenize{chapter_knowledge/goods_analysis:id19}}
绝大多数情况下,没有竞争对手是因为市场不存在、需求不存在。也可能是因为创业者把竞争对手理解得太狭隘了。

竞争对手不仅仅是那些很相似的产品,或者解决相似问题的产品,还包括整个行业生态,它们共同服务着我们的用户。

我们可以回顾用户生态的相关内容,结合对用户的理解,去了解更多的用户故事。了解用户每天在相关领域的各种所作所为、所思所想后,才能帮助我们更全面地发现竞品。


\subparagraph{ToB的「无竞品」}
\label{\detokenize{chapter_knowledge/goods_analysis:tob}}
即使市场上存在性质类似的产品,作为普通用户想访问和使用也不是那么容易

外部公开的相似ToB产品设计资料资料可能很少,但对内的话,如果稍微留心搜索寻找一下,是可以通过内网的论坛、云盘、设计交流站点、设计稿预览站点还有不定期举办的内部专业分享等,找到前人对于类似项目的设计文档与经验总结的,给自己的设计思路带来启发。

这些小的模块很多在我们熟悉的ToC产品里都能找到影子,具体到交互设计模式很多都是通用的

一边学、一边猜、一边悟,通过收集资料,不断分析拼凑自己的产品版图


\paragraph{竞品选择策略}
\label{\detokenize{chapter_knowledge/goods_analysis:id20}}
产品生命周期有所了解。主要包括四个发展阶段:导入期、成长期、成熟期和衰退期。

在品类不同的发展期间,用户对品类的认知是不一样的,对应的,竞品选择策略也是不一样的。
\begin{itemize}
\item {} 
导入期:家用轿车是更快的马车、不用马拉的马车

\item {} 
成长期:A领导,竞品很可能还是马车。第二、第三抢领导品牌A

\item {} 
成熟期:沃尔沃代表安全,宝马代表驾驶的乐趣,可能还有消费者认为,日系品牌代表省油等等

\item {} 
衰退期:需求下降,换一个赛道,特性的创新

\end{itemize}


\paragraph{步骤}
\label{\detokenize{chapter_knowledge/goods_analysis:id21}}

\subparagraph{第一种刚起步,从0\sphinxhyphen{}1}
\label{\detokenize{chapter_knowledge/goods_analysis:id22}}
step1:找到优质竞品

行业热门、人气最旺、融资最多、最具特色
\begin{itemize}
\item {} 
关键词,搜索:全部>只找最优秀的几个(前10)

\item {} 
行业调研过程中发现的优秀竞品

\item {} 
基础数据查找,进行筛选

\end{itemize}

step2:锁定核心竞品

step3:确认分析维度
\begin{itemize}
\item {} 
产品不同、行业不同、业不同、产品关注点不同,你需要跟老大沟通的

\item {} 
产品概述(介绍这款产品的业务,公司背景)

\item {} 
产品模式(模式分析,优劣对比)

\item {} 
用户细分(用户模式,用户画像)

\item {} 
基本运营现状(用户量、日活月活、单量等指标)

\item {} 
盈利模式(讲清楚盈利情况,讲细,都多少种,多少钱,角色差异。运营的资金投入对比其他产品线得是否是公司的发展重心)

\item {} 
产品技术(技术是否前沿、对于用户迅速增加的压力承受能力、稳定性、技术框架等)

\item {} 
数据分析(包含盈利情况、推广情况、用户群体覆盖面、市场占有率,总注册用户量/装机量/转化率/用活/在线时长/等。)

\item {} 
核心业务流程、核心功能、亮点(点,要细节,要细节,要细节)

\end{itemize}

step4:横向对比分析

step5:借鉴与规避竞品分析总结,结合我们自身情况,可以吸收的
\begin{itemize}
\item {} 
产品模式、用户细分、盈利模式、特色亮点也许是融资最多的

\item {} 
核心业务流程

\item {} 
核心功能

\item {} 
竞品总结(借鉴与规避)

\end{itemize}


\subparagraph{规避竞品的问题 6\sphinxfootnotemark[531]}
\label{\detokenize{chapter_knowledge/goods_analysis:id23}}%
\begin{footnotetext}[531]\sphinxAtStartFootnote
\sphinxnolinkurl{https://www.zhihu.com/pub/reader/119980992/chapter/1284104622898974720}
%
\end{footnotetext}\ignorespaces 
竞品遇到的问题是真真切切存在的,如果产品经理不认真地对待和规避,那么最终自己一定也会遇到这些问题。对于一款新产品来说,任何问题可能都会造成至少
10 万元的损失,更别提其他的时间成本、机会成本了。
\begin{enumerate}
\sphinxsetlistlabels{\arabic}{enumi}{enumii}{}{.}%
\item {} 
团队和产品不匹配:凭什么能够在自己不熟悉的项目里输出自己的价值。

\item {} 
只谈情怀,不谈收入:需要去预算收益,收入有安全感,才有希望。

\item {} 
步子迈得太大:围绕一个更细分的市场去搭建,通过一个小的点来切入。形成了绝对的竞争壁垒。

\item {} 
无法满足用户的需求:社交因为关系链本身无法迁移。

\end{enumerate}


\paragraph{方法}
\label{\detokenize{chapter_knowledge/goods_analysis:id24}}

\subparagraph{Base+Solution分析法}
\label{\detokenize{chapter_knowledge/goods_analysis:base-solution}}
Base:目标用户是什么?目标用户的核心需求是什么?通过什么解决方案能够满足?同其他产品相比,解决方案有什么差异化和卖点?如何推广营销?市场效果如何?

Solution:解决方案如何实现?还有多少空间?陌生用户进来如何使用?信息组织、交互如何?为什么要这么做?是否符合用户预期?配色、UI是否符合用户审美?用户会在哪里困惑?用户打开产品的频率如何?用户是否会向他人推荐产品?


\subparagraph{Yes/No法}
\label{\detokenize{chapter_knowledge/goods_analysis:yes-no}}
主要适用于功能层面,简单来说就是将功能点全盘罗列出,具有该功能点的产品A便标记为“Yes”,没有该功能点的B产品标记为“No”,通过比对可以清晰地了解功能点上产品间的异同。


\paragraph{不同阶段的竞品分析}
\label{\detokenize{chapter_knowledge/goods_analysis:id25}}

\subparagraph{产品定位规划阶段}
\label{\detokenize{chapter_knowledge/goods_analysis:id26}}
要解决的问题是:
\begin{itemize}
\item {} 
我们要怎么做?

\item {} 
如何来做才能赚到更多的钱?

\item {} 
如何比竞品做的更好?

\end{itemize}

了解竞品是怎么做的?
主要包含,用户细分,产品定位产品模式,业务模型,盈利模型,付费的转化漏斗模型等等。


\subparagraph{产品设计阶段}
\label{\detokenize{chapter_knowledge/goods_analysis:id27}}\begin{itemize}
\item {} 
我们的产品如何设计?

\item {} 
业务流程是什么?

\item {} 
都有那些功能?

\item {} 
功能逻辑的细节是什么?

\end{itemize}

这个阶段的竞品分析,更关注的是,以上这些内容竞品是怎么做的?


\subparagraph{产品的优化迭代阶段}
\label{\detokenize{chapter_knowledge/goods_analysis:id28}}
任何竞品分析都不可能是静态的,整个市场在变动,分析也应该长期保持更新:
\begin{enumerate}
\sphinxsetlistlabels{\arabic}{enumi}{enumii}{}{.}%
\item {} 
我们的产品存在哪些问题?

\item {} 
如何优化改进?

\item {} 
可以通过数据分析,并与竞品进行对比分析来,发现自身的问题

\end{enumerate}

根据产品的问题确定竞品分析的方式和内容


\paragraph{示例}
\label{\detokenize{chapter_knowledge/goods_analysis:id29}}
\begin{figure}[H]
\centering
\capstart

\noindent\sphinxincludegraphics{{goods_analysis_catalog}.png}
\caption{竞品分析目录\sphinxhref{https://t.qidianla.com/1130052.html}{13}\sphinxfootnotemark[532]}\label{\detokenize{chapter_knowledge/goods_analysis:id34}}\end{figure}
%
\begin{footnotetext}[532]\sphinxAtStartFootnote
\sphinxnolinkurl{https://t.qidianla.com/1130052.html}
%
\end{footnotetext}\ignorespaces \begin{itemize}
\item {} 
饿了么、美团外卖、百度外卖竞品分析by十三:\sphinxurl{https://www.jianshu.com/p/a49663820163}

\item {} 
生鲜电商APP竞品分析\sphinxhyphen{}盒马鲜生VS叮咚买菜:\sphinxurl{https://t.qidianla.com/1175887.html}

\item {} 
人人都是产品经理、PMCAFF竞品分析:\sphinxurl{https://t.qidianla.com/1156403.html}

\end{itemize}

不建议大家去网上看那些动辄几十页的竞品分析,找来各种不靠谱的数据,要知道绝大多数可以被网络上搜到的行业、某个App的下载量数据,基本都是假的。

不要去反推某个大厂App的前一个功能是怎么做的,因为这个真没有任何技术含金量,自嗨式的炫技,\sphinxstylestrong{既成的事实},自然不会锻炼到产品经理的思考能力。\sphinxhref{https://zhuanlan.zhihu.com/p/69502665}{8}%
\begin{footnote}[533]\sphinxAtStartFootnote
\sphinxnolinkurl{https://zhuanlan.zhihu.com/p/69502665}
%
\end{footnote}


\subparagraph{内容产品如何防盗版 5\sphinxfootnotemark[534]}
\label{\detokenize{chapter_knowledge/goods_analysis:id30}}%
\begin{footnotetext}[534]\sphinxAtStartFootnote
\sphinxnolinkurl{https://weread.qq.com/web/reader/8d632bc07208ed1c8d697c4k9bf32f301f9bf31c7ff0a60}
%
\end{footnotetext}\ignorespaces 
盗版也属于广义竞品的范畴。
\begin{itemize}
\item {} 
官方上场:版权方主动放出部分内容,占领流量入口,让想找盗版的人找不到盗版内容。

\item {} 
变目的为手段:随着盗版资源在市场上的传播,咨询业务也在无形中扩大了影响力。

\item {} 
产品本身创新:拉大盗版与正版的价值差距,给正版用户提供更多的增值服务。

\end{itemize}


\subparagraph{实际工作}
\label{\detokenize{chapter_knowledge/goods_analysis:id31}}
几乎很少去做一整个APP的竞品分析,也从来没有使用《用户体验要素》里头讲的战略层、范围层等理论去进行分析(知乎上关于如何找到产品实习工作的帖子,教大家用这样的方法写竞品,其实是\sphinxstylestrong{错的})。

因为在实际的工作中,做的人和看得人都是行业内从业者,对于市场盘子,竞品的体量心里大体都有个数,而时间都很宝贵,在大力倡导MVP(敏捷开发)的移动时代,每个版本的迭代一般也就一两个核心功能,所以一般竞品分析,我们就只做一两个核心功能的竞品分析就可以了。\sphinxhref{https://www.zhihu.com/question/23601989/answer/317794141}{9}%
\begin{footnote}[535]\sphinxAtStartFootnote
\sphinxnolinkurl{https://www.zhihu.com/question/23601989/answer/317794141}
%
\end{footnote}

\begin{figure}[H]
\centering
\capstart

\noindent\sphinxincludegraphics{{good_analysis_mindmap}.png}
\caption{竞品分析报告基本结构}\label{\detokenize{chapter_knowledge/goods_analysis:id35}}\end{figure}

产品经理如何进行竞品分析? \sphinxhyphen{} 留几手的回答 \sphinxhyphen{} 知乎
\sphinxurl{https://www.zhihu.com/question/23601989/answer/810093405}


\subsubsection{用户需求研究 1\sphinxfootnotemark[536]}
\label{\detokenize{chapter_knowledge/users_analysis:users-analysis}}\label{\detokenize{chapter_knowledge/users_analysis:id1}}\label{\detokenize{chapter_knowledge/users_analysis::doc}}%
\begin{footnotetext}[536]\sphinxAtStartFootnote
\sphinxnolinkurl{http://www.woshipm.com/operate/3627874.html}
%
\end{footnotetext}\ignorespaces 

\paragraph{用户}
\label{\detokenize{chapter_knowledge/users_analysis:id2}}

\subparagraph{什么是用户?}
\label{\detokenize{chapter_knowledge/users_analysis:id3}}
用户不是人,而是\sphinxstylestrong{多个需求的集合}。某个产品完全满足了某个人在某个场景下的某类需求,那么就可以说该场景下的这个人就是产品的一个用户


\subparagraph{用户的五个属性 {[}11{]}}
\label{\detokenize{chapter_knowledge/users_analysis:id4}}\begin{itemize}
\item {} 
异质性:每一个用户的偏好、认知、拥有的资源是不一样的,

\item {} 
情境性:没有情境就没有用户,同一用户在不同的情境下会有不容的反应和行为

\item {} 
可塑性:用户的偏好和认知会随着外界不同的信息刺激发生变化和演化

\item {} 
自利性:追求个人总效用最大化

\item {} 
有限理性:虽然追求理性,但是能力有限、判断经常出错,所以只能做到有限的程度

\end{itemize}


\subparagraph{场景例子 {[}15{]}}
\label{\detokenize{chapter_knowledge/users_analysis:id5}}\begin{itemize}
\item {} 
场景A:有个朋友喜欢夜跑,跑步时一定要听音乐,而且要听特别动感的。

\item {} 
场景B:喜欢在上下班的路上听音乐,地铁上或者公交车上。

\item {} 
场景C:很怀念中学时代,想找以前耳熟能详的歌曲,却无从下手。

\end{itemize}


\paragraph{UCD VS BCD {[}4{]}}
\label{\detokenize{chapter_knowledge/users_analysis:ucd-vs-bcd-4}}\begin{itemize}
\item {} 
UCD(User Centered Design): 以用户为中心的产品设计

\item {} 
BCD(Boss Centered Design): 以老板为中心的产品设计

\end{itemize}

无论产品的使用流程、产品的信息架构、人机交互方式等,以UCD为核心的设计都时刻高度关注并\sphinxstylestrong{考虑用户的使用习惯、预期的交互方式、视觉感受}等方面。

腾讯手机充值业务曾经收到老板提出的一个需求:将单次充值的最大额度,从之前的
500 元增加到 2000
元。因为老板自己有这个需求,容易认为其他用户可能也会有类似情况,也就是说这个需求在决策过程中犯了「以点带面、以偏概全」的毛病,放大了单一用户的小众个性化需求,取代了普适大众的真实需求。

谁最了解用户,谁最有发言权!


\subparagraph{用户导向不容易 {[}17{]}}
\label{\detokenize{chapter_knowledge/users_analysis:id6}}
Perspective
taking的心理学理论,人考虑问题习惯性先从自己出发,而不是对方出发。


\paragraph{了解用户}
\label{\detokenize{chapter_knowledge/users_analysis:id7}}

\subparagraph{为什么了解}
\label{\detokenize{chapter_knowledge/users_analysis:id8}}
涉及到方向性问题的时候,在混沌的信息中,在开放的无边界的信息中,找到适合的方向,这本身不是A/B测试能搞定的事情。

基于对用户的深刻洞察,才能谈价值发现,产品规划,产品设计,上线运营等。

\sphinxstylestrong{方法:}
\begin{itemize}
\item {} 
用户故事:关注以用户的视角描述其通过使用软件产品想要实现的\sphinxstylestrong{任务}和获得的价值。

\item {} 
同理心地图:关注描述\sphinxstylestrong{设身处地}以用户的视角在某个\sphinxstylestrong{情景}时遇到的问题,来挖掘目标与解决方案。

\item {} 
用户生态:关注用户\sphinxstylestrong{多样性},是否可以继续细分。

\item {} 
用户画像:关注用\sphinxstylestrong{关键特征}来描述的用户。

\item {} 
用户旅程:关注用户在情景的整个前中后的\sphinxstylestrong{过程}的行为、心理。

\end{itemize}


\paragraph{用户故事 {[}6{]}}
\label{\detokenize{chapter_knowledge/users_analysis:id9}}
用户故事的用途是以用户的视角描述其通过使用软件产品想要实现的任务和获得的价值。故事不同于传统需求规格说明书,以简化的形式促进团队交流,降低修改成本、灵活调整接受变化,同时故事以验收驱动的定义形式让所有干系人入对最终的目标建立共识。

以用户的语言来描述用户故事,以识别真正的用户故事而不是解决方案

即:用户故事——谁,在什么情况下,碰到了什么问题,有什么感受和情绪,现在又是怎么做的,现在的做法中又有哪些痛点,等等。
{[}9{]}

用户故事可分为三个层次:
\begin{enumerate}
\sphinxsetlistlabels{\arabic}{enumi}{enumii}{}{.}%
\item {} 
“主题”用户故事

\item {} 
“大”用户故事

\item {} 
“可开发”的用户故事

\end{enumerate}


\subparagraph{用户故事的INVEST原则}
\label{\detokenize{chapter_knowledge/users_analysis:invest}}\begin{enumerate}
\sphinxsetlistlabels{\arabic}{enumi}{enumii}{}{.}%
\item {} 
独立性(Independent)— 要尽可能的让一个用户故事独立于其他的用户故事。

\item {} 
可协商性(Negotiable)—
一个用户故事的内容要是可以协商的,用户故事不是合同。

\item {} 
有价值(Valuable)—
每个故事必须对客户具有价值(无论是用户还是购买方)。

\item {} 
可以估算性(Estimable)—开发团队需要去估计一个用户故事以便确定优先级,工作量,安排计划。

\item {} 
短小(Small)—
一个好的故事在工作量上要尽量短小,最好不要超过10个理想人/天的工作量,至少要确保的是在一个迭代或Sprint中能够完成。

\item {} 
可测试性(Testable)—一个用户故事要是可以测试的,以便于确认它是可以完成的。

\end{enumerate}


\paragraph{同理心地图}
\label{\detokenize{chapter_knowledge/users_analysis:id10}}
同理心地图给设计团队提供了一个思考框架,是帮助团队整理对用户的认识的一项工具。它帮助团队整合所观察和调研到的人和事物,并协助挖掘提出对用户深层次的理解目标、需求、观点、关注{[}20{]}、痛处、态度、行为)同时,同理心地图也是一个团队协同设计的工具,确保每位团队成员对使用者的理解都是相同的。

\begin{figure}[H]
\centering
\capstart

\noindent\sphinxincludegraphics{{empathy_map}.png}
\caption{empathy\_map}\label{\detokenize{chapter_knowledge/users_analysis:id37}}\end{figure}

简例:{[}15{]}
\begin{itemize}
\item {} 
目标用户可以细分:情怀型白领 / 运行型白领

\item {} 
使用场景:睡觉前 / 跑步时

\item {} 
用户目标:需要情怀式的曲目 / 不经常操作,连续播放动感音乐

\end{itemize}

如果想真正融入其中,对其有更深的了解,就一定要去\sphinxstylestrong{做对方所做的事情}。比如,产品经理要想了解某个产业的情况,最快的办法是自己做销售,真正地和销售走到一起,做销售要做的事情,接受销售要接受的考核。只有这样,产品经理才能真正了解中间的利益流转到底是怎样的、在销售过程中应该怎样做和产品的价值点到底在哪儿。


\paragraph{用户生态}
\label{\detokenize{chapter_knowledge/users_analysis:id11}}
在任何一个产品领域,用户都是多种多样的。所以,第二步我们要梳理用户生态。你需要了解,在产品所涉及的领域中,有哪几种用户,他们之间的关系是什么。

还要注意三点:
\begin{enumerate}
\sphinxsetlistlabels{\arabic}{enumi}{enumii}{}{.}%
\item {} 
颗粒度:某种用户可以继续细分,但分到什么程度,还没有定论,需要根据实际情况进行分析。比如家长要不要细分为爸爸和妈妈。

\item {} 
考虑边界:不同的用户和产品发生的关系有强有弱,最广义的用户,是指所有和产品有关系的人。那么,用户是否都要纳入我们日常的用户生态图,就是你需要考虑的“边界”。

\item {} 
优先级:已经被画在用户生态图中的用户,也是有重要、有次要,我们肯定是先照顾最重要的。

\end{enumerate}


\paragraph{用户画像}
\label{\detokenize{chapter_knowledge/users_analysis:id12}}
用户画像,我们要用一些关键特征来描述一个重要的用户群体。它可以帮助整个产品创新团队时刻牢记我们的产品是为谁服务的。

用户画像包含:
\begin{enumerate}
\sphinxsetlistlabels{\arabic}{enumi}{enumii}{}{.}%
\item {} 
基本信息,给这类用户的代表起个看起来真实的名字,选一个照片,设定性别、年龄、职业、日常的兴趣爱好。

\item {} 
描述用户的特定信息,也就是与产品领域相关的信息,比如生活方式、价值取向、心理预期等。

\item {} 
选几句在收集用户故事的时候,听到的用户说的有代表性的话,增强真实感。

\end{enumerate}

\begin{figure}[H]
\centering
\capstart

\noindent\sphinxincludegraphics{{QQ_users}.jpg}
\caption{QQ早期用户画像数据}\label{\detokenize{chapter_knowledge/users_analysis:id38}}\end{figure}

\begin{figure}[H]
\centering
\capstart

\noindent\sphinxincludegraphics{{user_persona}.png}
\caption{用户画像{[}22{]}}\label{\detokenize{chapter_knowledge/users_analysis:id39}}\end{figure}

TODO:https://zhuanlan.zhihu.com/p/28485415


\subparagraph{人的五个层次 {[}10{]}}
\label{\detokenize{chapter_knowledge/users_analysis:id13}}\begin{enumerate}
\sphinxsetlistlabels{\arabic}{enumi}{enumii}{}{.}%
\item {} 
感知层

\item {} 
角色框架层

\item {} 
资源结构层

\item {} 
人的能力圈

\item {} 
一个人对存在感的定义(这是一个人的内核,就是他对他自己为什么而存在,到底是怎么感知的。)

\end{enumerate}

存在感对于人就像生存对于动物一样,是触发情绪和推动行动的开关。最内核是存在感,它的外面一层是能力圈。如果一个人的存在感满足了,其实他的能力圈就不会再扩充了。

如果你明确知道自己想成为一个什么样的存在,你就会不断地改变自己的能力圈,改变自己的资源,甚至改变自己的样子。


\paragraph{找到核心用户}
\label{\detokenize{chapter_knowledge/users_analysis:id14}}
\begin{figure}[H]
\centering
\capstart

\noindent\sphinxincludegraphics{{RFM}.jpg}
\caption{RFM}\label{\detokenize{chapter_knowledge/users_analysis:id40}}\end{figure}

当我们寻找到RFM都高的群体,就是我们目标的核心用户。找到核心用户的下一步,其实是做核心用户的分群和拆解。核心用户仍然需要进一步的拆分,我们会从三个维度:特征维度,行为维度,需求维度来进行拆分,比如从性别,年龄,消费行为,使用行为还有潜在需求进行划分,这样能够精准的抓住我们的核心用户是谁。
{[}19{]}


\subparagraph{找到核心用户的需求}
\label{\detokenize{chapter_knowledge/users_analysis:id15}}
TGI : Target Group Index ( 目标群体指数)

TGI指数
\(=\frac{\text { 目标群体中具有某一特征的群体所占比例 }}{\text { 总体中具有相同特征的群体所占比例 }}\)
*标准数100


\subparagraph{如何去计算产品的上限}
\label{\detokenize{chapter_knowledge/users_analysis:id16}}
我们同样可以使用DAU来预测。在自然增长的情形下,第n天的日活理论上等于当天的新增用户,加上此前每一天的新增用户在当天的一个留存。

DAU(n)=A(n)+A(n\sphinxhyphen{}1)R(1)+A(n\sphinxhyphen{}2)R(2)+… …+A(1)R(n\sphinxhyphen{}1)

假设每日新增用户A相同,

DAU(n)=A(1+R(1)+… …+ R(n\sphinxhyphen{}1))

DAU(t)为第t天的日活,A(t)为第t天的新增用户,R(t)为新增用户在第t天后的留存。


\paragraph{用户旅程}
\label{\detokenize{chapter_knowledge/users_analysis:id17}}
如果说用户画像是静态的,那我们最后做的用户旅程,就是让用户“动起来”。

选一个重要的用户,思考他在解决相应问题的时候,都会碰到什么状况,做什么事,有什么感受和情绪。这时候,“有没有产品”依然不是重点,重点还是关注用户的言行举止。

用户旅程分为三段:
\begin{enumerate}
\sphinxsetlistlabels{\arabic}{enumi}{enumii}{}{.}%
\item {} 
做某事前的准备;

\item {} 
做某事的过程;

\item {} 
做完某事之后。

\end{enumerate}


\subparagraph{用户旅程地图}
\label{\detokenize{chapter_knowledge/users_analysis:id18}}
用户旅程地图(User Journey Map是和用户画像
persona)相辅相成的工具,用户画像代表的是具体的族群,而体验地图是分析这个族群为了实现某个目标而经历的过程的可视化呈现工具。它用于了解和解决客户需求和痛点,在这个过程中用户可能会使用多个设备和渠道(例如网站,手机app,社交媒体,电话,线下客等)
\begin{enumerate}
\sphinxsetlistlabels{\arabic}{enumi}{enumii}{}{.}%
\item {} 
阶段:用户实现某个目标所经历的具体步骤

\item {} 
行动:每一个步骤下用户所产生的具体行为习惯

\item {} 
想法:用户在这个过程中的想法和体会

\item {} 
情感曲线:用户在这个过程中不同阶段的情感波动

\item {} 
机会点:我们洞察到的能够改进的机会

\item {} 
改进点:将每个改进点对应的相应的责任人身上

\end{enumerate}


\paragraph{Persona 文档指导}
\label{\detokenize{chapter_knowledge/users_analysis:persona}}
\begin{figure}[H]
\centering
\capstart

\noindent\sphinxincludegraphics{{persona}.png}
\caption{Persona 文档}\label{\detokenize{chapter_knowledge/users_analysis:id41}}\end{figure}

可参考《About Face:交互设计精髓》{[}24{]}


\paragraph{价值主张画布}
\label{\detokenize{chapter_knowledge/users_analysis:id19}}
\begin{figure}[H]
\centering
\capstart

\noindent\sphinxincludegraphics{{value_map}.png}
\caption{价值主张画布}\label{\detokenize{chapter_knowledge/users_analysis:id42}}\end{figure}


\paragraph{故事板 {[}7{]}}
\label{\detokenize{chapter_knowledge/users_analysis:id20}}
故事板可以帮助用户预测并探索产品的用户体验,透过故事板的情境模拟以利设计师在设计过程中能去推测出使用者在使用过程中可能会遇上的问题,且帮助了解用户目前与问题相关的动机和经验,便于设计师能更进一步确立设计目标


\paragraph{同期群分析(Cohort Analysis)}
\label{\detokenize{chapter_knowledge/users_analysis:cohort-analysis}}
主要目的是分析相似群体随时间的变化(比如用户的回访)随看开发迭代的演进,产品上线第一个月使用你的产品的用户与第五个月使用你产品的用户感受到的体验是很不一样的。
我们把在同一个时间段(产品阶段)使用产品的用户划为同一期,针对他们的分析叫做同期群分析


\paragraph{需要了解到什么度}
\label{\detokenize{chapter_knowledge/users_analysis:id21}}
至少在你们公司,你应该是你们公司用户的专家,即其他人想要了解用户对某些场景或问题的看法时,如果想到咨询一个人的话,第一个想到的是你,那么你就是你们公司的用户专家。可以不断的问自己一个问题“自己是否可以称为用户专家,是否足够的洞察用户”,这需要时间的积累,在实践中回答这个问题,并不断的通过实践给出一个肯定的答案。


\subparagraph{怎么衡量了解的度}
\label{\detokenize{chapter_knowledge/users_analysis:id22}}
最简单直接的方法是假设验证法,即给定一个场景,给出你对用户的判断,然后以实际结果验证你的判断。不断的实践来提高对用户判断的准确度。

当给出任何场景,你对用户的判断八九不离十,知道用户是否存在这个问题?多少用户存在这个问题?用户当前是怎么解决这个问题的?是否值得做?做了之后用户是否能从原来的习惯中迁移过来?


\paragraph{指标 {[}4{]}}
\label{\detokenize{chapter_knowledge/users_analysis:id23}}
需求量、强度、频次、痛点、Arpu、期望、现有解决方案
\begin{itemize}
\item {} 
需求量:大众、小众

\item {} 
强度:刚需(必须需要);、弱需
越刚越容易付费;后验看“功能渗透率”(指当前使用该功能的人数占整体使用产品的人数比例。),渗透率越高,越说明这是用户的核心功能。{[}29{]}

\item {} 
频次:高、低频

\end{itemize}

\begin{figure}[H]
\centering
\capstart

\noindent\sphinxincludegraphics{{need_analysis}.png}
\caption{need\_analysis}\label{\detokenize{chapter_knowledge/users_analysis:id43}}\end{figure}
\begin{itemize}
\item {} 
痛点:解决某个需求时,遇到的最大障碍。

\item {} 
Arpu:用户价值。烧饼一两块、化妆品百来块、增高药上万

\item {} 
期望:超预期。才能拉新。

\end{itemize}

\begin{figure}[H]
\centering
\capstart

\noindent\sphinxincludegraphics{{lawyer_analysis}.png}
\caption{lawyer\_analysis}\label{\detokenize{chapter_knowledge/users_analysis:id44}}\end{figure}

为了需求找技术。


\paragraph{研究内容 {[}4{]}}
\label{\detokenize{chapter_knowledge/users_analysis:id24}}\begin{itemize}
\item {} 
用户特征:性别、年龄、职业、地域、学历、消费能力。TOFA(传统/时尚、节俭/花钱)

\item {} 
需求情景:在什么时候用,用的时候会发生什么?饿了么,来不及停止接单、在意配送时间准时保。

\item {} 
需求动机:聊天、结婚、约炮?微信熟人、陌陌陌生人不需加好友。

\item {} 
显性/隐性需求:隐性又是更重要

\item {} 
关注因素:在意什么?菜品口味、价格、送餐速度、干净卫生。大众用综合排序

\item {} 
认知过程:不知道》知道》了解》产生兴趣》学习

\item {} 
行为习惯:用户通常怎么做?由于认知决定。SICAS(Sense、Interest、Communication、Action、Share)
FOGG(motivation、ability、trigger)

\item {} 
行为心理:为何这么做?货比三家、怕吃亏上当。

\item {} 
使用过程:用户使用你产品或服务的过程。

\item {} 
决定因素:重大行为的决策。陌陌上找你喝酒,怕是酒托、仙人跳..

\end{itemize}


\subparagraph{需求情景–情节}
\label{\detokenize{chapter_knowledge/users_analysis:id25}}
主线:筛选饭店、点餐、支付、等餐、就餐。

分支:
\begin{enumerate}
\sphinxsetlistlabels{\arabic}{enumi}{enumii}{}{.}%
\item {} 
筛选饭店的方式:搜索、好评、默认推荐

\item {} 
查看送餐小哥什么时候能送到?要不要催促下?

\item {} 
饭到了很难吃,要不要给个差评?

\end{enumerate}

异常:退餐流程,这个流程中,又可以细分出N个情景

用户体验、满意度、需求满足程度
\begin{enumerate}
\sphinxsetlistlabels{\arabic}{enumi}{enumii}{}{.}%
\item {} 
产品不同,研究的内容和方法也会有差异,需要活学活用。

\item {} 
这条线上的每一个点,都会关联到你的(UML统一建模语言,推荐书籍《UML大战需求分析》),产品功能设计,运营策略,付费转化策略,营销推广的策略和\sphinxstylestrong{N个细节}。

\item {} 
产品的每一个细节,都跟用户需求有着千丝万缕的联系。大到产品定位规划,商业模式,竞争策略,小到每一句营销的文案编写,UI设计图那个字需要加大加粗,某个位置需要一个小图标。

\end{enumerate}


\paragraph{用户吐槽的小故事 {[}18{]}}
\label{\detokenize{chapter_knowledge/users_analysis:id26}}
腾讯内部有时候会举办各种产品投票,在一次票选【员工心中最需要改进的产品】活动中,QQ
浏览器排在了第一位。

几天后,腾讯内部 BBS 上出现了这样一条帖子:QQ
浏览器重金邀请同事们内测,团队郑重承诺:请大家尽情吐槽浏览器的所有问题,保证
100\% 及时回复,每 48 小时公开处理结果。

就这样,产品经理们收到了 1130 条反馈,归整为 606
类建议,团队马不停蹄开展了 10 轮讨论,在 15 天内完成了 72
项体验优化,推出了 CE 优化版。部门总经理逐字逐句修改内部 CE
的邮件,最后还在 200 多张致谢卡上一一签名。

3 年后,作为业界有口皆碑的浏览器,QQ
浏览器获得了「名品堂」奖项,这是腾讯奖励给优秀产品的最高荣誉。


\paragraph{洞察用户时常犯错误}
\label{\detokenize{chapter_knowledge/users_analysis:id27}}
1)以偏概全,因自己或周边人经常遇到某些场景,就以为绝大多数人会遇到类似场景,很感性的认知。举例:你朋友圈的热点可能真的只是你朋友圈的热点,在你父母那,在你高中同学那,在别的行业的大学同学那,甚至同行业同事那里,大家的热点都是有差异的。

2)常识性错误,比如我们知道一般老板会查看下属工作情况,老板也更关心公司的业绩统计数据,然后我们就可能认为下属资料和业绩统计分析会有较高的用户重合度,其实不一定,因为查看下属资料这个大概率是管理员做,但统计分析这种业务员也可以查看,甚至是老板指派专人管理。

3)过于相信数据,比如AI技术可以实现一些功能的自动化,我们通过自动化的开关来判定用户有没有使用,也通过用户对自动化数据的修改来判断用户是否真的将自动化使用起来。但数据表现都很好,不代表用户满意,用户可能只是不知道你给他自动做来那么多事情,甚至知道了,也觉得数据是错的,但选择忽略而已,需要更多的从用户真实的反馈中得到。

4)静态的看待用户的行为,无论我们做用户访谈,还是用户调研,得出的数据和内容是基于当时用户状态及对产品的了解,而用户在产品或服务使用的过程中,是会随时间的变化而变化的。比如对于C端用户勋章挑战类的功能,刚开始可能用户比较喜欢,参与度较高,但随着参与次数的提升,部分用户会有疲劳感,这在产品的设计中,就要考虑随时间周期变化的用户的反馈。


\paragraph{研究方法 {[}5{]}}
\label{\detokenize{chapter_knowledge/users_analysis:id28}}\begin{itemize}
\item {} 
用户访谈:主要是定性研究,是围绕一个特定的目的,通过不同的形式了解用户,从而获取受访用户对产品或者服务的感受、意见、建议,以及期望的过程。{[}14{]}

\item {} 
问卷调查:问卷设计一般都需要产品经理完成,然后可以找专业调研公司去实施。可参考
{[}24{]} 定量研究;工具:问卷星、腾讯问卷、金数据{[}25{]}

\item {} 
体验与观察

\item {} 
焦点小组:通常需要第三方专业公司提供服务,这种方式是一个主持人面对一组用户,按照访谈提纲进行半结构式的交谈,每次可访谈6~12个用户,是效率非常高的研究方法,但也是对主持人控场能力要求非常高的方法。

\item {} 
成为用户

\item {} 
参与式设计

\item {} 
卡片分类

\item {} 
亲和图法

\item {} 
可用性测试

\item {} 
数据分析

\item {} 
满意度调查

\end{itemize}


\subparagraph{用户访谈}
\label{\detokenize{chapter_knowledge/users_analysis:id29}}
发现需求背后的动机。

如:网易云音乐的跑步FM功能。得出结论。
\begin{itemize}
\item {} 
在跑步时听歌,除了简单的背景音乐外,还有更深层次的需求。

\item {} 
跑步很考验体能、耐力和毅力,心里如果一直想着自己跑得好累、跑了多少米、还有多少米,会感觉更累。听音乐能让人的注意力从疲劳的长跑中解放出来,专心聆听音乐,暂时\sphinxstylestrong{忘记疲劳},突破体力极限。

\item {} 
更重要的是,节奏感强的音乐非常利于控制跑步节奏,抵抗肌肉疲劳。让耳朵和思维专心跟随音乐节奏,身体就可以一直保持速度。这时音乐实际起到了引导跑步的作用,在跑步步频放慢时,刺激身体保持节奏;在跑步步频稳定时,让人尝试提高步频、跑得更快。{[}28{]}

\end{itemize}


\subparagraph{问卷调查}
\label{\detokenize{chapter_knowledge/users_analysis:id30}}
问卷设计方法:收集资料\sphinxhyphen{}研究问卷形式\sphinxhyphen{}列出标题和提纲\sphinxhyphen{}确定问卷\sphinxhyphen{}修改试测\sphinxhyphen{}重新修订\sphinxhyphen{}发放问卷

研究问卷问题以封闭式为主,题目10道左右,尽量让用户能在1分钟内完成一道题。不能有任何诱导用户的奖励,以免影响客观性。{[}27{]}

同样,问卷调查除了前面介绍、问卷目标的明确、问卷设计的考究。一定要注意:{[}26{]}
\begin{itemize}
\item {} 
目标不要过于多,这样的问卷会迷失方向;

\item {} 
要加上测试题方便剔除无效问卷;

\item {} 
目标用户投放渠道的选择不要偷懒,否则不真实:根据目标人群选择投放渠道,我的目标用户是大学生,所以我选择了学生群体做用户访谈。

\item {} 
结论很重要:因为它是你做问卷调查的目的,要为你的验证和猜想提供现实依据。

\end{itemize}

\begin{figure}[H]
\centering
\capstart

\noindent\sphinxincludegraphics{{questionaire}.png}
\caption{问卷调查}\label{\detokenize{chapter_knowledge/users_analysis:id45}}\end{figure}


\subparagraph{用户调研}
\label{\detokenize{chapter_knowledge/users_analysis:id31}}
意义:它是作为验证的手段。但它不能作为需求的来源。{[}23{]}

相关书籍《用户体验与可用性测试》《用户体验度量》


\paragraph{结果产出 {[}5{]}}
\label{\detokenize{chapter_knowledge/users_analysis:id32}}\label{\detokenize{chapter_knowledge/users_analysis:id33}}\begin{itemize}
\item {} 
用户画像

\item {} 
用户应用情景

\item {} 
主线情景

\item {} 
分支情景

\item {} 
异常情景

\item {} 
认知过程

\item {} 
关注因素

\item {} 
行为心理

\item {} 
决策心理

\item {} 
任务流程分析

\item {} 
逻辑与权重

\end{itemize}


\paragraph{用户访谈}
\label{\detokenize{chapter_knowledge/users_analysis:id34}}
\begin{figure}[H]
\centering
\capstart

\noindent\sphinxincludegraphics{{user_research}.jpg}
\caption{用户访谈示例:直播中后部}\label{\detokenize{chapter_knowledge/users_analysis:id46}}\end{figure}


\subparagraph{在求职和工作中会有什么作用呢?}
\label{\detokenize{chapter_knowledge/users_analysis:id35}}\begin{enumerate}
\sphinxsetlistlabels{\arabic}{enumi}{enumii}{}{.}%
\item {} 
在求职中,用户访谈是说服面试官的武器。

\end{enumerate}

在面试中,面试官可能会质疑你需求的合理性。例如,面试官可能会说:“我觉得你这个约别人骑行的需求是个伪需求。”如果你没有做用户访谈,那么你只能说:“我觉着这个需求肯定是存在的,因为我就有这个需求。”这样是不是显得底气不足?但是,如果你做过用户访谈,你就可以自信地说:“这个需求确实存在,因为我访谈了20
个目标用户,其中85\%的用户提到自己有这个需求,主要场景有两个,一个是当自己不会修车时可以找人帮忙,另一个是他们觉得一个人骑行很无聊。”这样的回答是不是更有说服力?
\begin{enumerate}
\sphinxsetlistlabels{\arabic}{enumi}{enumii}{}{.}%
\setcounter{enumi}{1}
\item {} 
在工作中,用户访谈是验证需求合理性的方法之一。

\end{enumerate}

作为产品经理,我们要“发现需求”,而不是“创造需求”。这就要求我们通过科学、严谨的方法去挖掘需求,而不是用拍脑袋的方式决定有什么需求。比较严谨的方式有两种,一种是数据分析,另一种是用户访谈。因此,在掌握了用户访谈的方法后,我们就可以在以后的工作中设计出更符合用户预期的产品。


\paragraph{消费心理 {[}8{]}}
\label{\detokenize{chapter_knowledge/users_analysis:id36}}\begin{itemize}
\item {} 
比附大腕以成就品牌:蒙牛一开始绑定伊利、借助内蒙古的优秀品牌。

\item {} 
通过情感联系来打造品牌:“钻石恒久远,一颗永留传”–“钻石有价,爱情无价”;贝尔电话–“女儿说她爱我们。”

\item {} 
掌握消费者的需求和心理:静的让你日日夜夜都感觉不到–伊莱克斯冰箱;“谁杀了兔子乔丹”–我穿耐克鞋,我是英雄

\item {} 
通过事件营销推广品牌:“水仙花”实验,向观众展示了水仙花在农夫山泉天然水和纯净水中的生长状况;刘翔与可口可乐,销量一度上升了30%。

\item {} 
珍惜消费群体:“飘柔”产品曾领先于洗发行业,为了低档市场的销售量,研发了“飘柔”系列的淋浴液及香皂,降低了品牌的市场价值。

\end{itemize}


\paragraph{AI 产品}
\label{\detokenize{chapter_knowledge/users_analysis:ai}}
用户需求来自大数据分析,用户行为关联传感器等新型的交互方式,用户心理依靠深度学习。

{[}3{]}: {[}4{]}: \sphinxurl{https://www.bilibili.com/video/BV1wz4y1y7sg?p=2} {[}5{]}:
\sphinxurl{https://www.bilibili.com/video/BV1wz4y1y7sg?p=3} {[}6{]}:
\sphinxurl{https://www.bilibili.com/video/BV1254y1D7Ht?from=search\&seid=14167562900175777805}
{[}7{]}: \sphinxurl{http://acadeck.com/?p=411} {[}8{]}:
\sphinxurl{https://wiki.mbalib.com/wiki/\%E6\%B6\%88\%E8\%B4\%B9\%E8\%80\%85\%E5\%BF\%83\%E7\%90\%86}
{[}9{]}: \sphinxurl{https://www.jianshu.com/p/60e79d46dde5} {[}10{]}:
\sphinxurl{https://www.jianshu.com/p/85ec807c56d3} {[}11{]}:
\sphinxurl{https://www.jianshu.com/p/02df7160b7b0} {[}12{]}: {[}13{]}:
\sphinxurl{http://www.woshipm.com/pd/841065.html} {[}14{]}:
\sphinxurl{https://weread.qq.com/web/reader/46532b707210fc4f465d044k6ea321b021d6ea9ab1ba605}
{[}15{]}: \sphinxurl{https://zhuanlan.zhihu.com/p/24855458} {[}16{]}:
\sphinxurl{https://www.zhihu.com/pub/reader/119980992/chapter/1284104632479215616}
{[}17{]}: \sphinxurl{https://www.zhihu.com/question/31154592/answer/51489241} {[}18{]}:
\sphinxurl{https://www.zhihu.com/market/paid\_column/1312360599620358144/section/1312363033470443520}
{[}19{]}: \sphinxurl{https://mp.weixin.qq.com/s/jK5vQWcUnKdro1kwqT8L8g} {[}20{]}:
\sphinxurl{https://tangjie.me/blog/230.html} {[}21{]}: \sphinxurl{https://36kr.com/p/1721300926465}
{[}22{]}:
\sphinxurl{https://coffee.pmcaff.com/article/1103103818721408/pmcaff?utm\_source=forum}
{[}23{]}: \sphinxurl{https://www.zhihu.com/question/29342383/answer/46616997} {[}24{]}:
\sphinxurl{http://www.woshipm.com/pmd/3024508.html} {[}25{]}:
\sphinxurl{https://www.zhihu.com/question/20791021/answer/1525593995} {[}26{]}:
\sphinxurl{https://t.qidianla.com/1149667.html} {[}27{]}:
\sphinxurl{https://t.qidianla.com/1156537.html} {[}28{]}:
\sphinxurl{https://www.zhihu.com/question/323588594/answer/890413615} {[}29{]}:
\sphinxurl{http://www.woshipm.com/pmd/3113347.html\#:~:text=\%E8\%80\%8C\%E5\%AF\%B9\%E4\%BA\%8E\%E6\%9F\%90\%E4\%B8\%AAAPP,\%E7\%9A\%84\%E5\%A4\%A7\%E9\%83\%A8\%E5\%88\%86\%E7\%94\%A8\%E6\%88\%B7\%E5\%88\%9A\%E9\%9C\%80\%E3\%80\%82}


\subsubsection{需求分析}
\label{\detokenize{chapter_knowledge/demand_analysis:id1}}\label{\detokenize{chapter_knowledge/demand_analysis::doc}}
产品60\%的错误存在于设计中,而设计的错误60\%的源于需求和分析活动。–《代码大全》的作者Steve
McConnell \sphinxhref{http://www.woshipm.com/pmd/4440884.html}{11}%
\begin{footnote}[537]\sphinxAtStartFootnote
\sphinxnolinkurl{http://www.woshipm.com/pmd/4440884.html}
%
\end{footnote}


\paragraph{报告文档 8\sphinxfootnotemark[538]}
\label{\detokenize{chapter_knowledge/demand_analysis:id2}}%
\begin{footnotetext}[538]\sphinxAtStartFootnote
\sphinxnolinkurl{https://tangjie.me/blog/83.html}
%
\end{footnotetext}\ignorespaces \begin{enumerate}
\sphinxsetlistlabels{\arabic}{enumi}{enumii}{}{.}%
\item {} 
需求分类:将需求以需要的类型分门别类的罗列好,并介绍清楚需求本意。

\item {} 
需求分析和判断:在这个部分介绍各个需求决策结果,将可行性需求留下,不可行需求放弃;通常这个部分只介绍放弃的需求和放弃的理由。

\item {} 
需求分位:将可行性需求进行分位表述,表明需求的轻重缓急,这个分位的决定因素有很多,需要参考三大因素进行评估;四象限定位法在普遍的公司里也会这样称呼,重要(紧急)、重要(不紧急)、不重要(紧急)、不重要(不紧急)。

\item {} 
需求分级:根据分位再分优化等级,将需求划分计划,\sphinxstylestrong{根据不同规划阶段分多个版本实现};如果需求很少,那么就一次性迭代实现了。

\end{enumerate}


\subparagraph{需求分类 10\sphinxfootnotemark[539]}
\label{\detokenize{chapter_knowledge/demand_analysis:id3}}%
\begin{footnotetext}[539]\sphinxAtStartFootnote
\sphinxnolinkurl{https://zhuanlan.zhihu.com/p/340058145}
%
\end{footnotetext}\ignorespaces \begin{enumerate}
\sphinxsetlistlabels{\arabic}{enumi}{enumii}{}{.}%
\item {} 
功能类:通过人与系统交互的方式,来帮助业务同学解决问题,完成任务。例如:订单修改功能、记录备注功能、线索分配功能等。此类需求需要业务同学具有较好的逻辑思维能力,能把需求逻辑或规则梳理描述清楚。

\item {} 
数据类:主要包括两块内容,一个是数据字段定义,即数据统计口径。很多时候出现数据不准确的问题都是因为数据口径不一致导致的。另外一个是筛选条件,当数据较多时,能够有效地对数据进行筛选。

\item {} 
体验类:C端产品,用户体验直接关系到用户转化和用户留存,关乎到产品的成败,至关重要;B端产品,关系到业务同学的工作效率,可以通过优化操作流程或页面呈现来提升用户体验。

\item {} 
性能类:一般是指响应速度、可靠性、安全性等。例如:页面过很久才能打开,响应速度慢;或者有时候能打开,有时间打不开,可靠性较差;或者可以随意修改数据,安全性较低。

\item {} 
BUG类:是指已上线功能未能按照需求逻辑得到预期结果。业务同学在描述BUG的同时,最好能提供CASE方便技术同学排查。BUG可能会带来未知的损失,需要立即被解决,所以这也是优先级最高的需求。

\end{enumerate}


\paragraph{如何将用户需求转换为产品需求?}
\label{\detokenize{chapter_knowledge/demand_analysis:id4}}
首先保持二八原则,\sphinxstylestrong{只有普遍用户的需求,才能内化为产品的需求。}比如某个需求就一个用户需要,其他大多数用户都不需要,你就不需要做。

通过现象看本质,收集用户需求以后,\sphinxstylestrong{多为自己几个为什么,找到用户的动机。}

例如:用户在沙漠中需要水,你就要问自己用户为什么需要水?用户有可能口渴了,那这时候你给他水就好,如果用户是因为太热,你能不能给他防晒服,甚至考虑一下用户体验,觉得防晒服太麻烦,提供防晒霜。有时候一个人并不能完全洞察用户的动机,需要团队的其他人员一起头脑风暴,甚至多问提这个需求的原始用户几个为什么,直到找到真正动机为止,然后结合产品本身衡量需求的性价比,最后综合团队实力,需求急切度确定最终产品需求。


\subparagraph{需求提取 5\sphinxfootnotemark[540]}
\label{\detokenize{chapter_knowledge/demand_analysis:id5}}%
\begin{footnotetext}[540]\sphinxAtStartFootnote
\sphinxnolinkurl{https://blog.csdn.net/eickandy/article/details/80294224}
%
\end{footnotetext}\ignorespaces 
“如果我最初问消费者他们想要什么,他们应该是会告诉我,‘要一匹更快的马!’”

——这是亨利·福特的一句经典名言,如今我们在《乔布斯传》里又见到了它。

客户需求有显性需求和隐性需求两大类。我们通过市场调查得知的往往都是一些诸如“我要一匹更快的马”这类显性需求。客户的显性需求并不是客户真正的需求。企业需要根据所收集的显性需求信息进行深度挖掘和捕获,以了解客户的隐性需求是什么,进而分析出客户的真正需求是什么(例如:用更短的时间、更快地到达目的地)。这就是一个需求分析的过程。


\subparagraph{Y 模型 6\sphinxfootnotemark[541]}
\label{\detokenize{chapter_knowledge/demand_analysis:y-6}}%
\begin{footnotetext}[541]\sphinxAtStartFootnote
\sphinxnolinkurl{https://www.jianshu.com/p/2af332aaa017}
%
\end{footnotetext}\ignorespaces 
\begin{figure}[H]
\centering
\capstart

\noindent\sphinxincludegraphics{{Y_model}.png}
\caption{Y 模型}\label{\detokenize{chapter_knowledge/demand_analysis:id18}}\end{figure}

配图中你可以看到一个大写的字母 Y,有三个线段、四个节点。
\begin{itemize}
\item {} 
“节点
1”代表的是用户需求场景,经常被简称为用户需求。这是起点,是表象,是表面的需求,是用户的观点和行为。

\item {} 
“节点
2”是用户需求背后的\sphinxstylestrong{目标和动机},是用户言行的原因。不过产品经理在思考用户目标时也要综合考虑公司、产品的目标。

\item {} 
“节点 3”是产品功能,是解决方案,是技术人员能看懂的描述。

\item {} 
“节点
4”是人性与价值观,或者说是用户心智,是需求的最深层体现,是需求的本质。

\end{itemize}

Y 模型的不同阶段,各自需要回答一些问题,可以总结为 6 个 W 和 3 个 H。

“节点 1”这个阶段的问题主要是
Who(目标用户是谁)、What(需求表现为什么)和
Where/When(何时何地,什么情况下)。

“节点 1”到“节点 2”和“节点 2”到“节点
4”这个阶段,对应的是对用户需求的层层深入。这个阶段要回答 Why
这个问题——要不停地往下深入挖掘需求,了解用户为什么会有这样的言行、为什么会有这样的目标和动机。

“节点 4”到“节点 2”再到“节点 3”的过程中,你要想清楚
How——也就是要想清楚问题应该怎么解决。这个叫浅出,先深入后浅出。

“节点 3”中,要回答 Which、How many、How much 三个问题。

Which
是指选哪一个方案,做哪一个功能,这背后其实是对价值的判断,比如怎么评估性价比和优先级。How
many 是指这一次做多少个功能,考验的是对迭代周期,产品包大小的把控。How
much
原意是多少钱,这里引申为多少资源,是对时间、金钱、团队等资源的评估。
\begin{quote}

在一个不成熟的领域或全新的市场,只做“节点
1、2、3”是玩得转的。但表层需求很快就会被相似的跟进产品满足,随着市场的成熟,产品很快会陷入同质化竞争和价格战,最终整个市场变成红海。而破局的方法就在“节点
4”(用户心智)。
\end{quote}

作为初创团队,你要做的重要事情只有两件,一是和用户交流,二是开发产品。
———保罗·格雷厄姆

创新会面临两大风险,一是市场风险,就是说你做的产品有没有人需要,二是技术风险,就是说你能不能把东西做出来。

和用户交流,就是 Y
模型的前半段,是解决市场风险,对应的心法就是用心听;而开发产品,就是 Y
模型的后半段,是解决技术风险,对应的心法是不要照着做。


\paragraph{痛点剖析 7\sphinxfootnotemark[542]}
\label{\detokenize{chapter_knowledge/demand_analysis:id6}}%
\begin{footnotetext}[542]\sphinxAtStartFootnote
\sphinxnolinkurl{https://blog.csdn.net/kepengs/article/details/92955349?utm\_medium=distribute.pc\_relevant.none-task-blog-BlogCommendFromMachineLearnPai2-9.baidujs\&dist\_request\_id=1328740.12762.16168265945036403\&depth\_1-utm\_source=distribute.pc\_relevant.none-task-blog-BlogCommendFromMachineLearnPai2-9.baidujs}
%
\end{footnotetext}\ignorespaces \begin{enumerate}
\sphinxsetlistlabels{\arabic}{enumi}{enumii}{}{.}%
\item {} 
痛点是个体合适普遍?

\item {} 
痛点的需求是否符合政策导向,是否合规?

\item {} 
痛点的需求是否高频应用?

\item {} 
用户是否愿意为痛点买单?愿意付出多大代价?

\item {} 
感受到痛点的用户是不是具有采购决策权?

\end{enumerate}


\paragraph{优先级安排 1\sphinxfootnotemark[543]}
\label{\detokenize{chapter_knowledge/demand_analysis:id7}}%
\begin{footnotetext}[543]\sphinxAtStartFootnote
\sphinxnolinkurl{https://www.bilibili.com/video/BV1254y1D7Ht?from=search\&seid=14167562900175777805}
%
\end{footnotetext}\ignorespaces 

\subparagraph{why? 2\sphinxfootnotemark[544]}
\label{\detokenize{chapter_knowledge/demand_analysis:why-2}}%
\begin{footnotetext}[544]\sphinxAtStartFootnote
\sphinxnolinkurl{https://zhuanlan.zhihu.com/p/22067195}
%
\end{footnotetext}\ignorespaces 
人类拥有无穷的欲望,却只拥有有限的资源。熊掌与鱼,不可兼得!


\subparagraph{对「优先级」的定义关系产品成败}
\label{\detokenize{chapter_knowledge/demand_analysis:id8}}
苹果早期 iPhone
的设计是优先级控制的典范。手机正面只有一个按键的设计,尽管现在看来似乎是理所当然的,但在当时却是非常勇敢和有争议的决定。我曾深入研究过
iOS
系统早期的设计,在很多地方的取舍做的非常到位,能够大胆砍掉之前手机系统常见的功能和界面元素,让重点变得更重点,让需要突出的内容变得更突出。

而相比之下,同一时期诺基亚的系统尽管拥有大量功能,在呈现给用户时并没有处理好优先级,对于用户相对要更复杂,如果读者还有印象,可以想象一下打开一个联系人,看看与之对应的常常的功能菜单。而那个时候微软的
Windows Mobile 系统,则是大量将 PC
上的体验搬到手机上,用户的认知资源和系统有限的显示和交互资源之间并不匹配。


\subparagraph{结构化优先级}
\label{\detokenize{chapter_knowledge/demand_analysis:id9}}\begin{enumerate}
\sphinxsetlistlabels{\arabic}{enumi}{enumii}{}{.}%
\item {} 
信息优先级要关注内容的组织关系和轻重缓急

\item {} 
视觉优先级重在引导用户视线和行为轨迹

\item {} 
交互优先级要区分主线和支线任务

\item {} 
需求优先级要平衡用户目标和商业目标

\item {} 
用户优先级要界定核心用户是谁

\end{enumerate}


\subparagraph{信息优先级}
\label{\detokenize{chapter_knowledge/demand_analysis:id10}}
余额宝:收益、总金额。字号大,想让你看!麦肯兹金字塔。


\subparagraph{视觉优先级}
\label{\detokenize{chapter_knowledge/demand_analysis:id11}}\begin{itemize}
\item {} 
报纸上的文字大小、颜色、区块

\item {} 
海报朝着商品上看。

\end{itemize}


\subparagraph{交互优先级}
\label{\detokenize{chapter_knowledge/demand_analysis:id12}}
区分主线和支线,\sphinxstylestrong{突出主线任务}
\begin{itemize}
\item {} 
读书app,点一下会设置,再点一下只能前后。读书!

\item {} 
支付宝,收付钱。

\item {} 
滴滴,预约用车、现在用车。感觉车很多。

\end{itemize}

建立故事,记忆更深刻。

豌豆荚几亿钱分十几个人!


\subparagraph{需求优先级}
\label{\detokenize{chapter_knowledge/demand_analysis:id13}}
需求优先级要平衡用户目标和商业目标。
\begin{itemize}
\item {} 
用户目标:炫耀。读取型号、找对应图片、程序合成

\item {} 
商业目标:更多用户。使用豌豆荚截图+\sphinxstylestrong{网址}。留个空间给裁。

\end{itemize}


\subparagraph{项目范围优先级制定一沟通计划}
\label{\detokenize{chapter_knowledge/demand_analysis:id14}}
By who、Who、How、Why、When、What。

制定沟通计划的目的是为项目交付周期的交流和相互支持提供指导。在敏捷项目里,面对面交流比文档要好,但是依然会有一些共享文件,比如报告和项目计划,需要留下档案。


\subparagraph{四象限 3\sphinxfootnotemark[545]}
\label{\detokenize{chapter_knowledge/demand_analysis:id15}}%
\begin{footnotetext}[545]\sphinxAtStartFootnote
\sphinxnolinkurl{https://www.bilibili.com/video/BV1254y1D7Ht?from=search\&seid=14167562900175777805}
%
\end{footnotetext}\ignorespaces 
优先级顺序:重要、紧急(立即做)>重要、不紧急(时间表)>不重要、紧急(委派)>不重要、不紧急(排除)

重要程度大致的排序如下:
\sphinxhref{https://weread.qq.com/web/reader/40632860719ad5bb4060856ke3632bd0222e369853df322}{4}%
\begin{footnote}[546]\sphinxAtStartFootnote
\sphinxnolinkurl{https://weread.qq.com/web/reader/40632860719ad5bb4060856ke3632bd0222e369853df322}
%
\end{footnote}
\begin{itemize}
\item {} 
不做会造成严重问题和恶劣影响的;

\item {} 
做了会产生巨大好处和极佳效果的;

\item {} 
同重要合作对象或投资人有关的;

\item {} 
同核心用户利益有关的;

\item {} 
同大部分用户权益有关的;

\item {} 
同效率或成本有关的;

\item {} 
同用户体验有关的。

\end{itemize}

紧急程度大致的排序如下:
\begin{itemize}
\item {} 
不做错误会持续发生,然后造成严重影响;

\item {} 
在一定时间内可控,但长期会有糟糕的影响;

\item {} 
做了立刻能解决很多问题、产生正面的影响;

\item {} 
做了在一段时间后可以有良好的效果。

\end{itemize}


\paragraph{需求分析失败}
\label{\detokenize{chapter_knowledge/demand_analysis:id16}}\begin{enumerate}
\sphinxsetlistlabels{\arabic}{enumi}{enumii}{}{.}%
\item {} 
不完整的需求:主要是因为我们撰写的“需求规格说明书”这类文档里面,里面充斥了很多技术术语如:数据字典管理、报表子系统、API接口、写死、耦合、黑白名单、新增客户之类。有时候会让一些技术功底并不深厚的用户很难理解这些技术文档。所以我们建议:要让用户能够更好地参与到完整的需求分析过程,我们需要采用一个“业务导向”的组织结构,而不是让用户将一大堆技术动作翻译成自己的业务场景中。最好是利用树形层次介机构,将宏观信息与微观信息进行有效的剥离。

\item {} 
缺乏用户的参与:在很多产品开发项目过程中,经常会出现缺乏用户参与的问题。出现这样情况主要可能是:需求人员自以为共同清楚用户需求,但不将自己的撰写好的需求文档进行二次确认。不会根据开发进度与客户保持同步沟通确认。当然客户也经常说这样的话:“你们先做,做出来后我们用用,有问题再改吧”。类似还有客户还有认为确认是事不关己、自己利益无关、不感兴趣的事。这时候对于需求分析员而言,真正能够让用户主观能动地参与进来,是要基于业务利益(解决问题、创造机会、提高管控力、推进项目进度等)的沟通

\item {} 
不切实际的用户期望:很多时候,用户总是天真地提出了大量的需求,有些是技术上根本无法实现的,有些是在原本较少费用和时间预算下无法实现的。所以在软件的无形和成本的不透明的情况下,导致我们认为客户在不切实际的需求。作为需求分析人员为了让我们客户提出一些实际可行的需求,应该主动帮助客户更好理解软件产品的成本。也就是说,单单告诉客户我们做不到是无效的,而应该说明为什么做不到才能够解决问题。

\item {} 
需求变更频繁:这就是我们经常碰到客户反悔的情况。客户常常会说:我不就是才提了两个需求,怎么就被你们说成变更很大、变更频繁做不了呢?也就是说用户根本没有意识到你变更对于软件项目工程会带来怎样的负面影响。目前大多数企业的需求分析过程,没有一个需求评审的过程,应该找到所有关键人员一起讨论一些需求到底该不该变、能不能变、能变多少。也有权决绝不合理的需求变更的要求。同时需求变更类型做一个统计,从根源上找出问题,改进整个需求过程

\item {} 
提供了不再需要的:也就是最后做出产品,发现功能不是用户能用上,好似一个鸡肋的伪功能。这真是一个浪费开发资源和时间的错举呀。那么针对这种情况,我们主要是做到事先预防,应该有效地划分需求优先级,真正基于业务领域知识来衡量需求的必要性和充分性。\sphinxhref{https://www.zhihu.com/column/c\_199236458}{9}%
\begin{footnote}[547]\sphinxAtStartFootnote
\sphinxnolinkurl{https://www.zhihu.com/column/c\_199236458}
%
\end{footnote}

\end{enumerate}


\paragraph{更多}
\label{\detokenize{chapter_knowledge/demand_analysis:id17}}
\sphinxurl{https://www.google.com/search?q=requirements+engineering\&rlz=1C1GCEA\_enJP909HK909\&oq=Requirements+Engineering\&aqs=chrome.0.0l8j46j0.902j0j1\&sourceid=chrome\&ie=UTF-8}


\subsubsection{产品体验报告}
\label{\detokenize{chapter_knowledge/experience_report:id1}}\label{\detokenize{chapter_knowledge/experience_report::doc}}
目的是锻炼和培养“产品敏感度”和“共情心理”.
特别注意需求提炼、场景描述这2个部分. 最重要的是, 要锻炼到一看到产品,
就能很快代入用户角色,
从用户的场景和视角来思考问题。\sphinxhref{https://www.zhihu.com/people/woyaonuliya/postss}{2}%
\begin{footnote}[548]\sphinxAtStartFootnote
\sphinxnolinkurl{https://www.zhihu.com/people/woyaonuliya/postss}
%
\end{footnote}

\begin{figure}[H]
\centering
\capstart

\noindent\sphinxincludegraphics{{product_experience_mindmap}.png}
\caption{产品体验报告基本结构\sphinxhref{https://g.yuque.com/zhongguodianxinyanjiuyuan/bgso10/fqs7tp}{3}\sphinxfootnotemark[549]}\label{\detokenize{chapter_knowledge/experience_report:id2}}\end{figure}
%
\begin{footnotetext}[549]\sphinxAtStartFootnote
\sphinxnolinkurl{https://g.yuque.com/zhongguodianxinyanjiuyuan/bgso10/fqs7tp}
%
\end{footnotetext}\ignorespaces 
1.需求分析\sphinxhref{https://www.jianshu.com/p/9fff898ce6bd}{1}%
\begin{footnote}[550]\sphinxAtStartFootnote
\sphinxnolinkurl{https://www.jianshu.com/p/9fff898ce6bd}
%
\end{footnote} 1.1 产品定义
1.2 用户需求 2.功能分析

3.竞品分析

\sphinxurl{https://www.bilibili.com/video/BV1ng4y1z7tW?from=search\&seid=6415503955533295161}

TODO:
\begin{itemize}
\item {} 
响巢看看产品体验报告:\sphinxurl{https://www.jianshu.com/p/905028cf416b}

\item {} 
「玩票」产品分析报告: \sphinxurl{https://www.jianshu.com/p/17ddec8fe63f}

\item {} 
携程旅行产品分析报告:\sphinxurl{https://www.jianshu.com/p/31e49097ab4e}

\item {} 
每天读点故事APP产品分析报告:\sphinxurl{http://www.woshipm.com/evaluating/1103753.html}

\item {} 
“全民K歌”产品分析报告:\sphinxurl{https://vickydyy.github.io/2019/06/22/KTV-Product-Analysis/}

\item {} 
今日头条App产品体验报告:\sphinxurl{https://www.jianshu.com/p/4f851659892a}

\end{itemize}


\subsubsection{竞争分析}
\label{\detokenize{chapter_knowledge/compete_analysis:id1}}\label{\detokenize{chapter_knowledge/compete_analysis::doc}}

\paragraph{定义}
\label{\detokenize{chapter_knowledge/compete_analysis:id2}}
把市场分析放在了产品分析之后,和竞品分析并入到竞争分析。


\subparagraph{竞品思维误区}
\label{\detokenize{chapter_knowledge/compete_analysis:id3}}\begin{enumerate}
\sphinxsetlistlabels{\arabic}{enumi}{enumii}{}{.}%
\item {} 
你所以为的竞品,以为是领导,只是自己认为的竞品,别人根本没把你当竞品。

\item {} 
只把关注点集中在了那几个竞争对手上,而企业的竞争是多方面的,复杂的。

\item {} 
运用竞争研究资料时,自己反而会有意无意的模仿。

\end{enumerate}


\subparagraph{竞争思维}
\label{\detokenize{chapter_knowledge/compete_analysis:id4}}
企业之间也是如此,当发生竞争的时候,彼此的利益就会发生转移,是在\sphinxstylestrong{抢夺利益。}

当用竞争思维去分析时,就不会去把关注点全集中在那几个头部品牌的表面分析上,会更深入的研究竞争者和竞争局面。这才是研究竞争的真正意义。

用竞争思维去研究的时候,自然就会看那些真正跟自己形成竞争的品牌,关注竞争对手同时,也看整个大局。

所以我说是做竞争研究分析,不做竞品研究分析。\sphinxhref{https://www.zhihu.com/question/19717171}{5}%
\begin{footnote}[551]\sphinxAtStartFootnote
\sphinxnolinkurl{https://www.zhihu.com/question/19717171}
%
\end{footnote}


\paragraph{怎么做}
\label{\detokenize{chapter_knowledge/compete_analysis:id5}}
6个要点
\begin{enumerate}
\sphinxsetlistlabels{\arabic}{enumi}{enumii}{}{.}%
\item {} 
行业的总体竞争态势;

\item {} 
画竞争地图;

\item {} 
行业领导者梳理;

\item {} 
识别核心竞争者(细分出竞争能力:资金、渠道、产品技术、品牌);

\item {} 
识别核心竞争者的传播策略;

\item {} 
识别异军突起的企业;

\end{enumerate}


\paragraph{竞争战略}
\label{\detokenize{chapter_knowledge/compete_analysis:id6}}
行业内:领先者、跟随者、新加入者。
\begin{itemize}
\item {} 
差异化:淘宝(C2C、性价比)、京东(B2C、配送快、购物体验好)、美丽说蘑菇街(女性电商导购到淘宝)、拼多多(拼一刀、货找人)

\item {} 
跟随:慧聪网跟随淘宝

\item {} 
成本优势:京东VS苏宁国美:偏远郊区来仓储VS闹市门店来摆卖

\item {} 
免费:360VS金山瑞星卡巴斯基,互联网思维:培养用户来赚钱VS软件思维:付软件。

\item {} 
好人/坏人:囚徒困境,好人+以牙还牙。

\end{itemize}


\paragraph{分析模型}
\label{\detokenize{chapter_knowledge/compete_analysis:id7}}\begin{itemize}
\item {} 
定位自己
\sphinxhref{https://www.zhihu.com/pub/reader/119919151/chapter/1283860051246137344}{4}%
\begin{footnote}[552]\sphinxAtStartFootnote
\sphinxnolinkurl{https://www.zhihu.com/pub/reader/119919151/chapter/1283860051246137344}
%
\end{footnote}:MOST
分析模型

\item {} 
分析外部
\sphinxhref{https://www.zhihu.com/pub/reader/119919151/chapter/1283860051246137344}{4}%
\begin{footnote}[553]\sphinxAtStartFootnote
\sphinxnolinkurl{https://www.zhihu.com/pub/reader/119919151/chapter/1283860051246137344}
%
\end{footnote}:PESTLE
分析模型

\item {} 
自己与外部通盘考虑:SWOT分析

\end{itemize}


\subparagraph{MOST 分析模型}
\label{\detokenize{chapter_knowledge/compete_analysis:most}}
MOST 分析模型中的 4
个字母分别代表了企业的使命(Mission)、目标(Objectives)、战略(Strategies)和战术(Tactics),产品经理在做商业分析时最好能够识别这些元素,它们可以帮助公司进行初步定位,对要实现的商业目标以及如何解决使命、目标、战略和战术的执行问题进行全面分析。


\paragraph{PESTLE 分析模型}
\label{\detokenize{chapter_knowledge/compete_analysis:pestle}}
PESTLE 分析模型中的 6 个字母分别为
Political、Economic、Sociological、Technological、Legal 和 Environmental
的首字母,它比 PEST
分析模型多了法律和环境两个维度,使用此模型可以评估各种外部因素对公司的影响,并确定如何解决这些因素可能造成的问题。


\paragraph{SWOT分析}
\label{\detokenize{chapter_knowledge/compete_analysis:swot}}
SWOT分析,又称态势分析,它是一种常用的战略规划分析模型,主要针对产品研发过程中涉及的一些关键要素进行分析。

SWOT分析是从优势(Strengths)、劣势(Weakness)、机会(Opportunity)、威胁(Threats)四个方面进行的。其中优势和劣势是针对自身而言的,即内部资源的优劣势;而机会和威胁则是针对外部而言的,可从市场、环境、社会等方面分析产品存在的机会和面临的威胁。

产品经理通过一系列调查将这些内容列举出来,并依照矩阵的形式进行排列,然后将产品设计过程中涉及的各种因素相互匹配加以分析,进而得出一系列相应的结论。记得别高估自己、低估对手!
\sphinxhref{https://weread.qq.com/web/reader/0c032c9071dbddbc0c06459k37632cd021737693cfc7149}{3}%
\begin{footnote}[554]\sphinxAtStartFootnote
\sphinxnolinkurl{https://weread.qq.com/web/reader/0c032c9071dbddbc0c06459k37632cd021737693cfc7149}
%
\end{footnote}

SWOT分析通常在产品规划期内使用,用以确定产品的战略方。
\begin{itemize}
\item {} 
SW分析:着眼于企业内部,由于企业是一个整体,且资源分散在各个部门或岗位之中,因此进行SW分析时必须从企业整个价值链中的每个环节入手,将涉及的每个环节与竞争对手进行对比,对比时也需要针对企业和产品进行特别分析。例如,设计一款智能音箱,需要分析企业能力,如产品部门的设计能力、研发部门的技术储备、生产制造工艺、生产能力、资金支持情况、人力资源储备、市场渠道、销售渠道等。分析产品时可以从新颖度、识别精度、制造工艺、价格优势等方面入手。

\item {} 
OT分析:随着社会、政治、经济、科技等方面的变化和发展,企业所处的环境也更为开放。由于这种外部环境变化是不受控制的,因此在产品分析过程中OT分析也变得至关重要。外部环境既存在威胁,也存在机会。环境威胁指的是由环境中一种不利的发展趋势所形成的不可预知的挑战,如果不采取相应的措施,这种不利趋势可能导致产品的优势被削弱。环境机会则是可能对产品形成优势的因素,通过这些机会,产品的竞争力将得到提高。

\end{itemize}

\begin{figure}[H]
\centering
\capstart

\noindent\sphinxincludegraphics{{taobao_SWOT}.png}
\caption{淘宝SWOT\sphinxhref{http://yun.itheima.com/course/671.html?2010stt}{6}\sphinxfootnotemark[555]}\label{\detokenize{chapter_knowledge/compete_analysis:id10}}\end{figure}
%
\begin{footnotetext}[555]\sphinxAtStartFootnote
\sphinxnolinkurl{http://yun.itheima.com/course/671.html?2010stt}
%
\end{footnotetext}\ignorespaces 

\paragraph{示例}
\label{\detokenize{chapter_knowledge/compete_analysis:id8}}

\subparagraph{美团 2\sphinxfootnotemark[556]}
\label{\detokenize{chapter_knowledge/compete_analysis:id9}}%
\begin{footnotetext}[556]\sphinxAtStartFootnote
\sphinxnolinkurl{https://www.bilibili.com/video/BV1wz4y1y7sg?p=5}
%
\end{footnotetext}\ignorespaces \begin{enumerate}
\sphinxsetlistlabels{\arabic}{enumi}{enumii}{}{.}%
\item {} 
将有限资源投放在二线城市市场,最后杀回一线

\item {} 
制空权:线上时o2o的入口,看出C端用户规模时关键

\item {} 
看重用户体验、留住用户。

\end{enumerate}


\subsubsection{MRD 1\sphinxfootnotemark[557]}
\label{\detokenize{chapter_knowledge/MRD:mrd-1}}\label{\detokenize{chapter_knowledge/MRD::doc}}%
\begin{footnotetext}[557]\sphinxAtStartFootnote
\sphinxnolinkurl{http://www.woshipm.com/pmd/131946.html}
%
\end{footnotetext}\ignorespaces 

\paragraph{定义}
\label{\detokenize{chapter_knowledge/MRD:id1}}
BRD:这值得吗?好处在哪?

MRD的定义:通过BRD明确这件事值得做之后,描述\sphinxstylestrong{应该怎么做,并说明这么做的原因。}如果说BRD是抛出了论题,那么MRD就是要我们用论点来支撑BRD,同时通过论证来得出我们采用什么样的方式获得BRD的商业目标。就是经过一系列分析后,拿出一套我们认为最适合干这个事情的方法以及指导实施的文档。

进入实施阶段,要有\sphinxstylestrong{更细致的市场与竞争对手分析},包括可通过哪些功能来实现商业目的,功能、非功能需求分哪几块,功能优先级等;产出物由Feature
List、业务逻辑图等,是从商业目标到技术实现的关键转化文档;\sphinxhref{https://quizlet.com/129588206/\%E4\%BA\%BA\%E4\%BA\%BA\%E9\%83\%BD\%E6\%98\%AF\%E4\%BA\%A7\%E5\%93\%81\%E7\%BB\%8F\%E7\%90\%86-\%E7\%AC\%94\%E8\%AE\%B0-flash-cards/s}{3}%
\begin{footnote}[558]\sphinxAtStartFootnote
\sphinxnolinkurl{https://quizlet.com/129588206/\%E4\%BA\%BA\%E4\%BA\%BA\%E9\%83\%BD\%E6\%98\%AF\%E4\%BA\%A7\%E5\%93\%81\%E7\%BB\%8F\%E7\%90\%86-\%E7\%AC\%94\%E8\%AE\%B0-flash-cards/s}
%
\end{footnote}


\paragraph{最优解基本条件 2\sphinxfootnotemark[559]}
\label{\detokenize{chapter_knowledge/MRD:id2}}%
\begin{footnotetext}[559]\sphinxAtStartFootnote
\sphinxnolinkurl{https://www.bilibili.com/video/BV1wz4y1y7sg}
%
\end{footnotetext}\ignorespaces \begin{enumerate}
\sphinxsetlistlabels{\arabic}{enumi}{enumii}{}{.}%
\item {} 
最懂用户

\item {} 
比竞争对手做的更好

\item {} 
要让用户知道你

\end{enumerate}

勤奋+努力+人脉资源+。。。


\paragraph{MRD的阅读对象}
\label{\detokenize{chapter_knowledge/MRD:mrd}}
MRD的汇报对象:未来参与产品的各个层级的同事,包括产品经理自己。
\begin{enumerate}
\sphinxsetlistlabels{\arabic}{enumi}{enumii}{}{.}%
\item {} 
MRD是最完善的产品诞生分析描述文档。

\item {} 
以后的一段时间,产品的各个衍生文档、产品依据、团队判断,都要参考MRD。

\item {} 
产品参与成员了解产品的各种背景、数据和方法的依据。

\end{enumerate}


\paragraph{主要内容}
\label{\detokenize{chapter_knowledge/MRD:id3}}

\subparagraph{文档说明}
\label{\detokenize{chapter_knowledge/MRD:id4}}
文档说明:文档基本信息;文档修改记录;文档目的;文档概要。
\begin{itemize}
\item {} 
文档基本信息:公司名称,产品名称,文档创建日期,创建人,创建人联系方式,部门、职位。

\item {} 
文档修改记录:日期,版本、修改人、修改记录、审核人。

\item {} 
文档目的:用于说明产品的市场、用户、产品规划、核心目标、产品路线图、项目规划等。

\item {} 
文档概要:文档说明;市场说明;用户说明;产品说明。

\end{itemize}


\subparagraph{市场说明(industry\_analysis)}
\label{\detokenize{chapter_knowledge/MRD:industry-analysis}}
{\hyperref[\detokenize{chapter_knowledge/industry_analysis:industry-analysis}]{\sphinxcrossref{\DUrole{std,std-ref}{行业分析}}}} (\autopageref*{\detokenize{chapter_knowledge/industry_analysis:industry-analysis}})
\begin{itemize}
\item {} 
市场说明:摘要(可选);

\item {} 
现在市场存在的问题和机会:根据需要撰写–两到三点:产品方面–形态复杂,用户体验差;技术方面–(语音压缩技术不成熟,外资搜索引擎对中文理解不够深刻);运营方面–(产业链篇下游,重实体,轻线上,造成瓜分旅行社利润,形成对立);用户方面(新的需求的出现,需求明显);商业模式方面。

\item {} 
目标市场分析:市场规模;市场特征;发展趋势(未来2\sphinxhyphen{}5年的发展评测);\sphinxstylestrong{时间边界}(这个市场的持续时间预估)。

\item {} 
市场分析结论:一般来说,这里会得到一个比较有市场商业价值的结论。

\end{itemize}


\subparagraph{用户说明(users\_analysis)}
\label{\detokenize{chapter_knowledge/MRD:users-analysis}}
{\hyperref[\detokenize{chapter_knowledge/users_analysis:users-analysis}]{\sphinxcrossref{\DUrole{std,std-ref}{用户需求研究 1}}}} (\autopageref*{\detokenize{chapter_knowledge/users_analysis:users-analysis}})
\begin{itemize}
\item {} 
用户说明:目标用户群体(要求准确:年龄段、收入、地区、学历);

\item {} 
目标群体特征:共性的为主分析;

\item {} 
建立虚拟用户角色:形象化,常用用户特征,用户名称,用户技能、与产品相关的用户特征。

\item {} 
制作用户角色卡片:对用户归类划分,抽取典型角色,能代表目标用户。

\item {} 
用户场景分析:演示性的场景,用户在时间、地点,完成的某个事的故事。

\item {} 
用户动机总结;用户目标总结(明确实质);分析影响用户使用的主要因素。

\item {} 
注意:从技术层面剖析市场,洞察用户心理案例分析。动机和目标是不一致的。

\end{itemize}


\subparagraph{产品说明(goods\_analysis 、compete\_analysis)}
\label{\detokenize{chapter_knowledge/MRD:goods-analysis-compete-analysis}}\begin{itemize}
\item {} 
产品定位:我们用什么样的产品满足用户或用户市场;针对什么用户,做什么事。

\item {} 
产品的核心目标:解决目标市场、用户的核心需求;核心目标的工作级别最高。

\item {} 
产品结构:整体结构,不是功能结构。是产品的核心目标,市场定位,产品定位的直接体现。

\item {} 
产品的路线图;以时间为节点的任务导向。

\item {} 
产品功能性需求:用户注册、留言等等。

\item {} 
非功能性需求:有效性、性能(响应时间 \& 吞吐量 \& 并发用户数 \&
资源利用率)、扩展性、安全性(可得性 \& 私密性)、健壮性(出错率 \&
自我恢复速度)、兼容性、可用性(易用性 \& 一致性 \&
观感需求)、可支持性\sphinxhref{http://www.xmamiga.com/3573/}{4}%
\begin{footnote}[560]\sphinxAtStartFootnote
\sphinxnolinkurl{http://www.xmamiga.com/3573/}
%
\end{footnote}(可扩展性
\& 可维护性 \& 可安装性)、运营需求、用户体验。

\end{itemize}


\paragraph{优秀MBR}
\label{\detokenize{chapter_knowledge/MRD:mbr}}\begin{itemize}
\item {} 
逻辑性强(有论点,论据,论证);

\item {} 
把抽象的东西形象化出来;

\item {} 
数据可靠,分析有理;有把握的主观,无把握的客观;

\item {} 
用词行文,简洁明了;

\item {} 
合理的产品进度分析;

\item {} 
重视非功能需求;

\item {} 
解释专业名词。

\end{itemize}


\subparagraph{可行性分析}
\label{\detokenize{chapter_knowledge/MRD:id5}}
\sphinxurl{https://github.com/twn422/PRD2017/issues/23}


\subparagraph{非功能需求示例 5\sphinxfootnotemark[561]}
\label{\detokenize{chapter_knowledge/MRD:id6}}%
\begin{footnotetext}[561]\sphinxAtStartFootnote
\sphinxnolinkurl{https://t.qidianla.com/1159980.html}
%
\end{footnotetext}\ignorespaces 

\subparagraph{安全需求}
\label{\detokenize{chapter_knowledge/MRD:id7}}\begin{enumerate}
\sphinxsetlistlabels{\arabic}{enumi}{enumii}{}{.}%
\item {} 
密码加密原则:使用128位MD5加密,每个用户拥有独立加密令牌

\item {} 
需登录才可使用的页面开启token验证

\item {} 
所有用户属性信息传输,在接口层进行数据加密

\item {} 
前端代码要求混淆

\item {} 
所有前段页面走https协议

\end{enumerate}


\subparagraph{性能需求}
\label{\detokenize{chapter_knowledge/MRD:id8}}\begin{enumerate}
\sphinxsetlistlabels{\arabic}{enumi}{enumii}{}{.}%
\item {} 
APP从启动到显示首页,不超过5s

\item {} 
单页面加载时间不超过3s

\item {} 
3S无响应需给出提示:网络繁忙,请稍后重试。且可以点击页面重新加载

\item {} 
用懒家在方式,对页面元素预加载处理

\item {} 
已读文章支持缓存,加载完后,断网仍能阅读

\end{enumerate}


\subparagraph{可用性需求}
\label{\detokenize{chapter_knowledge/MRD:id9}}
兼容性上:
\begin{enumerate}
\sphinxsetlistlabels{\arabic}{enumi}{enumii}{}{.}%
\item {} 
PC端网页要求对主流浏览器如IE、Chrome、360、QQ的兼容

\item {} 
H5网页要求对UC、QQ、微信、Safari等浏览器或客户端兼容\sphinxhref{https://www.zhihu.com/column/c\_199236458}{6}%
\begin{footnote}[562]\sphinxAtStartFootnote
\sphinxnolinkurl{https://www.zhihu.com/column/c\_199236458}
%
\end{footnote}

\item {} 
操作系统支持:IOS8.0以上IPHONE5以上机型,Android4.4以上华为、小米、三星、oppo、vivo等主流机型

\item {} 
弱网断网情况下无闪退,可退出当前操作

\item {} 
支持所有二级界面右划返回上一级

\item {} 
保持toast提示样式一致性

\item {} 
保持所有概念文案一致性

\item {} 
需要避免用户重复点击

\item {} 
需要为用户提供反馈入口

\end{enumerate}


\paragraph{市场调研能力}
\label{\detokenize{chapter_knowledge/MRD:id10}}

\subsubsection{PRD}
\label{\detokenize{chapter_knowledge/PRD:prd}}\label{\detokenize{chapter_knowledge/PRD::doc}}
商业需求文档(BRD)和市场需求文档(MRD)主要是前期项目调研的产物,而\sphinxstylestrong{研发和测试主要是以产品需求文档(PRD)为蓝}本来开展工作的


\paragraph{定义 1\sphinxfootnotemark[563]}
\label{\detokenize{chapter_knowledge/PRD:id1}}%
\begin{footnotetext}[563]\sphinxAtStartFootnote
\sphinxnolinkurl{http://www.woshipm.com/pmd/3319375.html}
%
\end{footnotetext}\ignorespaces 
PRD是产品汪思考后的产品方案呈现载体,主要用于需求开发过程中\sphinxstylestrong{沟通研发、测试、设计、运营等,也会用于存档便于查看和回溯。}

目的是让开发人员知道如何按照产品经理的思路为需求撰写代码,让UI
设计人员知道他们需要输出哪些UI
设计稿,让测试人员知道测试用例中需要包含哪些测试点。PRD
的撰写侧重点是需求描述、需求逻辑、需求原型。PRD
是所有产品经理都会接触到的,也是产品经理平时写得最多的文档,所以大家一定要好好研究。

内容主要包含整体说明、用例文档、产品Demo等,对产品功能具体描述;FSD功能详细说明,比较像用例文档,经常包含在PRD中,涉及很多技术内容,产品界面、业务逻辑等。与此同时,硬件系统的设计、数据库设计、表结构设计等工作也开始由架构师、系统分析师编写了。
\sphinxhref{https://quizlet.com/129588206/\%E4\%BA\%BA\%E4\%BA\%BA\%E9\%83\%BD\%E6\%98\%AF\%E4\%BA\%A7\%E5\%93\%81\%E7\%BB\%8F\%E7\%90\%86-\%E7\%AC\%94\%E8\%AE\%B0-flash-cards/}{9}%
\begin{footnote}[564]\sphinxAtStartFootnote
\sphinxnolinkurl{https://quizlet.com/129588206/\%E4\%BA\%BA\%E4\%BA\%BA\%E9\%83\%BD\%E6\%98\%AF\%E4\%BA\%A7\%E5\%93\%81\%E7\%BB\%8F\%E7\%90\%86-\%E7\%AC\%94\%E8\%AE\%B0-flash-cards/}
%
\end{footnote}


\paragraph{需求 到 PRD}
\label{\detokenize{chapter_knowledge/PRD:id2}}
梳理PRD怎么写,无法脱离于项目流程单独思考,需要带入流程。

\begin{figure}[H]
\centering
\capstart

\noindent\sphinxincludegraphics{{need2PRD}.jpg}
\caption{需求 到 PRD
\sphinxhref{https://www.yinxiang.com/everhub/note/435c8b2c-9127-43f3-a6e3-fc5f8898d893}{7}\sphinxfootnotemark[565]}\label{\detokenize{chapter_knowledge/PRD:id31}}\end{figure}
%
\begin{footnotetext}[565]\sphinxAtStartFootnote
\sphinxnolinkurl{https://www.yinxiang.com/everhub/note/435c8b2c-9127-43f3-a6e3-fc5f8898d893}
%
\end{footnotetext}\ignorespaces 
需求拟定:产品原型、交互说明、业务流、数据流、资金流


\subparagraph{战略层——用户需求\&产品目标}
\label{\detokenize{chapter_knowledge/PRD:id3}}
将需求背景、用户需求写在开头,后续撰写都要围绕这个核心,相当于定了方向,也可以指引阅读者更好更快的了解需求。
不管是定量目标还是定性目标,都要落实到PRD上,这是整个项目团队工作的预期成果。

比如36氪的目标用户就是创业者、投资人、互联网职场人和学生,创业者要找36氪报道项目,投资人要通过看36氪找可以投资的好项目,互联网职场人和学生要看36氪开拓视野,了解圈内大事,以补充自己的知识储备量。
另外注意,考虑用户时,如果是多边平台型产品,就不仅要考虑 C 端,还要顾及
B 端用户,比如广告主、机构负责人等等。

给公司带来多大价值:分金钱收益和流量收益,结合时间维度分别说明短期收益和长期收益。比如3个月内聚集用户,半年内展开招商,一年内有持续现金流等等。
注意:在描述收益时,尽量「量化」收益数值,尤其是需要和公司历史收益对比,和同行业其他家收益对比,和行业规模对比,才能做到具备一定说服力。\sphinxhref{https://www.yuque.com/weis/pm/el10i7}{5}%
\begin{footnote}[566]\sphinxAtStartFootnote
\sphinxnolinkurl{https://www.yuque.com/weis/pm/el10i7}
%
\end{footnote}


\subparagraph{范围层——功能规格\&内容需求}
\label{\detokenize{chapter_knowledge/PRD:id4}}
战略层中明确了用户需求和产品目标后,范围层就要确定做哪些功能、提供什么内容来实现产品目标,可以将各个功能点以及它们的优先级用脑图或excel画出来(工具随意),相当于把整个需求分拆成各个模块,便于理解和评估工作。


\subparagraph{结构层——交互设计\&信息架构}
\label{\detokenize{chapter_knowledge/PRD:id5}}
范围层把需求分拆之后,结构层再把各个部分用流程图或思维脑图连接起来,让阅读者能抽象出产品的使用流程或信息架构。

偏交易、工具的产品需求可以使用流程图,偏内容的产品需求可以使用思维脑图,此处放一个点餐交易的局部流程图,仅做示例参考。


\subparagraph{框架层——界面设计\&导航设计}
\label{\detokenize{chapter_knowledge/PRD:id6}}
结构层将需求分拆模块,框架层梳理整体流程,相信这时候PRD读者已经对需求有了清晰的思路,框架层主要对每个模块或页面的逻辑、布局做详细阐述。


\subparagraph{表现层——视觉设计}
\label{\detokenize{chapter_knowledge/PRD:id7}}
表现层主要包括但不限于视觉、听觉、触觉等内容,需要做一些高保真的设计,建议与UED、用户体验部门联合产出,这里不做过多展开。


\paragraph{过程}
\label{\detokenize{chapter_knowledge/PRD:id8}}
在写需求文档前,面对我们的用户——相关协同人员,产品经理需要去了解他们。了解他们的工作方式、工作习惯、工作态度、工作认知、工作能力等与工作相关的内容,同时,对他们与人相处的方式、生活习惯、兴趣爱好等等的了解,有助于产品经理更全面的了解,从而建立更加立体的用户画像。

在输出判断结果时会更准确,写需求文档会更有侧重点——哪些是他们需要知道的,哪些是他们需要特别详细表述的,哪些是需要特殊标注的,哪些是省略表述即可的。{[}11{]}
保证无遗漏、无歧义。把层次关系梳理出现,再用图表或流程图表现。拆分组块业务逻辑,梳理业务上下游。实际工作中,很可能你写的东西,开发压根懒得看,上司也压根不看,但是一定要有,等和开发撕逼的时候,拿出来,反正你写了,他不看,是他的责任!{[}14{]}


\paragraph{作用}
\label{\detokenize{chapter_knowledge/PRD:id9}}
\begin{figure}[H]
\centering
\capstart

\noindent\sphinxincludegraphics{{PM_interact}.png}
\caption{协同人员与其需求{[}40{]}}\label{\detokenize{chapter_knowledge/PRD:id32}}\end{figure}
\begin{itemize}
\item {} 
对于开发而言:开发可以通过需求文档知道产品的各个功能点,以及设计到的交互逻辑。

\item {} 
对于测试而言:测试通过需求文档来编写测试用例。

\item {} 
对于设计师而言:可以通过需求文档来确定交互细节。

\item {} 
对于运营而言:可以通过需求文档了解我们的战略及规范,了解到产品的亮点,从而制定相应的运营策略。

\item {} 
对于项目经理而言:可以根据需求文档来拆分相应的工作包。

\item {} 
对于产品经理自身而言:人的大脑容量是有限的,我们不能保证在产品迭代的后期还能清楚的知道每一个功能点的细节,所以需求文档就是一个很好的“复习”工具。特别是版本需求讨论、需求评审会、内容交叉、Bug处理、版本迭代时。

\item {} 
另外,当出现人员异动(人员项目变更、人员离职等)时,新接手的产品经理,可以通过需求文档来快速的全方面的了解产品。{[}15{]}

\end{itemize}


\paragraph{表现形式}
\label{\detokenize{chapter_knowledge/PRD:id10}}
产品需求文档的表现形式有很多种,常见的有
Word、图片和交互原型这三种形式,文档内容通常包含信息结构图、界面线框图、功能流程图、功能说明文档。虽然产品需求文档没有标准的规范,但是有两项是必不可少的,那就是文件标识和修改记录。文档在撰写过程中,我们可以自行不断的修改完善,但是如果正式发布或交给团队其他成员后,一旦有了修改,为了文档的同步,我们就需要标注出文档的修改内容,备注修改记录,这样可以方便大家查看和了解改动的内容。关于文件标识和修改记录,格式都大同小异。


\paragraph{目录: 2\sphinxfootnotemark[567]}
\label{\detokenize{chapter_knowledge/PRD:id11}}%
\begin{footnotetext}[567]\sphinxAtStartFootnote
\sphinxnolinkurl{http://www.woshipm.com/pmd/3516749.html}
%
\end{footnotetext}\ignorespaces \begin{enumerate}
\sphinxsetlistlabels{\arabic}{enumi}{enumii}{}{.}%
\item {} 
产品背景

\item {} 
版本历史

\item {} 
名词解释

\item {} 
产品综述

\item {} 
用户故事

\item {} 
需求详述

\item {} 
评审记录

\item {} 
其他问题描述

\end{enumerate}


\subparagraph{产品背景}
\label{\detokenize{chapter_knowledge/PRD:id12}}

\subparagraph{背景概述}
\label{\detokenize{chapter_knowledge/PRD:id13}}
含义解释:背景概述是用简单的语言大概概括一下大的背景,让人知道我们本次要讲的内容大概是什么。

描述正文:官网为用户提供产品试用,目前,完整的试用流程如下:

用户在官网进行注册,填写申请试用表单。商务(运营)在管理后台,对用户的申请进行授权操作(允许/拒绝)。

注释:这样的背景描述,是将云官网,本次的产品需求,用业务流程串联起来,从前端到后端。从业务流程出发,将业务串联起来,这是一种非常好的方式。用一个事件,将涉及的所有产品功能都串联起来,让本次讨论有主线。

一般来说,这段话的作用在于让人阅读后明白我们为什么要花时间做这件事,以及明白了这件事的意义所在。重点在WHY,关于WHY的重要性,大家可以看一个演讲叫做How
great leaders inspire action。

既然是完善和优化,那么产品经理就需要向运营、向研发证明,为什么这样的修改,相较于原来的流程,更好。

Example: \sphinxurl{http://www.woshipm.com/pd/4167090.html}

TODO:芒果金融APP PRD:\sphinxurl{https://t.qidianla.com/1156795.html}


\subparagraph{文档介绍 6\sphinxfootnotemark[568]}
\label{\detokenize{chapter_knowledge/PRD:id14}}%
\begin{footnotetext}[568]\sphinxAtStartFootnote
\sphinxnolinkurl{http://www.woshipm.com/pmd/707412.html}
%
\end{footnotetext}\ignorespaces 
版本控制是管理需求的一个重要方面,而作为每一个公布的需求文档的版本有着两个重要的部分:一是关于版本状态;二是修订历史。

为了尽量减少困惑、冲突、误传,文档的版本必须被统一确定,且在版本介绍中具备版本号、作者、日期、版本状态,在修订历史中记录文档版本、修订时间、变更人姓名、修订属性以及版本内容描述。


\subparagraph{开文介绍}
\label{\detokenize{chapter_knowledge/PRD:id15}}
开文介绍中主要是说明该文档的V1.0版本是关于哪一个产品、IOS还是Android(可以移动端用一个文档说明)、当前版本、V1.0版本作者、完成日期以及文档状态。


\subparagraph{版本历史}
\label{\detokenize{chapter_knowledge/PRD:id16}}
我这里面以QQ举例,例如QQ第一版本主要支持点对点的文字信息通讯功能。{[}12{]}
\begin{itemize}
\item {} 
这个时候我们叫做V1.0.0;

\item {} 
​第二个版本你又增加文件传输功能,这个时候叫做V1.1.0;

\item {} 
​有一天你发现一个bug,需要紧急上线一个版本,这个时候命名为V1.1.1

\end{itemize}


\subparagraph{名词解释}
\label{\detokenize{chapter_knowledge/PRD:id17}}
专业词汇


\subparagraph{其他问题描述}
\label{\detokenize{chapter_knowledge/PRD:id18}}
TODO: 4.5 法务说明

凡涉及隐私权、知识产权、专利权、商标、服务条款(TOS)、版权、合同责任、客户沟通等方面需要

标明来源。(法务应提供协助)

4.6 性能需求

技术提供

4.7 安全需求

技术提供

4.8 兼容需求

技术提供

4.9 时间显示说明

举例:

六 附录

6.1 名词解释

6.2 原型

6.3 其它


\subparagraph{兼容性需求 6\sphinxfootnotemark[569]}
\label{\detokenize{chapter_knowledge/PRD:id19}}%
\begin{footnotetext}[569]\sphinxAtStartFootnote
\sphinxnolinkurl{http://www.woshipm.com/pmd/707412.html}
%
\end{footnotetext}\ignorespaces 
云之家初始发版会在安卓、IOS、WEB同步发布,兼容系统版本与浏览器版本如下:


\subparagraph{修订历史}
\label{\detokenize{chapter_knowledge/PRD:id20}}
修订历史是一个版本的可追溯源,因为一般来说,各个公司都是处于铁打的产品需求文档流水的产品经理的状态,利用它,就可以对产品的整个发展历程有一个清晰的认识,便于把握产品发展路线。

新建默认为相应模块的首次使用,对于文档的修改以及增加的地方可加入超链接,同时在增加与修改的具体地方进行颜色标示或者其他标志来进行区分,方便其他人员进行查询。


\paragraph{AI产品}
\label{\detokenize{chapter_knowledge/PRD:ai}}
二者的差异主要集中在需求内容的部分。而需求内容这方面的差异又因PM角色的不同而有所不同。

\sphinxstylestrong{机器学习是满足用户需求的一种方式,而不是需求本身。}


\subparagraph{数据需求 8\sphinxfootnotemark[570]}
\label{\detokenize{chapter_knowledge/PRD:id21}}%
\begin{footnotetext}[570]\sphinxAtStartFootnote
\sphinxnolinkurl{https://www.sohu.com/a/258016767\_610300}
%
\end{footnotetext}\ignorespaces 
数据是机器学习的输入和输出。


\subparagraph{数据要求}
\label{\detokenize{chapter_knowledge/PRD:id22}}\begin{itemize}
\item {} 
需要什么数据? 答:找出产品或特定功能所需要的数据

\item {} 
哪些特征是已知的,将是有用的?等等。 答:已知属性称为特征

\item {} 
这些特征可用吗?如果不可用,获取成本是多少?
答:要预测的值称为标签。按可用性、有无困难和成本来排定优先级。

\end{itemize}


\subparagraph{数据采集策略}
\label{\detokenize{chapter_knowledge/PRD:id23}}\begin{itemize}
\item {} 
上述数据来自哪里?答:想到所有对这个任务有帮助,而且你也能拿得到的数据。如,可从公共记录中获得房子的年龄、用公共地图数据计算离最近的杂货店的距离。

\item {} 
现有数据是否存在质量问题?答:注意因为不同格式的特征值或语义,如,单位。

\item {} 
你认为需要多少数据?答:尽量多、迁移、预算、限定。

\end{itemize}


\subparagraph{隐私与安全}
\label{\detokenize{chapter_knowledge/PRD:id24}}\begin{itemize}
\item {} 
数据存储和处理的方式是否安全? 答:最好咨询专家。

\item {} 
你有收集/使用数据的权限吗? 答:考虑什么是应做的,什么是不该做的。

\item {} 
从用户的角度来看,新功能或产品的好处是否能超过他们在提供数据时的担忧?
答:用户得到什么好处。

\end{itemize}


\paragraph{测试完整性3\sphinxfootnotemark[571]}
\label{\detokenize{chapter_knowledge/PRD:id25}}%
\begin{footnotetext}[571]\sphinxAtStartFootnote
\sphinxnolinkurl{http://www.woshipm.com/pmd/21446.html}
%
\end{footnotetext}\ignorespaces 
现在你有一个PRD草稿,你需要测试它的完整性。工程师是否可以充分了解并达到目标?OA
Team(质量管理团队)是否有足够的信息来做出测试计划,是否可以开始做案例?

当投资人或相关人审核了PRD,确定了各个需要说明的方面,所有的问题得到解决,现在你就可以按PRD进行产品开发。


\paragraph{管理产品 3\sphinxfootnotemark[572]}
\label{\detokenize{chapter_knowledge/PRD:id26}}%
\begin{footnotetext}[572]\sphinxAtStartFootnote
\sphinxnolinkurl{http://www.woshipm.com/pmd/21446.html}
%
\end{footnotetext}\ignorespaces 
解决所有PRD中存在问题,如果不在PRD中就写进去。你的任务就是迅速解决问题并记录在PRD。

如果你做了你的工作并准备记录在PRD,项目审查就会变得非常简单,因为任何一个部份都历历在目。

记住PRD是一个“活”的文件,在要跟踪记录在产品开发期间的所有功能过程。最后你会发现很多额外的东西,如果你认为是必要的就在PRD中写进。

confluence:这个不多说了,非常好用的一款产品PRD在线编辑软件。\sphinxhref{http://www.woshipm.com/pmd/913343.html}{4}%
\begin{footnote}[573]\sphinxAtStartFootnote
\sphinxnolinkurl{http://www.woshipm.com/pmd/913343.html}
%
\end{footnote}


\paragraph{COD评分表方法}
\label{\detokenize{chapter_knowledge/PRD:cod}}

\paragraph{信息架构}
\label{\detokenize{chapter_knowledge/PRD:id27}}
组织系统、标签系统、导航系统、搜索系统


\paragraph{好的产品需求文档 6\sphinxfootnotemark[574]}
\label{\detokenize{chapter_knowledge/PRD:id28}}%
\begin{footnotetext}[574]\sphinxAtStartFootnote
\sphinxnolinkurl{http://www.woshipm.com/pmd/707412.html}
%
\end{footnotetext}\ignorespaces \begin{enumerate}
\sphinxsetlistlabels{\arabic}{enumi}{enumii}{}{.}%
\item {} 
页面(数字标示需要进行说明的按钮、图片、热区、说明文案等)

\item {} 
需求场景(简单的人物故事说明)

\item {} 
需求对应功能

\item {} 
业务说明(业务的流程)

\item {} 
页面说明

\item {} 
前置后置流程

\item {} 
补充说明(有其他需要说明的进行补充)

\end{enumerate}


\paragraph{完善评审流程和项目管理 10\sphinxfootnotemark[575]}
\label{\detokenize{chapter_knowledge/PRD:id29}}%
\begin{footnotetext}[575]\sphinxAtStartFootnote
\sphinxnolinkurl{https://www.zhihu.com/people/linelian/posts?page=1}
%
\end{footnotetext}\ignorespaces 
产品评审:对PRD进行阅读,然后评论,关键是提问。

整体流程如下:
\begin{enumerate}
\sphinxsetlistlabels{\arabic}{enumi}{enumii}{}{.}%
\item {} 
创建日程,将参加评审同学添加至日程,并编辑会议摘要,说明评审内容

\item {} 
日程开始后,通过日程快速创建会议群,群内发出评审PRD,大家在各自电脑打开阅读文档编辑评论

\item {} 
一次PRD评审控制在45分钟,PRD作者组织评审,一般会15分钟阅读文档,过程中PRD作者通过文字回答评论提问,阅读完成后文档底部点赞代表阅读完成,多数人点赞后开始对评论答疑讨论,并记录todo

\item {} 
拉齐需求,对战略目标,对同组同仁

\end{enumerate}


\paragraph{一体化产品需求文档}
\label{\detokenize{chapter_knowledge/PRD:id30}}
{[}17{]}

{[}11{]}: {[}12{]}: \sphinxurl{https://www.pianshen.com/article/89602055805/} {[}13{]}:
\sphinxurl{http://www.woshipm.com/pmd/2860477.html} {[}14{]}:
\sphinxurl{https://www.zhihu.com/question/19654911/answer/624768169} {[}15{]}:
\sphinxurl{https://www.zhihu.com/question/19654911/answer/1760290293} {[}16{]}:
\sphinxurl{http://www.woshipm.com/pmd/2860477.html} {[}17{]}:
\sphinxurl{http://www.woshipm.com/rp/379949.html}


\subsubsection{工具}
\label{\detokenize{chapter_knowledge/tools:id1}}\label{\detokenize{chapter_knowledge/tools::doc}}
工具只是工具,没有绝对的标准、只要适合就好

\sphinxurl{https://tangjie.me/blog/184.html}

\begin{figure}[H]
\centering
\capstart

\noindent\sphinxincludegraphics{{skill_need_tools}.png}
\caption{工作所需的工具}\label{\detokenize{chapter_knowledge/tools:id4}}\end{figure}

TODO: 产品经理必用的软件大全—产品手记推荐工具系列 \sphinxhyphen{} 「已注销」的文章 \sphinxhyphen{}
知乎 \sphinxurl{https://zhuanlan.zhihu.com/p/109261065}


\paragraph{Project}
\label{\detokenize{chapter_knowledge/tools:project}}
Microsoft Project

适用平台:Windows\&Mac

软件特点:
\begin{itemize}
\item {} 
全能型项目管理工具;

\item {} 
Microsoft家族软件成员;

\item {} 
无需联网即可使用;

\item {} 
有资源管理和人员管理模块;

\item {} 
有甘特图、燃尽图等模板;

\end{itemize}

缺点:
\begin{itemize}
\item {} 
收费,有永久激活版;

\item {} 
无法多人协作;

\item {} 
共享和演示功能弱;

\end{itemize}

甘特图:\sphinxurl{https://support.microsoft.com/zh-cn/office/\%E4\%BD\%BF\%E7\%94\%A8\%E7\%94\%98\%E7\%89\%B9\%E5\%9B\%BE\%E8\%A7\%86\%E5\%9B\%BE-0e84efa4-78ce-4cd1-baed-5159a55f78b4}
\begin{itemize}
\item {} 
\sphinxurl{https://www.iamxiarui.com/?paged=3\&cat=111}

\item {} 
\sphinxurl{https://blog.csdn.net/liwei16611/article/details}/100106262\#\# UML

\end{itemize}

\sphinxurl{https://www.iamxiarui.com/?p=1231}

\begin{figure}[H]
\centering
\capstart

\noindent\sphinxincludegraphics{{github_projects}.png}
\caption{Github project}\label{\detokenize{chapter_knowledge/tools:id5}}\end{figure}


\paragraph{Axure 1\sphinxfootnotemark[576]}
\label{\detokenize{chapter_knowledge/tools:axure-1}}%
\begin{footnotetext}[576]\sphinxAtStartFootnote
\sphinxnolinkurl{https://www.yinxiang.com/everhub/note/435c8b2c-9127-43f3-a6e3-fc5f8898d893}
%
\end{footnotetext}\ignorespaces 
Axure 9.0不用?\sphinxhref{https://www.bilibili.com/video/BV1it41137Xg?p=2}{4}%
\begin{footnote}[577]\sphinxAtStartFootnote
\sphinxnolinkurl{https://www.bilibili.com/video/BV1it41137Xg?p=2}
%
\end{footnote}
口碑差,实际上都用8.0、8.1

\sphinxstylestrong{Axure实战}见
\sphinxurl{https://github.com/StevenJokess/2bPM/blob/master/src/meituan\_PM\_recovered.rp}
\begin{itemize}
\item {} 
流程图实战:ref:\sphinxcode{\sphinxupquote{flow\_chart\_Axure}}

\item {} 
流程图页面化实战:ref:\sphinxcode{\sphinxupquote{flow\_chart2page\_Axure}}

\item {} 
低保真页面实战:ref:\sphinxcode{\sphinxupquote{page\_Axure}}

\item {} 
高保真页面实战:ref:\sphinxcode{\sphinxupquote{page\_done\_Axure}}

\end{itemize}

更多:
\begin{itemize}
\item {} 
\sphinxurl{https://www.axure.com.cn/axure/}

\item {} 
AxureShop:\sphinxurl{https://www.axureshop.com/}

\item {} 
\sphinxurl{https://axhub.im/}

\item {} 
AXURE8
原型绘制效率工具:\sphinxurl{https://coffee.pmcaff.com/article/2111722865562752/pmcaff?utm\_source=forum\&newwindow=1}

\item {} 
AXURE8
教学(从8到9,布局快捷键大变,更不方便,一步交互变四步,多目标藏+好里,很多bug):\sphinxurl{https://www.bilibili.com/video/BV1bt41137MY}

\item {} 
AXURE8:\sphinxurl{http://www.woshipm.com/xiazai/188927.html}

\item {} 
Axure(8+9)产品经理:\sphinxurl{https://edu.csdn.net/course/play/28296/389326}

\item {} 
Axhub 即Axure原型发布与协作平台:\sphinxurl{https://axhub.im/}

\item {} 
\sphinxurl{https://chanpinyx.com/}

\item {} 
\sphinxurl{http://www.iaxure.com/}

\end{itemize}


\subparagraph{简介}
\label{\detokenize{chapter_knowledge/tools:id2}}
\sphinxurl{https://www.iamxiarui.com/?p=1845}

在我的实践中,用Axure原型+注释的方式撰写,导出HTML或发布到Axshare或通过蓝湖上传的方式传达;效率最高效果最好。

Axure之于产品经理就像PhotoShop之于设计师

经典的虽然看起来没那么酷炫,但是它代表着易用性、拓展性、普适性
\begin{itemize}
\item {} 
易用性:可以找到大量教程,降低门槛快速上手

\item {} 
拓展性:可以获取大量素材,积累提升输出质量

\item {} 
普适性:可以兼容各种系统,提升工作协同效率

\item {} 
有人用墨刀做花里胡哨的交互,文字描述寥寥,结果连异常流程都覆盖不了

\item {} 
有人用word写长篇大论,结果开发根本不看,还要问你每个页面怎么跳转

\item {} 
有人用sketch,不配合专属插件的话,很容易画成没有结构性的页面流程图。而且是否具备普适性,取决于周围与你配合的人

\end{itemize}

工具本身没有问题,有问题一定是工具人的问题。

为了让你的PRD的用户使用效果更好,汲取意见是很重要的。有的时候不要埋怨开发为什么不认真看,去问问他们喜欢看什么样的,适当调整达成统一。

Tower、禅道、tapd等团队协同工具,会导致一部分人习惯把产品的结构通过协同工具中的单个需求建立父子关系连接,在每个子需求中上传需求图片+文字描述。这种方式更适合优化已有功能,功能可以在小范围内实现闭环,但是容易使人忽略细节的调整对全局的影响。

产品经理应该把控Axure源文件的颗粒度,哪些版本/模块在一个源文件里更新,从哪个版本/模块新建文档维护;文档或文件夹之间尽量去重、解耦。在原有文件上新增或调整的部分,更显眼的做出标注,让人可以一眼看出变动的部分。

在Axure里搭建当前版本/模块最完整的产品结构,并且及时更新。要明确一个认知,协同工具是给UI、开发、测试、运营看的,目的是让他们看起来方便,可以针对单个需求设置执行人员、排期、添加bug记录等,核心价值是项目管理。而不是为了产品经理维护需求文档方便,所以在需求的撰写和管理时,不要依赖协同工具。在追溯需求问题的时候,要能做到在Axure文件里便能找到记录,而不是去翻协同工具里的需求记录。

Axure基础操作总结:\sphinxurl{https://blog.csdn.net/zcl050505/article/details/110439551}


\paragraph{Visio}
\label{\detokenize{chapter_knowledge/tools:visio}}
画泳道图\sphinxhref{https://www.zhihu.com/column/c\_199236458}{6}%
\begin{footnote}[578]\sphinxAtStartFootnote
\sphinxnolinkurl{https://www.zhihu.com/column/c\_199236458}
%
\end{footnote}


\paragraph{Sketch 5\sphinxfootnotemark[579]}
\label{\detokenize{chapter_knowledge/tools:sketch-5}}%
\begin{footnotetext}[579]\sphinxAtStartFootnote
\sphinxnolinkurl{https://blog.csdn.net/Dylan\_zhijing/article/details/107825514?spm=1001.2014.3001.5502}
%
\end{footnotetext}\ignorespaces 
Sketch有利于产品与UI的协同工作。对于较小的需求,在白板上画出原型后可以直接进入UI阶段。

优点:
\begin{itemize}
\item {} 
方便画出高保真的原型图;

\item {} 
在静态界面的展示方面有一定优势;

\item {} 
无限画布,可以将各页面放到一个画布上帮助串联思考。

\end{itemize}

不足:
\begin{itemize}
\item {} 
对于初学者,基础功能的学习成本较高;

\item {} 
Sketch作为一款设计工具,会让使用者陷入“设计者思维”,在细节上耗费较多时间,拖延项目进度。

\end{itemize}


\paragraph{TAPD}
\label{\detokenize{chapter_knowledge/tools:tapd}}
TAPD开放产品,包含两个解决方案\sphinxhyphen{}轻量协作和敏捷研发。轻量协作:对标Tower,功能特点为看板+云文档+企业微信集成,为满足小团队的协作需要。敏捷研发功能特点包含需求、迭代、缺陷、测试计划/用例、发布评审、看板等。总体来看敏捷研发特性(能力)更强。整体UI特点:偏传统,多页应用。

\sphinxurl{https://www.zhihu.com/question/56575428/answer/352398510}


\paragraph{Markdown}
\label{\detokenize{chapter_knowledge/tools:markdown}}\begin{itemize}
\item {} 
Markdown保姆级教程之画图篇(流程图、序列图、饼图、甘特图):
\sphinxurl{https://www.bilibili.com/video/av88551739/}

\item {} 
官方帮助文档:\sphinxurl{https://mermaid-js.github.io/mermaid/\#/flowchart}

\item {} 
VScode插件:Markdown Preview Mermaid Support

\item {} 
更多:\sphinxurl{https://www.iminho.me/wiki/docs/mindoc/mermaid.md}

\end{itemize}


\paragraph{眼界}
\label{\detokenize{chapter_knowledge/tools:id3}}\begin{itemize}
\item {} 
快懂百科:\sphinxurl{https://www.baike.com/}

\item {} 
中文互联网数据资讯网:\sphinxurl{http://www.199it.com/}

\item {} 
FT中文网:\sphinxurl{https://www.ftchinese.com/}

\item {} 
品玩:\sphinxurl{https://www.pingwest.com/}

\item {} 
最新高质量的专业资讯、数据:\sphinxurl{https://www.zhihu.com/question/367641344}

\end{itemize}


\subsubsection{流程图 1\sphinxfootnotemark[580]}
\label{\detokenize{chapter_knowledge/flow_chart:id1}}\label{\detokenize{chapter_knowledge/flow_chart::doc}}%
\begin{footnotetext}[580]\sphinxAtStartFootnote
\sphinxnolinkurl{http://www.woshipm.com/pd/818876.html}
%
\end{footnotetext}\ignorespaces 

\paragraph{编写有效用例 6\sphinxfootnotemark[581]}
\label{\detokenize{chapter_knowledge/flow_chart:id2}}%
\begin{footnotetext}[581]\sphinxAtStartFootnote
\sphinxnolinkurl{https://www.yinxiang.com/everhub/note/f9ab87ee-73e6-4241-9428-9507cbfd007f}
%
\end{footnotetext}\ignorespaces \begin{itemize}
\item {} 
低精度:页面流程图(页面名称、页面内容、判断、触发动作和跳转)

\item {} 
中精度:静态页面(导航信息、组件元素、页面布局\sphinxhyphen{}栅格系统、文案信息、重要备注)

\item {} 
高精度:动态高保真原型

\end{itemize}


\paragraph{画流程图}
\label{\detokenize{chapter_knowledge/flow_chart:id3}}
画原型前应该先画流程图,所以画流程应该是最基本最必要的技能。但往往大多数新产品人却对流程的概念异常模糊。

下面将讲解定义、分类以及画法,及各种流程图的特点。


\paragraph{流程图}
\label{\detokenize{chapter_knowledge/flow_chart:id4}}
流程——顾名思义:水流的路程;事物进行中的次序或顺序的布置和安排。流程是自然而然就存在的,它可以不规范,可以不固定,可以充满问题。

图——Chart 或者 Diagram,
是将基本固化有一定规律的流程进行显性化和书面化,从而有利于传播与沉淀、流程重组参考。

由两个及以上的步骤,完成一个完整的行为的过程,可称之为流程;注意是两个及以上的步骤。

流程图的核心就在于如何排布事物进行的次序,不同的顺序可能造成截然不同的结果。

流程不可或缺的因素:对象、输入、动作、输出。
\begin{itemize}
\item {} 
对象就是执行人,也就是产品中的用户;

\item {} 
输入可以理解为前提、前置条件;

\item {} 
动作,就是产品中的操作,可以是点击、输入,等等;

\item {} 
输出,可以理解为结果、动作的目的。

\end{itemize}


\paragraph{目的}
\label{\detokenize{chapter_knowledge/flow_chart:id5}}\begin{enumerate}
\sphinxsetlistlabels{\arabic}{enumi}{enumii}{}{.}%
\item {} 
流程图为产品设计基石,可以保证产品的使用逻辑合理顺畅;为画原型做向导,也能反向检验流程图,发现完整流程中必不可少的原型图。

\item {} 
展现了基于用户选择的状态、页面视图和内容,更好地传达需求,用流程图来更好地表达产品逻辑。纯粹的原型图会让沟通对象陷入页面本身的焦点中,而流程图则可以告诉沟通对象要经历哪些过程、最终要实现怎样的目标。

\item {} 
查漏补缺,检验是否有遗漏的分支流程。

\end{enumerate}


\paragraph{常见问题 7\sphinxfootnotemark[582]}
\label{\detokenize{chapter_knowledge/flow_chart:id6}}%
\begin{footnotetext}[582]\sphinxAtStartFootnote
\sphinxnolinkurl{https://www.zhihu.com/pub/reader/119980992/chapter/1284104609896828928}
%
\end{footnotetext}\ignorespaces \begin{itemize}
\item {} 
没有任何前提条件:做饭的前提之一是有电或燃气

\item {} 
没异常:手机忽然没电关机了OR确认支付时系统卡死了

\item {} 
没角色:电商中买家、卖家、服务商各自的流程

\item {} 
没背景:说出,现阶段优化的目的、不足、方案、能否解决、能否预防

\end{itemize}


\paragraph{组件}
\label{\detokenize{chapter_knowledge/flow_chart:id7}}
\begin{figure}[H]
\centering
\capstart

\noindent\sphinxincludegraphics{{flow_chart_part}.png}
\caption{组件}\label{\detokenize{chapter_knowledge/flow_chart:id37}}\end{figure}


\paragraph{分类}
\label{\detokenize{chapter_knowledge/flow_chart:id8}}
流程图以描述对象分类,包括:业务流程图、页面流程图、功能流程图、数据流程图等。作为产品,经常谈的是业务流程图;作为交互设计师,则比较关心页面流程图;而作为系统分析师,数据流程图最关键。
\sphinxhref{http://www.woshipm.com/pd/675174.html}{4}%
\begin{footnote}[583]\sphinxAtStartFootnote
\sphinxnolinkurl{http://www.woshipm.com/pd/675174.html}
%
\end{footnote}

流程图其实是传统的管理业务流程图,包含基本流程图和\sphinxstylestrong{跨职能}流程图(泳道图)两种。


\subparagraph{业务流程图(Transaction Flow Diagram, TFD)}
\label{\detokenize{chapter_knowledge/flow_chart:transaction-flow-diagram-tfd}}
“大象塞进冰箱”–分解

业务流程图(TFD)是一种抽象地描述管理系统内各单位、人员(系统组织结构)之间的业务关系、作业顺序(业务流程)和管理信息流向,而不涉及具体操作与执行细节的图表。

以医院挂号流程为例:

\begin{figure}[H]
\centering
\capstart

\noindent\sphinxincludegraphics{{hospital_flow}.png}
\caption{医院挂号流程}\label{\detokenize{chapter_knowledge/flow_chart:id38}}\end{figure}

\begin{figure}[H]
\centering
\capstart

\noindent\sphinxincludegraphics{{service_flow_chart}.jpg}
\caption{电商购物的业务流程图}\label{\detokenize{chapter_knowledge/flow_chart:id39}}\label{\detokenize{chapter_knowledge/flow_chart:flow-chart-axure}}\end{figure}


\subparagraph{Axure实战}
\label{\detokenize{chapter_knowledge/flow_chart:axure}}
\begin{figure}[H]
\centering
\capstart

\noindent\sphinxincludegraphics{{flow_chart_axure}.png}
\caption{TFD
Axure\sphinxhref{https://www.bilibili.com/video/BV1WE411w7LW?from=search\&seid=9895003283584993406}{8}\sphinxfootnotemark[584]}\label{\detokenize{chapter_knowledge/flow_chart:id40}}\end{figure}
%
\begin{footnotetext}[584]\sphinxAtStartFootnote
\sphinxnolinkurl{https://www.bilibili.com/video/BV1WE411w7LW?from=search\&seid=9895003283584993406}
%
\end{footnotetext}\ignorespaces 
\begin{figure}[H]
\centering
\capstart

\noindent\sphinxincludegraphics{{meituan_login_flow}.png}
\caption{美团外卖App登录}\label{\detokenize{chapter_knowledge/flow_chart:id41}}\end{figure}
\begin{itemize}
\item {} 
\sphinxhref{http://www.woshipm.com/pd/1229962.html}{实例分析拆解:如何设计登录注册?}%
\begin{footnote}[585]\sphinxAtStartFootnote
\sphinxnolinkurl{http://www.woshipm.com/pd/1229962.html}
%
\end{footnote}

\item {} 
\sphinxhref{http://www.woshipm.com/pmd/358037.html}{真实案例分享|登录注册产品需求文档}%
\begin{footnote}[586]\sphinxAtStartFootnote
\sphinxnolinkurl{http://www.woshipm.com/pmd/358037.html}
%
\end{footnote}

\end{itemize}


\subparagraph{绘制思路一般是:}
\label{\detokenize{chapter_knowledge/flow_chart:id9}}\begin{itemize}
\item {} 
首先将业务按阶段划分,比如电商类可以分为下单和支付,单车类可以分为提车、骑行和停车;

\item {} 
然后列出每个阶段参与的功能模块,比如下单阶段,就有商品查看、登录/注册、信息记录、个人中心等功能。

\item {} 
最后按照时间顺序,画出业务需求在各个功能模块之间的流转情况。

\end{itemize}

有两个原则:
\begin{itemize}
\item {} 
先思考主干流程,再思考分支流程,主干流程逻辑准确,分支流程全面无遗漏;

\item {} 
表达清楚后台产生的各种判断及相应的前端展示,这将作为接口设计的重要根据。

\end{itemize}

心得:
\begin{itemize}
\item {} 
先画流程图,再画页面原型不要想一步到位,完善需要过程

\item {} 
以能解决问题为目的,不要过分追求细致

\end{itemize}


\subparagraph{页面流程图(Page Flow Diagram)}
\label{\detokenize{chapter_knowledge/flow_chart:page-flow-diagram}}\label{\detokenize{chapter_knowledge/flow_chart:axure-1}}
问题:
\begin{enumerate}
\sphinxsetlistlabels{\arabic}{enumi}{enumii}{}{.}%
\item {} 
这么多流程总不可能要在同一个页面吧,那需要几个页面呢?

\item {} 
页面里又应该有什么流程和功能呢?

\end{enumerate}
\begin{itemize}
\item {} 
业务流程图重要的是描述谁在什么条件下做了什么事。

\item {} 
而页面流程图是具体到了网站、系统、产品功能设计的时候,表现页面之前的流转关系——用户通过什么操作进了\sphinxstylestrong{什么页面}及后续的操作及页面。

\end{itemize}

定义:指电子产品具体所呈现的页面跳转流程图。其承载了业务流程图所包含的业务流转信息。

页面流程图依然是包含在业务流程图的。这恰恰符合定义中的要求,同时也印证了页面流程图的正确性。

我们将抽象的业务,映射在了具象的页面上,用软件的页面承载起了业务需求。而以上就是由业务流程图到页面流程图的转化过程。

京东购物车流程:\sphinxurl{https://www.processon.com/view/5715d26ce4b0d89bd25a2998}


\subparagraph{Axure实战}
\label{\detokenize{chapter_knowledge/flow_chart:id10}}
\begin{figure}[H]
\centering
\capstart

\noindent\sphinxincludegraphics{{flowchart2page_axure}.png}
\caption{TFD转化PFD Axure实战}\label{\detokenize{chapter_knowledge/flow_chart:id42}}\label{\detokenize{chapter_knowledge/flow_chart:flow-chart2page-axure}}\end{figure}


\subparagraph{好处}
\label{\detokenize{chapter_knowledge/flow_chart:id11}}\begin{itemize}
\item {} 
对于设计师或产品经理的好处:

\end{itemize}
\begin{enumerate}
\sphinxsetlistlabels{\arabic}{enumi}{enumii}{}{.}%
\item {} 
页面流程图一张页面助你讲完完整的用户与系统的交互故事,借助它,你更容易知道流程中的潜在地雷是什么,哪里的效率比较低,有助于系统化、全局化、周全性的思考

\item {} 
细化工作量的基础,通过页面流程图可准确评估需要多少张页面。

\item {} 
聚焦:页面流程图中的每个页面都不必追求精细——你的目标是规划行为路径,而不是单页面交互设计,所以完全无需考虑页面内容、布局。所以你会更加聚焦于用户目标和任务的完成。不必过早陷入细节。

\item {} 
关键是很快。线框图有可能有几十张,你画起来没那么快,而且一旦进入细节,则还需要慢慢深究。但是页面流程图也许就是几个小时的事情。你就可以对整个项目心中有数了。

\end{enumerate}
\begin{itemize}
\item {} 
对于开发工程师的好处:

\end{itemize}
\begin{enumerate}
\sphinxsetlistlabels{\arabic}{enumi}{enumii}{}{.}%
\item {} 
可作为评估工作量的重要依据——可帮助他们对工作量也心中有数。

\item {} 
可做为开展代码工作的重要参考——特别是前端开发,必须得知道每一种操作指向什么页面。

\item {} 
他们会映射功能逻辑,会给你更多好的建议。

\end{enumerate}


\subparagraph{更多}
\label{\detokenize{chapter_knowledge/flow_chart:id12}}
\sphinxurl{https://t.qidianla.com/924149.html}


\subparagraph{功能流程图(Function Flow Diagram)}
\label{\detokenize{chapter_knowledge/flow_chart:function-flow-diagram}}
定义:指单页面内或多页面之间的功能操作流程,其包含在页面流程中。

任何功能都是被包含在页面内的,但一个页面内往往不止一个功能,所以单单页面流程图可能无法完整表达所有流程,而这时就需要用功能流程图来更加具体表达每个页面内所包含的功能。

相比于业务流程图,功能流程图的特点是:
\begin{itemize}
\item {} 
只展现用户的操作,不展现后台的判断;

\item {} 
只展现正常流程,不展现异常流程;

\item {} 
只可查看用户的工作流程,无法作为开发的参考。

\end{itemize}


\subparagraph{数据流程图(Data Flow Diagram)}
\label{\detokenize{chapter_knowledge/flow_chart:data-flow-diagram}}
定义:特指软件产品中,描述数据在不同节点被处理的过程所画的图表。主要表达计算机程序对于业务的实现原理。用户在功能流程图中的每一个操作,对应都会反映在数据流程图中。同时,数据流程图也可以叫程序流程图(Program
Flow Diagram)。

它是一种能全面地描述信息系统逻辑模型的主要工具。它可以利用少数几种符号综合的反映出信息在系统中的流动、处理和存储的情况。数据流程图具有抽象性和概括性。

每个流程图中都有一个核心伴随着不同操作在整个系统中不断流转。比如业务流程图大多以人为核心,每个节点都是在传递人的不同行为。而页面流程图和功能流程图也类似,都是以人的操作行为为核心,在不同页面和功能间进行流转。但数据流程图不同,它是以数据为核心,展示整个系统中,数据是如何被处理的。其更偏技术思维,更多的是展现后台程序的实现原理。所以,常常是开发人员绘制此图,而产品经理涉及较少。


\subparagraph{理解业务}
\label{\detokenize{chapter_knowledge/flow_chart:id13}}
分别展示了一个产品的业务流程、页面流程、功能流程和数据流程。从中可以发现,由业务到页面,再到功能,再到数据处理,是顺序拓展的。一个产品的页面或功能,不是凭空出现的,而是依据业务层的各个节点和流程进行设计的。这就是为什么在做产品设计时一定要先理解业务的原因。

尽量将业务、页面、功能和数据区分清楚,并且逐层递进,不要把多种类型的流程图混杂一起。这样反而会将思想搞得混乱。


\paragraph{颗粒度}
\label{\detokenize{chapter_knowledge/flow_chart:id14}}
流程图的细致程度。

我在画流程图时也常常会犹豫纠结,这个功能点用不用描写得更详细?这条分支用不用标出来?这个和服务器的交互事件用不用在流程图体现?等等这些问题,也都是产品经理在日常画图时会遇到的。


\paragraph{流程图的结构}
\label{\detokenize{chapter_knowledge/flow_chart:id15}}
流程图中大致包含四种结构:顺序结构、条件结构(又称选择结构)、循环结构。基本上大多数流程图都是由这三种结构组成的。


\paragraph{线框图 2\sphinxfootnotemark[587]}
\label{\detokenize{chapter_knowledge/flow_chart:id16}}%
\begin{footnotetext}[587]\sphinxAtStartFootnote
\sphinxnolinkurl{https://www.bilibili.com/video/BV1254y1D7Ht?from=search\&seid=14167562900175777805}
%
\end{footnotetext}\ignorespaces \begin{itemize}
\item {} 
线框图只需要使用线条、方框和灰阶色彩填充,是低保真设计图。

\item {} 
线框图主要呈现主体信息群,勾勒结构和布局,表达用户交互界面的主视觉和描述。

\item {} 
线框图是一种低保真且静态的呈现方式,产品经理通常使用纸笔来表达自己的想法。

\end{itemize}

包括:
\begin{enumerate}
\sphinxsetlistlabels{\arabic}{enumi}{enumii}{}{.}%
\item {} 
内容大纲:这个产品包含什么内容

\item {} 
信息结构、布局:这个产品的内容怎么放

\item {} 
用户交互界面:这个产品用户怎么操作

\end{enumerate}


\paragraph{总体流程图}
\label{\detokenize{chapter_knowledge/flow_chart:id17}}
\begin{figure}[H]
\centering
\capstart

\noindent\sphinxincludegraphics{{whole_flow_chart}.png}
\caption{简书APP总体流程图\sphinxhref{https://www.jianshu.com/p/e89e97858be1}{10}\sphinxfootnotemark[588]}\label{\detokenize{chapter_knowledge/flow_chart:id43}}\end{figure}
%
\begin{footnotetext}[588]\sphinxAtStartFootnote
\sphinxnolinkurl{https://www.jianshu.com/p/e89e97858be1}
%
\end{footnotetext}\ignorespaces 

\paragraph{案例}
\label{\detokenize{chapter_knowledge/flow_chart:id18}}

\subparagraph{分享购物车}
\label{\detokenize{chapter_knowledge/flow_chart:id19}}
“发起者”角度

\begin{figure}[H]
\centering
\capstart

\noindent\sphinxincludegraphics{{share_shopping}.png}
\caption{流程图}\label{\detokenize{chapter_knowledge/flow_chart:id44}}\end{figure}

节点分别是:
\sphinxhref{https://coffee.pmcaff.com/article/2714966199749760/pmcaff?utm\_source=forum}{3}%
\begin{footnote}[589]\sphinxAtStartFootnote
\sphinxnolinkurl{https://coffee.pmcaff.com/article/2714966199749760/pmcaff?utm\_source=forum}
%
\end{footnote}
\begin{enumerate}
\sphinxsetlistlabels{\arabic}{enumi}{enumii}{}{.}%
\item {} 
用户是作为起点,来开始;

\item {} 
抵达的第一个页面,是购物车;

\item {} 
在购物车,有“一键分享”的按钮;

\item {} 
点击完“一键分享”后,吊起商品选择确认页面;支持“取消”商品的勾选;

\item {} 
用户点击确认后,吊起好友筛选列表;

\item {} 
在好友筛选列表中,选中某一个特定的好友;弹出“确认”或“取消”按钮;

\item {} 
用户点击“确认”后,则把之前选择好的商品商品列表发给Ta;

\end{enumerate}


\subparagraph{登录注册流程图 4\sphinxfootnotemark[590]}
\label{\detokenize{chapter_knowledge/flow_chart:id20}}%
\begin{footnotetext}[590]\sphinxAtStartFootnote
\sphinxnolinkurl{http://www.woshipm.com/pd/675174.html}
%
\end{footnotetext}\ignorespaces 
\begin{figure}[H]
\centering
\capstart

\noindent\sphinxincludegraphics{{login_flow_chart}.png}
\caption{登录注册流程图}\label{\detokenize{chapter_knowledge/flow_chart:id45}}\end{figure}

一个大的流程就是由许多小流程(一个流程一个小模块)组成,每个小流程(常用的,每个App流程基本改动不太大的)可反复使用,提高工作效率,这就有点像面向对象的封装思想。


\subparagraph{泳道图}
\label{\detokenize{chapter_knowledge/flow_chart:id21}}
可以理解为一种特殊的流程图,只不过泳道图会把部门和职能划分开。因此,泳道流程图是一种反映商业流程里,人与人或组织与组织之间关系的特殊图表。


\subparagraph{泳道图的作用}
\label{\detokenize{chapter_knowledge/flow_chart:id22}}\begin{enumerate}
\sphinxsetlistlabels{\arabic}{enumi}{enumii}{}{.}%
\item {} 
泳道图在商业流程里,可以直观地反映出人与人之间的关系,令每个人清楚的掌握自己所负责的事项任务。

\item {} 
对于企业而言,泳道图能够让工作部署更加流程,提升工作效率。

\item {} 
有助于研究整个流程中,人与人,或者是工作小组和工作小组之间交接的动作

\end{enumerate}


\subparagraph{步骤}
\label{\detokenize{chapter_knowledge/flow_chart:id23}}\begin{enumerate}
\sphinxsetlistlabels{\arabic}{enumi}{enumii}{}{.}%
\item {} 
罗列出参与此流程不同人员的各自工作内容,并输入到泳道图的左侧或者上方。

\item {} 
设计各个环节设计的流程图,并写入到各个泳道里。

\item {} 
对着写步骤环节进行深入的探讨,并将他们放置于合适的泳道上。

\item {} 
通过上述三步,基本给出了流程图的草稿,在此基础上再稍作调整即可完成。

\end{enumerate}


\paragraph{示例:}
\label{\detokenize{chapter_knowledge/flow_chart:id24}}
招聘的流程\sphinxhref{https://juejin.cn/post/6923717340127297549}{19}%
\begin{footnote}[591]\sphinxAtStartFootnote
\sphinxnolinkurl{https://juejin.cn/post/6923717340127297549}
%
\end{footnote}:

\begin{figure}[H]
\centering
\capstart

\noindent\sphinxincludegraphics{{hire_process}.png}
\caption{招聘的流程
泳道图展示\sphinxhref{https://www.bilibili.com/video/BV1254y1D7Ht?from=search\&seid=14167562900175777805}{2}\sphinxfootnotemark[592]}\label{\detokenize{chapter_knowledge/flow_chart:id46}}\end{figure}
%
\begin{footnotetext}[592]\sphinxAtStartFootnote
\sphinxnolinkurl{https://www.bilibili.com/video/BV1254y1D7Ht?from=search\&seid=14167562900175777805}
%
\end{footnotetext}\ignorespaces 

\subparagraph{AI落地}
\label{\detokenize{chapter_knowledge/flow_chart:ai}}
一个AI产品从需求到落地,大概需要经历以下环节:

\sphinxstylestrong{需求分析→数据采集→数据清洗→数据标注→训练迭代→测试验证→交付模型→生产环境部署}


\subparagraph{常见的绘制流程图的工具}
\label{\detokenize{chapter_knowledge/flow_chart:id25}}
(1)在线工具
\begin{itemize}
\item {} 
ProcessOn:\sphinxurl{https://www.processon.com/}

\item {} 
draw.io:\sphinxurl{https://www.draw.io/}

\item {} 
excalidraw: \sphinxurl{https://excalidraw.com/}

\end{itemize}

(2)客户端
\begin{itemize}
\item {} 
Microsoft Visio

\item {} 
edraw亿图

\item {} 
xmind

\item {} 
omniGraffle(mac)

\item {} 
StarUML

\end{itemize}


\paragraph{更多图}
\label{\detokenize{chapter_knowledge/flow_chart:id26}}\begin{itemize}
\item {} 
用例图

\item {} 
信息架构图

\item {} 
线框图

\item {} 
实体关系图\sphinxhref{http://www.woshipm.com/pmd/3864.html}{5}%
\begin{footnote}[593]\sphinxAtStartFootnote
\sphinxnolinkurl{http://www.woshipm.com/pmd/3864.html}
%
\end{footnote}

\item {} 
产品结构图

\item {} 
蜘蛛图

\item {} 
气泡图

\item {} 
散布图

\end{itemize}


\subparagraph{用例图 11\sphinxfootnotemark[594]}
\label{\detokenize{chapter_knowledge/flow_chart:id27}}%
\begin{footnotetext}[594]\sphinxAtStartFootnote
\sphinxnolinkurl{https://tangjie.me/blog/115.html}
%
\end{footnotetext}\ignorespaces 

\subparagraph{用例}
\label{\detokenize{chapter_knowledge/flow_chart:id28}}
用例(Use
Case)是一种描述产品需求的方法,使用用例的方法来描述产品需求的过程就是用例模型,用例模型是由用例图和每一个用例的详细描述文档所组成的。在技术和产品的工作领域里都有用例模型的技能知识。技术人员的用例主要是为了方便在多名技术人员协同工作,或者技术人员任务交接时,让参与者更好的理解代码的逻辑结构。产品人员的用例主要是为了方便技术研发和功能测试时,让参与者更好的理解功能的逻辑。

用例起源和发展于软件时代的产品研发,后来被综合到UML规范之中,成为一种标准化的需求表述体系。虽然用例在软件研发和技术工作中应用的非常广泛,但是在互联网产品规划和设计中,并不经常使用,互联网产品的需求表达为了敏捷效率,\sphinxstylestrong{通常采用原型加产品需求文档}。


\subparagraph{用例图}
\label{\detokenize{chapter_knowledge/flow_chart:id29}}
用例图并不是画成了图形的用例。用例图包含一组用例,每一个用例用椭圆表示,放置在矩形框中;矩形框表示整个系统。矩形框外画如图所示的小人,表示参与者。参与者不一定是人,可以是其它产品、软件或硬件等等。某一参与者与某一用例用线连起来,表示该参与者和该用例有交互。

\begin{figure}[H]
\centering
\capstart

\noindent\sphinxincludegraphics{{use_func_pic}.png}
\caption{用例图}\label{\detokenize{chapter_knowledge/flow_chart:id47}}\end{figure}

许多人通过UML认识了用例,UML定义为展现用例的图形符号。UML并不是为描述用例定义书写格式的标准,因此许多人误认为这些图形符号就是用例本身;然而,图形符号只能给出最简单的一个或一组用例的概要。UML是用例图形符号最流行的标准,但是除了UML标准,用例也有其它的可选择的标准。


\subparagraph{用例描述文档}
\label{\detokenize{chapter_knowledge/flow_chart:id30}}\begin{enumerate}
\sphinxsetlistlabels{\arabic}{enumi}{enumii}{}{.}%
\item {} 
用例名称:本用例的名称或者编号

\item {} 
行为角色:参与或操作(执行)该用例的角色

\item {} 
简要说明:简要的描述一下本用例的需求(作用和目的)

\item {} 
前置条件:参与或操作(执行)本用例的前提条件,或者所处的状态

\item {} 
后置条件:执行完毕后的结果或者状态

\end{enumerate}


\subparagraph{信息架构图}
\label{\detokenize{chapter_knowledge/flow_chart:information-infra}}\label{\detokenize{chapter_knowledge/flow_chart:id31}}

\subparagraph{信息架构}
\label{\detokenize{chapter_knowledge/flow_chart:id32}}
产品经理的工作需要设计业务架构、产品架构和信息架构。一个企业的业务架构决定了产品架构,产品架构决定了信息架构,是一个递进的关系。\sphinxhref{http://www.woshipm.com/pd/3236364.html}{14}%
\begin{footnote}[595]\sphinxAtStartFootnote
\sphinxnolinkurl{http://www.woshipm.com/pd/3236364.html}
%
\end{footnote}

\begin{figure}[H]
\centering
\capstart

\noindent\sphinxincludegraphics{{information_structure_flowchart}.png}
\caption{信息架构处于位置}\label{\detokenize{chapter_knowledge/flow_chart:id48}}\end{figure}

设计里面非常关键的就是信息架构。信息架构最主要的就是把一些核心任务给突出出来。根据你的用户画像,你要知道用户最核心的任务是哪些,根据这些核心任务再去排列组合你的信息架构。参考《信息架构,超越WEB设计》\sphinxhref{http://www.woshipm.com/pmd/3024508.html}{13}%
\begin{footnote}[596]\sphinxAtStartFootnote
\sphinxnolinkurl{http://www.woshipm.com/pmd/3024508.html}
%
\end{footnote}

\begin{figure}[H]
\centering
\capstart

\noindent\sphinxincludegraphics{{information_infra}.png}
\caption{信息架构}\label{\detokenize{chapter_knowledge/flow_chart:id49}}\end{figure}

信息架构完成之后,我们就要做具体的设计。设计要遵从设计规范。安卓也、
iOS也好,或者是各个公司都会有自己的设计规范,这都是最优秀的、最有经验的设计师输出的,所以如果我们能够遵照设计规范来做我们的设计,就一定是比较不错的一个设计了。
\sphinxhref{https://www.zhihu.com/question/311376037/answer/1628258822}{9}%
\begin{footnote}[597]\sphinxAtStartFootnote
\sphinxnolinkurl{https://www.zhihu.com/question/311376037/answer/1628258822}
%
\end{footnote}


\subparagraph{产品结构图}
\label{\detokenize{chapter_knowledge/flow_chart:id33}}
《用户体验要素》中产品结构层处在中间的位置,正处在一个又抽象到具象的过渡阶段。\sphinxhref{http://www.woshipm.com/pd/2611357.html}{18}%
\begin{footnote}[598]\sphinxAtStartFootnote
\sphinxnolinkurl{http://www.woshipm.com/pd/2611357.html}
%
\end{footnote}

产品结构图是一种让产品经理通过思维导图的方式梳理思路的方法,通过这种方法可以明确产品有多少个频道、有多少个页面、页面有多少个功能模块、功能模块有多少个元素,逐步的将脑海里的想法明确梳理成结构。虽然这种方法能够明确产品的结构,但是这样的思维导图也就只有产品经理自己能够看懂,因为对于设计和技术人员这是一个抽象的表述方式,如果没有详细的讲解,是很难理解的。

产品结构图是将产品原型具体化的一种方式,只是罗列了产品的频道页面和功能,但是没有详细的进行推演,关于细化方面是否符合产品逻辑,是否符合用户体验,这些都是没有深思过的,因此我们接下来就要进行原型设计,开始具体的考虑可行性。\sphinxhref{https://tangjie.me/blog/113.html}{12}%
\begin{footnote}[599]\sphinxAtStartFootnote
\sphinxnolinkurl{https://tangjie.me/blog/113.html}
%
\end{footnote}

产品结构图是综合展示产品信息和功能逻辑的图表,简单说产品结构图就是产品原型的简化表达;而不只是产品功能结构图的简称\sphinxhref{http://www.woshipm.com/pmd/844937.html}{15}%
\begin{footnote}[600]\sphinxAtStartFootnote
\sphinxnolinkurl{http://www.woshipm.com/pmd/844937.html}
%
\end{footnote}

一个公式:功能结构图(骨架)+信息结构图(血肉)=产品结构图(产品原型简化版)\sphinxhref{http://www.woshipm.com/pd/2611357.html}{16}%
\begin{footnote}[601]\sphinxAtStartFootnote
\sphinxnolinkurl{http://www.woshipm.com/pd/2611357.html}
%
\end{footnote}


\subparagraph{产品功能结构图}
\label{\detokenize{chapter_knowledge/flow_chart:id34}}
在一款产品的设计过程中,功能结构图是必须的,信息结构图视产品和PM自身而定,通常我们初步确定了产品功能结构图(产品功能框架)之后才开始绘制产品信息结构图。

在产品设计流程中,产品功能结构图是产品概念化阶段的初期输出,产品结构图是产品概念化的尾期阶段输出物,
\begin{itemize}
\item {} 
梳理需求,以鸟瞰的方式对整个产品页面中的功能结构形成一个直观的认识。

\item {} 
思考并明确产品的功能模块及其功能组成。\sphinxhref{http://www.woshipm.com/pd/2611357.html}{18}%
\begin{footnote}[602]\sphinxAtStartFootnote
\sphinxnolinkurl{http://www.woshipm.com/pd/2611357.html}
%
\end{footnote}

\item {} 
同时以产品结构图作为绘制原型的依据,可以避免我们在产品设计中边画边改,跳进死掐细节,不见森林的陷阱。\sphinxhref{http://www.woshipm.com/pmd/844937.html}{15}%
\begin{footnote}[603]\sphinxAtStartFootnote
\sphinxnolinkurl{http://www.woshipm.com/pmd/844937.html}
%
\end{footnote}

\end{itemize}


\subparagraph{信息结构图}
\label{\detokenize{chapter_knowledge/flow_chart:id35}}
信息结构:文章标题、发布时间、正文内容。\sphinxhref{https://tangjie.me/blog/213.html}{17}%
\begin{footnote}[604]\sphinxAtStartFootnote
\sphinxnolinkurl{https://tangjie.me/blog/213.html}
%
\end{footnote}

指脱离产品的实际页面,将产品的数据抽象出来,组合分类的图表。

作用:
\begin{itemize}
\item {} 
帮助PM梳理复杂内容的信息组成,避免信息内容在展示过程中出现遗漏、混乱、重复;

\item {} 
作为开发工程师建立数据库的参考依据;

\end{itemize}

\sphinxstylestrong{脱离实际页面}:将信息结构图完全按照页面的逻辑顺序来进行分类组合,严格意义上来说,这种图表不是一份合格的信息结构图。\sphinxhref{http://www.woshipm.com/pd/2611357.html}{16}%
\begin{footnote}[605]\sphinxAtStartFootnote
\sphinxnolinkurl{http://www.woshipm.com/pd/2611357.html}
%
\end{footnote}

\begin{figure}[H]
\centering
\capstart

\noindent\sphinxincludegraphics{{information_structure}.png}
\caption{信息结构图}\label{\detokenize{chapter_knowledge/flow_chart:id50}}\end{figure}

如何绘制呢?

功能结构图绘制完之后,需要思考一个问题,在这些场景中,涉及到了哪些对象,如果对编程有了解的朋友应该知道基于对象的编程思维,万物皆是对象\textasciitilde{}\sphinxhref{http://www.woshipm.com/pd/2611357.html}{18}%
\begin{footnote}[606]\sphinxAtStartFootnote
\sphinxnolinkurl{http://www.woshipm.com/pd/2611357.html}
%
\end{footnote}


\subparagraph{综合起来}
\label{\detokenize{chapter_knowledge/flow_chart:id36}}
示例:

\begin{figure}[H]
\centering
\capstart

\noindent\sphinxincludegraphics{{product_infra_eg}.png}
\caption{某产品的产品结构图}\label{\detokenize{chapter_knowledge/flow_chart:id51}}\end{figure}

如上图示例,“活动大全”的产品结构依次是:产品 \sphinxhyphen{}> 频道 \sphinxhyphen{}> 页面 \sphinxhyphen{}> 页面元素
\sphinxhyphen{}> 操作 \sphinxhyphen{}>
元素。我们换一个角度观看示例,产品结构图实际上就是一种结构化的产品原型。这样做的目的就是梳理产品结构逻辑,让我们清楚的知道产品有几个频道,频道下面有没有子频道或者有多少个页面,这些页面里又有哪些功能模块,这些功能模块里又有哪些元素。


\subsubsection{静态页面}
\label{\detokenize{chapter_knowledge/static_page:id1}}\label{\detokenize{chapter_knowledge/static_page::doc}}
导航信息、组件元素、页面布局\sphinxhyphen{}栅格系统、文案信息、重要备注
\sphinxhref{https://www.yinxiang.com/everhub/note/f9ab87ee-73e6-4241-9428-9507cbfd007f}{1}%
\begin{footnote}[607]\sphinxAtStartFootnote
\sphinxnolinkurl{https://www.yinxiang.com/everhub/note/f9ab87ee-73e6-4241-9428-9507cbfd007f}
%
\end{footnote}


\paragraph{低保真 Axure实战}
\label{\detokenize{chapter_knowledge/static_page:axure}}\label{\detokenize{chapter_knowledge/static_page:page-axure}}
大小 \sphinxhref{https://www.bilibili.com/video/BV1WE411w7LW?p=3}{2}%
\begin{footnote}[608]\sphinxAtStartFootnote
\sphinxnolinkurl{https://www.bilibili.com/video/BV1WE411w7LW?p=3}
%
\end{footnote}:
\begin{center}\sphinxincludegraphics{{page2_axure}.png}\end{center}

锁定: \begin{center}\sphinxincludegraphics{{lock_axure}.png}\end{center}

关闭按钮: \begin{center}\sphinxincludegraphics{{close_axure}.png}\end{center}

整体: \begin{center}\sphinxincludegraphics{{page2_axure}.png}\end{center}


\paragraph{高保真 Axure实战}
\label{\detokenize{chapter_knowledge/static_page:page-done-axure}}\label{\detokenize{chapter_knowledge/static_page:id2}}
\sphinxurl{https://library.ant.design/}

下载,解压

\begin{figure}[H]
\centering
\capstart

\noindent\sphinxincludegraphics{{ant_design2axure}.png}
\caption{导入原型库}\label{\detokenize{chapter_knowledge/static_page:id9}}\end{figure}

\begin{figure}[H]
\centering
\capstart

\noindent\sphinxincludegraphics{{remove_now_lib_axure}.png}
\caption{移出当前原型库}\label{\detokenize{chapter_knowledge/static_page:id10}}\end{figure}

图标:\sphinxurl{https://www.iconfont.cn/}

登录失败。。

\begin{figure}[H]
\centering
\capstart

\noindent\sphinxincludegraphics{{svg2image}.png}
\caption{svg到图片}\label{\detokenize{chapter_knowledge/static_page:id11}}\end{figure}

\begin{figure}[H]
\centering
\capstart

\noindent\sphinxincludegraphics{{image_color}.png}
\caption{图片黄色}\label{\detokenize{chapter_knowledge/static_page:id12}}\end{figure}

\begin{figure}[H]
\centering
\capstart

\noindent\sphinxincludegraphics{{awesomeicon_axure}.png}
\caption{awesomeicon\_axure}\label{\detokenize{chapter_knowledge/static_page:id13}}\end{figure}

\sphinxurl{https://www.flaticon.com/search?word=close}

对齐:先圆形后图标,居中

卡死了,为啥?

\begin{figure}[H]
\centering
\capstart

\noindent\sphinxincludegraphics{{kuang0_axure}.png}
\caption{0框}\label{\detokenize{chapter_knowledge/static_page:id14}}\end{figure}

勉强做完,实在登不上iconfont..QQ找不到,wechat用的flaticon

\begin{figure}[H]
\centering
\capstart

\noindent\sphinxincludegraphics{{page_done}.png}
\caption{制作完成}\label{\detokenize{chapter_knowledge/static_page:id15}}\end{figure}


\paragraph{交互设计图 —— 只做交互相关解释}
\label{\detokenize{chapter_knowledge/static_page:ui-design-docs}}\label{\detokenize{chapter_knowledge/static_page:id3}}

\subparagraph{什么是「交互文档」?}
\label{\detokenize{chapter_knowledge/static_page:id4}}\begin{itemize}
\item {} 
交互文档也称DRD,是用来告诉别人「页面设计细节」的一个说明文档。

\item {} 
包括页面跳转逻辑、页面交互逻辑、页面显示状态等细节的说明。\sphinxhref{https://www.zhihu.com/question/19825650/answer/758117107}{4}%
\begin{footnote}[609]\sphinxAtStartFootnote
\sphinxnolinkurl{https://www.zhihu.com/question/19825650/answer/758117107}
%
\end{footnote}

\end{itemize}


\subparagraph{为什么要写交互文档?}
\label{\detokenize{chapter_knowledge/static_page:id5}}\begin{itemize}
\item {} 
交互设计师的工作就是完善细节设计,那你设计的这些细节应该如何告诉开发呢?口述当然是不行的了,第一是记不住,第二是会理解错误,第三就是最重要的一点,没有历史记录。

\item {} 
现在办公用的聊天软件非常多,但是所有企业依然使用邮件的方式沟通,这正是因为邮件可以作为沟通证据保存下来,没有沟通记录就等于没说,交互文档也是一个道理,你要把你做的所有细节都写下来,给其他人发邮件,告诉对方要做什么、要怎么做。

\end{itemize}


\subparagraph{交互文档给谁看?}
\label{\detokenize{chapter_knowledge/static_page:id6}}
\sphinxstylestrong{交互设计文档}主要是\sphinxstylestrong{给前端和UI看的文档},所以文档的说明应\sphinxstylestrong{不包含业务解释、规则说明}相关的内容。可以借用需求文档的思路,还是先排版,页面分为“准高保证原型图”、“注释说明”、“交互”、“其他补充内容”。\sphinxhref{http://www.woshipm.com/pmd/418829.html}{3}%
\begin{footnote}[610]\sphinxAtStartFootnote
\sphinxnolinkurl{http://www.woshipm.com/pmd/418829.html}
%
\end{footnote}
\begin{itemize}
\item {} 
给产品经理看:交互逻辑、交互方式,需要与产品经理确认。产品经理是把控整个产品的一个角色,任何事情他都需要做到心里有数,包括交互设计。

\item {} 
给UI设计师看:交互设计会包含很多状态、很多细节、很多页面,UI设计师要确保每一个状态、每一个页面都画出来,这样在开发的时候才不会漏掉东西。

\item {} 
给开发人员看:代码逻辑和我们普通说话的逻辑不同,说话的时候可以是点击这里跳转页面,而程序员需要知道点击什么位置,使用什么方式跳转到哪一个页面,中途需要判断哪些规则,如果中途失败怎么办…因此交互文档就更重要了。

\item {} 
给测试人员看:测试岗位也称QA,属于研发部门旗下的一个岗位(很多小公司没有这个岗位),你需要把你所有的设计细节都告诉他,然后他去测试一遍最终效果与你设计的是否完全一致(不包括UI)。

\end{itemize}


\subparagraph{内容}
\label{\detokenize{chapter_knowledge/static_page:id7}}
\begin{figure}[H]
\centering
\capstart

\noindent\sphinxincludegraphics{{UI_design_docs}.png}
\caption{交互设计文档}\label{\detokenize{chapter_knowledge/static_page:id16}}\end{figure}
\begin{itemize}
\item {} 
文档页面标题:一般在每一页文档的顶部。写明当页内容是属于哪个模块或流程的,让看的人不容易迷失;

\item {} 
界面标题:注意命名,方便交互稿中的互相联系,如“跳转【XX页面】”,“回到【XX界面】状态”;

\item {} 
界面:界面尺寸建议按实际界面的比例缩小,避免你想当然的设计并不符合规范,也避免一个界面太大影响阅读效果;

\item {} 
设计说明:逻辑关系、操作流程或反馈、元素状态、字符限制、异常/特殊状态
等等,都可以放在设计说明中;

\item {} 
交互说明:提示弹框样式、功能实现方式、提示内容等响应用户物理操作的反馈画面。

\item {} 
流程线:说明界面间逻辑关系;

\item {} 
跳转链接:指向其他页面,例如某子流程,开发伙伴阅读起来会很方便。

\item {} 
注释说明:应该包括输入检测标准、缺省值、界面元素变化、显示上限等情景的说明。

\end{itemize}


\paragraph{更多}
\label{\detokenize{chapter_knowledge/static_page:id8}}\begin{itemize}
\item {} 
\sphinxurl{https://www.axurethemes.com/axure-icon-packs}

\item {} 
\sphinxurl{https://fontawesome.com/download}

\item {} 
\sphinxurl{https://forum.axure.com/t/4-steps-to-get-fontawesome-5-11-2-free-axure-9-working/65942}

\item {} 
\sphinxurl{https://fontawesome.com/icons?d=gallery\&p=2\&q=close}

\item {} 
\sphinxurl{https://github.com/bsdfzzzy/iconfont-all-in-one}

\item {} 
\sphinxurl{http://caibaojian.com/font-awesome.html}

\end{itemize}


\subsubsection{原型设计}
\label{\detokenize{chapter_knowledge/prototype_design:id1}}\label{\detokenize{chapter_knowledge/prototype_design::doc}}
重点:\sphinxstylestrong{可视+互动}

在产品定义阶段,用户参与产品设计和体验,解决75\%的问题才是最高效的。

原型的定义:是产品方案的输出物
\sphinxhref{https://www.zhihu.com/question/55997614/answer/615628989}{1}%
\begin{footnote}[611]\sphinxAtStartFootnote
\sphinxnolinkurl{https://www.zhihu.com/question/55997614/answer/615628989}
%
\end{footnote}。

\sphinxstylestrong{目的}:将想法转化为高效传达给其他人(程序员和ui)或者与用一起测试的形式,并且随看时间的推移,可以持续进行调整。你可以用它来自如的测试未完成的想法,从而达到最佳的结果。接受反馈,将反馈再次融入到你的原型设计中。只要对方能够听懂看懂就可以了,所以\sphinxstylestrong{使用手绘原型}是最高效率的方法。


\paragraph{原型设计时需要思考的问题 9\sphinxfootnotemark[612]}
\label{\detokenize{chapter_knowledge/prototype_design:id2}}%
\begin{footnotetext}[612]\sphinxAtStartFootnote
\sphinxnolinkurl{https://zhuanlan.zhihu.com/p/56954145}
%
\end{footnotetext}\ignorespaces \begin{enumerate}
\sphinxsetlistlabels{\arabic}{enumi}{enumii}{}{.}%
\item {} 
此原型的用户是谁?

\item {} 
原型需要达到的目的是什么?

\item {} 
用户的使用场景是什么?

\item {} 
需要提供什么样的用户体验?

\item {} 
后续的工作流程如何把握?

\end{enumerate}


\paragraph{原型设计和评审}
\label{\detokenize{chapter_knowledge/prototype_design:id3}}
由产品经理主导的。基于运营的需求设计原型,在原型设计完后,要经过内部评审和外部评审。

在内部评审中,产品经理要召集数据中台的架构师、模型设计师、数据开发工程师、后端开发工程师、前端开发工程师、UI设计师、测试工程师,说明整个功能的价值和详细的\sphinxstylestrong{业务流程、操作流程},确保大家理解一致。

接下来,产品经理和运营人员要针对原型做一次外部评审,把\sphinxstylestrong{有歧义的地方一并解决}。

对于比较重要的功能,产品经理需要发邮件让运营人员进一步确认,并同步给所有的产品/运营人员,保证大家的口径一致。


\paragraph{原型分类}
\label{\detokenize{chapter_knowledge/prototype_design:id4}}

\subparagraph{低保真原型}
\label{\detokenize{chapter_knowledge/prototype_design:id5}}

\subparagraph{手绘原型}
\label{\detokenize{chapter_knowledge/prototype_design:id6}}
线框图,不可交互,手绘来最快速的表现产品轮廓手绘原型在初期验证想法时非常高效,也方便讨论和重构,同时也适合敏捷开发时快速出原型;

\begin{figure}[H]
\centering
\capstart

\noindent\sphinxincludegraphics{{hand_draw}.png}
\caption{手绘原型}\label{\detokenize{chapter_knowledge/prototype_design:id19}}\end{figure}


\subparagraph{灰模原型}
\label{\detokenize{chapter_knowledge/prototype_design:id7}}
由图形设计软件制作而成,不可交互。最常用的软件是Photoshop和Fireworks,相对手绘原型,灰模更加清晰和整洁,也适用于正式场合的PPT形式宣讲,或打印在纸上,来展示主业务流程。平面原型:如果不会使用图形软件也可以使用Axure
RP设计,相比交互原型,灰模原型只是缺少交互效果,仅仅是将产品需求以线框结构的方式展示出来,让产品需求更加规整的直观展现。\sphinxhref{https://tangjie.me/blog/114.html}{8}%
\begin{footnote}[613]\sphinxAtStartFootnote
\sphinxnolinkurl{https://tangjie.me/blog/114.html}
%
\end{footnote}

\begin{figure}[H]
\centering
\capstart

\noindent\sphinxincludegraphics{{PS}.png}
\caption{PS设计}\label{\detokenize{chapter_knowledge/prototype_design:id20}}\end{figure}

\begin{figure}[H]
\centering
\capstart

\noindent\sphinxincludegraphics{{grey_axure}.png}
\caption{Axure平面原型}\label{\detokenize{chapter_knowledge/prototype_design:id21}}\end{figure}


\subparagraph{中保真原型}
\label{\detokenize{chapter_knowledge/prototype_design:id8}}
可简单交互(点击页面跳转),简单配色并配图,但字体、配色和配图不完美;低保真
Axure实战:ref:\sphinxcode{\sphinxupquote{page\_Axure}}


\subparagraph{高保真原型}
\label{\detokenize{chapter_knowledge/prototype_design:id9}}
开发和设计参与,原型可交互,可录入信息,可以实现所有流程,有动画效果,字体、配色和配图与真实产品一致;高保真
Axure实战:ref:\sphinxcode{\sphinxupquote{page\_done\_Axure}}


\subparagraph{线框图}
\label{\detokenize{chapter_knowledge/prototype_design:id10}}
\begin{figure}[H]
\centering
\capstart

\noindent\sphinxincludegraphics{{lineframe_chart}.jpg}
\caption{线框图\sphinxhref{https://www.bilibili.com/video/BV1Yx411f7d6?from=search\&seid=9942601070785163162}{7}\sphinxfootnotemark[614]}\label{\detokenize{chapter_knowledge/prototype_design:id22}}\end{figure}
%
\begin{footnotetext}[614]\sphinxAtStartFootnote
\sphinxnolinkurl{https://www.bilibili.com/video/BV1Yx411f7d6?from=search\&seid=9942601070785163162}
%
\end{footnotetext}\ignorespaces 

\paragraph{反馈圈分类}
\label{\detokenize{chapter_knowledge/prototype_design:id11}}\begin{itemize}
\item {} 
团队成员或朋友:团队成员,公司员工,社交圈朋友。特点:容易找到,易与沟通。使用低保真原型,了解主要流程是否存在问题。

\item {} 
行业专家:相关领域产品经理和产品专家,他们了解目标市场。特点:能提出逻辑清晰的意见。使用低,中保真原型。

\item {} 
客户:目标市场的客户。使用中、高保真原型,他们会提供非常棒的反馈意见

\item {} 
客户的客户:特定人群应用,如投资人等。使用高保真原型展示,获取真实反馈。

\end{itemize}


\paragraph{信息收集工具}
\label{\detokenize{chapter_knowledge/prototype_design:id12}}\begin{itemize}
\item {} 
记笔记;

\item {} 
手机录音;

\item {} 
在用户允许的情况下,录入表情;

\item {} 
原型操作过程手机或电脑页面录制。

\end{itemize}


\paragraph{工具推荐}
\label{\detokenize{chapter_knowledge/prototype_design:id13}}\begin{itemize}
\item {} 
低保真\sphinxhyphen{}线框\sphinxhyphen{}纸质原型:可以用于产品团队内部交流使用,通过头脑风暴,跑通住业务流程。确定MVP(Minimum
Viable Product, MVP)方案,确定中保真原型方案。

\item {} 
中保真原型–墨刀或axure:对于朋友、专家和客户可以提供具有交互功能的可录入信息的中保真原型,设计尽量保证完成,以便于手机用户的全部操作行为。

\item {} 
高保真原型:开发资源富裕的团队,如微信团队,可以实现每个想法的高保真原型快速输出和范围测试。

\end{itemize}


\subparagraph{墨刀}
\label{\detokenize{chapter_knowledge/prototype_design:id14}}
墨刀\sphinxhref{https://zhuanlan.zhihu.com/p/33997501}{3}%
\begin{footnote}[615]\sphinxAtStartFootnote
\sphinxnolinkurl{https://zhuanlan.zhihu.com/p/33997501}
%
\end{footnote}是一个款移动端原型制作工具,原型制作制作方便,可以快捷添加一些简单的交互动画,信息录入,逻辑判断等功能,支持微信分享,便于传播手机信息。

挑一种学一下就可以了(最易学习的是墨刀,Sketch学会了用起来最顺手快捷,Axure应该算是最不好用的一个了


\paragraph{产品演进路线图 2\sphinxfootnotemark[616]}
\label{\detokenize{chapter_knowledge/prototype_design:id15}}%
\begin{footnotetext}[616]\sphinxAtStartFootnote
\sphinxnolinkurl{https://www.bilibili.com/video/BV1254y1D7Ht?from=search\&seid=14167562900175777805}
%
\end{footnotetext}\ignorespaces 
产品演进路线图是为了给产品的利益相关者以及团队呈现产品长远发展的构思。使团队内部和利益相关者对项目的长远发展一目了然并达成共识
\begin{itemize}
\item {} 
近期:为了满足需求必须要包含的功能

\item {} 
中期:关注范围广、有一定的灵活性的功能

\item {} 
远期:顶层设计、宽泛的范围、更加灵活的功能

\end{itemize}


\paragraph{绘制操作流程 9\sphinxfootnotemark[617]}
\label{\detokenize{chapter_knowledge/prototype_design:id16}}%
\begin{footnotetext}[617]\sphinxAtStartFootnote
\sphinxnolinkurl{https://zhuanlan.zhihu.com/p/56954145}
%
\end{footnotetext}\ignorespaces \begin{itemize}
\item {} 
对于交互比较复杂的操作,需要绘制操作流程图,可以是Visio图,也可以是页面跳转说明

\item {} 
操作流程图中形状需要根据标准进行选择

\item {} 
需要包含正常操作逻辑和异常操作逻辑

\end{itemize}


\paragraph{原型发布和导出 9\sphinxfootnotemark[618]}
\label{\detokenize{chapter_knowledge/prototype_design:id17}}%
\begin{footnotetext}[618]\sphinxAtStartFootnote
\sphinxnolinkurl{https://zhuanlan.zhihu.com/p/56954145}
%
\end{footnotetext}\ignorespaces \begin{itemize}
\item {} 
原型发布时选择生成HTML文件,在相应的文件夹中,点击start.html就可以通过浏览器打开发布的文件(在线太慢,一般建议导出使用)

\item {} 
原型导出时,也可以将所有相关页面导出为PNG图片

\item {} 
给前端提供原型时,最好分别提供HTML文件和PNG图片

\end{itemize}


\paragraph{验证(Validate) 4\sphinxfootnotemark[619]}
\label{\detokenize{chapter_knowledge/prototype_design:validate-4}}%
\begin{footnotetext}[619]\sphinxAtStartFootnote
\sphinxnolinkurl{https://www.jianshu.com/p/cb6ae5a3f3fa}
%
\end{footnotetext}\ignorespaces 
验证产品原型

\begin{figure}[H]
\centering
\capstart

\noindent\sphinxincludegraphics{{after_validate}.png}
\caption{验证产品原型后续工作\sphinxhref{https://mp.weixin.qq.com/s?\_\_biz=MjM5MzE3MDQ3Mw==\&mid=2650404998\&idx=3\&sn=e4bf27058ac6a697bfb1ae3cbb319e14\&chksm=be964dc089e1c4d613d4dcf763e01fbc65dee8b08136e34ebf62c1d22cbc7d83c58502416f2a\&scene=21\#wechat\_redirect}{10}\sphinxfootnotemark[620]}\label{\detokenize{chapter_knowledge/prototype_design:id23}}\end{figure}
%
\begin{footnotetext}[620]\sphinxAtStartFootnote
\sphinxnolinkurl{https://mp.weixin.qq.com/s?\_\_biz=MjM5MzE3MDQ3Mw==\&mid=2650404998\&idx=3\&sn=e4bf27058ac6a697bfb1ae3cbb319e14\&chksm=be964dc089e1c4d613d4dcf763e01fbc65dee8b08136e34ebf62c1d22cbc7d83c58502416f2a\&scene=21\#wechat\_redirect}
%
\end{footnotetext}\ignorespaces 

\paragraph{题目 5\sphinxfootnotemark[621]}
\label{\detokenize{chapter_knowledge/prototype_design:id18}}%
\begin{footnotetext}[621]\sphinxAtStartFootnote
\sphinxnolinkurl{https://blog.nowcoder.net/n/9bd8651faead4a73ae344be0b74128de}
%
\end{footnotetext}\ignorespaces 
Axure文件的后缀名RP(Rapid Prototyping)指快速原型
\begin{itemize}
\item {} 
Action sheet:动作菜单/动作面板/行动列表

\item {} 
Picker:选择器/拾取器

\item {} 
Toast:吐司提示
产品设计中常把出现在屏幕中、用以引起注意,短暂出现后消失的提醒控件叫做

\item {} 
Status bar:全局修改状态栏

\item {} 
More: \sphinxhref{https://www.jianshu.com/nb/9076183}{6}%
\begin{footnote}[622]\sphinxAtStartFootnote
\sphinxnolinkurl{https://www.jianshu.com/nb/9076183}
%
\end{footnote}

\end{itemize}


\subsubsection{交互设计}
\label{\detokenize{chapter_knowledge/IXD_design:id1}}\label{\detokenize{chapter_knowledge/IXD_design::doc}}

\paragraph{什么是交互}
\label{\detokenize{chapter_knowledge/IXD_design:id2}}\label{\detokenize{chapter_knowledge/IXD_design:id3}}
互联网人所讲的交互,其实是“人机交互”的简称,即为完成某任务,人与计算机系统之间以某种交互方式进行的信息交换过程。这里指的计算机系统不仅包括PC机、笔记本电脑也包括手机、平板电脑、智能手表等智能设备。
\sphinxhref{https://zhuanlan.zhihu.com/p/26081162}{9}%
\begin{footnote}[623]\sphinxAtStartFootnote
\sphinxnolinkurl{https://zhuanlan.zhihu.com/p/26081162}
%
\end{footnote}


\paragraph{交互设计}
\label{\detokenize{chapter_knowledge/IXD_design:id4}}
交互设计,主要包含用户对系统的理解(即心智模型),为了提升系统的可用性或用户友好性而做的设计。

交互设计过程中,人作为核心要素应该被突出体现出来。

交互设计包括交互设计、界面设计、动效设计\sphinxhref{https://zhuanlan.zhihu.com/p/25942494}{6}%
\begin{footnote}[624]\sphinxAtStartFootnote
\sphinxnolinkurl{https://zhuanlan.zhihu.com/p/25942494}
%
\end{footnote}

交互设计\sphinxhref{https://t.qidianla.com/1174989.html}{1}%
\begin{footnote}[625]\sphinxAtStartFootnote
\sphinxnolinkurl{https://t.qidianla.com/1174989.html}
%
\end{footnote}是产品工作中的重要一节,一份优秀的交互设计文档可以高效地指导开发和测试展开工作,并减少不必要沟通,有利于提高产品的产出效率。\sphinxhref{http://www.woshipm.com/ucd/2294526.html}{2}%
\begin{footnote}[626]\sphinxAtStartFootnote
\sphinxnolinkurl{http://www.woshipm.com/ucd/2294526.html}
%
\end{footnote}

定义产品的视觉表现:包括用户界面(UI,User Interface)和交互设计(User
Interaction)和用户体验(User
Experience)。产品经理通常和UI设计师或交互设计及用户体验设计师一起完成产品设计工作。产出为低保真原型、UI视觉稿以及高保真原型。\sphinxhref{https://zhuanlan.zhihu.com/p/25796796}{5}%
\begin{footnote}[627]\sphinxAtStartFootnote
\sphinxnolinkurl{https://zhuanlan.zhihu.com/p/25796796}
%
\end{footnote}


\paragraph{交互设计师}
\label{\detokenize{chapter_knowledge/IXD_design:id5}}

\subparagraph{产品经理 VS 交互设计师}
\label{\detokenize{chapter_knowledge/IXD_design:vs}}\begin{itemize}
\item {} 
直白点:一个是写文档的,一个是画原型的。文档里面是用户故事,场景和用例描述。交互需要将这些文字性的需求描述转换为可视化的流程图,架构图,原型图和交互稿。\sphinxhref{https://www.zhihu.com/question/21015379/answer/182435115}{3}%
\begin{footnote}[628]\sphinxAtStartFootnote
\sphinxnolinkurl{https://www.zhihu.com/question/21015379/answer/182435115}
%
\end{footnote}

\item {} 
精细程度上:产品经理描述出大概的草图样式,交互会给出更贴近实际高保真原型图。产品和交互早期合作会一起设计概念草图,之后交互和视觉,程序等再完善为低保真,高保真的原型图。

\item {} 
职能分工上:产品经理收集整理分析需求,定义出产品版本的特性和系统边界。交互设计师根据已经确定的拆解人机交互模型,信息架构,框架流程,静态原型图,动态交互稿。

\item {} 
研发工作流上:产品是交互设计师的上游职能部门。产品经理通过需求分析找到\sphinxstylestrong{低成本高价值}的需求,输出需求文档,交互设计师解读文档,站在用户体验角度将其转化为可开发的原型交互文档。

\item {} 
用户体验要素上:产品经理偏战略层和范围层,交互设计师偏向于结构层和框架层。视觉设计师负责表现层。

\item {} 
负责对象上:产品经理向上负责,向老板负责。交互设计师向下负责,向用户负责。

\item {} 
组织关系上:同一个规模的组织范围内,产品管辖交互,交互管辖视觉。很少见视觉管辖交互,交互管辖产品。对交互很重视的公司,产品经理和交互设计师职能上可以平级。像阿里的高级交互顾问,那都是算计着KPI来决定哪条内容(需求)该摆放在哪个位置。

\item {} 
沟通上:后者结构化思维,前者形象化表达

\item {} 
工作参与度上:交互几乎是全程参与一个项目,产品经理只参与项目的前期工作即可,所有细碎的事务给交互来完成。交互要和产品经理一起进行早期的概念设计,自己独立绘制低保真原型,和视觉UI们一起设计高保真原型,必要的时候,再叫上程序员完成一个可展示的原型Demo。产品经理更多就参与前期的概念草图设计。

\item {} 
字迹风格上:产品都偏向于半个老板,字体都写得奔放潦草(看不懂!),交互的字迹都写得清晰工整。

\item {} 
页面上:产品经理在考虑背后的规则,而交互设计只是在考虑表面的逻辑。\sphinxhref{https://www.zhihu.com/question/21015379/answer/1365070268}{4}%
\begin{footnote}[629]\sphinxAtStartFootnote
\sphinxnolinkurl{https://www.zhihu.com/question/21015379/answer/1365070268}
%
\end{footnote}

\end{itemize}


\paragraph{UI}
\label{\detokenize{chapter_knowledge/IXD_design:ui}}
\sphinxstylestrong{UI,User Interface Design},用户交互设计。


\subparagraph{用户图像交互设计 GUI}
\label{\detokenize{chapter_knowledge/IXD_design:gui}}
界面如同机器的脸,在人机互动的过程中始终面对着我们的用户,起着十分重要的沟通作用。UI设计由UI设计师以PM所制作的原型图为蓝本进行艺术加工,其中包括色彩、icon、字体大小颜色间距等具体工作,所以这时候需要强调一点,那就是PM在原型设计时尽量不要带颜色,以免影响UI设计师的后续工作。所以,产品经理又被称为「产品狗」也是因为产品经理大部分时间都是在制作“黑白灰”三色的原型,可能是线框图,也可能是低保真原型图,但是如果有的公司有交互设计师,低保真原型图多由交互设计师辅助产品经理完成,因为低保真原型对细节的表达更加深入。

UI设计师一般需要了解心理学、设计学、语言学等方面的知识,更要掌握诸如界面一致性等设计原则。然而界面设计不仅仅是UI设计师的工作,PM在整个设计过程中要带领整个团队与UI设计师并肩战斗,但是平时工作中由于大家意见相左导致项目无限延期的事情屡见不鲜,所以此时PM的角色就显得更加重要,在UI设计师工作时要不断与其工作,沟通关于产品的功能、场景、用户、市场等信息,这样可以让UI设计师的脑海中有更多关于艺术设计的灵感,同时PM也应积极地促进UI设计师与前端工程师之间的沟通,以免UI设计在程序实现上出现无法预料的问题。\sphinxhref{https://zhuanlan.zhihu.com/p/26103663}{8}%
\begin{footnote}[630]\sphinxAtStartFootnote
\sphinxnolinkurl{https://zhuanlan.zhihu.com/p/26103663}
%
\end{footnote}

不要把交互设计片面理解成交互动效设计。交互动效仅仅是为了美化交互过程而做的设计,只是交互设计中的一部分,好的交互设计只能说是锦上添花,而并不是雪中送炭,而优秀的交互设计,是可以帮助用户优雅而又高效地完成所设想的任务,用户在整个过程中能感到愉悦而不受打扰。


\subparagraph{语音交互 VUI}
\label{\detokenize{chapter_knowledge/IXD_design:vui}}
机器学习促进了语音识别技术的发展,也促进发展了语音交互场景。AI语音交互的设计可能比手机/PC端的交互设计难很多,因为语音交互系统不是限定好的GUI操作界面,而是不便于规范且自由延展的自然语言。会话的开放性意味着
AI
交互设计者必须考虑用户可能采取的几乎无数的选择。要能够理解用户,了解他们的动机,然后合乎逻辑地思考如何引导他们完成一件事情。

阿里、谷歌、亚马逊语音交互设计规范\sphinxhref{https://www.yuque.com/weis/ai/qui8gs}{10}%
\begin{footnote}[631]\sphinxAtStartFootnote
\sphinxnolinkurl{https://www.yuque.com/weis/ai/qui8gs}
%
\end{footnote}


\paragraph{UE}
\label{\detokenize{chapter_knowledge/IXD_design:ue}}
用户体验(User
Experience,简写为UE),是用户在访问一个网站或使用一款产品时不仅包括感官更包括心理体验,比如\sphinxstylestrong{印象、感觉、成就感、舒适感},以及是否愿意再次体验或使用。而另外一个词UED(User\sphinxhyphen{}Experience
Design),即用户体验设计,就是指在进行产品设计、开发、维护时从用户的需求和用户的感受出发,以用户为中心进行产品设计、开发和维护,而不是让用户去适应产品本身。\sphinxhref{https://zhuanlan.zhihu.com/p/26035392}{7}%
\begin{footnote}[632]\sphinxAtStartFootnote
\sphinxnolinkurl{https://zhuanlan.zhihu.com/p/26035392}
%
\end{footnote}


\paragraph{具体工作中}
\label{\detokenize{chapter_knowledge/IXD_design:id6}}\begin{enumerate}
\sphinxsetlistlabels{\arabic}{enumi}{enumii}{}{.}%
\item {} 
产品将交互设计文档:ref:\sphinxcode{\sphinxupquote{UI\_design\_docs}}交给UI。

\item {} 
UI部门要设计首页风格,一般形成3\textasciitilde{}5套典型方案,然后提交2套方案部门内部最满意的方案给需求部门,经过多次过会后,最终定下一套UI定稿。需要注意的是,界面设计的评判标准即不是某个产品经理拍板的意见,也不是产品团队内部投票的结果,而应该是最终用户的感受。前后端交互\sphinxhref{https://vickydyy.github.io/2019/06/02/\%E5\%89\%8D\%E5\%90\%8E\%E7\%AB\%AF\%E4\%BA\%A4\%E4\%BA\%92/}{11}%
\begin{footnote}[633]\sphinxAtStartFootnote
\sphinxnolinkurl{https://vickydyy.github.io/2019/06/02/\%E5\%89\%8D\%E5\%90\%8E\%E7\%AB\%AF\%E4\%BA\%A4\%E4\%BA\%92/}
%
\end{footnote}、前端UI测试\sphinxhref{http://www.jfrcw.com/zhichang/215579.html}{13}%
\begin{footnote}[634]\sphinxAtStartFootnote
\sphinxnolinkurl{http://www.jfrcw.com/zhichang/215579.html}
%
\end{footnote}

\item {} 
UE(用户体验)部门开始针对原型进行操作上的优化调整,收集各类交互及用户体验方面的改善建议,比如“这个文字需要加下划线”、“主题颜色需要调整”等,过程中也可以邀请典型用户参与讨论。

\item {} 
经过多番修订,就可以定稿了,也就是传说中的视觉稿。

\end{enumerate}


\subsubsection{MVP}
\label{\detokenize{chapter_knowledge/MVP:mvp}}\label{\detokenize{chapter_knowledge/MVP::doc}}
传统的产品开发流程,先定义最终产品,然后逐步进行开发测试上线,这通常需要花费数月甚至数年的时间。当投入使用后,才知道产品是否和市场契合,用户是否喜爱。如果失败了,则浪费大量的人力物力财力。而MVP呢,倡导先开发一款具有基本功能和吸引力的产品,立即投入运行,视市场的反馈来决定是继续还是转型,有效地降低了风险、而且能及时获取用户反馈。

\begin{figure}[H]
\centering
\capstart

\noindent\sphinxincludegraphics{{MVP_build}.png}
\caption{MVP建造\sphinxhref{https://www.yuque.com/wuxinghua/01/uqgdgxs}{3}\sphinxfootnotemark[635]}\label{\detokenize{chapter_knowledge/MVP:id9}}\end{figure}
%
\begin{footnotetext}[635]\sphinxAtStartFootnote
\sphinxnolinkurl{https://www.yuque.com/wuxinghua/01/uqgdgxs}
%
\end{footnotetext}\ignorespaces 

\paragraph{原型与MVP 1\sphinxfootnotemark[636]}
\label{\detokenize{chapter_knowledge/MVP:mvp-1}}%
\begin{footnotetext}[636]\sphinxAtStartFootnote
\sphinxnolinkurl{https://www.jianshu.com/p/5b078398f632}
%
\end{footnotetext}\ignorespaces 
在《Managing Software Requirements:A Use Case
Approach》一书中,不止一次提到原型是应该用完即弃的,是个一次性的玩意儿。在日常项目中,也都是用Axure等工具进行高保真原型开发,而真实产品开发则是另起炉灶。

原型和产品隔离,一方面可以加速原型开发、激发创意,另一方面可以提高产品稳定性,因为避免了通过对原型进行修修补补来制造产品。而MVP呢?是\sphinxstylestrong{最小可行产品},也是第一代产品,是可以让用户使用的产品——尽管功能不丰富。原型背后的逻辑都是\sphinxstylestrong{软件模拟}的、是假的、用户是不能真正用其解决问题的,它只是看起来像真的而已。

斯坦福的一个创业团队打算在无人机上安装高清摄像头,拍摄(每一颗)农场作物的病害、施肥和灌溉情况。农场主可以根据采集并经过处理的数据来决定如何更好地播种,团队则可以通过销售数据来盈利。创业者们本想先购买无人机、超清摄像头、图像处理软件,然后花费数月时间来进行开发整合。但是Steve
Blank(Steve Blank和下文的Eric
Ries均为MVP推广者)的建议是:既然团队目标是想确定农场主是否愿意购买数据,而农场主并不关心数据是来自卫星、无人机或者魔法,那么,只需要租借一个手动控制的飞机模型、安装上普通相机,然后飞跃农场拍摄、手动处理数据,再验证农场主是否愿意为这些信息付费即可。

因此,在上文的两个例子中,Dropbox的视频是原型,Steve
Blank建议的遥控飞机是MVP。

\begin{figure}[H]
\centering
\capstart

\noindent\sphinxincludegraphics{{prototype_VS_MVP}.png}
\caption{原型和MVP比较}\label{\detokenize{chapter_knowledge/MVP:id10}}\end{figure}


\paragraph{最小可行性测试 2\sphinxfootnotemark[637]}
\label{\detokenize{chapter_knowledge/MVP:id1}}%
\begin{footnotetext}[637]\sphinxAtStartFootnote
\sphinxnolinkurl{https://www.zhihu.com/pub/reader/119980992/chapter/1284104623666458624}
%
\end{footnotetext}\ignorespaces 

\subparagraph{自然增长用户}
\label{\detokenize{chapter_knowledge/MVP:id2}}
自然增长就是抛开人工干预后的自然结果。用户增长的方法有很多,无论是直接广告推广、应用市场推广还是积分墙、资源互换、地推和拉身边的人脉,都是进行了人工干预的。

产品经理一定要为一款新产品预留出 1
个月左右的自然增长期,在这段时间,不进行任何的人工干预,只将应用上传至各大应用市场就足够了。接下来,产品经理要做的就是等待并关注自然增长的数据。

那么,为什么要这样做呢?

很简单,产品经理通过自然增长的数据可以明显看出一款产品的被需求程度。因为不进行人工干预,所有的用户增长量都来自应用市场,主要包括应用市场精品推荐(代表平台对于这款产品的认可)和用户自然搜索(代表用户基于某种需求而选择了这款产品)。两者相结合就能够体现出产品的被需求程度和产品的精良程度。


\subparagraph{主动分享用户}
\label{\detokenize{chapter_knowledge/MVP:id3}}
除了用户自然增长,还有一点是主动分享用户的数据,主要是指在不进行任何的人工干预的情况下,用户主动、自愿地进行分享的数据。这里一定不能依靠活动或一些奖励进行人工干预,一旦进行了人工干预,其背后的核心价值便无法区分了。当然,主动分享用户和自然增长用户一样,产品经理需要预留
1 个月的时间观测。

那么,主动分享用户又代表着什么呢?

应用市场官方将产品选入精品推荐,代表产品得到了某市场官方的认可,或者某市场官方刚好对此类产品有需求。而用户主动分享则代表用户主动下载并使用了这款产品以后,非常满意并认为这款产品远远超出了自己的预期。这也代表产品提供了超出预期的解决方案,为后续进行大规模的人工干预打下了良好的基础。

无论是自然增长用户的增加还是主动分享用户的增加,都在于产品品质已经过关;而产品品质过关的核心在于产品抓住了用户最核心的需求,从而提供了超出预期的解决方案。


\subparagraph{MVP必备模块 3\sphinxfootnotemark[638]}
\label{\detokenize{chapter_knowledge/MVP:mvp-3}}%
\begin{footnotetext}[638]\sphinxAtStartFootnote
\sphinxnolinkurl{https://www.yuque.com/wuxinghua/01/uqgdgxs}
%
\end{footnotetext}\ignorespaces 
MVP产品除了核心流程以外,还有几个必备模块,这里做一下简单的介绍


\subparagraph{便捷的反馈渠道}
\label{\detokenize{chapter_knowledge/MVP:id4}}
尽可能为用户在MVP产品内提供便捷的反馈机制,而不仅仅是微信群和QQ群。为什么用户发现问题了,最希望第一时间把自己的疑问或者不满反应出来,微信群和QQ群毕竟有滞后性;其次,微信群和αQ群接触的用户可数量不足,在群里活跃的可能就老是那几个人,一个个去私聊嘛,效率可能低。所以,我们在微信QQ以外还是结合套内部反馈机制比较好


\subparagraph{数据埋点不可少}
\label{\detokenize{chapter_knowledge/MVP:id5}}
MVP的目的就是要验证。因此相应的数据埋点也不可少(纸面原型和墨刀原型不能埋点)。但是简单开发的还是可以埋点的。但是传统的数据埋点方法耗时比较久,这里我推荐
growing
io,只需要把一段SDK代码埋入网页和应用中,产品经理和运营人员就可以快速自己埋点(也有类似产品,比如诸葛IO,神策,但是我没用过,不好评价)


\subparagraph{前期用户调研不可少}
\label{\detokenize{chapter_knowledge/MVP:id6}}
做MVP虽然从某种意义上可以理解为用户调硏的一种延伸,但是不能因为有了MVP就忽视了正式的用户调研。毕竟MVP也是要依赖于前期用户调研。


\subparagraph{更多}
\label{\detokenize{chapter_knowledge/MVP:id7}}

\subparagraph{MMP(Minimum Marketable Product)是什么? 4\sphinxfootnotemark[639]}
\label{\detokenize{chapter_knowledge/MVP:mmp-minimum-marketable-product-4}}%
\begin{footnotetext}[639]\sphinxAtStartFootnote
\sphinxnolinkurl{http://www.shinescrum.com/news\_and\_events/mvp}
%
\end{footnotetext}\ignorespaces 
MMP是将为客户提供的最小特性集合构成一个产品发布,
目标是抢占市场窗口,而且在发布后可以及早获取真实用户的反馈,为产品的迭代演进提供输入。MMP
(Minimum Marketable
Product)的思想是非常敏捷的,但那是在产品的Idea和商业模式得到验证后,开始启动产品研发的工作方法。

永远都不要期望一个版本里做尽可能多的需求,而是相反:以最少的特性先发布第一个版本,这个版本要体现产品的独特价值主张。这样的策略有以下几点优势:
\begin{enumerate}
\sphinxsetlistlabels{\arabic}{enumi}{enumii}{}{.}%
\item {} 
抢占市场先机

\item {} 
及早获取真实用户的反馈

\item {} 
依据第一版本的市场反馈,决策后续迭代的演进,从而减少不必要的需求开发,减少浪费。

\end{enumerate}

那么如何决定哪些特性在MMP里发布呢?
\begin{itemize}
\item {} 
首先依据核心价值主张,那些不在核心价值主张的特性,一定不在MMP里;

\item {} 
其次,对每个特性,问一个问题:如果没有它会怎么样?如果没有它用户可以完成基本场景的使用,那就可以没有。

\end{itemize}


\subparagraph{MDP}
\label{\detokenize{chapter_knowledge/MVP:mdp}}
MDP即Most Desirable
Product,最渴望的产品。MDP是为用户提供高价值、高满意度用户体验所必需的最基本产品,不只是功能上,还需要关注
UI、产品质量等。所以MDP产品一定是一个比较成熟的产品。

\begin{figure}[H]
\centering
\capstart

\noindent\sphinxincludegraphics{{MDP}.png}
\caption{MDP}\label{\detokenize{chapter_knowledge/MVP:id11}}\end{figure}


\paragraph{AI产品的MVP 5\sphinxfootnotemark[640]}
\label{\detokenize{chapter_knowledge/MVP:aimvp-5}}%
\begin{footnotetext}[640]\sphinxAtStartFootnote
\sphinxnolinkurl{http://www.woshipm.com/pmd/2817456.html}
%
\end{footnotetext}\ignorespaces 

\subparagraph{low\sphinxhyphen{}hanging fruit}
\label{\detokenize{chapter_knowledge/MVP:low-hanging-fruit}}
对于AI MVP来说,从最容易实现的目标(low\sphinxhyphen{}hanging
fruit)开始做起或许是不错的选择。\sphinxhref{https://www.itsiwei.com/26937.html}{6}%
\begin{footnote}[641]\sphinxAtStartFootnote
\sphinxnolinkurl{https://www.itsiwei.com/26937.html}
%
\end{footnote}


\subparagraph{API}
\label{\detokenize{chapter_knowledge/MVP:api}}
在MVP阶段,暂不用考虑数据安全等问题,完全可以直接调大厂的接口。


\subparagraph{Demo视频}
\label{\detokenize{chapter_knowledge/MVP:demo}}
B站视频的“朱茵的脸换成了杨幂的脸”证明了ZAO可行性。


\subparagraph{人工+智能}
\label{\detokenize{chapter_knowledge/MVP:id8}}
智能医生:找医生来冒充机器人,在线回复用户问题,首先要验证的是会不会有人来问机器人问题,会问到什么问题,会问到什么程度,试问机器人告诉你你的病要吃什么药,你敢吃吗?

当搜集到足够多的问题后,对问题进行统计,再来评估是用知识图谱还是知识库QA。


\subsubsection{业务说明(用例)}
\label{\detokenize{chapter_knowledge/service_analysis:id1}}\label{\detokenize{chapter_knowledge/service_analysis::doc}}

\paragraph{登录用例1\sphinxfootnotemark[642]}
\label{\detokenize{chapter_knowledge/service_analysis:id2}}%
\begin{footnotetext}[642]\sphinxAtStartFootnote
\sphinxnolinkurl{https://t.qidianla.com/1159980.html}
%
\end{footnotetext}\ignorespaces 
正常流程:输入手机号、输入密码,判断正误,登录框关闭,回到登录前页面,获取登录状态及用户信息显示。

异常流程:
\begin{enumerate}
\sphinxsetlistlabels{\arabic}{enumi}{enumii}{}{.}%
\item {} 
未输入手机号,登录按钮无效

\item {} 
未输入密码,登录按钮无效

\item {} 
手机号非11位,登录按钮无效

\item {} 
手机号获取焦点弹出输入框无法输入字母

\item {} 
手机号输入框超过11位数字后无法继续输入

\item {} 
输入手机号错误,点击登录,提示账号未注册

\item {} 
输入手机号正确,密码错误,点击登录,提示密码错误

\item {} 
输入手机号密码正确,网络异常,点击登录,提示网络异常,稍后再试

\item {} 
点击忘记密码,输入手机号未存在数据库,点击发送验证码,提示手机号未注册

\end{enumerate}


\paragraph{评论用例}
\label{\detokenize{chapter_knowledge/service_analysis:id3}}
正常流程:点击评论框获取焦点,弹出输入框,输入评论内容,点击发布,关闭输入框,刷新页面,更新评论。

异常流程:
\begin{enumerate}
\sphinxsetlistlabels{\arabic}{enumi}{enumii}{}{.}%
\item {} 
未输入内容,发布按钮失效

\item {} 
内容不足15字,发布按钮失效

\item {} 
内容超过140字不能继续输入

\item {} 
失去网络连接点击发布,提示网络异常,请稍后再试

\item {} 
网络延迟多次发布,提示同一用户不能重复发布相同内容评论

\item {} 
输入内容包含敏感词,使用*替代

\end{enumerate}


\subsubsection{策略分析}
\label{\detokenize{chapter_knowledge/strategy_analysis:id1}}\label{\detokenize{chapter_knowledge/strategy_analysis::doc}}

\paragraph{运营策略}
\label{\detokenize{chapter_knowledge/strategy_analysis:id2}}
运营手段,历史运营策略,粉丝活动等,

品牌策略,slogan变化,品牌形象包装变化等

分析后对其进行具体的品牌定位


\paragraph{产品策略}
\label{\detokenize{chapter_knowledge/strategy_analysis:id3}}
跟踪竞品版本,搜集个版本核心功能变化,版本介绍,引导页变化,分析归纳竞品策略变化


\paragraph{盈利模式}
\label{\detokenize{chapter_knowledge/strategy_analysis:id4}}
客户价值,企业与客户之间的链接逻辑

主要的盈利模式在哪里?横向分析竞品之间不同的赚钱方法


\subsubsection{项目管理}
\label{\detokenize{chapter_knowledge/project_manage:id1}}\label{\detokenize{chapter_knowledge/project_manage::doc}}

\paragraph{项目与产品的关系}
\label{\detokenize{chapter_knowledge/project_manage:id2}}
分而治之的软件需要通过集成方式整体交付,软件的生产过程也是一个多人、多组织协作的过程,也需要集成。把软件看成是一个产品,产品就有策划、研发、运营和退出各个阶段,每个阶段可能由不同的人或组织完成。软件的研发阶段就是一个一个项目的实施过程,包括立项、执行和完工。这样的过程组织起来,就是一条软件生产的流水线。从早期瀑布式的软件研发,到后来\sphinxstylestrong{敏捷研发过程、CMM,到目前风头正劲的DevOps},都是在解决软件生产流水线不同阶段的协作问题,敏捷针对软件定义、设计、构建(开发)阶段的协作,持续集成是构建(开发)与测试阶段的协作,持续交付是从定义阶段到部署(交付)阶段的协作。

\begin{figure}[H]
\centering
\capstart

\noindent\sphinxincludegraphics{{product_vs_project}.png}
\caption{项目与产品的关系\sphinxhref{http://www.uml.org.cn/ai/201707041.asp}{8}\sphinxfootnotemark[643]}\label{\detokenize{chapter_knowledge/project_manage:id18}}\end{figure}
%
\begin{footnotetext}[643]\sphinxAtStartFootnote
\sphinxnolinkurl{http://www.uml.org.cn/ai/201707041.asp}
%
\end{footnotetext}\ignorespaces 

\paragraph{定义}
\label{\detokenize{chapter_knowledge/project_manage:id3}}
PM能在产品的整个生命周期中,进行产品迭代的规划、各类资源的整合(协调团队各成员)以及生产进度的把控来保证在预算内按时交付产品。
\sphinxhref{https://www.iamxiarui.com/?p=1782}{2}%
\begin{footnote}[644]\sphinxAtStartFootnote
\sphinxnolinkurl{https://www.iamxiarui.com/?p=1782}
%
\end{footnote}

包括:
\begin{itemize}
\item {} 
确保资源投入

\item {} 
制定项目计划

\item {} 
根据计划跟踪项目发展

\item {} 
辨别关键路径

\item {} 
必要时争取追加投入

\item {} 
向主管报告项目进展情况\sphinxhref{https://zhuanlan.zhihu.com/p/25796796}{18}%
\begin{footnote}[645]\sphinxAtStartFootnote
\sphinxnolinkurl{https://zhuanlan.zhihu.com/p/25796796}
%
\end{footnote}

\end{itemize}


\paragraph{最简单的项目管理}
\label{\detokenize{chapter_knowledge/project_manage:id4}}
立项—>需求—>开发—>测试—>发布
\sphinxhref{https://quizlet.com/129588206/\%E4\%BA\%BA\%E4\%BA\%BA\%E9\%83\%BD\%E6\%98\%AF\%E4\%BA\%A7\%E5\%93\%81\%E7\%BB\%8F\%E7\%90\%86-\%E7\%AC\%94\%E8\%AE\%B0-flash-cards/s}{5}%
\begin{footnote}[646]\sphinxAtStartFootnote
\sphinxnolinkurl{https://quizlet.com/129588206/\%E4\%BA\%BA\%E4\%BA\%BA\%E9\%83\%BD\%E6\%98\%AF\%E4\%BA\%A7\%E5\%93\%81\%E7\%BB\%8F\%E7\%90\%86-\%E7\%AC\%94\%E8\%AE\%B0-flash-cards/s}
%
\end{footnote}

文档管理——流程管理——敏捷方法


\paragraph{整体流程}
\label{\detokenize{chapter_knowledge/project_manage:id5}}
\begin{figure}[H]
\centering
\capstart

\noindent\sphinxincludegraphics{{project_flow_chart}.png}
\caption{整体流程图\sphinxhref{https://t.qidianla.com/1175640.html}{16}\sphinxfootnotemark[647]}\label{\detokenize{chapter_knowledge/project_manage:id19}}\end{figure}
%
\begin{footnotetext}[647]\sphinxAtStartFootnote
\sphinxnolinkurl{https://t.qidianla.com/1175640.html}
%
\end{footnotetext}\ignorespaces 

\subparagraph{开发细节}
\label{\detokenize{chapter_knowledge/project_manage:id6}}
第一次内测→第二次内测→SIT内测→UAT测试→试运营→上线跟踪。
\sphinxhref{https://blog.csdn.net/zcl050505/article/details/112948498}{7}%
\begin{footnote}[648]\sphinxAtStartFootnote
\sphinxnolinkurl{https://blog.csdn.net/zcl050505/article/details/112948498}
%
\end{footnote}
\begin{enumerate}
\sphinxsetlistlabels{\arabic}{enumi}{enumii}{}{.}%
\item {} 
第一次内测,主要由研发人员自己开发并验证;

\item {} 
第二次内测,主要是研发提交给测试人员进行测试,并将bug打回,让研发人员重新返工;

\item {} 
SIT测试,系统集成测试 (System Integration Testing )
,由于不同模块是由不同人开发,当他们汇总起来,有可能出现问题,需要进行检测;

\item {} 
UAT测试,用户验收测试(User Acceptance
Test),也就是用户可接受测试,这个阶段基本上都是由产品经理主导,用户最开始拿到产品是不会使用的,这时候就需要产品经理当着用户的面去演示。

\item {} 
试运营,这个环节也有产品经理参与,当产品上线之后,找一两个核心人员去用一用,觉得没有问题后小面积用户使用确保产品没有问题。慢慢替代原有业务,把原来的业务淘汰使用现在的产品。

\item {} 
产品经理首先得制作操作手册,培训用户使用,主要教会对面负责人,让他们自己去使用,了解产品。

\item {} 
上线跟踪

\item {} 
产品上线后,要准备相关资料,真正的\sphinxstylestrong{外包项目靠售后服务挣钱}。

\item {} 
以后有任何小需求小功能都会找你,卖给他们的软件售后问题都可以找你,对方公司直接给\sphinxstylestrong{售后费},后期服务对方,所以这种产品赚钱在售后方面。

\end{enumerate}


\paragraph{VS Project Manager}
\label{\detokenize{chapter_knowledge/project_manage:vs-project-manager}}
产品和开发的是同样的团队和同样的人,但在驱动产品和驱动项目这两件事情上,最好还是有所差别。至少产品更关注的是产品、功能、方向和反馈;而项目则更关注进度、质量和测试等。
\begin{itemize}
\item {} 
做好评估。几乎所有项目最终未按计划执行,其最根本原因就是在项目开始阶段,没有对需求、技术、产品有足够充分的了解,也就没有后续开发中的可控力度。高估和低估都是有问题的,所以我们常用的做法就是非常重视前期的评估,宁愿多花时间,并且对有模糊边界或者有挑战的问题,留足buffer。

\item {} 
将计划落实到可执行的单元和可执行的人。有了评估,然后就是将计划落实到足够力度的任务,以任务驱动开发过程,任务落实到责任人,任务要标明截止日期。在此,通过一定的工具来管理,是十分必要而可控进度的。例如我们基于自主产品PingCode
的任务驱动方式,就可以很好的将开发计划落实到任务和可执行的人,以直观的方式来告诉负责人项目整体的状态、执行者的情况、被delay的事情有哪些。总之,工具的辅助需要团队开发想法的驱动。(再多说一句:PingCode不仅可以进行以上的项目管理,还覆盖了项目、任务、需求、缺陷、迭代规划、测试、目标管理的研发管理全流程。)

\end{itemize}


\paragraph{项目立项 15\sphinxfootnotemark[649]}
\label{\detokenize{chapter_knowledge/project_manage:id7}}%
\begin{footnotetext}[649]\sphinxAtStartFootnote
\sphinxnolinkurl{https://zhuanlan.zhihu.com/p/92156981}
%
\end{footnotetext}\ignorespaces \begin{itemize}
\item {} 
目标是什么?极为关键,见BRD

\item {} 
背景是什么?即“为什么做” 交代清楚项目背景,见BRD

\item {} 
内容和范围是什么?即“要做什么”了解项目边界的,见BRD

\item {} 
结果或KPI?即“如何证明做到了”
KPI:问题发生率的降低?成本的下降?收益的增加?

\item {} 
执行的基本思路和方案?即“怎么做”:见MRD,让评审人能知道未来的执行方向和策略

\item {} 
里程碑\&计划:拆解项目的关键节点,并用项目管理软件分解好每个里程碑下的工作计划

\item {} 
职责与分工:明确产品方案、分工职责、项目排期、 Review 周期
\sphinxhref{https://www.iamxiarui.com/?p=1369}{4}%
\begin{footnote}[650]\sphinxAtStartFootnote
\sphinxnolinkurl{https://www.iamxiarui.com/?p=1369}
%
\end{footnote}

\item {} 
资源需求:人,财,物(服务器的资源?硬件的资源?市场的资源?)

\item {} 
风险控制:分析可能失败的原因中提前控制、不可控的因素

\end{itemize}


\paragraph{制定项目计划}
\label{\detokenize{chapter_knowledge/project_manage:id8}}
明确好目标之后,就可以根据具体的,可量化的方案组织相关的干系人来评估工作量,根据工作量倒排项目计划表,将目标拆解到更小的时间颗粒度,并指定相关责任人进行任务跟进,如下图所示。

\begin{figure}[H]
\centering
\capstart

\noindent\sphinxincludegraphics{{project_plan}.png}
\caption{项目计划}\label{\detokenize{chapter_knowledge/project_manage:id20}}\end{figure}

在这个阶段需要明确各个环节的交付产物,并识别可能的项目风险,提前制定风险应对计划,例如本公司缺乏某方面的数据,需要从外部获取,或者相关人员配置不足,需要招聘或借调人力资源的支持等等。在项目进行的过程中持续监控,以确保项目的正常进行。\sphinxhref{https://www.jianshu.com/p/fb2fbd4f1e06}{14}%
\begin{footnote}[651]\sphinxAtStartFootnote
\sphinxnolinkurl{https://www.jianshu.com/p/fb2fbd4f1e06}
%
\end{footnote}


\paragraph{周报 6\sphinxfootnotemark[652]}
\label{\detokenize{chapter_knowledge/project_manage:id9}}%
\begin{footnotetext}[652]\sphinxAtStartFootnote
\sphinxnolinkurl{https://www.zhihu.com/pub/reader/119980992/chapter/1284104645191839744}
%
\end{footnotetext}\ignorespaces 
写好周报的重点在于方法论建设,知识传承的核心在于能够真正用到实际工作中且可以快速复制。而周报本身就是基于自身工作的且与团队成员有极强的契合度,因此更容易引起共鸣,产生实际的价值。产品经理切记,从入职的第一天起就要启动周报的知识传承。下面为大家提供一个周报模板。

\begin{figure}[H]
\centering
\capstart

\noindent\sphinxincludegraphics{{week_report}.jpg}
\caption{周报模板}\label{\detokenize{chapter_knowledge/project_manage:id21}}\end{figure}

这周的工作内容是推进A项目,下周还是推进A项目。如果项目一直被卡住,等到了写周报的时候似乎就没得可写了。总不能一直写“规划某个需求”,“设计某个产品”,“推动某个项目进度”,“整理某个产品的文档和资料”等……\sphinxhref{http://dadaghp.com/index/index/article\_detail/id/665.html}{17}%
\begin{footnote}[653]\sphinxAtStartFootnote
\sphinxnolinkurl{http://dadaghp.com/index/index/article\_detail/id/665.html}
%
\end{footnote}


\paragraph{开发看板}
\label{\detokenize{chapter_knowledge/project_manage:id10}}
\begin{figure}[H]
\centering
\capstart

\noindent\sphinxincludegraphics{{display_board}.png}
\caption{开发看板\sphinxhref{https://www.jianshu.com/p/266834df1808}{12}\sphinxfootnotemark[654]}\label{\detokenize{chapter_knowledge/project_manage:id22}}\end{figure}
%
\begin{footnotetext}[654]\sphinxAtStartFootnote
\sphinxnolinkurl{https://www.jianshu.com/p/266834df1808}
%
\end{footnotetext}\ignorespaces 
在开发需求的过程中,各需求的相关人不用再去寻找邮件,或者翻看电脑保存的文档。

每个人都可以通过看板,看到每个需求的实时状态。每个人都可以去拖动卡片。提前预知自己的工作量。比如,测试工程师就可以通过看板,大概预知有多少卡片在待测试状态,从而预估自己的工作量。


\paragraph{算法项目管理 6\sphinxfootnotemark[655]}
\label{\detokenize{chapter_knowledge/project_manage:id11}}%
\begin{footnotetext}[655]\sphinxAtStartFootnote
\sphinxnolinkurl{https://www.zhihu.com/pub/reader/119980992/chapter/1284104645191839744}
%
\end{footnotetext}\ignorespaces 
算法项目不同于应用研发项目,研发功能上在一定周期的配合后,团队内部对于\sphinxstylestrong{可行性方案、研发周期、最小可执行单元}会逐渐达成共识,项目评估后可以有明确的启动和结束节点。

算法项目更多依赖算法团队的学术和工程能力,对于算法团队未执行过的领域项,算法同学一方面会对数据提出过高的要求,另一方面对于可执行的效果情况也往往不能作出很好的评估。业务的时效性和算法的可行性上往往存在着代沟


\paragraph{项目管理知识 10\sphinxfootnotemark[656]}
\label{\detokenize{chapter_knowledge/project_manage:id12}}%
\begin{footnotetext}[656]\sphinxAtStartFootnote
\sphinxnolinkurl{https://zhuanlan.zhihu.com/p/192633890}
%
\end{footnotetext}\ignorespaces 
软件工程推进过程中,项目管理相关的技能方法与工具运用也非常的关键。其中各种研发流程与规范,例如敏捷开发,设计评审,代码评审,版本管控,任务看板管理等,都是实际项目推进中非常重要的知识技能点。这方面推荐学习一本经典的软件工程教材《构建之法》,了解软件项目管理的方方面面。进一步来说广义的项目管理上的很多知识点也是后续深入学习的方向,可以参考极客时间上的课程《项目管理实战20讲》。


\subparagraph{项目延期怎么解决的?怎么避免项目延期? 13\sphinxfootnotemark[657]}
\label{\detokenize{chapter_knowledge/project_manage:id13}}%
\begin{footnotetext}[657]\sphinxAtStartFootnote
\sphinxnolinkurl{http://www.woshipm.com/pmd/1642415.html}
%
\end{footnotetext}\ignorespaces 

\subparagraph{项目延期怎么处理?}
\label{\detokenize{chapter_knowledge/project_manage:id14}}\begin{enumerate}
\sphinxsetlistlabels{\arabic}{enumi}{enumii}{}{.}%
\item {} 
了解情况:了解不能按时上线的原因及需要延期多长时间上线;

\item {} 
知会需求方:确定影响范围,知会需求方,讨论解决方案;

\item {} 
确定解决方案:协调资源加班解决;或者砍掉非核心功能保障上线时间点。

\end{enumerate}


\subparagraph{怎么避免项目延期?}
\label{\detokenize{chapter_knowledge/project_manage:id15}}\begin{enumerate}
\sphinxsetlistlabels{\arabic}{enumi}{enumii}{}{.}%
\item {} 
保障产品文档质量,避免需求变更导致延期,前期沟通评审功课做足;

\item {} 
合理评估时间,设置任务里程碑,每天过进度,及时发现风险点;

\item {} 
建立一个内部网络空间,所有文档资源统一存放,供团队成员共享;

\item {} 
利用即时聊天工具,建立一个项目群,每天通报项目进度;建立项目邮件组,所有变更达成一致后,发送邮件确认;

\item {} 
制定制度和标准,实行奖惩措施,制度和标准及其重要。

\end{enumerate}


\subparagraph{自我考核}
\label{\detokenize{chapter_knowledge/project_manage:id16}}
在某个负责项目中运用项目管理方法,完成一个实际的需求评估,项目规划,设计与评审,开发执行,项目上线,监控维护流程,并对整个过程做复盘总结。


\paragraph{AI相关}
\label{\detokenize{chapter_knowledge/project_manage:ai}}
AI算法→AI工程→AI服务→AI应用
\sphinxhref{https://www.zhihu.com/question/363069393}{9}%
\begin{footnote}[658]\sphinxAtStartFootnote
\sphinxnolinkurl{https://www.zhihu.com/question/363069393}
%
\end{footnote}
\begin{itemize}
\item {} 
AI算法:就是大家平常理解的算法模型。

\item {} 
AI工程:因为解决一个实际问题中会涉及到多个模型的处理,多个模型之间如何协调、调用就是AI工程化要做的事情。(以CV为例,可能会涉及到去噪、编码、几何变换、增强、边缘检测、图像分割、特征提取等等)

\item {} 
AI服务:把AI工程化的输出能力以服务的方式提供出来,比如大家常见的公有云API调用的方式、Docker私有化容器部署等。(服务形式会根据提供的AI能力会有所不同,比如百度的OCR通用识别模型,还会带有标注、训练等功能)

\item {} 
AI应用:利用已有的AI服务能力结合实际的业务场景,输出用户价值。

\end{itemize}

我们需将AI部署作为一个有生命的系统,与传统的IT项目相反,人工智能项目是活生生的、会呼吸的解决方案。它们在部署的几乎任何阶段都不是静态的或不可预测的——特别是在新奇的用例中。\sphinxhref{https://emerj.com/ai-executive-guides/3-phases-of-ai-deployment/}{11}%
\begin{footnote}[659]\sphinxAtStartFootnote
\sphinxnolinkurl{https://emerj.com/ai-executive-guides/3-phases-of-ai-deployment/}
%
\end{footnote}

生物学类比:发育阶段——并非所有新生动物都能达到成年。动物竞争、食物短缺和环境变化对生存构成了挑战。一些动物成年后仍然需要食物和住所,但不再需要父母的持续关注。成年期可以被视为人工智能部署的第三个阶段;一旦到达,AI项目仍然需要基本资源,以保持正常运作,直到项目结束。

构建一个人工智能应用程序不是“即插即用”,而是指照顾一个有生命的东西,伴随着不断增长和各种各样的需求。

换句话说:
\begin{itemize}
\item {} 
人工智能不是IT

\item {} 
人工智能是概率性的,而不是确定性的

\item {} 
人工智能更像是研发,而不是软件

\item {} 
开发一个人工智能应用程序就是让生命成长并照顾一个有生命、有数据支持的有机体

\end{itemize}


\subparagraph{AI阶段}
\label{\detokenize{chapter_knowledge/project_manage:id17}}
\begin{figure}[H]
\centering
\capstart

\noindent\sphinxincludegraphics{{AI_3phases}.png}
\caption{企业AI部署的三个阶段}\label{\detokenize{chapter_knowledge/project_manage:id23}}\end{figure}

一个人工智能应用程序要经历三个阶段。与数据科学生命周期不同,应用程序在部署阶段中向后移动是不常见的,而且——理想情况下——这些阶段不是循环的,而是线性地进行。

对于人工智能项目领导者来说,完全理解使一个人工智能项目变成现实所涉及的技术细微差别并不重要——但对于项目领导者来说,在进入一个人工智能项目时,他们应该期待自己能够小心地通过所有这三个阶段,这才是重要的。过程的各个阶段都有独特的里程碑和目标,并允许项目团队避免匆忙集成的风险,并且有足够的时间进行迭代和修补应用程序。缺乏这个框架的团队将很难评估完成一个项目所需的努力,也很难看到项目顺利部署。


\subparagraph{概念证明(PoC)}
\label{\detokenize{chapter_knowledge/project_manage:poc}}
目标——确定人工智能是否能在一个业务功能中(通常是在一个有历史和测试数据的“沙箱”环境中)提供特定的好处。

挑战\sphinxhyphen{}决定正确的数据和特性来训练。选择具有足够高的潜在业务价值的问题。

在一个孤立的“沙箱”环境中进入下一阶段的标准(应用程序证明它能够显示有希望的结果——达到一些预先定义的成功标准)。

例子1
\sphinxhyphen{}电子商务公司采用产品推荐引擎。假设孵化期有了丰硕的成果,团队接受了新的人工智能相关流程的培训——推出推荐引擎作为默认用户体验,有大量团队致力于测试和调整推荐算法,并收集对结果的反馈。

例2
\sphinxhyphen{}一家采用预测分析应用程序的制造公司。假设孵化期取得了丰硕的成果,团队接受了与人工智能相关的新流程的培训——在所有特定类型的机器上安装传感器和连接器,创建一套中央仪表板来监控机器,并由专职人员维护和改进它们。

对于人工智能项目领导者来说,完全理解使一个人工智能项目变成现实所涉及的技术细微差别并不重要——但对于项目领导者来说,在进入一个人工智能项目时,他们应该期待自己能够小心地通过所有这三个阶段,这才是重要的。过程的各个阶段都有独特的里程碑和目标,并允许项目团队避免匆忙集成的风险,并且有足够的时间进行迭代和修补应用程序。缺乏这个框架的团队将很难评估完成一个项目所需的努力,也很难看到项目顺利部署。

TODO:
\sphinxurl{https://easyai.tech/blog/ai-executive-guides-data-science-lifecycle/}


\paragraph{More}
\label{\detokenize{chapter_knowledge/project_manage:more}}
\begin{figure}[H]
\centering
\capstart

\noindent\sphinxincludegraphics{{Project_management_mindmap}.png}
\caption{项目管理图}\label{\detokenize{chapter_knowledge/project_manage:id24}}\end{figure}


\subsubsection{沟通、协作、项目推动能力}
\label{\detokenize{chapter_knowledge/collaborate:id1}}\label{\detokenize{chapter_knowledge/collaborate::doc}}
人工智能的应用模式是对产品功能的赋能,因此在推进产品上线的过程中,跨团队、跨部门沟通是一种常态,有时候因为数据分散在各个业务模块,这时就更考验AI产品经理凝聚团队的能力,以及清楚表达项目价值的能力。

有的AI产品经理的日常工作就是需要和多方沟通、协调各方资源、达成产品目标。有人说“在产品的需求评审的时候,能够应答如流”是AI产品经理个人的荣耀时刻,但“顺利完成需求评审,跟进产品上线、把握项目进度,产品得到用户的认可”这个过程才是整个团队最为骄傲的时刻。AI产品经理的工作不仅表现出AI产品经理具备表达能力、逻辑能力和执行力,表现出跟进产品开发上线过程中的为人处事的能力,其实也是其个人人格魅力的很好体现。

对于产品经理来说,人工智能的产品应用显然会比一个普通的功能设计复杂得多,从项目初期的数据评估、设定项目的目标、可行性方案跟踪,到项目中期的研发、设计、测试、协调,到最后项目上线的效果评估,每一个环节都考验着AI产品经理沟通、协作和项目推动的能力。\sphinxhref{https://weread.qq.com/web/reader/40632860719ad5bb4060856ka5732aa0226a5771bce9dc4}{1}%
\begin{footnote}[660]\sphinxAtStartFootnote
\sphinxnolinkurl{https://weread.qq.com/web/reader/40632860719ad5bb4060856ka5732aa0226a5771bce9dc4}
%
\end{footnote}


\paragraph{理解和协调全局}
\label{\detokenize{chapter_knowledge/collaborate:id2}}
常有新人产品经理认为自己的工作职责是战略部署,是顶端规划,至于其他部门的工作人员如技术、设计、运营、市场、销售,做的都是填充性工作,完成细枝末节罢了。

然而,事情并非这样。残忍的现实是:产品需要你时,你必须什么都会,什么岗位都担当,大部分时间内,在其他人眼里,产品经理什么都不是,什么也不会,不少新人在洋洋自得和趾高气扬时,忘却了职场斗争和国内职业现状,把时间消耗在与其他部门争吵不休上,从而在自嗨中既无法专精,也无法跨界,更别谈做个复合型人才了。

你要知道的是:没有程序员,所有需求都是无从谈起的,没有 UI
和平面,交互和美观的体验都会是极差的,没有市场部,产品就难以被人认可并使用,没有运营,用户也无法留存。

所以,在提交方案后,依旧不可掉以轻心,自我修炼的环节才刚刚开始,硬实力让你的产品得以生存,软实力则让你的产品发扬光大。在该环节你要做的是:与各部门的各工作人员耐心沟通,协调好他们之间的关系,达成你的目的。

倘若其他部门间因你的需求产生意见不合,你也不可袖手旁观,千万不要认为麻烦没找到你就随便他们互相之间发生争执。要知道,争执的结果是拖延,最终影响的是你的进度。

身为产品经理,你要思索的不仅仅是手头上的
KPI。其实,其他部门的指标也是你的指标:设计美、交互强、增速快、留存高、盈利明显……这些看似与你无关的
KPI,都是你的 KPI。


\subsubsection{数据分析}
\label{\detokenize{chapter_knowledge/data_analysis:data-analysis}}\label{\detokenize{chapter_knowledge/data_analysis:id1}}\label{\detokenize{chapter_knowledge/data_analysis::doc}}
数据分析能力,基本的Excel、SPSS、R语言等工具只是锦上添花,更重要的是,产品经理是否有数据意识,是否针对思考的问题(例如要解决什么问题,为什么要解决这个问题,解决这个问题可能带来的收益,方案上线后问题解决的效果等),去寻求相应的数据支撑,以数据分析的结果为重要参考来解决。举个例子,用户的埋点数据跟踪,能够帮助我们了解用户实际使用产品的路径,通过判断其是否与我们预期一致,便能为该功能的后续优化提供很好的参考。
\sphinxhref{https://zhuanlan.zhihu.com/p/353225318}{6}%
\begin{footnote}[661]\sphinxAtStartFootnote
\sphinxnolinkurl{https://zhuanlan.zhihu.com/p/353225318}
%
\end{footnote}

AI产品经理同样要具备数据产品的能力,因为数据也是AI的三大要素之一。需要熟练使用sql和hive相关数据查询语言。对于一个AI产品经理来说,使用sql和hive是基本,因为这两门语言学习的门槛相对降低,自学都可以很快上手。通过数据分析可以快速找到后续模型效果优化和策略调整的方向,有据可依。对于使用Tableau这种,实话实说真的是小儿科,这种可视化的界面查询只能做一些简单的结果分析和精美图表,对于更好的策略调整和模型效果优化必须进行数据库查询。对于AI产品经理来说,仅仅只会数据分析也不够,还需要很强的数据sense才可以。
\sphinxhref{https://www.zhihu.com/question/57815929/answer/1730120649}{8}%
\begin{footnote}[662]\sphinxAtStartFootnote
\sphinxnolinkurl{https://www.zhihu.com/question/57815929/answer/1730120649}
%
\end{footnote}


\paragraph{数据建模过程}
\label{\detokenize{chapter_knowledge/data_analysis:id2}}
数据采集、SQL数据库、Python、统计学、数据分析核心模块、可视化、报告撰写,更多查看\sphinxhref{https://www.huaweicloud.com/articles/5aa4be414ac9d338001dd7112587e496.html}{13}%
\begin{footnote}[663]\sphinxAtStartFootnote
\sphinxnolinkurl{https://www.huaweicloud.com/articles/5aa4be414ac9d338001dd7112587e496.html}
%
\end{footnote}

\begin{figure}[H]
\centering
\capstart

\noindent\sphinxincludegraphics{{data_model}.png}
\caption{数据建模过程}\label{\detokenize{chapter_knowledge/data_analysis:id19}}\end{figure}


\paragraph{数据分析-》商业价值 7\sphinxfootnotemark[664]}
\label{\detokenize{chapter_knowledge/data_analysis:id3}}%
\begin{footnotetext}[664]\sphinxAtStartFootnote
\sphinxnolinkurl{https://quizlet.com/129588206/\%E4\%BA\%BA\%E4\%BA\%BA\%E9\%83\%BD\%E6\%98\%AF\%E4\%BA\%A7\%E5\%93\%81\%E7\%BB\%8F\%E7\%90\%86-\%E7\%AC\%94\%E8\%AE\%B0-flash-cards/}
%
\end{footnotetext}\ignorespaces 
足够熟悉产品-方向性假设——提取应用数据分析-得到(未留意)现象-解释现象——用户调研修正解释——指导产品发展方向


\paragraph{数据分析的四种模式 11\sphinxfootnotemark[665]}
\label{\detokenize{chapter_knowledge/data_analysis:id4}}%
\begin{footnotetext}[665]\sphinxAtStartFootnote
\sphinxnolinkurl{https://www.zhihu.com/people/terrypm/posts}
%
\end{footnotetext}\ignorespaces \begin{enumerate}
\sphinxsetlistlabels{\arabic}{enumi}{enumii}{}{.}%
\item {} 
描述性分析(Descriptive Analytics),即将已经发生事实用数据表述出来。

\item {} 
诊断性分析(Diagnostic
Analytics),即回答为什么会发生,通常使用数据钻取的手段就可实现。

\item {} 
预测性分析(Predictive
Analytics),即通过历史数据对未来的趋势进行预测。这个阶段会引入一些高级算法。

\item {} 
决策建议性分析(Prescriptive
Analytics),即通过分析可能影响行为结果的动态指标(或行为)并将指标和结果的关联关系进行量化,从而给出对结果产生最重要影响的指标,以及对应每个指标对结果产生不同影响程度的描述。有了以上这些分析,决策者可以将数据驱动决策真正落地。

\end{enumerate}

\begin{figure}[H]
\centering
\capstart

\noindent\sphinxincludegraphics{{data_drive}.png}
\caption{数据驱动}\label{\detokenize{chapter_knowledge/data_analysis:id20}}\end{figure}


\paragraph{分析方法}
\label{\detokenize{chapter_knowledge/data_analysis:id5}}
\begin{figure}[H]
\centering
\capstart

\noindent\sphinxincludegraphics{{data_analysis}.png}
\caption{数据分析}\label{\detokenize{chapter_knowledge/data_analysis:id21}}\end{figure}


\subparagraph{决策支持}
\label{\detokenize{chapter_knowledge/data_analysis:id6}}
决策支持是通过简单的求和以及易于理解的分析模型,帮助用户做出决策,比如对比本月同比和环比用户平均消费金额,从而决定通过什么决策活动来提高本月的用户平均消费金额。比如建立一个广告投入因素和新增用户的关系模型,就能够预测投入多少广告额,能带来多少新增用户。

简单的关系模型产品经理是能通过Excel表格分析出来的,如柱状图、折线图等。

如果一项因素引发问题的因素很复杂,则需要建立一个由多个因素组成的预测模型。通过这个模型,我们可以观察模型中某个因素对整体结果造成的影响。预测模型需要用到的统计方法有交叉列表统计、统计学假设检验
、多元回归分析等,这个阶段大部分产品经理都需要求助数据分析师的帮助了。

使用你收集的额外信息——比如点击率、页面停留时间、搜索历史和产品偏好——来了解用户在做什么,并帮助他们了解为什么你是最佳选择。
\sphinxhref{https://www.jianshu.com/p/a57a927a112b}{9}%
\begin{footnote}[666]\sphinxAtStartFootnote
\sphinxnolinkurl{https://www.jianshu.com/p/a57a927a112b}
%
\end{footnote}

\begin{figure}[H]
\centering
\capstart

\noindent\sphinxincludegraphics{{index_analysis}.png}
\caption{指标分析\sphinxhref{https://www.zhihu.com/question/412642864/answer/1784905476}{12}\sphinxfootnotemark[667]}\label{\detokenize{chapter_knowledge/data_analysis:id22}}\end{figure}
%
\begin{footnotetext}[667]\sphinxAtStartFootnote
\sphinxnolinkurl{https://www.zhihu.com/question/412642864/answer/1784905476}
%
\end{footnotetext}\ignorespaces 

\subparagraph{系统优化}
\label{\detokenize{chapter_knowledge/data_analysis:id7}}
系统优化指的是帮助用户构建让计算机执行的方案算法,常用的系统优化方法有机器学习。

相比简单模型的决策模型,系统通过机器学习方法分析出系统中更详细的因素,比如系统优化能分析出广告投入多少金额,能带来新用户的快速增长,以及广告投放中具体什么投放渠道,效果最好。

机器学习的优势在于能从数据中学习出其本身包含的模式和规律,并以此来建立模型。比今日头条,就是通过分析我们过去浏览的记录,利用机器学习建立模型,从而给我们推荐类似的内容。系统优化用到的统计方法有逻辑回归分析、聚类、主成分分析、决策树分析等。


\paragraph{经营分析}
\label{\detokenize{chapter_knowledge/data_analysis:id8}}
用户画像系统的标签数据通过API进入分析系统后,可以丰富分析数据的维度,支持进行多种业务对象的经营分析。


\paragraph{数据类型:}
\label{\detokenize{chapter_knowledge/data_analysis:id9}}\begin{itemize}
\item {} 
用户数据分析

\item {} 
活动数据分析

\item {} 
流量数据分析

\item {} 
销售数据分析

\item {} 
内容数据分析

\item {} 
商品数据分析

\item {} 
订单数据分析

\item {} 
渠道数据分析

\end{itemize}


\subparagraph{用户数据分析}
\label{\detokenize{chapter_knowledge/data_analysis:id10}}
留存、活跃、新增这个优先度递减。

用户数量:
\begin{itemize}
\item {} 
新用户数

\item {} 
老用户数

\item {} 
新/老用户数量比;

\end{itemize}

用户质量:
\begin{itemize}
\item {} 
新增用户:第一次启动应用的用户;

\item {} 
每日新增用户 DNU(Daily New Users):每日应用中的新登入用户数量

\item {} 
新增独立用户:全体应用的新增用户的总和(去重)

\item {} 
活跃用户 AU(Active
Users):当天启动一次的用户即为活跃用户,含新用户和老用户;

\item {} 
活跃独立用户:当天应用的活跃用户总和(去重)

\item {} 
DAU:DAU(Daily Active
User)日活跃用户数量。常用于反映网站、互联网应用或网络游戏的运营情况。

\item {} 
MAU:MAU(monthly active users)月活跃用户人数。

\item {} 
用户参与度

\item {} 
沉睡

\end{itemize}

这些用户价值指标,会导向一个最终的产品指标——付费用户:
\begin{itemize}
\item {} 
客单价

\item {} 
PU ( Paying User):付费用户

\item {} 
APA(Active Payment Account):活跃付费用户数

\item {} 
ARPU(Average Revenue Per User) :平均每用户收入

\item {} 
ARPPU (Average Revenue Per Paying User): 平均每付费用户收入

\end{itemize}


\subparagraph{AARRR模型}
\label{\detokenize{chapter_knowledge/data_analysis:aarrr}}\begin{itemize}
\item {} 
Acquisition\sphinxhyphen{}如何获取用户

\item {} 
Activation\sphinxhyphen{}如何提高用户活跃度

\item {} 
Retention\sphinxhyphen{}如何提升用户留存率

\item {} 
Revenue\sphinxhyphen{}如何提高收入

\item {} 
Refer\sphinxhyphen{}如何引导用户推荐产品给其他人

\end{itemize}

\begin{center}\sphinxincludegraphics{{AARRR}.jpg}\end{center}
摆地摊用法\sphinxhref{http://www.shuahuangpu.com/articles/110936.html}{10}%
\begin{footnote}[668]\sphinxAtStartFootnote
\sphinxnolinkurl{http://www.shuahuangpu.com/articles/110936.html}
%
\end{footnote}


\subparagraph{渠道数据分析}
\label{\detokenize{chapter_knowledge/data_analysis:id11}}
用户活跃:
\begin{itemize}
\item {} 
活跃用户:UV、PV

\item {} 
新增用户:注册量、注册同环比

\end{itemize}

用户质量:

留存:次日/7日/30日留存率
\begin{itemize}
\item {} 
用户留存率:在互联网行业中,用户在某段时间内开始使用应用,经过一段时间后,仍然继续使用该应用的用户,被认作是留存用户。这部分用户占当时新增用户的比例即是留存率,会按照每隔1单位时间(例日、周、月)来进行统计。

\end{itemize}

用户留存率中的40\sphinxhyphen{}20\sphinxhyphen{}10法则:如果你想让游戏、应用的DAU超过100万,那么日留存率应该大于40\%,周留存率和月留存率分别大于20\%和10\%。
\begin{itemize}
\item {} 
次日留存率:(当天新增的用户中,在往后的第1天还活跃的用户数)/第一天新增总用户数;

\item {} 
第2日留存率:(第一天新增用户中,在往后的第2天还有活跃的用户数)/第一天新增总用户数;

\item {} 
第7日留存率:(第一天新增的用户中,在往后的第7天还有活跃的用户数)/第一天新增总用户数;

\item {} 
第30日留存率:(第一天新增的用户中,在往后的第30天还有活跃的用户数)/第一天新增总用户数。

\end{itemize}

渠道收入:
\begin{itemize}
\item {} 
订单:订单量、日均订单量、订单同环比

\item {} 
营收:付费金额、日均付费金额、金额同环比

\item {} 
用户:人均订单量、人均订单金额

\end{itemize}


\subparagraph{流量分析}
\label{\detokenize{chapter_knowledge/data_analysis:id12}}\begin{itemize}
\item {} 
流量来源;

\item {} 
流量数量:UV、PV;

\item {} 
流量质量:浏览深度(UV、PV)、停留时长、来源转化、ROI(投资回报率,return
on investment);

\end{itemize}


\subparagraph{PV > UV:}
\label{\detokenize{chapter_knowledge/data_analysis:pv-uv}}\begin{itemize}
\item {} 
PV(访问量):即Page View,
具体是指网站的是页面浏览量或者点击量,页面被刷新一次就计算一次。如果网站被刷新了1000次,那么流量统计工具显示的PV就是1000
。

\item {} 
UV(独立访客):即Unique
Visitor,访问您网站的一台电脑客户端为一个访客。00:00\sphinxhyphen{}24:00内相同的客户端只被计算一次。

\end{itemize}

\begin{figure}[H]
\centering
\capstart

\noindent\sphinxincludegraphics{{UV}.png}
\caption{UV拆分}\label{\detokenize{chapter_knowledge/data_analysis:id23}}\end{figure}
\begin{itemize}
\item {} 
另外就是APP的埋点数据,这个功能的点击率是多少?这个功能有多少人打开,又有多少人使用了?有多少人在频繁使用这个功能?等等,这些埋点数据要时常关注。结合数据变化来反思功能设计的问题,从而优化产品。

\end{itemize}


\subparagraph{数据埋点}
\label{\detokenize{chapter_knowledge/data_analysis:id13}}
定义:产品上线前,提前写入代码,去统计一个产品的关键页面或关键动作的数据,以便产品上线后进行数据统计分析和产品迭代。\sphinxhref{https://www.jianshu.com/p/a57a927a112b}{9}%
\begin{footnote}[669]\sphinxAtStartFootnote
\sphinxnolinkurl{https://www.jianshu.com/p/a57a927a112b}
%
\end{footnote}

B端埋点工具:Google
Analytics(GA)、百度统计。开发埋入工作量小,可扩展性强,百度统计提供的数据可视化后台基本能应付一个产品的常规数据需求,性价比非常高。

人工智能时代:如何写一份高质量的埋点文档:\sphinxurl{https://www.jianshu.com/p/b791a7b37326}

\sphinxstylestrong{数据分析:}
\begin{itemize}
\item {} 
基本分析:网站的全体用户、分群用户、个体用户的浏览行为进行全面准确地监控和分析,从而优化站点
内容,提高留存率、转化率等;可统计:访客来源、设备信息、访客属性、页面访问量、停留时长、流转去向等

\item {} 
桑基图(能量分流图)

\item {} 
概要的迅速观察用户的整体访问路径和习惯,以及在哪些页面、什么情况下用户会终端访问

\item {} 
Cohort分析图(队列分析):留存分析法

\item {} 
访客分析

\item {} 
客观分析全面的用户行为数据

\item {} 
热力图

\item {} 
页面不同区域的热度图表

\end{itemize}

\sphinxstyleemphasis{B端和C端数据埋点的区别}

诉求
\begin{itemize}
\item {} 
B端(尤其业务系统):观察研究用户对各项产品功能的接受程度、使用情况、用户操作习惯等,进一步评估功能是否合理,能否帮助用户提升效率

\item {} 
C端:提升用户体验,细致的、全面的数据埋点

\end{itemize}

方案
\begin{itemize}
\item {} 
B端:web埋点(URL访问、跳转、按钮点击、文本框录入)

\item {} 
C端:app(交互行为进行细致的埋点,全面掌握用户的动作)

\end{itemize}


\subparagraph{产品数据分析}
\label{\detokenize{chapter_knowledge/data_analysis:id14}}\begin{itemize}
\item {} 
搜索功能:搜索人数/次数、搜索功能渗透率、搜索关键词;

\item {} 
关键路径漏斗等产品功能设计分析;

\end{itemize}


\paragraph{警惕指标作弊}
\label{\detokenize{chapter_knowledge/data_analysis:id15}}\begin{itemize}
\item {} 
DAU(日活跃用户数):买垃圾流量,做各种不靠谱的活动。

\item {} 
下载量:虚假宣传,夸大产品价值。

\item {} 
注册用户数:不考虑留存的注册返现。

\item {} 
活跃度:在“分子 / 分母”的公式上做文章,在分子、分母的定义上玩花样。

\item {} 
人均 PV:一篇文章分 N 页,人均停留时间也类似。

\item {} 
点击率:软件下载站上,各种花花绿绿的“下载”按钮,点好几次也不一定能点到真的下载链接。

\item {} 
使用时长:后台运行,或者故意“迷惑”用户,让用户无法快速完成任务。

\item {} 
付费用户数:首单 1 分钱。

\item {} 
复购率:首单 9 块 9,第二单 1 毛。

\item {} 
不只是制订指标的人,哪怕经常完成指标的你,也一定对上面这些投机取巧的做法深恶痛绝。但人性使然,我们不能去正面挑战它。

\end{itemize}

真正的成功指标可以反映出用户的“非受迫、无诱导的成功行为”。衡量指标要在执行开始前制订,而不是过程中根据“做的情况”调整。如果没有重大变化,不可以不断调整目标
\begin{itemize}
\item {} 
非受迫:用户没有被逼着做没价值的事情,比如有些 App
里的签到才能获得某个价值;

\item {} 
无诱导:用户的行为不是“奖励就有,没奖励就没有”,比如有红包才会转发;

\item {} 
成功行为:指的是指标考察的行为,本身就为用户创造了价值,而不只对公司有价值。

\end{itemize}


\subparagraph{商品数据分析}
\label{\detokenize{chapter_knowledge/data_analysis:id16}}\begin{itemize}
\item {} 
商品动销:GMV、客单价、下单人数、取消购买人数、退货人数、各端复购率、购买频次分布、运营位购买转化;

\item {} 
商品品类:支付订单情况(次数、人数、趋势、复购)、访购情况、申请退货情况、取消订单情况、关注情况/;

\end{itemize}


\subparagraph{订单数据分析}
\label{\detokenize{chapter_knowledge/data_analysis:id17}}\begin{itemize}
\item {} 
订单指标:总订单量、退款订单量、订单应付金额、订单实付金额、下单人数;

\item {} 
转化率指标:新增订单/访问UV、有效订单/访问UV;

\end{itemize}


\paragraph{AI 数据}
\label{\detokenize{chapter_knowledge/data_analysis:ai}}
从“数据”这个角度来说,从收集(TTS,3个月)、分析(看大量聊天对话数据,才能自己提炼规则feature)、应用(产品早期,数据的价值甚至大过技术模型算法)到测试(产品需求、TE测试、用户使用,数据集都是不一样的,越来越不可控)等等,每个环节都有很大不同。

从结果看,即使是大公司中级产品经理(总监级),也至少3\sphinxhyphen{}6个月来适用AI产品工作,甚至都很难有自己真正独到而深入的理解认知


\subparagraph{和基线比较 5\sphinxfootnotemark[670]}
\label{\detokenize{chapter_knowledge/data_analysis:id18}}%
\begin{footnotetext}[670]\sphinxAtStartFootnote
\sphinxnolinkurl{https://neal-lathia.medium.com/machine-learning-for-product-managers-ba9cf8724e57}
%
\end{footnotetext}\ignorespaces 
我们通常孤立地看待早期/MVP产品:我们只做一件事,然后把它推出去看顾客的反应。

ML产品是不同的,因为性能总是相对的——即使是第一次迭代。例如,如果您的高级ML算法是95\%的准确性,但您的简单基线是94\%的准确性,那么您投资了大量的工作,以获得1\%的收益。另一方面,如果您的ML算法的准确率是75\%,但简单的基线是50\%,那么您已经取得了巨大的飞跃。

这里有两点很重要:首先,性能总是相对于某些东西:您需要一个基线(baseline)。其次,为了能够进行比较,您需要定义良好的指标。

在ML产品中,这些指标通常分为离线指标(例如,“算法预测历史数据的准确性有多高?”)和在线指标(例如,“当我们使用这种算法部署产品时,我们能获得多少转化率?”)。


\subsubsection{产品迭代管理}
\label{\detokenize{chapter_knowledge/upgrade_manage:id1}}\label{\detokenize{chapter_knowledge/upgrade_manage::doc}}

\paragraph{产品迭代管理的目的}
\label{\detokenize{chapter_knowledge/upgrade_manage:id2}}
产品在研发上线,推向市场时,需要随时监控市场变化、用户需求变化,不断迭代产品,以延长产品生命周期。


\paragraph{产品迭代管理的要求}
\label{\detokenize{chapter_knowledge/upgrade_manage:id3}}
软件的持续优化、升级:充分利用研发资源,正确认识技术优化所需的资源,升级研发效率
\begin{enumerate}
\sphinxsetlistlabels{\arabic}{enumi}{enumii}{}{.}%
\item {} 
研发资源管理:研发人力资源安排图(时间、负责人、项目模块)

\item {} 
技术优化资源分配

\end{enumerate}
\begin{itemize}
\item {} 
初创:权利开发业务功能,10\%用来技术优化

\item {} 
瓶颈:业务需求满足疲态,技术架构、设计缺陷出现问题,50\%技术优化

\item {} 
重构:80\%资源做技术重构

\item {} 
稳定:10\%\sphinxhyphen{}20\%资源持续做技术优化

\end{itemize}
\begin{enumerate}
\sphinxsetlistlabels{\arabic}{enumi}{enumii}{}{.}%
\item {} 
双周迭代
局限性:MVP不一定能在迭代周期内交付、跨端项目复杂研发节奏相互依赖、难以准确预估工作投入

\end{enumerate}


\paragraph{迭代优化的过程}
\label{\detokenize{chapter_knowledge/upgrade_manage:id4}}
\sphinxstylestrong{闭环过程}:upgrade\sphinxhyphen{}>需求挖掘\sphinxhref{https://g.yuque.com/amir/pm/iu3lkk}{11}%
\begin{footnote}[671]\sphinxAtStartFootnote
\sphinxnolinkurl{https://g.yuque.com/amir/pm/iu3lkk}
%
\end{footnote}、反馈\sphinxhyphen{}>不满\sphinxhyphen{}>产品需求池\sphinxhyphen{}>评估\sphinxhyphen{}>排序筛选(其他的被排除)\sphinxhyphen{}>转化为功能(其他的用弥补方案)\sphinxhyphen{}>Feature
List\sphinxhyphen{}>upgrade


\paragraph{形式 9\sphinxfootnotemark[672]}
\label{\detokenize{chapter_knowledge/upgrade_manage:id5}}%
\begin{footnotetext}[672]\sphinxAtStartFootnote
\sphinxnolinkurl{https://www.zhihu.com/pub/reader/119967224/chapter/1284014011047555072}
%
\end{footnotetext}\ignorespaces 

\subparagraph{升级换代}
\label{\detokenize{chapter_knowledge/upgrade_manage:id6}}
任何一款产品的生命周期都是有限的,因为客户总有「审美疲劳」的一天,竞争对手也在不断地推陈出新。后续新产品一般在前一款产品开始进入成熟期的后期时推出。这样,既保证了产品的连续性,又保护了前一款产品的获利机会。在图
5\sphinxhyphen{}3 中,当第一代产品 A 进入衰退期时,第二代产品 B
开始进入成长期;当第二代产品 B 进入衰退期时,第三代产品 C
开始进入成长期。通过多代产品的持续更新升级,企业实现持续的利润增长。


\subparagraph{产品系列化}
\label{\detokenize{chapter_knowledge/upgrade_manage:id7}}
为了实现规模经济性,获取尽可能大的市场份额,企业可以在第一款基型产品的基础上,采用平台开发模式,针对不同细分市场开发系列新产品。基于共用平台系列化地开发新产品,不但可以大幅降低单款新产品的开发成本,大幅缩短新产品的开发和上市周期,而且可以大大延续产品的整体生命周期。

\begin{figure}[H]
\centering
\capstart

\noindent\sphinxincludegraphics{{market_update}.png}
\caption{系列产品开发的可能方向选择}\label{\detokenize{chapter_knowledge/upgrade_manage:id22}}\end{figure}
\begin{itemize}
\item {} 
成本降低类产品开发

\item {} 
重新定位类产品开发

\item {} 
更新换代类产品开发

\item {} 
延伸拓展类产品开发

\end{itemize}


\paragraph{需求管理背景}
\label{\detokenize{chapter_knowledge/upgrade_manage:id8}}
内部资源永远是紧张的,外部竞争却时刻在发生。所谓内部资源,主要是指研发资源。与AI相关的研发工程师在大多数企业中都是处于稀缺的状态,如何“好钢用在刀刃上”,那就考验产品经理的决策能力了。

当产品经理面对多个需求时该怎样做呢?我们没有办法把所有的需求同时做好,与其接收多个需求同时做,可能每个需求和功能都做不好,不如抓住一个需求做到完整。


\paragraph{需求管理}
\label{\detokenize{chapter_knowledge/upgrade_manage:id9}}
需求管理——加深对业务和产品的理解;需求优先级、工作迭代计划——业务准确判断力

需求排序
\sphinxhref{http://www.woshipm.com/pd/1887717.html}{1}%
\begin{footnote}[673]\sphinxAtStartFootnote
\sphinxnolinkurl{http://www.woshipm.com/pd/1887717.html}
%
\end{footnote}——第一要务就是分清产品需求的主次,解决问题的顺序:


\paragraph{需求来源部分}
\label{\detokenize{chapter_knowledge/upgrade_manage:id10}}
不同方向的需求来源略有不同,总体来说,产品经理的需求来源有以下几个方面:
\begin{itemize}
\item {} 
业务需求:由业务方,比如
BD、编审、运营等,直接提出的业务需求(与市场营销、产品运营等各部门协作,推广产品,适应长期发展\sphinxhref{https://weread.qq.com/web/reader/46532b707210fc4f465d044k98f3284021498f137082c2e}{8}%
\begin{footnote}[674]\sphinxAtStartFootnote
\sphinxnolinkurl{https://weread.qq.com/web/reader/46532b707210fc4f465d044k98f3284021498f137082c2e}
%
\end{footnote});

\item {} 
数据挖掘:通过数据挖掘和分析,发现的问题或需求;

\item {} 
竞对调研:通过分析竞对的产品,发现竞对比我们有优势或值得学习借鉴的地方;

\item {} 
实地观察:不论是 B 端产品还是 C
端产品,其实都有大量的机会实地观察用户行为。比如直接陪访城市运营等;

\item {} 
战略需要:俗话说「老大拍的」,这类是由公司 leader
直接安排的需求,通常表示公司未来发展方向或急需解决的关键问题;

\item {} 
其他还包括客服、商服反馈的问题,通过在线渠道或用户访谈获得的需求,还有微博、朋友圈吐槽等各种
SNS
途径。不论哪种途径获取的需求,产品经理都应该有一个自己的需求池,统一记录各种想法。

\end{itemize}

在不同需求来源中,最看重的产品经理自己发现/评估的需求,这是评价产品经理需求决策能力的重要指标。


\paragraph{收集和整理需求}
\label{\detokenize{chapter_knowledge/upgrade_manage:id11}}
收集和整理需求,就是将需求统一记录到需求池,方便团队内/间的传播。每个团队建议维护自己的需求池地址。

需求池只是一个备忘,记录下来的需求不一定都要做,也未必是值得做的需求才记录下来。


\subparagraph{需求收集}
\label{\detokenize{chapter_knowledge/upgrade_manage:id12}}\begin{itemize}
\item {} 
需求来源:见上。

\item {} 
需求内容:交互体验优化、业务调整要求、业务管理要求

\item {} 
需求采集:1V1面谈、问卷调研、轮岗实习

\item {} 
需求背后的 \sphinxstylestrong{真实问题} 以及 \sphinxstylestrong{需求的价值}

\item {} 
需求背后的真正问题是什么?

\item {} 
问题是否有简单快速的解法?

\item {} 
问题影响面有多大?个案问题是否值得研究解决

\item {} 
共性问题,优先级和紧急程度?

\end{itemize}


\subparagraph{需求池}
\label{\detokenize{chapter_knowledge/upgrade_manage:id13}}
企业建立需求池是为了让AI产品经理能够了解在一定时间内该业务的整体需求,从而能够更加全面地评估需求,需求池可以按照周或月的频次进行更新,这样更便于观察。


\subparagraph{需求池管理}
\label{\detokenize{chapter_knowledge/upgrade_manage:id14}}\begin{enumerate}
\sphinxsetlistlabels{\arabic}{enumi}{enumii}{}{.}%
\item {} 
目的:实现清晰准确的需求管理、迭代计划管理,使项目进度透明

\item {} 
需求池管理模板:

\end{enumerate}
\begin{itemize}
\item {} 
业务线/对应的系统

\item {} 
需求类型:产品需求、产品需求(插入)、技术需求、技术需求(插入)、线上bug

\item {} 
主题:需求概述

\item {} 
背景:原因引发

\item {} 
内容:需求具体描述

\item {} 
预期收益:用户、产品、市场的数据指标

\item {} 
来源:需求提出者

\item {} 
提出时间

\item {} 
优先级:重要紧迫度、目标、作用对象;需要与业务方在原则上达成一致

\item {} 
迭代版本

\item {} 
业务负责人

\item {} 
产品经理

\item {} 
研发负责人

\item {} 
测试负责人

\item {} 
状态

\item {} 
计划上线时间

\item {} 
实际上线时间

\item {} 
前端开始/结束日期

\item {} 
前端研发工作量

\item {} 
发版计划

\end{itemize}

\begin{figure}[H]
\centering
\capstart

\noindent\sphinxincludegraphics{{requirements_pool_mindmap}.png}
\caption{需求池管理文档基本结构\sphinxhref{https://g.yuque.com/zhongguodianxinyanjiuyuan/bgso10/dtrl4d}{10}\sphinxfootnotemark[675]}\label{\detokenize{chapter_knowledge/upgrade_manage:id23}}\end{figure}
%
\begin{footnotetext}[675]\sphinxAtStartFootnote
\sphinxnolinkurl{https://g.yuque.com/zhongguodianxinyanjiuyuan/bgso10/dtrl4d}
%
\end{footnotetext}\ignorespaces 

\paragraph{筛选评估需求如此困难的原因}
\label{\detokenize{chapter_knowledge/upgrade_manage:id15}}\begin{itemize}
\item {} 
相比去做更普适的项目,做那些你最喜欢的、自己会用的产品更令人满足。

\item {} 
相比去做直接对你的目标产生影响的项目,把注意力集中在那些聪明有趣的主意上更有诱惑力。

\item {} 
相比去做自己已经有信心的项目,去钻研新的想法更令人兴奋。(有些情况也可能反过来)

\item {} 
相比去做语权大的需求,拒绝伪需求冒着得罪人的更稳妥

\item {} 
相比追求面面俱到,只把主线功能做好显得多么简陋

\end{itemize}


\paragraph{需求属性的评估}
\label{\detokenize{chapter_knowledge/upgrade_manage:id16}}

\subparagraph{三维度}
\label{\detokenize{chapter_knowledge/upgrade_manage:id17}}\begin{itemize}
\item {} 
需求价值评估:需求的价值分为两个维度,一个维度是创造效益,另一个维度是节省成本,通过对需求背景的影响因子给定假设条件进行评估,就能大致估算出这个需求可以产生的预期效果;例如:当前的妥投率为95\%,未妥投原因中收件地址错误的原因占比75\%,假设通过增加收件人地址校验,解决85\%,那么需求的预期收益就是提升3.18\%的妥投率。

\item {} 
需求难度评估:难度可以从是否需要设计新模块开发、原有模块改造量、开发实现难度、是否涉及关联系统改造几个方面进行考量;如果一个需求既设计算法选项、参数优化、训练数据标注、模块封装、总控接入、前端改造那么这个需求的实现难度就属于很高的程度了。

\item {} 
需求周期评估:需求拆解后本项目组的开发周期加上\sphinxstylestrong{关联系统排期、开发的周期}就能确定需求的实现周期。

\end{itemize}


\subparagraph{RICE 5\sphinxfootnotemark[676]}
\label{\detokenize{chapter_knowledge/upgrade_manage:rice-5}}%
\begin{footnotetext}[676]\sphinxAtStartFootnote
\sphinxnolinkurl{https://weread.qq.com/web/reader/40632860719ad5bb4060856ke3632bd0222e369853df322}
%
\end{footnotetext}\ignorespaces 
RICE SCORE =
R\sphinxstyleemphasis{I}C/E,根据RICE评分即可对需求进行排序,该方法比较适用于大型项目,在一般项目中不常使用。


\subparagraph{Reach(接触数量)}
\label{\detokenize{chapter_knowledge/upgrade_manage:reach}}
接触数量是指用每个时间段的用户数或事件数来衡量,考察一个需求在一定时间段内会影响多少用户。这可能是“每季度客户数量”或“每月交易数量”,尽可能使用产品指标的实际测量结果


\subparagraph{Impact(影响程度)}
\label{\detokenize{chapter_knowledge/upgrade_manage:impact}}
影响程度是对目标产生可观影响的需求,以此来预估这个项目对个人产生的影响。可以分为巨大影响、高、中、低、极低几个标准。


\subparagraph{Confidence(信心指数)}
\label{\detokenize{chapter_knowledge/upgrade_manage:confidence}}
有些需求有创意但无数据支持而显得不明确,我们在评估时可以把信心指数考虑进去,可以分高为100\%、中为80\%、低为50\%三个档次。


\subparagraph{Effort(投入精力)}
\label{\detokenize{chapter_knowledge/upgrade_manage:effort}}
为了迅速行动并且事半功倍,估算项目需要团队的所有成员(产品、设计和工程)的总时间。投入精力的预估单位是人/月。


\subparagraph{MoSCoW}
\label{\detokenize{chapter_knowledge/upgrade_manage:moscow}}
美其名曰“全面”,以全面来打入市场。最终却是“样样做,样样差”。must
have、should have、could have、won’t have模型。

\begin{figure}[H]
\centering
\capstart

\noindent\sphinxincludegraphics{{MoSCoW}.png}
\caption{MoSCoW}\label{\detokenize{chapter_knowledge/upgrade_manage:id24}}\end{figure}
\begin{itemize}
\item {} 
位于“1”(Must have):用户价值高,难度低,优先制作

\item {} 
位于“2”(won’t have):用户价值低,难度高,延后制作甚至不做;

\item {} 
位于“3”、“4”:如果交货时间紧,“可以有”将第一批被删除,“应该有”紧随其后。

\end{itemize}


\subparagraph{Kano帮你找到用户满意度2\sphinxfootnotemark[677]}
\label{\detokenize{chapter_knowledge/upgrade_manage:kano2}}%
\begin{footnotetext}[677]\sphinxAtStartFootnote
\sphinxnolinkurl{https://www.huaweicloud.com/articles/280202e7d83cd36df93e5f027939cbaa.html}
%
\end{footnotetext}\ignorespaces 
一种在不同阶段按产品目标倒退需求优先级的思维方式,它将需求分为三类:

它将需求分为三类:
\begin{enumerate}
\sphinxsetlistlabels{\arabic}{enumi}{enumii}{}{.}%
\item {} 
基础功能。代表产品进入市场的基本门槛,保证能够满足用户普遍需求的最低标准。然而在后续的研发若投入大量精力,并不会显著提高用户的满意度或建立产品的竞争门槛,因此此类需求优先级较低。

\item {} 
性能需求。即在实现基础功能后,为了提升和优化产品性能的需求。这类需求可以在一定程度上提升用户满意度,但其他竞争对手同时也会在这方面持续投入,ROI通常为线性。

\item {} 
尖叫(兴奋)功能。用户使用产品后能够感受到喜悦和兴奋,这种产品可能是非常有创造性的,也有可能带来

\end{enumerate}

属性的成熟程度和情绪反应之间呈线性关系,主要针对于如易用性、成本、娱乐价值和安全性这样的产品特征。

狩野纪昭(Noriaki
Kano)将五种情绪反应可视化为图中的曲线,其中,y轴是情绪反应,x轴是特征的成熟程度。情绪反应的强度由特征如何充分呈现和其成熟程度驱动。

将需求划分为必备型、期望型、魅力型、无差异型、反向型五类,分别以英文字母M、O、A、I、R表示。
\begin{itemize}
\item {} 
必备型需求(M):需求满足时,用户不会感到满意。需求不满足时,用户会很不满意。

\item {} 
期望型需求(O):需求满足时,用户会感到很满意。需求不满足时,用户会很不满意。

\item {} 
魅力型需求(A):该需求超过用户对产品本来的期望,使得用户的满意度急剧上升。即使表现的不完善,用户的满意度也不受影响。

\item {} 
无差异型需求(I):需求被满足或未被满足,都不会对用户的满意度造成影响。

\item {} 
反向型需求(R):该需求与用户的满意度呈反向相关,满足该要求,反而会使用户的满意度下降。

\end{itemize}

better\sphinxhyphen{}worse系数:
\begin{itemize}
\item {} 
Better系数=(期望数+魅力数)/(期望数+魅力数+必备数+无差异数)

\item {} 
Worse系数= \sphinxhyphen{}1*(期望数+必备数)/(期望数+魅力数+必备数+无差异数)

\end{itemize}

Better系数越接近1,表示该具备度越高该需求对用户满意度提升的影响效果越大。Worse系数越接近\sphinxhyphen{}1,表示具备度越低该需求对用户满意度造成的负面影响越大。

\sphinxurl{http://www.woshipm.com/pd/4383131.html}


\subparagraph{维格斯法}
\label{\detokenize{chapter_knowledge/upgrade_manage:id18}}
该方法将需求分为4个维度来进行评估。
\begin{itemize}
\item {} 
实现需求给客户带来的收益。

\item {} 
不实现需求给客户带来的损害。(不做会怎么样?发现用户在解决这个问题的不满)

\item {} 
实现需求所需要耗费的成本。

\item {} 
实现需求的风险。其中收益和损害是从客户角度出发的,而成本和风险则是从实现角度出发的,是逻辑较清晰且通用的方法。

\end{itemize}


\subparagraph{优先级指数量表}
\label{\detokenize{chapter_knowledge/upgrade_manage:id19}}
优先级指数=(需求急迫性+功能价值+需求普遍性+数据支持度+资源准备度)/(开发成本+技术实现难度)

\begin{figure}[H]
\centering
\capstart

\noindent\sphinxincludegraphics{{need_judge}.png}
\caption{优先级指数量表}\label{\detokenize{chapter_knowledge/upgrade_manage:id25}}\end{figure}


\subparagraph{相似组分类法(Affinity Grouping) 6\sphinxfootnotemark[678]}
\label{\detokenize{chapter_knowledge/upgrade_manage:affinity-grouping-6}}%
\begin{footnotetext}[678]\sphinxAtStartFootnote
\sphinxnolinkurl{http://reader.epubee.com/books/mobile/f4/f4c52db61d39acb835e2709cbed1585e/text00009.html}
%
\end{footnotetext}\ignorespaces 
相似组分类法是一种让团队成员取得一致的好办法,首先需要团队成员进行头脑风暴,尽量将能想出来的需求写在卡片上,然后团队一起将每个卡片按照内容相似度进行分组,并给每个组起好名字,最后团队共同为每个组进行投票打分,选出优先级最高的组和这个组里优先级最高的卡片。


\paragraph{其他弥补方案 4\sphinxfootnotemark[679]}
\label{\detokenize{chapter_knowledge/upgrade_manage:id20}}%
\begin{footnotetext}[679]\sphinxAtStartFootnote
\sphinxnolinkurl{http://www.woshipm.com/pmd/4161341.html}
%
\end{footnotetext}\ignorespaces \begin{enumerate}
\sphinxsetlistlabels{\arabic}{enumi}{enumii}{}{.}%
\item {} 
平衡
如果是面向B端业务,那么所有业务线对自己的需求都是关注且紧迫的;这时候就需要学会平衡每个业务的需求,不能被业务完全牵着走,这样对产品规划会有极大影响。

\end{enumerate}

那么每当这时候,就需要可以一起拉通多个业务,来集中评估各方的诉求,宣导团队的资源是有限性的;让业务之间来取舍他们之间的优先级,这就是让“决策”转移到业务自身上。
\begin{enumerate}
\sphinxsetlistlabels{\arabic}{enumi}{enumii}{}{.}%
\setcounter{enumi}{1}
\item {} 
替代
任何的产品解决方案都是有备选方案的,那么在当前无法尽快满足用户的前提;可以优先采用临时方案,先满足用户最核心的需求,把其他延伸性需求先砍掉,待到条件成熟在上线完整需求。

\end{enumerate}

而这就是采用“替代”的方式,一定程度上去满足用户需求,这对挽留用户、提升用户口碑有极大帮助。
\begin{enumerate}
\sphinxsetlistlabels{\arabic}{enumi}{enumii}{}{.}%
\setcounter{enumi}{2}
\item {} 
延迟
这是一个不太“厚道”的方式,前面提到用户对他们自身的需求都关注比较强烈,受限当前的规划和限制条件,确实无法那无法尽快满足的情况,但用户是不会理解买账的;那么如何去“安抚‘用户情绪呢,把负面情绪尽可能降到最低?

\end{enumerate}

这就靠一个“拖”字决,可以先给对方的一个明确信号:我们会做(类似先画个饼)。

但是由于一些原因(把这些问题夸大),需要稍微往后一些才能支持,那么这个往后的时间就可以相对灵活可变的,在这里也主要是\sphinxstylestrong{安抚用户的情绪}为主。


\paragraph{Feature List}
\label{\detokenize{chapter_knowledge/upgrade_manage:feature-list}}
\begin{figure}[H]
\centering
\capstart

\noindent\sphinxincludegraphics{{Feature_list}.png}
\caption{功能需求单}\label{\detokenize{chapter_knowledge/upgrade_manage:id26}}\end{figure}


\paragraph{AI相关}
\label{\detokenize{chapter_knowledge/upgrade_manage:ai}}

\subparagraph{模型更新 7\sphinxfootnotemark[680]}
\label{\detokenize{chapter_knowledge/upgrade_manage:id21}}%
\begin{footnotetext}[680]\sphinxAtStartFootnote
\sphinxnolinkurl{https://yam.gift/2021/02/19/ExpSum/2021-02-19-AI-Engineer-Growing-I/}
%
\end{footnotetext}\ignorespaces 
具体又包括以下几个方面:
\begin{itemize}
\item {} 
模型全量更新。主要是指模型整体升级,比较推荐 tensorflow/serving: A
flexible, high\sphinxhyphen{}performance serving system for machine learning
models,不仅性能优秀,而且可以通过监控模型文件的变化自动升级到新的版本。同时,还支持
RPC 和 RestFul 两种接口,支持版本控制,支持多个模型,简直是业界良心。

\item {} 
模型在线学习。主要指模型根据线上数据实时或准实时更新模型的情况。这块目前工作几乎还未涉及,日后更新。

\item {} 
更新毛刺。主要指模型更新前后线上请求出现的延迟抖动现象。一般在规模很大时才会出现。爱奇艺团队针对
Tensorflow Serving 有过不错的改进尝试,被 Tensorflow
官方公号发表,具体参见:社区分享 | TensorFlow Serving
模型更新毛刺的完全优化实践。

\end{itemize}


\subsubsection{资源管理}
\label{\detokenize{chapter_knowledge/resource_manage:id1}}\label{\detokenize{chapter_knowledge/resource_manage::doc}}
资源管理(数据集、算法模型、策略、软硬件资源、需求管理)


\paragraph{产品研发规划1\sphinxfootnotemark[681]}
\label{\detokenize{chapter_knowledge/resource_manage:id2}}%
\begin{footnotetext}[681]\sphinxAtStartFootnote
\sphinxnolinkurl{http://www.woshipm.com/pmd/220940.html}
%
\end{footnotetext}\ignorespaces 
研发规划离我们就很近了,首先研发规划的输入应该是产品规划或产品年度规划,产品年度规划的内容需要通过产品研发规划落地。研发规划可以理解成大的项目计划,那自然应该包括项目范围和需求收集,需求分析和优先级的评估,研发资源的投入和成本,研发功能点,研发进度计划。如果存在多个子产品,还必须考虑研发产品组合规划,然后才是子产品的研发计划。

研发计划需要进一步细化到短周期的产品研发项目和版本,每一个版本有明确的项目范围和交付功能,有明确的资源申请和资源拖入,研发的每个版本最终通过立项后要严格按项目管理的方式进行计划,执行和跟踪。

研发规划里面另外一个重点就是技术规划,很多时候我们的研发规划和技术规划是放在一起的,但是不同的产品研发往往是涉及到相同的技术和平台。那么这部分应该抽取出来,根据产品开发方法论,本身也是包括了产品,平台和技术三个层面的内容。平台层和技术层如果足够强大,那么产品本身的灵活和可配置性也越强。

资源管理和管道管理在产品组合管理里面是一个重要内容,在研发规划里面一定要考虑到对资源的需求,特别是涉及到多产品规划的时候,资源冲突往往是一个很头疼的事情。而研发资源往往又不适合在同一时间兼顾多个产品或项目。资源规划本身包括了对资源技能的要求,时间的要求和数量的要求。这些都需要提前考虑到。


\paragraph{产品财务规划}
\label{\detokenize{chapter_knowledge/resource_manage:id3}}
产品财务规划是产品战略规划和商业模式中盈利模式分析和规划内容的进一步落地。财务规划必须回答问题是产品投入多大,需要多少成本?后续盈利模式如何,需要多久能够开始盈利。是否有进一步的可持续的盈利模式,还有哪些增值点,这些都必须考虑到。

资源一定要分阶段投入,产品也最好能够分迭代版本逐步退出。要知道产品研发周期越长,那么成本投入越大,企业本身的现金流压力也就越大。公司做产品规划,做产品的目的只有一个就是合法的赚取最大化的利润。这是很正常的一个事情,一方面是回报员工和股东,一方面是公司可持续长久发展。

产品财务规划首先是产品的投资回收期和内部收益率分析,其次是产品投入预算,预算的进一步分解等。有预算后续有执行,就容易进一步的按产品核算成本和收益。对于产品推出市场,又必须考虑产品的定价策略,产品定价策略需要考虑市场本身成熟度,客户关系积累,产品本身的发展多方面信息进行完善。


\subsubsection{Research}
\label{\detokenize{chapter_knowledge/research:research}}\label{\detokenize{chapter_knowledge/research::doc}}

\paragraph{工具}
\label{\detokenize{chapter_knowledge/research:id1}}
谷歌学术、MIT/卡梅隆/北大清华图书馆资源库、公司内部知识库(如商汤/阿里)


\paragraph{博客}
\label{\detokenize{chapter_knowledge/research:id2}}\begin{itemize}
\item {} 
\sphinxurl{https://medium.com/}

\item {} 
\sphinxurl{https://towardsdatascience.com/}

\item {} 
\sphinxurl{https://ai.googleblog.com/}

\item {} 
\sphinxurl{https://deepmind.com/blog/?category=research}

\item {} 
\sphinxurl{https://research.fb.com/}

\end{itemize}


\paragraph{论文}
\label{\detokenize{chapter_knowledge/research:id3}}

\subparagraph{顶级会议}
\label{\detokenize{chapter_knowledge/research:id4}}\begin{itemize}
\item {} 
NLP:ACL / NAACL / EMNLP …

\item {} 
CV: CVPR / ICCV / ECCV …

\item {} 
ML: ICML / ICLR / NIPS / AAAI …

\item {} 
Data Mining \& Information Retrieval: WWW / KDD / IJCAI …

\end{itemize}


\subparagraph{Sci\sphinxhyphen{}Hub}
\label{\detokenize{chapter_knowledge/research:sci-hub}}
\sphinxurl{http://www.sci-hub.io/}

备用站点:\sphinxurl{http://www.sci-hub.cc/}

中国版以及备用站点:\sphinxurl{http://www.sci-hub.cn/}、\sphinxurl{http://www.sci-hub.xyz/}


\subparagraph{谷歌学术}
\label{\detokenize{chapter_knowledge/research:id5}}
谷歌学术网址,\sphinxurl{http://scholar.glgoo.org/}、\sphinxurl{https://xs.glgoo.net/}、\sphinxurl{http://scholar.hedasudi.com/}

也有镜像网站合集http://www.dirmor.com/


\subparagraph{Library Genesis}
\label{\detokenize{chapter_knowledge/research:library-genesis}}
Library
Genesis号称是帮助全人类知识无版权传播的计划。网站上论文很多,下载方便,还有很多外文书籍和中文书籍,几乎每天都在更新。\sphinxurl{http://gen.lib.rus.ec/}


\subparagraph{arXiv}
\label{\detokenize{chapter_knowledge/research:arxiv}}
www.arxiv.org

arXiv,
是archive(归档)的意思,是一个由康乃尔大学维护的免费的多学科论文预出版(preprint)数据库。所谓预出版,就是说论文还没有经过同行评审,文责自负,文章质量参差不齐,所以一般不会作为正式的学术成果。不过有的学科习惯上先把文章公开到arXiv上,然后再提交到会议上。
arXiv 是我们搜索、浏览和下载学术论文的重要工具。近 30 年来,arXiv
为公众和研究社区提供了开放获取学术论文的服务。这些论文涉及物理学的庞大分支和计算机科学的众多子学科,如数学、统计学、电气工程、定量生物学和经济学等等。

Chrome extension that adds video explanations to research papers on
arxiv.org: \sphinxurl{https://github.com/amitness/papers-with-video}


\subparagraph{arxiv\sphinxhyphen{}sanity}
\label{\detokenize{chapter_knowledge/research:arxiv-sanity}}
特斯拉的人工智能高级总监 Andrej Karpathy:\sphinxurl{http://www.arxiv-sanity.com/}

感兴趣相关度排序、个人图书馆、推荐系统、都在看什么\sphinxhref{https://cloud.tencent.com/developer/article/1473703}{2}%
\begin{footnote}[682]\sphinxAtStartFootnote
\sphinxnolinkurl{https://cloud.tencent.com/developer/article/1473703}
%
\end{footnote}

\sphinxurl{https://arxiv.xixiaoyao.cn/}


\subparagraph{semanticscholar3}
\label{\detokenize{chapter_knowledge/research:semanticscholar3}}
\sphinxurl{https://www.semanticscholar.org/}


\subparagraph{metacademy}
\label{\detokenize{chapter_knowledge/research:metacademy}}
\sphinxurl{https://metacademy.org/}

metacademy看作一副机器学习和人工智能的知识图谱,在上面搜索任意机器学习或者人工智能的知识概念,它会告诉你学习这个知识点需要什么前置知识,需要多长时间掌握,并罗列出相关的课程。


\subparagraph{AMiner3}
\label{\detokenize{chapter_knowledge/research:aminer3}}
输入关键词就能找到对应的学者: \sphinxurl{https://www.aminer.cn/}

\sphinxurl{https://www.aminer.cn/ai2000/country/China}


\subparagraph{著名学者}
\label{\detokenize{chapter_knowledge/research:id6}}\label{\detokenize{chapter_knowledge/research:id7}}
AI 领域大多数有影响力的人物(比如谷歌的 Peter Norvig,Facebook AI 研究的
Yann LeCun 以及微软的 Eric Horvitz)


\subparagraph{论文}
\label{\detokenize{chapter_knowledge/research:id8}}
\sphinxurl{https://openreview.net/}

CV领域Paper论文常见单词(一) \sphinxhyphen{} 王大东的文章 \sphinxhyphen{} 知乎
\sphinxurl{https://zhuanlan.zhihu.com/p/58860096}

\sphinxurl{https://ying-zhang.github.io/misc/2016/we-love-paper/}


\paragraph{代码}
\label{\detokenize{chapter_knowledge/research:id9}}

\subparagraph{Papers with Code1\sphinxfootnotemark[683]}
\label{\detokenize{chapter_knowledge/research:papers-with-code1}}%
\begin{footnotetext}[683]\sphinxAtStartFootnote
\sphinxnolinkurl{https://www.jiqizhixin.com/articles/2020-10-09-5}
%
\end{footnotetext}\ignorespaces 
机器学习资源网站 Papers with Code
自创立以来,凭借丰富的开放资源和卓越的社区服务,成为机器学习研究者最常用的资源网站之一。网站还根据细分领域的指标对论文进行了整理和排序。以图像配准(Image
Registration)为例。指标、论文、代码,安排得明明白白。

2019 年底,Papers with Code 正式并入 Facebook
AI。最近,它又有了新举措:与论文预印本平台 arXiv 展开合作,支持在 arXiv
页面上添加代码链接。

Browse State\sphinxhyphen{}of\sphinxhyphen{}the\sphinxhyphen{}Art: \sphinxurl{https://paperswithcode.com/sota}


\subparagraph{Github \sphinxhyphen{} Awesome3}
\label{\detokenize{chapter_knowledge/research:github-awesome3}}
以 awesome 为名,进行某个领域资料的高质量整合。


\subparagraph{实践}
\label{\detokenize{chapter_knowledge/research:id10}}
推荐网站:\sphinxurl{https://www.kaggle.com/}


\paragraph{mathpix}
\label{\detokenize{chapter_knowledge/research:mathpix}}
手写/截图 转 LaTex公式:\sphinxurl{https://mathpix.com/}

LaTex如果所有公式都要自己手打还是很痛苦的。(虽然很多时候一篇Deep
Learning方向的paper公式数量只有十个左右(这还是在强行加上LSTM等被翻来覆去写烂的公式的情况下))

\sphinxurl{http://deepdive.nn.157239n.com/}


\paragraph{报告}
\label{\detokenize{chapter_knowledge/research:id11}}
\sphinxurl{http://www.zft-park.com.cn/index.php?m=Article\&a=show\&id=384}


\subsubsection{估值}
\label{\detokenize{chapter_knowledge/Valuation:id1}}\label{\detokenize{chapter_knowledge/Valuation::doc}}

\paragraph{尽职调查}
\label{\detokenize{chapter_knowledge/Valuation:id2}}
上裁判文书网、执行信息网、信用中国扒个涉诉涉执行情况


\paragraph{上市1\sphinxfootnotemark[684]}
\label{\detokenize{chapter_knowledge/Valuation:id3}}%
\begin{footnotetext}[684]\sphinxAtStartFootnote
\sphinxnolinkurl{https://www.bilibili.com/video/av21295743/}
%
\end{footnotetext}\ignorespaces 
问:中国大型的互联网公司为什么纷纷跑到海外上市呢?下面请中国法学理事会理事李可书博士为大家解答一下境内外上市的区别在哪里。

  李可书:我们注意到中国大型的互联网公司,往往选择赴海外上市,而且美国上市是优选的,这是一个目前很常见的现象。
  那么,我们再看为什么,比如说阿里,它为什么选择在美国上市,这是一个我们需要去思索的点,首先我想分析一下它为什么选择在美国上市的原因。第一个原因,是因为在A股上市准入门槛比较高,规定的条件比较多,这种条件因为阿里满足不了,首先第一个条件,我国的《证券法》和《公司法》规定的非常明确,在A股上市的公司必须是依照法律规定,在中国境内设立的股份有限公司。而阿里巴巴呢,是注册在海外的,它这种外资的身份,使得它不满足A股上市的要求,这是第一个原因,因为A股要求必须注册在国内的企业。
  第二个要求,它也满足不了,是什么呢?关于A股上市的发行对象的要求,对于A股的上市来说,要求将股东的人数控制在200人,大家知道阿里之前大量的通过职工持股,收了自己员工大量的股票,如果需要将股东人数进行压缩的话,那么需要大量清理职工持股,这也是阿里不希望做到的。所以呢,基于这两点法律的要求,阿里很难去完全满足,这是第一个阿里当时考虑在美国上市的原因之一。
  第二个原因,是因为审核的时间,因为在国内上市呢,我们国内目前审核制,注册制没有正式实行之前一直是审核制,审核制决定了我们的审核是需要时间的,正常是两到三年,这个时间对于大量的互联网公司,包括阿里在内的大量互联网公司他们觉得时间太长,耗不起这个时间,而在国外呢,比如在美国上市,实际上是非常快的,一般来讲是半年左右就完成上市。所以从时间上考虑,在美国上市,这种境外上市,比在A股上市来说有一定的吸引力。这个是阿里当时选择在美国上市的第二个原因。
  第三个原因,是关于A股对于公司的盈利有明确的财务指标的要求,包括阿里在内的京东也一样,大家知道京东前几年一直没有盈利,所以它无法满足A股上市的要求,这是第三个原因。
  所以基于这些原因,导致包括阿里、京东这些大量的互联网公司选择在境外上市,当然未来他们是否选择回归,这可能是未来我们需要考虑的点,在我国的这种A股市场的制度进行变更之后。


\paragraph{商汤科技}
\label{\detokenize{chapter_knowledge/Valuation:id4}}
成立于2014年的商汤科技,仅仅3年时间估值就暴涨到20亿美金。2017年7月,商汤科技宣布完成4.1亿美元B轮融资,创下当时全球人工智能领域单轮融资额记录。

2018年4月,商汤科技完成阿里巴巴集团领投的6亿美元C轮融资,再次创下全球人工智能领域融资记录;一个月后,商汤科技再度获得6.2亿美元C+轮融资;三个多月后,商汤科技再度获得软银10亿美金的融资,估值也飙升至60亿美金。

现在绝大部分技术型的、平台型的公司还是一To
B的场景,但投资机构却把它们当作To
C的公司来投。这样的公司,后续还需要多轮的融资支持成长。如果天使轮一下子把估值做到1亿,那A轮总得3亿,做到F轮怎么办?


\paragraph{Labby Inc2\sphinxfootnotemark[685]}
\label{\detokenize{chapter_knowledge/Valuation:labby-inc2}}%
\begin{footnotetext}[685]\sphinxAtStartFootnote
\sphinxnolinkurl{https://www.labbyinc.com/labbys-ai-enabled-optical-sensing-technology-secures-usd-480-000-seed-investment}
%
\end{footnotetext}\ignorespaces 
Labby Inc, an early\sphinxhyphen{}stage startup specializing in AI\sphinxhyphen{}enabled optical
sensing solutions for raw milk testing, today announced it has raised
\$480,000 in seed funding. AgriTech Capital, a strategy and investment
firm specializing in innovation and technology in the agribusiness
sector.

The global dairy farming industry loses \$32 billion annually due to
mastitis infections. At a minimum, twenty\sphinxhyphen{}five percent of cows each year
are impacted regardless of how well managed a farm is. Farmers lack a
way to quickly and easily identify mastitis at an early stage so they
can take preventative measures to reduce the impact on yields. With
Labby’s solution, farmers and dairy processors finally have a way to
quickly and easily test raw milk gaining visibility into animal health,
milk quality, and feed efficiency, enabling them to optimize their
operations.


\subsubsection{产品定价 1\sphinxfootnotemark[686]}
\label{\detokenize{chapter_knowledge/price:id1}}\label{\detokenize{chapter_knowledge/price::doc}}%
\begin{footnotetext}[686]\sphinxAtStartFootnote
\sphinxnolinkurl{http://www.woshipm.com/pd/3817590.html}
%
\end{footnotetext}\ignorespaces 
商业化产品的定价链条如下图所示。将定价链条的上下游关联在一起形成的核心定价工作主要有三项:第一,制定计费方式;二,制定产品价格;三,产品SKU管理。


\paragraph{计费}
\label{\detokenize{chapter_knowledge/price:id2}}
计费方式是产品定价不可分割的一部分,并且是产品定价的基础。不同类型的商业化产品的计费方式不同,比如实物产品、软件产品、广告产品、内容产品、会员产品、数据产品等。

需要说明的是,在各类产品越来越精细化的价格管理背后,计费方式的管理也更加精细化,既包括内部计费控制,又包括给予客户的计费说明。下图所示为融云的计费方式,其中每个计费标准都通过独立页面做单独说明(不做展示)。


\paragraph{定价 2\sphinxfootnotemark[687]}
\label{\detokenize{chapter_knowledge/price:id3}}%
\begin{footnotetext}[687]\sphinxAtStartFootnote
\sphinxnolinkurl{http://www.woshipm.com/ai/968453.html}
%
\end{footnotetext}\ignorespaces 
产品经理需要站在用户角度考虑产品定价策略,深入理解场景和用户的痛点在哪。
\begin{quote}

举一个简单的例子:食堂打饭这个场景中,最后一个环节通常是需要一个收银员根据你挑选的饭菜金额收费,这要依靠准确的识别和速算。
\end{quote}
\begin{quote}

如果你设计一个菜品识别(机器视觉)、报价、收费的收费机器人,你怎么给这个产品定价?
如果只是看表面,你一定觉得这个产品简直是太完美了。
如果的机器误识别率低,而且运算速度快,用户只要将菜品放在摄像头前刷卡就行了,最直接的价值就是节省了一个人的劳动力。
\end{quote}

但是你要仔细想想,食堂档口的老板会这么想吗?收银员只是在用餐高峰期充当收费的角色,在不忙的时候可能会被安排洗碗、擦地、甚至需要在后厨兼做一些帮厨的工作。

尽管在用餐高峰收费这个环节的劳动力是被省下了,但是机器人能替代人完成其他任务吗?

因此,这款产品的定价一定不会很高。


\paragraph{价格歧视}
\label{\detokenize{chapter_knowledge/price:id4}}

\subparagraph{一级价格歧视,按人来定价}
\label{\detokenize{chapter_knowledge/price:id5}}
一级价格歧视又被称为完全价格歧视,其假定供给方完全了解每个消费者对任何数量的商品或服务所愿意支付的最大金额,并以此针对每个消费者设定不同价格,以实现对每个消费者的消费剩余的全部占有。

由于信息的割裂和缺失,在现实中供给方不可能完整的了解每个消费者的保留价格,所以一级价格歧视属于极端情况。但在当前大数据背景下,企业根据用户历史行为数据、行业共享数据等海量数据进行分析,已经能够极大程度的逼近一级价格歧视的理想情况。

比较典型的案例就是前段时间热传的“大数据杀熟”事件,其中就包括滴滴打车在同时间、同路段的快车对不同用户显示不同价格,以及携程上同时间、同房型的酒店对不同用户显示不同价格。这里面就涉及到企业根据数据分析对每个用户进行画像后的差异化定价,在定价的同时也会配套使用差异化发放红包和优惠券等形式来降低用户敏感度和负面影响。


\subparagraph{二级价格歧视,按量来定价}
\label{\detokenize{chapter_knowledge/price:id6}}
二级价格歧视又称为数量歧视,是供给方对消费者的需求曲线有一定了解,针对同一商品或服务的不同购买量来进行差别定价,但每个消费者购买相同数量的价格一样。

其核心是通过不同购买量来实现价格歧视,最常见的就是各类交易平台使用的团购、满赠和满减等方式,这类定价策略一般为正向二级价格歧视,即消费者购买的数量越多,实际商品的单价越便宜,而其中价格下降导致的收益下降小于价格下降带来的销量的增加而增加的收益,故供给方的整体利润是增加的。

除了正向二级价格歧视,逆向二级价格歧视在生活中我们也能看到,最典型的就是阶梯电价和水价,如①月度用电量在180度以内,每度0.6元;②超过180度后,181度$_{\text{360度之间,每度为0.7元;③超过360度后,361度}}$540度之间,每度为0.8元……以此类推,这里的定价便是买的越多价格越贵。而这种定价策略产生的前提条件是:
\begin{enumerate}
\sphinxsetlistlabels{\arabic}{enumi}{enumii}{}{.}%
\item {} 
供给方提供的商品或服务在特定区域处于垄断(或相对接近)地位;

\item {} 
能提升资源配置效率并降低贫富差距问题。

\end{enumerate}


\subparagraph{三级价格歧视,按类来定价}
\label{\detokenize{chapter_knowledge/price:id7}}
三级价格歧视是供给方针对不同的消费者进行分类,再根据不同类用户所代表市场的需求价格弹性不同来制定不同价格策略。当某类用户群的需求价格弹性越大,则定价越低;需求价格弹性越小,则定价越高。从而在需求弹性小的用户群中获取更多的消费者剩余价值。

这种定价方式就涉及到对消费者进行用户分类,常见的方式有:

1)优惠券的发放(不包含新人礼包),如淘宝和网易严选等平台采用的店铺优惠券主动领取,而非直接发放或降价的方式,即通过消费者主动的方式来划分用户的价格敏感度;

2)分享拉人降价,如拼多多的拼团或美团、饿了么的分享红包模式,针对消费者价值敏感度高的用户进行优惠刺激,将潜在消费者变成客户。


\subsubsection{推广产品}
\label{\detokenize{chapter_knowledge/more_users:id1}}\label{\detokenize{chapter_knowledge/more_users::doc}}
宣传和介绍产品

随时监控市场变化、用户需求变化,不断迭代产品。

与市场营销、产品运营等各部门协作,让更多的人使用产品,获得更多反馈意见,不断迭代产品,以便适应更长期的发展。


\subsection{前人经验}
\label{\detokenize{chapter_experience/index:chap-exper}}\label{\detokenize{chapter_experience/index:id1}}\label{\detokenize{chapter_experience/index::doc}}

\subsubsection{早期工作}
\label{\detokenize{chapter_experience/early_phase:id1}}\label{\detokenize{chapter_experience/early_phase::doc}}
我做写匠,和之前做计算机视觉领域的产品还不太一样,写匠主要用的技术是自然语言处理(NLP)和机器学习(ML)。\sphinxhref{https://medium.com/@liwdai/ai-pm-\%E4\%B9\%8B\%E9\%9A\%90\%E6\%80\%A7\%E9\%83\%A8\%E5\%88\%86\%E7\%9A\%84\%E5\%B7\%A5\%E4\%BD\%9C-be6de08d1c05}{1}%
\begin{footnote}[688]\sphinxAtStartFootnote
\sphinxnolinkurl{https://medium.com/@liwdai/ai-pm-\%E4\%B9\%8B\%E9\%9A\%90\%E6\%80\%A7\%E9\%83\%A8\%E5\%88\%86\%E7\%9A\%84\%E5\%B7\%A5\%E4\%BD\%9C-be6de08d1c05}
%
\end{footnote}
NLP
是一个交叉学科,难度很大,有些问题从语言学角度都没有解决,导致技术受限,产品本身要分解到底层又牵扯到语言学领域的问题,对
PM
初学者来说水很深,不太建议一转行就进来踩坑到底层,看多了资料真的会盲目,十分受挫。

而我在接触 NLP 产品的早期(入职0–6个月)会有这些工作:

(1)理解用户场景和需求背景:
\begin{itemize}
\item {} 
理解产品本身的需求场景、做同类竞品分析和测试…

\item {} 
进行一些用户调研、访谈、…

\item {} 
思考写匠产品本身核心竞争力是什么,核心功能和辅助功能分配?

\item {} 
产品结构,整体规划…

\end{itemize}

(2)拆解需求,哪些和 AI 相关:
\begin{itemize}
\item {} 
核心功能:要反馈给用户哪些写作建议,如:基于认知科学的xx原理、涉及到的某些文学属性和评价指标、语言风格、…

\item {} 
如何拆解这些需求?基于规则还是基于统计?比如:提取并抽象出各种语言规则…

\item {} 
算法研发过程中有哪些问题?评价指标有哪些?怎么算好指标?如:

\end{itemize}

好指标:
\begin{itemize}
\item {} 
Actionable,可以指导具体行为。

\item {} 
Accessible,简单,容易理解。

\item {} 
Auditable,可以进行验证。

\end{itemize}
\begin{enumerate}
\sphinxsetlistlabels{\arabic}{enumi}{enumii}{}{.}%
\item {} 
FPR\&TNR、TPR

\item {} 
精确率Precision(全文找到错别字是不是有误报的情况)、召回率Recall(全文找到的错别字的数量是不是漏了)、F1值

\item {} 
综合评价指标F\sphinxhyphen{}measure

\item {} 
ROC曲线和AUC

\item {} 
容量和速度(一次可以容纳多少字数、速度如何,会不会卡)

\item {} 
损失函数(MAE:目标值和预测值之差的绝对值之和)

\end{enumerate}
\begin{itemize}
\item {} 
如何提升这些指标?比如:提高数据量…

\item {} 
如何扩大语料库规模和训练数据?

\item {} 
比如:从网上获取语料库,建立通用语料库,专用语料库、各行业的词表…然后处理数据集,数据整理,…

\item {} 
根据需求拆解,算法的输入和输出是什么?数据如何显示?哪些是用户需要,哪些是后台需要?…

\item {} 
新旧版本的算法评测如何对比?比如:

\item {} 
制定算法评测指标,进行评测集标注,做算法评测,分析结果…

\end{itemize}

(3)产品终于从 0 \textasciitilde{} 1 进入到工程开发阶段:
\begin{itemize}
\item {} 
产品逻辑、原型设计、交互设计…

\item {} 
算法、前端、后端的业务逻辑拆分…

\item {} 
团队写作、项目管理、各种开会…

\item {} 
测试用例,功能测试、性能测试、…

\item {} 
版本管理,后续迭代哪些功能…

\item {} 
偶尔申请一些第三方服务…

\item {} 
偶尔组织聚餐和团建… … 这就需要我额外学习:

\item {} 
AI 在 自然语言处理中的应用有哪些,如问答系统、文本摘要、情感分析、…

\item {} 
自然语言处理本身的基础能力,如分词、词性标注、句法分析、…

\item {} 
读一些和 NLP、ML、PM 工作直接或间接相关的书,如:

\item {} 
AI+NLP:《NLP汉语自然语言处理原理与实践》、《机器学习》、《认知语言学》、《面向\sphinxhyphen{}
机器学习的自然语言处理标注》、…

\item {} 
PM:《认知与设计》、《破茧成蝶》、《设计心理学》、…

\item {} 
工作方法要高级批量化,提升认知水平,帮助自己理性思维的认知科学理论,应用到自己的学习中…(推荐老板的公众号:【开智学堂】、【心智工具箱】)

\end{itemize}


\paragraph{困难}
\label{\detokenize{chapter_experience/early_phase:id2}}
上述工作中,最开始对我来说最难的地方是:拆解需求,如何根据目标,提取出抽象规则,制定算法能力的评价标准,…比如:怎么把这些文学底层理论知识变成算法硬性的指标,如果底层的问题没想清楚或者是解决不了,即使把代码写好了,模型训练好了,最后用户也会觉得怎么那么硬邦邦的,很奇怪,就觉得这产品很不实用。而即使有了思路和方法,
在执行过程也发现很大阻力,比如数据各种问题。后来就给自己挖到坑里面了,浪费了很多时间,虽然在这个坑里自己也蛮大收获的。


\subsubsection{行业巨变}
\label{\detokenize{chapter_experience/issue:id1}}\label{\detokenize{chapter_experience/issue::doc}}

\subsubsection{盒马鲜生}
\label{\detokenize{chapter_experience/hema:id1}}\label{\detokenize{chapter_experience/hema::doc}}

\paragraph{观察}
\label{\detokenize{chapter_experience/hema:id2}}
“观察”的方式:

保持空无,抛下预设 ↓ 用客体视角觉察出自己内心与行为的关系 ↓
再试着深入“阅读”他人内心与行为的关系 ↓ 结合规律,分析出外界真实的需要 ↓
在生活与工作中做出策略调整或反应 ↓ 保持练习,达到情商和洞察力的提高


\paragraph{练习“关联性”思维的方式:}
\label{\detokenize{chapter_experience/hema:id3}}
抛开过去那种任何事都想着“自己干”的想法,问自己3个问题:
\begin{enumerate}
\sphinxsetlistlabels{\arabic}{enumi}{enumii}{}{.}%
\item {} 
我现在要做的事情,有没有利他性?

\item {} 
可以不可以与他人形成合力?

\item {} 
最终取得的成果,能不能多方共享?

\end{enumerate}


\bigskip\hrule\bigskip


如果3个问题想清楚了没问题,那么不怕拒绝,厚着脸皮干就完了!

日常要留心,自己和他人身上,有哪些可以“做成事”的资源,这并不是要人学会自利,而是需要培养自己的协作性。自己的专业知识,钱,甚至体力,时间,人脉圈,都是能一起互相协作的资源。

除了人与人的资源关联性,还可以培养物与物相互跨界联系的能力。

比如在阿里,训练公关的新闻策划能力,就有一种称之为“两只试管法”的日常思考方法,你可以想象成左手握一个产品试管,右手握一个情绪试管,然后两种试剂倒在了一起,产生神奇的化学反应。

比如:

盒马鲜生(线下的果蔬生鲜服务设施/一种都市快节奏生活方式)+
房价(情绪饱满的高敏感民生话题)= 品牌概念:盒区房

进口水果 + 北上广的生活压力(情绪饱满的消费焦虑)= 热门话题:车厘子自由


\subsubsection{Jamin}
\label{\detokenize{chapter_experience/Jamin:jamin}}\label{\detokenize{chapter_experience/Jamin::doc}}

\paragraph{GOAL:learn}
\label{\detokenize{chapter_experience/Jamin:goal-learn}}

\paragraph{靠拢互联网}
\label{\detokenize{chapter_experience/Jamin:id1}}

\paragraph{作品证明自己}
\label{\detokenize{chapter_experience/Jamin:id2}}

\paragraph{内推}
\label{\detokenize{chapter_experience/Jamin:id3}}

\subsubsection{taobao}
\label{\detokenize{chapter_experience/taobao:taobao}}\label{\detokenize{chapter_experience/taobao::doc}}
也许张勇最能理解蒋凡,因为他们都是那种,在关键时刻孤独地扮演过“扳道工”角色的人,无论当时对他们来说,自己在不在最重要的位置上。

在蒋凡身上,有着外界所说的“一眼看穿底层逻辑”的能力。也是当下信息爆炸的时代,一种透过乱七八糟的消息迷雾,看到复杂事物中最简单的常识的能力。

“无”招胜有招《笑傲江湖》里风清扬传给令狐冲的第一句话。


\paragraph{拼多多为什么能够在阿里眼皮下迅速崛起呢!?}
\label{\detokenize{chapter_experience/taobao:id1}}
如果说是把握了下沉市场还是流于表面,你用矛盾的观点看本质:

第1点,2015\textasciitilde{}2017年间,大量阿里生态内的小小B端的角色,如底层商家、淘客、羊毛党因为阿里战略调整,对外发生了外溢,这些互联网游牧民走到哪,哪里就形成了新的细小供应链。这些人离开阿里要吃饭啊,这是最主要矛盾。

第2点,低价智能机和微信支付相结合,带来了小镇青年整体电商用户盘子扩大,这些人的日常时间要怎么打发,身边可能连个高级商场都没有,这是次主要矛盾。

这些东西,身处五环内的你在那个年代里,光看数据是不会马上发现的,只有靠细微的洞察才能感知到:

1)快递小哥的包裹里是不是开始有了别的平台的商品?

2)老家父母亲戚的朋友圈,是不是很多东西变了?

3)地方台的的综艺节目里面,广告赞助商是不是出现了不认识的牌子?(可惜很多北上广人不看电视)

4)那些像游牧民族一样的羊毛党,被你屏蔽朋友圈的微商妈妈又在忙什么?

透过现象看本质,拼多多就是抓住了这些要素悄悄长大的。


\paragraph{怎么抓“主要矛盾”}
\label{\detokenize{chapter_experience/taobao:id2}}\begin{enumerate}
\sphinxsetlistlabels{\arabic}{enumi}{enumii}{}{.}%
\item {} 
首先是重新平衡天猫、淘宝的重心,执两用中,平衡“大多数用户”和B端之间的消费和供给,这不是拿捏尺度的平面问题,而是一个对顶层架构重新分析、设计的立体问题。

\item {} 
选用模式更适合五环外市场的聚划算做渠道下沉,向低线城市渗透、并且覆盖全年龄段,尽快封堵挤压拼多多的继续扩张

\item {} 
发力短视频、抖音、网红,直播这些内容场景,再通过大数据精准推送,通过占领用户时间,赢得市场,让B端人群比如主播网红下沉去填补C端的使用手机时间。

\item {} 
带领品牌商家下沉。之前很多品牌集中在打一二线市场,原有的渠道网络对于下沉市场是滞后的。但随着阿里的强势运营,优质的中部商家做敲门砖品牌迅速得以下沉。提前占住山头,让对手仰攻。

\end{enumerate}

随着最近淘宝特价推出,结合淘宝、聚划算、天猫、淘小铺全面出击,阿里军团的刀枪剑戟朝向了同一个方向,B端搭建架构,C端占领时间,蒋凡完成了对北上广人群和下沉市场的一记全垒打!


\subsubsection{tencent}
\label{\detokenize{chapter_experience/tencent:tencent}}\label{\detokenize{chapter_experience/tencent::doc}}

\paragraph{运营转产品}
\label{\detokenize{chapter_experience/tencent:id1}}
运营:短期价值 产品:长期价值


\paragraph{总结}
\label{\detokenize{chapter_experience/tencent:id2}}
心态:认识自己,而非彻底否定 把握可控:自我介绍和提问环节 不设限:多尝试
低谷期:学 规划:把握住运气


\paragraph{简历坑}
\label{\detokenize{chapter_experience/tencent:id3}}

\subparagraph{做}
\label{\detokenize{chapter_experience/tencent:id4}}
更关注工作经历 简历是产品,HR即用户


\subparagraph{投}
\label{\detokenize{chapter_experience/tencent:id5}}
越新越可能,不合适也可以投


\paragraph{面试坑}
\label{\detokenize{chapter_experience/tencent:id6}}
多试试,现场面试


\paragraph{公司}
\label{\detokenize{chapter_experience/tencent:id7}}
小公司:多面手 大公司:专才

{[}1{]}: \sphinxhref{https://zhuanlan.zhihu.com/p/257044198}{从半年低谷期到入职腾讯产品经理,他的经验是什么?\sphinxhyphen{}
知群群星闪耀时}%
\begin{footnote}[689]\sphinxAtStartFootnote
\sphinxnolinkurl{https://zhuanlan.zhihu.com/p/257044198}
%
\end{footnote}


\subsubsection{蚂蚁借呗的金融产品解析}
\label{\detokenize{chapter_experience/ant_jiebei:id1}}\label{\detokenize{chapter_experience/ant_jiebei::doc}}
信用贷款是指以借款人的信誉发放的贷款,借款人不需要提供担保。

其特征就是:借款人无需提供抵押品或第三方担保仅凭自己的信誉就能取得贷款,并以借款人信用程度作为还款保证的。

信用贷成为了众多平台变现的方式:对于变现困难的平台来讲,可以借鉴信用贷的形式,实现金融变现。

一探究竟:
\begin{enumerate}
\sphinxsetlistlabels{\arabic}{enumi}{enumii}{}{.}%
\item {} 
市场规模:艾瑞咨询预测,未来两年互联网消费金融2019年的规模将达到3.4万亿元

\item {} 
蚂蚁借呗:根据风控和准入标准筛选额度,定位为小额、高息,面向支付宝用户,门槛,无抵押。

\item {} 
产品特征:对比传统的流程(线下网点,打印个人信用资料)支付宝申请简便、当日放款、自动还款防忘。

\item {} 
准入条件:门槛–个人实名认证、芝麻信用在600分以上,无在阿里其它平台内有不良记录和纠纷

\item {} 
放款机构:重庆市蚂蚁商城小额贷款有限公司和广发银行股份有限公司合作联合放贷

\item {} 
授信额度:授信额度从1千到30万不等,用户在阿里生态内购物、理财、捐款、转账、上传学历车辆信息等能提高用户的授信额度,用户变更收货地址、多次借贷、核心联系人逾期会降低自己的授信额度。

\item {} 
可用额度:等于或者小于授信金额;授信额度减去已经提取额度,即下次可贷金额

\item {} 
借款期限:非固定期限(按天计算)和固定期限(按月计算:3、6、12个月)。

\item {} 
借款利率:
日计算利率,蚂蚁借呗的借款日利率为0.015\%\sphinxhyphen{}0.6\%的区间,乘以365天即年化5.5\%\sphinxhyphen{}21.9\%。

\item {} 
征信情况:蚂蚁借呗主要依靠人行征信和芝麻信用评断用户的信用情况

\item {} 
贷款审批:流程快慢与数据、分控有关,流程分“借款申请”、“借款审核”与“放款还款”

\item {} 
还款方式:蚂蚁借呗目前只先息后本和每月等额两种还

\item {} 
结清管理:提前部分结清或者全部结清,
一般结清需要根据提前还款金额收一定比例的手续费,蚂蚁借呗目前没有收取提前结清手续费,但是在贷款期限内,提前结清手续费的收取标准也可能产生变化。恶意频繁多次的提前还款,来达到套现、刷分的目的,有可能导致借呗账户额度被降低或关闭。

\item {} 
贷后管理:如果用户有多笔按月借款类贷款,那么蚂蚁统一把还款日固定成统一的一天

\item {} 
逾期管理:还款日扣划或者冻结用户的网商银行账户,支付宝账号以及在阿里平台内其它应收款项,直到用户的欠款还清,否则一直持续的扣划或者冻结相关账号;逾期后蚂蚁借呗的未偿还本金贷款利率按照放款利率提高50\%,如果资金未按贷款用途使用,那么罚息在放款利率水平上提升100\%,未偿还的利息按照罚息利率计算复率;如果用户逾期严重,那么借呗可以通知用户的关联人逾期情况,同时该笔借款会被蚂蚁借呗委托给第三方催收公司和律师事务所,进行电话催收甚至是法律诉讼。同时逾期信息也将会计入芝麻信用和上报至人行征信系统。

\end{enumerate}


\subsubsection{diudiu}
\label{\detokenize{chapter_experience/diudiu:diudiu}}\label{\detokenize{chapter_experience/diudiu::doc}}

\paragraph{求职}
\label{\detokenize{chapter_experience/diudiu:id1}}
一次就够,1/2000:\sphinxurl{https://baike.baidu.com/item/\%E5\%BC\%A0\%E9\%A2\%96/3405319}


\paragraph{动态迭代}
\label{\detokenize{chapter_experience/diudiu:id2}}
了解评价,作品找工作

「快速迭代」:缩短决策与行动之间的周期,先让自己快速动起来。\sphinxhref{https://zhuanlan.zhihu.com/p/146486072}{1}%
\begin{footnote}[690]\sphinxAtStartFootnote
\sphinxnolinkurl{https://zhuanlan.zhihu.com/p/146486072}
%
\end{footnote}


\paragraph{作品核心}
\label{\detokenize{chapter_experience/diudiu:id3}}
核心的用户需求 产品定位 规划


\paragraph{话题引子}
\label{\detokenize{chapter_experience/diudiu:id4}}
分析(用户、需求等)–》落地(设计、输出)


\paragraph{规划路径}
\label{\detokenize{chapter_experience/diudiu:id5}}
不纠结

\sphinxurl{https://izhiqun.com/web/share/experience}


\paragraph{关注潜力}
\label{\detokenize{chapter_experience/diudiu:id6}}

\paragraph{准备}
\label{\detokenize{chapter_experience/diudiu:id7}}

\paragraph{不闭门造车}
\label{\detokenize{chapter_experience/diudiu:id8}}

\subsubsection{钉钉}
\label{\detokenize{chapter_experience/dingding:id1}}\label{\detokenize{chapter_experience/dingding::doc}}
2014年,阿里经历了强推社交产品“来往”的巨大挫折,在智能手机全国开始普及的年代,因为社交用户基数大,而且极度高频的入口级特性,社交产品所能带来的安全感是各大互联网厂商都极度渴望的,所以你可以理解为什么马化腾会把微信横空出世称为:\sphinxstylestrong{抢到第一张移动互联网船票}。

绝望会让一个人抛弃原有的脑子里对世界所有的理解,进入一种\sphinxstylestrong{彻底放空和内省状态},这时候才能静下心来观察和阅读世界真正的需要。

作为一个产品经理可能会反思,任何大而广的东西一定有弱点,如果说微信的社交面是一条横线,需要观察寻找的,是哪里可以诞生一条尚未挖掘的纵线。

静心向内看就会有答案,那就是阿里生态圈的万千小B企业,如果你进入用户的心中去“观察”他们的想法。你就会用心眼看到后面的答案。

钉钉的成功最深处,是在碎片化办公的大环境下,人性中饱含的对深度工作专注和效率的追求。而在这一点上,无论是老板还是员工,只要他还算是
“想做事的人” 那就是共通的!

钉钉所有的员工,入职后第一课就是被要求,放下已知,带着空杯进入那些小B企业中,同工同吃,“观察”和阅读用户内心真正的需要。


\subsubsection{4年产品工作总结}
\label{\detokenize{chapter_experience/4years:id1}}\label{\detokenize{chapter_experience/4years::doc}}
关于学习成长,关于交流分享,关于职场选择。


\subsubsection{剧本杀}
\label{\detokenize{chapter_experience/jubensha:id1}}\label{\detokenize{chapter_experience/jubensha::doc}}
“剧本杀”最初源自线下游戏“谋杀之谜”,是一款 LARP
(实时角色扮演)游戏。不同游戏的剧本内容各不相同,但是玩法基本大同小异:

游戏开始阶段,每一名玩家选择扮演剧本中的一个角色,其中有一名玩家会在其他玩家不知情的情况下\sphinxstylestrong{扮演凶手},其他玩家需要在故事情节以及所搜寻到的证据的分析推理下,\sphinxstylestrong{共同找出真凶}。

剧本杀多是以封闭和半封闭半开放的剧本为主。封闭的剧本好比爬楼梯,一步步的探索最终获得事件的真相;开放的剧本就好比寻宝,一丝丝的痕迹与线索拼凑在一起得知最后的事实。


\paragraph{起源与走红1\sphinxfootnotemark[691]}
\label{\detokenize{chapter_experience/jubensha:id2}}%
\begin{footnotetext}[691]\sphinxAtStartFootnote
\sphinxnolinkurl{http://www.woshipm.com/it/1374466.html}
%
\end{footnotetext}\ignorespaces 
起源于风靡欧美的线下派对,2016年首档明星推理综艺秀\sphinxstylestrong{《明星大侦探》}第三季播出结束后播放量突破32亿次,收获了大批“剧本杀”粉丝。

2018年上半年,随着几款连麦推理社交游戏的上架,“剧本杀”迅速走红


\paragraph{游戏环节分析}
\label{\detokenize{chapter_experience/jubensha:id3}}
五个环节:
\begin{enumerate}
\sphinxsetlistlabels{\arabic}{enumi}{enumii}{}{.}%
\item {} 
选择角色:共侦探、嫌疑人、真凶。进入游戏,默认进入语音群聊,并选择想要扮演的剧本角色房间后,此阶段时间较短,待所以选角完成,进入下一阶段

\item {} 
人物剧本:所有玩家阅读各自人物剧本,剧本内容包含人物过往经历、重要线索、时间点等情节。剧本阅读完毕,玩家们在语音群聊房间内依次自我介绍,并讲述过往经历及重要线索。尽量入戏,你就是这个人物,不要莫得感情的读剧本。\sphinxhref{https://www.murdermysterypa.com/thread-3693-1-1.html}{2}%
\begin{footnote}[692]\sphinxAtStartFootnote
\sphinxnolinkurl{https://www.murdermysterypa.com/thread-3693-1-1.html}
%
\end{footnote}

\item {} 
搜集线索:所有玩家共同阅读线索信息,同时就具体线索展开讨论。以发问、闲聊等方式透露给其他玩家(关键信息可隐瞒)。此阶段包含对多个角色的线索搜集及玩家讨论(可以选择分享以甩锅或保留以掩盖)的过程,直至所有的线索搜集并讨论完成,方可结束。桌游版剧本杀通常将证据卡片分布在不同场景和人物身上,每位玩家对每个场景和人物有搜索限制(每个本不一样)。\sphinxhref{https://www.zhihu.com/question/270386766/answer/615240050}{5}%
\begin{footnote}[693]\sphinxAtStartFootnote
\sphinxnolinkurl{https://www.zhihu.com/question/270386766/answer/615240050}
%
\end{footnote}严格以上,只是为玩家提供一些思路去范围缩小或怀疑对象!误导类、决定性类、无用类和一般线索,别轻易定性,思考怎样串联完整的证据链\sphinxhref{https://zhuanlan.zhihu.com/p/66137913}{8}%
\begin{footnote}[694]\sphinxAtStartFootnote
\sphinxnolinkurl{https://zhuanlan.zhihu.com/p/66137913}
%
\end{footnote}。

\item {} 
圆桌讨论:搜证阶段结束后,玩家们再次共同讨论并寻找故事真凶,此阶段\sphinxstylestrong{不提供}任何剧本及线索,所以各玩家首先需要\sphinxstylestrong{回忆}之前各个阶段的逻辑推理再作讨论。因为之后会共同投票,更加鼓励玩家把自己的想法和推理讲出来,让其他玩家信服

\item {} 
最终投票:就圆桌阶段讨论的结果,玩家们共同投票,选择凶手。投票阶段一般会计时,玩家需要在计时结束前完成投票。投票结束后,公布结局真相,并给出答案解析,游戏结束。

\end{enumerate}

目前主流的“剧本杀”游戏,玩家的主体精力基本放在“搜集线索”环节,需要进行多次搜证并对线索加以讨论。在房间人数上大多采用多人局,最多支持8人,其中《我是谜》app还提供了1\sphinxhyphen{}2人局,在极大程度解决了用户匹配的问题。游戏基本采用纯音频形式进行游戏,未来或许会出现类似“狼人杀”的视频面杀形式。


\paragraph{世界观→动机→逻辑→线索4\sphinxfootnotemark[695]}
\label{\detokenize{chapter_experience/jubensha:id4}}%
\begin{footnotetext}[695]\sphinxAtStartFootnote
\sphinxnolinkurl{https://www.zhihu.com/question/270386766/answer/692364483}
%
\end{footnotetext}\ignorespaces \begin{itemize}
\item {} 
世界观:隐藏在背景故事和个人故事里,从故事的完整性上,更好对照人物关系,动机。

\item {} 
人物:服装(带面具,面纱,口罩,有胡子)、发色、服装(样式,颜色)、身份(和死者事件的关系及目的)
凶手:
\sphinxstyleemphasis{禁忌}:群攻(没有逻辑地拉仇恨)、乱私聊(暴露关系)、线索(作案销毁物证、第一时间搜)、时间线(读剧本作案卡壳)、多嘴(爆出当时只有凶手才知道)。\sphinxstyleemphasis{正确}:栽赃第二嫌疑人(引诱其他玩家盘问、隔山观虎斗)、骗拉友军(洗清别人嫌疑、后期隐藏)、散布假证据、沉着冷静流畅地编(记下其他角色作案时间,别冲突\sphinxhref{https://zhuanlan.zhihu.com/p/66137913}{8}%
\begin{footnote}[696]\sphinxAtStartFootnote
\sphinxnolinkurl{https://zhuanlan.zhihu.com/p/66137913}
%
\end{footnote})

\item {} 
动机:分外、内因。路人(双眼能佐证所有人的时间线、驳不在场证明)、动机证人(知道过去别人不知道的)、昏迷失忆(剧本交互导致的)

\item {} 
逻辑:无故放弃杀害、动机和行为不符。是个人故事、时间线,动机的综合体。

\item {} 
线索:不会骗人,佐证你的逻辑线,动机,世界观。明辨线索为一次行为形成的还是多次行为形成的。注意人设是否前后不一。关键信息:尸体死状,从伤口等找到凶器,\sphinxstylestrong{注意}多种凶器都有可能造成,结合时间线和动机综合分析!

\end{itemize}


\paragraph{用户分析}
\label{\detokenize{chapter_experience/jubensha:id5}}
“剧本杀”游戏的玩家群体,从技术竞技性、社交追求度和游戏排他性可以大致分为以下几类:竞技型玩家、社交型玩家、语音型玩家。

群体细分:
\begin{itemize}
\item {} 
竞技性玩家:喜爱游戏本身,极度追求技术,对游戏剧本的内容,游戏结果以及整体流程体验会更加看重

\item {} 
社交型玩家:青睐社交属性较重,热爱结交朋友,更看重社交方面的体验

\item {} 
语音型玩家:热爱包含语音连麦功能的游戏,更关注连麦时语音的质量

\end{itemize}


\paragraph{游戏关键要素}
\label{\detokenize{chapter_experience/jubensha:id6}}
线下游戏迁移至线上app
\begin{itemize}
\item {} 
好评:例如,“玩法很新颖”、“有一种自己演电影的感觉”、“很锻炼逻辑思维能力”

\item {} 
问题:例如,“希望剧本的筛选能更用心”、“有些人麦的回音很大”、“无法闭麦”等。

\end{itemize}
\begin{enumerate}
\sphinxsetlistlabels{\arabic}{enumi}{enumii}{}{.}%
\item {} 
剧本内容:依靠粉丝网友的投稿,奖励机制吸引优质写手,签约长期合作的剧本作者。合理精彩的剧本会带来更好的游戏体验

\item {} 
实时互动:除了阅读剧本,搜集线索的过程,90\%以上的游戏体验核心都在语音聊天上。网络不稳定等原因导致语言不通畅

\item {} 
音质效果:因为语音连麦聊天的问题,杂音、丢音、回声、噪声等问题严重影响游戏体验

\item {} 
稳定流畅:网络异常问题导致的游戏过程断续,音视频技术可调用第三方的SDK(满足快速上线与稳定)

\end{enumerate}


\paragraph{对产品经理的好处3\sphinxfootnotemark[697]}
\label{\detokenize{chapter_experience/jubensha:id7}}%
\begin{footnotetext}[697]\sphinxAtStartFootnote
\sphinxnolinkurl{http://www.woshipm.com/pmd/3064843.html}
%
\end{footnotetext}\ignorespaces \begin{enumerate}
\sphinxsetlistlabels{\arabic}{enumi}{enumii}{}{.}%
\item {} 
锻炼逻辑思维能力、辩论能力:以理服人

\item {} 
锻炼独立思考、自我判断能力:辨别伪装

\item {} 
锻炼记忆力:沟通需求不遗漏,后置位发言的人很有优势的原因。

\item {} 
锻炼随机应变能力:应对措手不及的突击、质问、威胁\sphinxhref{https://www.zhihu.com/question/270386766/answer/415647339}{7}%
\begin{footnote}[698]\sphinxAtStartFootnote
\sphinxnolinkurl{https://www.zhihu.com/question/270386766/answer/415647339}
%
\end{footnote}

\item {} 
锻炼团队配合能力:协调好团队,以共同决策。信息闭塞

\item {} 
锻炼辨别真伪需求能力:分辨表面和内在实质

\item {} 
交朋友:放松工作压力的同时社交

\item {} 
锻炼撒谎能力:见人说人话见鬼说鬼话。

\item {} 
锻炼抗压能力:项目组其他岗位(设计、开发、测试、运营)的,随时都可能并发产生。破绽后洗白。

\item {} 
锻炼控制情绪的能力:以理服人,防自爆(抗压力是所有软实力的基础。)

\item {} 
锻炼表演能力:换角度思考别人的问题,才能演好自己的角色

\item {} 
锻炼观察能力:表情不自然、言行怪异\sphinxhref{https://www.zhihu.com/question/270386766/answer/655939057}{6}%
\begin{footnote}[699]\sphinxAtStartFootnote
\sphinxnolinkurl{https://www.zhihu.com/question/270386766/answer/655939057}
%
\end{footnote}

\item {} 
锻炼交流能力:避免信息闭塞的自我相信。追问、私聊、讨论

\end{enumerate}


\subsubsection{失败}
\label{\detokenize{chapter_experience/fail:id1}}\label{\detokenize{chapter_experience/fail::doc}}

\paragraph{别死}
\label{\detokenize{chapter_experience/fail:id2}}
互联网本身竞争就激烈,往往是老大和老二打架,最后死的却是老三,前几年风光无限的明星产品经理,

过几年之后就沦为一个故事家,出来打情怀牌,吃老本的人也大把大把的存在。

这还是好的情况,很多做了N年的产品经理,本职工作并不能有很好的突破,行业不行,公司不行,自己空有改变世界的心,可惜最后也沦为了重复造轮子的工具人。

但是不可否认,那些处在金字塔尖,动不动做着行业排名数一数二产品经理,日子还算过得滋润。

那些没有成功的产品经理也并不代表他们不优秀,可能就差一次机会而已。

现在之所以会出现一些唱衰产品经理的声音,主要是和疫情经济影响和移动互联网增长乏力带来的双重影响。


\paragraph{「改变世界」到底是什么呢?}
\label{\detokenize{chapter_experience/fail:id3}}
解决掉一个用户烦恼的问题,就是在改变世界。


\subsubsection{AI PM的隐性工作}
\label{\detokenize{chapter_experience/recessive_work:ai-pm}}\label{\detokenize{chapter_experience/recessive_work::doc}}

\paragraph{清晰界定}
\label{\detokenize{chapter_experience/recessive_work:id1}}
面试官希望你有“能力”,可能很多人就理解为要有优势;再比如,面试官希望你懂做直播,可能就被理解为做“网红”带货。同样,还有很多人认为对电商运营很懂,因为平时自己也帮朋友出主意,帮朋友做一些淘宝店铺的运营,但是真正的工作远比这些要复杂。
\sphinxhref{https://weread.qq.com/web/reader/46532b707210fc4f465d044kb6d32b90216b6d767d2f0dc}{2}%
\begin{footnote}[700]\sphinxAtStartFootnote
\sphinxnolinkurl{https://weread.qq.com/web/reader/46532b707210fc4f465d044kb6d32b90216b6d767d2f0dc}
%
\end{footnote}


\paragraph{隐性部分}
\label{\detokenize{chapter_experience/recessive_work:id2}}
「产品经理的工作包括显性部分和隐性部分。为外界所知的显性部分,通常其实只占这个工作的十分之一左右;不为外人道的隐性部分,则占了这个工作的十分之九甚至更多。」

要掌握 AI PM 的软技能,要先关注工作本身和产品的细节。
\sphinxhref{https://medium.com/@liwdai/ai-pm-\%E4\%B9\%8B\%E9\%9A\%90\%E6\%80\%A7\%E9\%83\%A8\%E5\%88\%86\%E7\%9A\%84\%E5\%B7\%A5\%E4\%BD\%9C-be6de08d1c05}{1}%
\begin{footnote}[701]\sphinxAtStartFootnote
\sphinxnolinkurl{https://medium.com/@liwdai/ai-pm-\%E4\%B9\%8B\%E9\%9A\%90\%E6\%80\%A7\%E9\%83\%A8\%E5\%88\%86\%E7\%9A\%84\%E5\%B7\%A5\%E4\%BD\%9C-be6de08d1c05}
%
\end{footnote}


\paragraph{隐性工作 – AI}
\label{\detokenize{chapter_experience/recessive_work:ai}}
一般来说,偏 AI 的隐性部分的工作主要包括以下几个方面:
\begin{enumerate}
\sphinxsetlistlabels{\arabic}{enumi}{enumii}{}{.}%
\item {} 
制定产品策略和算法能力的边界;

\item {} 
梳理算法研发前的准备阶段的工作流程;

\item {} 
理解数据标注和算法原理的基本常识。

\end{enumerate}


\subparagraph{制定产品策略和算法能力的边界}
\label{\detokenize{chapter_experience/recessive_work:id3}}
我最近在上一个策略产品课,对 AI PM 的工作范围划分有了新的认知。AI PM
除了做项目管理、设计用户可感知的功能以外,还需要制定用户不可感知的、返回数据的策略逻辑,例如:搜索策略、分析策略、推荐策略、展示策略等。

下面以我做的产品的校对功能为例,思考 AI PM
在制定策略和分析算法能力边界中的常见问题。

(1)首先,在错误列表的展示策略中,应该给用户返回什么类型的「错误」?
\begin{itemize}
\item {} 
算法需要检查出哪些类别的错误,明显的错别字当然属于错误,但是对于火星文、不规范中英文混合、口语经常出现的句式杂糅、新捏造词汇、数字格式乱用等情况,也属于错误的范畴吗?

\item {} 
这些「错误」是否是用户关心的错误?是否全部都提示给用户?

\item {} 
这些「错误」要基于什么标准下的语言规范,来划分错误的层级?

\item {} 
有些经常无法被检查出来的「上下文语义错误」,如何制定这些规则?如果没有固定的规则,是否有其他的处理方式?

\end{itemize}

(2)应该提供何种操作体验?
\begin{itemize}
\item {} 
在文中出现相同错误时,「批量修改」和「单次修改」中应该选择哪种方案?

\item {} 
实时检查文章时,用户在打字过程中即可看到实时检测结果。可是如何判断用户的输入状态,是输入全篇文章、一段还是一句?此外,该如何设定实时检查的间隔时间,不同时间的设定会有什么影响?在分析各种影响因素后,再选择合适的方案。

\end{itemize}

(3)系统对用户操作应作出怎样的反馈?
\begin{itemize}
\item {} 
对检查出来的错误词,算法匹配了正确词用于替换或修改参考。此功能是否提供多个修改词?

\item {} 
错误词被用户首次修改后,系统是否记住用户的此次修改?还需要考虑的是,当该词在其他语境中出现且并不是错误时,又该如何处理?

\end{itemize}


\subparagraph{梳理算法研发前的准备阶段的工作流程}
\label{\detokenize{chapter_experience/recessive_work:id4}}
如果是从 0 到 1
的产品,在算法研发之前,首先需要收集语料库或标注数据,用于算法工程师做模型训练。

在 AI PM 接到需求的准备阶段,需要将任务安排到每个人的执行细节。但梳理 AI
工作流程或拆解任务的方法,并不能在资料中学到。刚开始我也只能摸爬滚打地不断尝试,后来有了一定的经验后,才制定了适合团队工作的标准流程。

下面举例说说,为了提升产品校对功能的算法精度(召回率),我对工作任务进行了拆解。主要采取扩大规则库的手段,去收集语料库、数据处理、制定规则等。根据这个目标,具体的拆解步骤如下:
\begin{enumerate}
\sphinxsetlistlabels{\arabic}{enumi}{enumii}{}{.}%
\item {} 
首先根据需求,要确定需要的语料库类型以及购买渠道。(这活是由我和算法工程师一起讨论确认的)

\item {} 
将购买好的数据入库到云端 or
其他保存数据的地方,比如移动硬盘。(这是由另一位数据同事负责)

\item {} 
数据下载完成后,查看具体数据,分析整理有用数据和需要清理的无效数据。(为了不浪费算法工程师的时间,基本是我看数据做的分析)

\item {} 
数据预处理:处理垃圾语料、整理文件、对数据进行分类。(查阅 NLP
语料库处理方式的方法 + 和算法工程师讨论方案后,我做的整理)

\item {} 
提取本次任务所需语料库,数据清洗,过滤杂质(技术同事根据我整理的规范,写脚本处理)

\item {} 
算法如何利用这些数据制定语言规则或训练不同的模型。(这就不关我的事儿了)

\item {} 
训练集抽样标注(PM
写标注文档,抽时间和标注团队对数据抽样标,能看出很多细节问题,及时反馈给算法工程师处理)

\item {} 
测试集标注(同上)

\item {} 
算法跑测试脚本后,生成评测报告(和算法工程师讨论需要哪些指标,以及测试方法,主要由算法工程师来定)

\item {} 
测试报告分析结果(算法和 PM
从技术和产品的角度分别写测试结果的分析报告)

\item {} 
根据分析报告,将实验结果中有用的规则(符合用户场景的常见错误)写入规则库。(这部分主要是算法工程师负责)

\item {} 
算法版本升级,产品更新(PM 负责产品版本管理,写发布报告)

\item {} 
将数据处理方法和标注\sphinxhyphen{}测试方法等写到资料仓库的 wiki 的 handbook
手册中,方便下次遇到同样任务时,可以复用流程。

\end{enumerate}


\subparagraph{理解数据标注和算法原理的基本常识}
\label{\detokenize{chapter_experience/recessive_work:id5}}
大部分的 PM 其实不需要了解这些,数据标注知识对有标注需求的 AI PM
或技术底层的 AI 算法 PM 来说,更有用些。

插一句题外话,入门 AI
PM,最简单就是数据标注,因为可以通过数据来理解基本的算法原理和常识,PM
从项目具体行动中去做,去学,才能理解背后的规则。 这里以我在商汤做的 CV
标注为例:刚开始写标注文档的时候发现,标注团队会问我很多问题。因为刚开始我并不熟悉标注和算法原理,因此我需要花很多时间去咨询算法工程师,理解每一个特殊的
case
要怎么标。这样来回沟通不仅浪费大家的时间,也导致由于数据标错而产生的多余返工时间。所以,理解原理能帮助产品做得更好。下面举两个标注的例子:

(1)OCR 营业执照图片标注出现模糊情况,AI PM
先要理解机器识别和肉眼识别的能力区别。在不理解的情况下,标注文档会写不清楚,导致标注员不知道怎么标,甚至出现数据大批量标错,算法的质量下降
or 测试结果不准确的后果。

(2)真实场景中会有各种情况的数据,每种情况都需要写清楚数据的类别与问题边界,同时给出每种类型的标注的详细定义。如果标注定义不清晰,将会影响算法的精度。


\paragraph{隐性工作 — PM}
\label{\detokenize{chapter_experience/recessive_work:pm}}
一般来说,偏 PM 的隐形部分的工作主要包括以下几个方面:
\begin{enumerate}
\sphinxsetlistlabels{\arabic}{enumi}{enumii}{}{.}%
\item {} 
控制项目进度,按时交付产品;

\item {} 
团队协作与沟通;

\item {} 
产品开发周期中的问题处理。

\end{enumerate}


\subparagraph{控制项目进度,按时交付}
\label{\detokenize{chapter_experience/recessive_work:id6}}
首先 AI PM
要根据需求背后的用户目标排需求优先级,制定相应的解决方案。在解决方案落地的过程中,梳理不同部门和产品结构的关系,向各部门提出需求,控制整体的项目进度,提前预测产品在开发周期中可能面临的风险。

以我之前在商汤做的大型 AI
项目为例:当时的情况是,客户要求的时间很紧,客户的测试通过指标是关键字段的识别准确率不低于98\%。根据这个目标,我梳理了工作流程后,将各个部门的工作流程和完成进度都严格控制在下图的这张进度表上,并且每日跟进具体问题,及时应对风险。

\begin{center}\sphinxincludegraphics{{progress_chart}.jpeg}\end{center} \sphinxincludegraphics{{progress_chart2}.jpeg}
\begin{description}
\item[{一般来说,toB 的 AI PM 要非常\sphinxstylestrong{严格地控制项目进度},按时交付产品;同时需要在项目管理中的对风险做出预判,因为从设计到开发的过程中会遇到各种问题,就需要AI PM全程跟踪和参与,甚至身体力行地去做一些事推进整个项目进展。}] \leavevmode
额外提一下,有时 AI PM
要\sphinxstylestrong{同时处理多个问题,注意控制自己的时间分配};还需要清楚在什么阶段要做什么事,哪些必须提前做,哪些可以后置;项目管理的目的是保证产品的进展,确定按时交付。

\end{description}


\subsection{项目实践}
\label{\detokenize{chapter_project/index:chap-project}}\label{\detokenize{chapter_project/index:id1}}\label{\detokenize{chapter_project/index::doc}}

\subsubsection{AI行业分析}
\label{\detokenize{chapter_project/AI_industry_analysis:ai}}\label{\detokenize{chapter_project/AI_industry_analysis::doc}}

\paragraph{PEST}
\label{\detokenize{chapter_project/AI_industry_analysis:pest}}

\subparagraph{政治}
\label{\detokenize{chapter_project/AI_industry_analysis:id1}}
2021年作为新十四五规划开局之年,在新规划建议中,多次提及人工智能这个关键词,而且科技前沿领域的公关,排在第一位的就是“新一代人工智能”\sphinxhref{https://www.weiyangx.com/382066.html}{16}%
\begin{footnote}[702]\sphinxAtStartFootnote
\sphinxnolinkurl{https://www.weiyangx.com/382066.html}
%
\end{footnote}

\sphinxhref{http://www.gov.cn/zhengce/content/2017-07/20/content\_5211996.htm}{《新一代人工智能发展规划》}%
\begin{footnote}[703]\sphinxAtStartFootnote
\sphinxnolinkurl{http://www.gov.cn/zhengce/content/2017-07/20/content\_5211996.htm}
%
\end{footnote}


\subparagraph{技术}
\label{\detokenize{chapter_project/AI_industry_analysis:id2}}
\begin{figure}[H]
\centering
\capstart

\noindent\sphinxincludegraphics{{AI_tech_map}.png}
\caption{人工智能技术场景体系层级划分(2020)\sphinxhref{https://www.weiyangx.com/356538.html}{17}\sphinxfootnotemark[704]}\label{\detokenize{chapter_project/AI_industry_analysis:id28}}\end{figure}
%
\begin{footnotetext}[704]\sphinxAtStartFootnote
\sphinxnolinkurl{https://www.weiyangx.com/356538.html}
%
\end{footnotetext}\ignorespaces 
\sphinxhref{http://www.stdaily.com/cxzg80/kejizixun/2019-02/19/content\_750862.shtml}{CB
Insights调研出2019年人工智能行业25大趋势}%
\begin{footnote}[705]\sphinxAtStartFootnote
\sphinxnolinkurl{http://www.stdaily.com/cxzg80/kejizixun/2019-02/19/content\_750862.shtml}
%
\end{footnote}

\sphinxhref{http://www.etimeweekly.com/2021/03/11/ai\%E5\%9F\%BA\%E7\%A1\%80\%E8\%AE\%BE\%E6\%96\%BD\%E5\%B8\%82\%E5\%9C\%BA2021\%E5\%B9\%B4\%E5\%85\%A8\%E7\%90\%83\%E6\%B4\%9E\%E5\%AF\%9F\%E5\%8A\%9B\%E5\%92\%8C\%E4\%B8\%9A\%E5\%8A\%A1\%E5\%9C\%BA\%E6\%99\%AF-oracle\%EF\%BC\%8Cmicrosoft\%EF\%BC\%8Cintel-c/}{全球AI基础设施市场规模,现状和预测2021\sphinxhyphen{}2026}%
\begin{footnote}[706]\sphinxAtStartFootnote
\sphinxnolinkurl{http://www.etimeweekly.com/2021/03/11/ai\%E5\%9F\%BA\%E7\%A1\%80\%E8\%AE\%BE\%E6\%96\%BD\%E5\%B8\%82\%E5\%9C\%BA2021\%E5\%B9\%B4\%E5\%85\%A8\%E7\%90\%83\%E6\%B4\%9E\%E5\%AF\%9F\%E5\%8A\%9B\%E5\%92\%8C\%E4\%B8\%9A\%E5\%8A\%A1\%E5\%9C\%BA\%E6\%99\%AF-oracle\%EF\%BC\%8Cmicrosoft\%EF\%BC\%8Cintel-c/}
%
\end{footnote}


\subparagraph{2020 AI Hype Cycle}
\label{\detokenize{chapter_project/AI_industry_analysis:ai-hype-cycle}}
\begin{figure}[H]
\centering
\capstart

\noindent\sphinxincludegraphics{{2020_AI_Gartner}.png}
\caption{2020人工智能技术成熟度曲线报告}\label{\detokenize{chapter_project/AI_industry_analysis:id29}}\end{figure}

新内容:
\sphinxhref{http://www.iotworld.com.cn/html/News/202009/31046f2ae4fd6885.shtml}{4}%
\begin{footnote}[707]\sphinxAtStartFootnote
\sphinxnolinkurl{http://www.iotworld.com.cn/html/News/202009/31046f2ae4fd6885.shtml}
%
\end{footnote}
\begin{enumerate}
\sphinxsetlistlabels{\arabic}{enumi}{enumii}{}{.}%
\item {} 
健康护照(健康码)

\item {} 
形成性人工智能:一种能够用动态变更对情况作出响应的AI。比如可随时间动态适应的AI,以及可生成新颖的模型来解决特定问题的技术等。

\item {} 
人工智能增强设计

\item {} 
复合人工智能

\item {} 
嵌入式人工智能

\item {} 
生成性人工智能:一种可以创建新颖内容(图像,视频等),或者变更已有内容的AI。新生成的产物跟原始的很像,但不完全相同。这项技术可生成深度伪造的内容,可能会衍生出严重的假信息并带来名誉风险,预计在未来五年内,伪造内容会越来越多。\sphinxhref{https://www.gartner.com/cn/information-technology/articles/5-trends-drive-the-gartner-hype-cycle-for-emerging-technologies-2020}{8}%
\begin{footnote}[708]\sphinxAtStartFootnote
\sphinxnolinkurl{https://www.gartner.com/cn/information-technology/articles/5-trends-drive-the-gartner-hype-cycle-for-emerging-technologies-2020}
%
\end{footnote}

\item {} 
负责任的人工智能

\item {} 
人工智能增强开发

\item {} 
自我监督学习

\item {} 
小数据

\item {} 
复合型AI

\end{enumerate}

删除: \sphinxhref{https://moore.live/news/247633/detail/}{7}%
\begin{footnote}[709]\sphinxAtStartFootnote
\sphinxnolinkurl{https://moore.live/news/247633/detail/}
%
\end{footnote}

与去年相比,Gartner将13种技术删除、重新分类或者转移到其他技术曲线中,例如今年Gartner将支持VPA的无线扬声器从所有曲线中删除;AI开发人员工具包现在被分类到AI开发人员和教学工具包类别下;AI
PaaS现在属于AI云服务;与AI相关的C\&SI服务、AutoML、可解释AI(2020年划归到负责任的AI类别下)、
图形
分析、强化学习移至2020年数据科学和机器学习技术成熟度曲线中;会话式用户界面、
语音识别
、虚拟助理转至2020年自然语言技术成熟度曲线中;量子计算移至2020年计算基础设施技术成熟度曲线中;机器人流程自动化软件从AI技术成熟度曲线中删除。


\subparagraph{人工智能工程化(AI Engineering) 5\sphinxfootnotemark[710]}
\label{\detokenize{chapter_project/AI_industry_analysis:ai-engineering-5}}%
\begin{footnotetext}[710]\sphinxAtStartFootnote
\sphinxnolinkurl{https://www.gartner.com/cn/newsroom/press-releases/2021-top-strategic-technologies-cn}
%
\end{footnotetext}\ignorespaces 
Gartner的研究表明,只有53\%的项目能够从人工智能(AI)原型转化为生产。首席信息官和IT领导者发现,由于缺乏创建和管理生产级人工智能管道的工具,人工智能项目的扩展难度很大。为了将人工智能转化为生产力,就必须转向人工智能工程化这门专注于各种人工智能操作化和决策模型(例如机器学习或知识图)治理与生命周期管理的学科。

人工智能工程化立足于三大核心支柱:\sphinxstylestrong{数据运维、模型运维和开发运维}。强大的人工智能工程化策略将促进人工智能模型的性能、可扩展性、可解释性和可靠性,完全实现人工智能投资的价值。


\subparagraph{渐入冷静期}
\label{\detokenize{chapter_project/AI_industry_analysis:id3}}
最近两年 AI
的发展其实是呈现一个冷静趋势,非常多的明星企业面临了衰落和倒闭。包括
Anki Vector Robot, 芯片新星 Wave computing 以及吴恩达夫妇投资的
drive.ai,还有大量自动驾驶,头戴式 VR
眼镜和聊天机器人公司,都倒在了上市之前的道路上。从 AI
投资的角度来看,前几年投资人主要关注这个团队有没有明星的科学家,例如有没有图灵奖获得者,有没有大牛存在,然后就更关注算法效果是不是国际领先的,再到这两年,大家更关注的是有没有落地的案例,有没有客户,以及公司的收入如何。\sphinxhref{https://www.infoq.cn/article/Vw5WdUPVIZd0tVFdgBae}{14}%
\begin{footnote}[711]\sphinxAtStartFootnote
\sphinxnolinkurl{https://www.infoq.cn/article/Vw5WdUPVIZd0tVFdgBae}
%
\end{footnote}


\paragraph{市场}
\label{\detokenize{chapter_project/AI_industry_analysis:id4}}

\subparagraph{市场规模}
\label{\detokenize{chapter_project/AI_industry_analysis:id5}}
比如,深圳市人工智能行业协会发布的《2019
人工智能产业发展白皮书》中提出:预计到 2020 年,我国人工智能市场规模约
990
亿元。如果你是人工智能行业的产品经理,就可以通过整体的规模来倒推自己产品的规模了。
\sphinxhref{https://www.zhihu.com/pub/reader/119980992/chapter/1284104620428685312}{9}%
\begin{footnote}[712]\sphinxAtStartFootnote
\sphinxnolinkurl{https://www.zhihu.com/pub/reader/119980992/chapter/1284104620428685312}
%
\end{footnote}

据德勤报告预计,2025年世界人工智能市场规模将超过6万亿美元,2017\sphinxhyphen{}2025年CAGR达30\%;IDC预计2020年中国人工智能市场规模达42.5亿美元,年增长率达51.5\%,2023年市场规模达119亿美元。\sphinxhref{http://pdf.dfcfw.com/pdf/H3\_AP202007081390272095\_1.pdf}{18}%
\begin{footnote}[713]\sphinxAtStartFootnote
\sphinxnolinkurl{http://pdf.dfcfw.com/pdf/H3\_AP202007081390272095\_1.pdf}
%
\end{footnote}


\subparagraph{产业格局}
\label{\detokenize{chapter_project/AI_industry_analysis:id6}}
\begin{center}\sphinxincludegraphics{{AI_industry_KG}.png}\end{center} \sphinxincludegraphics{{2020_AI}.png}\sphinxhref{http://www.woshipm.com/pd/873240.html}{1}%
\begin{footnote}[714]\sphinxAtStartFootnote
\sphinxnolinkurl{http://www.woshipm.com/pd/873240.html}
%
\end{footnote}
\begin{center}\sphinxincludegraphics{{iresearch_AI}.png}\end{center}
{\color{red}\bfseries{}|AI\sphinxhyphen{}100\sphinxhyphen{}startup\sphinxhyphen{}2020\textbackslash{} |\textbackslash{} `12 <https://www.cbinsights.com/research/2020\sphinxhyphen{}top\sphinxhyphen{}100\sphinxhyphen{}ai\sphinxhyphen{}startups\sphinxhyphen{}where\sphinxhyphen{}are\sphinxhyphen{}they\sphinxhyphen{}now/>`\_\_
|Most\sphinxhyphen{}Valuable\sphinxhyphen{}AI\sphinxhyphen{}Startups\sphinxhyphen{}V3\textbackslash{} |\textbackslash{} `13 <https://www.cbinsights.com/research/most\sphinxhyphen{}valuable\sphinxhyphen{}private\sphinxhyphen{}ai\sphinxhyphen{}companies/>`\_\_
|AI领先企业主要投资领域|} \begin{center}\sphinxincludegraphics{{AI_Unicorn}.png}\end{center}


\subparagraph{发展报告}
\label{\detokenize{chapter_project/AI_industry_analysis:id7}}\begin{itemize}
\item {} 
\sphinxhref{https://www.aminer.cn/research\_report/5de27b53af66005a44822b12}{《2019人工智能发展报告》}%
\begin{footnote}[715]\sphinxAtStartFootnote
\sphinxnolinkurl{https://www.aminer.cn/research\_report/5de27b53af66005a44822b12}
%
\end{footnote}

\item {} 
\sphinxhref{http://report.iresearch.cn/report/202004/3548.shtml}{2020年中国AI基础数据服务行业研究报告}%
\begin{footnote}[716]\sphinxAtStartFootnote
\sphinxnolinkurl{http://report.iresearch.cn/report/202004/3548.shtml}
%
\end{footnote}

\item {} 
\sphinxhref{https://www.ccidgroup.com/info/1105/32595.htm}{赛迪展望 |
一文了解“2021年中国人工智能产业发展趋势”}%
\begin{footnote}[717]\sphinxAtStartFootnote
\sphinxnolinkurl{https://www.ccidgroup.com/info/1105/32595.htm}
%
\end{footnote}

\end{itemize}


\subparagraph{信息途径}
\label{\detokenize{chapter_project/AI_industry_analysis:id8}}
机器之心、量子位
\sphinxhref{https://blog.csdn.net/Dylan\_zhijing/article/details/107548246}{3}%
\begin{footnote}[718]\sphinxAtStartFootnote
\sphinxnolinkurl{https://blog.csdn.net/Dylan\_zhijing/article/details/107548246}
%
\end{footnote};然后还有易观智库和
QuestMobile 的一些调研报告、\sphinxurl{https://www.itsiwei.com/category/ai}


\paragraph{优势:极快、极简}
\label{\detokenize{chapter_project/AI_industry_analysis:id9}}
人工智能可以处理人1秒中可以想出答案的问题,这个问题还需要有以下几个特点:大规模,重复性,限定领域,快速反馈。

人工智能产品设计要以操作极度简单为标准,但是前端的简单代表后端的复杂,系统越复杂,才能越智能。

同样,人工智能的发展依赖于产业生态的共同推进,上游芯片提供算力保障,中游人工智能厂商着力研发算法模型,下游应用领域提供落地场景

\begin{figure}[H]
\centering
\capstart

\noindent\sphinxincludegraphics{{qushi}.png}
\caption{趋势}\label{\detokenize{chapter_project/AI_industry_analysis:id30}}\end{figure}


\paragraph{分工 10\sphinxfootnotemark[719]}
\label{\detokenize{chapter_project/AI_industry_analysis:id10}}%
\begin{footnotetext}[719]\sphinxAtStartFootnote
\sphinxnolinkurl{http://www.changgpm.com/thread-387-1-1.htmls}
%
\end{footnotetext}\ignorespaces 

\subparagraph{基础设施提供者}
\label{\detokenize{chapter_project/AI_industry_analysis:id11}}
基础设施提供者,为整个产品体系提供了计算能力、产品与外界沟通的工具,并通过基础平台实现支撑。比如当前的阿里云、腾讯云、百度智能云,等等AI基础设施平台,我们只需要购买其服务,就可以基于平台现有的软硬件和模型算法,实现企业的个性化AI产品打造。


\subparagraph{数据提供者}
\label{\detokenize{chapter_project/AI_industry_analysis:id12}}
数据提供者是体系的数据来源,为后续的数据处理提供充足的“养料”。比如一些大数据公司、广告公司,他们拥有者丰富的数据资源,在以前这些数据可能只会应用于企业内部的角色分析,但是如今却可以将这些数据进行清洗,为第三方企业提供数据服务,例如数据增补、数据开源、以及数据销售。不过鉴于数据安全,国家政策会在一定程度上限制,但这并不影响人工职能的发展。


\subparagraph{数据处理者}
\label{\detokenize{chapter_project/AI_industry_analysis:id13}}
数据处理者,代表着各种人工智能技术和服务提供商,主要负责智能信息表示与形成、智能推理、智能决策及智能执行与输出等工作。数据处理者,在某个智能领域拥有成熟的解决方案,例如旷世科技(Face
++,致力于图像识别领域)、科大讯飞(强大的智能语音服务商),数据处理者能够帮助第三方快速进行AI产品方案的落地。


\subparagraph{系统协调者}
\label{\detokenize{chapter_project/AI_industry_analysis:id14}}
系统协调者,负责系统的集成、需求的定义、资源的协调、解决方案的封装,以及除研发以外一切可以保障人工智能产品顺利运行和再行业落地所需的工作;系统些调者的主要的目标就是实现AI产品服务的商业化落地,也是保障前三个角色价值落地的根本。

我们从数据流开始说起,人工智能的产品体系是一个动态流程,本质上是围绕数据采集、存储、计算展开的。
\begin{enumerate}
\sphinxsetlistlabels{\arabic}{enumi}{enumii}{}{.}%
\item {} 
数据提供者使用各种手段获得原始数据。

\item {} 
数据处理者对数据进行加工。

\item {} 
数据处理者进行模型训练,获得可以使用对模型。

\item {} 
用模型对新数据进行预测。

\end{enumerate}

“数据–信息–知识–智慧”的过程,再随着动态循环,就是“训练–推断–再训练–再推断”的过程。产品经理需要完成系统集成、需求定义、资源协调、解决方案封装的保障工作。


\paragraph{BAT}
\label{\detokenize{chapter_project/AI_industry_analysis:bat}}
百度A(AI)B(Big
data)C(Cloud)战略,阿里腾讯也有各自云服务,大数据中心,人工智能实验室,这些大公司胜在基础架构层、数据量和资本优势上,拥有大量的人工智能科学家,可以持续优化算法,提升算法模型的准确度。


\paragraph{准确性}
\label{\detokenize{chapter_project/AI_industry_analysis:id15}}

\subparagraph{需要达到99.9999\%}
\label{\detokenize{chapter_project/AI_industry_analysis:id16}}
如手术机器人,自动驾驶技术,智慧交通等,这些产品和服务直接关系到人的生死,要求具有极高的准确度,需要AI科学家持续的优化,只有达到近乎百分之百的准确度才会商用。


\subparagraph{达到99\%或者95\%就可以}
\label{\detokenize{chapter_project/AI_industry_analysis:id17}}
如面部识别,语音机器人,无人机农药喷洒,艺术设计,搜索引擎,精准营销等,这些产品和服务对于精确度要求不高,因为即使不精确也不会直接造成人员伤亡。


\paragraph{垄断程度}
\label{\detokenize{chapter_project/AI_industry_analysis:id18}}

\subparagraph{高}
\label{\detokenize{chapter_project/AI_industry_analysis:id19}}
行业的垄断程度越高,头部公司的体量越大,最初可能因为缺乏AI技术而采购技术,当技术环境成熟,BAT和google这类公司开源了大量技术后,行业垄断型公司会则会搭建自己的AI团队,搭建自己的大数据,云计算和AI实验室,以运营商行业为例,资源垄断型市场,三家独大,每家都在搭建自己的大数据分析平台,也在搭建自己的人工智能实验室。


\subparagraph{低}
\label{\detokenize{chapter_project/AI_industry_analysis:id20}}
如衣食住行相关的制造业和零售行业,因为分散,他们有需求,但是没有足够体量和资本自己搭建AI团队,所以他们会将AI技术作为一项工具,以合理的价格采购成套服务,来实现+AI的升级。

如同当年的互联网+和+互联网一样,也会演化出AI+和+AI的发展方向。


\subparagraph{象限图}
\label{\detokenize{chapter_project/AI_industry_analysis:id21}}
我认为第一象限因为BAT拥有科学家优势,虽然垄断程度高的企业很有钱,但是因为BAT有数据优势和科学家优势,在这个领域BAT优势明显,可以向企业提供独特的AI服务,提升垄断企业效率,这部分产品需要靠AI科学家驱动。

第三象限虽然技术门槛低,垄断程度低,会出现大量小AI公司进入这个市场,BAT进入这个市场拥有足够的品牌优势,因为市场需求量较大,BAT可以考虑做开放平台,为有垂直领域的AI公司体统底层服务,如果自己来做,这部分服务和产品将是运营和产品来主要驱动。

第二象限暂时来看不太适合进场,第四象限垄断企业会自己组建AI团队来做,我们能看到,手机制造这个还不算垄断的行业中,因为资本实力雄厚,各个厂家已经在组建自己的AI研发团队。

\begin{figure}[H]
\centering
\capstart

\noindent\sphinxincludegraphics[width=600\sphinxpxdimen]{{产品象限}.png}
\caption{产品象限}\label{\detokenize{chapter_project/AI_industry_analysis:id31}}\end{figure}


\paragraph{应用场景2\sphinxfootnotemark[720]}
\label{\detokenize{chapter_project/AI_industry_analysis:id22}}%
\begin{footnotetext}[720]\sphinxAtStartFootnote
\sphinxnolinkurl{https://www.zhihu.com/question/57373956/answer/155398900}
%
\end{footnotetext}\ignorespaces 
1.场景比较规范,2.需要经验,
3.且数据量大,4.但是反复度高的工作岗位,5.如果监管准入门槛比较低就更好。
1和5可促进快速落地,2、3、4适合深度学习复现场景。

医疗+AI,门槛着重考虑;安防+AI,门槛重在渠道,和海康;无人驾驶,需要规范,市场、大众、政府、产品供应、交通设施等都需要规范。


\paragraph{2B}
\label{\detokenize{chapter_project/AI_industry_analysis:b}}

\subparagraph{民营企业}
\label{\detokenize{chapter_project/AI_industry_analysis:id23}}\begin{itemize}
\item {} 
赚更多的钱

\item {} 
转型的决心和行动力:只要技术是有用的,可以提升效率或压缩成本的

\item {} 
途径:BAT可以考虑在尽可能多民营企业家聚集的场合,推广真实高效的+AI产品和服务

\end{itemize}


\subparagraph{国营企业}
\label{\detokenize{chapter_project/AI_industry_analysis:id24}}\begin{itemize}
\item {} 
国营企业即承担创造价值的责任,也同时承担着保证国有资产不流失的责任,组织内部员工多是对上级和自己的职位负责,所以创新一定要稳妥

\item {} 
用友和亚信等软件开发团队多是长期驻厂,提供运维服务和新需求开发

\item {} 
核心诉求是不犯错,未必有功,但求无过

\end{itemize}


\subparagraph{创业公司}
\label{\detokenize{chapter_project/AI_industry_analysis:id25}}
AIStartups: \sphinxurl{https://github.com/lipiji/AIStartups}


\paragraph{上市}
\label{\detokenize{chapter_project/AI_industry_analysis:id26}}
截至3月12日,CV四小龙中,旷视和依图2家都中止过上市进程;智能语音领域的云知声在问询后被终止;最烧钱的AI芯片领域短时间难有企业上市;营收稍好的硬件领域,也有优必选等企业折戟IPO。

\sphinxurl{https://www.jiemian.com/article/5806409.html}

从2020年全球知名的AI芯片企业——Wave Computing
公司破产,AI企业再难获得VC亲睐,独立造血不足的情况,第一批AI公司甚至已经开始倒下,现在对于活着的AI来说,能不暴雷已经算是发展行情不错。

最近,东南亚电商平台Shopee
3月份发布的财报坐实,原依图科技CTO颜水成已在2020年末离开,加盟Shopee。而据内部人士消息,格灵深瞳CTO邓亚峰也已经离职。核心高管离职,对拟上市企业无疑是重大打击。

当下的情况是,投了很多资金、寄于厚望的AI独角兽近乎全部折戟上市,也算是投资人继O2O后,又押错的一个时代。强如李开复也在2020年公开承认,“不少AI公司割了投资人的韭菜。”


\paragraph{访谈}
\label{\detokenize{chapter_project/AI_industry_analysis:id27}}
EE
Times:你怎么看这种现象?\sphinxhref{https://www.eet-china.com/news/202005080936.html}{15}%
\begin{footnote}[721]\sphinxAtStartFootnote
\sphinxnolinkurl{https://www.eet-china.com/news/202005080936.html}
%
\end{footnote}

Ernst:在很大程度上,这反映了中国加入全球高科技产业创新竞赛时间较晚;此外,我认为很多研发活动仍被局限在官方科研机构,而企业更多扮演“生产者”角色,没有体现出研发与工程能力,在营销与策略规划方面也没有发挥作用。尽管有很多在市场与组织改革方面的努力,中国在强化产业界与学数界之间的知识交流方面,还有一段路要走。

EE Times:然后还有专利政策。

Ernst:事实上,
中国企业现在过于专注在增加专利申请案的数量,一旦获得注册,就似乎不太关注那些专利的状况。更重要的是,在能够达到高引证(citations)的专利识别、开发、维护以及质量的改善方面,缺乏后劲。

中国AI技术的最大挑战 EE
Times:所以在你看来,中国的AI技术发展遇到的最大挑战是什么?

Ernst:中国创新体系的分散化突显了中国AI发展的一个基础性困境;在中美贸易战爆发前,
中国AI业者在能够反映他们竞争优势的领域创新,透过当地数量庞大的低人力成本大学毕业生来开采大数据库,专注于在中国快速成长的大众化AI应用市场竞争。中国在国际贸易与全球生产网络的深度融合,提供取得全球知识来源的充足机会,让这种策略成为可行;在某种程度上中国业者能用外国技术,不需要投资内部的基础性与应用研发,就能繁荣成长。但随着美国升高技术出口限制,这些业者要取得相同的收益就越来越困难。


\paragraph{More:}
\label{\detokenize{chapter_project/AI_industry_analysis:more}}
\begin{figure}[H]
\centering
\capstart

\noindent\sphinxincludegraphics{{data_AI_industry}.jpg}
\caption{data\_AI\_industry}\label{\detokenize{chapter_project/AI_industry_analysis:id32}}\end{figure}
\begin{itemize}
\item {} 
\sphinxurl{https://mattturck.com/data2020/}

\item {} 
中国人工智能产业发展联盟:\sphinxurl{http://aiiaorg.cn/}

\item {} 
中国人工智能产业知识产权白皮书2020:\sphinxurl{http://www.ai-research.online/\#/whitepaper/detail/51}

\item {} 
\sphinxurl{https://daxueconsulting.com/category/artificial-intelligence-industry-in-china/}

\item {} 
\sphinxurl{https://www.ulapia.com/reports/search?query=AI}

\item {} 
\sphinxurl{https://www.iyiou.com/search?p=\%E4\%BA\%BA\%E5\%B7\%A5\%E6\%99\%BA\%E8\%83\%BD}

\item {} 
\sphinxurl{https://emerj.com/ai-executive-guides/}

\item {} 
IT桔子的工智能创投数据厍:\sphinxurl{https://www.itjuzi.com/ai}

\item {} 
人工智能行业研究报告(147份):\sphinxurl{https://zhuanlan.zhihu.com/p/346793543}

\end{itemize}


\subsubsection{AI公司}
\label{\detokenize{chapter_project/AI_company:ai}}\label{\detokenize{chapter_project/AI_company::doc}}

\paragraph{理解企业}
\label{\detokenize{chapter_project/AI_company:id1}}
\sphinxurl{https://www.zhihu.com/market/paid\_column/1251475507390050304/section/1251475513652604928}

“那些口号喊得响、低估场景挑战、高估技术能力的公司大多会在泡沫中死掉。”GMIS全球机器智能峰会后,刘维在接受网易智能专访时这样评价AI创业现状。
\sphinxhref{https://mp.weixin.qq.com/s?\_\_biz=MzI3NTU3ODk1MQ==\&mid=2247484933\&idx=1\&sn=e7b99f0686f5f4c6f9d41bc22a012881\&chksm=eb03ef2ddc74663bc8f0ccca0f64c71a72e9e5583986806f81d86a799beca3d56ac970f461f9\&scene=21\#wechat\_redirect}{9}%
\begin{footnote}[722]\sphinxAtStartFootnote
\sphinxnolinkurl{https://mp.weixin.qq.com/s?\_\_biz=MzI3NTU3ODk1MQ==\&mid=2247484933\&idx=1\&sn=e7b99f0686f5f4c6f9d41bc22a012881\&chksm=eb03ef2ddc74663bc8f0ccca0f64c71a72e9e5583986806f81d86a799beca3d56ac970f461f9\&scene=21\#wechat\_redirect}
%
\end{footnote}


\paragraph{阵营}
\label{\detokenize{chapter_project/AI_company:id2}}\begin{enumerate}
\sphinxsetlistlabels{\arabic}{enumi}{enumii}{}{.}%
\item {} 
互联网巨头,包括“超第一梯队”的跨国公司Google和微软亚洲研究院,以及第一梯队的巨头
— — 大家熟悉的百度、阿里和腾讯;以及第二梯队的今日头条和滴滴等公司。

\item {} 
人工智能创业公司,以某种人工智能技术为主营业务的创业公司,典型的如自动驾驶领域的Momenta、地平线、驭势科技;视觉识别领域的格灵深瞳、商汤科技和旷视科技等。

\item {} 
将人工智能融入到自身业务中的其它创业公司,如学霸君、泼辣熊和智齿科技等。
\sphinxhref{https://zhuanlan.zhihu.com/p/33524676}{4}%
\begin{footnote}[723]\sphinxAtStartFootnote
\sphinxnolinkurl{https://zhuanlan.zhihu.com/p/33524676}
%
\end{footnote}

\end{enumerate}

人工智能公司(主要针对创业公司)主要分为三个阵营:\sphinxhref{https://www.sohu.com/a/364264851\_114819}{5}%
\begin{footnote}[724]\sphinxAtStartFootnote
\sphinxnolinkurl{https://www.sohu.com/a/364264851\_114819}
%
\end{footnote}
\begin{enumerate}
\sphinxsetlistlabels{\arabic}{enumi}{enumii}{}{.}%
\item {} 
研究核心技术的AI公司(Core AI
Companies)核心人工智能,主要针对人工智能基础设施的搭建。产品经理侧重于对底层技术框架的理解。

\item {} 
应用人工智能公司(Application AI
Companies):通常的表现形式是提供一种基础功能,客户可以通过调用封装好的API进行对自身产品的武装或填充,而无需自己研究基础功能。因为往往对于一些中小公司而言,拥有的数据量有限,无法通过机器学习技术完成对每一个基础功能的建模和应用部署,因此需要借助这样公司提供的开放API能力,然后自身做垂直应用。产品对行业的理解力和对行业趋势(参考《跨越鸿沟》《创新者的窘境》\sphinxhref{http://www.woshipm.com/pmd/3024508.html}{21}%
\begin{footnote}[725]\sphinxAtStartFootnote
\sphinxnolinkurl{http://www.woshipm.com/pmd/3024508.html}
%
\end{footnote})的洞察力是核心;应用AI技术公司的商业模式以TO
B为主,产品经理的KPI是项目回款,因此产品经理需要有一定的商务技能(售前、销售);同时因为需要定制化开发,产品经理要明确区分标准化产品和定制化产品;

\item {} 
行业人工智能公司(Industry AI
Companies):三个阵营中最接近终端用户的公司,提供垂直领域的AI服务,帮助用户解决具体场景中的具体问题。产品对行业的理解力和对行业趋势的洞察力是核心

\end{enumerate}

核心人工智能公司往往对产品经理在技术层面要求最高,应用人工智能其次,行业垂直应用人工智能公司是对产品经理的业务深度或行业理解深度要求最高。


\paragraph{国际互联网}
\label{\detokenize{chapter_project/AI_company:id3}}
\begin{figure}[H]
\centering
\capstart

\noindent\sphinxincludegraphics{{international_AI}.png}
\caption{国际互联网企业产业布局图谱\sphinxhref{https://weread.qq.com/web/reader/40632860719ad5bb4060856kc9f326d018c9f0f895fb5e4}{10}\sphinxfootnotemark[726]}\label{\detokenize{chapter_project/AI_company:id19}}\end{figure}
%
\begin{footnotetext}[726]\sphinxAtStartFootnote
\sphinxnolinkurl{https://weread.qq.com/web/reader/40632860719ad5bb4060856kc9f326d018c9f0f895fb5e4}
%
\end{footnotetext}\ignorespaces 

\paragraph{中国公司总览}
\label{\detokenize{chapter_project/AI_company:id4}}
{\color{red}\bfseries{}|}AI公司{\color{red}\bfseries{}|}\sphinxhref{https://daxueconsulting.com/ai-landscape-china/}{7}%
\begin{footnote}[727]\sphinxAtStartFootnote
\sphinxnolinkurl{https://daxueconsulting.com/ai-landscape-china/}
%
\end{footnote}
{\color{red}\bfseries{}|}中国人工智能产业链图谱{\color{red}\bfseries{}|}\sphinxhref{https://www2.deloitte.com/content/dam/Deloitte/cn/Documents/innovation/deloitte-cn-innovation-ai-whitepaper-zh-181126.pdfs}{11}%
\begin{footnote}[728]\sphinxAtStartFootnote
\sphinxnolinkurl{https://www2.deloitte.com/content/dam/Deloitte/cn/Documents/innovation/deloitte-cn-innovation-ai-whitepaper-zh-181126.pdfs}
%
\end{footnote}
新基建IT产业图谱: \sphinxurl{https://www.analysys.cn/article/detail/20019748}

\begin{figure}[H]
\centering
\capstart

\noindent\sphinxincludegraphics{{AI_100_company}.png}
\caption{潜力100强\sphinxhref{http://finance.eastmoney.com/a/202007141554661012.html}{17}\sphinxfootnotemark[729]}\label{\detokenize{chapter_project/AI_company:id20}}\end{figure}
%
\begin{footnotetext}[729]\sphinxAtStartFootnote
\sphinxnolinkurl{http://finance.eastmoney.com/a/202007141554661012.html}
%
\end{footnotetext}\ignorespaces 

\paragraph{报告}
\label{\detokenize{chapter_project/AI_company:id13}}
《中国人工智能软件及应用市场半年度研究报告》:\sphinxurl{https://www.idc.com/getdoc.jsp?containerId=prCHC46625720}
《中国AI云服务市场(2020上半年)跟踪》:\sphinxurl{https://www.idc.com/getdoc.jsp?containerId=prCHC47212020}
预测:\sphinxurl{https://www.idc.com/getdoc.jsp?containerId=prCHC47222921}


\paragraph{初步判断是不是真AI公司}
\label{\detokenize{chapter_project/AI_company:id14}}\begin{itemize}
\item {} 
公司大小:大公司才有大数据。小公司很可能调了API。

\item {} 
融资: 融资有多少轮,目前融了多少资

\item {} 
知名度

\end{itemize}


\paragraph{排行榜}
\label{\detokenize{chapter_project/AI_company:id15}}
2018年中国企业人工智能技术发明专利排行榜(TOP100):\sphinxurl{https://www.douban.com/note/709381421/?from=author}


\paragraph{BAT的机会}
\label{\detokenize{chapter_project/AI_company:bat}}
个人感觉BAT做AI有机会,在第一象限有技术和数据优势。在第三象限有数据和品牌优势,如果做垂直领域,可以通过招聘获取垂直领域的认知,垂直领域的市场拓展是最困难的,下面将从企业属性来分析这个问题。第四项象限,BAT有数据优势,可以通过合作方式互通互联。
\sphinxhref{https://medium.com/@liwdai/\%E8\%BD\%AC\%E5\%9E\%8Bai\%E4\%BA\%A7\%E5\%93\%81\%E7\%BB\%8F\%E7\%90\%86\%E9\%9C\%80\%E8\%A6\%81\%E6\%8E\%8C\%E6\%8F\%A1\%E7\%9A\%84\%E7\%A1\%AC\%E7\%9F\%A5\%E8\%AF\%86-\%E4\%B8\%80-ai\%E4\%BA\%A7\%E5\%93\%81\%E7\%BB\%8F\%E7\%90\%86\%E8\%83\%BD\%E5\%8A\%9B\%E6\%A8\%A1\%E5\%9E\%8B\%E5\%92\%8C\%E5\%B8\%B8\%E8\%A7\%81ai\%E6\%A6\%82\%E5\%BF\%B5\%E6\%A2\%B3\%E7\%90\%86-99ccd4a7c214}{19}%
\begin{footnote}[730]\sphinxAtStartFootnote
\sphinxnolinkurl{https://medium.com/@liwdai/\%E8\%BD\%AC\%E5\%9E\%8Bai\%E4\%BA\%A7\%E5\%93\%81\%E7\%BB\%8F\%E7\%90\%86\%E9\%9C\%80\%E8\%A6\%81\%E6\%8E\%8C\%E6\%8F\%A1\%E7\%9A\%84\%E7\%A1\%AC\%E7\%9F\%A5\%E8\%AF\%86-\%E4\%B8\%80-ai\%E4\%BA\%A7\%E5\%93\%81\%E7\%BB\%8F\%E7\%90\%86\%E8\%83\%BD\%E5\%8A\%9B\%E6\%A8\%A1\%E5\%9E\%8B\%E5\%92\%8C\%E5\%B8\%B8\%E8\%A7\%81ai\%E6\%A6\%82\%E5\%BF\%B5\%E6\%A2\%B3\%E7\%90\%86-99ccd4a7c214}
%
\end{footnote}

\begin{figure}[H]
\centering
\capstart

\noindent\sphinxincludegraphics{{BAT}.jpg}
\caption{BAT}\label{\detokenize{chapter_project/AI_company:id21}}\end{figure}


\paragraph{Baidu}
\label{\detokenize{chapter_project/AI_company:baidu}}
2016 ACL Fellow 百度 CTO
王海峰。\sphinxhref{https://www.jiqizhixin.com/articles/2019-11-28-4}{1}%
\begin{footnote}[731]\sphinxAtStartFootnote
\sphinxnolinkurl{https://www.jiqizhixin.com/articles/2019-11-28-4}
%
\end{footnote}

作为百度集团首席技术官,王海峰负责百度搜索、语音搜索、图像搜索、信息流、手机百度、小度机器人、自然语言处理、知识图谱、互联网数据挖掘等业务,并曾创始了百度语音、图像、推荐及个性化、深度学习、度秘等多个技术方向。由王海峰领导研发的百度翻译产品目前支持
28 种语言、756 个方向的自动翻译,并于 2015 年 5
月上线了全球首个融合神经网络机器翻译和统计机器翻译模型的大规模在线翻译系统。其领导的「基于大数据的互联网机器翻译核心技术及产业化」还荣获了
2015 年国家科技进步奖,这也是我国互联网企业首次获得该奖项。

2016 年,王海峰当选 ACL
Fellow,成为了首位获此荣誉的中国大陆科学家。会士评选委员会在对王海峰的评语中写道:王海峰在机器翻译、自然语言处理和搜索引擎技术领域,在学术界和工业界都取得了杰出成就,对于
ACL 在亚洲的发展也做出了卓越贡献。

在前沿技术领域,拥有七大实验室、聚集数十位世界顶级AI科学家的百度研究院,正在聚焦前瞻基础研究,探索技术前沿方向。比如,在量子领域,除“量桨”之外,百度还研发出了国际领先、国内第一的云上量子脉冲系统“量脉”。在区块链领域,百度超级链XuperChain实现了核心技术自主可控,专利申请量200多件。在工业物联网安全领域,百度AIoT安全能力覆盖六大场景,覆盖终端设备超过1.5亿。

在芯片方面,百度自主研发的百度昆仑AI芯片、百度鸿鹄AI芯片,正在为新基建提供可靠的动力。百度昆仑芯片是业界实际性能最高的AI芯片,也是首次在工业领域大规模应用的中国自研AI芯片。百度鸿鹄芯片作为远场语音芯片,适配于车载、智能家具等场景,一颗芯片就能解决所有语音交互问题,是一个足以改变行业的技术革新。

百度智能云已经取得了一系列傲人的成果:智慧金融已服务近200家金融客户,涉及营销、风控等十几个金融场景;智慧医疗的产品已经服务300多家医院和超过1500家基层医疗机构,服务人次超过了2500万;智慧城市则已经逐渐落地北京海淀、重庆、苏州等城市,成为新一代城市智能基础设施,让城市变得更智慧;智慧能源领域中,企业级AI中台、知识中台在国家电网、南方电网等头部客户落地应用,支撑20多个业务场景,覆盖两条特高压智能化线路、150多个智慧变电站、4万多条输电线路的监拍智能化,每天代替人工巡视能源路线超过7万公里;智能制造领域中,已覆盖14大行业,30余家企业,16个合作伙伴,触达32类垂直场景,在3C、汽车、钢铁、能源等行业已规模落地。

智能交通方面,百度Apollo依托百度领先的AI能力,接连中标重庆、合肥、阳泉等地车路协同新基建项目,Apollo
Robotaxi自动驾驶出租车服务也已在长沙全面开放试运营。第一季度,知名研究公司Navigant
Research将百度Apollo列为全球四大自动驾驶领域领导者之一。
\sphinxhref{http://www.mysecretrainbow.com/ai/17083.html}{6}%
\begin{footnote}[732]\sphinxAtStartFootnote
\sphinxnolinkurl{http://www.mysecretrainbow.com/ai/17083.html}
%
\end{footnote}

在工业互联网领域,百度智能云在四季度分别与贵州省贵阳市政府、山东省济南市工信局签署战略合作,打造地方级AI工业互联网平台,全面推动企业数字化、智能化转型,助力当地经济形成新的经济增长点。
\sphinxhref{http://finance.eastmoney.com/a/202102181812494141.html}{8}%
\begin{footnote}[733]\sphinxAtStartFootnote
\sphinxnolinkurl{http://finance.eastmoney.com/a/202102181812494141.html}
%
\end{footnote}

在2020年的百度世界大会上,百度大脑6.0发布,核心技术已经具备“知识增强的跨模态深度语义理解”能力,基于掌握的5500亿海量知识,百度大脑综合语音、语言、视觉等不同信息,实现跨模态语义理解,获得对世界的统一认知。有了这一能力,机器就能听懂语音,看懂图像视频,理解语言,进而理解真实世界。

而在应用层面,百度的自然语言处理能力已经在百度搜索、小度音箱、信息流推荐等一系列产品应用中发挥重要作用,对外输送到金融、通信、教育、互联网等行业,助力各行各业产业智能化升级。

除此之外,百度还在量子计算、生命科学等前沿技术领域积极布局,探索人工智能技术的持续突破。2018年百度就成立了量子计算研究所,2020年9月,百度发布国内首个云原生量子计算平台——量易伏,量易伏实现了量子计算和云计算的深度融合,可以实现对量子硬件的操控。2020年12月20日,百度发布了基于飞桨的生物计算平台\sphinxhyphen{}螺旋桨
Pad
dleHelix,可以重点满足生物医药、疫苗设计和精准医疗方面的AI需求。\sphinxhref{http://news.moore.ren/industry/269550.htm}{20}%
\begin{footnote}[734]\sphinxAtStartFootnote
\sphinxnolinkurl{http://news.moore.ren/industry/269550.htm}
%
\end{footnote}

\begin{center}\sphinxincludegraphics{{baidu_AI}.png}\end{center} \sphinxincludegraphics{{baidu_AI2}.png}

成果展示:\sphinxurl{https://ai.baidu.com/forum/topic/list/187}


\paragraph{Alibaba}
\label{\detokenize{chapter_project/AI_company:alibaba}}
\begin{figure}[H]
\centering
\capstart

\noindent\sphinxincludegraphics{{ali_AI}.jpg}
\caption{阿里AI\sphinxhref{https://www.zhihu.com/question/278914587/answer/1246774889}{16}\sphinxfootnotemark[735]}\label{\detokenize{chapter_project/AI_company:id22}}\end{figure}
%
\begin{footnotetext}[735]\sphinxAtStartFootnote
\sphinxnolinkurl{https://www.zhihu.com/question/278914587/answer/1246774889}
%
\end{footnotetext}\ignorespaces 

\subparagraph{Aliyun}
\label{\detokenize{chapter_project/AI_company:aliyun}}
机器学习PAI Studio
\sphinxhref{https://www.aliyun.com/product/bigdata/product/learn}{2}%
\begin{footnote}[736]\sphinxAtStartFootnote
\sphinxnolinkurl{https://www.aliyun.com/product/bigdata/product/learn}
%
\end{footnote} \begin{center}\sphinxincludegraphics{{tianchi}.jpg}\end{center}
\begin{center}\sphinxincludegraphics{{tianchi_develop}.jpg}\end{center}


\subparagraph{taobao}
\label{\detokenize{chapter_project/AI_company:taobao}}
AI虚拟主播
\sphinxhref{https://developer.aliyun.com/article/778214?spm=a2c6h.13262185.0.0.1d0a4ee6o0ncC3}{15}%
\begin{footnote}[737]\sphinxAtStartFootnote
\sphinxnolinkurl{https://developer.aliyun.com/article/778214?spm=a2c6h.13262185.0.0.1d0a4ee6o0ncC3}
%
\end{footnote}


\paragraph{腾讯}
\label{\detokenize{chapter_project/AI_company:id16}}

\subparagraph{微信AI 12\sphinxfootnotemark[738]}
\label{\detokenize{chapter_project/AI_company:ai-12}}%
\begin{footnotetext}[738]\sphinxAtStartFootnote
\sphinxnolinkurl{http://www.changgpm.com/thread-214-1-1.html}
%
\end{footnotetext}\ignorespaces \begin{itemize}
\item {} 
微信对话开放平台:自定义你的AI客服机器人

\item {} 
腾讯小微硬件开放平台:用一个小程序指挥智能硬件

\end{itemize}


\subparagraph{实验室}
\label{\detokenize{chapter_project/AI_company:id17}}
AI Lab \sphinxhref{https://www.jiqizhixin.com/articles/2019-05-24-14}{13}%
\begin{footnote}[739]\sphinxAtStartFootnote
\sphinxnolinkurl{https://www.jiqizhixin.com/articles/2019-05-24-14}
%
\end{footnote}
优图实验室AI手语识别
\sphinxhref{https://www.jiqizhixin.com/articles/2019-05-16-16}{14}%
\begin{footnote}[740]\sphinxAtStartFootnote
\sphinxnolinkurl{https://www.jiqizhixin.com/articles/2019-05-16-16}
%
\end{footnote}


\paragraph{创业公司}
\label{\detokenize{chapter_project/AI_company:id18}}
创业公司有很多方法可以并且确实成功地与大公司竞争。即使在一个相对较小的数据集上,你也可以在很多领域取得巨大的成就。


\paragraph{More}
\label{\detokenize{chapter_project/AI_company:more}}
\sphinxurl{https://www.tusimple.com/} \sphinxurl{https://github.com/amusi/CV-Company-List}
开源:\sphinxurl{https://www.oschina.net/company} \sphinxurl{http://www.birdbot.cn/}

AI 证券:

\sphinxurl{http://search.stcn.com/was5/web/search?token=0.1584090199903.75\&channelid=252914\&searchword=AI\&catid=\&order=rel\&before=\&after=};


\bigskip\hrule\bigskip


阿波罗的官网地址是: \sphinxurl{http://apollo.auto/}

源代码,文档与数据下载地址为: \sphinxurl{https://github.com/apolloauto}


\subsubsection{AI 人才}
\label{\detokenize{chapter_project/AI_talents:ai}}\label{\detokenize{chapter_project/AI_talents::doc}}

\paragraph{所有人的危机}
\label{\detokenize{chapter_project/AI_talents:id1}}
\begin{figure}[H]
\centering
\capstart

\noindent\sphinxincludegraphics{{AI_risk}.jpg}
\caption{所有人的危机\sphinxhref{https://www.slideshare.net/Happy.Prototyper/mix2018ai-ai-vp}{7}\sphinxfootnotemark[741]}\label{\detokenize{chapter_project/AI_talents:id17}}\end{figure}
%
\begin{footnotetext}[741]\sphinxAtStartFootnote
\sphinxnolinkurl{https://www.slideshare.net/Happy.Prototyper/mix2018ai-ai-vp}
%
\end{footnotetext}\ignorespaces 

\paragraph{全球}
\label{\detokenize{chapter_project/AI_talents:id2}}
\begin{figure}[H]
\centering
\capstart

\noindent\sphinxincludegraphics{{AI_talent}.jpg}
\caption{AI中美发展情况}\label{\detokenize{chapter_project/AI_talents:id18}}\end{figure}


\paragraph{中国}
\label{\detokenize{chapter_project/AI_talents:id3}}
据数据显示,中国的人工智能专利申请数量占全球总量的37.1\%,位居全球第一。论文总产出量达到141840篇,位居全球第二。

现阶段中国也是全球人工智能产业投融资最为活跃的国家之一,其中总投融资事件数量占全球的31.7\%,投融资资金总额占全球的60\%。

虽然我国在专利申请和论文产出方面已经跻身全球领先序列,但人工智能人才短缺问题依旧存在。报告称,我国从事人工智能基础研究的学者仅占全球总量的11\%,科研机构仅占5\%,仍落后于全球顶尖水平。\sphinxhref{https://tech.sina.com.cn/roll/2020-07-19/doc-iivhvpwx6203309.shtml}{1}%
\begin{footnote}[742]\sphinxAtStartFootnote
\sphinxnolinkurl{https://tech.sina.com.cn/roll/2020-07-19/doc-iivhvpwx6203309.shtml}
%
\end{footnote}


\subparagraph{底层研究弱}
\label{\detokenize{chapter_project/AI_talents:id4}}
“越是人工智能上层(算法层、应用层)的研究,我国研究者对世界作出的贡献越多;越是底层(系统层、芯片层),我国研究者的贡献越少。底层研究能力的缺失不仅给我国人工智能基础研究拖后腿,更重要的是,将使得我国智能产业成为一个空中楼阁,走上信息产业受核心芯片和操作系统制约的老路。”


\paragraph{十大紧缺岗位}
\label{\detokenize{chapter_project/AI_talents:id5}}
在算法研究岗、应用开发岗和实用技能岗这三大岗位类型中,算法研究岗和应用开发岗的学历准入门槛远高于其他岗位。

据报告显示,45.1\%的算法研究岗和41.9\%的应用开发岗要求应聘人员具有硕士及以上学历;实用技能岗和产品经理岗的准入门槛为本科及以上,相关的岗位占比分别为88.8\%和91.8\%。

\begin{figure}[H]
\centering
\capstart

\noindent\sphinxincludegraphics{{gangwei}.png}
\caption{岗位}\label{\detokenize{chapter_project/AI_talents:id19}}\end{figure}


\paragraph{人工智能产业人才岗位类型}
\label{\detokenize{chapter_project/AI_talents:id6}}
薪酬方面,学历准入门槛高于其他岗位的算法研究岗和应用开发岗也得到较好体现。据报告显示,目前高达56.5\%的算法研究岗和46.1\%的应用开发岗的单月薪酬达到35k以上。

(1)算法研究员、工程师来自各类高校研究生,人工智能浪潮崛起第一波受益人,因为没有产品,从学术研究,从sdk卖起,要的全是算法,价格水涨船高。

(2)后台、前端人才也是来自学校和各类互联网开发的转型,需求也是旺盛,量大,但是可以从各类技术栈不变的情况下迁移一下业务就行了,比如c++继续写c++,换个业务做人脸识别而已,不外乎还是数据的管理、计算、代码效率封装别人sdk之类的,问题不大,迁移成本低,人才很容易就多起来。

(3)产品,如果你的业务还是画画原型,做一下简单的业务功能设计,那从互联网产品经理过来就行了。问题也不大,迁移一下业务而已。但是如果要从事一个专业的行业/领域产品经理,特别是要懂人工智能技术,理解它的能力边界,去开拓应用场景,结合业务和技术,找到权衡点,做出产品,这类产品经理,就是稀缺的。我建议的也是做这类型的产品经理。这类,从学校出来就直接做,也很少,因为需要的能力类型比较丰富。这类的产品经理会去做什么样的产品呢,后面说。

(4)售前销售就不说了。

着重对比下算法和产品经理吧

(1)算法分两种:研究和工程,如果是做研究,那是发论文,出新方法,新方法应用到业务上,让公司业务及产品能力保持领先的,但是就需要超强持续的学习和研究能力。但是这类人,没有多少公司养的起,哪怕大厂,也会把大家拉到业务线去做业务。所以就变成研究+工程。算法工程师呢,专心搞业务模型训练,搞点数据、搞模型,模型训练师而已,谈不上高大上,随着市场上这类能力的人越来越多,加上自动化训练更加成熟,的确没有什么竞争力。所以如果要做算法,肯定就是要深入走研究+工程的道路,单走算法工程,35岁后估计就没人要了吧。

(2)产品经理:就是你要的广度,这个广度,需要了解算法、sdk、工程化、什么容器技术吧、k8s吧,技术架构,然后上面是业务层什么的,然后产品设计、项目管理、然后不同的产品,又有很多有趣的东西,你要研究行业吧,你要研究上下游吧,你要研究产业发展吧。从广度上,满足了你的想象,这也是我喜欢的一个原因,我自己也的确没办法专注到数学研究上,所以转型了。


\paragraph{对比其他互联网岗位}
\label{\detokenize{chapter_project/AI_talents:id7}}
\begin{figure}[H]
\centering
\capstart

\noindent\sphinxincludegraphics{{talents_compete}.png}
\caption{AI对比其他互联网岗位}\label{\detokenize{chapter_project/AI_talents:id20}}\end{figure}


\paragraph{对比新职业}
\label{\detokenize{chapter_project/AI_talents:id8}}
\begin{figure}[H]
\centering
\capstart

\noindent\sphinxincludegraphics{{talents_new}.png}
\caption{对比新职业\sphinxhref{https://my.oschina.net/u/3861898/blog/4405417}{5}\sphinxfootnotemark[743]}\label{\detokenize{chapter_project/AI_talents:id21}}\end{figure}
%
\begin{footnotetext}[743]\sphinxAtStartFootnote
\sphinxnolinkurl{https://my.oschina.net/u/3861898/blog/4405417}
%
\end{footnotetext}\ignorespaces 
数据显示,2019年春招旺季人才需求增幅最高的15个职位中,人工智能类占据六席。其中图像识别、语音识别、图像处理等应用层岗位的人才需求增速显著加快,图像识别工程师的人才需求增幅同比高达110.9\%。深度学习、机器学习等基础层研究职位人才需求增速低于应用层,但也呈现出了强劲的增长势头。\sphinxhref{http://www.kejilie.com/lanjingtmt/article/rUVjeu.html}{6}%
\begin{footnote}[744]\sphinxAtStartFootnote
\sphinxnolinkurl{http://www.kejilie.com/lanjingtmt/article/rUVjeu.html}
%
\end{footnote}


\paragraph{学历要求 3\sphinxfootnotemark[745]}
\label{\detokenize{chapter_project/AI_talents:id9}}%
\begin{footnotetext}[745]\sphinxAtStartFootnote
\sphinxnolinkurl{http://finance.southcn.com/f/2021-03/05/content\_192173681.htm}
%
\end{footnotetext}\ignorespaces 
智联招聘的数据显示,人工智能对从业者的技术功底和学习能力最为看重,对本科及以上学历人才要求占比86.3\%,其中要求硕士及以上岗位占三成。随着互联网产业越来越依靠数据资产,数据相关技能不光是对数据工程师的要求,也成为人工智能、运维支持等岗位的要求;产品经理虽然不需要对底层数据进行处理,但一定的数据认知与分析能力也成为必备。而在人工智能领域,除了编程算法与数据技能之外,机器学习、深度学习等必备技能,使AI人才在技术竞争中呈现差异化。


\paragraph{人工智能岗位的薪酬}
\label{\detokenize{chapter_project/AI_talents:id10}}
除互联网相关行业外,随着新兴技术极大赋能传统行业、工业互联网呈现井喷式发展,生产制造业等第二产业对互联网人才的需求只增不减。据智联招聘2020年大数据显示,仪器仪表及工业自动化与大型设备/机电设备/重工业行业都有很高的AI人才需求,这与它们对人工智能等数字化技术的广泛运用不无关系,例如工业云平台、AI/机器视觉工业应用等。

产业互联网的蓬勃发展也带动技术人才需求水涨船高,招聘需求规模占18.1\%。从求职竞争程度看,2020年,互联网产业竞争最激烈的岗位为设计,平均每个招聘职位有32.3份简历竞争,其次是产品,平均每个招聘职位有31.6份简历竞争,至于技术,则是平均28.5人抢一职。

竞争激烈的人工智能岗位使得其薪酬也更集中在高薪区间,智联招聘数据显示,在前十名中占据一半,依次为机器学习(20895元/月)、算法工程师(19944元/月)、深度学习(19431元/月)、自然语言处理(NLP)(18720元/月)与图像算法(18601元/月)。此外,在软件研发中,Golang为“最高薪编程语言”推动“码农”收获17886元/月的平均薪资,语音/视频/图形开发、脚本开发、云计算也成为高薪职类。具体来看招聘职位数最高的20个城市,北京占据近两成,深圳与上海各占10.5\%和8.5\%。广州虽以4.8\%紧随其后,但与成都、杭州接近,仅差0.2个百分点。


\paragraph{个人品牌建立}
\label{\detokenize{chapter_project/AI_talents:id11}}
确定自我品牌要求,并从小事开始,为品牌塑造努力。我想得现在还不算太晚,重新审视人生之旅的地图,深思熟虑,定出新的起点并迈出步伐。


\subparagraph{品牌塑造能力}
\label{\detokenize{chapter_project/AI_talents:id12}}
品牌的本质是什么?是提供差异化的价值:\sphinxhref{http://www.xmamiga.com/372/}{8}%
\begin{footnote}[746]\sphinxAtStartFootnote
\sphinxnolinkurl{http://www.xmamiga.com/372/}
%
\end{footnote}
\begin{itemize}
\item {} 
能为别人提供什么价值

\item {} 
最擅长的领域是什么

\item {} 
跟其他人相比,最具竞争力的特点是什么

\item {} 
有没有一种需求,能跟自已的领域建立连接

\end{itemize}


\paragraph{人才素质要求}
\label{\detokenize{chapter_project/AI_talents:id13}}
人工智能时代对人才素质要求的影响首先直接体现在人工智能技术领域,将在技术层面以及在应用层面直接对人才提出硬性技术能力和软性素质能力两方面的更高要求。在技术的研发上,随着未来人工智能技术发展到一定阶段,企业可能会产生对掌握更高阶技术人才的需求,潜在方向包括机器人培训与监督、机器人外形设计、机器人性格设计等技术能力,这类顶尖的技术人才往往有过硬的学术背景与科研实力,大多拥有计算机科学(Computer
Science)或者电气工程学(Electrical
Engineering)等专业科学学科的博士学位。而在技术的应用上,企业将需要更多既掌握技术能力又具有良好软性素质能力的复合型人才。这类人才应具备交叉学科背景及综合能力,如同时有能力搭建计算机程序和商业模型;同时,他们还需要快速学习能力以理解商业逻辑,更需要跨界合作能力与各方沟通洽谈,从而真正将人工智能技术落地为各行业的具体应用。
\sphinxhref{https://www.financialnews.com.cn/hq/yw/201804/P020180412355549093101.pdf}{9}%
\begin{footnote}[747]\sphinxAtStartFootnote
\sphinxnolinkurl{https://www.financialnews.com.cn/hq/yw/201804/P020180412355549093101.pdf}
%
\end{footnote}

人工智能产业人才岗位能力要求\sphinxhref{https://www.miitec.cn/home/index/detail?id=2252}{15}%
\begin{footnote}[748]\sphinxAtStartFootnote
\sphinxnolinkurl{https://www.miitec.cn/home/index/detail?id=2252}
%
\end{footnote}

\begin{figure}[H]
\centering
\capstart

\noindent\sphinxincludegraphics{{AI_universal_talents}.png}
\caption{AI时代通用人才\sphinxhref{http://www.woshipm.com/zhichang/3146016.html}{14}\sphinxfootnotemark[749]}\label{\detokenize{chapter_project/AI_talents:id22}}\end{figure}
%
\begin{footnotetext}[749]\sphinxAtStartFootnote
\sphinxnolinkurl{http://www.woshipm.com/zhichang/3146016.html}
%
\end{footnotetext}\ignorespaces 
《AI技术人才成长路线图》
\sphinxurl{https://blog.csdn.net/zw0Pi8G5C1x/article/details/79947077}


\paragraph{学历和工作经验要求}
\label{\detokenize{chapter_project/AI_talents:id14}}
在学历和技能方面,《目录》显示,大部分岗位要求本科及以上学历,也有 9\%
的企业岗位招聘专科人才。除了学历背景,数字经济六大重点领域对于人才的通用能力也存在多元需求,有
44\% 的企业要求熟练掌握外语语种,22\% 的企业要求应聘者精通计算机。

在从业经验方面,近四成的企业倾向于招收 3\sphinxhyphen{}5
年工作经验的求职者。同时,也有 9\%
的岗位表示接受应届毕业生,企业愿意利用自身较为完善的人才培养体系帮助紧缺专业人才完成从高校到岗位的转化。\sphinxhref{https://www.infoq.cn/article/0d349rm8zninvibeksdd}{13}%
\begin{footnote}[750]\sphinxAtStartFootnote
\sphinxnolinkurl{https://www.infoq.cn/article/0d349rm8zninvibeksdd}
%
\end{footnote}


\paragraph{人才培养模型}
\label{\detokenize{chapter_project/AI_talents:id15}}
人工智能技术服务专业(专业代码:610217),是 2019
年增补专业,主要面向人工智能产业及其应用相关的企事业单位,在人工智能技术应用开发、系统运维、产品营销、技术支持等岗位群,咶从事人工智能应用产品开发与测试、数据处理、系统运维、产品营销、技术支持等工作。

\begin{figure}[H]
\centering
\capstart

\noindent\sphinxincludegraphics{{form_AI_talents}.png}
\caption{人工智能技术服务(610217)学习路径及人才培养模型}\label{\detokenize{chapter_project/AI_talents:id23}}\end{figure}

2019年3月教育部公布了《2018年度普通高等学校本科专业备案和审批结果的通知》,人工智能被列入新增审批本科专业名单,全国共有35所高校获首批建设资格。这35所高校分别是:北京科技大学、上海交通大学、厦门大学、电子科技大学、北京交通大学、同济大学、山东大学、西南交通大学、天津大学、南京大学、武汉理工大学、西安交通大学、东北大学、东南大学、四川大学、西安电子科技大学、大连理工大学、南京农业大学、重庆大学、兰州大学、吉林大学、浙江大学、北京航空航天大学、北京理工大学、哈尔滨工业大学、西北工业大学、中北大学、长春师范大学、南京信息工程大学、江苏科技大学、安徽工程大学、江西理工大学、中原工学院、湖南工程学院、华南师范大学。

2020年2月教育部公布《2019年度普通高等学校本科专业备案和审批结果》,新增人工智能专业的高校达180所。从学校数量来看,北京、江苏、山东、四川的新增院校较多;从学校层次来看,近两年新增人工智能本科专业的院校既有北京航天航空大学、北京理工大学、哈尔滨工业大学、浙江大学、南京大学、上海交通大学、复旦大学、同济大学、武汉大学等传统老牌名校,也有如安徽信息工程学院、泉州信息工程学院、东华理工大学等普通院校,共同推进人工智能基础研究型人才和应用型人才的培养。\sphinxhref{https://shimo.im/docs/ryYGVtYQvPGGdHjG/read}{12}%
\begin{footnote}[751]\sphinxAtStartFootnote
\sphinxnolinkurl{https://shimo.im/docs/ryYGVtYQvPGGdHjG/read}
%
\end{footnote}


\paragraph{薪酬报告}
\label{\detokenize{chapter_project/AI_talents:id16}}
科锐的薪酬报告(鉴于猎头公司的报告一般工资虚高)、robertwalters的薪酬报告、mercer之类的报告
\sphinxhref{https://www.zhihu.com/question/63188172/answer/515405404}{10}%
\begin{footnote}[752]\sphinxAtStartFootnote
\sphinxnolinkurl{https://www.zhihu.com/question/63188172/answer/515405404}
%
\end{footnote}


\subsubsection{AI 金融}
\label{\detokenize{chapter_project/AI_Finance:ai}}\label{\detokenize{chapter_project/AI_Finance::doc}}
是金融科技 {\hyperref[\detokenize{chapter_AI+Finance/FinTech:fintech}]{\sphinxcrossref{\DUrole{std,std-ref}{金融科技(FinTech)}}}} (\autopageref*{\detokenize{chapter_AI+Finance/FinTech:fintech}}) 的其中一种AI应用,建议有时间先阅读金融科技
{\hyperref[\detokenize{chapter_AI+Finance/FinTech:fintech}]{\sphinxcrossref{\DUrole{std,std-ref}{金融科技(FinTech)}}}} (\autopageref*{\detokenize{chapter_AI+Finance/FinTech:fintech}})。


\paragraph{金融业结合AI最积极 {[}13{]}}
\label{\detokenize{chapter_project/AI_Finance:ai-13}}
IDC《中国AI落地白皮书》中也提到,金融产业对于AI的应用最为积极,不论项目落地数量还是成熟度也相对更高。

机器之心500强:\sphinxurl{https://www.jiqizhixin.com/articles/2019-10-22-12}

\begin{figure}[H]
\centering
\capstart

\noindent\sphinxincludegraphics{{AI_Finance_company}.png}
\caption{AI 金融公司{[}23{]}}\label{\detokenize{chapter_project/AI_Finance:id31}}\end{figure}


\subparagraph{为何?}
\label{\detokenize{chapter_project/AI_Finance:id1}}\begin{enumerate}
\sphinxsetlistlabels{\arabic}{enumi}{enumii}{}{.}%
\item {} 
金融产业凭借自己高度数字化、信息化的良好基础,相比其他产业更容易打通技术入口。

\item {} 
尤其金融产业的前台环节,业务接待、产品销售等流程,同样也属于劳动力高度密集的产业,应用AI所带来的降本增效效应也会更加显著。

\item {} 
加上金融行业在技术方面的尝试更加积极,大量金融机构都拥有自己的技术研发部门,使得整个行业在接纳AI赋能上拥有更好的基础,不需要长时间的市场教育。

\end{enumerate}


\subparagraph{结合难点}
\label{\detokenize{chapter_project/AI_Finance:id2}}
不在于技术,而是对于数据安全和隐私问题的要求天然要比其他行业更高。

上云风潮后部署不是简单对接API就可以的。

2019年趋势:
\begin{itemize}
\item {} 
技术服务者调整自己的云化方案,通过私有云、混合云等多种部署

\item {} 
金融机构选择自己研发或采购技术,对自身云平台的能力进行AI更新。

\end{itemize}


\subparagraph{应用点}
\label{\detokenize{chapter_project/AI_Finance:id3}}
\begin{figure}[H]
\centering
\capstart

\noindent\sphinxincludegraphics{{AI_for_Finance}.png}
\caption{人工智能行业落地主要场景领域{[}23{]}}\label{\detokenize{chapter_project/AI_Finance:id32}}\end{figure}
\begin{itemize}
\item {} 
智能客服催收

\item {} 
监管:Regtech

\item {} 
智能保顾:太保阿尔法发布时曾轰动一时,蚂蚁慧小保现在已经淡出,微信只做风险评估

\item {} 
投顾(投资教育、投资建议、风险匹配等) {[}14{]}

\item {} 
投研(舆情分析、投资决策、决策模型等)

\item {} 
智能风控(多维信用体系、全面风险管理) VS 普通风控{[}15{]}

\item {} 
金融产品运营(智能营销、用户/产品运营等)

\item {} 
征信: 个人征信未来在中国的发展和应用 \sphinxhyphen{} King James的文章 \sphinxhyphen{} 知乎
\sphinxurl{https://zhuanlan.zhihu.com/p/22280599}
阿里和腾讯这样的互联网公司做第三方征信是不是本身就是不合理的? \sphinxhyphen{} King
James的回答 \sphinxhyphen{} 知乎
\sphinxurl{https://www.zhihu.com/question/33731819/answer/120081634}

\end{itemize}


\subparagraph{端侧}
\label{\detokenize{chapter_project/AI_Finance:id4}}
腾讯、京东等企业都推出了类似于“金融无人舱”的概念,将人脸识别、语音交互等技术通过麦克风阵列、智能摄像头等方式部署在端侧。

通过端侧部署,可以通过统一的硬件配置,让技术模型不再需要面对因移动端设备多样化而提升鲁棒性的麻烦。像是因为不同设备前置摄像头配置不同,用户所处环境也会影响光线。因此人脸核验、证件核验的识别算法都要提升鲁棒性。但整体化的硬件配置,就不再需要担心这些问题。

同时类似“无人舱”概念的出现,绕过银行App这一入口,把AI的接触点直接搬到了线下,让很多不习惯使用App的用户,在线下也能与\sphinxstylestrong{AI能力相遇},不仅减轻了人工负担,也让业务管理更加统一化。


\paragraph{“AI+金融”产业链}
\label{\detokenize{chapter_project/AI_Finance:id5}}
\begin{figure}[H]
\centering
\capstart

\noindent\sphinxincludegraphics{{financial_AI}.png}
\caption{“AI+金融”产业链{[}6{]}}\label{\detokenize{chapter_project/AI_Finance:id33}}\end{figure}

2020年中国AI+金融行业发展分析报告 {[}8{]}

金融类产品的上、下游关系一般是这样的:银行类机构—金融科技公司—第三方服务公司—渠道—员工。银行类机构掌控着最大化的资金和对应的牌照,金融科技公司借助自身的技术手段向银行类机构提供优质的客户,同时金融科技公司又在一定程度上借助第三方服务公司获取所需,然后通过渠道进行产品分销,最终又借助员工落地。
{[}10{]}


\subparagraph{明确上、下游的利益 {[}11{]}}
\label{\detokenize{chapter_project/AI_Finance:id6}}
虽然银行类机构不同时期有不同的要求,但是本质上还是倾向于通过新增开户量提升自身的余额(存款),同时也借助自身资金的优势获取一定的利益。当然,银行类机构比较看重合规和风控。也就是说,银行类机构对获取利益的需求并不像一般企业那么急迫,更多的还是在合规和风控的前提下完成行内的任务。

金融科技公司更多的是提供科技层面上的解决方案,利用自身的技能优势帮助银行解决流程、合规、风控和效率等问题,从而获取利益。也可以说,金融科技公司本质上是中间的服务商,将上、下游打通,然后收取一定的服务费。

第三方服务公司的涉及面比较广,它着重强调的是垂直于某一领域的技能,而不是综合的服务提供商,如专门输出实名认证方案的服务公司、专门查询企业注册信息的公司,以及征信服务机构等一系列专门的机构。其核心是拓展更多的场景和接口,这样不仅可以通过服务来赚钱,还能够在与上游的交互中进一步完善自己的数据,提高服务效率,让自身变得更有价值。

渠道本质上就是一个分销商,它借助自身渠道侧的资源,承担销售的角色,通过收取服务费或利差来获利。一般来说,渠道是交叉服务于多家公司的,因此中间有许多竞争对手,甚至有不同行业的竞争对手。这样做可以丰富自身业务,赚取更多的复合型收益;还能够在与上游谈判的过程中保持一定的竞争力;同时也分散了风险,不会因某款产品出现问题而无事可做。

员工就是实际的执行者,对某款产品进行实际销售,员工的利益一般分为基本工资和销售提成两部分。


\paragraph{为什么是金融+AI而非AI+金融呢4\sphinxfootnotemark[753]}
\label{\detokenize{chapter_project/AI_Finance:aiai-4}}%
\begin{footnotetext}[753]\sphinxAtStartFootnote
\sphinxnolinkurl{https://tanxianlian.com/2020/05/15/\%e9\%87\%91\%e8\%9e\%8dai\%e7\%9a\%84\%e6\%9c\%aa\%e6\%9d\%a5\%e7\%95\%85\%e6\%83\%b3/}
%
\end{footnotetext}\ignorespaces 
这两者的前后连接顺序体现的是主动和被动,引导和被引导的关系。之所以是金融+AI,体现的是这是金融行业对AI技术\sphinxstylestrong{主动性的利用需求},而非被动型的推动。


\paragraph{为什么金融可以+AI呢?}
\label{\detokenize{chapter_project/AI_Finance:id7}}
因为金融业务开展的基础本质上是基于信息(数据)进行,AI可以对数据进行更好的利用,从而提升金融业务的效率。

具体来说,有这么几个主要的业务范围:
\begin{itemize}
\item {} 
KYC(客户了解):具体包括客户背景信息调查、客户核身、用户画像、客户偏好等;

\item {} 
交易决策:例如信贷领域的风控,理财领域的风险等级评估、产品推荐,保险领域的保险方案设计、理赔验证,以及所有细分领域都包含的反欺诈等;

\item {} 
客户服务:例如售前营销、售后服务等。

\end{itemize}


\paragraph{金融数据}
\label{\detokenize{chapter_project/AI_Finance:id8}}
\begin{figure}[H]
\centering
\capstart

\noindent\sphinxincludegraphics{{finance_data_ecosystem}.jpg}
\caption{金融机构数据生态系统}\label{\detokenize{chapter_project/AI_Finance:id34}}\end{figure}

自2000年以来,以Hadoop为代表的分布式存储和计算技术迅猛发展,极大地提升了互联网数据管理能力,引发全社会开始重新审视数据的价值,数据也被作为一种重要的战略资源对待。而大数据作为一种新资源、新技术、新理念,为数据赋予了新的意义。从资源视角看,大数据是一种新的资源;从技术视角看,大数据代表了新一代数据管理和分析技术;从理念视角看,大数据打开了一种全新的数据驱动思维角度。


\paragraph{平安}
\label{\detokenize{chapter_project/AI_Finance:id9}}
\begin{figure}[H]
\centering
\capstart

\noindent\sphinxincludegraphics{{pingan}.png}
\caption{平安的数据应用的架构}\label{\detokenize{chapter_project/AI_Finance:id35}}\end{figure}


\paragraph{金融新基建 2\sphinxfootnotemark[754]}
\label{\detokenize{chapter_project/AI_Finance:id10}}%
\begin{footnotetext}[754]\sphinxAtStartFootnote
\sphinxnolinkurl{https://www.leiphone.com/news/202012/7ovvkzByXnPQjnlD.html}
%
\end{footnotetext}\ignorespaces 
在金融新基建榜中,乐信、水滴、弘玑Cyclone、洞见、同盾五家公司凭借各自优势在众多优秀竞争者中脱颖而出。

他们分别荣获“最佳新消费AI平台奖”、“最佳保险科技数据中台奖”、“最佳智能自动化平台方案奖”、“最佳隐私计算平台奖”和“最佳智能分析决策奖”。


\paragraph{特有的知识体系}
\label{\detokenize{chapter_project/AI_Finance:id11}}
很多行业都有特有的知识体系,不深入工作5年以上,很难做到融会贯通。比如,互联网金融行业的风险控制产品经理如果原来没有在金融体系、银行体系工作过,那么很难做出优秀的风险控制产品。这种风险控制产品可不是随随便便在办公室里想想,或者打开某个竞争对手的产品看看就能够想清楚的。
{[}7{]}

智能风控主要依托高维度的大数据和人工智能技术对风险进行及时有效的识别、预警、防范。智能风控整个流程主要分为四个阶段:第一阶段,数据采集,数据是智能风控的基础,主要数据来源为网络行为数据、授权数据、交易时产生的数据、第三方数据等;第二阶段,行为建模,在这个过程中,需要对大量数据进行结构化处理,形成最有效的信用评估组合;第三阶段,用户画像,通过前期的数据采集和行为建模,形成对每个用户的画像;第四阶段,风险定价,主要包括行为监控、反欺诈违约和催收。金融业务风控新挑战和智能风控基本流程见图。

\begin{figure}[H]
\centering
\capstart

\noindent\sphinxincludegraphics{{risk_management}.jpg}
\caption{智能风控}\label{\detokenize{chapter_project/AI_Finance:id36}}\end{figure}

现有的智能风控公司主要分为三类:第一类是研发自用型,所研发的智能风控系统主要用于自身业务的发展。例如拍拍贷的“魔镜”大数据风控系统、鑫合汇的“鑫盾”风控系统、爱钱进的“云图”动态风控系统等。第二类是纯技术输出型,为商业银行、互联网金融公司、消费金融公司、P2P公司等提供信用评估审核、智能风控、反欺诈等金融解决方案。例如百融金服的“风险罗盘”、明略数据的明智系统“金融风控大脑”等。第三类是混合型,既支持自身业务的发展,也对外输出技术能力。这一类型的企业一般以建立生态为目的,希望以技术输出来丰富自身的数据。比如蚂蚁金服对中小企业开放的风控产品“蚁牛”和个人征信产品“芝麻信用”、京东金融的供应链金融产品“京保贝”、网易金融的“北斗”风控系统等。

智能风控一定程度上确实突破了传统风控的局限,在利用更高维度、更充分的数据时降低了人为的偏差,减少了风控的成本。然而,智能风控的核心数据还不够完善,优秀的风控人才也供不应求,征信的建设也处于初步阶段。智能风控的运用和完善,任重而道远。
{[}18{]}


\subparagraph{为什么AI或者大数据技术在金融风控领域用的最成熟?{[}21{]}}
\label{\detokenize{chapter_project/AI_Finance:ai-21}}
第一,数据量方面。金融领域的交易数量和用户数量巨大,很容易支持大规模的数据应用。

第二,大部分情况下,我们的模型在金融和反欺诈领域是不需要很严格的解释,避免了可解释性差的问题。另外,从零搭建一个新的引擎、新的算法时,更多时候也会看重业务规则和人类的经验专家体系,再结合机器学习等技术。


\subparagraph{AI 驱动的金融风控解决方案 {[}22{]}}
\label{\detokenize{chapter_project/AI_Finance:ai-22}}\begin{itemize}
\item {} 
大数据管理挑战 \sphinxhyphen{}> 知识图谱解决数据治理难题

\item {} 
特征提取挑战 \sphinxhyphen{}> 深度学习挖掘弱数据的金融价值

\item {} 
数据建模挑战 \sphinxhyphen{}> 集成学习框架有效整合各类风险因子

\end{itemize}


\paragraph{智能投顾 1\sphinxfootnotemark[755]}
\label{\detokenize{chapter_project/AI_Finance:id12}}%
\begin{footnotetext}[755]\sphinxAtStartFootnote
\sphinxnolinkurl{https://zhuiyi.ai/solution/securities}
%
\end{footnotetext}\ignorespaces 
金融似乎是人工智能乐于“入侵”的领域,仅智能投顾就涌现了近百家平台。顾名思义,智能投顾就是人工智能+投资顾问的结合体,借助大数据识别用户的风险喜好,再通过通过算法和模型定制风险资产组合。优势在于费用低、服务效率高、覆盖人群广,且在一定程度上满足了“千人千面”的理财需求。国外有Wealthfront、Betterment、Future
Advisor等知名智能投顾平台,国内也出现了钱景、拿铁财经、理财魔方等模仿者,就连记账软件网易有钱也开始向智能投顾转型。不过在政策和牌照的压力下,智能投顾能走多远仍不得而知。{[}20{]}

智能投顾,用服务新模式,打造差异化品牌

业务同质化让券商竞争激烈,企业希望通过服务的创新打造出差异化,吸引更多用户。追一AIForce的智能投顾助手YIFA提供了实时个股诊断、多条件筛选的能力,再结合快速交易能力,让投资者随时掌握个股动态,抓住转瞬即逝的交易机会。

智能投顾助手积累了行业头部的3000+常用知识点,让客户能在自营券商APP中闭环完成查询、交易和学习。创新的服务模式在不断增加客户信任度和粘性,提升品牌价值。

低成本高质量的智能外呼有效覆盖更多场景证券行业的高频度服务,让每个用户触点上的服务质量,成为决定券商运营效率和客户满意度的关键。

外呼可以提供各类电话沟通服务,包括开户的回访、对离职员工名下的客户进行回访、风险抽查、满意度调查、新股中签缴费提醒等等。他还能提供自动的业务咨询等经纪服务,既降低人力成本,又提升服务能力,提升覆盖度。

客户画像师, 挖掘数字金矿价值
大量的客户数据和运营数据在碎片化的场景中,难以获得有效沉淀与分析利用。

追一AIForce的客户画像师Feature,基于强大的语义理解能力,可以分析每一通外呼电话和各个渠道的客户交互内容。打破数据黑盒,将信息整理为结构化的数据,构建出消费者画像,从而辅助决策或主动服务,实现精细化运营与精准营销。


\subparagraph{步骤}
\label{\detokenize{chapter_project/AI_Finance:id13}}
\begin{figure}[H]
\centering
\capstart

\noindent\sphinxincludegraphics{{AI_Customer_service}.png}
\caption{客户服务智能化}\label{\detokenize{chapter_project/AI_Finance:id37}}\end{figure}

智能客服虽然在一定程度上能够,提高服务水平,但对于投资顾问所能提供的专业投资服务还有待智能化落地。智能投顾相对传统投顾的优势如下图所示:


\subparagraph{智能投顾 VS 传统投顾 {[}14{]}}
\label{\detokenize{chapter_project/AI_Finance:vs-14}}
\begin{figure}[H]
\centering
\capstart

\noindent\sphinxincludegraphics{{AI_invest_vs_traditional}.png}
\caption{智能投顾 VS 传统投顾}\label{\detokenize{chapter_project/AI_Finance:id38}}\end{figure}

2013年以来,金融机构用户规模大幅增加,传统的投顾手段难以服务大体量用户群体,在数字化发展的智
能化时代,与用户深度互动、不断优化用户投资体验才能赢得用户,智能投顾产品在优化用户体验、实现深度互动、提高服务效率、降低服务成本方面具有天然发展优势。头部金融机构在智能投顾方面的探索、实践也在推动着智能投顾在行业内推广开来。

“投”与“顾”
不平衡,用户深度互动缺失:目前,金融的智能投顾产品处于探索初级阶段,投资端智能化程度显著不足,顾问端仅仅优
化了操作的便捷性,缺乏与用户的深度互动,智能投顾探索不应局限于金融机构,而应引入更多外部科技力量推动真正智能投顾的实现。


\paragraph{金融保险(Finance and insurance)}
\label{\detokenize{chapter_project/AI_Finance:finance-and-insurance}}\begin{itemize}
\item {} 
4Paradigm

\item {} 
BioCatch

\item {} 
DataVisor

\item {} 
HyperScience

\item {} 
Behavox

\item {} 
AppZen {[}16{]}

\end{itemize}


\paragraph{人工智能平台中的 AI——实现规模应用的“哆啦 A 梦”}
\label{\detokenize{chapter_project/AI_Finance:ai-a}}
在人工智能平台前,金融行业特别是银行中的建模大都还是 SAS、SPSS
等统计建模软件的天下,虽然它们在评分卡等领域曾经辉煌过,但在大数据时代的长河里,它们渐渐失去了往日的光芒。这时,具有大数据基因,且整合了大数据机器学习框架以及多种计算机语言的人工智能平台应运而生。其不仅利用分布式计算部署能力和容器技术让计算能力和速度进一步提升,而且还降低了建模计算的使用门槛,让前线的业务人员也能体验小白上手大数据建模的快感,同时也能让建模与业务场景结合地更紧密,让建模结果更好地赋能业务。{[}19{]}


\paragraph{AI 在金融领域落地面临困难和挑战 3\sphinxfootnotemark[756]}
\label{\detokenize{chapter_project/AI_Finance:ai-3}}%
\begin{footnotetext}[756]\sphinxAtStartFootnote
\sphinxnolinkurl{http://www.ramywu.com/work/2018/05/18/AI-in-Finance-Survey/}
%
\end{footnotetext}\ignorespaces \begin{enumerate}
\sphinxsetlistlabels{\arabic}{enumi}{enumii}{}{.}%
\item {} 
深度学习模型的构建比较困难
目前并没有成熟的理论对深度学习模型的构造提供指导,主要还是依靠研究学者不断实验、不断探索

\item {} 
深度学习模型的稳健性和适用性有待商榷
深度学习模型能否适用于特定领域的分析和预测,需要大量实验进行验证。目前相关理论研究还处于对单一模型的优化处理,并没有提炼出通用的规律性方法和框架,从而限制了最终模型的稳健性和广泛适用性。

\item {} 
深度学习模型较难正确地阐述金融数据分析结果背后的经济学原理
深度学习模型在分析金融数据时,削弱了利用经济学解释最终结果的因果关系、以及隐藏于数据背后的经济学原理。

\end{enumerate}


\paragraph{2020 金融AI}
\label{\detokenize{chapter_project/AI_Finance:id14}}
金融科技进入“强监管”时代,行业合规有序发展◆金融科技行业正式进入“强监管”时代,市场的喧嚣与浮躁开始隐退,各类机构在探索创新与合规的平衡中不断前行。首份金融科技发展顶层文件出台,明确金融科技创新与服务的边界,整个行业进入合规有序发展阶段。金融机构积极拥抱金融科技,通过调整内部信息技术架构、成立科技子公司,推动技术从后台走向前台和中台,赋能业务发展。金融科技出海热潮持续进化,一批以提供获客、风控、运营等金融技术服务的企业开始扬帆远航,寻求新的发展机遇。整体来看,监管规范、新技术与金融业的融合应用、技术驱动下的经营模式与业务合作模式创新都是行业普遍关注和积极实践的焦点。

亿欧智库认为金融科技2020年十大关键词为:金融开放、金融科技监管、监管科技、消费金融、小微金融、开放银行、第三方支付、财富管理、保险科技。


\paragraph{开放银行概念兴起,联邦学习技术将成为行业新的生产力}
\label{\detokenize{chapter_project/AI_Finance:id15}}
“开放银行”概念起源于英国,2013年由英国“竞争和市场管理局”(CMA)推出,并在2016年3月正式发布了《开放银行标准》。开放银行的本质是为各类小型金融机构提供共享信息的安全通道,来帮助各类金融机构提供多元化的金融服务,并创新银行产品。那么如何建立安全的信息共享通道就成了开放银行发展的关键。此外,鉴于国内对于用户金融信息的“防泄密”要求逐步严格,对于直接开放金融数据进行交互的模式是不合规的。由此,如何合规的进行金融机构间的数据交互成了中国银行业探索“开放银行”业务的关键。
{[}8{]}

联邦学习的诞生就是为了解决这个难题,其技术本质是分布式加密机器学习,在保护原始数据隐私安全的情况下进行联合建模,共同分享计算结果。而在银行数字化的进程中,商业银行机构逐步将对数据的需求发展至“捕捉基于场景下的动态数据“从而实现高效获客和低成本风控。基于银行需求和合规要求,联邦学习技术的发展也将对开放银行模式起到决定性作用。该项技术的积累与突破,也将成为AI金融各赛道企业探索新AI+金融落地场景和商业模式的战略蓝海。


\subparagraph{联邦学习}
\label{\detokenize{chapter_project/AI_Finance:id16}}\begin{itemize}
\item {} 
调用部署在第三方模型的时候,输入的就不是具体的业务数据而是模型参数

\item {} 
解决了数据泄露问题

\item {} 
目前,蚂蚁、腾讯、京东、微众银行,它们各自都有很成熟的联邦学习解决方案了

\end{itemize}


\paragraph{核心创始人与产品要匹配 {[}12{]}}
\label{\detokenize{chapter_project/AI_Finance:id17}}
某互联网金融平台的定位是从事金融服务的公司,创始团队中 70\%
的人员都来自金融行业,主要以银行为主。在这种情况下,团队就与产品形成了有效的融合,因此很容易就看出这个行业现阶段存在的一些问题,也清楚这些问题可以通过哪些路径去解决。团队人员大多来自银行,他们对合规风控极其敏感。因此,他们不仅能够系统化地输出解决方案,还能有效地控制风险,从而达到平衡。


\paragraph{风险 {[}19{]}}
\label{\detokenize{chapter_project/AI_Finance:id18}}

\subparagraph{微观金融风险}
\label{\detokenize{chapter_project/AI_Finance:id19}}

\subparagraph{金融市场风险}
\label{\detokenize{chapter_project/AI_Finance:id20}}
大量金融市场参与者同时应用人工智能技术时可能会出现金融市场稳定性风险。例如,如果以机器学习为基础的交易者胜过其他交易者,可能导致更多的交易者采用类似的机器学习策略,放大金融震荡。此外,机器学习交易策略中的可预测模式可能存在被犯罪分子用来操纵市场价格的风险。


\subparagraph{金融机构风险}
\label{\detokenize{chapter_project/AI_Finance:id21}}
对大部分人而言,人工智能的决策过程如同一个“黑箱子”,透明度的缺乏可能导致监管机构和市场投资者难以判断决策过程


\subparagraph{市场集中化风险}
\label{\detokenize{chapter_project/AI_Finance:id22}}
如果未来人工智能技术集中于少数领先第三方技术提供商,可能会导致金融系统中某些功能的集中度变高。此外,若某些金融机构拥有海量自有大数据,或最前沿的技术可能因研发成本高昂而只有大公司负担得起,也可能会导致其市场地位上升,加剧市场集中化。


\subparagraph{市场漏洞风险}
\label{\detokenize{chapter_project/AI_Finance:id23}}
机器学习的交易算法存在一定不可预测性,若出现金融市场冲击,可能较难解释其成因。此外,如果人工智能在高频交易中被广泛使用,大量买入卖出可能会同时进行,导致市场波动性增加。人工智能的应用还可能允许更少流动性缓冲、更高杠杆,从而导致潜在的流动性或高杠杆风险。


\subparagraph{关联性风险}
\label{\detokenize{chapter_project/AI_Finance:id24}}
金融体系存在互相联动的特点,如果众多金融机构在某一关键部分依赖于相同数据或算法,那么当这些数据或算法出现问题时,问题可能会从单个节点向整个市场扩散。因此,集体采用人工智能工具可能会带来关联性风险。


\subparagraph{技术限制风险}
\label{\detokenize{chapter_project/AI_Finance:id25}}
如果人工智能模型没有经过适当的培训或反馈,例如不充足的压力测试,则使用者可能无法及时发现潜在的技术风险,特别是在使用者未能充分理解人工智能本质及限制的情况下。


\paragraph{就业岗位}
\label{\detokenize{chapter_project/AI_Finance:id26}}
人工智能技术将在金融行业内创造三类就业岗位:技术型、运营型和业务型。技术型岗位包括数据科学家、系统架构师、开发工程师、算法及系统测试师等;运营型岗位负责大数据与人工智能产品相关系统的运行与维护,确保相关产品的质量稳定、法律和业务合规性;业务型是介于技术和业务之间的复合型岗位,包括能够在技术部门、业务部门以及服务部门之间充当业务需求及技术算法解释角色的算法解释分析师,同时也需要能够快速了解、学习前沿技术并与现有业务进行结合的商务拓展专家。
{[}19{]}


\paragraph{专利}
\label{\detokenize{chapter_project/AI_Finance:id27}}\begin{itemize}
\item {} 
2020年金融科技专利报告:\sphinxurl{https://www.01caijing.com/article/273000.htm}

\item {} 
2020年中国金融科技发明专利排行榜:\sphinxurl{https://www.maigoo.com/news/554316.html}

\end{itemize}


\paragraph{示例}
\label{\detokenize{chapter_project/AI_Finance:id28}}
金融行业怎么用AI?蚂蚁金服是这么做的:\sphinxurl{https://tech.antfin.com/community/articles/625}

AI + 金融:10家头部人工智能厂商金融产品盘点:
\sphinxurl{http://www.rpa-cn.com/zuixinzixun/AIshijiao/2020-07-20/2638.html}


\paragraph{智能投顾}
\label{\detokenize{chapter_project/AI_Finance:id29}}
\sphinxurl{http://www.pbcsf.tsinghua.edu.cn/portal/article/index/id/5130.html}


\paragraph{更多}
\label{\detokenize{chapter_project/AI_Finance:id30}}
\sphinxurl{https://istock.ssetech.com.cn/wiki/doku.php?id=start}

人工智能与金融AI研究报告精选(286份):\sphinxurl{https://www.jrwenku.com/22053.html}

{[}5{]}: {[}6{]}:
\sphinxurl{https://www2.deloitte.com/content/dam/Deloitte/cn/Documents/innovation/deloitte-cn-innovation-ai-whitepaper-zh-181126.pdf}
{[}7{]}:
\sphinxurl{https://weread.qq.com/web/reader/46532b707210fc4f465d044ke4d32d5015e4da3b7fbb1fa}
{[}8{]}: \sphinxurl{https://mp.weixin.qq.com/s/1jOCiQMMYIqDFWOLv-6n-A} {[}9{]}:
\sphinxurl{https://www.yinxiang.com/everhub/note/e7f0c50e-dc27-488f-a9f9-35c121e20bb1}
{[}10{]}:
\sphinxurl{https://www.zhihu.com/pub/reader/119980992/chapter/1284104631833292800}
{[}11{]}:
\sphinxurl{https://www.zhihu.com/pub/reader/119980992/chapter/1284104632080130048}
{[}12{]}:
\sphinxurl{https://www.zhihu.com/pub/reader/119980992/chapter/1284104622652002304}
{[}13{]}: \sphinxurl{http://www.woshipm.com/ai/3263320.html} {[}14{]}:
\sphinxurl{http://www.changgpm.com/thread-202-1-1.html} {[}15{]}:
\sphinxurl{http://www.woshipm.com/pmd/2356222.html} {[}16{]}:
\sphinxurl{https://easyai.tech/blog/best-ai-company-2019/} {[}17{]}: {[}18{]}:
\sphinxurl{https://weread.qq.com/web/reader/e77325105e4e55e77af47dbkd3d322001ad3d9446802347}
{[}19{]}: \sphinxurl{https://www.infoq.cn/article/1obcmwjkaqyux5xjmy7j} {[}20{]}:
\sphinxurl{http://www.woshipm.com/it/508568.html} {[}21{]}:
\sphinxurl{http://www.woshipm.com/pmd/859851.html} {[}22{]}:
\sphinxurl{https://www.modb.pro/doc/23865} {[}23{]}:
\sphinxurl{http://www.199it.com/wp-content/uploads/2021/03/Image28-126.png}


\subsubsection{AI 健身}
\label{\detokenize{chapter_project/AI_fit:ai}}\label{\detokenize{chapter_project/AI_fit::doc}}

\paragraph{行业分析}
\label{\detokenize{chapter_project/AI_fit:id1}}
根据CBNData\&天猫发布的《2020健身大企业新趋势研究》,国人健身意识逐年增高,预计2020年健身器材市场规模将达500亿元。90后为主的新消费群体以及已婚已育女性成为健身大器械下单的主要人群。在器材中方面,跑步机成为家用场景下健身器材的中流砥柱。

张卓彧从投资角度进一步分析认为,国内健身市场正从休闲化向专业化发展,居家健身器材品牌需要硬件和内容双向升级,这更加抬高了健身器械市场的进入壁垒,赛道会逐步跑出头部玩家。而目前,国内健身器械品牌不仅在国内有广阔的发展土壤,并且有出海机会。

“比起欧美,中国有供应链优势。2020年,美国健身器材硬件进口市场中,来自中国大陆的产品比例占到64.5\%,来自中国台湾的产品占28.1\%。但未来当供应链发展到一定成熟阶段时,成本优势就没有太大差异了。所以健身器械品牌还需要做产品性能的提升和品牌溢价,未来健身器材智能化是一个发展方向。”张卓彧表示。

根据前瞻产业研究院发布的《2019年中国体育市场现状及发展趋势分析》统计,2019年,中国数字体育月活跃用户超1.2亿人。其中数字健身用户超2000万人。《2019\sphinxhyphen{}2020中国健身房市场发展白皮书》指出,Keep疫情期间月活跃用户量较2019年同期增长超过20\%。


\paragraph{痛点 3\sphinxfootnotemark[757]}
\label{\detokenize{chapter_project/AI_fit:id2}}%
\begin{footnotetext}[757]\sphinxAtStartFootnote
\sphinxnolinkurl{https://post.smzdm.com/p/andllwop/}
%
\end{footnotetext}\ignorespaces 
健身行业目前存在着诸多痛点:传统的健身来回通勤时间成本太高、健身教练鱼龙混杂、普遍缺乏数字化功能,或者是可以与用户实时交互的功能。所以健身过程比较枯燥,且不益于科学精准指导,用户粘性和健身积极性都不高。


\paragraph{例子}
\label{\detokenize{chapter_project/AI_fit:id3}}
2017年,天猫魔盒的研发者杜武平,就自立门户打造了一款全息互动训练系统——魔力智屏(Smart
Wall)。通过全触控的墙屏和地屏,对场内的人实现心率监测、运动检测、AI评分、墙地屏互动等功能。

坐落在北京望京区域的VENTO,就通过AI设备会对用户进行弹跳力、灵敏度、肌群柔韧度测试,然后将数据写入智能手环,与智能健身器材进行连接,来陪伴客户的整个运动过程,并实现自动化的调节。


\paragraph{失败}
\label{\detokenize{chapter_project/AI_fit:id4}}
也相继关闭了北京和上海的两家门店,单纯的智能设备始终无法取代人工,给用户带来私教式的指导。尽管节省了人力成本,却无力吸引到足够的会员来支撑运营。


\paragraph{产品体验 3\sphinxfootnotemark[758]}
\label{\detokenize{chapter_project/AI_fit:id5}}%
\begin{footnotetext}[758]\sphinxAtStartFootnote
\sphinxnolinkurl{https://post.smzdm.com/p/andllwop/}
%
\end{footnotetext}\ignorespaces 

\subparagraph{myShape}
\label{\detokenize{chapter_project/AI_fit:myshape}}
\sphinxurl{http://myshape.ai/mirror}

myShape包含了自研的3D动捕技术、姿态识别纠错算法、运动力学的专业知识以及一套强交互性的健身内容。

包含了自研的3D动捕技术、姿态识别纠错算法、运动力学的专业知识以及一套强交互性的健身内容。

TODO:


\paragraph{华为智慧屏 22\sphinxfootnotemark[759]}
\label{\detokenize{chapter_project/AI_fit:id6}}%
\begin{footnotetext}[759]\sphinxAtStartFootnote
\sphinxnolinkurl{https://consumer.huawei.com/cn/support/content/zh-cn00977206/}
%
\end{footnotetext}\ignorespaces 
适用版本:HarmonyOS 1.1, HarmonyOS 2.0

华为智慧屏的一大特色是拥有一个AI慧眼,除了支持视频通话,还有一个特色功能AI
健身,将你的客厅变身健身中心,智慧屏化身私教,操作步骤也很简单,只需语音说“我要健身”即可打开此功能,开启智慧健身第一步。\sphinxhref{https://zhuanlan.zhihu.com/p/87779620}{24}%
\begin{footnote}[760]\sphinxAtStartFootnote
\sphinxnolinkurl{https://zhuanlan.zhihu.com/p/87779620}
%
\end{footnote}

您可以按照以下方式操作:

1、在智慧屏主页选择 AI健身 。

3、进入AI健身应用,选择 推荐/健身专区/肩颈放松/瑜伽专区/广场舞专区
任意页签。

4、选择任意健身课程,在课程详情界面,选择 智能模式 或 普通模式
,根据界面提示,开始健身。

智能模式:摄像头实时收集您的健身动作和运动情况,并显示在智慧屏上,通过智能分析及时给出反馈和指导。

普通模式:智慧屏不会显示您的健身情况,您可以根据智慧屏的画面指导自行健身。

在智慧屏AI健身应用中,您可以选择健身训练课程,摄像头实时记录您的健身动作,并及时提醒指导,让您在家也能科学运动健身。

在屏幕上看到自己的健身姿态,可以和视频动作对比。

华为团队自研HUAWEI
Fitness算法,可以实时识别和追踪人体的骨骼关节点,通过关节点的位置变化识别用户的健身动作,并通过各个肢体在完成健身动作中的姿态变化客观地评价用户动作是否标准(完美,真棒,很好,加油。),让用户的运动更科学,效果更好

距离可能比较近,摄像头无法拍摄到用户的全身,此时比较难识别和评价用户的健身动作。作了针对性处理,只要摄像头捕捉到的用户肢体能满足当前健身动作的最核心的评价需求,即可完成动作的识别、评价和指导功能。


\subparagraph{3D体型追踪仪 VisbodyFit、Visbody D}
\label{\detokenize{chapter_project/AI_fit:d-visbodyfitvisbody-d}}
三维人体扫描国家标准起草单位唯一企业身份
\sphinxhref{https://www.visbodyfit.com/a/xinwenzixun/2021/0219/142.html}{20}%
\begin{footnote}[761]\sphinxAtStartFootnote
\sphinxnolinkurl{https://www.visbodyfit.com/a/xinwenzixun/2021/0219/142.html}
%
\end{footnote}
166项三维人体扫描技术领域专利数量居全球第一
清华\sphinxhyphen{}伯克利深圳学院人体视觉领域 唯一技术合作伙伴
西安电子科技大学成立图像视觉实验室唯一合作伙伴
谷歌、上海交大在顶级会议ECCV 联合发表体测领域专业论文
牛耳奖“人工智能领域年度最佳技术创新奖”
阿里巴巴诸神之战全AIoT赛道全球总决赛亚军\sphinxhref{https://www.visbodyfit.com/a/xinwenzixun/2020/0622/29.html}{16}%
\begin{footnote}[762]\sphinxAtStartFootnote
\sphinxnolinkurl{https://www.visbodyfit.com/a/xinwenzixun/2020/0622/29.html}
%
\end{footnote}
体测领域唯一一家国家级高新技术企业
西安市硬科技之星示范企业/西安市瞪羚企业 AIWIN医疗赛道TOP3
华为鲲鹏云生态合作伙伴

2017年,维塑以多年的技术研究为基础,成功研发了国内首款3D智能体测设备——维塑3D体型追踪仪。站在维塑上转一圈,即可收获个人专属的360°
可旋转3D人体模型,还有9项毫米级身体围度数据与12项身体成分分析数据。智能3D体态评估功能可以直观判断体态问题,帮助健身人群高效运动同时避免运动损伤等多种问题。维塑目前已销往全国200多个城市,用户超百万。目前VisbodyFit已经覆盖全国165
个城市超过
1500家健身机构,以先进设计理念,高新技术,优秀的用户体验打破了体测领域国外产品称霸的局面
\sphinxhref{https://www.visbodyfit.com/a/xinwenzixun/2020/0622/30.html}{17}%
\begin{footnote}[763]\sphinxAtStartFootnote
\sphinxnolinkurl{https://www.visbodyfit.com/a/xinwenzixun/2020/0622/30.html}
%
\end{footnote}
被超级猩猩、金吉鸟、李欣普拉提等知名品牌全线采购
\sphinxhref{https://www.visbodyfit.com/a/xinwenzixun/2020/1221/138.html}{21}%
\begin{footnote}[764]\sphinxAtStartFootnote
\sphinxnolinkurl{https://www.visbodyfit.com/a/xinwenzixun/2020/1221/138.html}
%
\end{footnote}

在2018年,维塑在三维人体扫描领域拥有百余项国家专利,获得了相关专利数量全国第一全球第三的成绩,并持有三维数字成像技术、基于深度学习的数字人体分析技术、云平台技术、BDA人体成分算法等多项核心技术。\sphinxhref{https://www.visbodyfit.com/a/xinwenzixun/2020/0622/31.html}{15}%
\begin{footnote}[765]\sphinxAtStartFootnote
\sphinxnolinkurl{https://www.visbodyfit.com/a/xinwenzixun/2020/0622/31.html}
%
\end{footnote}

首次新增“4象限重心平衡检测”,通过足底压力的分布检测人体重心平衡情况。
\sphinxhref{https://www.visbodyfit.com/a/xinwenzixun/2020/0611/27.html}{18}%
\begin{footnote}[766]\sphinxAtStartFootnote
\sphinxnolinkurl{https://www.visbodyfit.com/a/xinwenzixun/2020/0611/27.html}
%
\end{footnote}

维塑专利技术能实现从视觉传感器捕捉的人体信息中提取:生命体征(身高、体温、心跳以及血压等),身体成分(体脂率、骨骼肌、基础代谢以及无机盐含量等),身体围度信息、3D体型体态信息以及体能体质(包括体适能与关节灵活度分析等在内的人体动态评估)等多维度数据。

维塑3D体测仪被超级猩猩、金吉鸟、李欣普拉提等知名品牌全线采购

维塑科技希望通过搭建身体云平台,将用户、机构、行业各方的需求链接在一起,开启智能健康管理新局面:硬件设备将数据上传至云端进行计算和存储,个人用户可通过手机查看检测结果;企业客户可通过客户端进行管理;不同业态的合作伙伴可通过API接口集成数据,打破孤岛,实现价值最大化。\sphinxhref{https://www.visbodyfit.com/a/xinwenzixun/2020/0622/29.html}{16}%
\begin{footnote}[767]\sphinxAtStartFootnote
\sphinxnolinkurl{https://www.visbodyfit.com/a/xinwenzixun/2020/0622/29.html}
%
\end{footnote}

VisbodyFit赋能于各类大健康领域机构,包括医院相关科室、康复中心、健身中心、瑜伽馆、美容塑形机构以及家庭健身,帮助人们更好地掌控身体健康、塑造自己的健康生活。

2020年4月,维塑新品Visbody D正式发售。Visbody
D将作为顶级体测设备面世,带来更专业更高端的体测技术,并针对后疫情时期的客户需求做了更深的挖掘。\sphinxhref{https://www.visbodyfit.com/a/xinwenzixun/2020/1221/138.html}{21}%
\begin{footnote}[768]\sphinxAtStartFootnote
\sphinxnolinkurl{https://www.visbodyfit.com/a/xinwenzixun/2020/1221/138.html}
%
\end{footnote}
\begin{enumerate}
\sphinxsetlistlabels{\arabic}{enumi}{enumii}{}{.}%
\item {} 
Visbody
D支持手势操作,用户只需挥手即可完成选择及确认操作。手势识别技术的难点在于准确率。每台Visbody
D出厂均经过万次测试,确保用户拥有便捷流畅的体验。

\item {} 
除头部、肩部、骨盆的评估外,Visbody
D新增X/O/K/D腿型及膝关节评估,并增加不良体态可能导致的风险预警,让体态评估更加符合用户需求及认知水平。

\item {} 
加高精度压力传感器,通过足底压力的分布检测人体重心平衡情况

\item {} 
Visbody
D身体成分检测技术已升级为医疗级,可检测细胞内外液含量,通过细胞外液和细胞内液比率可判断是否存在体液失衡,从而评估营养情况,判断免疫力水平。

\item {} 
3年前,本地存储+单次结果+纸质报告还是体测的唯一选择。维塑首次推出云端档案+历史数据查看+微信报告功能

\end{enumerate}

\begin{figure}[H]
\centering
\capstart

\noindent\sphinxincludegraphics{{Visbody}.jpg}
\caption{Visbody D \& R}\label{\detokenize{chapter_project/AI_fit:id23}}\end{figure}


\subparagraph{FITURE魔镜}
\label{\detokenize{chapter_project/AI_fit:fiture}}
FITURE(成都拟合未来科技有限公司)致力于通过科技帮助大众建立健康的生活方式。提供“硬件+AI+内容+服务”一体式健康生态

FITURE魔镜解决的痛点不只是场景、成本等问题,它其实是为用户提供智能健康综合解决方案的科技产品。获得ELLEMEN
2020理容大奖榜单之“年度最佳科技产品”

“FITURE Motion
Engine”智能运动追踪系统,是适应各种极端场景的人体检测模型、高精度的姿态识别模型,是抽象化连续姿态的动作识别引擎,使得你站在这面魔镜前,无需穿戴任何产品或传感器辅助,你的一举一动都会被镜面上的摄像头和传感器捕捉,这些信息会成为判断标准,系统会通过屏幕里的AI教练会实时指导你的的动作姿势。

在家单独锻炼难免会缺少氛围,FITURE魔镜的游戏化互动健身课程,提供即刻置身虚拟游戏世界的沉浸式体验,为家庭健身增添了许多趣味性,用户甚至可以发起线上挑战赛,有社交的健身才更有动力。

FITURE于2020年9月底完成了A轮融资,并刷新了全球健身行业A轮融资的纪录,成为红杉资本、腾讯、C资本、凯辉基金、黑蚁资本、CPE(中信产业基金)、BAI(贝塔斯曼亚洲投资基金)、全明星基金等头部基金追捧的宠儿。
\sphinxhref{https://coffee.pmcaff.com/article/13654236\_j}{8}%
\begin{footnote}[769]\sphinxAtStartFootnote
\sphinxnolinkurl{https://coffee.pmcaff.com/article/13654236\_j}
%
\end{footnote}

NFS2020年度CEO峰会暨猎云网创投颁奖盛典上,FITURE
入选“2020年度最具投资价值创新企业TOP20”。

在此之外,AI健身myShape,已于今年9月推出旗下首个智能健身镜,并与器械公司乔山合作推出健身镜产品。而近乎同一时间,健身O2O平台沸腾时刻也推出智能健身镜。同样在中国香港的健身市场,刚刚上市的健身内容公司OliveX,发布了第一款健身镜产品。

抛开产品细节,各家健身镜的功能近乎相似。镜子硬件、课程内容、AI交互,构成核心的功能模块。相比跑步机、单车,做家庭健身镜在成为更多中国健身公司的选择。\sphinxhref{https://www.visbodyfit.com/a/xinwenzixun/2020/1130/132.html}{14}%
\begin{footnote}[770]\sphinxAtStartFootnote
\sphinxnolinkurl{https://www.visbodyfit.com/a/xinwenzixun/2020/1130/132.html}
%
\end{footnote}


\subparagraph{面试问题参考: 9\sphinxfootnotemark[771]}
\label{\detokenize{chapter_project/AI_fit:id7}}%
\begin{footnotetext}[771]\sphinxAtStartFootnote
\sphinxnolinkurl{https://coffee.pmcaff.com/article/2729281195713664/pmcaff?utm\_source=forum}
%
\end{footnotetext}\ignorespaces \begin{enumerate}
\sphinxsetlistlabels{\arabic}{enumi}{enumii}{}{.}%
\item {} 
请问在独立推动项目时,遇到的最困难的经历是什么样的?请详细说明为什么这些点难以解决,及你的应对策略。

\item {} 
你认为疫情下,用户对日常锻炼的需求会有什么变更?长期健身者的用户画像会有什哪些可能的变化?

\item {} 
有没有自己比较满意的项目经历,解释为什么会觉得满意。是什么原因让项目成功?

\end{enumerate}


\subparagraph{Keep AI 虚拟教练 6\sphinxfootnotemark[772]}
\label{\detokenize{chapter_project/AI_fit:keep-ai-6}}%
\begin{footnotetext}[772]\sphinxAtStartFootnote
\sphinxnolinkurl{https://coffee.pmcaff.com/article/12061874\_j}
%
\end{footnotetext}\ignorespaces 
为了追寻我们最初的目标:让用户能够最有效的运动。让更多的人运动起来。

人们日益增长的健身需求,和健身教练数量不足之间的矛盾。为用户提供个性化的专属训练计划

Keep 展示了
TOF(深度摄像头)动作打分的新技术。通过拍摄用户运动过程,Keep 的 APP
可以在你锻炼的时候进行动作指导,并实时提示标准度。

Keep
成立了人工智能研究院,秦曾昌博士任首席科学家兼人工智能研究院的院长。在和一家国内手机厂商合作,很快就会推出基于深度摄像头的应用。「即使是一张平面的照片,我们也可以重建出
3D 的人体姿态。」

新品Q60智能电视,搭载了Keep AI大屏互动健身产品,基于Keep
动作库和数据模型,结合Keep的AI算法,借助电视端摄像头功能,在捕捉用户动作轨迹的同时进行标准度打分和提供健身指导。同时,Keep
AI大屏互动健身产品整套操作系统中加入了智能语音控制,通过对话便可控制开始、结束、切换动作等流程。\sphinxhref{https://coffee.pmcaff.com/article/13242929\_j}{7}%
\begin{footnote}[773]\sphinxAtStartFootnote
\sphinxnolinkurl{https://coffee.pmcaff.com/article/13242929\_j}
%
\end{footnote}


\paragraph{快快第二代智能健身系统搭 10\sphinxfootnotemark[774]}
\label{\detokenize{chapter_project/AI_fit:id8}}%
\begin{footnotetext}[774]\sphinxAtStartFootnote
\sphinxnolinkurl{https://coffee.pmcaff.com/article/13646585\_j}
%
\end{footnotetext}\ignorespaces 
配合上课的智能运动臂带,0.01秒监测人体脉搏波形,媲美医用级心率带,每次课后都会进行72项指标的运动评估,为用户带来了“千人千面”智能健身体验;快快智能运动膝带,采用双腿4颗9轴传感器,支持跑步模式+课程模式,提高运动效率还能保护膝盖安全。

快快第二代智能健身系统搭载的165吋智慧大屏,两个乒乓桌的面积大,可以做到1:1真人等比上课,自研解码算法,在国内率先实现了60fps高帧率物理分辨率6K直播技术。在运动过程中,系统还支持每秒数百组数据实时回显,可满足多人多组PK赢取金币,让运动娱乐化。

可以说,快快第二代智能健身系统通过搭建高清数字矩阵和终端智慧教室,加速了健身场景的智能化改造,用自己的AI能力去给更多产品赋能,进入到了一个技术和数据推动的阶段。


\paragraph{Freeletics应用程序 12\sphinxfootnotemark[775]}
\label{\detokenize{chapter_project/AI_fit:freeletics-12}}%
\begin{footnotetext}[775]\sphinxAtStartFootnote
\sphinxnolinkurl{https://ai.51cto.com/art/202011/632683.htm}
%
\end{footnotetext}\ignorespaces 
Freeletics应用程序,它利用一系列AI流程来创建自定义锻炼,然后对其进行维护和修改以优化用户的喜好和发展。Freeletics应用程序首先收集少量个人数据。然后,它与其他用户和锻炼的海量数据库进行交叉引用,以创建建议的开始程序。

然后,该应用程序将跟踪用户的进度并接受反馈,以继续使他们的锻炼达到满意程度。无论是一般健身,针对单个肌肉群或身体部位,减肥或其他健身目标,Freeletics都使用机器学习。通过机器学习,用户将获得反馈,该反馈将使用户长期坚持的常规操作归零。其他应用程序使用人体姿势估计来检测和分析体育活动中的人体姿势。


\paragraph{Asensei智能穿戴}
\label{\detokenize{chapter_project/AI_fit:asensei}}
Asensei是一家智能服装开发人员,提供一套衬衫和裤子,能够跟踪用户进行的诸如弓步,下蹲和类似常规动作等身体运动的运动。Asensei智能服装使用运动捕捉和AI技术将用户的角度和运动范围与可接受的运动形式规范进行比较,并可以实时纠正用户以养成良好的运动习惯。

Sensoria提供了类似的基于AI的可穿戴系统,该系统专门针对慢跑和跑步而设计。Sensoria平台从智能服装(Sensoria自己的服装或其他具有IoT功能的服装)中收集数据。数据可测量一系列运动和生物特征。这包括心率,脚着地和脚踏的速度以及跑步时的冲击力。

Sensoria分析不仅为锻炼程序提供了优化和改进的建议,而且可以监视和发现等待中的潜在伤害并识别动力学链中的薄弱环节。Sensoria系统的设计既注重健康又适合从事积极生活方式的用户。


\paragraph{运动技术的可穿戴设备}
\label{\detokenize{chapter_project/AI_fit:id9}}
纳迪(Nadi)是一位专注于瑜伽的智能服装创造者,使用许多相同的AI和身体捕捉技术来生产智能瑜伽裤。

这些绑腿以无线方式连接到用于移动设备的应用程序,并且该移动应用程序提供了通过预定例程进行的瑜伽教程。同时,绑腿本身利用一系列轻柔的振动来提供指导,以指导例程的每个步骤都应关注身体的哪些部位。


\subparagraph{健身环大冒险}
\label{\detokenize{chapter_project/AI_fit:id10}}
去年10月18日上市时,官方销售价格(含)为79.99美元,约合560元人民币。2月22日,京东平台上的游戏报价已达1799+元,较最初的官方定价提升3倍,一度卖到断货。由此,《健身环大冒险》被网友们戏称为“2020年度最佳理财产品”。3月12日,游戏在淘宝平台售价跌为1248元。尽管后期有小幅上涨,但当人们开始复工复产后,价格就一直回落,最终下降到1000元左右。

AI私教又不够人性化,不能检测到玩家的每一个动作是否标准,也不晓得玩家是否穿着拖鞋、瘫在床上来偷懒作弊。


\subparagraph{超级猩猩APP 21\sphinxfootnotemark[776]}
\label{\detokenize{chapter_project/AI_fit:app-21}}%
\begin{footnotetext}[776]\sphinxAtStartFootnote
\sphinxnolinkurl{https://www.visbodyfit.com/a/xinwenzixun/2020/1221/138.html}
%
\end{footnotetext}\ignorespaces 
\begin{figure}[H]
\centering
\capstart

\noindent\sphinxincludegraphics{{fit_app}.png}
\caption{来源:MobData研究院公开资料}\label{\detokenize{chapter_project/AI_fit:id24}}\end{figure}

区别于超级猩猩的线上直播课程,目前所有App端的课程对有邀请资格的用户免费。根据超级猩猩线上负责人刻奇透露,本次公测随机邀请部分活跃猩友参与体验,首日邀请用户超过500人,并将逐步扩大邀请范围。

而社群功能在超级猩猩App中被设定为「圈子」,目前没有对全体用户开放创建圈子的权限,仅认证后的教练和官方可以创建,这样主要是为了方便形成良好的讨论氛围,降低用户探索成本,之后会逐步开放。

此外,用户还可在APP上通过教练账户页面直接预约课程,还可以在课中和课后,使用门禁密码推送和照片推送功能。据官方称,推出APP旨在了解用户需求、完善服务场景和提高用户体验。

App是精细化运营的需求。超级猩猩全国门店数量约100家,累计会员数量100万人。从服务会员体量上来说,仅靠微信生态的线上运营用户的方式已不能满足超级猩猩更加精细化的服务需求,另外目前超级猩猩扩店的进度早已放缓,从而企业的重心转移到更精细化的用户运营上,接下来就是如何用线上做存量和增量。

疫情期间,超级猩猩在「一直播」平台直播,一次直播几个教练带动十万人参与运动,明星教练的效率前所未有的高,但效益是怎样的呢?

App是精细化运营的需求。超级猩猩全国门店数量约100家,累计会员数量100万人。从服务会员体量上来说,仅靠微信生态的线上运营用户的方式已不能满足超级猩猩更加精细化的服务需求,另外目前超级猩猩扩店的进度早已放缓,从而企业的重心转移到更精细化的用户运营上,接下来就是如何用线上做存量和增量。

而在2020年12月主打“零售课程”已经拿到D轮融资的超级猩猩截至2020年底累计开店数达113家。


\paragraph{优质教练}
\label{\detokenize{chapter_project/AI_fit:id11}}
就国内健身器械商转型对标的Peloton而言,用户月活高达21.1次,相比于去年同期的12.6次,翻了将近一倍。

高月活的背后,是平台上的优质教练。类似于SoulCycle的明星教练,在美国社媒Reddit和Instagram上,用户对于Peloton明星教练,如粉丝般的狂热追逐,热度不亚于Netflix的演员,对明星教练和优质内容的认可,是构成用户几乎每天使用Peloton的关键。


\paragraph{更多 23\sphinxfootnotemark[777]}
\label{\detokenize{chapter_project/AI_fit:id12}}%
\begin{footnotetext}[777]\sphinxAtStartFootnote
\sphinxnolinkurl{https://36kr.com/p/1087630409318663}
%
\end{footnotetext}\ignorespaces 
咕咚、75派推出智能跳绳,通过传感器计数,还可将数据同步到手机APP,乔山旗下品牌Matrix
Fitness引入iFit交互平台,金史密斯、云麦加入小米生态链体系,舒华跑步机也加入到华为DFH闭环生态圈等。


\paragraph{技术能力}
\label{\detokenize{chapter_project/AI_fit:id13}}
依赖于开发者在三维动态捕捉、深度学习建模等领域的技术能力
\sphinxhref{https://www.tmtpost.com/4257148.html}{1}%
\begin{footnote}[778]\sphinxAtStartFootnote
\sphinxnolinkurl{https://www.tmtpost.com/4257148.html}
%
\end{footnote}

微软研究院利用消费者手中的智能手机相机进行远程医疗等领域的非接触式生理测量。测量人体随时间而产生的细微变化,而通过捕捉这些肉眼无法察觉的变化能够提供非常多的生理信息。\sphinxhref{https://ai.51cto.com/art/202012/633705.htm}{13}%
\begin{footnote}[779]\sphinxAtStartFootnote
\sphinxnolinkurl{https://ai.51cto.com/art/202012/633705.htm}
%
\end{footnote}


\paragraph{体感游戏 4\sphinxfootnotemark[780]}
\label{\detokenize{chapter_project/AI_fit:id14}}%
\begin{footnotetext}[780]\sphinxAtStartFootnote
\sphinxnolinkurl{https://www.infoq.cn/article/qiciiwtdpujamorfuijq}
%
\end{footnotetext}\ignorespaces 
体感游戏不仅仅意味着手机产品可以打开全新的游戏类别,更多的是可以与智能穿戴、大屏产品以及其他类似于外置手柄一类的外设连带售卖。这与手机厂商近来拓宽产品线、增加
IoT SKU 的经营理念的相符的。在 PC
版“吃鸡”风行时,游戏配置对于硬件的高要求甚至掀起了一阵换机潮。或许一款优秀的手机体感游戏,也能带动很多
IoT 产品的出售。


\subparagraph{Switch+健身环 25\sphinxfootnotemark[781]}
\label{\detokenize{chapter_project/AI_fit:switch-25}}%
\begin{footnotetext}[781]\sphinxAtStartFootnote
\sphinxnolinkurl{https://mp.weixin.qq.com/s?\_\_biz=MzU1MzExMzA5OA==\&mid=2247600065\&idx=1\&sn=0546543150c6dae8e40d192e7d8d92c2\&chksm=fbf4c4a4cc834db201e0624427588f671bcc0300f09c46d16c27bbc29729d42d344a9b349750\&scene=132\#wechat\_redirects}
%
\end{footnotetext}\ignorespaces 
家门都不用出

「健身环大冒险」:要打怪升级,你需要完成深蹲、抱膝式、向下推压等一系列动作,不断解锁新的健身技能。

「舞力全开2021」,你的女团男团梦、街舞dancer梦,多样的曲风搭配不同难度,30分钟有氧模式,有歌单不用花时间选歌,省事。


\subparagraph{Keep}
\label{\detokenize{chapter_project/AI_fit:keep}}
毫无基础培训一个月就上岗当健身教练?Keep不懂健身?这里讲述的是中国最真实的健身行业
\sphinxhyphen{} King James的文章 \sphinxhyphen{} 知乎 \sphinxurl{https://zhuanlan.zhihu.com/p/26689698}


\paragraph{会员卡营销 26\sphinxfootnotemark[782]}
\label{\detokenize{chapter_project/AI_fit:id15}}%
\begin{footnotetext}[782]\sphinxAtStartFootnote
\sphinxnolinkurl{https://weread.qq.com/web/reader/46532b707210fc4f465d044k4e73277021a4e732ced3b55}
%
\end{footnotetext}\ignorespaces 

\subparagraph{借鉴互联网裂变新玩法。}
\label{\detokenize{chapter_project/AI_fit:id16}}
比如,你可以只卖月卡,一方面告诉用户健身中心不会关闭,另一方面可以过滤一些不能坚持健身的用户。当你明确和用户说不给不能坚持健身的用户办会员卡是一个门槛时,反而会刺激用户的自尊心,用户就会坚持健身。然后,你可以让老用户介绍新用户来办卡,让老用户获得会员延期。再比如,你可以通过拼团、组队砍价等形式拉新用户。


\subparagraph{借鉴互联网打卡玩法。}
\label{\detokenize{chapter_project/AI_fit:id17}}
你可以给年卡会员设计一个全年打卡300次的全返奖励机制,用户只要一年内来过300天,健身中心就可以给他们全额退还年费,在一般情况下只有10\%的用户才会坚持下来。


\paragraph{问题 11\sphinxfootnotemark[783]}
\label{\detokenize{chapter_project/AI_fit:id18}}%
\begin{footnotetext}[783]\sphinxAtStartFootnote
\sphinxnolinkurl{http://www.woshipm.com/ai/990247.html}
%
\end{footnotetext}\ignorespaces \begin{enumerate}
\sphinxsetlistlabels{\arabic}{enumi}{enumii}{}{.}%
\item {} 
用户决策成本高,健身房售课缴费动辄包年包季,可看到健身效果却需要3\sphinxhyphen{}6个月。

\item {} 
从业人员复制率低,培养一名优秀教练的难度不亚于培养一名医生,现如今整个健身行业都以销售为导向,教练缺乏经验,自然也不能给用户提供好的服务。

\item {} 
行业中同质化竞争严重,标准化的团操、标准化的设备、标准化的收费模式之下,商家很难建立起壁垒,也容易陷入无底线的竞争。

\end{enumerate}


\paragraph{机会 11\sphinxfootnotemark[784]}
\label{\detokenize{chapter_project/AI_fit:id19}}%
\begin{footnotetext}[784]\sphinxAtStartFootnote
\sphinxnolinkurl{http://www.woshipm.com/ai/990247.html}
%
\end{footnotetext}\ignorespaces \begin{itemize}
\item {} 
用户层面的体验机会。像Enflux、肌动科技和其他智能设备、智能健身房的出现,可以对用户健身的体验和效率进行改变,逐渐减少健身中的用户决策成本。

\item {} 
行业层面的壁垒机会。当AI带来健身体验的个性化,用户在进行决策时考虑也就不仅仅是场馆距离家近不近、费用多少这些容易被无底线竞争破坏的因素,而开始考虑这家健身房的解决方案是否适合自己。

\item {} 
产业层面的模式机会。当体验更好的智能健身解决方案普及度越来越高时,整个健身行业也就不再囿于卖课、卖卡、卖加盟这样单一的盈利方式。在未来专门适用于健身的“体育云”、健身算法训练师、更丰富的智能设备等等都可能出现,整个健身产业的营收模式会变得更加丰富。

\end{itemize}


\paragraph{相关项目}
\label{\detokenize{chapter_project/AI_fit:id20}}
\sphinxurl{https://www.bilibili.com/video/BV1yp4y1i7R9/?spm\_id\_from=trigger\_reload}
\sphinxurl{https://github.com/Xu-Hanjia-SDUWH/Human-Activity-Recognition-HIIT}


\paragraph{社区氛围 27\sphinxfootnotemark[785]}
\label{\detokenize{chapter_project/AI_fit:id21}}%
\begin{footnotetext}[785]\sphinxAtStartFootnote
\sphinxnolinkurl{https://zhuanlan.zhihu.com/p/93475087}
%
\end{footnotetext}\ignorespaces 
添加2个互动指针: 好感度、熟悉度。 把原来男女都是一键分享的模式,
改为女性可以根据对男性“好感度”和“熟悉度”决定分享的范围。
这样女性用户觉得更有安全感。
同时男用户为了提高在女方心里的“好感度”和“熟悉度”,
会主动提高互动的频率, 努力活跃气氛。

社区可以设计一个“想看”功能。
让男用户邀请指定的女用户拍摄某一类运动视频。
其他男用户可以加入一块邀请, 表明“想看”。 对女性用户来说,
如果有好几个男用户都邀请自己拍视频, 其实小小地满足了下女孩子的虚荣心,
女孩子也会很乐意拍摄。


\paragraph{Amesome}
\label{\detokenize{chapter_project/AI_fit:amesome}}
\sphinxurl{https://36kr.com/projectDetails/79445}


\paragraph{示例}
\label{\detokenize{chapter_project/AI_fit:id22}}
Keep竞品分析报告:\sphinxurl{https://t.qidianla.com/1163465.html}


\subsubsection{AI 硬件 1\sphinxfootnotemark[786]}
\label{\detokenize{chapter_project/AI_hardware:ai-1}}\label{\detokenize{chapter_project/AI_hardware::doc}}%
\begin{footnotetext}[786]\sphinxAtStartFootnote
\sphinxnolinkurl{https://www.jianshu.com/p/111d9fcc005e?utm\_campaign=maleskine\&utm\_content=note\&utm\_medium=seo\_notes\&utm\_source=recommendation}
%
\end{footnotetext}\ignorespaces 
智能硬件比如现在流行的智能音箱和智能手环等


\paragraph{不确定性 6\sphinxfootnotemark[787]}
\label{\detokenize{chapter_project/AI_hardware:id1}}%
\begin{footnotetext}[787]\sphinxAtStartFootnote
\sphinxnolinkurl{https://zhuanlan.zhihu.com/p/22551432}
%
\end{footnotetext}\ignorespaces 
智能硬件产品不仅具有极端的不确定性还具有比互联网产品经理需要花更多精力打磨的地方,比如供应链,核心元件:例如芯片、电池等。总的体会来说智能硬件产品经理包含了部分互联网产品经理的职能,在这条路上个人最大额体会是:确定的事情是留给机器人(智能硬件)做的,不确定的研究创新是给人(产品经理)做的,看来不停地专研是产品经理的常态!


\paragraph{方案+研究+生产}
\label{\detokenize{chapter_project/AI_hardware:id2}}
做方案:
\begin{enumerate}
\sphinxsetlistlabels{\arabic}{enumi}{enumii}{}{.}%
\item {} 
研究市场需求

\item {} 
出MDR,定义市场需求(定目标人群、定卖点、定关键特性、定价)

\item {} 
和ID(工业设计)团队一起出方案,出效果图

\item {} 
拿方案在内部评审可行性,直到比较有把握

\item {} 
ID开始设计和打样,做Mockup

\item {} 
做市场调查,找用户提提意见

\item {} 
拿方案去对客户宣讲,看看客户的反馈(一般是合作过的老客户,比如运营商、渠道商;新客户都是拿成品去拓展的)

\item {} 
根据市场调研反馈修改设计

\end{enumerate}

研发和生产阶段:

\begin{figure}[H]
\centering
\capstart

\noindent\sphinxincludegraphics{{AI_hardware_process}.jpg}
\caption{AI赋能智能软硬件整体产品设计与开发过程}\label{\detokenize{chapter_project/AI_hardware:id33}}\end{figure}
\begin{enumerate}
\sphinxsetlistlabels{\arabic}{enumi}{enumii}{}{.}%
\item {} 
ID团队出正式的方案(中间需要反复修改设计、去喷漆厂调色、做高保真Mockup)

\item {} 
MD(结构设计)出结构方案

\item {} 
硬件出硬件方案

\item {} 
MD和硬件一起选购元器件,定BOM

\item {} 
软件研发软件方案

\item {} 
开模,需多次修模

\item {} 
量产

\end{enumerate}


\paragraph{硬件AI PM}
\label{\detokenize{chapter_project/AI_hardware:ai-pm}}
智能硬件产品经理一种是规划应用AI技术,统筹规划设计生产出智能硬件产品的产品经理角色。

硬能力核心流程如下图:

\begin{figure}[H]
\centering
\capstart

\noindent\sphinxincludegraphics{{hardwareAI_PM}.png}
\caption{硬件AI PM}\label{\detokenize{chapter_project/AI_hardware:id34}}\end{figure}


\paragraph{硬件PM VS 互联网PM}
\label{\detokenize{chapter_project/AI_hardware:pm-vs-pm}}

\subparagraph{知识面不同: 2\sphinxfootnotemark[788]}
\label{\detokenize{chapter_project/AI_hardware:id3}}%
\begin{footnotetext}[788]\sphinxAtStartFootnote
\sphinxnolinkurl{http://www.woshipm.com/pmd/1815501.html}
%
\end{footnotetext}\ignorespaces \begin{itemize}
\item {} 
软件产品经理通常需要关注市场和用户调研和分析、产品设计、用户体验、UI设计、开发技术、数据分析、运营维护、拉新促活和召回、商业变现、推荐推送等方面的知识。

\item {} 
硬件产品经理除了需要关注市场和用户调研和分析、产品设计之外还需要关注、方案设计、ID开模、电子电路、成本控制、包装设计、质量把控、营销渠道、售后服务等各方面知识。

\end{itemize}


\subparagraph{合作人不同:}
\label{\detokenize{chapter_project/AI_hardware:id4}}
\begin{figure}[H]
\centering
\capstart

\noindent\sphinxincludegraphics{{PM_Internet_VS_hardware}.png}
\caption{硬件PM VS 互联网PM}\label{\detokenize{chapter_project/AI_hardware:id35}}\end{figure}
\begin{itemize}
\item {} 
互联网产品经理一般合作的团队成员主要有设计、开发、测试、运营、市场这几个主要岗位,如果是大公司还有需求分析师、交互设计师、用户研究员等相关岗位。

\item {} 
硬件产品经理一般是和ID设计、平面设计师、结构设计师、电子工程师、软件工程师、采购、品控、销售、售后、技术支持、仓库管理员、供应商、代工厂、模具成、SMT厂、包装厂等相关职位以及合作伙伴进行合作。因为需要面对很多外部人员以及多方合作,所以硬件产品经理也需要具有来更好地协调能力和规划能力。

\end{itemize}


\subparagraph{定位不同:参谋 VS 领班 3\sphinxfootnotemark[789]}
\label{\detokenize{chapter_project/AI_hardware:vs-3}}%
\begin{footnotetext}[789]\sphinxAtStartFootnote
\sphinxnolinkurl{http://www.woshipm.com/pmd/134575.html}
%
\end{footnotetext}\ignorespaces \begin{itemize}
\item {} 
硬件团队中职能团队要广的多,老板难以掌控所有的工作,所以必须专设两类助手——产品经理(找业务方向)和项目经理(推动项目执行)。尽管PM也没有管人的”实权“,但因为身背业绩,所以在业务的发言权要大很多——比如例会上,一般是老板坐镇,所有团队负责人参与,然后产品经理和项目经理轮流汇报工作。在这个过程中,任何问题都可能被提出来,由对应职能部门负责人解释。一旦被被挑战,压力非常大。

\item {} 
互联网团队通常采用扁平化管理,产品、设计、研发、运营都是彼此协作的
职能团队,职位高低上并无不同。由于团队中一般不专设项目经理,就把PM抓来同时兼任。由于这种”项目经理“并无分配资源的”实权“,又必须推动项目实
施,所以”产品汪“天天跪舔”攻城狮“就一点都不奇怪了。

\end{itemize}


\subparagraph{KPI不同:收入考核 VS 用户考核}
\label{\detokenize{chapter_project/AI_hardware:kpi-vs}}\begin{itemize}
\item {} 
硬件产品没有互联网那么多眼花缭乱的商业模式。硬件PM的KPI非常简单粗暴,就是收入和利润是否达标。无论市场眼光准不准、方案完成度高不高、产品有没有竞争力、情怀能不能打动人,最终都体现在一件事上——能不能把东西卖掉并赚到钱!

\item {} 
互联网产品一般不向用户收费,很多产品在上线多年后都不能变现,用收入考核产品经理根本不现实。所以产品经理一般是与运营一起背用户增长、流量等。

\end{itemize}


\subparagraph{关注点不同:商业价值 VS 用户体验}
\label{\detokenize{chapter_project/AI_hardware:vs}}\begin{itemize}
\item {} 
互联网PM更关注完整的用户体验,对用户心理的揣摩更细腻,比如用切换卡导航还是抽屉导航,用图标按钮还是文字链等等。

\item {} 
硬件PM虽然也关注用户体验,但更需要衡量每一点改善的投入产出比,如果不能带来销量的提升就尽可能砍掉。比如我从来没见过哪个传统电视机厂家把遥控器做的好用一些,因为这对销售帮助极小。

\end{itemize}


\subparagraph{项目追求不同:高完成度VS快速上线}
\label{\detokenize{chapter_project/AI_hardware:id5}}\begin{itemize}
\item {} 
发布硬件产品,则需要从市场到设计、研发、生产、销售、售后全部团队的重度参与,一旦量产就不能容许有严重BUG。所以在立项起各阶段的检查就需要极为严格,在方案达标后必须及时关闭迭代,保证产品的高完成度。硬件产品试错成本很高,必须谨慎计划,严肃执行。

\item {} 
互联网常常拉一拨人几个月就上线顺便公测,效果不错就加大投入,效果不佳就赶紧换方向或砍掉。上线后就算有严重BUG,也不算事儿,有问题下一版升级就好了。互联网产品试错成本很低,可以小步快跑,随时掉头

\end{itemize}


\subparagraph{成本意识不同:非常关注成本VS不太关注成本}
\label{\detokenize{chapter_project/AI_hardware:id6}}\begin{itemize}
\item {} 
硬件对配置斤斤计较:由于每件硬件产品最终都是一笔交易,所以控制成本就极为重要。因此从元器件采购到确定BOM都需要PM深度参与,当然专业意见还是技术部门出,订货过程由项目经理跟。

\item {} 
互联网产品不考虑成本:一般初期投入成本很低(只有一些人力投入),用户增长的边际成本也极低。所以互联网PM基本不太需要考虑成本,更关心如何吸引用户。

\end{itemize}


\paragraph{互联网思维对硬件的束缚}
\label{\detokenize{chapter_project/AI_hardware:id7}}
互联网的“反硬件基因”,能用app解决的事情,坚决不用硬件


\paragraph{AI产品商业简史}
\label{\detokenize{chapter_project/AI_hardware:ai}}\begin{itemize}
\item {} 
三大应用:语音、视觉、机器翻译

\item {} 
四大品类:智能音响、家庭机器人、翻译机、AI相机

\end{itemize}


\paragraph{智能音响商业史}
\label{\detokenize{chapter_project/AI_hardware:id8}}
Echo诞生之初着力点是运算能力和高音质,因此价格199美金。此时市场为蓝海,通过降价策略吸引了一部分观众后,用户体验后感觉良好。随后google感到危机加入战场,凭借其互联网搜索引擎的家底占据了一部分市场。随着技术的成熟,价格下降成为趋势,通过低价位,echo进一步巩固市场地位,google随后也向低档下手。最后屏幕音响扰乱市场。


\subparagraph{智能音响商业策略}
\label{\detokenize{chapter_project/AI_hardware:id9}}
三大基本策略:应用渗透(以产品服务的渗透率为第一目标)+生态延伸(自身产品生态的延伸,如HomePod)+价值割据(围绕用户价值改进产品,建立优势巩固壁垒)

企业需要考虑的:战略贯穿(以引流为目的,盈利优先级低)+结局导向(确保能落地)+全局商战(企业的利弊权衡+需求程度和用户接受度+软件加硬件加商业化运作)


\subparagraph{智能机器人发展史}
\label{\detokenize{chapter_project/AI_hardware:id10}}
14年的妖风初起,资本夸大机器人市场;15年巅峰在望,消费级机器人品类增加;16年由盛转衰,回归理性;17年资本降温,强弩之末,或突围或止损;18年回首,一地鸡毛。

底层需求与价值:教育(早教机器人、编程机器人)、娱乐(玩具)、效率(扫地机器人、音响)

极点产品设计:极点形态(用户选择)、极点功能(需求)、合理的价格区间

商用机器人市场大于消费级机器人,仅仅炫技而无法落地的机器人很难生存

商用机器人:物流机器人+农场机器人+安防机器人+公关+外骨骼+医疗


\subparagraph{智能翻译机发展史}
\label{\detokenize{chapter_project/AI_hardware:id11}}
翻译机相对app的卖点:使用可靠性、识别准确性、操作简易性

翻译机的商业策略:产品演化+抢占市场


\subparagraph{机器视觉产品应用}
\label{\detokenize{chapter_project/AI_hardware:id12}}
技术赋能,给老产品带来新体验


\paragraph{互联网思维做不好AI硬件}
\label{\detokenize{chapter_project/AI_hardware:id13}}
\begin{figure}[H]
\centering
\capstart

\noindent\sphinxincludegraphics{{Internet_VS_hardware}.png}
\caption{互联网思维和硬件思维的差异}\label{\detokenize{chapter_project/AI_hardware:id36}}\end{figure}
\begin{itemize}
\item {} 
功能:互联网思维是设计功能、满足需求,硬件领域则是要达到用户的预期。如Echo

\item {} 
设计:软件产品强调极致,硬件产品关注全局整体性。如AirPods

\item {} 
价格:互联网的免费思维背后是流量,硬件则应该一开始就考虑产品定位和定价。硬件的渗透是渐进的,无法复制软件的导流。

\item {} 
开发机制:软件快节奏,容错能力强;硬件重质量,容错能力弱

\end{itemize}


\paragraph{为什么体验好的产品,卖不好}
\label{\detokenize{chapter_project/AI_hardware:id14}}
体验最好不代表最合适。合适的重要性远大于体验,提升体验着成本的上身,用户只买对的,不买贵的。销量是衡量产品最普适的指标。


\paragraph{AI硬件的3种模式}
\label{\detokenize{chapter_project/AI_hardware:ai3}}\begin{itemize}
\item {} 
硬件模式(AI+硬件):硬件是主体,AI可有可无。如智能手机的拍照、无人机的镜头、加入新技术的玩具、带有语音助手的耳机、智能手表、智能家居(音响主控,其他的家电被连接)、相机、眼镜

\item {} 
互联网模式(AI+管道):智能音响、智能翻译机、智能电视、家庭监控

\item {} 
资本模式(资本+梦想):下一代交互/运算平台(消费者购买的是产品,不会为梦想买单,如TNT)+AI机器人(技术or泡沫?)

\end{itemize}


\paragraph{AI硬件创新}
\label{\detokenize{chapter_project/AI_hardware:id15}}\begin{itemize}
\item {} 
用户决策:轻决策(门槛低、风险低、心动就会买,忌花里胡哨抬高价格)+重决策(门槛高、风险高、没有必要就不买,不能妥协性能)

\item {} 
产品演进:压缩成本(轻决策,平民化)+提高价值(重决策,价值穿透)

\item {} 
推广路径:轻决策,依靠平价,快速渗透;重决策,穿透核心价值用户后才能抵达大众,强贯穿

\end{itemize}


\paragraph{智能硬件}
\label{\detokenize{chapter_project/AI_hardware:id16}}
智能硬件看似复杂,拆解出来脉络很清晰。包含硬件(HW)、软件(SW)、外观(ID)、结构(MD)、互联网平台。

其中软件包含板级支持包(BSP)、底层引导程序(bootloade)、系统与应用程序、算法,这些不展开来讲,找固件打包的工程师就
OK ,一般所有的程序都汇总到他那儿了。

作为项目经理,不太需要进行深入的了解,当然能够深入更好,但作为产品经理还是更深入一点较好。

互联网平台,这个包含云服务、后台、App、小程序等。常见的是前三个。跟进对应的工程师就好。

\begin{figure}[H]
\centering
\capstart

\noindent\sphinxincludegraphics{{hardwareAI_flow_chart}.png}
\caption{AI 智能硬件流程图}\label{\detokenize{chapter_project/AI_hardware:id37}}\end{figure}


\paragraph{项目研发}
\label{\detokenize{chapter_project/AI_hardware:id17}}
项目研发分为EVT阶段、DVT阶段、PVT阶段、MP阶段和维护阶段

EVT 阶段:(Engineering Verification
Test),指工程验证。一般在工程样机之前的研发行为,我都称之为工程验证。

这个阶段,目的是工程验证。尽可能的发现设计问题,方案对比。

最终拿到的是工程样机,用于样机整机测试,判定是否可以开模。

DVT 阶段:(Design Verification
Test),指设计验证测试。最终拿到的是试产的整机样机,用于多方联调,验证优化。

上一个阶段,完成产品的雏形,这个阶段继续上个阶段的设计开发、优化。MD
详细设计完成,开始投模、试模、修模、颜色调制等。

试产模具,组装整机,进行硬件/结构的整机测试。软硬件、结构、互联网平台多方联调。比如软硬件的稳定性、可靠性、性能等;软件与互联网平台(云服务/App等)联调测试;硬件与结构的联调测试,比如散热、结构强度等。

另外,这在这阶段关于产品的贴纸、说明书、包装等可以开始设计/打样,然后等待,因为这些时间周期比较短。

如果软硬件状态比较理想,在这个阶段尽早安排认证。因为认证周期非常长,基本在
40 天左右,别等到产品快要量产了,认证还没出来,影响销售。

总之,这个阶段就是联调、测试、试模、打板、试产。

PVT 阶段 :(Process Verification
Test),指生产验证。进行小批量产,摸清生产工艺,测试工艺,为大批量产做准备。

这个阶段依然会进行各种验证,以及解决上一阶段遗留的一些小问题。但主要的精力放在一致性、设计(细节,比如按键手感不好,干涉等)调整上。

各部门处于生产支持模式,比如工程部制作
SOP(标准作业程序),结构部帮忙解决生产上的结构问题。与生产相关的测试工具、生产工具、烧录工具、产测工具的支持。

所有的生产支持文件规定当送到工厂,量产软件/量产硬件BOM/量产结构BOM,结构/元器件终版签样。

总之,这个阶段就是为了保证产品量产。
量产顺利,效率高,不良率最低,产品一致性够高。


\paragraph{智能硬件设计流程 5\sphinxfootnotemark[790]}
\label{\detokenize{chapter_project/AI_hardware:id18}}%
\begin{footnotetext}[790]\sphinxAtStartFootnote
\sphinxnolinkurl{https://weread.qq.com/web/reader/40632860719ad5bb4060856k98f3284021498f137082c2e}
%
\end{footnotetext}\ignorespaces 
智能硬件从智能穿戴设备开始,在智能硬件领域已经扩展出了诸如智能电视、智能家居、智能汽车、医疗健康、智能玩具、机器人等人工智能应用。如今比较典型的智能设备包括Google
Glass、三星Gear、Fitbit、麦开水杯、咕咚手环、Tesla、无屏电视等。智能硬件涉及领域广泛,与此相关的行业也非常多。一个完整的智能硬件产品通常拥有一个包含双流程的产品设计流程,如下图所示。

\begin{figure}[H]
\centering
\capstart

\noindent\sphinxincludegraphics{{hardware_design}.png}
\caption{智能硬件产品设计流程}\label{\detokenize{chapter_project/AI_hardware:id38}}\end{figure}


\subparagraph{需求分析}
\label{\detokenize{chapter_project/AI_hardware:id19}}
确认整体的业务场景、了解应用的技术、明确需要满足人什么需求,甚至对整体市场的情况进行评估,这个阶段是AI产品经理调研需求定义产品的阶段,是一个智能硬件产品生命周期的开始。


\subparagraph{产品形态定义}
\label{\detokenize{chapter_project/AI_hardware:id20}}
AI产品经理在这个阶段需要完成产品的整体方案,包括硬件和软件的相关功能,将产品形态,以文字、图片、模型等方式展示出来,完成对产品形态的定义。


\subparagraph{双流程设计需求}
\label{\detokenize{chapter_project/AI_hardware:id21}}
采集并完成产品方案设计后,会按照硬件设计和软件设计流程同步进行。在硬件的设计流程中,会涉及一些更加专业的流程,如BOM规划、ID工艺。


\subparagraph{BOM规划}
\label{\detokenize{chapter_project/AI_hardware:bom}}
BOM(Bill of
Material)指的是硬件产品所需物料明细表,BOM详细记录了一个项目所用到的所有材料及相关属性,母件与所有子件的从属关系、单位用量及其他属性,在有些系统称为材料表或配方料表。当AI明确产品经理的需求后,工业设计团队和研发团队会分工设计产品的结构、外观,包括对核心部件的选择,从而完成BOM规划,通过合理的BOM规划,可以最大限度地减少资源浪费,通过物料清单,AI产品经理能够了解基本的成本。


\subparagraph{ID工艺}
\label{\detokenize{chapter_project/AI_hardware:id}}
ID设计指的是工艺产品设计,主要指的是产品外观设计,该部分会有专业人员进行设计,ID设计需要考虑产品的美观、易用等性能。


\subparagraph{结构工艺}
\label{\detokenize{chapter_project/AI_hardware:id22}}
完成BOM规划和ID设计后,设计团队会进行结构工艺,如注射开模,然后进行小量试产,就会产生试用产品。

智能硬件还包括软件设计流程,该部分流程同大部分互联网产品设计流程一致,配合产品功能,需要进行软件功能设计,包括方案设计,如有相关操作界面,还需要增加界面设计、完成开发测试的流程。①
方案设计。软件设计部分需要了解数据存储方式和数据交互方式;硬件产品部分数据是存储在本地的,这与常见的互联网产品不同,如智能音箱唤醒词,需要特别注意的是,由于数据存储方式的差异而产生的边界情况。②
界面设计。智能硬件的屏幕不再是标准的手机界面,如可能是如手表的圆形界面,此外色彩呈现和交互方式也与手机有所不同,AI产品经理需要确认产品的载体及支持的展示方式,太过复杂的效果可能无法呈现。③
开发测试。软件部分的开发测试主要侧重于进行数据逻辑的验证,在硬件设计和软件设计阶段完成后,会进行软硬联调的工作。


\paragraph{智能硬件成本预估}
\label{\detokenize{chapter_project/AI_hardware:id23}}
智能硬件产品的成本主要包括原材料成本、生产成本和第三方成本。


\subparagraph{原材料成本}
\label{\detokenize{chapter_project/AI_hardware:id24}}
原材料成本是产品的直接成本,是组成产品的所有原材料的成本之和,一个硬件产品的原材料成本通常包含如下几种。
\begin{enumerate}
\sphinxsetlistlabels{\arabic}{enumi}{enumii}{}{.}%
\item {} 
PCB成本,PCB物料成本和PCB板上元器件成本,包括IC(主IC、电源管理IC、RF
IC、其他类IC)、存储(FLASH、RAM)、屏幕、电池、电阻电容电感的物料成本等。

\item {} 
结构物料成本,包括产品上盖、下盖、中框和按键等。

\item {} 
配件物料成本,包括电源适配器、数据线、耳机等,适配器基本为标配。

\item {} 
包装物料成本,包括外包装、内纸托等。

\item {} 
文档类物料成本,包括使用说明书,法规类说明文档等。

\end{enumerate}


\subparagraph{生产成本}
\label{\detokenize{chapter_project/AI_hardware:id25}}
生产成本指的是将产品原材料组装生产、研发、成品过程中所产生的费用,主要包括以下几项费用。
\begin{enumerate}
\sphinxsetlistlabels{\arabic}{enumi}{enumii}{}{.}%
\item {} 
生产组装费用,烧录、SMT、插件、包装费用。

\item {} 
生产检测类费用,产品性能类测试费用、产品法规类检测费用、产品品质类检测费用等。③
批量生产费用,工厂的一切日常活动都反映到机器产能和人工产能上,批量和产能越高,费用越低。

\item {} 
研发成本,主要为人力成本。

\item {} 
ID设计成本,产品外观设计费用;外观手模制作费用。

\item {} 
模具开模成本,产品ID开模、结构物料开模(屏蔽罩等)费用。

\item {} 
物料打样成本,样机制作成本。

\end{enumerate}


\subparagraph{第三方成本}
\label{\detokenize{chapter_project/AI_hardware:id26}}
由产品生产方支付给第三方公司的费用,主要包括:第三方专利费用、第三方软件授权费用、服务器费用、流量费用、云费用等。


\paragraph{都要懂硬件 7\sphinxfootnotemark[791]}
\label{\detokenize{chapter_project/AI_hardware:id27}}%
\begin{footnotetext}[791]\sphinxAtStartFootnote
\sphinxnolinkurl{https://zhuanlan.zhihu.com/p/163236280}
%
\end{footnotetext}\ignorespaces 
准产品经理是个学校培养的硬件工程师,但是志在硬件产品经理,首先他处在了产品经理起跑线的前沿,因为高段位的产品经理做到后来都是懂硬件的,哪怕最热门的数据产品经理或者什么什么产品经理,都是要懂硬件的,因为所有的数据都靠硬件采集、运算和存储。

小米智能音箱:\sphinxurl{https://www.bilibili.com/video/BV1wt411Y7zh}


\paragraph{嵌入式AI硬件 8\sphinxfootnotemark[792]}
\label{\detokenize{chapter_project/AI_hardware:ai-8}}%
\begin{footnotetext}[792]\sphinxAtStartFootnote
\sphinxnolinkurl{https://www.bilibili.com/video/BV1Zp4y1Q7ub?from=search\&seid=1470711389248919578}
%
\end{footnotetext}\ignorespaces 
\begin{figure}[H]
\centering
\capstart

\noindent\sphinxincludegraphics{{qianruduan}.jpg}
\caption{嵌入式AI硬件}\label{\detokenize{chapter_project/AI_hardware:id39}}\end{figure}


\paragraph{乔布斯}
\label{\detokenize{chapter_project/AI_hardware:id28}}
乔布斯在硬件产品领域的出身:

乔布斯并非硬件技术学院科班出身,但他却是200多项美国专利的发明人或共同发明人,我们都知道一个专利分为专利发明人、专利申请人,专利权人。

乔布斯是专利发明人,也就是说这些专利事是他干的,因为专利法所称发明人或者设计人,指对发明创造的实质性特点做出创造性贡献的人。在完成发明创造过程中,只负责组织工作的人、为物质技术条件的利用提供方便的人或者从事其他辅助工作的人,不是发明人或者设计人。

发明人是对该件专利技术具有贡献的人,一般来说在企业内,研发人员是发明人。

乔布斯不是硬件科班系出身但几乎是最懂PC电脑用户需求硬件的人。

\sphinxstylestrong{乔布斯在产品团队中的职责:}

从传奇级科技网络产品经理乔布斯做过的早期、中期和后期产品可以看到,乔布斯不仅仅是个懂用户创造并引领用户需求的会议型产品经理,更多的时候乔布斯知道用什么技术引领用户需求。


\paragraph{典型企业 9\sphinxfootnotemark[793]}
\label{\detokenize{chapter_project/AI_hardware:id29}}%
\begin{footnotetext}[793]\sphinxAtStartFootnote
\sphinxnolinkurl{http://www.woshipm.com/pd/990245.html}
%
\end{footnotetext}\ignorespaces \begin{itemize}
\item {} 
大疆产品追求先进性,企业文化是积极尽志,求真品城。风格激进,敢为天下先。创始人有鲜明的法家风格。以结果为导向,狼性文化,苛求真知灼见,拒绝平庸之辈。

\item {} 
OPPO产品追求差异性,企业文化是本分,敢为天下后,后中争先。创始人和兄弟品牌,都是鲜明的道家风格,平常心,不议同行,稳中求胜,把事作对。

\item {} 
小米产品追求经济性,企业特点是效率,唯快不破,不追求利润。虽然是硬件企业,背后却是有互联网模式的伟大愿景。

\end{itemize}


\paragraph{智能硬件产品需求文档 10\sphinxfootnotemark[794]}
\label{\detokenize{chapter_project/AI_hardware:id30}}%
\begin{footnotetext}[794]\sphinxAtStartFootnote
\sphinxnolinkurl{https://zhuanlan.zhihu.com/p/345731185}
%
\end{footnotetext}\ignorespaces 
1、需求修订历史

记录修订内容及时间,有便于团队进行了解

2、项目简介

产品是做什么的?

3、使用场景

产品将用在什么场景?

4、产品原则

产品原则性要求,如低功耗,7*24小时,安全可靠等

5、产品组成关系

各软件组成关系,硬件的组成关系

6、功能性需求

产品的具体功能有哪些,包括软件跟硬件功能,需重点详细描述,软件需带原型图

7、性能需求

产品需要什么性能,如功耗,使用寿命,运行速度等

8、接口需求

产品的内部通讯接口,以及产品的外部接口,包括产品端口

9、存储需求

元器件的性能,存储的大小,速度等

10、安全需求

对产品的安全性要求,如防静电、防雷击、防浪涌等

11、机械、电子设计需求

产品外壳材质、尺寸大小、PCB大小,丝印,端口位置、通风扇热等

12、环境需求

产品在什么产品下可以使用如高低温、湿度环境等

13、设计约束条件

如产品的最高成本限制、产品的最高功耗限制、产品效果性能的最低指标

14、可生产性需求

考虑产品在生产装配过程中部件之间的配合、定位等方面的问题,保证产品可以快速地、高效地且以最低的成本进行装配生产

15、可测试性需求

产品的各项功能、性能都可以被便捷地、全面地测试到位,并在测试中能够迅速而真实地获取产品的各部分状态和相关信息

16、核心元器件

将已经确定的核心元器件与团队成员进行介绍,提供元器件的型号、元器件功能、技术指标、性能指标等信息

17、嵌入式固件需求

对于嵌入式固件的功能和性能的说明,包括业务逻辑方面的处理、远程的配置控制、安全方面的保证机制、设备的OTA升级、设备的状态监控、设备的远程代理及设备出现问题后可以自动恢复的“看门狗”程序等。


\paragraph{TODO}
\label{\detokenize{chapter_project/AI_hardware:todo}}
更多:

\sphinxurl{http://www.woshipm.com/pmd/2337370.html}
\sphinxurl{http://www.woshipm.com/pmd/3301527.html}


\paragraph{AI硬件产品经理11\sphinxfootnotemark[795]}
\label{\detokenize{chapter_project/AI_hardware:ai11}}%
\begin{footnotetext}[795]\sphinxAtStartFootnote
\sphinxnolinkurl{http://www.chanpin100.com/article/113861}
%
\end{footnotetext}\ignorespaces 

\subparagraph{智能硬件产品经理}
\label{\detokenize{chapter_project/AI_hardware:id31}}
智能硬件比如现在流行的智能音箱和智能手环等,而智能硬件产品经理一种是规划应用AI技术,统筹规划设计生产出智能硬件产品的产品经理角色。产品经理需要对整个产品设计、开发、测试、试产和量产等过程监控,确保产品顺利按时保质保量的完成。硬件产品经理可能还需要对生产流程、质量控制方法等有一定了解,软件产品是没有生产工艺这一环节的。


\subparagraph{算力产品经理}
\label{\detokenize{chapter_project/AI_hardware:id32}}
大部分人了解到的AI硬件产品经理都是前面这一种,但其实在大型的AI厂商中还会存在后面这一类硬件产品经理。因为AI的三大要素是数据、算法和算力。对应的有数据产品经理、算法产品经理,自然也就有算力产品经理。只不过这部分产品经理更多是从服务器厂商出来的,因为AI的算力基本就是服务器资源了,对应的厂商有华为、浪潮、联想等。这一类产品经理相对来说对AI技术不需要很了解,对算力资源的配置和使用足够了解就可以了,简单了解不同场景应用哪类算力资源最合适即可。


\subsubsection{产品整个流程1\sphinxfootnotemark[796]}
\label{\detokenize{chapter_project/process:id1}}\label{\detokenize{chapter_project/process::doc}}%
\begin{footnotetext}[796]\sphinxAtStartFootnote
\sphinxnolinkurl{http://www.woshipm.com/pd/313514.html}
%
\end{footnotetext}\ignorespaces 

\paragraph{研究背景 3\sphinxfootnotemark[797]}
\label{\detokenize{chapter_project/process:id2}}%
\begin{footnotetext}[797]\sphinxAtStartFootnote
\sphinxnolinkurl{http://www.woshipm.com/pd/841065.html}
%
\end{footnotetext}\ignorespaces 
1、提高研发计划性
产品开发流程每个环节都涉及时间排期,这些时间管理要素可以有效控制项目时间表。

2、提高研发效率
通过明确开发团队每个角色的职责和协作方式,让每个成员只需严格按照规范做好自己的工作即可高效协作,降低沟通成本。

3、保证产品质量
通过确保每个环节的输入输出结果,让最终产出的产品得到有效保证。

4、及时发现问题 通过各环节过程数据,方便管理人员深入了解问题。


\paragraph{组建团队}
\label{\detokenize{chapter_project/process:id3}}
产品研发核心团队通常由产品经理(1名)、研发经理(1名)、研发人员(5\sphinxhyphen{}10名)组成。产品开发涉及的职责分配到各位成员身上。

角色见::ref:\sphinxcode{\sphinxupquote{prod\_people}}


\paragraph{工作流程 2\sphinxfootnotemark[798]}
\label{\detokenize{chapter_project/process:id4}}%
\begin{footnotetext}[798]\sphinxAtStartFootnote
\sphinxnolinkurl{http://www.woshipm.com/zhichang/459131.html}
%
\end{footnotetext}\ignorespaces 
收集产品需求(需求池)→ 评审需求→ 竞品分析 → 产品原型设计 → Demo评审→
UI评审→ 开发跟踪→上线前的测试 → 产品上线后的bug收集 →
对客服的培训,可根据实际情况酌情进行调整。

\begin{figure}[H]
\centering
\capstart

\noindent\sphinxincludegraphics{{process}.png}
\caption{流程}\label{\detokenize{chapter_project/process:id35}}\end{figure}


\subparagraph{AI PM 特殊点 6\sphinxfootnotemark[799]}
\label{\detokenize{chapter_project/process:ai-pm-6}}%
\begin{footnotetext}[799]\sphinxAtStartFootnote
\sphinxnolinkurl{http://www.changgpm.com/thread-253-1-1.html}
%
\end{footnotetext}\ignorespaces \begin{enumerate}
\sphinxsetlistlabels{\arabic}{enumi}{enumii}{}{.}%
\item {} 
需求把握:AI产品还处于探索期(找刚需),
ToC产品的产品形态甚至典型用户群体(用户画像)都还不明确,所以信息(行业信息、竞品信息、用户信息)收集、创意思考、产品验证的工作会更加困难。

\item {} 
闭环验证:在进行产品核心价值的设计和验证工作时,AI产品经理除了以数据分析为驱动,还需要有大胆的思路、敏锐的洞见。首先,从手机场景升级到AI场景,产品使用场景发生了翻天覆地的变化;其次,AI产品的用户门槛远高于互联网产品,用户量和用户数据的规模远比不上互联网产品的量级;再次,AI产品与硬件关系紧密,产品迭代周期更长,更难收集有效数据。

\item {} 
交互设计:场景巨变使得交互方式从纯软件(界面、触摸屏)、纯硬件的形式,升级到多模态交互等更复杂的人机交互形式,
AI场景还没有形成清晰的交互体系。很多人没有意识到新时代不仅是新技术驱动的,更关键的标志是新交互(还可能是新硬件)。

\item {} 
功能设计:比如,搜索产品和AI问答产品是很不同的东西,同时,产品一旦和硬件相关,难度就会陡增。

\item {} 
数据分析:语音交互产生的数据分析难度远高于触摸交互。因为触摸屏有效表达用户意图的概率非常高,而通过语音识别得到的信息往往和用户意图有很大的偏差

\end{enumerate}


\paragraph{七步}
\label{\detokenize{chapter_project/process:id5}}\begin{enumerate}
\sphinxsetlistlabels{\arabic}{enumi}{enumii}{}{.}%
\item {} 
市场调研

\item {} 
需求管理

\item {} 
产品设计

\item {} 
产品研发

\item {} 
产品测试

\item {} 
产品发布上线

\item {} 
项目跟进优化

\end{enumerate}


\paragraph{市场调研}
\label{\detokenize{chapter_project/process:id6}}\begin{itemize}
\item {} 
市场调查:分析行业现状和市场规模,发现并掌握目标市场和用户需求的变化趋势;

\item {} 
用户调研:通过用户访谈,可用性测试,调查问卷,数据分析的方法对用户需求进行挖掘和分析;

\item {} 
竞品分析:剖析产品的竞争对手,对其产品进行用户体验分析。

\item {} 
盈利分析:估算产品成本,验证产品需求。

\end{itemize}


\paragraph{需求管理}
\label{\detokenize{chapter_project/process:id7}}\begin{itemize}
\item {} 
产品规划:确定目标市场、产品定位、发展规划及路线图;

\item {} 
提炼需求:由市场或运营部门收集的需求来进行分析,提炼核心功能;

\item {} 
根据竞品分析,市场调研来对功能进行优先级排序;

\item {} 
版本规划:每个版本重点开发什么,预估研发进度,上线日期。

\end{itemize}


\paragraph{产品设计}
\label{\detokenize{chapter_project/process:id8}}\begin{itemize}
\item {} 
编写产品需求文档,确认产品周期。

\item {} 
产品原型要做的是梳理和完善产品需求流程,降低团队沟通成本。

\item {} 
跟设计师确立产品设计规范,从视觉效果角度确立选用图标类型,文字大小,模块间距,宽高大小等。

\end{itemize}


\paragraph{产品研发}
\label{\detokenize{chapter_project/process:id9}}\begin{itemize}
\item {} 
组织讨论,对需求进行评估及确认研发周期、提测时间、预发布时间点、正式发布时间点;

\item {} 
App的开发环境搭配,确定技术框架,以及研发各种基础系统等;

\item {} 
跟踪和推动项目进度,确保项目计划的完成;

\item {} 
布局产品运营工具,方便后期分析与用户跟踪。

\end{itemize}


\paragraph{产品测试}
\label{\detokenize{chapter_project/process:id10}}\begin{itemize}
\item {} 
测试周期是直接跟着开发周期一起做;

\item {} 
测试设备:确定要兼容的系统版本,手机品牌类型,手机分辨率等;

\item {} 
按照产品需求文档进行测试。

\end{itemize}


\paragraph{产品发布上线}
\label{\detokenize{chapter_project/process:id11}}\begin{itemize}
\item {} 
发布环境的搭建,包括预发布环境、生产环境、灰度发布环境的准备等工作。

\item {} 
而正式上线的工作,则包括数据库上线、程序文件上线等工作。

\item {} 
应用商店ASO优化并根据不同的应用商店作出调整;

\item {} 
协作运营部门做产品推广。

\end{itemize}


\paragraph{项目跟进优化}
\label{\detokenize{chapter_project/process:id12}}\begin{itemize}
\item {} 
根据用户反馈对功能进行改进,对用户体验进行优化;

\item {} 
对产品数据进行监控,分析产品运营效果、用户使用行为及需求,以便对产品进行持续性优化和改进;

\item {} 
根据数据挖掘新需求。

\end{itemize}
\begin{enumerate}
\sphinxsetlistlabels{\arabic}{enumi}{enumii}{}{.}%
\item {} 
研发工作总结:需要对产品研发过程做总结,不论是产品上的还是流程配合上的,为后续加强沟通协作、产品运营打好基础。

\item {} 
上线后收集用户反馈:为了更好的收集用户反馈,需要在所有产品上都增加反馈入口,以便用户提交反馈内容。每天都需要花费相当比例的时间去浏览,并将反馈建议\sphinxstylestrong{转化产品需求点加入需求池。}

\item {} 
产品体验可用性测试:邀请一批真实的典型客户,针对典型场景使用产品,用户研究员在一旁观察、聆听、记录,从而发现产品中存在的可用性缺陷。必须性,因为产品运营团队的员工往往潜意识里会认为用户一定会怎样操作,但是事实上用户很大概率上都不会按照他们希望的进行操作,甚至会陷入茫然根本用不下去。而通过可用性测试,就可以找到问题点,通过优化体验设计\sphinxstylestrong{降低用户使用门槛}。

\item {} 
运维系统优化:升级版本上线工作、服务监控、应用状态统计、日常服务状态巡检、突发故障处理、服务日常变更调整、集群管理、服务性能评估优化、数据库管理优化、随着应用PV增减进行应用架构的伸缩、安全、运维开发等工作。

\end{enumerate}


\paragraph{人工智能规划流程}
\label{\detokenize{chapter_project/process:id13}}
业内较为常见的设计流程是CRISP\sphinxhyphen{}DM(Cross\sphinxhyphen{}Industry Standard Process for
Data Mining,跨行业数据挖掘标准流程)

\begin{center}\sphinxincludegraphics{{CRISP-DM}.png}\end{center} \sphinxincludegraphics{{CRISP-DM_ability}.png}

在1996年的时候,SPSS,戴姆勒\sphinxhyphen{}克莱斯勒和NCR公司发起共同成立了一个兴趣小组,目的是为了建立数据挖掘方法和过程的标准。并在1999年正式提炼出了CRISP\sphinxhyphen{}DM流程。

这个流程确定了一个数据挖掘项目的生命周期,包括以下六个阶段(我做了部分修改):
\begin{enumerate}
\sphinxsetlistlabels{\arabic}{enumi}{enumii}{}{.}%
\item {} 
商业理解:了解进行数据挖掘的业务原因和数据挖掘目标(细节落实到算法各项指标的任务)。

\item {} 
数据理解:深入了解可用于挖掘的数据。

\item {} 
数据准备:对待挖掘数据进行合并,汇总,排序,样本选取等操作。

\item {} 
建立模型:根据前期准备的(训练、测试)数据选取合适的模型。

\item {} 
模型评估:使用在商业理解阶段设立的业务成功标准对模型进行评估。

\item {} 
结果部署:使用挖掘后的结果提升业务的过程。

\end{enumerate}

是否可以继续进行下一个阶段取决于是否有达到最初的业务目标,如果业务目标没有达到,那么就要考虑是否是数据不够充分或算法需要调整,一切都以业务目标为导向。

AI项目在产品开发过程和AI产品本身都需要一个“反馈循环”。因为人工智能产品本质上是基于研究的,所以实验和迭代开发是必要的。传统软件开发的输入和结果通常是确定的,而人工智能开发周期是概率性的。不管项目管理框架是什么,这都需要对项目的建立和执行方式进行一些重要的修改。\sphinxhref{https://www.oreilly.com/radar/bringing-an-ai-product-to-market/}{5}%
\begin{footnote}[800]\sphinxAtStartFootnote
\sphinxnolinkurl{https://www.oreilly.com/radar/bringing-an-ai-product-to-market/}
%
\end{footnote}


\subparagraph{商业理解}
\label{\detokenize{chapter_project/process:id14}}
海微购是一家从事跨境电商业务的创业公司,在前几年抓住了海淘的趋势,用户量和交易额都还不错。在新的财年,公司希望能在去年的基础上将GMV提高10\%,并以此为目标制定新一年的工作计划。


\subparagraph{了解客户和确定业务目标}
\label{\detokenize{chapter_project/process:id15}}
在内部,人工智能项目经理必须让利益相关者参与进来,以确保与最重要的决策者和顶级业务指标保持一致。

产品经理必须确保AI项目收集关于客户行为的定性信息。因为这可能不是直观的,需要注意的是,传统的数据测量工具在测量规模上比情绪更有效。对于大多数AI产品,产品经理\sphinxstylestrong{对点击率(CTR)和其他量化指标的兴趣不如对AI产品对用户的效用}感兴趣。因此,传统的产品研究团队必须与人工智能团队合作,以确保将正确的直觉应用到人工智能产品开发中,因为人工智能从业者可能缺乏适当的技能和经验。ctr很容易测量,但是如果您构建了一个旨在优化这类指标的系统,您可能会发现该系统牺牲了实际的实用性和用户满意度。在这种情况下,无论AI产品对这些指标的贡献有多好,它的产出最终都不会服务于公司的目标。

根据电商零售额公式(零售额=流量转化率客单价*复购率),公司认为:在获客成本较高的市场环境,以及本公司经营的海淘产品复购周期较长的情况下,应优先提高转化率和客单价两项指标。根据SMART目标制定原则,确定下一次迭代的产品目标为:猜你喜欢模块中的商品点击量需提高20\%,交叉销售额提高10\%。

如果你没有做适当的研究,你很容易将注意力集中在错误的度量上。我们采访的一家中型数字媒体公司报道称,他们的营销、广告、战略和产品团队曾经想要建立一个人工智能驱动的用户流量预测工具。市场营销团队建立了第一个模型,但因为它来自市场营销,所以该模型针对点击率和潜在客户转化率进行了优化。广告团队更感兴趣的是每潜在成本(CPL)和终身价值(LTV),而策略团队则与企业指标(收益影响和总活跃用户)保持一致。结果,很多工具的用户都不满意,尽管人工智能运行得很完美。最终的结果是开发了针对不同指标进行优化的多种模型,并重新设计了工具,以便能够将这些输出清晰、直观地呈现给不同类型的用户。


\subparagraph{确定数据挖掘目标}
\label{\detokenize{chapter_project/process:id16}}
需要明确数据挖掘的问题是一个分类问题,聚类问题还是一个预测问题,以便于后续的建模阶段选择合适的算法。另外,还需要确定的是数据挖掘的范围,是针对所有用户大范围调整,还是先针对小规模的部分用户进行A/B
Test,待验证成功之后再全面推行。

数据挖掘成功的标准是什么,例如:推荐的准确率要提高40\%,或用户的流失概率降低20\%等,通过可量化的指标评估整个工作的效果。


\subparagraph{制定项目计划}
\label{\detokenize{chapter_project/process:id17}}
根据具体的,可量化的方案组织相关的干系人来评估工作量。根据工作量倒排项目计划表,将目标拆解到更小的时间颗粒度,并指定相关责任人进行任务跟进

\begin{figure}[H]
\centering
\capstart

\noindent\sphinxincludegraphics{{AI_plan}.png}
\caption{项目计划}\label{\detokenize{chapter_project/process:id36}}\end{figure}

在这个阶段需要明确各个环节的交付产物,并识别可能的项目风险,提前制定风险应对计划。

例如:本公司缺乏某方面的\sphinxstylestrong{数据,需要从外部获取,}或者相关人员配置不足,需要招聘或借调人力资源的支持等等。在项目进行的过程中持续监控,以确保项目的正常进行。


\subparagraph{数据准备}
\label{\detokenize{chapter_project/process:id18}}

\subparagraph{数据探索和实验}
\label{\detokenize{chapter_project/process:id19}}
人工智能项目的这一阶段费时费力,但能否完成是未来成功的最重要指标之一。产品需要平衡资源投资和在没有充分了解数据环境的情况下继续发展的风险。获取数据通常很困难,尤其是在受监管的行业。一旦获得了相关数据,理解什么是有价值的,什么是简单的噪音就需要严格的统计和科学。人工智能产品经理可能不会自己做研究;他们的角色是指导数据科学家、分析师和领域专家对数据进行以产品为中心的评估,并为有意义的实验设计提供信息。我们的目标是对存在的数据有一个可衡量的信号,对数据相关性有一个可靠的洞察,并对在哪里集中精力设计特性有一个清晰的愿景。


\subparagraph{数据处理和特征工程}
\label{\detokenize{chapter_project/process:id20}}
数据处理和特征工程是每个AI项目中最困难也是最重要的阶段。人们普遍认为,在一个典型的产品开发周期中,数据科学家80\%的时间都花在特性工程上。自动化和深度学习的趋势和工具确实减少了构建原型(如果不是实际产品的话)所需的时间、技能和努力。尽管如此,构建一个卓越的特性管道或模型架构总是值得的。AI产品经理应该确保项目计划考虑到所需的时间、精力和人员。


\subparagraph{数据理解}
\label{\detokenize{chapter_project/process:id21}}
明确了业务目标之后,我们需要针对数据挖掘的问题收集相关的数据,并对数据进行初步的处理,目标是熟悉数据,探索数据与数据之间的内在联系,并识别数据的质量是否有问题。

用户画像,选择典型的主要客户——例如:最近有过购买记录,并且在一定时间内持续活跃的用户。而不能选择已经流失的,或者是从来没有购买记录的无效客户。

对于参与建模的特征,需要选择那些与业务目标息息相关的数据,以购买商品转化为例:从业务经验来看与之相关的数据有用户的月均消费额度、用户的商品偏好、商品的曝光率、好评率等等。


\subparagraph{构建产品}
\label{\detokenize{chapter_project/process:id22}}

\subparagraph{AI产品交互设计}
\label{\detokenize{chapter_project/process:ai}}
AI产品经理必须从一开始就成为设计团队的一员,以确保产品提供所需的结果。考虑产品的使用方式是很重要的。在最好的人工智能产品中,用户无法分辨底层模型如何影响他们的体验。他们既不知道也不关心应用程序中是否存在人工智能。以Stitch
Fix为例,它使用了多种算法方法来提供定制风格的建议。当Stitch
Fix用户与人工智能产品交互时,他们会与预测和推荐引擎交互。他们在体验过程中与之互动的信息是一种人工智能产品——但他们既不知道,也不关心,他们所看到的一切背后是人工智能。如果算法做出了完美的预测,但用户无法想象佩戴所展示的物品,该产品仍然是一个失败的产品。在现实中,ML模型远非完美,因此更有必要确定用户体验。

要做到这一点,产品经理必须确保设计与工程同等重要。设计师更倾向于用户行为的定性研究。显示用户满意度的信号是什么?如何开发出令用户满意的产品?苹果公司通过iPod、iPhone和iPad产品开创的设计理念,即制造“可以正常工作”的东西,是他们业务的基础。这就是你需要的,你一开始就需要输入。界面设计不是事后添加的东西。


\subparagraph{建模和评估}
\label{\detokenize{chapter_project/process:id23}}
人工智能项目的建模阶段是令人沮丧和难以预测的。这个过程本质上是迭代的,有些AI项目在这一点上失败了(原因很充分)。这一步之所以困难很容易理解:很少有朝着目标稳步前进的感觉。你不断地试验,直到有效果;这可能发生在第一天,或者第100天。当没有有形的“产品”可以展示给每个人的劳动和投资时,AI产品经理必须激励团队成员和利益相关者。一种保持动力的策略是推动短期的突破,以超越表现基线。另一种方法是启动多个线程(甚至可能是多个项目),这样一些线程就能够演示进度。


\subparagraph{原型及MVP}
\label{\detokenize{chapter_project/process:mvp}}
“不要一开始就试图设计和构建完美的系统。相反,快速建立和训练一个基本的系统——也许只需几天。即使基本系统与你能建立的“最佳”系统相去甚远,检查基本系统是如何运作的也是有价值的:你会很快找到线索,向你展示最有前途的方向,在那里投入你的时间。
\sphinxhref{https://radiant-brushlands-42789.herokuapp.com/towardsdatascience.com/take-your-machine-learning-models-to-production-with-these-5-simple-steps-35aa55e3a43c}{9}%
\begin{footnote}[801]\sphinxAtStartFootnote
\sphinxnolinkurl{https://radiant-brushlands-42789.herokuapp.com/towardsdatascience.com/take-your-machine-learning-models-to-production-with-these-5-simple-steps-35aa55e3a43c}
%
\end{footnote}

Entrepreneurial product managers are often associated with the phrase
“Move Fast and Break Things.” AI product mangers live and die by
“Experiment Fast So You Don’t Break Things Later.” Take any social media
company that sells advertisements. The timing, quantity, and type of ads
displayed to segments of a company’s user population are overwhelmingly
determined by algorithms. Customers contract with the social media
company for a certain fixed budget, expecting to achieve certain
audience exposure thresholds that can be measured by relevant business
metrics. The budget that is actually spent successfully is referred to
as fulfillment, and is directly related to the revenue that each
customer generates. Any change to the underlying models or data
ecosystem, such as how certain \sphinxstylestrong{demographic features are weighted},
can have a dramatic impact on the social media company’s revenue.
Experimenting with new models is essential–but so is yanking an
underperforming model out of production. This is only one example of why
rapid prototyping is important for teams building AI products. AI PMs
must create an environment in which continuous experimentation and
failure are allowed (even celebrated), along with supporting the
processes and tools that enable experimentation and learning through
failure.

Qualitative data collection tools (such as SurveyMonkey, Qualtrics, and
Google Forms) should be joined with interface prototyping tools (such as
Invision and Balsamiq), and with data prototyping tools (such as Jupyter
Notebooks) to form an ecosystem for product development and testing.

\begin{figure}[H]
\centering
\capstart

\noindent\sphinxincludegraphics{{AI_MVP}.png}
\caption{AI MVP}\label{\detokenize{chapter_project/process:id37}}\end{figure}


\subparagraph{数据质量和标准化}
\label{\detokenize{chapter_project/process:id24}}
在大多数组织中,数据质量要么是工程问题,要么是IT问题;除非它阻碍了下游流程或项目,否则产品团队很少处理它。这种关系对于开发AI产品的团队来说是不可能的。“垃圾输入,垃圾输出”也适用于人工智能,所以优秀的人工智能pm必须关心数据的健康状况。

原则:
\begin{itemize}
\item {} 
小心“数据清理”方法会破坏数据。如果它改变了底层数据的核心属性,那么它就不是数据清理。

\item {} 
寻找数据中的特性(例如,来自遗留系统的数据会截断文本字段以节省空间)。

\item {} 
在计划和实施数据采集时,要了解糟糕的下游标准化的风险(例如任意词干提取、停止词删除)。

\item {} 
确保数据存储、关键管道和查询都有适当的文档记录,具有结构化的元数据和易于理解的数据流。

\item {} 
考虑时间如何影响您的数据资产,以及季节性影响和其他偏差。

\item {} 
要理解用户体验选择和调查设计可能会引入数据偏差和人为因素。

\end{itemize}


\subparagraph{通过技术领导来加强AI产品管理}
\label{\detokenize{chapter_project/process:id25}}
除了优秀的产品感觉、UI/X体验、客户知识、领导技能等,不太可能每个AI
PM都拥有世界级的技术直觉。但不要因此而产生悲观情绪。因为一个人不可能是所有事情的专家,AI
pm需要与技术领导者(例如,技术领导者或首席科学家)建立合作关系,后者了解技术的现状,熟悉当前的研究,并相信技术领导者受过教育的直觉。

找到这个关键的技术合作伙伴是很困难的,特别是在今天竞争激烈的人才市场。然而,并不是一切都失去了:有许多优秀的技术产品领导者伪装成有能力的工程经理。


\subparagraph{模型评估}
\label{\detokenize{chapter_project/process:id26}}
开始最后的部署阶段之前,最重要的事情是彻底的评估模型,根据在商业理解阶段定义的业务目标来评估我们努力的成果。


\subparagraph{评估结果}
\label{\detokenize{chapter_project/process:id27}}
数据挖掘没有达成业务目标的结果,也不一定意味着建模的失败,有多种可能性导致不成功的结果。

例如:业务目标一开始定得不够合理,与业务目标密切相关的数据未收集到,数据准备出现了错误,训练数据和测试数据不具备代表性等等。这时候我们就要回到之前的步骤,来检视到底是哪个环节出现了问题。


\subparagraph{结果部署}
\label{\detokenize{chapter_project/process:id28}}
建模的作用是从数据中找到知识,获得的知识需要以便于用户使用的方式重新组织和展现,这就是结果部署阶段的工作。根据业务目标的不同,结果部署简单的可能只需要提交一份数据挖掘报告即可,也有可能复杂到需要将模型集成到企业的核心运营系统当中。


\subparagraph{部署前}
\label{\detokenize{chapter_project/process:id29}}
不违反某些指标阈值是非常重要的。这些度量通常被称为护栏指标,它们确保了产品分析不会给决策者错误的信号,让他们知道什么才是对业务真正重要的。

拼车公司:人工智能产品可以轻易地减少来自难以到达地点的用户的请求,从而减少平均拾取时间。然而,这种行为会对公司的整体业务结果产生负面影响,并最终减缓服务的采用。

当一个措施成为目标时,它就不再是一个好的措施(古德哈特定律)。任何衡量标准都会被滥用。“让我们想想如何滥用拾取时间度量”


\subparagraph{制定部署计划}
\label{\detokenize{chapter_project/process:id30}}
根据业务要求和运算性能的的不同,部署的模型可分为:离线模型,近线模型和在线模型三种

离线模型一般适用于重量级的算法,部署于大数据集群仓库,运算的时间需要以小时计,并且时效上通常是T+1的。

近线模型适用于轻量级算法,一般在\sphinxstylestrong{内存和Redis(一种支持Key\sphinxhyphen{}Value等多种数据结构的存储系统,适用于高速消息队列)}中进行,运算的速度以秒为单位。而在线模型则通常根据业务规则来设置,在内存中运行,运行的速度以毫秒计。

另外,部署还需要考虑不同编程语言对于算法模型的调取兼容性,在这个阶段算法工程师需要与原有业务的开发工程师进行联调协作,确保业务系统能够正确的调用算法模型结果。


\subparagraph{部署}
\label{\detokenize{chapter_project/process:id31}}
与传统的软件工程项目不同,AI产品经理必须大量参与构建过程。工程经理通常负责确保软件产品的所有组件都被正确地编译成二进制文件,并按照版本仔细地组织构建脚本,以确保可再现性。许多成熟的DevOps过程和工具,经过多年成功的软件产品发布的磨练,使这些过程更加易于管理,但它们是为传统的软件产品开发的。在ML/AI生态系统中根本不存在类似的工具和过程;即使这样,它们也很少成熟到可以大规模使用。因此,人工智能项目经理必须采取高度接触的、定制的方法来指导人工智能产品的生产、部署和发布。


\subparagraph{模型监督和维护}
\label{\detokenize{chapter_project/process:id32}}
对于AI产品,模型性能和应用程序性能必须同时监控。当AI产品执行超出规范时触发的警报可能需要以不同的方式传递;如果没有AI团队的支持,现场SRE团队可能无法诊断模型或数据管道的技术问题。

我们知道算法模型是基于历史数据得来的,但是在模型部署并运行一段时间之后,可能业务场景的改变,使得核心数据已经发生了变化,原有的模型性能下降,可能已经无法满足当前的业务需要。
\sphinxhref{https://radiant-brushlands-42789.herokuapp.com/towardsdatascience.com/my-first-year-as-a-project-manager-for-artificial-intelligence-ai-16127a4a37c2}{7}%
\begin{footnote}[802]\sphinxAtStartFootnote
\sphinxnolinkurl{https://radiant-brushlands-42789.herokuapp.com/towardsdatascience.com/my-first-year-as-a-project-manager-for-artificial-intelligence-ai-16127a4a37c2}
%
\end{footnote}

例如,一款帮助服装制造商了解该购买哪种材料的人工智能产品将随着时尚的变化而过时。如果AI产品成功了,它甚至可能导致这些变化。您必须检测模型何时过时,并根据需要对其进行重新培训。

这就需要我们在模型部署上线的同时,同步上线模型的监督和维护系统,持续跟踪模型的运行状况。

该框架允许部署的模型不间断地运行,同时根据总体样本培训第二个模型。如果第二种模型的性能优于原来的,它可以简单地被替换掉——通常没有任何停机时间!

当发现模型结果在出现短期异常值时,排查异常的原因,例如:运营活动或者节假日等因素,当发现模型长期表现不佳时,则要考虑是否是用户和产品的数据构成已经发生了变化。如果是因为数据构成已经发生变化,则需要重新通过CRISP\sphinxhyphen{}DM流程构建新的模型。

优雅地处理产品故障:为用户提供一种立即重新标记数据以进一步改进模型的方法。举例来说,iPhone
的语音邮件转录服务对其低信度是透明的,并为用户提供了一个选项来帮助苹果通过提交语音记录来改进转录。\sphinxhref{http://www.uml.org.cn/devprocess/201910163.asp}{8}%
\begin{footnote}[803]\sphinxAtStartFootnote
\sphinxnolinkurl{http://www.uml.org.cn/devprocess/201910163.asp}
%
\end{footnote}


\subparagraph{持续集成和持续部署}
\label{\detokenize{chapter_project/process:id33}}\begin{itemize}
\item {} 
\sphinxurl{https://martinfowler.com/articles/cd4ml.html}

\item {} 
\sphinxurl{https://www.oreilly.com/library/view/continuous-delivery-reliable/9780321670250/}

\end{itemize}


\subparagraph{生成最终报告}
\label{\detokenize{chapter_project/process:id34}}
最后,不要忘了向项目的相关干系人发送一份最终的项目总结报告。


\subsubsection{数据处理流程拆解}
\label{\detokenize{chapter_project/Data Process:id1}}\label{\detokenize{chapter_project/Data Process::doc}}
AI开发项目与数据相关的流程大概有以下

\begin{figure}[H]
\centering
\capstart

\noindent\sphinxincludegraphics{{data_process}.png}
\caption{与数据相关的流程}\label{\detokenize{chapter_project/Data Process:id8}}\end{figure}

注意:一般在数据未产生且亦可通过程序自动标注的情况:数据采集和数据标注合并需求方案会在一起定义输出文档给研发同事。\sphinxhref{https://shimo.im/docs/jxCw6W6XrGqwkqwd/read}{3}%
\begin{footnote}[804]\sphinxAtStartFootnote
\sphinxnolinkurl{https://shimo.im/docs/jxCw6W6XrGqwkqwd/read}
%
\end{footnote}

一般的项目数据相关的工作内容包含:从数据的构建到数据标注到后期维护管理等。如下图,罗列关于产品经理需要了解的数据相关认知,但AI产品经理入门,侧重点在数据采集、数据标注、数据维护。(下图建议在网页版打开)

\begin{figure}[H]
\centering
\capstart

\noindent\sphinxincludegraphics{{data_process_detail}.png}
\caption{侧重数据采集、数据标注}\label{\detokenize{chapter_project/Data Process:id9}}\end{figure}


\paragraph{数据工作 2\sphinxfootnotemark[805]}
\label{\detokenize{chapter_project/Data Process:id2}}%
\begin{footnotetext}[805]\sphinxAtStartFootnote
\sphinxnolinkurl{https://weread.qq.com/web/reader/40632860719ad5bb4060856ka1d32a6022aa1d0c6e83eb4}
%
\end{footnotetext}\ignorespaces 
数据准备可以分为3个阶段,数据来源—数据定义—数据交付,在这3个阶段中,需要AI产品经理具备规划、收集、整理数据的能力。在数据准备的过程中,我们可以根据不同阶段考虑以下问题。


\subparagraph{数据来源}
\label{\detokenize{chapter_project/Data Process:id3}}\begin{itemize}
\item {} 
数据源。数据源在哪?这些数据是否存在不同的地方,以及如何进行关联?

\item {} 
数据规模。数据是否能够或足够进行建模,有多少数据来描述这个场景?

\item {} 
数据更新。数据是怎么更新的?这些数据的维度是什么?

\end{itemize}


\subparagraph{数据定义}
\label{\detokenize{chapter_project/Data Process:id4}}\begin{itemize}
\item {} 
数据清洗。用什么样的方法清洗和整理数据?

\item {} 
输入和输出。设置什么样的“输入”和“输出”,能够保证测试集训练出来的机器能更好地运用在实际场景中?

\item {} 
数据标注。如何更迅速高效地标注数据?不同的输入方式之间有什么层级关系?我们应该用什么形式来展现这些层级关系?

\end{itemize}


\subparagraph{数据交付}
\label{\detokenize{chapter_project/Data Process:id5}}\label{\detokenize{chapter_project/Data Process:id6}}
在产品的实际交互中,应记录哪些数据? 用什么样的形式提供数据?
如何使用数据,通过接口或是批量推送?

获取正确数据的过程并非我们想象中那么简单,因为不同团队对于数据的维护程度是不一样的,在获取正确数据的过程中,沟通成本是非常高的,在AI产品构建过程中,除了业务场景的调研,数据的准备环节也是很重要的,因为数据是AI产品设计的基础,所有的工作自始至终都是围绕数据展开的。


\paragraph{数据处理流程拆解}
\label{\detokenize{chapter_project/Data Process:id7}}
\begin{figure}[H]
\centering
\capstart

\noindent\sphinxincludegraphics{{data_process}.png}
\caption{数据处理\sphinxhref{http://sjrzld.com/a/AI0273.html}{4}\sphinxfootnotemark[806]}\label{\detokenize{chapter_project/Data Process:id10}}\end{figure}
%
\begin{footnotetext}[806]\sphinxAtStartFootnote
\sphinxnolinkurl{http://sjrzld.com/a/AI0273.html}
%
\end{footnotetext}\ignorespaces 
一、数据标注

数据的质量直接会影响到模型的质量,因此数据标注在整个流程中绝对是非要重要的一点。

1、一般来说,数据标注部分可以有三个角色

标注员:标注员负责标记数据。

审核员:审核员负责审核被标记数据的质量。

管理员:管理人员、发放任务、统计工资。

只有在数据被审核员审核通过后,这批数据才能够被算法同事利用。

2、数据标记流程

任务分配:假设标注员每次标记的数据为一次任务,则每次任务可由管理员分批发放记录,也可将整个流程做成“抢单式”的,由后台直接分发。

标记程序设计:需要考虑到如何提升效率,比如快捷键的设置、边标记及边存等等功能都有利于提高标记效率。

进度跟踪:程序对标注员、审核员的工作分别进行跟踪,可利用“规定截止日期”的方式淘汰怠惰的人。

质量跟踪:通过计算标注人员的标注正确率和被审核通过率,对人员标注质量进行跟踪,可利用“末位淘汰”制提高标注人员质量。

而且,模型的效果,需要在这两个指标之间达到一个平衡。

测试同事需要关注特定领域内每个类别的指标,比如针对识别人脸的表情,里面有喜怒哀乐等分类,每一个分类对应的指标都是不一样的。测试同事需要将测试的结果完善地反馈给算法同事,算法同事才能找准模型效果欠缺的原因。同时,测试同事将本次模型的指标结果反馈给产品,由产品评估是否满足上线需求。

2、数据标记流程

任务分配:假设标注员每次标记的数据为一次任务,则每次任务可由管理员分批发放记录,也可将整个流程做成“抢单式”的,由后台直接分发。

标记程序设计:需要考虑到如何提升效率,比如快捷键的设置、边标记及边存等等功能都有利于提高标记效率。

进度跟踪:程序对标注员、审核员的工作分别进行跟踪,可利用“规定截止日期”的方式淘汰怠惰的人。

质量跟踪:通过计算标注人员的标注正确率和被审核通过率,对人员标注质量进行跟踪,可利用“末位淘汰”制提高标注人员质量。

对于众包平台来讲,国外首选亚马逊众包平台,ImageNet就是通过这个平台进行标注的。而国内也有百度众包、京东众智、龙猫数据等众包平台可供选择。

二、模型训练

这部分基本交由算法同事跟进,但产品可依据需求,向算法同事提出需要注意的方面;

举个栗子:

背景:一个识别车辆的产品对大众车某系列的识别效果非常不理想,经过跟踪发现,是因为该车系和另外一个品牌的车型十分相似。那么,为了达到某个目标(比如,将精确率提高5\%),可以采用的方式包括:

补充数据:针对大众车系的数据做补充。值得注意的是,不仅是补充正例(“XXX”应该被识别为该大众车系),还可以提供负例(“XXX”不应该被识别为该大众车系),这样可以提高差异度的识别。

优化数据:修改大批以往的错误标注。

产品将具体的需求给到算法工程师,能避免无目的性、无针对性、无紧急程度的工作。

三、模型测试

测试同事(一般来说算法同事也会直接负责模型测试)将未被训练过的数据在新的模型下做测试。

如果没有后台设计,测试结果只能由人工抽样计算,抽样计算繁琐且效率较低。因此可以考虑由后台计算。

一般来说模型测试至少需要关注两个指标:

精确率:识别为正确的样本数/识别出来的样本数

召回率:识别为正确的样本数/所有样本中正确的数

举个栗子:全班一共30名男生、20名女生。需要机器识别出男生的数量。本次机器一共识别出20名目标对象,其中18名为男性,2名为女性。则

精确率=18/(18+2)=0.9

召回率=18/30=0.6

再补充一个图来解释:

四、产品评估

“评估模型是否满足上线需求”是产品必须关注的,一旦上线会影响到客户的使用感。

因此,在模型上线之前,产品需反复验证模型效果。为了用数据对比本模型和上一个模型的优劣,需要每次都记录好指标数据。

假设本次模型主要是为了优化领域内其中一类的指标,在关注目的的同时,产品还需同时注意检测其他类别的效果,以免漏洞产生。


\subsubsection{模型过程}
\label{\detokenize{chapter_project/Model Process:id1}}\label{\detokenize{chapter_project/Model Process::doc}}

\paragraph{整体框架}
\label{\detokenize{chapter_project/Model Process:id2}}
\begin{figure}[H]
\centering
\capstart

\noindent\sphinxincludegraphics{{ML_structure}.png}
\caption{整体框架}\label{\detokenize{chapter_project/Model Process:id11}}\end{figure}


\paragraph{生命周期 1\sphinxfootnotemark[807]}
\label{\detokenize{chapter_project/Model Process:id3}}%
\begin{footnotetext}[807]\sphinxAtStartFootnote
\sphinxnolinkurl{https://ucbrise.github.io/cs294-ai-sys-fa19/assets/lectures/lec03/03\_ml-lifecycle.pdf}
%
\end{footnotetext}\ignorespaces 
\begin{figure}[H]
\centering
\capstart

\noindent\sphinxincludegraphics{{ML_Lifecycle}.png}
\caption{模型的生命周期}\label{\detokenize{chapter_project/Model Process:id12}}\end{figure}


\paragraph{模型开发}
\label{\detokenize{chapter_project/Model Process:id4}}
数据采集\sphinxhyphen{}》数据清理\&可视化

需要数据工程师来帮助:
\begin{itemize}
\item {} 
识别数据的潜在来源

\item {} 
从多个来源连接数据

\item {} 
定位缺失值和异常值

\item {} 
绘制趋势以识别异常

\end{itemize}

—》特征工程\&模型设计\sphinxhyphen{}》训练\&评估

数据科学家为此而生:
\begin{itemize}
\item {} 
构建信息丰富的特性

\item {} 
功能设计新的模型架构

\item {} 
重新调整超参数

\item {} 
验证预测准确性

\end{itemize}


\subparagraph{特征和特征工程}
\label{\detokenize{chapter_project/Model Process:id5}}
特征:输入的属性或特征

为什么从模型开发中直接生成训练有素的模型是一个坏主意

为了解决新问题,正在构建模型
\begin{itemize}
\item {} 
需要跟踪组合和验证端到端的准确性。

\item {} 
需要对模型进行单元和集成测试

\end{itemize}

TODO:

Dynamic Features: features can often be modified faster than models:
\begin{itemize}
\item {} 
Useful for addressing fast changing dynamics (e.g., user preferences
can be encoded in click history features).

\item {} 
Issue:resulting potential covariate shift can be problematic

\end{itemize}


\subparagraph{超参数}
\label{\detokenize{chapter_project/Model Process:id6}}
参数和更普遍的配置细节不是通过培训直接确定的
\begin{itemize}
\item {} 
手动设置或使用交叉验证进行调整

\item {} 
为什么不直接学习?

\end{itemize}

\begin{figure}[H]
\centering
\capstart

\noindent\sphinxincludegraphics{{find_hyperparameter}.png}
\caption{寻找超参}\label{\detokenize{chapter_project/Model Process:id13}}\end{figure}

Training Pipelines \sphinxhyphen{}> Code Trained Models \sphinxhyphen{}> Binaries


\subparagraph{训练科技}
\label{\detokenize{chapter_project/Model Process:id7}}
\begin{figure}[H]
\centering
\capstart

\noindent\sphinxincludegraphics{{train_tech}.png}
\caption{训练科技}\label{\detokenize{chapter_project/Model Process:id14}}\end{figure}


\paragraph{推断}
\label{\detokenize{chapter_project/Model Process:id8}}
目的:在深度神经复杂的突发运行下在\textasciitilde{}10ms内进行预测


\subparagraph{并包括反馈}
\label{\detokenize{chapter_project/Model Process:id9}}\begin{enumerate}
\sphinxsetlistlabels{\arabic}{enumi}{enumii}{}{.}%
\item {} 
模型更新:当新数据来重新培训。

\end{enumerate}
\begin{itemize}
\item {} 
\sphinxstylestrong{周期性}:权衡利用批处理和验证。因为模型可能会过期一段时间。

\item {} 
\sphinxstylestrong{持续地(在线学习)}:最新鲜的模型。需要验证,学习率?…非常复杂。

\end{itemize}
\begin{enumerate}
\sphinxsetlistlabels{\arabic}{enumi}{enumii}{}{.}%
\setcounter{enumi}{1}
\item {} 
特征更新:新数据可能会改变特征

\end{enumerate}
\begin{itemize}
\item {} 
例如:更新用户的点击历史\sphinxhyphen{}》新预测

\item {} 
比在线学习更健壮

\end{itemize}

新的训练数据到达并改变损失面。而不是从以前的解的随机权重开始重新开始


\subparagraph{反馈圈}
\label{\detokenize{chapter_project/Model Process:id10}}
Models can bias the data they collect
\begin{itemize}
\item {} 
Example: content recommendation

\item {} 
Future models may reflect earlier model bias

\end{itemize}

Exploration –Exploitation Trade\sphinxhyphen{}off
\begin{itemize}
\item {} 
Exploration:observe diverse outcomes

\item {} 
Exploitation:leverage model to takepredicted best action

\end{itemize}

Solutions:
\begin{itemize}
\item {} 
Randomization (ε\sphinxhyphen{}greedy): occasionally ignore the model

\item {} 
Bandit Algorithms/Thompson Sampling: optimally balance exploration
and exploitation àactive area of research

\end{itemize}


\subsubsection{原型检验1\sphinxfootnotemark[808]}
\label{\detokenize{chapter_project/inspect:id1}}\label{\detokenize{chapter_project/inspect::doc}}%
\begin{footnotetext}[808]\sphinxAtStartFootnote
\sphinxnolinkurl{http://www.woshipm.com/pmd/21446.html}
%
\end{footnotetext}\ignorespaces 
容易犯一个常见的错误,他们对产品设计规范太有信心,结果一旦得到beta的测试他们就必须调整产品。但是肯定beta测试版并不是进行重大改变的时候,所以才会有许多首次发布的产品离目标太远。


\paragraph{可行性测试}
\label{\detokenize{chapter_project/inspect:id2}}
一个直接的问题就是产品是否可以开发,你的工程师和设计师应当介入技术的可行性调查和探索可用办法。有些办法是行不通的,但是有其他的办法可行是非常有希望的。
工程师会发现在产品的某个阶段不可能逾越,现在知道比以后知道要好。


\paragraph{可用性测试}
\label{\detokenize{chapter_project/inspect:id3}}
产品设计师将要和你紧密工作共同提出产品功能,让它能适应不同的用户。可用性测试常常会找出遗漏的产品要求,同时确认产品最初的要求是否是必须的。在你拿出一个成功的用户体验之前需要多做一些测试工作。可用性的目的是在真正的用户身上测试,从产品目标用户得到质量反馈的测试是非常艺术和科学的。当然产品经理和产品设计将模仿使用,但是实际是没有人能取代真实的目标用户。

观看其他人试着使用您正在(或已经)创建/设计/建造的产品,目的是
\sphinxhref{https://36kr.com/p/1721288867841}{2}%
\begin{footnote}[809]\sphinxAtStartFootnote
\sphinxnolinkurl{https://36kr.com/p/1721288867841}
%
\end{footnote}
\begin{enumerate}
\sphinxsetlistlabels{\arabic}{enumi}{enumii}{}{.}%
\item {} 
让它使用起来更容易;

\item {} 
证明它容易使用。

\end{enumerate}


\paragraph{概念测试(Product Concept Testing)}
\label{\detokenize{chapter_project/inspect:product-concept-testing}}
光是可用和可行是不足的。真正的问题是你的用户想要购买吗—你的用户有多喜欢\sphinxhyphen{}你做的有什么价值。这测试可能与可用性测试联系在一起。

对于一部份小产品,您的想法写在纸就足够了,但是对于多数产品,为了预计产品是否达到目标,复杂用户互作用或新技术的使用、某种形式原型都是非常重要的。

原型也许是一个物理设备,或者它也许是软件产品的一个预览版本。关键是它需要足够现实,您能用原型在实际目标顾客身上测试,并且他们可以给您质量反馈。

以前做原型主要有两个障碍。第一是缺乏良好的原型工具,需要花费很多的时间制作原型;另一个是管理方不知道原型和真实产品的区别,在不可预计的情况下,按照最终产品来要求原型。


\subsubsection{产品文档}
\label{\detokenize{chapter_project/product_document:id1}}\label{\detokenize{chapter_project/product_document::doc}}

\paragraph{问题}
\label{\detokenize{chapter_project/product_document:id2}}\begin{itemize}
\item {} 
需求描述不清楚:前端新增页面是新开页面,还是当前页面打开。

\item {} 
逻辑不通:产生需求变更,不仅引起众怒,研发人员也质疑产品经理的能力。

\item {} 
需求遗漏:功能的排序规则不对,然后让研发修改。

\end{itemize}


\paragraph{目标}
\label{\detokenize{chapter_project/product_document:id3}}\begin{enumerate}
\sphinxsetlistlabels{\arabic}{enumi}{enumii}{}{.}%
\item {} 
灵活运用不同文档格式描述功能、业务等,而不拘泥于框架,将你的想法最大程度用文档呈现;

\item {} 
使你的文档逻辑有条理、框架更清晰、重点更突出,减少与开发人员的沟通成本;

\item {} 
利用不同的规范化文档,在接收需求、管理需求、验收需求的过程中提高工作效率。

\end{enumerate}


\paragraph{产品版本V、R、M介绍}
\label{\detokenize{chapter_project/product_document:vrm}}

\subparagraph{产品大版本(Version)}
\label{\detokenize{chapter_project/product_document:version}}
代表公司某一产品或其系列产品,并与唯一的产品配套表对应。例如V1等。
根据市场定位或开发平台的不同,一个产品分为若干个V 级版本。每个V
级版本根据市场竞争需要、技术与成本因素等,有一个总体开发计划,按计划向市场发布若干个子版本(Release),因此V
级版本包含若干个Release版本。


\subparagraph{特性版本(Release)}
\label{\detokenize{chapter_project/product_document:release}}
每个Release
版本包含若干个特性,可以形成一个具体的系列产品,因此,系列产品与特性版本等同。什么特性纳入一个Release
版本,需要综合考虑市场竞争、技术与成本方面的因素,系列产品也可有自己的特性版本,系列产品可以在特性版本号上用特别的字母或数字表示。


\subparagraph{更改版本(Modification)}
\label{\detokenize{chapter_project/product_document:modification}}
如果需要在已经发布的Release
版本基础上更改(新增、改变、删除)某个特性,或者对特性下的功能、性能需求进行更改(新增、改变、删除),导致产品规格或总体方案变动,就会产生更改版本(Modification),更改版本是对发布后的Release版本的更新。


\subsubsection{Roadmap1\sphinxfootnotemark[810]}
\label{\detokenize{chapter_project/Roadmap:roadmap1}}\label{\detokenize{chapter_project/Roadmap::doc}}%
\begin{footnotetext}[810]\sphinxAtStartFootnote
\sphinxnolinkurl{http://www.woshipm.com/pd/3591815.html}
%
\end{footnotetext}\ignorespaces 
产品路线图是传达产品战略方向的生动、可视化文档。这些高层次的总结将所有涉及产品管理和产品开发的活动部分结合在一起。他们传达了两个愿望(通过产品愿景)和计划(通过倡议),同时也将产品战略与公司战略联系起来。有效地完成后,产品路线图将传达您正在构建的内容背后的“原因”。


\paragraph{目标}
\label{\detokenize{chapter_project/Roadmap:id1}}

\subparagraph{实现以下几点:}
\label{\detokenize{chapter_project/Roadmap:id2}}\begin{itemize}
\item {} 
描述产品愿景和战略

\item {} 
提供执行战略的指导文件

\item {} 
使内部利益相关者保持一致

\item {} 
促进对选项和方案规划的讨论

\item {} 
沟通产品开发计划的进展和状态

\item {} 
帮助向外部利益相关者(包括客户)传达您的战略

\end{itemize}


\subparagraph{还可帮助:}
\label{\detokenize{chapter_project/Roadmap:id3}}\begin{itemize}
\item {} 
IT团队可以使用路线图来规划和跟踪架构项目、软件实现和各种其他项目。

\item {} 
营销团队可以使用路线图来规划从营销策略到内容日历的所有内容。

\item {} 
业务和业务运营团队可以使用路线图来规划扩展和其他组织范围的目标。

\item {} 
DevOps可能会使用路线图来更好地连接他们负责整合的组织的各个部分。

\item {} 
工程团队可以使用路线图来计划冲刺。

\end{itemize}


\paragraph{长什么样}
\label{\detokenize{chapter_project/Roadmap:id4}}\begin{enumerate}
\sphinxsetlistlabels{\arabic}{enumi}{enumii}{}{.}%
\item {} 
产品愿景
你的产品愿景,通常被称为产品愿景宣言或使命宣言,是一个简洁的、高层次的、有抱负的陈述。

\end{enumerate}

为什么你的产品存在,如果你的产品成功了,世界将会变成怎样?

它是一盏指路明灯——在当下遥不可及,但却始终代表着预定的方向。

在Roadmap上阐述产品愿景,一方面有助于我们的产品始终围绕着这个核心价值观而不偏离轨道,另一方面,愿景是我们激励其他部门的重要力量。
\begin{enumerate}
\sphinxsetlistlabels{\arabic}{enumi}{enumii}{}{.}%
\setcounter{enumi}{1}
\item {} 
说明
通常需要说明roadmap上时间段的起点和终点分别代表什么意思。时间段不同颜色代表什么意思,比如可以是不同颜色代表不同的技术平台,也可以是代表不同的市场。

\item {} 
关键细分
这个关键细分取决于你所在的行业以及你的受众,通常来讲,如果是客户的话,可能是他们在购买你产品的时候主要考虑的几个因素之一。比如手机的话,可以是“米系列”,“红米系列”,汽车的话,可以是“轿车”,“SUV”和“MPV”等。关键细分可以是一层,也可以是多层,但最好不要超过三层。

\item {} 
产品信息
通常分为当前在售和未来推出。这里可以在产品名称下面加一些关键特征参数,主打的卖点或关键购买因素等信息。如下图里面12M,1.55μm是CMOS的一些关键特征参数。汽车的话诸如“2.0T,涡轮增压,190马力,国VI,双擎”。这样可以方便受众快速的找到自己关心的产品的进展。

\end{enumerate}


\paragraph{总结}
\label{\detokenize{chapter_project/Roadmap:id5}}
有利且危险:可以帮助产品经理更好地可视化和沟通产品战略,形成健康的权衡。如果他们太过细节和死板,则会造成很多工作浪费,给客户和管理层产生一些无法实现的期望。


\subsubsection{Scrum}
\label{\detokenize{chapter_project/Scrum:scrum}}\label{\detokenize{chapter_project/Scrum::doc}}

\paragraph{瀑布式(Waterfall)开发}
\label{\detokenize{chapter_project/Scrum:waterfall}}
瀑布式(Waterfall)开发:概念设计>>设计>>编程>>测试修正
\sphinxhref{https://ones-ai.gitbooks.io/ones-ai}{1}%
\begin{footnote}[811]\sphinxAtStartFootnote
\sphinxnolinkurl{https://ones-ai.gitbooks.io/ones-ai}
%
\end{footnote}


\paragraph{问题 14\sphinxfootnotemark[812]}
\label{\detokenize{chapter_project/Scrum:id1}}%
\begin{footnotetext}[812]\sphinxAtStartFootnote
\sphinxnolinkurl{https://blog.csdn.net/weixin\_45036344/article/details/102837495}
%
\end{footnotetext}\ignorespaces \begin{itemize}
\item {} 
不清楚用户在哪里:这是利用传统产品开发方法创业失败最核心的原因,缺少用户和有效的商业模式;

\item {} 
过分强调产品的发布时间:这个时间只是意味着开发结束的时间,而并不意味着充分掌握用户信息,也并不意味着团队知道如何开展市场营销活动,还没有知道用户以及市场在哪里就盲目地推出产品,这样做完全是本末倒置;

\item {} 
过分强调执行,忽略探索和学习:由于创业公司总是强调要快速完成任务,所以团队只要按部就班地运用已有的营销知识和销售技巧,保持以往的工作即可,无须学习新知识,而创业过程面临无数的新问题就是需要不断地学习和探索才能找到答案;

\item {} 
市场营销活动和销售工作缺乏明确目标:产品开发的优点是具有明确定义的目标,各个阶段界限分明,但是营销和销售的情况却往往不一样,是混乱的,围绕着各种可量化的任务进行执行,而忽略了其真正要探寻的问题,即:理解用户需求,发现用户购买产品的规律、利用合理的商业模式获取利润;

\item {} 
用产品开发的方法指导销售:流程是(愿景→产品开发→大量招聘销售人员→组建销售部门),这个策略的假设在于,一旦开发完成,产品就会有生意,但实际上多数情况下会事与愿违;

\item {} 
用产品开发的方法指导市场营销:流程是(愿景→产品开发→市场公关早期营销→品牌推广创造需求),所有的这些都发生在用户的消费行为之前,产品定位、营销策略、需求创造都没有经过实际的市场检验,即产品发布之后才能知道营销策略的有效性,团队无异于失控的火车一样撞向终点,没有任何机制可以让它暂停接受检修;

\item {} 
仓促扩张:通常,公司高层决定扩大经营规模的依据有三个,即产品开发方法、商业计划、预期收益,但这些依据的前提都是基于产品一旦会成功。还是缺少“停车检修”的机制;

\item {} 
恶性循环:仓促扩张会直接诱发公司陷入恶性循环,同时加快恶化速度,这些扩张的支出加速消耗公司的现金流,但致命的是团队对市场和用户还是知之甚少,处于业绩的压力下不断更换策略,但仍然没有停下来反思,反而加快烧钱的速度希望事情有所转机;

\item {} 
忽视市场类型的影响:常见的市场类型有现有市场(开发市场上已有的产品)、全新市场(开发全新产品,开拓全新市场)、细分市场(开发改良产品,进一步细分现有市场),面对不同类型的市场,用户对产品的接受程度和接受时间完全不同(跨越鸿沟),所以营销策略以及销售策略大相径庭,因而维持公司经营所需的现金流也不同;

\item {} 
好高骛远:创业公司如果采用传统的产品开发方法(瀑式),一两年就会出现致命的问题,因为他们抱有不切实际的期望,即用产品开发方法指导与产品开发无关的经营活动(如寻找市场、发掘用户、制定商业模式)、用户数量会随着产品开发的进度自动增长、只要产品发布上市就会被用户接受。除了三条外,投资者以及公司高层的压力也会起到推波助澜的作用,同时创业者往往会夸大市场份额和业务前景并故意忽略市场类型。

\end{itemize}

迭代周期可能短到一周,但分析、设计、编码、集成和测试等开发阶段一应俱全。而在采用瀑布开发的项目中,单个阶段也许就得花费数年时间。
与瀑布模式相比,敏捷放弃了从开始到结束控制项目形态的总体计划和规范,而选择了许多中途修正。


\paragraph{动因}
\label{\detokenize{chapter_project/Scrum:id2}}
来自于市场的压力(逼迫我们用更少的成本来制作更符合真实客户需求的软件)

用Scrum的开发模式小步快跑,可以以速度最快、最经济的方式逐步加入满足消费者的功能,明确用户价值。

瀑布式开发的致命症结就在于它的预先设计(BDUFs,Big Design
Up\sphinxhyphen{}Front)无法完全清晰,并且变更成本巨大


\paragraph{敏捷软体开发宣言}
\label{\detokenize{chapter_project/Scrum:id3}}\begin{itemize}
\item {} 
个人与互动 重于流程与工具

\item {} 
可用的软体 重于详尽的文件

\item {} 
与客户合作 重于合约协商

\item {} 
回应变化 重于遵循计划

\end{itemize}

也就是说,虽然右侧项目有其价值,但我们更重视左侧项目。
\sphinxhref{https://agilemanifesto.org/iso/zhcht/manifesto.html}{10}%
\begin{footnote}[813]\sphinxAtStartFootnote
\sphinxnolinkurl{https://agilemanifesto.org/iso/zhcht/manifesto.html}
%
\end{footnote}


\paragraph{Scrum流程}
\label{\detokenize{chapter_project/Scrum:id4}}
在发布产品获得用户反馈后才正式开始——由真实用户反馈影响需求规划、产品迭代的“小步快跑”方式,才是Scrum的精髓所在


\subparagraph{发布产品}
\label{\detokenize{chapter_project/Scrum:id5}}
营销推广模型,源自《用图秀演讲》:

背景介绍、描述阻碍、点燃希望、震撼登场、展现价值、精雕细琢、给出诱惑。
\sphinxhref{https://www.yinxiang.com/everhub/note/f9ab87ee-73e6-4241-9428-9507cbfd007f}{7}%
\begin{footnote}[814]\sphinxAtStartFootnote
\sphinxnolinkurl{https://www.yinxiang.com/everhub/note/f9ab87ee-73e6-4241-9428-9507cbfd007f}
%
\end{footnote}


\subparagraph{产品Backlog}
\label{\detokenize{chapter_project/Scrum:backlog}}
最顶端的核心需求拆分成同样颗粒度大小的需求——这样才能准确估计工作量——然后把需求排入版本中。


\subparagraph{User Story}
\label{\detokenize{chapter_project/Scrum:user-story}}
随着讨论和了解的细节增多,能够浮现出更多关于产品设计的细节和场景,也便越能进一步准确估算每个需求的工作量——不确定的特性是无法精确估算难度的。


\subparagraph{Sprint}
\label{\detokenize{chapter_project/Scrum:sprint}}
固定的时间限,大约2\textasciitilde{}4周,每个版本产出功能性的垂直切片,他们各自就像一个个小型项目


\subparagraph{Release}
\label{\detokenize{chapter_project/Scrum:release}}
进入到准上市状态,这称作一次发布。一个典型的发布需要经历2\textasciitilde{}4个月,节奏类似于一个经典项目中的里程碑。


\subparagraph{监控产品}
\label{\detokenize{chapter_project/Scrum:id6}}
产品回顾会 《敏捷回顾:团队从优秀到卓越之道》

罗伯特议事原则

六顶思考帽:
\sphinxhref{https://www.yinxiang.com/everhub/note/f9ab87ee-73e6-4241-9428-9507cbfd007f}{7}%
\begin{footnote}[815]\sphinxAtStartFootnote
\sphinxnolinkurl{https://www.yinxiang.com/everhub/note/f9ab87ee-73e6-4241-9428-9507cbfd007f}
%
\end{footnote}
\begin{itemize}
\item {} 
白色:客观而中立、数据:ref:\sphinxcode{\sphinxupquote{data\_analysis}}和事实

\item {} 
绿色:创造力和想象力

\item {} 
黄色:价值和肯定、乐观、建设性

\item {} 
黑色:否定、怀疑、批判

\item {} 
红色:表达情绪、直觉、预感

\item {} 
蓝色:控制和调节思维过程,用来控制其他思考帽的使用顺序,规划和负责整个思考过程,给出结论。

\end{itemize}


\paragraph{步骤}
\label{\detokenize{chapter_project/Scrum:id7}}\begin{enumerate}
\sphinxsetlistlabels{\arabic}{enumi}{enumii}{}{.}%
\item {} 
通过创建需求池项目,进行对产品Backlog(需求列表)的管理。

\item {} 
建立发布计划和Sprint迭代计划,清晰地区分阶段性目标。

\item {} 
把需求排入相应的版本中,进行对发布计划和迭代的任务管理。

\end{enumerate}


\subparagraph{好的Product Backlog}
\label{\detokenize{chapter_project/Scrum:product-backlog}}
好的Product Backlog应该具备以下特性DEEP:
\sphinxhref{https://medium.com/@b98606021/\%E6\%B7\%B1\%E5\%85\%A5\%E6\%B7\%BA\%E5\%87\%BA-\%E6\%95\%8F\%E6\%8D\%B7\%E8\%BB\%9F\%E9\%AB\%94\%E9\%96\%8B\%E7\%99\%BC-scrum-4a9d357ac0a4}{11}%
\begin{footnote}[816]\sphinxAtStartFootnote
\sphinxnolinkurl{https://medium.com/@b98606021/\%E6\%B7\%B1\%E5\%85\%A5\%E6\%B7\%BA\%E5\%87\%BA-\%E6\%95\%8F\%E6\%8D\%B7\%E8\%BB\%9F\%E9\%AB\%94\%E9\%96\%8B\%E7\%99\%BC-scrum-4a9d357ac0a4}
%
\end{footnote}
\begin{enumerate}
\sphinxsetlistlabels{\arabic}{enumi}{enumii}{}{.}%
\item {} 
适当详细:Detailed appropriately,意思就是在Backlog 的Item
并不是每个内容都有仔细地描述,在每次Sprint Meeting
之前,笔者建议采用八二法则,只要共20\%
的有详细描述即可(意即到开发团队可以拆解出Task地步),其他80\%
只要有大概的叙述即可(因为未来的变动),通常这样的数量对于开发团队已经足够。

\item {} 
估算过的:Estimated,代表撰写的Item 有询问过相关人等去估算该Item
大约要花多少的时间与成本完成,同时也估算重要性质,以利优先顺序的排列。

\item {} 
持续更新:Emergent
(虽然中文跟英文不太一样,但认为翻译持续更新比较能表达Emergent的意思),要一直持续的更新Item内容,去因应每周甚至每日所接受到最新的需求,这样交付下去的Item才会跟最终客户期待的产品认知不会落差太大。

\item {} 
优先顺序:Prioritized,产品负责人必须要去考虑如何安排Item
的优先顺序,必须要从不同的方向考量,包含商业价值、成本、和其他利害关系人的因素。

\end{enumerate}


\paragraph{XP核心实践}
\label{\detokenize{chapter_project/Scrum:xp}}\begin{enumerate}
\sphinxsetlistlabels{\arabic}{enumi}{enumii}{}{.}%
\item {} 
团队协作:通过客户、开发团队、项目经理三方共同参加的会议来确定开发计划。

\item {} 
规划策略:
计划是持续的、循序渐进的。每2周,开发人员就为下2周估算候选特性的成本,而客户则根据成本和商务价值来选择要实现的特性。

\item {} 
结对编程:系统的每一行代码都是两个人用一个键盘完成的。

\item {} 
测试驱动开发:先写测试,后写代码。

\item {} 
重构:不断优化系统设计,使之保持简单。

\item {} 
简单设计:为明确的功能进行最优的设计,不考虑未来可能需要的功能。

\item {} 
代码集体所有权:开发队伍中任何人可以修改任何其他人的代码,代码不属于某个个人。

\item {} 
持续集成:至少每天将整个系统集成一次,保持一个能运转的系统。

\item {} 
客户测试:客户自己也是软件开发队伍的重要一份子。

\item {} 
小版本发布:尽快发布,尽早发布。

\item {} 
每周40小时工作制:保证休息,保持体力。

\item {} 
编码标准:必须有统一的编码规范,确保代码的可读性。

\item {} 
系统隐喻:将整个系统联系在一起的全局视图;它是系统的未来影像,是它使得所有单独模块的位置和外观变得明显直观。如果模块的外观与整个隐喻不符,那么你就知道该模块是错误的。
\sphinxhref{http://www.woshipm.com/pmd/956881.html}{9}%
\begin{footnote}[817]\sphinxAtStartFootnote
\sphinxnolinkurl{http://www.woshipm.com/pmd/956881.html}
%
\end{footnote}

\end{enumerate}


\paragraph{角色}
\label{\detokenize{chapter_project/Scrum:id8}}

\subparagraph{Product Owner}
\label{\detokenize{chapter_project/Scrum:product-owner}}
负责明确目标、提出需求并排出优先级


\subparagraph{Scrum Master}
\label{\detokenize{chapter_project/Scrum:scrum-master}}
项目经理一般也是版本负责人,他会控制需求变更、增加需求的数量,提前预告风险,以及明确版本交付质量。

PM必须提醒大家每个版本都是产品的垂直切片,不能推迟到下一个版本才修复上个版本的bug和资源。


\subparagraph{Team}
\label{\detokenize{chapter_project/Scrum:team}}
Scrum相关著述推荐团队有7\textasciitilde{}9个成员(Schwaber 2004

将大团队拆分成多个小型Scrum团队负责某个模块,比如关卡原型制作团队。每个敏捷团队都包含所有职能的开发人员,设计、开发、测试等,他们共同为团队的产品交付物负责。


\paragraph{每日站会}
\label{\detokenize{chapter_project/Scrum:id9}}
一种有效的方法是让每个将要构建系统的人都在一个房间里,并让团队对每个项目的难度、人数和影响形成一致的估计。然后,您可以创建一个影响和易用性图表,根据投资回报对每个项目进行排序,并相应地进行优先级排序。在现实中,确定优先级是一个混乱和不稳定的过程,因为项目经常有依赖性,面临人员限制或与其他涉众截止日期的冲突。为了与其他团队或优先级保持一致,经常需要减少范围或牺牲质量。

目的:
\begin{enumerate}
\sphinxsetlistlabels{\arabic}{enumi}{enumii}{}{.}%
\item {} 
让所有团队成员统一步调,一起向着“完成Sprint”的目标进发;

\item {} 
承诺第二天要完成的工作,并重申团队对Sprint目标的承诺;

\item {} 
识别出团队面临的所有障碍;

\item {} 
使团队成员成为“一条绳子上的蚂蚱“,每个人都需要了解其它人面临的困难,以便在会后找到解决方法。

\end{enumerate}

Q:为什么一定要是”站“会呢?

A:因为要通过站立的形式让大家明确,这是一个需要快速解决的短会,而非冗长艰苦的会议。


\paragraph{反馈}
\label{\detokenize{chapter_project/Scrum:id10}}
销售,服务环节,运营环节,甚至公司团队成员本身,在ONES系统上我们可以用反馈池进行详尽记录:


\paragraph{敏捷开发 2\sphinxfootnotemark[818]}
\label{\detokenize{chapter_project/Scrum:id11}}%
\begin{footnotetext}[818]\sphinxAtStartFootnote
\sphinxnolinkurl{https://www.jianshu.com/p/e53974f9cbc9}
%
\end{footnotetext}\ignorespaces 
把一个大的产品功能模块拆解为多个相互独立的小功能模块,每次只上线一部分功能,在保证产品可用的前提下,一步步迭代完成整个功能的上线,这种方式就叫敏捷开发。
\sphinxhref{https://weread.qq.com/web/reader/8d232b60721a488e8d21e54kc51323901dc51ce410c121b}{4}%
\begin{footnote}[819]\sphinxAtStartFootnote
\sphinxnolinkurl{https://weread.qq.com/web/reader/8d232b60721a488e8d21e54kc51323901dc51ce410c121b}
%
\end{footnote}

例如,一款新的电商App要做一个购物车功能,第一个版本上线最基础的商品结算功能,第二个版本上线移除商品功能,第三个版本上线降价商品提醒功能等,不要在一次迭代中把所有的功能都做完。


\subparagraph{宣言}
\label{\detokenize{chapter_project/Scrum:id12}}\begin{itemize}
\item {} 
个体和交互胜过过程和工具。

\item {} 
可以工作的软件胜过面面俱到的文档。

\item {} 
客户合作胜过合同谈判。

\item {} 
响应变化胜过遵循计划。

\end{itemize}


\subparagraph{计划}
\label{\detokenize{chapter_project/Scrum:id13}}\begin{itemize}
\item {} 
任何过大的素材都应该被分解成小一点的部分,任何过小素材都应该和其它小的素材合并。

\item {} 
如果知道了开发速度,客户就能够对每个素材的成本有所了解。

\item {} 
迭代期间用户素材的实现顺序属于技术决策范畴。

\item {} 
一旦迭代开始,客户就不能改变该迭代期内需要实现的素材。

\end{itemize}


\subparagraph{测试}
\label{\detokenize{chapter_project/Scrum:id14}}
编写单元测试是一种验证行为,更是一种设计行为。同样,它更是一种编写文档的行为。编写单元测试避免了相当数量的反馈循环,尤其是功能验证方面的反馈循环。

首先编写测试可以:
\begin{itemize}
\item {} 
程序中的每一项功能都有测试来验证它的操作的正确性。

\item {} 
迫使我们使用不同的观察点。

\item {} 
迫使自己把程序设计为可测试的,从而迫使我们解除软件中的耦合。(forces
us to decouple the software)

\item {} 
作为一种无价的文档形式。

\end{itemize}


\subparagraph{重构}
\label{\detokenize{chapter_project/Scrum:id15}}
每一个软件模块都有三个职责:
\begin{enumerate}
\sphinxsetlistlabels{\arabic}{enumi}{enumii}{}{.}%
\item {} 
它运行起来所完成的功能。

\item {} 
它要应对变化。

\item {} 
要和阅读它的人进行沟通。

\end{enumerate}


\paragraph{项目看板}
\label{\detokenize{chapter_project/Scrum:id16}}
看板方法源自丰田的“及时生产”JIT=just\sphinxhyphen{}in\sphinxhyphen{}time)系统。

项目看板清晰地展示了:需求池中的哪些功能待开发;哪些功能进入UI设计阶段;哪些需求在开发阶段;哪些需求在测试阶段;哪些需求已经上线;哪些需求需要延期等。项目看板可以明确哪类问题需要谁去跟进,从而保证项目按照项目排期表稳步推进

看板方法可以动态显示瓶颈:你之所以能找到这些瓶颈,是因为限制了在制品(work\sphinxhyphen{}in\sphinxhyphen{}progress,
WIP)的数量会显示出瓶颈。

卡片代表了工作项,列代表了开发工序,卡片会从第一步工序流动到最后一步。每一列顶部的数字用来限制每一列最多允许放置卡片的数量。

\begin{figure}[H]
\centering
\capstart

\noindent\sphinxincludegraphics{{kanban}.png}
\caption{看板}\label{\detokenize{chapter_project/Scrum:id22}}\end{figure}

一些列分割成了两列,这是为了用来说明正在进行中的项与哪些已经完成并准备好被下游工序拉走的项。


\subparagraph{项目排期表}
\label{\detokenize{chapter_project/Scrum:id17}}
项目排期表为了保证项目按时上线,会使用项目排期表确定每个参与者的具体工作内容及起止时间。

\begin{figure}[H]
\centering
\capstart

\noindent\sphinxincludegraphics{{project_table_structure}.png}
\caption{项目排期表结构}\label{\detokenize{chapter_project/Scrum:id23}}\end{figure}

项目排期表示例如图所示。

\begin{center}\sphinxincludegraphics{{project_table}.png}\end{center} \sphinxincludegraphics{{project_schedule}.jpg} \sphinxincludegraphics{{project_devide}.jpg}


\paragraph{敏捷产品 5\sphinxfootnotemark[820]}
\label{\detokenize{chapter_project/Scrum:id18}}%
\begin{footnotetext}[820]\sphinxAtStartFootnote
\sphinxnolinkurl{https://zhiya360.com/135801.html}
%
\end{footnotetext}\ignorespaces 
对于产品小团体交付给设计小团体前,我们要做需求、方案、原型三个方面的敏捷冲刺
\begin{enumerate}
\sphinxsetlistlabels{\arabic}{enumi}{enumii}{}{.}%
\item {} 
需求敏捷

\item {} 
方案敏捷

\item {} 
原型敏捷

\end{enumerate}


\subparagraph{需求敏捷}
\label{\detokenize{chapter_project/Scrum:id19}}
所有公司都用专门的问题反馈线:

客户\sphinxhyphen{}>Customer Service\sphinxhyphen{}>Support Engineer\sphinxhyphen{}>PM\sphinxhyphen{}>SDM\sphinxhyphen{}>SDE


\paragraph{算法的敏捷迭代}
\label{\detokenize{chapter_project/Scrum:id20}}
深度学习算法在一定程度上比较难针对badcase(误杀或者漏杀)进行快速的迭代,这样的情况下,一方面可以通过产品功能上进行补充(黑白样本库),一方面目前学界逐步在解决小样本场景,部分业务场景已经在小样本上有可工程化的应用。

敏捷开发不再追求 MVP(Minimum Viable
Product,即最小化可行产品),而是追求 MDP(Minimum Data
Product),指训练算法的一个迭代所用的最小化数据集。
\sphinxhref{http://www.qidianlife.com/Singular/index.php?m=Home\&c=Discover\&a=article\&id=2211}{13}%
\begin{footnote}[821]\sphinxAtStartFootnote
\sphinxnolinkurl{http://www.qidianlife.com/Singular/index.php?m=Home\&c=Discover\&a=article\&id=2211}
%
\end{footnote}


\subparagraph{更多}
\label{\detokenize{chapter_project/Scrum:id21}}
2020 Scrum
Guide(含中英文版)\sphinxhref{http://www.shinescrum.com/downloads/Scrum\%E6\%8C\%87\%E5\%8D\%972020\%E4\%B8\%AD\%E8\%8B\%B1\%E6\%96\%87\%E7\%89\%88.zip}{Download}%
\begin{footnote}[822]\sphinxAtStartFootnote
\sphinxnolinkurl{http://www.shinescrum.com/downloads/Scrum\%E6\%8C\%87\%E5\%8D\%972020\%E4\%B8\%AD\%E8\%8B\%B1\%E6\%96\%87\%E7\%89\%88.zip}
%
\end{footnote}

\sphinxhref{https://gitbook.cn/gitchat/geekbook/5c73a5c56203c926b7ba8cc1}{中国式敏捷}%
\begin{footnote}[823]\sphinxAtStartFootnote
\sphinxnolinkurl{https://gitbook.cn/gitchat/geekbook/5c73a5c56203c926b7ba8cc1}
%
\end{footnote}


\subsubsection{需求价值的分析方法}
\label{\detokenize{chapter_project/separate_need:id1}}\label{\detokenize{chapter_project/separate_need::doc}}
对需求的分析可以从用户、场景、需求、流程四个维度进行快速拆解。
\begin{enumerate}
\sphinxsetlistlabels{\arabic}{enumi}{enumii}{}{.}%
\item {} 
用户
用户是谁,有什么特性,用户画像具体是什么样的?如果对用户不了解还需要进一步做用户研究。

\item {} 
场景
用户在什么场景中有需求,场景可以是用户在什么时间和地点,会发生什么,如果不了解场景,最好到用户的使用场景中,当面去看用户怎么操作。

\item {} 
需求
用户的需求是无穷的,需求的背后一定是要结合用户具体的场景,脱离了场景的需求都是伪需求。

\item {} 
流程
流程决定了此需求和其它需求的关系,如何让用户最爽的满足一系列需求,就形成了流程,在B端是业务流程图,在C端是产品架构设计。

\end{enumerate}

我在日常的产品工作中,不大可能拿很多模型分析需求,论证需求,这样比较低效,公众号留言
需求画布,下载最新的需求画布模板。新手可以照着模板思考每个需求价值,老鸟时间长了表格可以扔掉,对每个需求心理默念即可。


\subsubsection{价值 1\sphinxfootnotemark[824]}
\label{\detokenize{chapter_project/valuable:id1}}\label{\detokenize{chapter_project/valuable::doc}}%
\begin{footnotetext}[824]\sphinxAtStartFootnote
\sphinxnolinkurl{http://www.woshipm.com/pmd/4339402.html}
%
\end{footnotetext}\ignorespaces 

\paragraph{产品创新的价值}
\label{\detokenize{chapter_project/valuable:id2}}\begin{itemize}
\item {} 
发明价值

\end{itemize}

发明价值是能被称为创新的最基础的前提,也就是说你得弄点世上原本不存在的新东西出来。

我对发明和创新这两个词有着不同的理解:发明,只需要东西新,但创新,还需要“有用”。

有很多被发明出来的科研成果都被束之高阁,只能在实验室里等待适合的应用场景,你可以说它们有发明价值,但暂时还不是创新。因为它们还没有没找到用户价值。注意,这不是说科研不重要,科学研究当然很重要,如果拉长时间的尺度,你会发现底层的发明经常能带来重大的产业变革。只不过,在商业环境下,我们做产品,不能局限于发明价值。
\begin{itemize}
\item {} 
用户价值

\end{itemize}

任何一个产品要达到第二层价值,都不容易。毫不夸张地说,每年上市的新产品中,有
90\% 都缺乏用户价值,是“没人需要的新玩意”。

可你该如何达成用户价值呢?

这需要我们去理解用户、去深挖需求、感受场景,分析竞品等等,这叫想清楚;再把问题转化为合适的解决方案,多快好省地做出来;还需要推出去,让尽可能多的目标用户用上我们的产品。这是做好产品创新,最最基础的标准动作。

用户价值(主观效用)具备的特征:
\sphinxhref{https://www.jianshu.com/p/02df7160b7b0}{3}%
\begin{footnote}[825]\sphinxAtStartFootnote
\sphinxnolinkurl{https://www.jianshu.com/p/02df7160b7b0}
%
\end{footnote}
\begin{enumerate}
\sphinxsetlistlabels{\arabic}{enumi}{enumii}{}{.}%
\item {} 
认知依存:用户的认知决定了他的偏好

\item {} 
情境依存:有情境才有用户,脱离情境就没有用户

\item {} 
经验反馈:具有经验反馈演化的特征,是不断变化的

\end{enumerate}
\begin{itemize}
\item {} 
商业价值

\end{itemize}

有用户价值的产品中,又有很大一部分没法赚钱盈利,它们要么靠团队的情怀加资金积蓄、要么靠一个更大组织里的其他团队来输血、要么靠风险投资人对未来可能性的认可。

但这些都只能短期解决问题,长期来看,是无法支撑的。所以,一个可以长期独立生存的产品,一定要有\sphinxstylestrong{自我造血}能力。没法创造商业价值的产品,只能是一个真正的闭环产品的部分模块。

一个产品是否有商业价值,也是评判一个产品人是否具备“端到端”能力的标准之一,在大公司里做产品总监、甚至产品副总裁,也未必能训练到这个能力,但要自己创业成功,必须具备这个能力。
\begin{itemize}
\item {} 
社会价值

\end{itemize}

社会价值在于,一个产品不但自给自足,还产生了正向的外部性影响,可以让更多的社会角色受益。

产品有了社会价值,也就意味着,它在该领域的生态中,占据了相对重要的生态位,不那么容易“死”掉了。


\paragraph{Naive?}
\label{\detokenize{chapter_project/valuable:naive}}
“只要是为社会疯狂创造价值的企业,它的收入、利润早晚会兑现。”

做产品经理也一样,只要一直做的是有价值的产品,个人价值也终将会兑现。


\paragraph{价值分析维度}
\label{\detokenize{chapter_project/valuable:id3}}
产品价值分析维度分为四部分:分别是方案闭环、价值闭环、资源闭环、财务闭环。
\begin{enumerate}
\sphinxsetlistlabels{\arabic}{enumi}{enumii}{}{.}%
\item {} 
方案闭环
某个需求和产品是否能够满足用户的基本需求,大部分初中级产品每天的工作就是接需求和做需求,方案做了一两年,大部分产品经理都能够做到一个方案思考逻辑没毛病,方案自然闭环。

\item {} 
价值闭环
一个产品能够做到方案闭环,最多只能说这是一个功能,如果一个产品能称为产品,它必定要有价值闭环,价值是对于用户有用、好用。

\item {} 
资源闭环
大部分产品失败是在资源闭环上存在缺陷,说白了能力大于理想,特别是B端产品,涉及供应链或者各种商户资源,如果没有资源把商户快速圈到平台中,就无法解决交易供给问题,产品功能做得再好,最后也成不了。

\item {} 
财务闭环
移动互联网十年之路,何其辉煌,如果回到2010年,大家能想到大部分移动互联网公司其实都不赚钱,赚钱的公司最后活下来靠的是金融业务。不赚钱,有用户没有收入,财务收入cover不了财务支出,这是大部分互联网公司的痛。

\end{enumerate}


\paragraph{产品价值实现路径}
\label{\detokenize{chapter_project/valuable:id4}}\begin{enumerate}
\sphinxsetlistlabels{\arabic}{enumi}{enumii}{}{.}%
\item {} 
mvp

\end{enumerate}

“如果你的产品的第一个版本没有让你感到尴尬,那么你的推出太晚了”,Reid
Hoffman

mvp指的是用\sphinxstylestrong{最小的成本}做对用户有价值的最小可用产品(对没有太强相关的\sphinxstylestrong{克制!})

当不知道做什么的时候,用户需求也把握不准的时候,先做做一个简单的产品方案去试点,这就是MVP。

MVP的产品不是单一的形态。只要可以让你的用户直观地感知到或者让他们实际使用起来,能激发他们真实的使用体验和感受就是可行的。
\begin{enumerate}
\sphinxsetlistlabels{\arabic}{enumi}{enumii}{}{.}%
\item {} 
pmf

\end{enumerate}

pmf指的是产品与市场匹配的产品

判断产品的价值和市场的价值是否匹配,但是如何判断PMF的临界点呢,怎么才算是产品价值和市场匹配呢?

可以参考C端,30\%的新用户次日留存,当然不同的产品类型这个数据可能有差异,如果实在不清楚,找用户做个问卷调查,看看他们对产品的真实反馈。你觉得我怎么样?貌似是向用户的一句表白,大部分收到的反馈是痛彻心扉。

因为高手从来不问你觉得我怎么样,他们不需要用户说,自己能感受到。
\begin{enumerate}
\sphinxsetlistlabels{\arabic}{enumi}{enumii}{}{.}%
\setcounter{enumi}{2}
\item {} 
Aha moment

\end{enumerate}

Aha moment是指产品使用户眼前一亮的时刻,是用户发现产品核心价值的时刻

当产品pmf满足之后,就可以尝试寻找产品的Aha
moment了,一般可以通过数据分析去发现产品的ahamoment。

在数据分析的时候去找那些用户使用了的功能或者服务,产品留存数据特别好,如果不使用,产品留存就会比较差的临界点。比如Facebook
的aha moment 是新用户有6个好友,留存率就特别高。

从pmf到Aha moment 是连续的过程,不能从mvp跨越pmf直接达到 Aha moment。

丁香医生的少楠他给我讲了丁香医生当时是怎么发展起来的,起先团队也不知道怎么做,当时产品模型搭建好之后,看似有很多方向,但是却一块也不强。

没有天生的赢家,在产品成功之前,楠哥和大部分人一样,也是懵逼的。接着看数据发现青春痘问诊数据特别好,所以后来丁香医生就专攻青春痘问诊,包括现在青春痘问诊也是核心业务,这就是找到产品的ahamoment,然后快速的复制。

反观有些产品,别说aha
moment了,pmf都没有,留存率特别低,所以只能回到用户中去再次调研,继续调整产品。就像挖一口井一样,有些地方天生不适合挖井,有些地方有地下水,但是没等到最后,团队因为投入问题而解散,这是大部分失败产品的命运。
\begin{enumerate}
\sphinxsetlistlabels{\arabic}{enumi}{enumii}{}{.}%
\item {} 
产品反馈
\sphinxhref{https://coffee.pmcaff.com/article/2753643635311744/pmcaff?utm\_source=forum}{4}%
\begin{footnote}[826]\sphinxAtStartFootnote
\sphinxnolinkurl{https://coffee.pmcaff.com/article/2753643635311744/pmcaff?utm\_source=forum}
%
\end{footnote}

\end{enumerate}

能够方便的集成在产品中,支持安卓iOS公众号小程序和WEB端,兼容PC视图和移动端视图\sphinxhyphen{}>可以识别反馈用户的身份,环境信息\sphinxhyphen{}>可以选择反馈分类,支持图文反馈\sphinxhyphen{}>有完善的通知机制和自动化机制\sphinxhyphen{}>所有反馈集中在后台显示,可回复可流转\sphinxhyphen{}>反馈回复能通知到用户\sphinxhyphen{}>用户可对回复进行评价,是否解决用户问题\sphinxhyphen{}>有完善的数据统计机制,能统计问题/回复数量,可统计运营人员工作量\sphinxhyphen{}>能导出完整的反馈/回复数据,便于进一步分析,构建用户画像,指导产品迭代。

如果是你来设计,是否能设计的比这个流程还有完善,如果要完成这些功能,需要多长的研发周期

埋点:进行二次或三次的拆解,了解每个节点数据流向,后续才能建立数据模型,分析用户操作。

异常监控:人力有限,我们无法做到24小时实时盯着数据。若功能中有影响用户核心操作的功能点,还需要让研发帮忙建立异常监控,以保证产品核心流程能够满足用户正常使用。
\begin{enumerate}
\sphinxsetlistlabels{\arabic}{enumi}{enumii}{}{.}%
\item {} 
完善的产品社区体系

\end{enumerate}

MIUI社区对于MIUI的成功来说可谓功不可没。

能在产品社区持续反馈产品问题,与产品方深入沟通自己的需求和使用场景,做任何调研都积极配合,有这么一帮铁杆种子用户,产品想不做成都难。

不过社区是一把双刃剑,良好的社区氛围能感染很多用户成为产品铁粉儿,成为产品持续优化迭代的源泉。但难就难在怎么做好社区氛围,氛围不好的社区就会成为一个产品负面消息的集散地。

良好的社区氛围和社区产品的体验以及社区运营能力有很大关系。

社区产品必须具备基本的登录、发贴、回贴、消息、个人主页功能,而帖子审核、置顶、关闭评论、点赞、标签、举报又是非常基础的运营需求。


\paragraph{价值的选择问题}
\label{\detokenize{chapter_project/valuable:id5}}
我经历的好几家公司都是以市为目的。之前在公司的时候,经常有投资者来公司参观,公司的很多决策也会按照给投资人描绘的进行布局,有些项目根本不赚钱,但是为了让故事好看,这些项目还是一定要做。

产品的工作向来不是如此纯粹,还得和内心做抗争,还得被程序员误解需求不过脑子,哀莫大于心死,一个项目死了,收拾心情,换个项目又是一条好汉。

从A股股王茅台对比美国的股王苹果,可见一二,国内的产品氛围大多还是销售(流量)驱动,国内的环境离真正的产品价值驱动还有一段距离要走。

在大环境不好的时候,做选择,选择相对价值高的事情。人和人的区别就是从做事是否有价值中拉开差距的,价值的累加就是赢的感觉,谁也不想做输家。


\subsubsection{APP体验}
\label{\detokenize{chapter_project/APP_experience:app}}\label{\detokenize{chapter_project/APP_experience::doc}}

\subsection{面试}
\label{\detokenize{chapter_interview/index:chap-interview}}\label{\detokenize{chapter_interview/index:id1}}\label{\detokenize{chapter_interview/index::doc}}

\subsubsection{该怎么准备面试 {[}1{]}}
\label{\detokenize{chapter_interview/index:id2}}\begin{itemize}
\item {} 
面试前的准备:简历信息梳理\sphinxhyphen{}自我认知\sphinxhyphen{}公司行业认知\sphinxhyphen{}产品体验\sphinxhyphen{}业务侧重思考理解

\item {} 
面试流程:自我介绍\sphinxhyphen{}为何转行\sphinxhyphen{}项目梳理\sphinxhyphen{}公司情况\sphinxhyphen{}岗位情况\sphinxhyphen{}请教问题

\end{itemize}


\paragraph{心态}
\label{\detokenize{chapter_interview/xintai:id1}}\label{\detokenize{chapter_interview/xintai::doc}}

\subparagraph{低姿态不能换来成功 1\sphinxfootnotemark[827]}
\label{\detokenize{chapter_interview/xintai:id2}}%
\begin{footnotetext}[827]\sphinxAtStartFootnote
\sphinxnolinkurl{https://weread.qq.com/web/reader/46532b707210fc4f465d044k182326e0221182be0c5ca23}
%
\end{footnotetext}\ignorespaces 
情绪图:

\begin{figure}[H]
\centering
\capstart

\noindent\sphinxincludegraphics{{emotion}.jpg}
\caption{emotion}\label{\detokenize{chapter_interview/xintai:id10}}\end{figure}


\subparagraph{在商业公司中很少有学徒制}
\label{\detokenize{chapter_interview/xintai:id3}}
其实能够看得出来,不管你摆出什么样的低姿态,你的潜台词就是“请相信我一次吧,我愿意学,请给我一次机会吧”“我可以不要工资,等我学会了你再给我工资”。这表示你不够成熟,并没有感受到商业社会的激烈竞争。在古代,你可以上山学艺或者跪拜求师,但是现在社会发展这么快,每家公司都要努力赢利才能生存下去,单凭你摆出一个低姿态就决定录用你是不可能的。没有经验不要紧,不一定要用低姿态去乞求别人,

你可以展示出你的潜力、你的创造力、你的创新力,哪怕只展示出比别人勤奋也可以。比如,你喜欢某个公司的职位,而其他人可能只大概了解这个公司是做什么的、是什么行业的。你可以收集相关的报道、社交媒体上的客户投诉和反馈信息,可以整理同行的信息,可以体验竞争对手的产品,可以写出自己的产品体验报告等,这些都可以体现出你有没有努力,有多么渴望得到这份工作,愿意付出多少。


\subparagraph{你喜欢的并不一定就是你的,你需要让自己配得上}
\label{\detokenize{chapter_interview/xintai:id4}}
查理·芒格曾经说过这样一个道理:“要得到你想要的某样东西,最可靠的办法是让你自己配得上它。”你如果想进入大公司工作,那么千万不要只是停留在想的阶段,也不要停留在羡慕的阶段,而要真正行动起来。既然你设定了目标,就应该想办法达到目标。你喜欢大公司,但是也要明白大公司是否喜欢你,凭什么喜欢你。所以,你必须要对自己做比较客观的定位分析和能力分析。你喜欢大公司不代表你就能去大公司工作,你要达到大公司的要求才能够去大公司工作。


\subparagraph{公司招聘你是为了让你完成工作,创造价值}
\label{\detokenize{chapter_interview/xintai:id5}}
公司招聘你加入公司,就是想让你完成工作,创造价值。在你不能创造价值时,公司不会给你发几个月工资培养你的工作能力。所以,你必须要先满足公司的最低要求。你在进入公司后接受培训是为了更高效的工作,并且能胜任更多的工作。所以,你千万不要想着以低姿态求职成功,即便成功了,可能也是运气好。


\subparagraph{把个人感情与工作任务分开 2\sphinxfootnotemark[828]}
\label{\detokenize{chapter_interview/xintai:id6}}%
\begin{footnotetext}[828]\sphinxAtStartFootnote
\sphinxnolinkurl{https://www.deepshare.net/index.php?c=article\&id=210}
%
\end{footnotetext}\ignorespaces 
没多少人喜欢找工作这件事,你又怎知面对你的 HR
和面试官不讨厌他们自己的工作,但这不妨碍每个人扮演好自己的职场角色,做好成人世界的游戏。

毕竟,对立心态在职场对自己毫无好处。

解决了心态问题,接下来为找到工作行动起来就是自然而然的事了。

这里限于篇幅就不赘述找工作细节了,具体求职攻略我会以后单独更新。但有一点关键「元」方法我一定要写一写。

找工作是一件比找对象还讲究「合适」的事。每当看到一份招聘启事(JD)时,脑子里应该立刻放大并\sphinxstylestrong{高频闪烁「match(匹配)」一词}。

HR 并不关心你的人生经历或你的履历是否高大上,TA
只关心你是不是这个\sphinxstylestrong{职位的「match」},你是否在简历和面试中清楚有力地表达了自己具备这个职位所需的技能或拥有学习这项技能的潜能,并给出了具体经历来证明你说的论点。这一点无比重要,不论什么水准的求职者都可能犯这个错误,即把求职的焦点从「match」上移开。


\subparagraph{积累业务经验}
\label{\detokenize{chapter_interview/xintai:id7}}
公司里面有很多东西并不是直接使用开源代码就能够发挥作用的,在公司里面无论做什么事情,最重要的一点就是对业务的理解。

在做业务的过程中,通常都会经历很多的坑,无论是别人主动挖的,还是自己踩坑踩出来的,都是自身宝贵的财富和经验。


\subparagraph{理解业务架构 3\sphinxfootnotemark[829]}
\label{\detokenize{chapter_interview/xintai:id8}}%
\begin{footnotetext}[829]\sphinxAtStartFootnote
\sphinxnolinkurl{https://www.deepshare.net/index.php?c=article\&id=212}
%
\end{footnotetext}\ignorespaces 
很多人以为算法工程师的工作就是把从论文和公开课里学到的 fancy
的算法用到业务里。这个基本上就是大错特错了。首先,绝大多数的先进算法只是相对于
baseline
算法有了一点点小的提升,这对于做科研是有意义的,毕竟积少成多。但是在工程中,这些算法的性价比是极低的,收益不大,却要大幅度调整系统,增加系统复杂度,得不偿失。

所以除非是非常颠覆性的想法,大幅度提升性能,像是
DNN,ResNet,Word2Vec,Bert,Seq2Seq
这样的模型算法,才会在业界广泛的应用。

那么更多时候,算法工程师的工作是结合业务,用上述提到的这些基本的模型去优化业务流。比如,在广告领域,原来要求完全匹配用户的搜索词,我可不可以用
Seq2Seq 模型改写出几个类似的搜索词?或者,原来都是 Counting
Feature,我可不可以用 DNN Embedding 来做一些离散 Feature?

换句话说,模型都是最基本的模型,但是怎么结合业务,选对模型,用对地方,才是真正考验算法工程师能力的地方。而要充分理解业务架构,并且能够在复杂的业务代码中自由的翱翔,你的工程能力一定不能差。


\subparagraph{社交网络 4\sphinxfootnotemark[830]}
\label{\detokenize{chapter_interview/xintai:id9}}%
\begin{footnotetext}[830]\sphinxAtStartFootnote
\sphinxnolinkurl{https://www.deepshare.net/index.php?c=article\&id=222}
%
\end{footnotetext}\ignorespaces 
一旦你完成了步骤 a)和步骤
b),社交网络能真正帮助你找到好工作。如果你不和人交谈,你会错过很多你可以发挥很好的机会。每天都结识新人是非常重要的。如果不是面对面交流,那么在
LinkedIn
上也可以。这样坚持多天后,你就拥有了一个庞大而强大的社交网络。社交网络不是为了让你认识的人为你做内推。在我开始找工作的时候,我经常犯这种错误,直到我偶然看到马克梅隆(Mark
Meloon)的这篇优秀文章(\sphinxurl{https://www.markmeloon.com/climbing-relationship-ladder-get-data-science-job/}),他谈到:建立人与人之间真正的关系,首先主动帮助他人是十分重要的。社交网络中的另一个重要步骤是分享您的消息。例如,如果你擅长某事,就可以在
Facebook 和 LinkedIn
上分享相关博客的链接。这不仅可以帮助他人,还可以帮助你。一旦你有足够好的社交网络,你的影响力就会成倍增加。你永远不知道,社交网络中的某个人通过对你的帖子进行点赞或评论,可能会帮助你接触更广泛的受众群体,其中可能包括正在寻找符合你专业经验的招聘人员。


\paragraph{选择}
\label{\detokenize{chapter_interview/choose:id1}}\label{\detokenize{chapter_interview/choose::doc}}

\subparagraph{职业价值观排序}
\label{\detokenize{chapter_interview/choose:id2}}
12选3
\sphinxhref{https://weread.qq.com/web/reader/46532b707210fc4f465d044k8f132430178f14e45fce0f7}{1}%
\begin{footnote}[831]\sphinxAtStartFootnote
\sphinxnolinkurl{https://weread.qq.com/web/reader/46532b707210fc4f465d044k8f132430178f14e45fce0f7}
%
\end{footnote}
\begin{enumerate}
\sphinxsetlistlabels{\arabic}{enumi}{enumii}{}{.}%
\item {} 
追求创新

\item {} 
追求通过工作去帮助他人

\item {} 
追求声望和地位

\end{enumerate}

理由:
\begin{enumerate}
\sphinxsetlistlabels{\arabic}{enumi}{enumii}{}{.}%
\item {} 
创新是区别我与别人的唯一方法。

\item {} 
帮助是我能提供的真的价值。

\item {} 
声望与地位能反馈出我的帮助是否合适。

\end{enumerate}


\subparagraph{个人评估 2\sphinxfootnotemark[832]}
\label{\detokenize{chapter_interview/choose:id3}}%
\begin{footnotetext}[832]\sphinxAtStartFootnote
\sphinxnolinkurl{https://weread.qq.com/web/reader/46532b707210fc4f465d044kd3d322001ad3d9446802347}
%
\end{footnotetext}\ignorespaces \begin{enumerate}
\sphinxsetlistlabels{\arabic}{enumi}{enumii}{}{.}%
\item {} 
我的个性是什么

\item {} 
我的工作兴趣是什么

\item {} 
我在工作中的梦想是什么

\item {} 
我想成为什么样的人

\item {} 
我做人和做事的价值观是什么

\item {} 
我具有什么样的天赋与特长

\end{enumerate}


\subparagraph{工作评估}
\label{\detokenize{chapter_interview/choose:id4}}\begin{enumerate}
\sphinxsetlistlabels{\arabic}{enumi}{enumii}{}{.}%
\item {} 
这个工作有非常好的发展空间吗

\item {} 
这个工作可长期从事吗

\item {} 
这个工作需要掌握一门技艺或者本领吗

\item {} 
这个工作有创造性、不易被取代吗

\item {} 
这个工作能激发我的成就感吗

\item {} 
这个工作符合我的性格特点和职业需求吗

\end{enumerate}


\subparagraph{终极评估}
\label{\detokenize{chapter_interview/choose:id5}}\begin{enumerate}
\sphinxsetlistlabels{\arabic}{enumi}{enumii}{}{.}%
\item {} 
如果世界上所有工作的收入都一样,那么我会从事这个工作吗

\item {} 
我希望自己的孩子以后也从事这样的工作吗

\item {} 
如果还有十年的寿命,那么我还会从事这个工作吗

\end{enumerate}


\subparagraph{愿景和规划 3\sphinxfootnotemark[833]}
\label{\detokenize{chapter_interview/choose:id6}}%
\begin{footnotetext}[833]\sphinxAtStartFootnote
\sphinxnolinkurl{https://weread.qq.com/web/reader/46532b707210fc4f465d044kc7432af0210c74d97b01b1c}
%
\end{footnotetext}\ignorespaces 
愿景是指对未来的前景和发展方向的高度概括的描述。愿景是由核心价值观、核心目的和对未来的展望组成的。愿景最好是未来10年以上的远大目标和对目标的生动描述。稻盛和夫曾经说,要想象一个彩色的未来,越具体越好,这是成功的首要条件,不要以现在的能力束缚对未来的想象!

规划是指个人或组织制订的比较全面、长远的发展计划,是对未来整体性、长期性、基本性问题的思考和考量,是设计未来整套行动的方案。


\subparagraph{行业 4\sphinxfootnotemark[834]}
\label{\detokenize{chapter_interview/choose:id7}}%
\begin{footnotetext}[834]\sphinxAtStartFootnote
\sphinxnolinkurl{https://weread.qq.com/web/reader/46532b707210fc4f465d044k6f4322302126f4922f45dec}
%
\end{footnotetext}\ignorespaces 
你回过头来看看你的简历,是在一个领域、一个行业深耕多年,还是觉得哪个行业有前途就去哪个行业工作、觉得哪个领域最近受到资本市场追捧就去哪个领域工作呢?很多人跳槽漫无目的,只要有一份工作就行,只要还是做产品经理就行,只要觉得这个公司做的事情相对靠谱,商业逻辑听上去比较合理、有趣就选择加入这个公司,完全没有认真地思考过自己在经验上的积累、在行业和领域上的沉淀,导致很多行业的工作都做过,每次都要重新调研用户需求,每次都至少要花3个月的时间才能够真正地感受到用户想要什么,这样做无疑是在浪费生命。

如果你作为产品经理,非常有心计,每次都探索用户的本质需求,并且能够跨领域、跨行业地思考它们之间的共性,探索如何将在一个行业中所获得的能力迁移到另一个全新的领域,那么这也算一种成长。


\subparagraph{大公司 or 小公司}
\label{\detokenize{chapter_interview/choose:or}}
\begin{figure}[H]
\centering
\capstart

\noindent\sphinxincludegraphics{{company_vs}.png}
\caption{大公司:raw\sphinxhyphen{}latex:\sphinxtitleref{小公司的优缺点}}\label{\detokenize{chapter_interview/choose:id10}}\end{figure}
\begin{itemize}
\item {} 
什么公司值得去:\sphinxurl{https://www.kanzhun.com/}

\item {} 
找人才:\sphinxurl{http://www.hunt007.com/employee/searchlist.htm}

\end{itemize}


\subparagraph{做大的几种效应 6\sphinxfootnotemark[835]}
\label{\detokenize{chapter_interview/choose:id8}}%
\begin{footnotetext}[835]\sphinxAtStartFootnote
\sphinxnolinkurl{https://www.bilibili.com/video/BV19v411k75u?from=search\&seid=11494051329064518502}
%
\end{footnotetext}\ignorespaces \begin{itemize}
\item {} 
头部效应:一步领先步步领先,资源聚集

\item {} 
规模效应:边际成本递减,供/需端某物多多益善

\item {} 
协同效应:多种事物互相促进“一鱼多吃”

\item {} 
网络效应:供需同体者的同质化网络(价值正比于N\textasciicircum{}2)

\item {} 
多边效应:供需多方(>=2)彼此强化

\end{itemize}


\subparagraph{人生战略}
\label{\detokenize{chapter_interview/choose:id9}}
人生战略的雏形目标。\sphinxhref{https://coffee.pmcaff.com/article/2147290812813440/pmcaff?utm\_source=forum\&newwindow=1}{5}%
\begin{footnote}[836]\sphinxAtStartFootnote
\sphinxnolinkurl{https://coffee.pmcaff.com/article/2147290812813440/pmcaff?utm\_source=forum\&newwindow=1}
%
\end{footnote}
\begin{itemize}
\item {} 
使命:让所有有价值的产品项目能不被扼杀在萌芽上。

\item {} 
愿景:做一款智能投研产品实现我的使命,发现别人价值,并帮助别人成长。。

\item {} 
战略:偏向投资、整合,为更多投资人与创业者建立桥梁。

\item {} 
战术:野蛮、快速试错;AI;建立更多联系。

\item {} 
价值观:推进科技进步,实现人类幸福。

\end{itemize}


\paragraph{要求}
\label{\detokenize{chapter_interview/require:id1}}\label{\detokenize{chapter_interview/require::doc}}

\subparagraph{硬性要求}
\label{\detokenize{chapter_interview/require:id2}}
硬性要求是指马上就能甄别出来的,比如:
\begin{itemize}
\item {} 
学历要求:一般要求是专科,有一些公司也要求必须是本科,甚至是985、211,学历一般要求很难伪装或造假,不然后果很严重,你们懂的。

\item {} 
专业要求:中文、市场营销、电子商务、计算机科学与技术(优先)等都可以;

\item {} 
工作年限:一般一年起的工作经验就行,但是必须是互联网相关经验,最好是产品经理实习经验和工作经验。三年工作经验的童鞋不一定在面试中有多大优势。白纸有的时候还是更大优势的,因为白纸可以由用人方通过公司内部培训定制其工作能力,比某些有经验的求职者更容易驾驭。

\end{itemize}


\subparagraph{能力要求}
\label{\detokenize{chapter_interview/require:id3}}
能力要求,比如对某类产品的了解程度、对工作中需要使用的软件工具的熟悉程度等。
\begin{itemize}
\item {} 
熟练使用Visio、Project、Axure、OmniGraffle等软件独立完成产品PRD文档、Demo制作及MRD撰写;

\item {} 
具备熟练的数据分析,总结能力,能够通过各种运营数据的收集,归纳,分析对产品的设计和运营进行及时的调整及改进;

\item {} 
有产品意识、用户意识;

\item {} 
各种软能力,比如自我管理能力、逻辑思维、沟通学习能力、抗压能力等。\sphinxhref{https://zhuanlan.zhihu.com/p/25821858}{1}%
\begin{footnote}[837]\sphinxAtStartFootnote
\sphinxnolinkurl{https://zhuanlan.zhihu.com/p/25821858}
%
\end{footnote}

\end{itemize}


\paragraph{令人喜爱的新人1\sphinxfootnotemark[838]}
\label{\detokenize{chapter_interview/new_like:id1}}\label{\detokenize{chapter_interview/new_like::doc}}%
\begin{footnotetext}[838]\sphinxAtStartFootnote
\sphinxnolinkurl{http://www.woshipm.com/pmd/284339.html}
%
\end{footnotetext}\ignorespaces 

\subparagraph{寻找亮点}
\label{\detokenize{chapter_interview/new_like:id2}}
\begin{figure}[H]
\centering
\capstart

\noindent\sphinxincludegraphics{{shining_point}.png}
\caption{寻找亮点}\label{\detokenize{chapter_interview/new_like:id22}}\end{figure}


\subparagraph{心态}
\label{\detokenize{chapter_interview/new_like:id3}}
在解决需求的过程当中,对产品产生关注和信任,在对工作方法和心态都了解之后,自然而然的就变成产品经理了。产品经理是对某个问题提出一系列解决方案和推动落实的那个人,你要对用户负责,对产品负责,更要对你自己负责


\subparagraph{懂行}
\label{\detokenize{chapter_interview/new_like:id4}}
如果大家平时就能写一下文章,或者对一些新出的产品给一些自己的见解或者改进方案,不妨整理好放在iPad里面,面试带过去效果会很好!了解行业动态推荐关注:虎嗅网、36氪等科技媒体,学习产品基本功,推荐关注:人人都是产品经理.


\subparagraph{经历}
\label{\detokenize{chapter_interview/new_like:id5}}
撬开BAT大门的基本配置=1\sphinxhyphen{}2份有料的互联网实习+1次以上的项目经历+个人作品或高质量比赛大奖。


\subparagraph{识人}
\label{\detokenize{chapter_interview/new_like:id6}}
过来人的帮助不仅能够让你获取更多的内部信息和资源(很多实习机会都是内部消化的),还可以在关键的时候给你一些比较全面的辅导和指引。所以可以通过参加一些活动,或者进入一些交流的微信群、QQ群等认识一些在互联网工作的师兄师姐或者HR


\subparagraph{包装}
\label{\detokenize{chapter_interview/new_like:id7}}
得到了讯息,看到了机会,也有经历之外,还需要懂得如何把最好、最真实的自己呈现出来,这就很需要个人修养了。


\subparagraph{学习}
\label{\detokenize{chapter_interview/new_like:id8}}
学会Xmind,Mindmanager,Visio,PS,Axure这些软件,另一方面还要多看书


\subparagraph{更关注能力 2\sphinxfootnotemark[839]}
\label{\detokenize{chapter_interview/new_like:id9}}%
\begin{footnotetext}[839]\sphinxAtStartFootnote
\sphinxnolinkurl{https://weread.qq.com/web/reader/46532b707210fc4f465d044kc1632f5021fc16a5320f3dc}
%
\end{footnotetext}\ignorespaces 

\subparagraph{在工作能力上是否与目标职位匹配}
\label{\detokenize{chapter_interview/new_like:id10}}
用户研究能力、产品分析能力、需求管理能力、用户体验能力、数据分析能力、技术理解能力、项目管理能力、行业和商业分析能力等。

面试官会问很多工作经历的细节信息,其实就是要听到你是怎么工作的,运用了什么方法,得到了什么结论,做出了什么决策。你千万不要以为面试官只是对你过去的公司感兴趣,对你取得的优秀业绩感兴趣,那些项目和取得的业绩数字都是历史结果,都已经不可能改变,也不可能在新公司完全复制。

面试官更关心的是你的能力是否与目标职位匹配。能力匹配不一定要经过很长时间,只要你的工作方法对、能力强,就可以在短时间内达到别人3年或5年的积累。


\subparagraph{在行业经验上是否与目标职位匹配}
\label{\detokenize{chapter_interview/new_like:id11}}
面试官希望你在行业经验上有3年的积累,首先是希望你对这个行业很了解、很熟悉,对该行业的产品比较熟悉,知道该行业的商业模式、产品生态等宏观层面的分析结论,并且对该行业产品的全流程工作有深刻的理解,有完整的业务产品包装能力,对产品的各个环节的设计思路清晰,对数据敏感,并熟知行业内各家玩法及相关数据。


\subparagraph{在思考与表达上是否与目标职位匹配}
\label{\detokenize{chapter_interview/new_like:id12}}
不要特别在意工作经验,要在意的是如何表达出3年工作经验所具备的工作能力、如何把以往的工作经验与目标职位的要求进行对位的复盘总结、如何在表达时讲出恰好面试官想听到的能力。在面试的时候,你要尽量表达对全流程的熟悉,能够展现出对行业的洞察力,这才能够体现出你有一定的行业积累。


\subparagraph{最好不是真的“新人” 3\sphinxfootnotemark[840]}
\label{\detokenize{chapter_interview/new_like:id13}}%
\begin{footnotetext}[840]\sphinxAtStartFootnote
\sphinxnolinkurl{https://www.zhihu.com/pub/reader/119583028/chapter/1057335985628672000}
%
\end{footnotetext}\ignorespaces 
进入到一个新行业,任何人都需要花时间去学习、去沉淀,对于企业方的招聘需求来说,他们要招的是产品经理而不是产品助理,产品经理的了解新领域时间对于企业来说是种浪费,企业方更倾向于招到一个能够「快速执行」的产品经理。


\subparagraph{应届毕业生的成功}
\label{\detokenize{chapter_interview/new_like:id14}}
对于应届毕业生来说,成功是什么?最显著的标志就是转正。产品经理带领应届毕业生参与了具体的项目,如果他整体的表现得到了同事们的一致认可,那么转正也是必然的事情了。


\subparagraph{融入真实的职场}
\label{\detokenize{chapter_interview/new_like:id15}}
众所周知,学校和职场是有很大的差别的。相对来说,学校非常单纯,虽然也有利益,但是更多的还是同学之间的感情,而在职场上,利益因素更多一些。

因此,应届毕业生要做的第一件事情就是让自己尽快地融入职场,而应届毕业生的导师要做的第一件事情就是帮助其尽快地融入职场,要告诉他们职场上的规章制度,也要告诉他们职场上的一些注意事项。


\subparagraph{实际工作中的价值体现}
\label{\detokenize{chapter_interview/new_like:id16}}
无论通过哪种渠道,应届毕业生都会或多或少地了解到工作所带来的价值,但是了解到的和实际工作中的往往都是有出入的。因此,应届毕业生需要真正融入具体的工作中,去执行、去体会。产品经理要让应届毕业生参与具体的项目,在执行过程中感受自己所负责的工作的具体职责及最终呈现出的价值是什么。


\subparagraph{认识身边的每一个人}
\label{\detokenize{chapter_interview/new_like:id17}}
应届毕业生通过参与具体的工作,能够提升自己的专业能力,也能够快速、高效地认识身边的每一个人。在职场中,如果两个人之间没有工作上的交集,这种认识就是流于表面的、浅层的。如果两个人曾经一起做过同一件事情,那么就更容易拉近彼此的距离和感情。在职场中,工作本身就是最重要的,想要了解一个人还是要看他在工作中的表现,一起工作意味着合作,合作就会拉近彼此的距离。

导师需要引荐毕业生,让更多的人认识他,帮助他建立人脉关系。很多时候,毕业生转正最重要的其实就是身边人的评价。即使应届毕业生到公司已经
3
个月了,也很难真正地体现出自己的能力。身边人的评价更容易说明应届毕业生是否能真正融入其中,包括主动性和意愿等。


\subparagraph{系统性的知识框架}
\label{\detokenize{chapter_interview/new_like:id18}}
除了上述三点,还有最重要的一点就是系统性的知识框架。如果应届毕业生想要获得持续的成长,就要对产品经理这一岗位所需要的系统性的知识框架有清晰的了解。因此,导师一定要做好以下三件事。


\subparagraph{搭建知识框架}
\label{\detokenize{chapter_interview/new_like:id19}}
产品经理基于岗位的职责和需求可以将知识框架分为通用技能、专业技能和组织影响力。

通用技能可以进一步地拆分为沟通能力、主动学习能力、行业理解能力、执行力等。

专业技能可以进一步地拆分为用户调研能力、需求挖掘能力、原型交互能力、产品规划能力、数据分析能力、用户运营能力、渠道管理能力等。

组织影响力则可以进一步地拆分为主动分享、方法论沉淀、活动组织能力等。

当然,每一项都可以进一步地拆分成更小的颗粒。比如,沟通能力可以拆分为表达能力、思辨能力、倾听能力、说服能力等。


\subparagraph{对照打分}
\label{\detokenize{chapter_interview/new_like:id20}}
搭建完产品经理的知识框架以后,产品经理接下来要做的是对每一项进行权重设定,在此基础上对每一位应届毕业生进行对照打分,分值分别对应产品经理这一岗位上的相应位置和层级。

其实,这样做的目的就是让应届毕业生能够清晰地了解到自己现阶段在产品经理这个岗位上所处的位置和水平。如果想要提升自己,就要先认清自己。


\subparagraph{查漏补缺}
\label{\detokenize{chapter_interview/new_like:id21}}
对照打分结束之后,产品经理要有针对性地进行查漏补缺,先从两个方面入手。一方面是目前得分最高的项目,分析和总结此项目得分高的原因,思考继续保持的方法;另一方面则是目前得分最低的项目,分析和总结此项目得分低的原因,思考如何快速地提高此项目的分值。

例如,得分最高的是行业理解能力,因为这个应届毕业生在学校时就喜欢关注互联网行业的各种新闻报道,无论是官方报道、小道消息、社区论坛还是权威的行业报告,他都会在第一时间进行了解。因此,对于这一项,他只要继续保持\sphinxstylestrong{行业的敏感度}就足够了。

得分最低的是用户调研能力,产品经理需要制订一个专项提升计划,如一个月内学会定性调研之面访,然后让这个应届毕业生参与其中。比如,先成功地约至少
10
位目标用户来公司参加头脑风暴会议,在头脑风暴会议中担任助理,协助处理基本的会务和会议记录等。在此基础上,这个应届毕业生要不断地深入学习、实际操作、反复总结,直到此项也达到某一分值为止。


\paragraph{简历 CV}
\label{\detokenize{chapter_interview/CV:cv}}\label{\detokenize{chapter_interview/CV::doc}}
\begin{center}\sphinxincludegraphics{{CV}.jpg}\end{center} \sphinxincludegraphics{{CV2}.png}


\subparagraph{LinkedIn and GitHub}
\label{\detokenize{chapter_interview/CV:linkedin-and-github}}
\sphinxurl{https://career-resource-center.udacity.com/linkedin-github-profiles}


\subparagraph{CV VS Resume 5\sphinxfootnotemark[841]}
\label{\detokenize{chapter_interview/CV:cv-vs-resume-5}}%
\begin{footnotetext}[841]\sphinxAtStartFootnote
\sphinxnolinkurl{https://www.zhihu.com/question/20355548}
%
\end{footnotetext}\ignorespaces \begin{itemize}
\item {} 
Resume,\sphinxstylestrong{简述}于求职相关的教育背景和工作经历,其目的在于说服用人单位雇用自己;

\item {} 
CV,Curriculum
Vitae事集中说明学术工作,不重视与文化程度和学习成绩无直接关系的资料,包括更多的证书、出版物、论文等成就;

\item {} 
Cover
letter是一种介绍自己的方式,去陈述一些简历无法完全展示的细节内容,来描述申请人的兴趣和资历。

\end{itemize}


\subparagraph{设计一份与众不同的简历 2\sphinxfootnotemark[842]}
\label{\detokenize{chapter_interview/CV:id1}}%
\begin{footnotetext}[842]\sphinxAtStartFootnote
\sphinxnolinkurl{http://www.woshipm.com/zhichang/4371937.html}
%
\end{footnotetext}\ignorespaces 
简历不是写出来的,而是设计出来的,介绍如何设计一份与众不同的简历、如何在众多模板简历中脱颖而出、如何让简历真的成为敲门砖、如何在1分钟内打动面试官、简历都有哪些投递的方式等。


\subparagraph{简历命名}
\label{\detokenize{chapter_interview/CV:id2}}
带职级的岗位(求职意向)、行业、工作年限

高级产品经理\sphinxhyphen{}隔壁老王\sphinxhyphen{}B端金融\sphinxhyphen{}6年。

从基本描述的已知信息中,获取到产品经理的个人标签,以便初步判断是否具有核心竞争力。
\sphinxhref{https://www.zhihu.com/search?type=content\&q=AI\%E4\%BA\%A7\%E5\%93\%81\%E7\%BB\%8F\%E7\%90\%86}{6}%
\begin{footnote}[843]\sphinxAtStartFootnote
\sphinxnolinkurl{https://www.zhihu.com/search?type=content\&q=AI\%E4\%BA\%A7\%E5\%93\%81\%E7\%BB\%8F\%E7\%90\%86}
%
\end{footnote}


\subparagraph{STAR}
\label{\detokenize{chapter_interview/CV:star}}
\begin{figure}[H]
\centering
\capstart

\noindent\sphinxincludegraphics{{STAR}.jpg}
\caption{STAR}\label{\detokenize{chapter_interview/CV:id8}}\end{figure}


\subparagraph{0\sphinxhyphen{}1创业}
\label{\detokenize{chapter_interview/CV:id3}}\begin{enumerate}
\sphinxsetlistlabels{\arabic}{enumi}{enumii}{}{.}%
\item {} 
好好想想你所学的技术可以应用在哪个行业的哪个环节,可以帮助这个行业或者是某一个企业降低成本、提高效率?

\item {} 
想象一下普通的用户是如何使用你的技术完成某项工作?

\item {} 
把想象到的东西全部写出来画出来,整理为流程图、结构图、原型图

\item {} 
着手开发吧,服务端和前端写的糙一点也没关系,实在不会的话就找几个接私活的,把你的需求和他们讲一遍

\item {} 
开发出来的东西肯定又难用又难看,找几个会说真话的朋友挑挑毛病,再改改

\item {} 
根据第一点,去找某个行业中的某个企业,推销自己的产品

\end{enumerate}
\begin{itemize}
\item {} 
卖出去了的话,你就可以开始集资创业了

\item {} 
卖不出去的话,你有了0\sphinxhyphen{}1的创业经验了

\item {} 
把1\sphinxhyphen{}5写出来,就是最好的简历

\end{itemize}


\subparagraph{简历框架}
\label{\detokenize{chapter_interview/CV:id4}}
\begin{figure}[H]
\centering
\capstart

\noindent\sphinxincludegraphics{{CV_frame}.png}
\caption{简历框架}\label{\detokenize{chapter_interview/CV:id9}}\end{figure}


\subparagraph{原则}
\label{\detokenize{chapter_interview/CV:id5}}\begin{itemize}
\item {} 
突出原则:条理清楚、重点突出、详略得当。优先级原则:根据企业岗位需求,个人经历,适当排序。

\item {} 
精简原则:把HR最想看的和自己最想展现的写出来。

\item {} 
数据原则:在内容上,能用数据表示的一就用数据表示。

\item {} 
细节原则:简历中不要出现错别字,不要忘写电话,不要写错邮箱,等等。

\item {} 
诚信原则:做人做事都要讲究一个诚信,简历亦是如此。\sphinxhref{https://t.qidianla.com/1165033.html}{7}%
\begin{footnote}[844]\sphinxAtStartFootnote
\sphinxnolinkurl{https://t.qidianla.com/1165033.html}
%
\end{footnote}

\end{itemize}


\subparagraph{问题 1\sphinxfootnotemark[845]}
\label{\detokenize{chapter_interview/CV:id6}}%
\begin{footnotetext}[845]\sphinxAtStartFootnote
\sphinxnolinkurl{http://www.woshipm.com/zhichang/4160330.html}
%
\end{footnotetext}\ignorespaces \begin{itemize}
\item {} 
简历过度包装

\item {} 
简历过分冗长:抓主要放次要,为什么做这个产品,这个产品解决了什么问题,你在其中扮演了什么角色,做出了哪些重要策略,算法模型存在哪些劣势。

\item {} 
内容含糊不清

\item {} 
数据敏感度差:数据指标的说明(包括基础指标和对比指标),并且自己对指标有深刻的合逻辑的理解。

\item {} 
借用他人的成果

\item {} 
不诚实:从业经验从社保缴纳记录、银行工资流水、个人完税证明等多个方面都可以佐证

\end{itemize}


\subparagraph{简历筛选与沟通}
\label{\detokenize{chapter_interview/CV:id7}}
作为面试官,会先快速阅读:相关经验、公司经历、教育背景等信息

相关经验按照 STAR
分析法,公司经历主要是否是互联网公司,是否一线互联网公司。如果有经历,一定要写,因为有的是结果导向,有的是过程导向差别很大。同时要关注候选人做过的事情,是面向互联网用户的还是面向公司内部用户的,因为面向互联网用户更关注用户体验这些细节部分,后者则更多强调通用性、拓展性,以及开发效率。

教育背景主要作为参考。

对候选人来说,自己的职责、做的事情、取得的成绩一定要写清楚,说白了就是目的性要强、重点要突出,要让面试官很快能把握到:你的公司做的产品是什么样的,模式是什么样的,你取得的成绩是业绩、指标提升了还是效率提高了,这样才能快速把握你的亮点。


\paragraph{金融产品经理的简历 1\sphinxfootnotemark[846]}
\label{\detokenize{chapter_interview/AI_Finance_CV:id1}}\label{\detokenize{chapter_interview/AI_Finance_CV::doc}}%
\begin{footnotetext}[846]\sphinxAtStartFootnote
\sphinxnolinkurl{https://www.zhihu.com/pub/reader/120098547/chapter/1321093149016838144}
%
\end{footnotetext}\ignorespaces 
金融产品经理是最近几年才出现的新职位,早期都是从金融机构内部选拔的人才,做出来的产品没有互联网化的感觉,很多时候在用户体验上只能说功能具备了,与大家经常用的微信、支付宝等产品的体验差距较大。不过随着更多互联网人才加入金融行业,这种情况得到了迅速解决,甚至现在到了反向的阶段,也就是要求加入的互联网产品经理必须要懂金融行业的知识。


\subparagraph{方向}
\label{\detokenize{chapter_interview/AI_Finance_CV:id2}}
第一种:懂金融业务的产品经理。首先这是一个产品经理职位,是负责金融业务的产品经理,要负责设计产品的前中后台,原则上这些都与金融业务相关。这种产品经理要能够非常理解金融业务。

第二种:懂产品的金融业务经理。首先这是一个业务经理职位,负责将各种金融产品落地。比如,在设计一个理财产品时,要知道\sphinxstylestrong{年化利率、最大购买金额、收益计算方式、赎回机制、赎回费用}等。这些都是非常专业的金融业务。同时,这个业务经理还要懂得如何展示这个产品,如何让用户更容易理解什么是年化利率、定投、起投金额、净值、债券等。这里的懂产品更像一种「翻译」工作,要把晦涩难懂的金融知识、金融产品更好地展示给用户。


\subparagraph{招聘要求}
\label{\detokenize{chapter_interview/AI_Finance_CV:id3}}
懂产品的金融业务经理,从工作范围上来看要负责银行、证券相关的业务规划,还要关注产品最终的销售情况,要求有
5
年以上的相关行业经验。这充分说明了这个职位更在意金融相关知识。这样的职位更要求能够讲清楚金融业务,对产品的功能设计的要求反而是次要的。

\begin{figure}[H]
\centering
\capstart

\noindent\sphinxincludegraphics{{finance_PM}.png}
\caption{金融产品经理}\label{\detokenize{chapter_interview/AI_Finance_CV:id4}}\end{figure}

一个懂金融业务的产品经理,要求对金融风控非常熟悉,对金融机构业务链路熟悉,这种专业知识真的不是想学就能马上学会的,并且就算学会了,这些知识还需要在实际工作中摸索运用,产品经理要不断地积累通过互联网产品实现金融业务逻辑的相关经验。但是同时我们可以看到在招聘要求中有些特别的要求,期望应聘者具备战略规划能力,有分析师或者管理咨询经验,最好在四大咨询公司工作过,最好能够有出色的
PPT 撰写及 Excel 建模能力。这都给这个职位的应聘者提出了更高要求。

\begin{figure}[H]
\centering
\capstart

\noindent\sphinxincludegraphics{{finance_Risk_control_PM}.png}
\caption{金融风控产品经理}\label{\detokenize{chapter_interview/AI_Finance_CV:id5}}\end{figure}

所以在简历中,我觉得你要更多地展示在\sphinxstylestrong{金融行业积累的经验,要能够把金融行业的基本流程、基本理念}写清楚,要让面试官非常容易认可你在金融领域的能力和经验。同时,你要告诉面试官你熟悉互联网产品的相关基础知识,我认为这就足够打动面试官了。


\paragraph{不能去的公司}
\label{\detokenize{chapter_interview/not_go:id1}}\label{\detokenize{chapter_interview/not_go::doc}}

\subparagraph{公司级别的划分标准}
\label{\detokenize{chapter_interview/not_go:id2}}
在业内并没有严格的公司级别的划分标准,一般来说,已上市的集团型公司为大型公司,如阿里巴巴、腾讯、百度、美团、小米、网易等;经过C
轮以上融资但未上市的独角兽公司为中型公司;经过C
轮以下融资或者未经过融资的公司为创业型公司,这类公司非常多。


\subparagraph{创业公司成长 4\sphinxfootnotemark[847]}
\label{\detokenize{chapter_interview/not_go:id3}}%
\begin{footnotetext}[847]\sphinxAtStartFootnote
\sphinxnolinkurl{https://coffee.pmcaff.com/article/2568729127965824/pmcaff?utm\_source=forum}
%
\end{footnotetext}\ignorespaces 
公司初创时期,寻找合伙人,这是第一阶段,要做的是“长心”,即寻找与你志同道合的朋友合伙,搭建公司核心的文化与价值观;有了合伙人之后,就要开始搭建初创团队,这是第二阶段,要做的是“搭骨架”,即寻找对你和公司有信念的员工,确保公司有效运转;等到公司发展壮大,需要招聘“牛人”,这是第三阶段,要做的是“长肉”,即进行团队迭代,让公司稳步发展


\subparagraph{隐患}
\label{\detokenize{chapter_interview/not_go:id4}}
大多数创业公司,一定一定会进入瓶颈期,这时各种隐患都会冒出来,比如加班多、薪酬不够多(还有猎头来挖人,一对比就。。)、流程制度不规范、士气不高、融资没结论等等。这时,就需要价值观、职业素养、公司凝聚力来救驾了


\subparagraph{靠谱的创业}
\label{\detokenize{chapter_interview/not_go:id5}}
很多靠谱的创业,在公司没有注册下来的时候,就已经开始有业务了;在装修的时候,就已经开始做销售有收入了;准备期过完,已经可以步入正轨了。
\sphinxhref{https://blog.csdn.net/liwei16611/article/details/100894158}{5}%
\begin{footnote}[848]\sphinxAtStartFootnote
\sphinxnolinkurl{https://blog.csdn.net/liwei16611/article/details/100894158}
%
\end{footnote}


\subparagraph{不能去的创业公司:}
\label{\detokenize{chapter_interview/not_go:id6}}\begin{enumerate}
\sphinxsetlistlabels{\arabic}{enumi}{enumii}{}{.}%
\item {} 
上一轮融资(IT桔子和36kr可知)已经烧差不多,自身盈利能力还没起来,需要靠下一轮融资继续经营。如果这时候融不到钱,就会大裁员,作为新员工、应届生,估计第一个被裁。

\item {} 
不好好面试的创业公司。创业公司的早期员工需要是精兵强将,否则养人的费用远高于员工的产出,本身就不盈利的话会死很快。所以从面试流程可以看出这家公司是不是认真的在招每一个人。

\item {} 
不签署劳工合同、不交五险一金的创业公司。合同是保护员工的,不签署对员工很不利,五险一金更不用说了。

\end{enumerate}


\subparagraph{老提当年}
\label{\detokenize{chapter_interview/not_go:id7}}
创始人实则是一个码人的角色,不要说自己是什么什么出身,那即代表你远离你的「出身」很久了,long
long ago 的 story 不必再提,好汉不提当年弱,更不要提当年勇。

码人,就是让一直擅长某一个领域的大神、大牛、Expert
在某一个公司管理部门发挥其价值。


\subparagraph{期权? 3\sphinxfootnotemark[849]}
\label{\detokenize{chapter_interview/not_go:id8}}%
\begin{footnotetext}[849]\sphinxAtStartFootnote
\sphinxnolinkurl{https://www.zhihu.com/pub/reader/119583028/chapter/1057335985750228992}
%
\end{footnotetext}\ignorespaces 
大多数初创公司的期权都是没有用的,具体情况如下。
\begin{enumerate}
\sphinxsetlistlabels{\alph}{enumi}{enumii}{}{.}%
\item {} 
初创公司画饼

\end{enumerate}

A 创业公司,招人时老板总谈降薪拿期权。当某员工降薪拿期权进入 A
公司,工作了一年多后,他发现公司的成长完全不及预期,失望之下,他试探了下外面的机会,发现同类型职位的薪资竟然是现公司的
2\textasciitilde{}3 倍。
\begin{enumerate}
\sphinxsetlistlabels{\alph}{enumi}{enumii}{}{.}%
\setcounter{enumi}{1}
\item {} 
成长型公司打土豪易、分田地难

\end{enumerate}

B 公司,国内风光一时的互联网上市公司,曾经期权是他们 offer
中重要的一环,不过,承诺最终难以兑现,随着原先许诺好的期权变成了 18:1
换股,早期员工也失去了憧憬。
\begin{enumerate}
\sphinxsetlistlabels{\alph}{enumi}{enumii}{}{.}%
\setcounter{enumi}{2}
\item {} 
上市公司的成熟分配机制

\end{enumerate}

以阿里、百度、奇虎等大公司为典型。期权被视为一种长期激励,而非利益捆绑。对优秀员工的激励方式是高薪资
+ 部分期权。大公司由于已上市,兑现相对容易。


\paragraph{笔试题}
\label{\detokenize{chapter_interview/exam:id1}}\label{\detokenize{chapter_interview/exam::doc}}

\subparagraph{题型}
\label{\detokenize{chapter_interview/exam:id2}}
行测题一般由以下几种题型组成:\sphinxhref{https://zhuanlan.zhihu.com/p/83786056}{1}%
\begin{footnote}[850]\sphinxAtStartFootnote
\sphinxnolinkurl{https://zhuanlan.zhihu.com/p/83786056}
%
\end{footnote}
\begin{enumerate}
\sphinxsetlistlabels{\arabic}{enumi}{enumii}{}{.}%
\item {} 
阅读理解

\item {} 
数据计算

\item {} 
逻辑推理

\item {} 
数字推理

\item {} 
图形推理

\end{enumerate}


\subparagraph{解题技巧}
\label{\detokenize{chapter_interview/exam:id3}}

\subparagraph{阅读理解题}
\label{\detokenize{chapter_interview/exam:id4}}
①
排除法,错误项一般都有以偏概全、无中生有、偷换概念、太过绝对、与题矛盾等错误;

②
还有一个技巧是“优先选首尾,优先选宏观”,指的是首尾部分往往是文段的主旨,答案越宏观越可靠。


\subparagraph{数据计算题}
\label{\detokenize{chapter_interview/exam:id5}}
增长量与增长率:增长量 = 现期量 \sphinxhyphen{} 基期量 增长率 = 增长量 / 基期量 *
100\%
同比与环比:同比是指和相同时期相比,比如与去年同月相比;环比指与相连的上一个统计周期相比,比如与上个季度相比。


\subparagraph{逻辑推理题}
\label{\detokenize{chapter_interview/exam:id6}}
第一步:写推理公式,(1) W → X (2) \sphinxhyphen{}T → \sphinxhyphen{}S (3) T → \sphinxhyphen{}X,X → \sphinxhyphen{}T (4) Y且Z →
W

第二步:写逆否公式,原命题成立则逆否命题必然成立,(1’) \sphinxhyphen{}X → \sphinxhyphen{}W (2’) S →
T (3’) X → \sphinxhyphen{}T,T → \sphinxhyphen{}X (4’) \sphinxhyphen{}W→ \sphinxhyphen{}Y或\sphinxhyphen{}Z

第三步:代入初始条件进行推导,现在接通S和Z,由(2’)式得T必然接通,由(3)式得X必然断开,由(1’)式得W必然断开,由(4’)式得Y和Z至少一个断开,得出\sphinxhyphen{}Y(注意:原命题和逆否命题等价,所以每个条件选一个用一次就行了)

第四步:得出结论,于是正确答案是S、T、\sphinxhyphen{}W、\sphinxhyphen{}X、\sphinxhyphen{}Y、Z,选A


\subparagraph{数字推理题}
\label{\detokenize{chapter_interview/exam:id7}}
① 基本数列:背下来!

等差数列:1,3,5,7,9 等比数列:2,4,8,16,32 质数列:2,3,5,7,11,13
平方数列:1,4,9,16,25 立方数列:1,8,27,64,125

②
含分数数列:遇到含分数的数列,可以尝试全部化为分数形式,尝试单独寻找分子分母数列的规律,例题可写成2/1,4/2,8/3,16/4,分子分母分别为等比和等差数列,可知正确答案为D

③
奇偶组合数列:数列的奇数项和偶数项分别是有规律的数列,需要分开找规律(这类题一般给的数比较多)

④
裂项数列:数列的项可以拆分,拆分出来的子数列呈一定规律,如2,12,30可以拆分为1×2,3×4,5×6,或2030600,1015150,1512180可以拆分为20×30=600,10×15=150,15×12=180(这类题一般给的项比较少)

⑤
递推数列:一般是由前两项推第三项,推导的方法各异(也有\sphinxstylestrong{前三项推第四项})

加减运算:1,3,4,7,11,18(c=a+b)

乘除运算:2,2,4,8,32,256(c=a×b)

组合运算:4,2,4,6,20,114(c=a×b\sphinxhyphen{}a)

⑥
孪生数列:数列连起来没有任何规律,一般是找两个两个间的规律,例如3,8,1,0,6,35,4,15,这个数列每两个一组就很容易看出规律,所以一般会给出7个数,求第8项

⑦
特殊数列:一些脑洞大开的数列,直接计算不出啥规律,如8324,669,4054,80173,每个数都有三个封闭空间,所以必须选符合这个规律的选项


\subparagraph{图形推理题}
\label{\detokenize{chapter_interview/exam:id8}}\begin{enumerate}
\sphinxsetlistlabels{\arabic}{enumi}{enumii}{}{.}%
\item {} 
对称轴规律:具有相同对称关系,对称轴数量、对称轴方向相同等

\item {} 
是否一笔画

\item {} 
布尔运算规律:两个图形叠加得第三个图形

\item {} 
旋转、对称、平移运动规律

\item {} 
数量规律:点(交点数、端点数)、线(线数、曲线数、直线数)、面(封闭域数)、角(角数)、素(每种图形的数量,图形种数)

\item {} 
接触规律:点接触、面接触

\end{enumerate}


\subparagraph{主观题解题技巧}
\label{\detokenize{chapter_interview/exam:id9}}

\subparagraph{需求分析法则}
\label{\detokenize{chapter_interview/exam:id10}}\begin{itemize}
\item {} 
市场调查部:负责玩家调查研究、对于市场游戏趋势开展研究。进行市场调查很需要资金,当然需要的资金投入越多获得的分析报告将更详尽,对公司的作用越大。

\item {} 
策划部:策划是游戏的总工程师,其作用不言而喻。没有资金的支持可不行。

\item {} 
程序部:游戏说到底也还是一个程序,程序研发的投入每家公司都很重视。

\item {} 
美术部:精美的场景、酷炫的人物和怪物、特有的光效对吸引玩家来说是多么重要。

\item {} 
技术部:服务器的采购、运维、防黑客、防病毒少得了资金投入吗?

\item {} 
测试部:没有游戏的测试,一大堆BUG的游戏是没有人喜欢的,对测试的投入不容忽视。
运营部:投放广告、游戏推广、营销策划、渠道管理等不投入怎么行?

\end{itemize}


\subparagraph{对比分析法则}
\label{\detokenize{chapter_interview/exam:id11}}
对比分析法则:往竞品上联想,和同类型产品对比、和另一类人作对比,对比一下你就知道该做啥了

为了扩大百度百家号的影响力,百家号布局短视频领域,为了帮助短视频创业者们在现实的竞争市场突围,推出了百万年薪计划,为有潜力的优质短视频创作者提供更好的流量扶持和品牌曝光,让更多优秀的短视频作者获得百万年薪。如何为百万年薪计划设置优质短视频评价体系?

解题指导:可以和最熟悉的大学老师做个对比。大牛教授(院士、长江),类比KOL,这些人得砸钱,不要看他创作了多少内容,而是名人效应;领基本工资的小讲师,类比新手作者,不能因为知名度低就一分不给,要有新人保护机制;一些科研冷门专业的老师,类比小众领域作者,他们创作内容虽然受众少,但是生命周期比蹭热点的内容长,长期来看也很有价值,为了防止他们分不到钱所以得把工作量考虑进去;学校也会按成果评title发工资,所以最终还是要建立起结果导向的评价体系,依据观看、点赞、评论、收藏等指标来发钱。


\subparagraph{场景模拟法则}
\label{\detokenize{chapter_interview/exam:id12}}
场景模拟法则:如果让你做一个东西,你就在脑海中模拟购买到使用的完整过程,从而发现用户需要

例题:给盲人设计一款闹钟

解题指导:盲人看不见但听得见,可以听收音机,所以可以做电台购物营销让盲人了解到你的产品;盲人使用时首先需要调时间,但是他看不见,所以要有语音播报功能;早上闹钟响后需要关掉,盲人要在黑暗中摸索,所以可以做个枕下闹钟,关闭按钮需要往大了做才能摸到;盲人视觉退化,听觉一般很灵敏,所以最好可以调节声音大小;盲人醒来后白天太寂寞了,可以加上电台功能,可以当收音机使用


\subparagraph{商业价值法则}
\label{\detokenize{chapter_interview/exam:id13}}
商业价值法则:出题人虽然没有要求你考虑变现,但是你要是能想到就是加分项

例题:给你100元在一个陌生城市生活7天,怎么规划?

解题指导:不能光想着合理分配,应该在能保证解决基本生活需求前提下,剩下的钱第一天就全部投出去挣钱,实现商业价值最大化,比如去批发市场买小玩意儿,然后去学校门口卖,挣第一桶金


\subparagraph{常用面试题 2\sphinxfootnotemark[851]}
\label{\detokenize{chapter_interview/exam:id14}}%
\begin{footnotetext}[851]\sphinxAtStartFootnote
\sphinxnolinkurl{https://zhuanlan.zhihu.com/p/23664545}
%
\end{footnotetext}\ignorespaces \begin{enumerate}
\sphinxsetlistlabels{\arabic}{enumi}{enumii}{}{.}%
\item {} 
考察笔试者对安卓规范掌握程度,4.0以前的规范还未成形,因此不那么重要

\item {} 
考察笔试者的收集分析能力,从这里可以看出应试者涉及产品领域的广度

\item {} 
考察笔试者对产品发展趋势的关注度与理解

\item {} 
了解笔试者是否随大流,是否有自己独特的兴趣

\item {} 
考察笔试者的用户需求挖掘能力,同时免费给面试官的视频项目搜集用户需求

\item {} 
考察面试者在以往的工作中对自己的产品是否有深入思考
是否有找到用户痛点,核心用户群,是否满足核心用户的需求

\item {} 
考察面试者对硬件的了解

\item {} 
考察笔试是否是面试者的真实成绩,同时,面试的双向交流比笔试的单向交流更能了解面试者的产品素质

\item {} 
考察面试者是否对自己常使用的产品有过思考

\item {} 
考察面试者对UI,交互,特色功能的掌握与理解

\item {} 
考察面试者对UI的理解(可能由于纯银的产品小清新居多,所以比较看重UI)

\item {} 
考察面试者的信息搜集能力,从侧面了解面试者在互联网行业所处层次
(如,连dribble都不去的设计师,设计水平不会有多高)

\item {} 
考察面试者是否对自己常使用的产品有过思考

\item {} 
考察面试者的数据分析能力

\item {} 
考察面试者的项目沟通与协作能力

\end{enumerate}


\paragraph{面试官 1\sphinxfootnotemark[852]}
\label{\detokenize{chapter_interview/interviewer:id1}}\label{\detokenize{chapter_interview/interviewer::doc}}%
\begin{footnotetext}[852]\sphinxAtStartFootnote
\sphinxnolinkurl{https://mp.weixin.qq.com/s?\_\_biz=MzIyOTAyOTEyNw==\&mid=2649631891\&idx=1\&sn=e5069d1c3e77ebd781da522ad787fb48\&chksm=f05268fbc725e1edc0987f8c94c5e497fd043177f819d0fe00aae3e6bf60423156ada713f83c\&mpshare=1\&scene=1\&srcid=0411XEbDZhF7chwQG50zMtJA\&key=52f65e2fc335f0816695259594ca021e3d2476a2cafaa96f3994e7588555cfa65895fc5e48257cb85115b1e2a25142ef955e698982337df732dbd52505ff6ffd2769e5aa3847377e4fb6a6594941e866\&ascene=0\&uin=OTYyNDg4NjIx\&devicetype=iMac+MacBookPro14\%2C1+OSX+OSX+10.12.5+build(16F2073)\&version=12020810\&nettype=WIFI\&lang=zh\_CN\&fontScale=100\&pass\_ticket=BRibOyqRAz6gRljQC9sbQ9pSXaaPwwqIN7vjp9uDpWetLencjvDMAKSRN\%2FIVeI4k}
%
\end{footnotetext}\ignorespaces 
如果也能在交流中搞清楚面试官的风格、偏好、真实需求,再决定说什么,怎么说,这就好比打听到了标底再投标,无往而不利。


\subparagraph{产品思维 4\sphinxfootnotemark[853]}
\label{\detokenize{chapter_interview/interviewer:id2}}%
\begin{footnotetext}[853]\sphinxAtStartFootnote
\sphinxnolinkurl{https://weread.qq.com/web/reader/46532b707210fc4f465d044kecc32f3013eccbc87e4b62e}
%
\end{footnotetext}\ignorespaces 
积极主动地把面试官当成“用户”,把简历当成“产品”,充分利用产品思维,你的面试过程就是产品运营推广的过程。在关键环节中获胜,在众多应聘者中脱颖而出。

从这个视角来看,你可以仔细审视一下你过去的一些选择是不是你的产品的加分项。

用户思维,要把面试官当成你的用户,要知道面试官认为什么样的人才是合格的产品经理,而不是自己认为自己是一个合格的产品经理。
\sphinxhref{https://weread.qq.com/web/reader/46532b707210fc4f465d044kb6d32b90216b6d767d2f0dc}{5}%
\begin{footnote}[854]\sphinxAtStartFootnote
\sphinxnolinkurl{https://weread.qq.com/web/reader/46532b707210fc4f465d044kb6d32b90216b6d767d2f0dc}
%
\end{footnote}


\subparagraph{真实需求}
\label{\detokenize{chapter_interview/interviewer:id3}}
想招个人,从github上把开源项目A抄下来,然后攥弄攥弄接到自己的数据库上,每天生成个报表页面给领导看。这个,才叫真实需求。知道了这个,你要做的,不是照上面那串词儿买一系列的《二十一天精通某某某》临时抱佛脚,而是去github上把A项目弄清楚,同时别忘了在自我介绍的时候说一句“我对A项目挺熟的”。


\subparagraph{不同公司的期望面试 6\sphinxfootnotemark[855]}
\label{\detokenize{chapter_interview/interviewer:id4}}%
\begin{footnotetext}[855]\sphinxAtStartFootnote
\sphinxnolinkurl{https://zhuanlan.zhihu.com/p/163236280s}
%
\end{footnotetext}\ignorespaces 
小公司创业公司往往面试4、5家就会发现有类似的套路,其核心是希望这个产品经理进来解决产品需求不清晰,产品设计实现执行慢,产品市场运营不理想等等问题的能力。

大公司路径比较成熟,看重行业经验的积累,看重某一个技能点的深度,和职业规划与大公司企业文化的匹配度。


\subparagraph{流水作业型}
\label{\detokenize{chapter_interview/interviewer:id5}}
有些面试官的风格,是简单直接的:一上来二话不说,扔过来一个问题让你自个儿想,他跑到一边继续闷头回邮件。

一般是业务繁忙的中层干部,为了快点儿搞定面试任务,设计了一条流水线,翻来覆去就那么几个问题,给个“Pass”或“No
Pass”了事。他根本不怎么跟你交流,怎么对付呢?其实在这种流水线里,多数问题都是套路活,在各大公司的面试题集锦里都能找到原形。所以,为了顺利兑付这样的面试,有目的的刷题,是必不可少的。

另外,这种风格的面试官,喜欢的是踏踏实实干活的人,所以在态度上,要表现得沉静乃至木讷,特别要流露出对频繁跳槽的厌倦,和对007式加班生活无限的憧憬。

还有最关键的一点,为了不让他觉得你“太行”,就算是这些题目你对答如流,也千万要做出冥思苦想的样子,而且要找个不太紧要的地方,卖个破绽,让面试官在居高临下的指摘和纠正中心满意足地写下“Pass”。


\subparagraph{职业经理型}
\label{\detokenize{chapter_interview/interviewer:id6}}

\subparagraph{骗方案}
\label{\detokenize{chapter_interview/interviewer:id7}}
网上搜了一些相关吐槽,发现互联网圈和非互联网圈都狠普遍。设计师被骗设计方案,市场被骗推广方案,产品经理被骗产品方案,被骗PRD,运营被骗运营方案。

我们产品圈流传着这样一个段子,当某个产品经理接到老板任务时候,不是需求分析,不是竞品分析,而是让HR找一些人来面试,面试时候把问题抛出来,看面试者怎么解决,一天面试下来,把方案汇总一下,就可以找老板汇报了。所以你会在拉钩或者boss直聘上看到,一个岗位一直挂那儿,总感觉没人去应聘的,长时间HR不处理简历,突然有一天喊你去面试了,面试完了就杳无音信了。哎,吃相真丑!

当然一些大公司也不例外,我就知道某大型电商平台也特别喜欢干这事。但是骗学生就有点过分了,无非是浪费人家劳动力。甚至某知名企业校园招聘时候,让面试者做企业宣传视频并传播,根据视频质量和视频的播放量决定复试名额,然而潜规则是,早就定好复试的人选,你视频做的好和不好跟你进复赛没啥关系,不过让学生做了一次免费推广罢了。


\subparagraph{骗方案的套路}
\label{\detokenize{chapter_interview/interviewer:id8}}\begin{enumerate}
\sphinxsetlistlabels{\arabic}{enumi}{enumii}{}{.}%
\item {} 
你自己的产品/运营方案,你觉得怎样?怎么修改?

\item {} 
整理一个完整的方案吧/刚刚那个需求,写个PRD发给我吧。

\item {} 
套信息

\item {} 
要以往的PRD/要以往的设计稿等等

\item {} 
HR电话催

\end{enumerate}


\subparagraph{给予面试者最基本的尊重}
\label{\detokenize{chapter_interview/interviewer:id9}}
一个免费的方案背后,是候选人的经验积淀,时间成本以及对工作的憧憬。


\subparagraph{我们要看到面试者的诚意}
\label{\detokenize{chapter_interview/interviewer:id10}}
如果拿方案来体现所谓的诚意,实在以偏概全了吧!面试者顶着烈日/冒雨前来面试不能体现诚意?打扮得体,一身正装过来面试不能体现诚意?面试前搜集了公司资料,对公司产品有所了解没有诚意?我作为产品经理,都投资了一把你们的理财产品没有诚意?偏偏是你要给我出方案,才是我的诚意,这是什么逻辑!

面试尊重是相互的,诚意也是相互的,我准时准点到了,面试官是不是也准时开始面试?我做了充分准备,调研了很多公司资料过来面试,面试官是不是面试前也仔细阅读了我的简历?我认认真真的面试,面试官是不是真诚和我沟通还是敷衍的过下场子而已呢?


\subparagraph{甄别,是不是骗方案}
\label{\detokenize{chapter_interview/interviewer:id11}}
\sphinxstylestrong{对于一些机密性的信息,尽量打马虎眼},这起码是对你上家公司的尊重,因为面试官不可能一上来问你一些很机密的东西,如果感觉有可能被骗了,不妨用一些假数据来迷惑一下,专业的运营经理会对你的假数据很敏感,多问几个就会找到破绽的,他们神情会表现出来的,而套信息的,基本都是记下来,然后不仔细问了,因为他们套信息的目的已经达到了。

尽可能的保护自己作品,他们问你要方案,如果你很想进这家公司,\sphinxstylestrong{尽量以自己做成PPT}下次我来当面演示给大家看。其实只要不是居心叵测的公司,他们很乐意这样,因为能够更具体了解面试者的实力。

你也可以拒绝,人家问为什么,机灵点,我和上家公司签了保密协议之类的,若对方原形毕露耍无赖非要你发,面试出来就可以把这家公司拉黑了。不到万不得已尽量不要撕破脸,毕竟一个圈子的,圈子很小,名声很重要。


\subparagraph{这个职位的理想人选是什么样的? 3\sphinxfootnotemark[856]}
\label{\detokenize{chapter_interview/interviewer:id12}}%
\begin{footnotetext}[856]\sphinxAtStartFootnote
\sphinxnolinkurl{https://www.yuque.com/weis/pm/up33vm}
%
\end{footnotetext}\ignorespaces 
我真的很喜欢这个问题,因为这个问题以一种全新的方式诠释了招聘方对你的期望。如果你的面试官可以凭空创造一个填补这个职位的理想人选,那会是什么样的人?有时候他们会觉得你就很理想,但有时候他们也会说一些和你的背景、技能或偏好都不相符的人选。你可以通过这种方式很好地了解你是否适合这家公司。


\subparagraph{你如何衡量开发团队、个人或公司成功与否?}
\label{\detokenize{chapter_interview/interviewer:id13}}
这又是一个过程问题。我想知道我的工作以及团队的工作会被如何评估。如果他们无法回答这个问题,那就换个思路,问他们是怎么判断自己做得不好的。在我看来,如果看不到做得好的地方,只能看到做错的地方,那这绝对是一个危险信号。如果你不知道成功是什么样子,那你怎么能成功呢?


\subparagraph{新员工的入职计划是什么?你该如何让新入职成员融入团队?}
\label{\detokenize{chapter_interview/interviewer:id14}}
除非你是初次求职,不然我认为这些问题应该放在最后。头一次找工作的人要了解重要的入职计划和培训计划。但老油条也可以通过这些问题了解他们的答案。我想知道他们会如何帮助新的开发人员开始工作。他们是否思考过如何让新入职的职员更容易地融入新公司?当然,如果公司没有这些计划也并无大碍,因为大部分公司都没有。

类似问题还有,他们是否会雇佣初级开发人员以及他们如何与这些开发人员共事,但问这个问题的前提是我们已经工作过一段时间了——我们并不是职场菜鸟。我已经工作快三年了,但我并不想给任何人任何建议。高级工程师可以提这个问题,他们不容易被误认为菜鸟。这样他们可以知道这家公司是如何看待员工价值的。


\paragraph{常见问题}
\label{\detokenize{chapter_interview/question:id1}}\label{\detokenize{chapter_interview/question::doc}}

\subparagraph{如何验证学习效果?}
\label{\detokenize{chapter_interview/question:id2}}
给自己编30道面试题,对着镜子模拟面试,直到能得满分为止。再换下面的30道,这样一轮轮下来,遍历自己的认知边界。发现边界太窄,就继续扩张边界。

面试AI团队,最忌讳对AI认知过浅。咱们换位思考,假设你是面试官,遇到的求职者讲不清朴素贝叶斯、SVM、NLP瓶颈边界、神经网络类型和原理,不能深入分析某个AI场景……公司又不是慈善机构,凭什么给他工作机会?


\subparagraph{专业能力}
\label{\detokenize{chapter_interview/question:id3}}
至于产品经理的专业能力,面试时一般从四个角度判断:
\begin{enumerate}
\sphinxsetlistlabels{\arabic}{enumi}{enumii}{}{.}%
\item {} 
批判性思维;

\item {} 
同理心(不要卖伤天害理的东西);

\item {} 
用户模型;

\item {} 
产品技能熟练度。

\end{enumerate}


\subparagraph{介绍一下你自己}
\label{\detokenize{chapter_interview/question:id4}}
介绍一下自己的姓名,年龄、毕业院校,工作经历。简单的介绍,保持在三分钟以内,给面试官问问题的时间。

工作经历主要讲一些你牛逼的工作经历,例如:你加入XX公司以后,销售额增加了多少、用户翻了多少倍…这样一些。有些人工作经历比较多,3年跳了好几家公司,建议你合并一下,不然面试官会觉得你这个人没有定力,在其他家公司干的时间都不长,在我公司能干多久?

至于你的毕业院校牛逼的肯定要说出来,如果觉得学校不好,不好意思说那就不说吧。

只能简单说几句,几年职业生涯一笔带过,自我介绍不超过1分钟的。说明这个候选人在描述一个事情上是说不好的,不懂得组织文字和语句,这样工作时往往\sphinxstylestrong{不能很好的表达自己的观点。}\sphinxhref{https://www.yuque.com/weis/pm/emr7ca}{10}%
\begin{footnote}[857]\sphinxAtStartFootnote
\sphinxnolinkurl{https://www.yuque.com/weis/pm/emr7ca}
%
\end{footnote}

讲的和简历上写的不一样,或者漏掉重要的经历。这种候选人需要特别注意,后面要多询问,如果不是口误或者忘了的话,简历很可能就是过度包装的。

别写流水帐,让这些经历更好地为这次求职背书。\sphinxhref{https://www.zhihu.com/collection/618263456}{23}%
\begin{footnote}[858]\sphinxAtStartFootnote
\sphinxnolinkurl{https://www.zhihu.com/collection/618263456}
%
\end{footnote}


\subparagraph{比例控制 22\sphinxfootnotemark[859]}
\label{\detokenize{chapter_interview/question:id5}}%
\begin{footnotetext}[859]\sphinxAtStartFootnote
\sphinxnolinkurl{http://dadaghp.com/index/index/article\_detail/id/670.html}
%
\end{footnotetext}\ignorespaces \begin{itemize}
\item {} 
应急毕业生(产助):可塑性35\%、沟通能力 25\%、逻辑能力 25\%、行业经历
5\%、决策能力 5\%、业绩体现 5\%;

\item {} 
1到5年产品经验:行业经历 20\%、业绩体现 20\%,沟通能力
15\%、逻辑能力15\%、决策能力 15\%、可塑性15\%;

\item {} 
5年以上产品经验:决策能力 35\%、业绩体现 30\%、行业经历 25\%、沟通能力
10\%、逻辑能力 0\%。

\end{itemize}


\subparagraph{我自己的介绍}
\label{\detokenize{chapter_interview/question:id6}}
我叫蔡舒起,年龄23,现求职一份金融AI产品经理岗位。
\begin{itemize}
\item {} 
金融方面:2020届的山西大学金融学专业的毕业生,获得三好学生学业奖学金,拥有会计、证券、基金从业证书,毕业论文是用Python的Statsmodel库里的VAR模型完成的《中美股市的联动性分析》,提出百度股价被低估能上300USD的想法,实现后。现百度股票主打AI第一股重新去香港上市。

\item {} 
AI方面:大一参加数学建模了解了神经网络算法,后在网络金融课分享了鱼书的ppt,得知金融方面的非结构信息常用深度学习算法,之后阅读过二十几本人工智能相关书籍,在20年用4个月自学完了《动手学深度学习》预览版并动手翻译MXNet到PyTorch的GAN:raw\sphinxhyphen{}latex:\sphinxtitleref{DCGAN},复现PyTorch、DJL等一系列深度学习移动端框架的Demo。

\item {} 
产品经理方面:在百度一面失败后,考虑到了百度开发者版的想法,与MXNet的开发者交流无人社区冷淡情况,学习十余本产品有关的书,并自行总结了AI产品经理的工作内容和技能要求,用Docker部署成《To
be AI PM》书。

\end{itemize}


\subparagraph{应届生 13\sphinxfootnotemark[860]}
\label{\detokenize{chapter_interview/question:id7}}%
\begin{footnotetext}[860]\sphinxAtStartFootnote
\sphinxnolinkurl{https://www.zhihu.com/question/57815929}
%
\end{footnotetext}\ignorespaces \begin{enumerate}
\sphinxsetlistlabels{\arabic}{enumi}{enumii}{}{.}%
\item {} 
对应届生做过的一件或几件重要的事深入询问,实习过的产品、参与过的项目、活动、学习、个人爱好等皆可。

\item {} 
看他对这件事的理解:各种选择的原因、各种收获或挫败的总结、各种问题的分析、做得好与不好的原因、其它可能性的判断等。

\item {} 
如果他能准确说清前因后果,差不多就有希望成长为合格 PM
了。如果还能有真知灼见,那么就有优秀 PM
的潜质,很可能他未来也能推动所负责的产品往前多走一步。

\end{enumerate}


\subparagraph{真正的选拔}
\label{\detokenize{chapter_interview/question:id8}}
我真正想听的是这种问题:
\begin{itemize}
\item {} 
你过往经历里方向背景是B端还是C端?

\item {} 
你设计的系统处理业务是属于电商类?社区类还是文娱?

\end{itemize}

这张图怎么用呢?举例来说如果一家公司要求学历为大专及以上,那么如果你在面试的自我介绍中\sphinxstylestrong{直截了当的告诉面试官我毕业于某某大学},那么面试官心中这里就直接在给你加上了学历匹配的5分。

\begin{figure}[H]
\centering
\capstart

\noindent\sphinxincludegraphics{{pipei}.png}
\caption{匹配阶梯}\label{\detokenize{chapter_interview/question:id19}}\end{figure}

根据上面六项做完匹配后,根据最终的得分我们可以将候选人直接分为这几类档:

\begin{figure}[H]
\centering
\capstart

\noindent\sphinxincludegraphics{{dang}.png}
\caption{候选人分档}\label{\detokenize{chapter_interview/question:id20}}\end{figure}

直观来说与其招到一个能把原型画到100分但是薪资相对于其他90分左右的产品提高了30\%,对于企业来说是没有多少意义的。

反之如果在高层级产品上选对了人,\sphinxstylestrong{他能在重大决策上有100分的判断},例如\sphinxstylestrong{在直播还在百花齐放的时代,有人先转型涉足了短视频领域做出了抖音,虽然他的薪资成本提高了30\%但是带来的意义是巨大的。}

因此在公司层面来说\sphinxstylestrong{对执行层人员的选择更多是选合适,而不是一定要非常优秀。}

这里我把这种现象称之为\sphinxstylestrong{低端饱和竞争}。

所以大家在后面的产品发展道路上除了要去磨练自己的一般性技能外,还要去主动学习更高层次的产品经理所要掌握的技能,\sphinxstylestrong{不要用战术上勤奋掩盖战略上懒惰。}


\subparagraph{战略}
\label{\detokenize{chapter_interview/question:id9}}

\subparagraph{一是围绕他过去做过的某件事(一般是某个产品或某个功能,也可以是其他)进行深入提问。}
\label{\detokenize{chapter_interview/question:id10}}\begin{itemize}
\item {} 
你开始为什么要做这个东西?

\item {} 
你是怎么想到的?

\item {} 
针对以后的情况你考虑了哪些方面和做了些什么?

\item {} 
开始做的过程中遇到了什么问题?

\item {} 
你是怎么解决的?发现了什么?

\item {} 
什么东西和原来理解的不一样了?

\item {} 
又发现了哪些相关的洞察,你和外部市场上的其他人对其认知不一样?

\item {} 
在迭代过程中又是怎么做的?

\item {} 
你现在对这件事情的认知有什么变化?

\item {} 
如果再回到当初,你还会做什么?

\end{itemize}

TODO: \sphinxurl{https://www.infoq.cn/article/kkfAX67GAlsRSP65JvBy}


\subparagraph{为什么想做产品经理 2\sphinxfootnotemark[861]}
\label{\detokenize{chapter_interview/question:id11}}%
\begin{footnotetext}[861]\sphinxAtStartFootnote
\sphinxnolinkurl{http://www.woshipm.com/zhichang/315041.html}
%
\end{footnotetext}\ignorespaces \begin{itemize}
\item {} 
兴趣爱好:我热爱AI,相信AI的认知(DL)与决策能力(RL)能极大提升生产力,帮助人们,比如我了解到美团用语音机器人联系人员复工,解放了很多重复的电话员的工作。而且更关心技术实际的社会经济作用,也就是“场景驱动技术”,去关注最需要的业务,技术研发不是为了炫技也没必要一味拿着钉子找榔头。

\item {} 
能力匹配:我个人有提到百度开发版的主意,这个想法来自于我身为程序员toC用户本身想法而来,我也了解到就算百度员工也用谷歌。那不管从个人效率,公司战略,乃至国家安全角度,这个产品都是非常重要的

\item {} 
职业规划\sphinxhref{https://www.bilibili.com/video/BV1UK4y1f7kK?from=search\&seid=11977543152973696126}{17}%
\begin{footnote}[862]\sphinxAtStartFootnote
\sphinxnolinkurl{https://www.bilibili.com/video/BV1UK4y1f7kK?from=search\&seid=11977543152973696126}
%
\end{footnote}:目前近三五年是想在产品经理岗上将AI技术赋能给更多企业与个人,尤其是金融和体育方向,这都是我比较热爱和了解的方向。

\end{itemize}


\subparagraph{你用了我们的产品么?对我们的产品有啥建议?}
\label{\detokenize{chapter_interview/question:id12}}
正确回答:(先吹嘘一番)公司做的这个业务市场规模很大,很有前景,而且我们公司是做的比较好的,产品体验也不错,尤其XX地方,设计的很好,用户体验很棒,但个人认为在一些细节上还有优化的空间,XX功能如果XX做的话会更好一些。

“你对我们公司的产品有什么看法?”一个合格的产品经理会\sphinxstylestrong{避其锋芒},委婉地表达对公司的敬意、对行业的熟悉、对竞品的研究,而不会感觉终于有了一个当面吐槽的机会,一股脑地说出很多意见或者建议,哪怕这些是正确的,也会让别人觉得不舒服。你在外界看到的信息,在公司内部的员工肯定已经看过了,并且研究过了,因为这是他们的工作,所以他们不太可能认可你的意见。因此,你千万要沉住气,要保持合理的谦虚。
\sphinxhref{https://weread.qq.com/web/reader/46532b707210fc4f465d044kb6d32b90216b6d767d2f0dc}{12}%
\begin{footnote}[863]\sphinxAtStartFootnote
\sphinxnolinkurl{https://weread.qq.com/web/reader/46532b707210fc4f465d044kb6d32b90216b6d767d2f0dc}
%
\end{footnote}


\subparagraph{谈谈竞品?}
\label{\detokenize{chapter_interview/question:id13}}
既然是产品面试,在已知所在岗位要面向的产品时,竞品分析报告必须得提前做好!这里我也是早就做了准备,主要是针对腾讯云/百度云/阿里云做了个分析(产品功能/推广/运营模式/商业化)。

分析比较两款同类产品的优劣/特点,或是让你下载他们公司的产品说一下意见和使用感受,这点可以在接到面试通知的时候就在网上查阅了解下,然后提前准备好。
\sphinxhref{https://www.zhihu.com/search?type=content\&q=\%E4\%BA\%A7\%E5\%93\%81+\%E7\%AC\%94\%E8\%AF\%95}{16}%
\begin{footnote}[864]\sphinxAtStartFootnote
\sphinxnolinkurl{https://www.zhihu.com/search?type=content\&q=\%E4\%BA\%A7\%E5\%93\%81+\%E7\%AC\%94\%E8\%AF\%95}
%
\end{footnote}


\subparagraph{你最喜欢的产品是什么?3\sphinxfootnotemark[865]}
\label{\detokenize{chapter_interview/question:id14}}%
\begin{footnotetext}[865]\sphinxAtStartFootnote
\sphinxnolinkurl{http://www.woshipm.com/pmd/2891945.html}
%
\end{footnotetext}\ignorespaces 
用户体验的五个层次出发,显得有条理:战略、范围、结构、框架、表现

比如从战略层,产品的核心是解决了什么用户痛点,这个市场规模能有多大。那么你回答的信息整合和用户调性,本质上都是它的产品定位以及配套的业务范围,也许是整个团队需要努力保持的核心竞争力。

结构方面,关注流和推荐流的层级设计,比如抖音是推荐主导,微博是关注主导,也能体现产品设计的侧重点。

构架和表现层,其实移动互联网时代的交互已经很久没有大的变化了,所以大多数正常产品挑不出大毛病,偶尔有一些细节彩蛋就会让人很惊喜,比如你说的blabla……”


\subparagraph{问清题目4\sphinxfootnotemark[866]}
\label{\detokenize{chapter_interview/question:id15}}%
\begin{footnotetext}[866]\sphinxAtStartFootnote
\sphinxnolinkurl{https://zhuanlan.zhihu.com/p/108911948\#\%E4\%B8\%80\%E4\%B8\%AA\%E9\%9D\%9E\%E5\%B8\%B8\%E7\%AE\%80\%E5\%8D\%95\%E7\%9A\%84\%E4\%BE\%8B\%E5\%AD\%90}
%
\end{footnotetext}\ignorespaces 
许多面试官在面试的时候,会故意先抛出一个模糊的问题。实际上,他们希望面试者能够经过一些询问理解问题。在这个过程中,面试者能够展现出自己对问题的分析能力以及沟通的能力。前者的重要性参见编程珠玑第一章:明确问题,战役就成功了90\%。后者的重要性在于,问清题目的这个交流过程与面试者入职之后与同事讨论问题的形式非常类似。显而易见,一个能够很难沟通的面试者也很难成为一个很好沟通的同事。


\subparagraph{面试的坑 21\sphinxfootnotemark[867]}
\label{\detokenize{chapter_interview/question:id16}}%
\begin{footnotetext}[867]\sphinxAtStartFootnote
\sphinxnolinkurl{https://t.qidianla.com/934094.html}
%
\end{footnotetext}\ignorespaces 
虽然我们是小白,在很多问题的想法和见解上都不成熟,但只要不是特别离谱的,当面试官质疑的时候,都不要那么快推翻自己的看法,这时候考验的是产品经理的抗压能力和资源争取能力,太容易妥协毫无自己想法很快就掉进面试官的挖的坑里了。


\subparagraph{To B的产品,跟To c 的产品,在设计产品过程中有什么不同?5\sphinxfootnotemark[868]}
\label{\detokenize{chapter_interview/question:to-b-to-c-5}}%
\begin{footnotetext}[868]\sphinxAtStartFootnote
\sphinxnolinkurl{https://zhuanlan.zhihu.com/p/33524676}
%
\end{footnotetext}\ignorespaces 
toB
公司产品要保证的是交付给客户的解决方案的结果和质量。\sphinxhref{http://www.ramywu.com/work/2018/04/09/Get-Ready-For-AI-PM/}{8}%
\begin{footnote}[869]\sphinxAtStartFootnote
\sphinxnolinkurl{http://www.ramywu.com/work/2018/04/09/Get-Ready-For-AI-PM/}
%
\end{footnote}所以
AI PM
在推动产品落地的过程中,需要各种团队协作、跨部门沟通、向上汇报等,因为工作目标是交付项目,在项目管理本身有很多蛮内隐知识在里面够学的了。
对2C平台来说,每个阶段的侧重点是不同的,前期更注重日活,后期看GMV。\sphinxhref{https://m.k.sohu.com/d/495625828?channelId=1\&page=1}{7}%
\begin{footnote}[870]\sphinxAtStartFootnote
\sphinxnolinkurl{https://m.k.sohu.com/d/495625828?channelId=1\&page=1}
%
\end{footnote}


\subparagraph{常见问题 15\sphinxfootnotemark[871]}
\label{\detokenize{chapter_interview/question:id17}}%
\begin{footnotetext}[871]\sphinxAtStartFootnote
\sphinxnolinkurl{https://zhuanlan.zhihu.com/p/350981809}
%
\end{footnotetext}\ignorespaces \begin{itemize}
\item {} 
请描述下你和客户沟通的技巧

\item {} 
请描述一个和客户沟通后,对产品理解加深的例子

\item {} 
怎么判断研发给你的排期是否合理

\item {} 
请描述下你之前负责的一个产品的产品规划

\item {} 
你目前所在的行业的市场规模有多大?测算逻辑是什么?

\item {} 
你们公司目前的市场占有率是多大?测算逻辑是什么?

\item {} 
产品做前期市场推广时,如何找高价值客户的?

\item {} 
请简述下你对行业未来发展趋势的看法

\item {} 
你目前负责产品的人均产出是多少?和公司其他产品比较呢?

\item {} 
你对XX算法的了解有多少?可以简单介绍下这方面的技术

\item {} 
你负责产品的主要竞品有哪些?主要优劣势?

\item {} 
请描述一个你成功/失败的项目和原因分析

\end{itemize}


\subparagraph{自提:}
\label{\detokenize{chapter_interview/question:id18}}\begin{itemize}
\item {} 
为何前面有快消产品经理所注重的品牌产品经理的内容?

\item {} 
什么时候能够得到反馈?

\item {} 
如能入职贵司,您希望在三个月,半年内甚至一年内,对我工作成绩有怎么期望?
\sphinxhref{https://zhuanlan.zhihu.com/p/33395387}{14}%
\begin{footnote}[872]\sphinxAtStartFootnote
\sphinxnolinkurl{https://zhuanlan.zhihu.com/p/33395387}
%
\end{footnote}

\item {} 
Axure卡死的情况?比如点了代理之后代理界面在缩略图的弹出来但实际不显示,只能退出。。

\end{itemize}

业务方面:\sphinxhref{https://shimo.im/docs/vyCrK3rQQ6KC9Ryp/read}{18}%
\begin{footnote}[873]\sphinxAtStartFootnote
\sphinxnolinkurl{https://shimo.im/docs/vyCrK3rQQ6KC9Ryp/read}
%
\end{footnote}
\begin{itemize}
\item {} 
工作的具体内容是什么,公司对这个岗位的定位是什么?(JD的文字一般都不多,何况还会有些包装的情况,所以了解清楚具体要做的工作内容非常重要,避免入职后才发现与预期不符)

\item {} 
成为你心目中理想的产品经理候选人需要具备什么条件?

\item {} 
产品去年的数据表现如何,如果是新产品,是谁牵着立项的,\sphinxstylestrong{目前的进展如何了}

\item {} 
其他对业务好奇的问题

\end{itemize}

团队方面:
\begin{itemize}
\item {} 
要了解目标岗位因何设立,是团队扩张新设的岗位,还是之前的人离职了,前面的人为什么离职

\item {} 
汇报线是什么样的

\item {} 
\sphinxstylestrong{整个团队有多少人,产品团队有多少人,大概有哪些分工,未来会达到什么样的规模}(一面的时候问面试官部门和具体业务,二面前进行充分准备和了解呐。\sphinxhref{https://zhuanlan.zhihu.com/p/87293782}{19}%
\begin{footnote}[874]\sphinxAtStartFootnote
\sphinxnolinkurl{https://zhuanlan.zhihu.com/p/87293782}
%
\end{footnote})

\end{itemize}

leader方面:
\begin{itemize}
\item {} 
目前负责的工作内容

\item {} 
\sphinxstylestrong{管理风格以及对下属的期待}

\item {} 
历史成绩和择业原因

\item {} 
其他对leader好奇的内容

\item {} 
你觉得在这里做产品经理最有挑战性的是什么?\sphinxhref{https://www.jianshu.com/p/09e7ae756bc6}{20}%
\begin{footnote}[875]\sphinxAtStartFootnote
\sphinxnolinkurl{https://www.jianshu.com/p/09e7ae756bc6}
%
\end{footnote}

\end{itemize}


\paragraph{模拟面试}
\label{\detokenize{chapter_interview/simulate_interview:id1}}\label{\detokenize{chapter_interview/simulate_interview::doc}}
TODO:https://t.qidianla.com/1165227.html

TODO: \sphinxurl{https://t.qidianla.com/author/qdxy-pmjy}

面试流程:专业面→跨岗位面→BOSS面→HR面
\sphinxhref{https://t.qidianla.com/1165227.html}{1}%
\begin{footnote}[876]\sphinxAtStartFootnote
\sphinxnolinkurl{https://t.qidianla.com/1165227.html}
%
\end{footnote}

\begin{figure}[H]
\centering
\capstart

\noindent\sphinxincludegraphics{{ability_test}.png}
\caption{能力测试}\label{\detokenize{chapter_interview/simulate_interview:id13}}\end{figure}


\subparagraph{电话面}
\label{\detokenize{chapter_interview/simulate_interview:id2}}
对方对你做一个初步的了解,根据你的沟通交流、表达来看下逻辑思维能力如何。


\subparagraph{专业面}
\label{\detokenize{chapter_interview/simulate_interview:id3}}
针对你的简历提问


\subparagraph{个人介绍}
\label{\detokenize{chapter_interview/simulate_interview:id4}}

\subparagraph{我自己的介绍}
\label{\detokenize{chapter_interview/simulate_interview:id5}}
我叫蔡舒起,年龄23,现求职一份金融AI产品经理岗位。
\begin{itemize}
\item {} 
金融方面:2020届的山西大学金融学专业的毕业生,获得三好学生学业奖学金,拥有会计、证券、基金从业证书,毕业论文是用Python的Statsmodel库里的VAR模型完成的《中美股市的联动性分析》,提出百度股价被低估能上300USD的想法,实现后。现注意到百度股票主打AI第一股重新去香港上市。ToB和ToG业务强调服务,要随时响应客户需求,我的金融背景能快速学会用客户的语言进行业务沟通。\sphinxhref{https://cloud.ofweek.com/news/2020-11/ART-178801-8500-30469911.html}{8}%
\begin{footnote}[877]\sphinxAtStartFootnote
\sphinxnolinkurl{https://cloud.ofweek.com/news/2020-11/ART-178801-8500-30469911.html}
%
\end{footnote}

\item {} 
AI方面:我的整个经历,大一参加数学建模了解了神经网络算法,后在网络金融课分享了鱼书的ppt,得知金融方面的非结构信息常用深度学习算法,之后阅读过二十几本人工智能相关书籍,在20年用4个月自学完了《动手学深度学习》预览版并动手翻译MXNet到PyTorch的GAN:raw\sphinxhyphen{}latex:\sphinxtitleref{DCGAN},复现PyTorch、DJL等一系列深度学习移动端框架的Demo。

\item {} 
产品经理方面:虽然没有工作经历,但还是努力修炼了一系列的能力。战略能力上,在百度一面失败后,考虑到了百度开发者版的想法。沟通能力上,与MXNet的开发者交流无人社区冷淡情况。学习上,学习十余本产品有关的书,并自行总结了AI产品经理的工作内容和技能要求,用Docker部署成《To
be AI PM》书。

\end{itemize}


\subparagraph{百度aistudio面试}
\label{\detokenize{chapter_interview/simulate_interview:aistudio}}
当时是问了我常用的框架,我说PyTorch,一个是学术开源的多,一个是官方论坛讨论的多。后来我反思应该把这种关系说成飞轮效应,相互促进。

后面接连问了几个问题,我当时没反应和记忆过来,后来发现应该是考虑需求收集的能力,应该备纸笔来辅助记忆。

后面面试官提到应该做些项目来锻炼我的产品思维,并给我推荐了《产品方法论》。

\sphinxurl{https://github.com/StevenJokess/d2l-en-read/blob/moreme/chapter-generative-adversarial-networks/aistudio-job.md}


\subparagraph{说服szha开放评论区}
\label{\detokenize{chapter_interview/simulate_interview:szha}}
\sphinxurl{https://github.com/apache/incubator-mxnet/issues/18931}

\begin{figure}[H]
\centering
\capstart

\noindent\sphinxincludegraphics{{MXNet_forum_questions}.png}
\caption{评论区的问题}\label{\detokenize{chapter_interview/simulate_interview:id14}}\end{figure}

1、摆数据

我会拿出相关数据来证明改动的必要性,数据可以是页面的点击数、用户的跳出率等。

截了PyTorch论坛的图,肉眼可见地他们的论坛运营很活跃,几乎每个问题都有人回答。几乎每个问题都有人回答,而对比MXNet的论坛,几个月都没几个问题,而且问题几乎没人回答都烂尾的。

2、讲事实

其次,我会讲事实。在数据论证的基础上陈述项目规划目标和领导预期,并结合所要改动的功能点进行说明,论证修改后会获得双方共赢的良好效果。

3、适当妥协

我会理解技术拒绝更改的原因,然后在合理的范围内做适当的妥协,如在不影响大局的前提下,适当的延长项目时间等,关键是项目有所输出并保证质量。

4、向上沟通

最后,向上沟通,如果基础的沟通无效,不妨将双方的leader加入到沟通的过程中,通过邮件或者是微信等一些实时沟通的方式,条理清晰的陈述己方的观点,并让leader做最后评判。

后来加了他的微信,说了两个关键点:一个是保证问题解决,一个是时间要尽快。

5、复盘

在处理本次冲突的基础上,还是要客观的分析冲突产生的原因是由于沟通机制存在问题,还是由于双方的沟通方式没有磨合好,PM需要根据实际情况提出积极的解决方案,提前规避下次的冲突产生。在很多同类型的沟通问题中也确实存在由于PM的思考不严谨,需求频繁更改的情况,如果确实是此类原因导致的沟通不顺畅,那么及时反思和承认错误也是必要的,没有一成不变的沟通方式,良好的沟通应该是在双方的理解和信任的基础上不断完善的。”

后来加了他的微信,说了两个关键点:一个是保证问题解决,一个是时间要尽快。

不过一切StevenJokes的记录,都因为一次情绪崩溃而丧失。


\subparagraph{心态崩溃的原因}
\label{\detokenize{chapter_interview/simulate_interview:id6}}\begin{itemize}
\item {} 
校招社招两头为难:由于疫情没能实习和春招,错过应届生省份,又想转行AI。

\item {} 
家里催的急引发的稀缺心态:由于家里经济条件不好会导致大脑的注意力被稀缺资源俘获。过度关注当前利益而无法考虑长远利益,使得过于抒发了负面zs情绪\sphinxhref{https://www.zhihu.com/question/20791021/answer/652756690}{2}%
\begin{footnote}[878]\sphinxAtStartFootnote
\sphinxnolinkurl{https://www.zhihu.com/question/20791021/answer/652756690}
%
\end{footnote}

\item {} 
感觉未来就业无望:由于并非名校,且金融专业尤其看学历和背景。

\end{itemize}

所以,心态崩溃根本原因是对人生未来的迷茫无助、家庭的经济能力差导致的,如果有一份较为稳定光明的工作,比如这份工作,我的\sphinxstylestrong{情绪控制能力}还是可以的,毕竟,没几个能在家里承受压力憋将近一年去修炼技能的。


\subparagraph{GAN、DCGAN}
\label{\detokenize{chapter_interview/simulate_interview:gandcgan}}\begin{itemize}
\item {} 
\sphinxurl{https://preview.d2l.ai/d2l-en/master/chapter\_generative-adversarial-networks/gan.html}

\item {} 
\sphinxurl{https://preview.d2l.ai/d2l-en/master/chapter\_generative-adversarial-networks/dcgan.html}

\end{itemize}

部分测试:

GAN:
\begin{itemize}
\item {} 
\sphinxurl{https://github.com/StevenJokess/d2l-en-read/blob/moreme/chapter-generative-adversarial-networks/my5gan.ipynb}

\item {} 
\sphinxurl{https://github.com/StevenJokess/d2l-en-read/blob/moreme/chapter-generative-adversarial-networks/D\_block\_ok.ipynb}

\item {} 
\sphinxurl{https://github.com/StevenJokess/d2l-en-read/blob/moreme/chapter-generative-adversarial-networks/G\_block1.ipynb}

\item {} 
\sphinxurl{https://github.com/StevenJokess/d2l-en-read/blob/moreme/chapter-generative-adversarial-networks/gan\_torch.ipynb}

\end{itemize}

DCGAN:
\begin{itemize}
\item {} 
\sphinxurl{https://github.com/StevenJokess/d2l-en-read/blob/moreme/chapter-generative-adversarial-networks/dcgan\_torch5.ipynb}

\item {} 
\sphinxurl{https://github.com/StevenJokess/d2l-en-read/blob/moreme/chapter-generative-adversarial-networks/dcgan\_train1.ipynb}

\item {} 
\sphinxurl{https://github.com/StevenJokess/d2l-en-read/blob/moreme/chapter-generative-adversarial-networks/dcgan\_pt3.ipynb}

\item {} 
\sphinxurl{https://github.com/StevenJokess/d2l-en-read/blob/moreme/chapter-generative-adversarial-networks/dcgan\_n2train.ipynb}

\item {} 
\sphinxurl{https://github.com/StevenJokess/d2l-en-read/blob/moreme/chapter-generative-adversarial-networks/colab\_pt\_DCGAN/DCGAN\_ptlr5.ipynb}

\end{itemize}

之前测试的:
\begin{itemize}
\item {} 
\sphinxurl{https://github.com/StevenJokess/d2l-en-read/blob/moreme/Ch17\_GAN/17-1pt.ipynb}

\item {} 
\sphinxurl{https://github.com/StevenJokess/d2l-en-read/blob/moreme/Ch17\_GAN/17-2pt.ipynb}

\end{itemize}

虽然我的commit记录没了,但我的记录就是原\#@tab pytorch里的内容
\begin{itemize}
\item {} 
\sphinxurl{https://github.com/d2l-ai/d2l-en/pull/1400}

\item {} 
\sphinxurl{https://github.com/d2l-ai/d2l-en/pull/1422}

\end{itemize}

我解释情绪与mli、astonzhang沟通:\sphinxurl{https://github.com/d2l-ai/d2l-en/pull/1449}


\subparagraph{为什么选择这一行}
\label{\detokenize{chapter_interview/simulate_interview:id7}}
我即是产品,一个好的产品是“要、能、赚!”
\begin{itemize}
\item {} 
社会需要:对比大量的金融从业人员,真的AI产品经理是价值很大且稀缺的,AI产品由于代码、模型的分发边际成本低,可能影响上亿的人。

\item {} 
能力胜任:金融行业知识(更好与开发交流开发背景)、编程AI知识(更好协调开发)、产品经理知识(做好项目管理)。

\item {} 
回报丰厚:物质生活上,有稳定较为丰厚的现金流来维持生活;精神生活上,能满足我对于AI本身的向往追求,实现自我价值。

\end{itemize}


\subparagraph{职业规划}
\label{\detokenize{chapter_interview/simulate_interview:id8}}\begin{itemize}
\item {} 
短期:把眼前的事情做得足够的扎实,修炼产品经理的技能。

\item {} 
中短期(3\sphinxhyphen{}5年):职位从产品经理到高级产品经理(对\sphinxstylestrong{金融智能投研业务}有通盘的了解)最好能到产品总监(整合各方面的资源,在商业上帮助公司)的位置。

\item {} 
中长期(7+年):在中短期的人脉行业积累下,锻炼嗅觉,去做投资,帮助更多企业和青年人成长。

\end{itemize}


\subparagraph{对小冰的认知}
\label{\detokenize{chapter_interview/simulate_interview:id9}}\begin{itemize}
\item {} 
chatbot:与C端用户聊天,这一点考虑到情感是不做商业化。

\item {} 
艺术教育:和有声读物去版权合作、画家6个虚拟人格办画展、小冰机智过人里根据诗词作曲。

\end{itemize}

\sphinxstylestrong{基于深度学习的自然语言生成}在金融领域的应用
\begin{itemize}
\item {} 
智能投研:在金融领域,小冰是目前\sphinxstylestrong{全球范围内规模第一的金融文本摘要生成}平台。去试图保证了稳定性、准确性以及时效性,特别是上市公告高并发、非密集的时候可以大大减压。万得资讯
ToB

\item {} 
金融风控:基于小冰人工智能技术生成的文本、大数据金融知识图谱来做风控。

\item {} 
金融资讯:实时翻译来提供中英双语AI金融资讯,每日经济新闻、华尔街见闻。ToC

\end{itemize}

业务扩展:短、长摘要;金融研报;垂直搜索;知识图谱;全景可视化

\begin{figure}[H]
\centering
\capstart

\noindent\sphinxincludegraphics{{Fin_xiaoice}.png}
\caption{微软小冰在金融\sphinxhref{https://www.youtube.com/watch?v=vu6YmD5TX7E}{7}\sphinxfootnotemark[879]}\label{\detokenize{chapter_interview/simulate_interview:id15}}\end{figure}
%
\begin{footnotetext}[879]\sphinxAtStartFootnote
\sphinxnolinkurl{https://www.youtube.com/watch?v=vu6YmD5TX7E}
%
\end{footnotetext}\ignorespaces 
\sphinxurl{https://github.com/StevenJokess/2bPM/blob/master/chapter\_company/xiaoice.md}


\subparagraph{准备提问}
\label{\detokenize{chapter_interview/simulate_interview:id10}}

\subparagraph{官网的问题}
\label{\detokenize{chapter_interview/simulate_interview:id11}}
\sphinxurl{https://e.xiaoice.com/Home?r=\%2F}
\begin{itemize}
\item {} 
为何不是HTTPS且自动转向HTTPS?安全需求

\item {} 
初次领养时,验证码错误不提示,反而在登录才会出现?异常流程的问题。

\item {} 
直接登录也可以手机验证码领养,那为何要多个领养的注册界面标签?多余流程的问题

\item {} 
名字不能超过10个字,为啥不早提示?异常流程没有提前告诉。

\item {} 
可以用“………………………………”做名字?如果传唤怎么传?可用性需求

\item {} 
QQ手机APP,搜索并关注公众号“小冰”(ID:ms\sphinxhyphen{}xiaoice)

\item {} 
在微信中关注公众号“小冰”(ID: xiaoice\sphinxhyphen{}ms),激活小冰。

\item {} 
我是古灵精怪的人工智能少女小冰,我来自微软。\sphinxhref{https://www.yeelight.com/zh\_CN/product/donut}{6}%
\begin{footnote}[880]\sphinxAtStartFootnote
\sphinxnolinkurl{https://www.yeelight.com/zh\_CN/product/donut}
%
\end{footnote}

\item {} 
哄情绪低落的人类开心起来,来避免冲动投资与撤资。\sphinxhref{https://www.bilibili.com/video/av796273602/}{5}%
\begin{footnote}[881]\sphinxAtStartFootnote
\sphinxnolinkurl{https://www.bilibili.com/video/av796273602/}
%
\end{footnote}

\end{itemize}

\begin{figure}[H]
\centering
\capstart

\noindent\sphinxincludegraphics{{xiaoice_zhaopin_issue}.jpg}
\caption{小冰Moka招聘界面\sphinxhref{https://app.mokahr.com/m/candidate/applications/deliver-query/xiaobing}{4}\sphinxfootnotemark[882]}\label{\detokenize{chapter_interview/simulate_interview:id16}}\end{figure}
%
\begin{footnotetext}[882]\sphinxAtStartFootnote
\sphinxnolinkurl{https://app.mokahr.com/m/candidate/applications/deliver-query/xiaobing}
%
\end{footnotetext}\ignorespaces 

\subparagraph{小冰框架 5\sphinxfootnotemark[883]}
\label{\detokenize{chapter_interview/simulate_interview:id12}}%
\begin{footnotetext}[883]\sphinxAtStartFootnote
\sphinxnolinkurl{https://www.bilibili.com/video/av796273602/}
%
\end{footnotetext}\ignorespaces \begin{itemize}
\item {} 
登录:要先注册不明显

\item {} 
注册必须要密码,而登录却可以只靠验证码

\item {} 
没有更改手机

\item {} 
注册界面
\sphinxhref{https://www.yeelight.com/zh\_CN/product/donut}{6}%
\begin{footnote}[884]\sphinxAtStartFootnote
\sphinxnolinkurl{https://www.yeelight.com/zh\_CN/product/donut}
%
\end{footnote}:还是© 2020
Microsoft。 4月8日已经变了。

\item {} 
是否可以从金融专业的翻译入手\sphinxhref{https://www.youtube.com/watch?v=vu6YmD5TX7E}{7}%
\begin{footnote}[885]\sphinxAtStartFootnote
\sphinxnolinkurl{https://www.youtube.com/watch?v=vu6YmD5TX7E}
%
\end{footnote}

\item {} 
\sphinxurl{https://singer.xiaoice.com/}: 大小约140M ; \sphinxurl{http://xstudio.pub/}
大小约220M

\end{itemize}


\subparagraph{Boss面}
\label{\detokenize{chapter_interview/simulate_interview:boss}}
行业的发展情况、产业链的发展、整体行业发展的大趋势类的话题,以及聊一下跟公司的文化价值观、愿景,看你是否认同公司的文化,如果能聊到共性,产生共鸣\sphinxhref{https://t.qidianla.com/1165227.html}{3}%
\begin{footnote}[886]\sphinxAtStartFootnote
\sphinxnolinkurl{https://t.qidianla.com/1165227.html}
%
\end{footnote}


\paragraph{HR}
\label{\detokenize{chapter_interview/HR:hr}}\label{\detokenize{chapter_interview/HR::doc}}

\subparagraph{HR是如何筛选简历的?9\sphinxfootnotemark[887]}
\label{\detokenize{chapter_interview/HR:hr-9}}%
\begin{footnotetext}[887]\sphinxAtStartFootnote
\sphinxnolinkurl{https://blog.csdn.net/pA2elX78qaJTADH/article/details/80768104?utm\_medium=distribute.pc\_relevant.none-task-blog-BlogCommendFromMachineLearnPai2-11.control\&dist\_request\_id=6f05adc5-b97c-4da8-ae1f-b2d8c5388ac8\&depth\_1-utm\_source=distribute.pc\_relevant.none-task-blog-BlogCommendFromMachineLearnPai2-11.control}
%
\end{footnotetext}\ignorespaces 
简历中突出优势、层次鲜明是必须的,招聘专员每天面对海量简历,往往10秒钟扫下简历中是否有AI、大厂、明星产品背景,统一的筛选标准有助于提升效率。而Boss直聘等个人渠道则不太一样,Boss往往更耐心,视角更专业。


\subparagraph{应聘前}
\label{\detokenize{chapter_interview/HR:id1}}
已知信息:企业的业务模式,企业服务的客户,应聘岗位名称,工作职责和任职要求

你的诉求:产品面试整个阶段应该了解什么信息,向谁了解
\begin{enumerate}
\sphinxsetlistlabels{\arabic}{enumi}{enumii}{}{.}%
\item {} 
应聘者的时间相对是自由的,而HR分配到使用boss进行沟通邀约的时间是相对固定并且较少的
\sphinxhref{https://wen.woshipm.com/question/detail/5tfpes.html?sf=wipm}{7}%
\begin{footnote}[888]\sphinxAtStartFootnote
\sphinxnolinkurl{https://wen.woshipm.com/question/detail/5tfpes.html?sf=wipm}
%
\end{footnote}

\item {} 
一般大厂会直接招实习,但是这家企业没有明确说是实习岗位,而是挂产品助理,外加产品经理的标签来加持,证明企业在行业内的竞争力应该一般,至少在人才吸引方面不会太强;

\item {} 
因为应聘者并没有表现出对岗位对公司的诚意,即到场面试。boss上的所有操作,目的是邀约而不是面试,这是两个概念,所以在什么环节干什么事,想要了解岗位的具体情况,就要到场面试,无论对于HR更好的了解你还是你更充分的了解企业,面试都是最佳选择,既体现尊重又便于双方高效的相互判断。

\end{enumerate}


\subparagraph{HR面试是拿Offer前的最后一关 4\sphinxfootnotemark[889]}
\label{\detokenize{chapter_interview/HR:hroffer-4}}%
\begin{footnotetext}[889]\sphinxAtStartFootnote
\sphinxnolinkurl{https://weread.qq.com/web/reader/8d232b60721a488e8d21e54k66f3299023a66f041e16858}
%
\end{footnotetext}\ignorespaces 
业务面试重点考察求职者的能力能否胜任产品经理的工作,HR
面试从其他方面对求职者做最终的把关,并且由于HR
负责控制公司的整体人力成本,因此与薪酬相关的内容也是在HR
面试中完成的,如果求职者和HR
没有就薪资达成一致,那么也会导致面试失败。因此,要想拿到Offer,即使轻松通过了业务面试也不要大意,HR
面试依然要好好准备。


\subparagraph{谈薪资 8\sphinxfootnotemark[890]}
\label{\detokenize{chapter_interview/HR:id2}}%
\begin{footnotetext}[890]\sphinxAtStartFootnote
\sphinxnolinkurl{http://www.woshipm.com/zhichang/2301423.html}
%
\end{footnotetext}\ignorespaces 
“我相信公司会给我一个合理的薪资安排。”

增加面试官的工作负担,并将决策风险转给了面试者。对于这种不愿意当风险承担者(价高就被拒)的情况,对于一些要求比较高的或者说看中主人翁精神的面试官可能直接就会拒绝你,去选择其他明确的候选人。

当HR问期望薪资时,不要直接给出一个数,而是先详细了解一下对方公司的薪资构成,包括月base、年终奖、绩效制度,以及福利体系,如各类奖金、各类补助的制度、五险一金待遇、假期待遇等。\sphinxhref{https://shimo.im/docs/vyCrK3rQQ6KC9Ryp/read}{10}%
\begin{footnote}[891]\sphinxAtStartFootnote
\sphinxnolinkurl{https://shimo.im/docs/vyCrK3rQQ6KC9Ryp/read}
%
\end{footnote}再给出一个你的期望薪资,并补上一句“我可以接受一定程度的薪资浮动”。

\begin{center}\sphinxincludegraphics{{salary}.png}\end{center} \sphinxincludegraphics{{salary_new}.png} \sphinxincludegraphics{{salary_graduate_PM}.jpg}


\subparagraph{考察重点分析}
\label{\detokenize{chapter_interview/HR:id3}}
传统行业HR 面试更加注重求职者的稳定性,所以HR
所提的问题更多的是私人问题(如家庭成员、籍贯等),因为HR
要确认求职者在未来几年内可以为公司持续创造价值。而互联网行业的人员流动性较大,所以互联网行业的HR
对求职者稳定性的考察会稍微弱一点,而是会重点考察其他方面。


\subparagraph{性格}
\label{\detokenize{chapter_interview/HR:id4}}
求职者最好是一个相对开朗、善于交际的人,也就是具有优秀的沟通能力。求职者还必须是一个对一切充满好奇、敢于质疑、追求卓越、能掌控大局,同时又心思缜密、执行能力很强的人。


\subparagraph{最大的缺点是什么?}
\label{\detokenize{chapter_interview/HR:id5}}
对现实的惨状太真实地吐露。


\subparagraph{你最大的兴趣爱好是什么?}
\label{\detokenize{chapter_interview/HR:id6}}
看书思辨与指点江山。


\subparagraph{如果在需求评审时,研发人员说无法实现这个需求,你会怎么做}
\label{\detokenize{chapter_interview/HR:id7}}
在遇到这个问题时不用害怕,求职者可以分3步来解决这个问题。
\begin{enumerate}
\sphinxsetlistlabels{\arabic}{enumi}{enumii}{}{.}%
\item {} 
了解无法实现的原因,是当前技术发展水平的限制,还是公司研发水平的限制?一般来说,只要是竞品可以实现的需求,理论上就不存在当前技术发展水平的限制,而更可能是公司研发水平的限制。

\item {} 
如果确实无法实现,那么产品经理可咨询研发人员,看\sphinxstylestrong{有哪些替代方案}。如果直线走不通,就可以尝试绕路。只要是能实现产品效果的方案,就可以对其进行评估。

\item {} 
如果有替代方案,就评估替代方案对项目的影响。例如,评估替代方案是否会导致项目延期、是否会导致产品方案变更。如果这些都在产品经理可接受的范围内,就可以适当地妥协。

\end{enumerate}


\subparagraph{P8的薪资}
\label{\detokenize{chapter_interview/HR:p8}}
P8的薪资=基本薪资(月薪4\sphinxhyphen{}5万,一年45\sphinxhyphen{}60万)+年终+股票。那P8能年薪200万左右,靠的是什么呢?不是靠基本薪资,是年终+股票,尤其是股票占大头。。股票要分4年才能拿完,满2年可以拿一半,如果满2年就走人,另外一半拿不到。另外股票要交税,45\%,拿到手并不多的。

股票的价值来源于哪里,来源于行业的红利+阿里这个平台的成功,反馈回员工身上的。平台利润+行业红利提供了远超出你本身能力的金钱收获。

Y哥,2K对于你来讲,太小的钱了。但对于学生来说,2K意味着很多。我不知道你有没有在大学食堂吃过饭,这些学生一顿饭才花8块钱。有些学生为了省1,2块钱,盘子里都是米饭,都看不见菜。2K块钱,不是小钱,是他们2个月的生活费。对于你们来说,为什么不花这2K块钱,买来这个学生的感激涕零,为公司全力付出。就像做饭一样,已经做了8分熟,为什么不加把火,做成10分熟。少2K块钱,他心不甘,又去看这个公司那个公司,心在摇摆。咱们能不能多花2K块钱,让他彻底心安,一心一意等待入职。\sphinxhref{https://www.zhihu.com/people/guosheng-hu/answers/by\_votes}{2}%
\begin{footnote}[892]\sphinxAtStartFootnote
\sphinxnolinkurl{https://www.zhihu.com/people/guosheng-hu/answers/by\_votes}
%
\end{footnote}


\subparagraph{禁忌离职原因1\sphinxfootnotemark[893]}
\label{\detokenize{chapter_interview/HR:id8}}%
\begin{footnotetext}[893]\sphinxAtStartFootnote
\sphinxnolinkurl{http://www.woshipm.com/zhichang/459131.html}
%
\end{footnotetext}\ignorespaces 
(1)因为收入低而离职

这样回答会让HR觉得你计较个人得失,认为你工作的
意义就是为了收入,并会猜想如果有更高收入的地方你会毫不犹豫的离职,而对你做出负面判断。

(2)因为分配不公平而离职

绩效工资、浮动奖金等是很多企业用来刺激员工提高工作效率的手段,用以体现努力和结果的结合,加之工资保密制度的实施,面试者用此作为借口会让HR以为你有喜欢打探别人隐私的嫌疑。

(3)因为人际关系复杂而离职

团队精神是大多数企业要求员工要具备的素质,拿人际关系复杂做原因HR可能会觉得你在人际交往中有所欠缺,没办法很好的融入群体。

(4)因为上司的为人问题而离职

一味的讲述上司的毛病,会在一定程度上说明你是缺乏工作上的适应性,同时,HR也会联想到你遇见麻烦的客户时会不会也凭好恶行事。

跳槽,只不过是一种双方解除之前的契约,重新和另外一家公司签订契约的过程而已,直面这个本质,你会抛弃很多顾虑。

我们在和其他公司合作时候,并不是说我和这家合作公司关系不好我就不合作了,而是这个渠道带来的ROI太低了,回报远远低于我的付出,那么我们会毅然决然的解除和这个渠道的合作。同样的,当你不开心了,当你觉得拿的少了,都不是你终止这份契约的标准,唯一的标准就是,你的付出回报比太低了,低于你的底线了,此时就解除这份契约吧。\sphinxhref{http://www.woshipm.com/zhichang/906380.html}{5}%
\begin{footnote}[894]\sphinxAtStartFootnote
\sphinxnolinkurl{http://www.woshipm.com/zhichang/906380.html}
%
\end{footnote}


\subparagraph{选择}
\label{\detokenize{chapter_interview/HR:id9}}
Offer
的选择是综合考虑行业、城市、公司、待遇等多个方面的因素而得出的一个最终结果。


\subparagraph{行业}
\label{\detokenize{chapter_interview/HR:id10}}
行业发展:互联网教育、医疗有很大的发展空间,因为这两个方向的需求是刚需


\subparagraph{城市}
\label{\detokenize{chapter_interview/HR:id11}}
互联网发展最好的5个城市分别是北京、深圳、杭州、上海和广州

\sphinxhref{https://blog.csdn.net/Dylan\_zhijing/article/details/107444119?spm=1001.2014.3001.5502}{成都}%
\begin{footnote}[895]\sphinxAtStartFootnote
\sphinxnolinkurl{https://blog.csdn.net/Dylan\_zhijing/article/details/107444119?spm=1001.2014.3001.5502}
%
\end{footnote}


\subparagraph{个人兴趣}
\label{\detokenize{chapter_interview/HR:id12}}
仅考虑发展趋势还不够,还要结合自己的个人兴趣,你只有从内心喜欢这个行业,才能发挥最大的主观能动性,最大限度地发挥自己的创意,才更有可能升职加薪。最理想的状态是自己喜欢的方向恰巧未来的发展空间很大。如果二者不能很好地融合,那么我的建议是行业发展的优先级大于个人兴趣。


\subparagraph{公司}
\label{\detokenize{chapter_interview/HR:id13}}
要综合比较公司规模和业务线的重要程度。如果是大型公司的核心业务和创业型公司的核心业务对比,就选择大型公司;如果是大型公司的边缘业务和创业型公司的核心业务对比,就选择创业型公司;如果是大型公司的边缘业务和创业型公司的边缘业务对比,就选择大型公司。


\subparagraph{待遇}
\label{\detokenize{chapter_interview/HR:id14}}
在考虑待遇时,不能只看月薪,而要综合考虑。要了解清楚季度奖、年终奖的数额,有无饭补、房补、交通补助,公积金的缴纳基数及比例,有无加班费等。

2021互联网校招薪资爆料:\sphinxurl{https://blog.nowcoder.net/n/2668a1d85a174bec94e1fea21dc01551}

头条、美团、滴滴、阿里、腾讯、百度、华为、京东职级体系及对应薪酬:
\sphinxurl{https://www.cnblogs.com/dhcn/p/11983157.html}

\begin{figure}[H]
\centering
\capstart

\noindent\sphinxincludegraphics{{salary_content}.png}
\caption{薪资结构与公司的成本\sphinxhref{http://www.woshipm.com/zhichang/807191.html}{11}\sphinxfootnotemark[896]}\label{\detokenize{chapter_interview/HR:id16}}\end{figure}
%
\begin{footnotetext}[896]\sphinxAtStartFootnote
\sphinxnolinkurl{http://www.woshipm.com/zhichang/807191.html}
%
\end{footnotetext}\ignorespaces 

\subparagraph{有什么需要我特别注意的部门吗?}
\label{\detokenize{chapter_interview/HR:id15}}
在大公司,他们认为「销售」部门需要特别注意,他们会让你知道销售就像是这里的神祇,不要惹恼他们。在小一点的公司,他们会告诉你这方面没什么需要注意的。通过这个问题你可以了解一些第一天工作要知道的事——实际发号施令的是谁?是否存在有些人觉得不值但也有些人很喜欢的项目?如果他们不介意告诉你一些秘辛,这会在你入职的头几周帮到你。这个问题也表示你很想融入公司,想和周围的人进行适当的沟通。


\paragraph{回报}
\label{\detokenize{chapter_interview/reward:id1}}\label{\detokenize{chapter_interview/reward::doc}}

\subparagraph{工资 1\sphinxfootnotemark[897]}
\label{\detokenize{chapter_interview/reward:id2}}%
\begin{footnotetext}[897]\sphinxAtStartFootnote
\sphinxnolinkurl{https://weread.qq.com/web/reader/46532b707210fc4f465d044k65132ca01b6512bd43d90e3}
%
\end{footnotetext}\ignorespaces 

\subparagraph{定义}
\label{\detokenize{chapter_interview/reward:id3}}
工资是员工因向所在的组织提供劳务而获得的各种形式的酬劳。狭义的工资指货币和可以转化为货币的报酬。工资包括四个部分,分别是固定工资、浮动工资、短期奖励工资、长期奖励工资。


\subparagraph{组成}
\label{\detokenize{chapter_interview/reward:id4}}
固定工资通常也叫基本工资,不随业绩或工作成果变化。浮动工资具备一定的激励作用,如绩效工资、公司分红或者利润分成等。当然,广义的工资还包括一些福利,比如公司向员工提供的各种保险、非工作日加班工资、额外的津贴和其他服务,比如单身公寓、免费工作餐等。

短期奖励工资是针对一年或者一年以内的特定绩效提供奖励的一种工资,比如年终奖,有些公司会把年底双薪或者三薪算进去。长期奖励工资是针对一年或者一年以上的特定绩效提供奖励的一种工资,比如股票、期权等,每家公司都不一样,在跳槽的时候最好问清楚。


\subparagraph{能力}
\label{\detokenize{chapter_interview/reward:id5}}
优秀的人的能力应该是螺旋式上升的。吴军老师讲过一个谷歌公司新员工培训的案例,你可以用一支笔在白纸上画出大一点的Z字,最下面的一条边称为基线。你的所有工作都应该建立在这条线的基础上,而不是从这条线下面开始做起,这点很重要。举个例子,当你应聘产品经理时,这条基线就应该是具备基本的办公技能,熟悉电脑操作等,你不可能进入公司工作后现学。接下来,你参加各种培训,看各种书籍,每天刷朋友圈的好文章,目的就是提高自己的基线。只有提高了基线,你才能进入下一个圈层。

Z
字上面那条边代表了极限值,是无法突破的。但是你有了目标,不断地提升自己的能力就是为了更接近上面那条线,所以你在跳槽时应该更有目标,应该在一个行业中积累,或者在一个技能领域沉淀,不然之前的基线就没用了。就好像你练了两年游泳,又去练田径,然后去练射击,最后可能一事无成。我们找工作也是一样的,要注意自己的基线,每一次跳槽、每个阶段的努力都是为了让能力螺旋式上升。就像在登山时我们需要有能够攀爬上去的绳索或者阶梯,而不是一直在山脚下不停地切换目标。


\paragraph{工作}
\label{\detokenize{chapter_interview/offer:id1}}\label{\detokenize{chapter_interview/offer::doc}}
多申请了几家:如果手里没有颇具竞争力的
offer,你在薪资谈判中就会处于严重的劣势。面试练习也的确给了我很大的帮助。

我了解的公司更多了。居然有那么多不错的工作我都没有考虑过!一些最有趣的职位(以及最好的
offer)来自我最初没有考虑过的公司。

我找工作花了挺长时间(从投出第一份简历到接受 offer
大约有半年时间),但我也是精疲力尽:那几个月我基本都是在机场、酒店、面试间度过的,不断接电话、和
HR 谈薪资。

不要指望在这个时期做什么工作。正如一个同事所说:「你的头脑总是被那些招聘反馈占满,没有多余的精力去想
ICML。」

几乎所有的面试官都会抽时间询问职位、团队或公司的相关问题,我喜欢询问工作与生活的平衡、工作的难点或他们对工作不太喜欢的地方,大多数人都会诚实坦率地回答这些问题,这会展示未来工作一些尖锐的问题所在。


\subparagraph{公司认可什么价值}
\label{\detokenize{chapter_interview/offer:id2}}
商业价值:
\begin{itemize}
\item {} 
业务价值

\item {} 
技术价值

\item {} 
工程价值

\end{itemize}


\subparagraph{公司如何评判价值}
\label{\detokenize{chapter_interview/offer:id3}}\begin{itemize}
\item {} 
业务成就

\item {} 
技术难题

\item {} 
工程设施

\end{itemize}


\bigskip\hrule\bigskip


我认为面试人员常常故意不具体说明一些问题,就是想要看看我的反应,并且愿意提供帮助或与我讨论细节。这种面试从不像是一种对抗过程,而更像是同事之间的讨论。

关于招聘人员

我从没有见过像招聘人员这样将胡萝卜加大棒的把戏玩得如此炉火纯青的一群人。

他们会告诉你「这是我们能给的最高待遇」(一周后就变了),还会跟你强调「他们会为你破例,因为你是一个非常优秀的求职者」(而实际上提供给你的东西和别人相差无几)。

他们还会凭空捏造出严格的
deadline(五分钟之后就会告诉你他们完全接受延期),他们还会告诉你,他们不会重新谈判(但只要你拿出更有竞争力的
offer,他们还是会重新跟你谈),还会说他们在过去两个月内在面试其他候选人(但他们绝对不会放弃跟你谈)。

感觉招聘人员总是想确定我是不是真的想选择另一份 offer
而不是他们的,还是只是利用他们与另一家公司重新谈判。我想这一切都在意料之中,而我最好的建议是始终保持礼貌、耐心和执着。


\bigskip\hrule\bigskip


先分享一下我的经验吧。我帮我们组和其他机器学习组面过50多个candidates(含校招或社招、全职或实习)。面试前一般会有prebrief,就是hiring
manager会描述对candidate的机器学习或者深度学习(DL)的水平的预期,然后面试官相应地设计面试题。

假设candidate真的把整本书吃透了,根据我面试的数据:

Breadth方面:

能过。

Depth方面:

如果candidate本身做过深度学习项目,我一般会结合这个项目问一些深一点的问题,大部分问题回归本质后,答案都在书里。

如果candidate无DL项目经验,我会从一个基础的问题开始和candidate聊,一般这个candidate会有意无意地显示出自己比较擅长的方面,然后我会在这方面问越来越深的问题,判断candidate在最擅长的方面能达到什么水平。假设这个方面与深度学习有关,我一般再深挖的问题无非与方法细节、数学分析、实现细节有关,这些书里都涉及了。

现在工业界用深度学习比较普遍,所以大部分机器学习工作都直接要求深度学习背景了。比如prebrief的时候很多hiring
manager会明确告诉我机器学习breadth/depth只需要关注candidate在深度学习方面的水平。但如果一个职位强调传统机器学习方法的话,我觉得在本书基础上可以根据职位需求多准备一些传统机器学习的基础知识。


\paragraph{背景调查}
\label{\detokenize{chapter_interview/background_survey:id1}}\label{\detokenize{chapter_interview/background_survey::doc}}

\subparagraph{定义}
\label{\detokenize{chapter_interview/background_survey:id2}}
企业在求职者面试通过之后,发offer之前,一般会对求职者进行背景调查。所谓的背景调查就是通过各种渠道(一般找你之前公司的hr或者你之前的公司领导)核实面试者在简历上提供的信息,核实的内容包括但不限于工作经历(比如你真实的工作时间是否和简历上写的一致)、教育经历、薪资水平、工作中的表现以及真实的离职原因等。


\subparagraph{为什么会}
\label{\detokenize{chapter_interview/background_survey:id3}}
由于市场上简历注水的现象比比皆是,外加各种培训机构的包装,企业进行背景调查也是为了规避用人风险,这种风险包括胜任力风险、法律风险、职业操守风险和成本风险,而一份完善的员工背景调查报告,可以为企业节省不必要的花销,降低企业招聘时间成本、人力成本等。避免损害企业和股东的利益。

很多人选择简历造假,是因为他们不知道有这么个关卡。这个关卡代表着,如果要造学历假,得把人事档案,学校档案,成绩单,学信网等一系列信息都改了;要造假工作经历,得把劳动合同,人事档案,社保记录,纳税记录,银行流水,同事印象等一系列也都改了,而这些修改大部分是无法做到的。

一个人在简历上造假表明求职者对目标公司有\sphinxstylestrong{投机}之嫌,HR
有理由相信这个人进公司以后是不会对他的职位负责的;另外,简历造假,说明求职者的诚信有问题,HR
有理由相信这个人有一定可能会损害公司利益。\sphinxhref{https://www.zhihu.com/collection/618263456}{2}%
\begin{footnote}[898]\sphinxAtStartFootnote
\sphinxnolinkurl{https://www.zhihu.com/collection/618263456}
%
\end{footnote}


\subparagraph{什么情况会}
\label{\detokenize{chapter_interview/background_survey:id4}}

\subparagraph{与资金有关的专业岗位}
\label{\detokenize{chapter_interview/background_survey:id5}}
比如会计、出纳、财务、投资类岗位,出于对资金安全的考虑,企业会对这些岗位的工作人员进行背景调查、了解员工的工作能力、犯罪记录和诚信情况。


\subparagraph{接触的核心机密的岗位}
\label{\detokenize{chapter_interview/background_survey:id6}}
比如产品经理、产品总监、技术总监、架构师等。这些人一般可以接触企业的核心机密,核心机密是企业的核心竞争力,企业对这样的岗位,企业在招聘这些人员时都会非常谨慎,宁可花费一定的资金对录用的求职者进行犯罪记录、诚信状况等背景调查。


\subparagraph{中高层管理岗位}
\label{\detokenize{chapter_interview/background_survey:id7}}
中高层管理岗位,也接触企业核心机密,掌控企业的运营战略,有的甚至还把握着核心客户资源,他们对于企业的日常运营,甚至未来发展顺利与否,都有举足轻重的影响。大多数的企业都会对中高层岗位的聘用进行背景调查。


\paragraph{招不到合适}
\label{\detokenize{chapter_interview/hire:id1}}\label{\detokenize{chapter_interview/hire::doc}}
做产品、招牛人、自成长,都是要知道“边界”。

如何才能知道“边界”呢?

从表面看,是格局+正直。站得高,才能看得远,而有清澈的心,才能借由阳光,看到边界。

从本质看,是谦虚。因为边界是动态的,每一步,内外环境都在变,即使前一天看得清楚,后一天都可能一叶障目。

就像当初朱啸虎主动去找(投)滴滴,不仅因为这个方向的市场空间大,更因为”滴滴用了唯一正确的切入方式“——四个不做(不做黑车、不做加价、不做帐户、不做硬件)。

做产品、招牛人、自成长,都是要知道“边界”。

如何才能知道“边界”呢?

从表面看,是格局+正直。站得高,才能看得远,而有清澈的心,才能借由阳光,看到边界。

从本质看,是谦虚。因为边界是动态的,每一步,内外环境都在变,即使前一天看得清楚,后一天都可能一叶障目。


\subparagraph{邀请合适的产品经理 2\sphinxfootnotemark[899]}
\label{\detokenize{chapter_interview/hire:id2}}%
\begin{footnotetext}[899]\sphinxAtStartFootnote
\sphinxnolinkurl{https://weread.qq.com/web/reader/8d632bc07208ed1c8d697c4ka5732aa0226a5771bce9dc4}
%
\end{footnotetext}\ignorespaces 
如果我们遇到了一个各方面都很合适的产品经理,怎么邀请他加入呢?这种产品经理往往在行业里很抢手,他备选的机会也很多。对此,我们可以借助这样几种激励方式:

名、利、权、情。“名”即名誉,如好听的岗位抬头;“利”即利益,如优厚的薪资待遇;“权”即权力,如对产品的决策权限;“情”即情感,如对产品愿景的认同。任何高阶岗位都可以用类似的激励方式。


\paragraph{初入职场}
\label{\detokenize{chapter_interview/enter:id1}}\label{\detokenize{chapter_interview/enter::doc}}
完成日常工作之余,还主动了解公司架构、业务规范、部门内外合作流程。\sphinxhref{https://t.qidianla.com/1166342.html}{1}%
\begin{footnote}[900]\sphinxAtStartFootnote
\sphinxnolinkurl{https://t.qidianla.com/1166342.html}
%
\end{footnote}


\subparagraph{一定不要做什么 2\sphinxfootnotemark[901]}
\label{\detokenize{chapter_interview/enter:id2}}%
\begin{footnotetext}[901]\sphinxAtStartFootnote
\sphinxnolinkurl{http://www.jfrcw.com/zhichang/215586.html}
%
\end{footnotetext}\ignorespaces 
无论你的前任是怎么离开的,总会有些说不清道不明的原因,否则好的项目凭什么轮到你?所以在遇到项目真正的困难前,以下几件事情一定不要做,给自己增加难度了。
\begin{enumerate}
\sphinxsetlistlabels{\arabic}{enumi}{enumii}{}{.}%
\item {} 
当众诋毁前任产品经理,以及产品现状
这是一种最外行的行为,就像天天有人要教张小龙怎么做微信的吃瓜群众一般,上来就大批评产品的现状是多么的烂,前任产品是多么的无能。

\end{enumerate}

传到前任产品耳里不可怕,但如果传到之前为此付出过心血的研发、测试、运营、设计团队那里,你还准备让人跟你一起好好干嘛?

经常有人问,你这个绘本上的字体怎么不能调整字号,那谁谁家的就可以;你这个语音评测怎么这么慢,别人家的立即就能立即反馈结果。

最开始我还是比较在意的,直到有一天有人安慰到,说:“壳,我相信你当时做了最正确的判断。”是的,你不了解当时的情况,以及这样做的背景;没必要等到自己做的时候,同样的话送给了你自己,让众人嘲笑。
\begin{enumerate}
\sphinxsetlistlabels{\arabic}{enumi}{enumii}{}{.}%
\setcounter{enumi}{1}
\item {} 
给团队及领导画饼
有自信是好事,有行业经验也是好事,但换了个新环境、新团队、新公司就想把之前的经验照搬,给大家定目标,给领导画饼。

\end{enumerate}

见过两次有人接手我的项目:

一个是没做过这样类似的项目,但是是从大厂来的,想当然的迷之自信;笔者花半年时间才从2万做到10万顶峰日活,就敢给团队定个次年200万日活的目标,瞬间之前的研发、运营、设计都来给我吐槽,啥不也不懂,啥也不是。

另一个是之前做过类似项目,可能还挺成功,来了一个月却不和我进行深入的沟通和交流,成天沉迷商业模型的设计和PPT;结果去给高层做汇报,被当场diss到体无完肤;都说成功是不可复制的,你过去的成功未必能在新的平台和团队里做好,何况最后我仔细看了一眼,真是空中楼阁,浮夸。
\begin{enumerate}
\sphinxsetlistlabels{\arabic}{enumi}{enumii}{}{.}%
\setcounter{enumi}{2}
\item {} 
立即推进项目 这种可能是少数产品经理特有的一些特性。

\end{enumerate}

笔者之前带过整条业务线,总会希望各小组(研发、运营、设计等)保持住一种“状态”,有点像一辆车从静止到跑上高速,总会需要有一段加速的过程;那么我会希望在更重要的需求确定下来之前,先做一些简单不会错的事情,让大家热身起来,也是个磨合。

说实话,结果其实有好有坏,但实际推进过程中会遇到很多问题,特别是很难回答灵魂问题:“你的目标是什么,你做这个的收益是什么?”如果答不上来,难免会遇到一些尴尬,这种尴尬如果持续扩大,会产生更多的不信任。
\begin{enumerate}
\sphinxsetlistlabels{\arabic}{enumi}{enumii}{}{.}%
\setcounter{enumi}{3}
\item {} 
立即推倒项目
比立即推进项目更恶劣的情况就是直接把原有项目推倒重来,按照自己的想法来。

\end{enumerate}

怎么说呢,这个按我上面的思路再想想,你即不是皇帝,也不是CEO,初来乍到如果又没有什么成绩,why?不靠谱和不信任的标签将会长期被贴在身上。

我司的CTO来公司观察和学习小半年后,才做出了大刀阔斧的改革,他也没有借着之前的管理经验就硬上呀。

二、“接盘”的第一步:了解现状

因为你将接手的项目已经存在,甚至上线过很久,迭代过很多版本;虽然可能会有前任产品,或者上级领导做工作交接,但总会有很多不具体的地方,很多细节即使交接也未必清楚,或者年代久远没有记录。

先了解公司产品出于什么阶段了,现在做开发的成员人数。\sphinxhref{https://www.zhihu.com/question/38295860/answer/76188176}{3}%
\begin{footnote}[902]\sphinxAtStartFootnote
\sphinxnolinkurl{https://www.zhihu.com/question/38295860/answer/76188176}
%
\end{footnote}


\subsection{深入学习数据}
\label{\detokenize{chapter_data_dive/index:chap-more}}\label{\detokenize{chapter_data_dive/index:id1}}\label{\detokenize{chapter_data_dive/index::doc}}

\subsubsection{数据}
\label{\detokenize{chapter_data_dive/data:id1}}\label{\detokenize{chapter_data_dive/data::doc}}

\paragraph{数据定义}
\label{\detokenize{chapter_data_dive/data:id2}}
数据(data),被视为“科学的度量、知识的来源”,在物理上存储以字节(byte)为计量单位,是关于事件之一组“离散且客观”的事实描述③,是数据原子(data
atomic)、数据项(data item)、数据对象(data object)、数据集(data
set)的统称\sphinxhref{https://scholar.harvard.edu/files/ctang/files/data\_industry\_draft\_in\_chinese.pdf}{1}%
\begin{footnote}[903]\sphinxAtStartFootnote
\sphinxnolinkurl{https://scholar.harvard.edu/files/ctang/files/data\_industry\_draft\_in\_chinese.pdf}
%
\end{footnote},分为模拟数据和数字数据两种。

描述数据的数据称为元数据(mata
data);处理数据的数据,如程序或软件,称为数据工具(data tool)。


\paragraph{数据的衡量}
\label{\detokenize{chapter_data_dive/data:id3}}
人工智能产品对数据除了有量的要求,还有质的要求,衡量数据质量的标准包括四个R:关联度relevancy(首要因素)、可信性reliability(关键因素)、范围range、时效性recency。


\subparagraph{可行度}
\label{\detokenize{chapter_data_dive/data:id4}}
2018年4月,易观数据发布《中国共享单车行业数据报告》,报告中称,2018年2月,ofo小黄车、摩拜单车和哈罗单车的市场覆盖率分别为50.89\%、49.14\%和5.64\%。哈罗单车CEO杨磊在朋友圈中转发相关消息吐槽:这种机构,给出来的数据就是个笑话。\sphinxhref{http://tech.sina.com.cn/i/2018-06-02/doc-ihcikcew4309938.shtml}{3}%
\begin{footnote}[904]\sphinxAtStartFootnote
\sphinxnolinkurl{http://tech.sina.com.cn/i/2018-06-02/doc-ihcikcew4309938.shtml}
%
\end{footnote}


\paragraph{数据的价值}
\label{\detokenize{chapter_data_dive/data:id5}}
2020年4月,《中共中央、国务院关于构建更加完善的要素市场化配置体制机制的意见》(以下简称“《意见》”)发布,数据作为一种新型生产要素被写入国家文件中,与土地、劳动力、资本、技术等传统要素并列为要素之一。

《意见》明确,加快培育数据要素市场,推进政府数据开放共享、提升社会数据资源价值、加强数据资源整合和安全保护。\sphinxhref{https://www.ofweek.com/security/2020-09/ART-510006-8900-30458012.html}{5}%
\begin{footnote}[905]\sphinxAtStartFootnote
\sphinxnolinkurl{https://www.ofweek.com/security/2020-09/ART-510006-8900-30458012.html}
%
\end{footnote}

数据由于具备可复制和可传播性,其本质上是不能被安全共享的,但在数据要素时代,我们虽然不共享数据本身,但\sphinxstylestrong{数据价值}应该被共享。\sphinxhref{https://www.ofweek.com/security/2020-09/ART-510006-8900-30458012.html}{6}%
\begin{footnote}[906]\sphinxAtStartFootnote
\sphinxnolinkurl{https://www.ofweek.com/security/2020-09/ART-510006-8900-30458012.html}
%
\end{footnote}


\subparagraph{数据获取地址:}
\label{\detokenize{chapter_data_dive/data:id6}}\begin{itemize}
\item {} 
ICPSR:www.icpsr.umich.edu

\item {} 
美国政府开放数据:www.data.gov

\item {} 
加州大学欧文分校:archive.ics.uci.edu/ml

\item {} 
数据堂:www.datatang.com

\end{itemize}

公开数据集:\sphinxhref{https://github.com/HuangCongQing/AI\_competitions}{4}%
\begin{footnote}[907]\sphinxAtStartFootnote
\sphinxnolinkurl{https://github.com/HuangCongQing/AI\_competitions}
%
\end{footnote}
\begin{itemize}
\item {} 
Google数据集搜索:\sphinxurl{https://toolbox.google.com/datasetsearch}

\item {} 
Datahub,分享高质量数据集平台:\sphinxurl{https://datahub.io/}

\item {} 
用于上传和查找数据集的机器学习数据集存储库:\sphinxurl{https://www.webdoctx.com/www.mldata.org}

\item {} 
datafountain收集数据集:\sphinxurl{https://www.datafountain.cn/dataSets}

\item {} 
tinymind收集数据集:\sphinxurl{https://www.tinymind.cn/sites\#group\_22}
看到的一篇文章,里面有介绍很多数据集的:世界上最有价值的不是石油而是数据(附数据资源下载链接)

\end{itemize}


\subsubsection{数据产业}
\label{\detokenize{chapter_data_dive/data_industry:id1}}\label{\detokenize{chapter_data_dive/data_industry::doc}}
将信息产业简单理解为信息化,从技术效果上看是将现实世界中的事物以数据的形式存储到计算机等设备中,是一个生产“数据”的过程,当前已累积并形成了多种领域或行业的数据资源。挖掘这些数据资源、提取有用信息,将涌现“取之不尽,用之不竭”的数据创新;赋予这些数据创新商业模式,就是产业化,会形成一种影响世界经济格局的战略性新兴产业,我们称之为数据产业(data
industry),数据产业是信息产业的逆反、衍生与升级。

数据产业是一种能提供“服务”的“技术”密集型产业,其所依赖的“技术”——数据科技,所对应的是数据科学(Data
Science)。


\subsubsection{合规数据}
\label{\detokenize{chapter_data_dive/compliance_data:id1}}\label{\detokenize{chapter_data_dive/compliance_data::doc}}

\paragraph{定义}
\label{\detokenize{chapter_data_dive/compliance_data:id2}}
?


\paragraph{获取途径}
\label{\detokenize{chapter_data_dive/compliance_data:id3}}\begin{enumerate}
\sphinxsetlistlabels{\arabic}{enumi}{enumii}{}{.}%
\item {} 
基于互联网的公开合规数据的挖掘;

\item {} 
合法的第三方数据源的获取;

\item {} 
企业内部分散的数据的统一整合。

\end{enumerate}

如果把这三种数据的价值能够统一梳理整合呈现,将会形成自己企业真正的“有价值数据资产”,进而形成基于数据的“核心竞争力”。\sphinxhref{https://www.weiyangx.com/346934.html}{1}%
\begin{footnote}[908]\sphinxAtStartFootnote
\sphinxnolinkurl{https://www.weiyangx.com/346934.html}
%
\end{footnote}


\subsubsection{侵权}
\label{\detokenize{chapter_data_dive/pirate:id1}}\label{\detokenize{chapter_data_dive/pirate::doc}}

\paragraph{何为侵权}
\label{\detokenize{chapter_data_dive/pirate:id2}}
❌
不遵守商业协议,私自销售商业源码\sphinxhref{https://githubmemory.com/repo/FanWenTao-Felix/Archer-1\#\%E4\%BE\%B5\%E6\%9D\%83\%E5\%90\%8E\%E6\%9E\%9C}{1}%
\begin{footnote}[909]\sphinxAtStartFootnote
\sphinxnolinkurl{https://githubmemory.com/repo/FanWenTao-Felix/Archer-1\#\%E4\%BE\%B5\%E6\%9D\%83\%E5\%90\%8E\%E6\%9E\%9C}
%
\end{footnote}
❌ 以任何理由将?源码用于申请软件著作权 ❌
将商业源码以任何途径任何理由泄露给未授权的单位或个人 ❌
开发完毕项目,没有为甲方购买企业授权,向甲方提供了?代码 ❌
基于?拓展研发与?有竞争关系的衍生框架,并将其开源或销售


\paragraph{侵权后果}
\label{\detokenize{chapter_data_dive/pirate:id3}}\begin{itemize}
\item {} 
情节较轻:第一次发现警告处理

\item {} 
情节较重:封禁账号,踢出商业群,并保留追究法律责任的权利

\item {} 
情节严重:与本地律师事务所合作,以公司名义起诉侵犯计算机软件著作权

\end{itemize}


\subsubsection{隐私}
\label{\detokenize{chapter_data_dive/private:id1}}\label{\detokenize{chapter_data_dive/private::doc}}
机器学习隐私泄露:
\begin{itemize}
\item {} 
不可靠的数据收集者泄露信息(直接泄露)

\item {} 
攻击者分析机器学习模型输出结果,逆向推出训练数据中的用户敏感信息(间接泄露)

\end{itemize}


\paragraph{法律}
\label{\detokenize{chapter_data_dive/private:id2}}
在法律层面,欧盟《通用数据保护条例》(GDPR)规范了企业收集、管理、删除客户和个人数据\sphinxhref{https://www.msra.cn/zh-cn/news/features/privacy-protection-in-machine-learning}{3}%
\begin{footnote}[910]\sphinxAtStartFootnote
\sphinxnolinkurl{https://www.msra.cn/zh-cn/news/features/privacy-protection-in-machine-learning}
%
\end{footnote}

华为小冰虚拟男友是安全的吗,怎么看待有网友爆料会自动开启摄像头窃取用户隐私的事情?是可信的吗?
\sphinxhyphen{} 知乎 \sphinxurl{https://www.zhihu.com/question/418430367}

当我们的技术团队,就是机器语义理解的团队,说我们自己做输入法可能会做的更好的时候,我当然很赞成。因为至少,在安全性方面,我们可以做的足够好。\sphinxhref{https://www.uisdc.com/wechat-10-years}{2}%
\begin{footnote}[911]\sphinxAtStartFootnote
\sphinxnolinkurl{https://www.uisdc.com/wechat-10-years}
%
\end{footnote}


\paragraph{解决方案}
\label{\detokenize{chapter_data_dive/private:id3}}
隐私计算是一门综合技术,具体来说,目前主要包括三个方向。\sphinxhref{https://www.ofweek.com/security/2020-09/ART-510006-8900-30458012.html}{7}%
\begin{footnote}[912]\sphinxAtStartFootnote
\sphinxnolinkurl{https://www.ofweek.com/security/2020-09/ART-510006-8900-30458012.html}
%
\end{footnote}

其一为基于密码学的多方安全计算(MPC:Multi\sphinxhyphen{}party
Computation)技术。通过秘密分享、遗忘传输、混淆电路或同态加密等特殊的加密算法和协议,从而支持在加密数据上直接进行计算。理论上,在不考虑代价的“理想”情况下,多方安全计算技术能实现任意的计算“功能”,并且达到比较高的安全性。但是由于数据通信量骤增,计算效率损失大和需要极高的算力要求等因素,MPC的技术产品化还有一定的限制,相关的技术解决方正在积极探索。

其二为基于人工智能的联邦学习技术。在横向维度,每个参与者在本地训练计算自己的样本,只分享模型训练的梯度;纵向维度,各参与者训练各自的embedding(“向量映射”),共同训练上层模型。两个维度的融合,从而让多个相互不信任的数据拥有方不必共享数据的基础上联合进行模型训练。

其三为基于可信硬件的安全沙箱计算(TEE: Trusted Execution
Environment\sphinxhref{http://gaocegege.com/Blog/fedlearn}{5}%
\begin{footnote}[913]\sphinxAtStartFootnote
\sphinxnolinkurl{http://gaocegege.com/Blog/fedlearn}
%
\end{footnote})技术。其核心思想是构建一个硬件安全区域,数据仅在该安全区域内进行计算,利用可信任执行环境TEE防止操作系统恶意地查看应用执行环境的内容;利用安全沙箱防止恶意应用通过特殊调用控制操作系统。

业界解决隐私泄露和数据滥用的数据共享技术路线主要有两条。一条是基于硬件可信执行环境(TEE:
Trusted Execution
Environment)技术的可信计算,另一条是基于密码学的多方安全计算(MPC:Multi\sphinxhyphen{}party
Computation)。\sphinxhref{http://gaocegege.com/Blog/fedlearn}{5}%
\begin{footnote}[914]\sphinxAtStartFootnote
\sphinxnolinkurl{http://gaocegege.com/Blog/fedlearn}
%
\end{footnote}


\subparagraph{深度学习隐私保护}
\label{\detokenize{chapter_data_dive/private:id4}}\begin{itemize}
\item {} 
宽松差分隐私:绝对的差分隐私会导致天平倾向隐私,而导致系统不可用

\item {} 
集成模型:一种基于知识迁移的深度学习隐私保护框架。引入学生模型和教师模型\sphinxhref{https://segmentfault.com/a/1190000023428141}{8}%
\begin{footnote}[915]\sphinxAtStartFootnote
\sphinxnolinkurl{https://segmentfault.com/a/1190000023428141}
%
\end{footnote}

\end{itemize}

不足:隐私性降低,泄露风险的可能性变大。另外,差分隐私仅能实现单点的隐私保护,若不同记录之间存在关联,攻击者仍可对满足差分隐私的算法进行攻击。


\subparagraph{对金融领域的重要意义}
\label{\detokenize{chapter_data_dive/private:id5}}
无论是联邦学习还是共享智能,很多技术实践都优先选择了在金融领域落地。相较于其他领域,金融领域对数据的管控更为严格,对数据隐私更加重视,因此也是最需要通过技术手段解决数据孤岛问题的领域。

周俊表示,在金融领域,共享智能侧重在解决“开放”这个大领域中的问题,比如联合营销、联合风控等,这两个场景相对更容易看到具体实施效果。相比其他领域,金融领域对数据保护看的更重,数据的流转在该领域中更难,因此采用共享智能技术,可以做到更好的隐私保护,实现数据可用不可见,是一个关键的助推器。

举例来说,通过数据融合,蚂蚁金服的共享智能帮助中和农信大幅度提高了风控性能,把原来传统的线下模式,变成线上自动过审模式,完成授信只需5分钟,8个月累计放款31.9亿,授信成功人数44万人,业务覆盖20多个省区,300多县城,10000多个乡村。


\subsubsection{隐私计算}
\label{\detokenize{chapter_data_dive/privacy_computing:id1}}\label{\detokenize{chapter_data_dive/privacy_computing::doc}}
隐私计算涵盖了信息搜集者、发布者和使用者在信息产生、感知、发布、传播、存储、处理、使用、销毁等全生命周期过程的所有计算操作,并包含支持海量用户、高并发、高效能隐私保护的系统设计理论与架构。
简单来说,隐私计算是从数据的产生、收集、保存、分析、利用、销毁等环节中对隐私进行保护的方法。

隐私计算的概念最早是在2016年提出的,隐私计算是面向隐私信息全生命周期保护的计算理论和方法,是隐私信息的所有权、管理权和使用权分离时隐私度量、隐私泄漏代价、隐私保护与隐私分析复杂性的可计算模型与公理化系统。隐私计算涵盖了信息搜集者、发布者和使用者在信息产生、感知、发布、传播、存储、处理、使用、销毁等全生命周期过程的所有计算操作,并包含支持海量用户、高并发、高效能隐私保护的系统设计理论与架构。简单来说,隐私计算是从数据的产生、收集、保存、分析、利用、销毁等环节中对隐私进行保护的方法。\sphinxhref{https://www.odaily.com/post/5138174}{1}%
\begin{footnote}[916]\sphinxAtStartFootnote
\sphinxnolinkurl{https://www.odaily.com/post/5138174}
%
\end{footnote}


\paragraph{三大矛盾}
\label{\detokenize{chapter_data_dive/privacy_computing:id2}}
隐私数据的处理过程当中还面临着三个内部矛盾:安全、效率、数据孤岛。

安全方面,目前的大数据行业主要依托于可信第三方的计算服务。这些第三方包括主要应用于科研领域的超算中心和主要应用于商业领域的数据中心。大数据行业的高性能、高投入需求让规模化、集中化的运算成为了市场主流,2011年起,我国规划建设了255个数据中心,总设计服务器规模728万台,承担了我国大部分民用数据的计算服务。但这些集中化、规模化的数据中心可能出现问题也并非危言耸听:就在今年2月,由于阿里云代码托管平台的项目权限设置存在歧义,导致开发者操作失误,造成至少40家以上企业的200多个项目代码泄露,其中涉及到万科集团、咪咕音乐、51信用卡旗下51足迹、百度无人车合作伙伴ecarx等知名企业。

效率方面,在隐私信息的生命周期中,受益于密码学发展,隐私的加密化、匿名化和脱敏技术都已经非常成熟,可以大规模应用在隐私获取、储存、流转等环节中。但大数据时代的到来,让隐私数据的处理成为了一个难题:大规模的加密数据处理一定会导致计算性能下降,而非加密数据处理又极大概率会导致隐私信息的泄露。

数据孤岛是指的是数据被保存在无法自由流动的环境之下,互相独立存储、独立维护。数据被视为数字时代的石油,每家企业都想守着自己的数据挖掘出巨大的商业价值。甚至数据隐私本身的保护服务,就蕴藏着商业利润。Gartner就预测2019年全球消费者安全软件支出将达到66亿美元。至于各地的政府部门本身,由于责任边界、数据共享的技术条件等问题,也缺乏足够的动力来推动。

有了隐私计算+区块链技术就不一样了。你可以选择把你的信息以加密方式都存在区块链上,当你需要用你的信息去填各种表格的时候,可以直接用加密方式提供。对方拿到了加密后的个人信息,可以直接拿到区块链上去验证。这样对方既可以确保你信息的真实性又免于了直接拿到你的信息。


\paragraph{例子:}
\label{\detokenize{chapter_data_dive/privacy_computing:id3}}
你叫李红,身份证号是
310101199708311528,人长得温婉可人。你存在区块链上的信息可能成为了
il99dskkdsf3234dsfs9893jdsjjadsf
等一串长长的密文,人脸像也被哈希加密。当你入住酒店的时候,你无需出示你的身份证,只需要把密文
il99dskkdsf3234dsfs9893jdsjjadsf
发给需要你信息的酒店,秘钥只有你自己知道。酒店可以通过智能AI对你进行人脸识别,然后你的数据会在一个可信的计算环境中和你链上的加密人脸数据进行比对,确定你的入住身份。同时也会比较你的身份密文数据和公安系统通缉要犯库中的数据密文,如果匹配不成功,那么你就
OK
了,可以入住。整个过程中,酒店方\sphinxstylestrong{不会知道你的姓名和住址},但是又能够确认是你本人,而且不是通缉要犯,没用假身份证,让你可以办理入住。


\subsubsection{知识产权}
\label{\detokenize{chapter_data_dive/intellectual property:id1}}\label{\detokenize{chapter_data_dive/intellectual property::doc}}
《中国人工智能产业知识产权白皮书2020》已于2021年2月3日正式发布:

\sphinxurl{http://www.iprdaily.cn/article\_27004.html}


\subsection{深入学习AI}
\label{\detokenize{chapter_AI_dive/index:ai}}\label{\detokenize{chapter_AI_dive/index:chap-ai-dive}}\label{\detokenize{chapter_AI_dive/index::doc}}
​


\subsubsection{理解AI}
\label{\detokenize{chapter_AI_dive/understand_AI:ai}}\label{\detokenize{chapter_AI_dive/understand_AI::doc}}

\paragraph{境界}
\label{\detokenize{chapter_AI_dive/understand_AI:id1}}

\subparagraph{知其然境界}
\label{\detokenize{chapter_AI_dive/understand_AI:id2}}
第一个是知其然,也叫知其所以然境界,知道当下的人工智能到底是什么,知道机器学习和深度学习大概是个什么东西,不会过分的去神话AI,知道目前AI的优势,更知道目前AI的局限,如果你从事相关产品经理或者做相关领域投资,达到第一境界即可。


\subparagraph{应用者境界}
\label{\detokenize{chapter_AI_dive/understand_AI:id3}}
应用者境界是大部分人工智能算法工程师所在的境界,主要就是明白算法原理,知道如何实现,核心在于知道如何把他应用在一个实际的业务场景之中,如果你是一个利用AI来解决业务问题的算法工程师,那么你至少需要达到这个境界。


\subparagraph{工程师境界}
\label{\detokenize{chapter_AI_dive/understand_AI:id4}}
工程师境界也是很难的一种境界,需要用很强的理论背景和工程实现能力,能独立复现最新的论文,深刻理解论文的实现原理,并能在上面做一些小创新。如果你想成为更牛逼的AI算法工程师,那么就需要达到这个经济。


\subparagraph{科学家境界}
\label{\detokenize{chapter_AI_dive/understand_AI:id5}}
科学家境界是很少很少一部分人能达到的境界,主要是那些从事人工智能研究的科学家,他们能原创出很多的算法和理论,解决一些最前沿的难题。比如深度学习的鼻祖Hinton,生成对抗网络发明人
Ian goodfellow,Xgboost发明人陈天奇等等。


\subsubsection{AI 研究}
\label{\detokenize{chapter_AI_dive/AI_Research:ai}}\label{\detokenize{chapter_AI_dive/AI_Research::doc}}

\paragraph{国际上AI机构哪家强?}
\label{\detokenize{chapter_AI_dive/AI_Research:id1}}
DeepMind、OpenAI和FAIR。

需要说明的是,这三家实验室都是属于纯AI研究实验室,但“背靠大树好乘凉”,这三家顶级AI实验室背后都有“金主”,分别背靠着谷歌、微软和Facebook等互联网及软件巨头。


\subparagraph{DeepMind}
\label{\detokenize{chapter_AI_dive/AI_Research:deepmind}}
DeepMind原来是一家英国的人工智能公司,公司创建于2010年,最初名称是DeepMind科技(DeepMind
Technologies Limited),在2014年被谷歌收购。

谷歌每年都会给DeepMind拨款数亿美元,微软在OpenAI创始投资者10亿美元的基础上也投资了10亿美元,Facebook未对
FAIR的投资资金进行分类,但也是耗资不菲。

DeepMind 最出名的是 AlphaGo
,它在围棋游戏中挑战并击败了世界上最好的人类棋手,甚至还有一部关于
AlphaGo 战胜韩国围棋传奇李世石的 Netflix 纪录片。


\subparagraph{OpenAI}
\label{\detokenize{chapter_AI_dive/AI_Research:openai}}
OpenAI成立于2015年底,总部位于旧金山,其创始人是特斯拉CEO埃隆·马斯克(没错,就是现在的世界首富)、Y
Combinator总裁阿尔特曼、天使投资人彼得·泰尔(Peter
Thiel)以及其他硅谷巨头,最初成立它的动机是出于对强人工智能潜在风险的担忧。

OpenAI
开发了游戏人工智能软件,已经在诸多等游戏中击败人类。然而,它更为出名的则是自然语言处理预处理模型GPT\sphinxhyphen{}3和人工智能图像生成器
DALL\sphinxhyphen{}E,这些软件已经达到了可以思考人生、生成的图像可以假乱真的地步。


\subparagraph{FAIR}
\label{\detokenize{chapter_AI_dive/AI_Research:fair}}
FAIR成立的想法开始于 2013 年,Facebook的创始人扎克伯格,首席技术官 Mike
Schroepfer 以及公司其他持有股票的领导,都在寻找着未来 10 到 20
年让公司保持竞争力的技术,包括计算机视觉、自然语言处理和对话型AI等。

为了更好地理解FAIR与Facebook的关系,举个例子,其实就相当于国内“达摩院”与阿里巴巴的关系。\sphinxhref{https://www.weiyangx.com/379999.html}{1}%
\begin{footnote}[917]\sphinxAtStartFootnote
\sphinxnolinkurl{https://www.weiyangx.com/379999.html}
%
\end{footnote}


\paragraph{论文写作}
\label{\detokenize{chapter_AI_dive/AI_Research:id2}}
\sphinxurl{https://tobiaslee.top/2018/04/03/How-to-Write-a-Paper/}

北京十大推动中国科技发展的人工智能实验室:\sphinxurl{https://blog.csdn.net/yoggieCDA/article/details/102803735?utm\_medium=distribute.pc\_relevant.none-task-blog-2\%7Edefault\%7EOPENSEARCH\%7Edefault-6.baidujs\&dist\_request\_id=\&depth\_1-utm\_source=distribute.pc\_relevant.none-task-blog-2\%7Edefault\%7EOPENSEARCH\%7Edefault-6.baidujs}


\paragraph{开源软件}
\label{\detokenize{chapter_AI_dive/AI_Research:id3}}
中国人工智能开源软件发展白皮书:\sphinxurl{http://www.aioss.cn/}


\subsubsection{code}
\label{\detokenize{chapter_AI_dive/code:code}}\label{\detokenize{chapter_AI_dive/code::doc}}
\sphinxurl{http://oj.acmcoder.com/ExamNotice.html}


\paragraph{Paste}
\label{\detokenize{chapter_AI_dive/code:paste}}
\begin{figure}[H]
\centering
\capstart

\noindent\sphinxincludegraphics{{MD_Paste}.png}
\caption{Paste}\label{\detokenize{chapter_AI_dive/code:id1}}\end{figure}


\subsubsection{特征工程}
\label{\detokenize{chapter_AI_dive/feature_engineering:id1}}\label{\detokenize{chapter_AI_dive/feature_engineering::doc}}
并非所有的数据都对我们用,所以我们要甄别,这个过程叫特征工程\sphinxhref{https://mp.weixin.qq.com/s/URwrV1WjKC6y\_UH\_5IWHyQ}{2}%
\begin{footnote}[918]\sphinxAtStartFootnote
\sphinxnolinkurl{https://mp.weixin.qq.com/s/URwrV1WjKC6y\_UH\_5IWHyQ}
%
\end{footnote}

特征工程主要包含如下几个环节:数据观察—>数据清洗—>特征加工—>特征选择—>特征规约。好的特征工程不仅需要我们对模型算法有深入的理解,还要有较强的专业领域知识。我们将特征工程主要涵盖的环节以及各环节需要解决的问题总结在下方的导图中,以供大家参考。
\sphinxhref{https://zhuanlan.zhihu.com/p/144488073}{1}%
\begin{footnote}[919]\sphinxAtStartFootnote
\sphinxnolinkurl{https://zhuanlan.zhihu.com/p/144488073}
%
\end{footnote}

\begin{figure}[H]
\centering
\capstart

\noindent\sphinxincludegraphics{{feature_engineer}.png}
\caption{特征工程}\label{\detokenize{chapter_AI_dive/feature_engineering:id2}}\end{figure}


\paragraph{Python}
\label{\detokenize{chapter_AI_dive/feature_engineering:python}}
选定了特征之后,我们来生成特征矩阵X,把刚才我们决定保留的特征都写进来。

常用代码\sphinxhref{https://mp.weixin.qq.com/s/URwrV1WjKC6y\_UH\_5IWHyQ}{2}%
\begin{footnote}[920]\sphinxAtStartFootnote
\sphinxnolinkurl{https://mp.weixin.qq.com/s/URwrV1WjKC6y\_UH\_5IWHyQ}
%
\end{footnote}:

import numpy as np import pandas as pd df =
pd.read\_csv(‘customer\_churn.csv’) X = df.loc{[}:,{[}‘CreditScore’,
‘Geography’, ‘Gender’, ‘Age’, ‘Tenure’, ‘Balance’, ‘NumOfProducts’,
‘HasCrCard’, ‘IsActiveMember’, ‘EstimatedSalary’{]}{]}


\subsubsection{ML 1\sphinxfootnotemark[921]}
\label{\detokenize{chapter_AI_dive/ML:ml-1}}\label{\detokenize{chapter_AI_dive/ML::doc}}%
\begin{footnotetext}[921]\sphinxAtStartFootnote
\sphinxnolinkurl{https://www.pianshen.com/article/66921228716/}
%
\end{footnotetext}\ignorespaces 

\paragraph{与AI:raw\sphinxhyphen{}latex:\sphinxtitleref{DL}:raw\sphinxhyphen{}latex:{\color{red}\bfseries{}`}Data {\color{red}\bfseries{}`}Mining等的关系}
\label{\detokenize{chapter_AI_dive/ML:ai-raw-latex-dl-raw-latex-data-mining}}
\begin{figure}[H]
\centering
\capstart

\noindent\sphinxincludegraphics{{ML_relate2AI}.png}
\caption{与AI:raw\sphinxhyphen{}latex:\sphinxtitleref{DL}{\color{red}\bfseries{}:raw\sphinxhyphen{}latex:`\textbackslash{}Data `Mining等的关系\textbackslash{} `10 <http://www.mysecretrainbow.com/ai/17264.html>`\_\_}}\label{\detokenize{chapter_AI_dive/ML:id22}}
\begin{sphinxlegend}\end{sphinxlegend}
\end{figure}


\paragraph{类别}
\label{\detokenize{chapter_AI_dive/ML:id7}}
第一类是构造间隔理论分布:聚类分析和模式识别
\begin{itemize}
\item {} 
人工神经网络

\item {} 
决策树

\item {} 
感知器

\item {} 
支持向量机

\item {} 
集成学习AdaBoost

\item {} 
降维与度量学习

\item {} 
聚类

\item {} 
贝叶斯分类器

\end{itemize}

第二类是构造条件概率:回归分析和统计分类
\begin{itemize}
\item {} 
高斯过程回归

\item {} 
线性判别分析

\item {} 
最近邻居法

\item {} 
径向基函数核

\end{itemize}

第三类是通过再生模型构造概率密度函数
\begin{itemize}
\item {} 
最大期望算法

\item {} 
概率图模型:包括贝叶斯网和Markov随机场

\item {} 
Generative Topographic Mapping

\end{itemize}

第四类是近似推断技术
\begin{itemize}
\item {} 
马尔可夫链

\item {} 
蒙特卡罗方法

\item {} 
变分法

\end{itemize}

\begin{figure}[H]
\centering
\capstart

\noindent\sphinxincludegraphics{{ML}.png}
\caption{ML\sphinxhref{http://www.uml.org.cn/devprocess/201910163.asp}{3}\sphinxfootnotemark[922]}\label{\detokenize{chapter_AI_dive/ML:id23}}\end{figure}
%
\begin{footnotetext}[922]\sphinxAtStartFootnote
\sphinxnolinkurl{http://www.uml.org.cn/devprocess/201910163.asp}
%
\end{footnotetext}\ignorespaces 

\paragraph{ML VS software engineering}
\label{\detokenize{chapter_AI_dive/ML:ml-vs-software-engineering}}
术语“软件工程”第一次出现是在1965年,也就是编程语言开始出现的15年后。大约20年后,软件工程研究所成立,以管理软件工程过程。今天,我们已经普遍接受了软件工程的最佳实践。另一方面,机器学习直到20世纪90年代才作为一个独立的领域开始蓬勃发展。深度学习是ML的一个子集,它在许多问题上的准确性创造了新纪录,包括图像识别和自然语言处理,直到2012年AlexNet的兴起才被广泛讨论。与软件工程相比,ML仍处于起步阶段,因此缺乏行业标准、度量标准、基础设施和工具。公司仍在探索最佳实践,扼杀应用程序。\sphinxhref{https://radiant-brushlands-42789.herokuapp.com/towardsdatascience.com/how-to-manage-machine-learning-products-part-1-386e7011258a}{5}%
\begin{footnote}[923]\sphinxAtStartFootnote
\sphinxnolinkurl{https://radiant-brushlands-42789.herokuapp.com/towardsdatascience.com/how-to-manage-machine-learning-products-part-1-386e7011258a}
%
\end{footnote}


\paragraph{选择}
\label{\detokenize{chapter_AI_dive/ML:id8}}
\begin{figure}[H]
\centering
\capstart

\noindent\sphinxincludegraphics{{ML_cheat_sheet}.png}
\caption{Machine Learning Algorithm Cheat
Sheet\sphinxhref{https://docs.microsoft.com/en-us/azure/machine-learning/algorithm-cheat-sheet}{8}\sphinxfootnotemark[924]}\label{\detokenize{chapter_AI_dive/ML:id24}}\end{figure}
%
\begin{footnotetext}[924]\sphinxAtStartFootnote
\sphinxnolinkurl{https://docs.microsoft.com/en-us/azure/machine-learning/algorithm-cheat-sheet}
%
\end{footnotetext}\ignorespaces 
如何选择?\sphinxhref{https://docs.microsoft.com/en-us/azure/machine-learning/how-to-select-algorithms}{9}%
\begin{footnote}[925]\sphinxAtStartFootnote
\sphinxnolinkurl{https://docs.microsoft.com/en-us/azure/machine-learning/how-to-select-algorithms}
%
\end{footnote}

在具体算法选择上,基于Python的scikit\sphinxhyphen{}learn机器学习算法库提供一套算法选择方法,参考这一部分(不局限于图中的算法和方法,由于这张图大多考虑了scikit中算法的实现情况)具体介绍一下算法的选择如下:
\sphinxhref{https://zhuanlan.zhihu.com/p/36870462}{4}%
\begin{footnote}[926]\sphinxAtStartFootnote
\sphinxnolinkurl{https://zhuanlan.zhihu.com/p/36870462}
%
\end{footnote}
\begin{enumerate}
\sphinxsetlistlabels{\arabic}{enumi}{enumii}{}{.}%
\item {} 
首先统计数据的容量当数据过小(小于50条)时,建议收集更多的数据,因为过小的数据训练的算法容易受噪声的影响比较大,算法效果一般。

\item {} 
判断是否为预测一个类别的问题,如果是并且训练数据中包含标签信息则为分类问题。

\item {} 
如果是预测一个类别的问题但是训练数据中不包含标签信息则是一个聚类问题

\item {} 
如果不是一个分类问题,是预测一个具体的数值问题一般为回归问题,如果不是预测具体数值对数据进行分析,挖掘数据中的异常值等问题,这时可以考虑一下是否为降维问题。

\item {} 
对于分类问题,如果数据量小于100k,建议用线性SVM的方法,如果效果不好根据是否为文本信息考虑用贝叶斯方法或者K临近分类法。如果数据量过大可以考虑加入正则化的方法来防止过拟合的问题来保证模型的稳定性。

\item {} 
对于聚类问题,如果我们知道需要划分的数据集个数一般使用Kmeans等聚类方法即可。如果无法获知聚类的个数一般使用mean\sphinxhyphen{}shift的基于密度的算法可以对模型进行聚类评估,

\item {} 
对于回归问题,如果数据量不大,直接使用SVM之类的回归即可,当然如果数据量过大可以考虑使用L1,L2的正则化方法来对权值进行正则化来防止过拟合问题的出现。这部分算法的选择与分类问题很相似。

\item {} 
对于降维问题,如果是考虑为分类问题的输入维度进行削减,一般考虑LDA方法可以很好的对每个类别上的数据进行降维处理。如果单纯对输入维度进行降维,将原有维度信息转移到新的维度(根据维度的正交化来达到降维的目的)一般使用PCA方法是比较主流的方法。

\end{enumerate}

对于算法的选择,有时不能找到确定的方法,也就是说很难根据数据是使用场景就完全锁定了那一个具体的算法,但是根据却可以缩小到指定的几个常用算法。然后通过测试集和训练集在这几个算法上做一些Demo。根据Demo反应的质量决定最终使用的算法那个。看似比较费力,其实是比较稳妥和精准的方法。


\paragraph{机器学习框架 6\sphinxfootnotemark[927]}
\label{\detokenize{chapter_AI_dive/ML:id9}}%
\begin{footnotetext}[927]\sphinxAtStartFootnote
\sphinxnolinkurl{https://zhuanlan.zhihu.com/p/192633890}
%
\end{footnotetext}\ignorespaces 
机器学习方面的新框架层出不穷,一方面我们需要掌握经典框架的使用方式,理解其模块构成,接口规范的设计,一定程度上来说其它新框架也都需要遵循这些业界标准框架的模块与接口定义。另一方面对于新框架或特定领域框架,我们需要掌握快速评估,上手使用,并且做一定改造适配的能力。一些比较经典的框架有:
\begin{itemize}
\item {} 
通用机器学习:scikit\sphinxhyphen{}learn,Spark ML,LightGBM

\item {} 
通用深度学习:Keras/TensorFlow,PyTorch

\item {} 
特征工程:tsfresh, Featuretools,Feast

\item {} 
AutoML:hyperopt,SMAC3,nni,autogluon

\item {} 
可解释机器学习:shap,aix360,eli5,interpret

\item {} 
异常检测:pyod,egads

\item {} 
可视化:pyecharts,seaborn

\item {} 
数据质量:cerberus,pandas\_profiling,Deequ

\item {} 
时间序列:fbprophet,sktime,pyts

\item {} 
大规模机器学习:Horovod,BigDL,mmlspark

\item {} 
Pipeline:MLflow, metaflow,KubeFlow,Hopsworks

\item {} 
迁移学习:\sphinxurl{http://www.mysecretrainbow.com/ai/17262.html\#1}

\end{itemize}

一般的学习路径主要是阅读这些框架的官方文档和tutorial,在自己的项目中进行尝试使用。对于一些核心接口,也可以阅读一下相关的源代码,深入理解其背后的原理。

\begin{figure}[H]
\centering
\capstart

\noindent\sphinxincludegraphics{{chapter_AI_dive/../img/ML_lifecycle}.png}
\caption{Machine Learning
Lifecycle\sphinxhref{https://databricks.com/solutions/machine-learning}{7}\sphinxfootnotemark[928]}\label{\detokenize{chapter_AI_dive/ML:id25}}\end{figure}
%
\begin{footnotetext}[928]\sphinxAtStartFootnote
\sphinxnolinkurl{https://databricks.com/solutions/machine-learning}
%
\end{footnotetext}\ignorespaces 

\paragraph{指标}
\label{\detokenize{chapter_AI_dive/ML:id10}}\begin{itemize}
\item {} 
准确率(P值)是针对我们预测结果而言的,它表示的是预测为正的样本中有多少是真正的正样本。

\item {} 
召回率(R值)是针对我们原来的样本而言的,它表示的是样本中的正例有多少被预测正确了。

\end{itemize}


\paragraph{流程 20\sphinxfootnotemark[929]}
\label{\detokenize{chapter_AI_dive/ML:id11}}%
\begin{footnotetext}[929]\sphinxAtStartFootnote
\sphinxnolinkurl{http://www.woshipm.com/pmd/2942899.html}
%
\end{footnotetext}\ignorespaces 

\subparagraph{目标定义}
\label{\detokenize{chapter_AI_dive/ML:id12}}
确认机器学习要解决的问题本质以及衡量的标准。

机器学习的目标可以被分为:分类、回归、聚类、异常检测等。


\subparagraph{数据采集}
\label{\detokenize{chapter_AI_dive/ML:id13}}
原始数据作为机器学习过程中的输入来源是从各种渠道中被采集而来的。


\subparagraph{数据预处理}
\label{\detokenize{chapter_AI_dive/ML:id14}}
普通数据挖掘中的预处理包括数据清洗、数据集成、数据转换、数据削减、数据离散化。

深度学习数据预处理包含数据归一化(包含样本尺度归一化、逐样本的均值相减、标准化)和数据白化。需要将数据分为三种数据集,包括用来训练模型的训练集(training
set),开发过程中用于调参(parameter tuning)的验证集(validation
set)以及测试时所使用的测试集(test set)。

数据标注的质量对于算法的成功率至关重要。


\subparagraph{模型训练}
\label{\detokenize{chapter_AI_dive/ML:id15}}
模型训练流程:每当有数据输入,模型都会输出预测结果,而预测结果会用来调整和更新W和B的集合,接着训练新的数据,直到训练出可以预测出接近真实结果的模型。


\subparagraph{准确率测试}
\label{\detokenize{chapter_AI_dive/ML:id16}}
用第三步数据预处理中准备好的测试集对模型进行测试。


\subparagraph{调参}
\label{\detokenize{chapter_AI_dive/ML:id17}}
参数可以分为两类,一类是需要在训练(学习)之前手动设置的参数,即超参数(hypeparameter),另外一类是通常不需要手动设置、在训练过程中可以被自动调整的参数(parameter)。

调参通常需要依赖经验和灵感来探寻其最优值,本质上更接近艺术而非科学,是考察算法工程师能力高低的重点环节。


\subparagraph{模型输出}
\label{\detokenize{chapter_AI_dive/ML:id18}}
模型最终输出应用于实际应用场景的接口或数据集。


\paragraph{金融应用}
\label{\detokenize{chapter_AI_dive/ML:id19}}
利用传统的回归分析等方法来建模交易策略有两个弊端:首先,所用数据维度有限,仅限于交易数据;其次,模型可处理的变量有限,模型的有效与否取决于所选取变量的特征和变量间的组合,而这很大程度上取决于研究员对数据的敏感程度。利用机器学习技术,结合预测算法,可以依据历史经验和新的市场信息不断演化,预测股票、债券等金融资产价格的波动及波动间的相互关系,以此来创建符合预期风险收益的投资组合。然而,机器学习可能是个相对缓慢的过程,且该过程无法通过其他统计方法来提供担保行为。机器学习虽可能适用于寻找隐藏的趋势、信息和关系,但在金融领域的应用和效果仍存在较大不确定性。市场上对于金融领域的机器学习仍存在一定程度的炒作。\sphinxhref{http://www.cstf.org.cn/newsdetail.asp?types=36\&num=1165}{21}%
\begin{footnote}[930]\sphinxAtStartFootnote
\sphinxnolinkurl{http://www.cstf.org.cn/newsdetail.asp?types=36\&num=1165}
%
\end{footnote}


\paragraph{更多}
\label{\detokenize{chapter_AI_dive/ML:id20}}
\sphinxurl{https://yulinzhao.wordpress.com/category/3\_machine-learning/}

\sphinxurl{https://mitpress.ublish.com/ereader/7093/?preview=\#page/}
\begin{enumerate}
\sphinxsetlistlabels{\arabic}{enumi}{enumii}{}{.}%
\item {} 
\sphinxurl{https://www.jianshu.com/p/ed9ae5385b89}

\item {} 
\sphinxurl{https://www.jianshu.com/p/0359e3b7bb1b}

\end{enumerate}

《美团机器学习实践》笔记:\sphinxurl{https://www.dazhuanlan.com/2019/10/17/5da8114a3b457/}

至少你要知道什么是二分类问题,什么是ground
truth、熵(entropy)的概念,dynamic
learning的概念等等。\sphinxhref{http://www.uml.org.cn/DevProcess/201712283.asp}{2}%
\begin{footnote}[931]\sphinxAtStartFootnote
\sphinxnolinkurl{http://www.uml.org.cn/DevProcess/201712283.asp}
%
\end{footnote}
\begin{itemize}
\item {} 
\sphinxurl{https://www.reddit.com/r/MachineLearning/}

\item {} 
\sphinxurl{http://www.mlebook.com/wiki/doku.php}

\end{itemize}

不容错过的 20
个机器学习与数据科学网站\sphinxhref{https://libertydream.github.io/2020/02/16/20\%E4\%B8\%AA\%E4\%B8\%8D\%E5\%AE\%B9\%E9\%94\%99\%E8\%BF\%87\%E7\%9A\%84AI\%E8\%B5\%84\%E6\%BA\%90/}{11}%
\begin{footnote}[932]\sphinxAtStartFootnote
\sphinxnolinkurl{https://libertydream.github.io/2020/02/16/20\%E4\%B8\%AA\%E4\%B8\%8D\%E5\%AE\%B9\%E9\%94\%99\%E8\%BF\%87\%E7\%9A\%84AI\%E8\%B5\%84\%E6\%BA\%90/}
%
\end{footnote}
\begin{itemize}
\item {} 
Andreas Mueller
的这门免费的课程《应用机器学习》\sphinxhref{https://www.infoq.cn/article/IPDVRNxwJVsx3ZGrgwzW}{16}%
\begin{footnote}[933]\sphinxAtStartFootnote
\sphinxnolinkurl{https://www.infoq.cn/article/IPDVRNxwJVsx3ZGrgwzW}
%
\end{footnote}(Applied
Machine Learning)。

\item {} 
Coursera:《机器学习》\sphinxhref{https://www.coursera.org/learn/machine-learning?utm\_source=gg\&utm\_medium=sem\&utm\_content=07-StanfordML-US\&campaignid=685340575\&adgroupid=32639001781\&device=c\&keyword=machine\%20learning\%20programming\%20tutorial\&matchtype=b\&network=g\&devicemodel=\&adpostion=\&creativeid=243289762754\&hide\_mobile\_promo\&gclid=EAIaIQobChMIhY7m1Pfa6wIVtR6tBh0UCQAJEAAYASAAEgJcV\_D\_BwEhttps://www.coursera.org/learn/machine-learning?utm\_source=gg\&utm\_medium=sem\&utm\_content=07-StanfordML-US\&campaignid=685340575\&adgroupid=32639001781\&device=c\&keyword=machine\%20learning\%20programming\%20tutorial\&matchtype=b\&network=g\&devicemodel=\&adpostion=\&creativeid=243289762754\&hide\_mobile\_promo\&gclid=EAIaIQobChMIhY7m1Pfa6wIVtR6tBh0UCQAJEAAYASAAEgJcV\_D\_BwE}{17}%
\begin{footnote}[934]\sphinxAtStartFootnote
\sphinxnolinkurl{https://www.coursera.org/learn/machine-learning?utm\_source=gg\&utm\_medium=sem\&utm\_content=07-StanfordML-US\&campaignid=685340575\&adgroupid=32639001781\&device=c\&keyword=machine\%20learning\%20programming\%20tutorial\&matchtype=b\&network=g\&devicemodel=\&adpostion=\&creativeid=243289762754\&hide\_mobile\_promo\&gclid=EAIaIQobChMIhY7m1Pfa6wIVtR6tBh0UCQAJEAAYASAAEgJcV\_D\_BwEhttps://www.coursera.org/learn/machine-learning?utm\_source=gg\&utm\_medium=sem\&utm\_content=07-StanfordML-US\&campaignid=685340575\&adgroupid=32639001781\&device=c\&keyword=machine\%20learning\%20programming\%20tutorial\&matchtype=b\&network=g\&devicemodel=\&adpostion=\&creativeid=243289762754\&hide\_mobile\_promo\&gclid=EAIaIQobChMIhY7m1Pfa6wIVtR6tBh0UCQAJEAAYASAAEgJcV\_D\_BwE}
%
\end{footnote}(Machine
Learning),吴恩达(Andrew Ng)。

\item {} 
优达学城(Udacity):机器学习工程纳米学位\sphinxhref{https://www.udacity.com/course/machine-learning-engineer-nanodegree--nd009t}{18}%
\begin{footnote}[935]\sphinxAtStartFootnote
\sphinxnolinkurl{https://www.udacity.com/course/machine-learning-engineer-nanodegree--nd009t}
%
\end{footnote}(Machine
Learning Engineering Nanodegree)。

\item {} 
Google速成ML课程\sphinxhref{https://developers.google.com/machine-learning/crash-course/static-vs-dynamic-training/video-lecture?hl=zh-cn\#:~:text=\%E4\%BB\%8E\%E5\%B9\%BF\%E4\%B9\%89\%E4\%B8\%8A\%E8\%AE\%B2\%EF\%BC\%8C\%E8\%AE\%AD\%E7\%BB\%83,\%E6\%A8\%A1\%E5\%9E\%8B\%E9\%87\%87\%E7\%94\%A8\%E5\%9C\%A8\%E7\%BA\%BF\%E8\%AE\%AD\%E7\%BB\%83\%E6\%96\%B9\%E5\%BC\%8F\%E3\%80\%82}{19}%
\begin{footnote}[936]\sphinxAtStartFootnote
\sphinxnolinkurl{https://developers.google.com/machine-learning/crash-course/static-vs-dynamic-training/video-lecture?hl=zh-cn\#:~:text=\%E4\%BB\%8E\%E5\%B9\%BF\%E4\%B9\%89\%E4\%B8\%8A\%E8\%AE\%B2\%EF\%BC\%8C\%E8\%AE\%AD\%E7\%BB\%83,\%E6\%A8\%A1\%E5\%9E\%8B\%E9\%87\%87\%E7\%94\%A8\%E5\%9C\%A8\%E7\%BA\%BF\%E8\%AE\%AD\%E7\%BB\%83\%E6\%96\%B9\%E5\%BC\%8F\%E3\%80\%82}
%
\end{footnote}

\end{itemize}


\paragraph{技术债务}
\label{\detokenize{chapter_AI_dive/ML:id21}}
“技术债务”,指为了产品快速迭代,做了很多临时性的代码处理。但是在未来的某一天,这些遗留问题都会以BUG方式体现出来,导致付出更大的维护成本\sphinxhref{http://www.woshipm.com/pmd/3024508.html}{14}%
\begin{footnote}[937]\sphinxAtStartFootnote
\sphinxnolinkurl{http://www.woshipm.com/pmd/3024508.html}
%
\end{footnote}

技术债务或许可以通过重构代码,改善单元测试,删除僵尸代码,减少依赖,精简
API
和改良文档说明进行清算。其目的不在于添加新功能,而是着眼于未来的提升,减少错误和提高可维护性。延期偿还只会加重负担,隐性债务之所以危险正是因为它是悄无声息间积攒下的。\sphinxhref{https://libertydream.github.io/2020/05/10/ML\%E9\%9A\%90\%E6\%80\%A7\%E5\%80\%BA\%E5\%8A\%A1/}{12}%
\begin{footnote}[938]\sphinxAtStartFootnote
\sphinxnolinkurl{https://libertydream.github.io/2020/05/10/ML\%E9\%9A\%90\%E6\%80\%A7\%E5\%80\%BA\%E5\%8A\%A1/}
%
\end{footnote}


\paragraph{ML system}
\label{\detokenize{chapter_AI_dive/ML:ml-system}}
机器学习系统在以下方面与其他软件系统不同:\sphinxhref{https://cloud.google.com/solutions/machine-learning/mlops-continuous-delivery-and-automation-pipelines-in-machine-learning?hl=zh-cn}{13}%
\begin{footnote}[939]\sphinxAtStartFootnote
\sphinxnolinkurl{https://cloud.google.com/solutions/machine-learning/mlops-continuous-delivery-and-automation-pipelines-in-machine-learning?hl=zh-cn}
%
\end{footnote}
\begin{itemize}
\item {} 
团队技能:在机器学习项目中,团队通常包括数据科学家或机器学习研究人员,他们主要负责进行探索性数据分析、模型开发和实验。这些成员可能不是经验丰富的、能够构建生产级服务的软件工程师。

\item {} 
开发:机器学习在本质上具有实验性。您应该尝试不同的特征、算法、建模技术和参数配置,以便尽快找到问题的最佳解决方案。您所面临的挑战在于跟踪哪些方案有效、哪些方案无效,并在最大程度提高代码重复使用率的同时维持可重现性。

\item {} 
测试:测试机器学习系统比测试其他软件系统更复杂。除了典型的单元测试和集成测试之外,您还需要验证数据、评估经过训练的模型质量以及验证模型。

\item {} 
部署:在机器学习系统中,部署不是将离线训练的机器学习模型部署为预测服务那样简单。机器学习系统可能会要求您部署多步骤流水线以自动重新训练和部署模型。此流水线会增加复杂性,并要求您自动执行部署之前由数据科学家手动执行的步骤,以训练和验证新模型。

\item {} 
生产:机器学习模型的性能可能会下降,不仅是因为编码不理想,而且也因为数据资料在不断演变。换句话说,与传统的软件系统相比,模型可能会通过更多方式衰退,而您需要考虑这种降级现象。因此,您需要跟踪数据的摘要统计信息并监控模型的在线性能,以便系统在值与预期不符时发送通知或回滚。

\end{itemize}

More\sphinxhref{https://libertydream.github.io/2020/05/10/ML\%E9\%9A\%90\%E6\%80\%A7\%E5\%80\%BA\%E5\%8A\%A1/}{15}%
\begin{footnote}[940]\sphinxAtStartFootnote
\sphinxnolinkurl{https://libertydream.github.io/2020/05/10/ML\%E9\%9A\%90\%E6\%80\%A7\%E5\%80\%BA\%E5\%8A\%A1/}
%
\end{footnote}


\subsubsection{MLOps}
\label{\detokenize{chapter_AI_dive/MLOps:mlops}}\label{\detokenize{chapter_AI_dive/MLOps::doc}}

\paragraph{什么是MLOps}
\label{\detokenize{chapter_AI_dive/MLOps:id1}}
MLOps就是机器学习时代的DevOps。它的主要作用就是连接模型构建团队和业务,运维团队,建立起一个标准化的模型开发,部署与运维流程,使得企业组织能更好的利用机器学习的能力来促进业务增长。

最近几年,大家对于机器学习项目落地愈发重视起来,对业务的理解,模型应用流程等都做的越来越好,也有越来越多的模型被部署到真实的业务场景中。但是当业务真实开始使用的时候,就会对模型有各种各样的需求反馈,算法工程师们就开始需要不断迭代开发,频繁部署上线。随着业务的发展,模型应用的场景也越来越多,管理和维护这么多模型系统就成了一个切实的挑战。

回顾这个发展,是不是感觉似曾相识?20年前软件行业在数字化演进道路上也遇到过类似的挑战。我们从部署一个Web服务到要部署几十甚至上百个不同的应用,在各种规模化交付方面的挑战之下,诞生了DevOps技术。像虚拟化,云计算,持续集成/发布,自动化测试等软件工程领域的各类最佳实践基本都跟这个方向有关。在不远的将来,或许智能模型也会与今天的软件系统一样普遍。一个企业需要使用非常多的业务系统来实现数字化流程,同样也需要非常多的模型来实现数据驱动的智能决策,衍生出更多与模型相关的开发运维,权限,隐私,安全性,审计等企业级需求。

因此最近几年,MLOps也逐渐成为了一个热门话题。有了好的MLOps实践,算法工程师一方面能更专注于擅长的模型构建过程,减少对模型部署运维等方面的“感知”,另一方面也让模型开发迭代的方向更加清晰明确,切实为业务产生价值。就像今日的软件工程师很少需要关注运行环境,测试集成,发布流程等细节,但却做到了一天数次发布的敏捷高效,未来算法工程师应该也能更专注于数据insights获取方面,让模型发布成为几乎无感又快速的自动化流程。


\paragraph{MLOps的原则}
\label{\detokenize{chapter_AI_dive/MLOps:id2}}

\subparagraph{Automation}
\label{\detokenize{chapter_AI_dive/MLOps:automation}}
在整个workflow中所有可以自动化的环节,我们都应该进行自动化,从数据的接入到最后的部署上线。Google那篇经典的MLOps指导中就提出了3个层级的自动化,非常值得借鉴,后面我们会详细介绍。


\subparagraph{Continuous}
\label{\detokenize{chapter_AI_dive/MLOps:continuous}}
一说起DevOps,大家就很容易联想到CI/CD,也从侧面印证这条原则的重要性。MLOps在持续集成,持续部署,持续监控的基础上,还增加了持续训练的概念,即模型在线上运行过程中可以持续得到自动化的训练与更新。我们在设计开发机器学习系统时,要持续思考各个组件对“持续”性的支持,包括流程中用到的各种artifacts,他们的版本管理和编排串联等。


\subparagraph{Versioning}
\label{\detokenize{chapter_AI_dive/MLOps:versioning}}
版本化管理也是DevOps的重要最佳实践之一,在MLOps领域,除了pipeline代码的版本管理,数据,模型的版本管理属于新涌现的需求点,也对底层infra提出了新的挑战。


\subparagraph{Experiment Tracking}
\label{\detokenize{chapter_AI_dive/MLOps:experiment-tracking}}
实验管理可以理解为version control中commit
message的增强。对于涉及模型构建相关的代码改动,我们都应该能记录当时对应的数据,代码版本,以及对应的模型artifacts存档,作为后续分析模型,选择具体上线的版本的重要依据。


\subparagraph{Testing}
\label{\detokenize{chapter_AI_dive/MLOps:testing}}
机器学习系统中主要涉及到3种不同的pipeline,分别是数据pipeline,模型pipeline和应用pipeline(类似于模型与应用系统的集成)。针对这3个pipeline,需要构建对应的数据特征测试,模型测试以及应用infra测试,确保整体系统的输出与预期的业务目标相符,达到将数据insights转化为业务价值的目的。这方面Google的ML
test score是一个很好的参考。


\subparagraph{Monitoring}
\label{\detokenize{chapter_AI_dive/MLOps:monitoring}}
监控也是一项软件工程的传统最佳实践。上面提到的ML test
score中也有一部分是与监控相关。除了传统的系统监控,例如日志,系统资源等方面外,机器学习系统还需要对输入数据,模型预测进行监控,确保预测的质量,并在出现异常情况时自动触发一些应对机制,例如数据或模型的降级,模型的重新训练与部署等。


\paragraph{模型部署}
\label{\detokenize{chapter_AI_dive/MLOps:id3}}
通过测试后,我们就可以把模型部署上线啦。这里又根据业务形态的不同分成很多不同的类型,例如:
\begin{itemize}
\item {} 
Batch预测pipeline

\item {} 
实时模型服务

\item {} 
Edge device部署,如手机app,浏览器等

\end{itemize}

这方面涉及到的话题也非常多,包括模型的序列化,推理性能优化,模型压缩等等。

模型部署的assets除了模型本身外,也需要包含end\sphinxhyphen{}to\sphinxhyphen{}end测试用例,
测试数据和相应的结果评估等。可以在模型部署完成后再执行一遍相关测试用例,确保开发和部署环境中得到的结果一致。

对于输出较为critical的模型,还需要考虑一系列model
governance的需求满足。例如在模型部署前需要进行各类人工审核,并设计相应的sign\sphinxhyphen{}off机制。顺带一提responsible
AI近年来也是越来越受到重视,在MLOps中的各个环节也需要关注相应功能的支持。


\paragraph{模型监控}
\label{\detokenize{chapter_AI_dive/MLOps:id4}}
最后,对于线上模型的运行,我们需要持续进行监控,包括:
\begin{itemize}
\item {} 
模型依赖组件的监控,例如数据版本,上游系统等

\item {} 
模型输入数据的监控,确保schema与分布的一致性

\item {} 
离线特征构建与线上特征构建输出的一致性监控,例如可以对一些样本进行抽样,比对线上线下结果,或者监控分布统计值

\item {} 
模型数值稳定性的监控,对NaN和Inf等情况进行记录

\item {} 
模型计算资源开销方面的监控

\item {} 
模型metric方面的监控

\item {} 
模型更新周期的监控,确保没有长时间未更新的模型

\item {} 
下游消费数量的监控,确保没有处于“废弃”状态的模型

\item {} 
对于排查问题有用的日志记录

\item {} 
对于提升模型有用的信息记录

\item {} 
外界攻击预防监控

\item {} 
上述的各类监控都要配合相应的自动/人工应对机制。

\end{itemize}

以模型效果监控为例,当效果出现下降时,我们需要及时介入排查处理,或触发重训练。对于重训练来说,需要综合考虑模型效果变化,数据更新频率,训练开销,部署开销,重新训练的提升度等,选择合适的时间点进行触发。虽然有很多模型也支持在线实时更新,但其稳定性控制,自动化测试等都缺少标准做法的参考,大多数情况下,重新训练往往比在线更新训练的效果和稳定性更好。

TODO:https://cloud.google.com/solutions/machine\sphinxhyphen{}learning/mlops\sphinxhyphen{}continuous\sphinxhyphen{}delivery\sphinxhyphen{}and\sphinxhyphen{}automation\sphinxhyphen{}pipelines\sphinxhyphen{}in\sphinxhyphen{}machine\sphinxhyphen{}learning?hl=zh\sphinxhyphen{}cn


\subsubsection{深度学习算法}
\label{\detokenize{chapter_AI_dive/DL:id1}}\label{\detokenize{chapter_AI_dive/DL::doc}}

\paragraph{DL 模型的表达能力}
\label{\detokenize{chapter_AI_dive/DL:dl}}
相较传统统计模型来说,深度神经网络有着数量众多的参数。如果用 MDL
衡量深度神经网络的复杂度,并将参数数量视为模型描述长度,那模型看起来可会惨不忍睹。模型描述
L(H) 很容易失控性疯涨。

但为使表达能力够强,对神经网络而言这许许多多的参数
必不可少。正因为神经网络对灵活多样的数据表示具备出色的捕获能力,它才能在许多应用中取得辉煌战绩。\sphinxhref{https://libertydream.github.io/2020/06/28/\%E6\%B7\%B1\%E5\%BA\%A6\%E5\%AD\%A6\%E4\%B9\%A0\%E6\%A8\%A1\%E5\%9E\%8B\%E4\%B8\%BA\%E4\%BB\%80\%E4\%B9\%88\%E6\%B2\%A1\%E8\%BF\%87\%E6\%8B\%9F\%E5\%90\%88/}{4}%
\begin{footnote}[941]\sphinxAtStartFootnote
\sphinxnolinkurl{https://libertydream.github.io/2020/06/28/\%E6\%B7\%B1\%E5\%BA\%A6\%E5\%AD\%A6\%E4\%B9\%A0\%E6\%A8\%A1\%E5\%9E\%8B\%E4\%B8\%BA\%E4\%BB\%80\%E4\%B9\%88\%E6\%B2\%A1\%E8\%BF\%87\%E6\%8B\%9F\%E5\%90\%88/}
%
\end{footnote}


\paragraph{神经网络组成 1\sphinxfootnotemark[942]}
\label{\detokenize{chapter_AI_dive/DL:id2}}%
\begin{footnotetext}[942]\sphinxAtStartFootnote
\sphinxnolinkurl{https://www.yinxiang.com/everhub/note/e7f0c50e-dc27-488f-a9f9-35c121e20bb1}
%
\end{footnotetext}\ignorespaces 
解析:
\begin{itemize}
\item {} 
公式:\(A^{(n+1)}=\delta^{n}\left(W^{n} A^{n}+b^{n}\right)\)

\item {} 
Weights(网络权重):表示不同神经元处理的特征对下个神经元的重要性

\item {} 
bias (偏移):主观偏见的处理

\item {} 
激活函数:对每一层整体的输出做改进,把每层的结果做非线性的变化,去更好地拟合数据分布,或者说来更好地展示给下一层级,常见的激活函数有
ReLU、tanh、Sigmoid 等等,我们用 δn() 表示

\end{itemize}

原理:目标就是在给定一个任务的情况下,找到最优的 Weights 和 bias,使得
Loss 最低。采用反向传播,把 Loss
误差从最后逐层向前传递,使当前层知道自己在哪里,然后再更新当前层的
Weights 权重和 bias 偏移,进而减小最后的误差
\begin{itemize}
\item {} 
常见的深度学习算法:CNN、LSTM、RNN、Seq2Seq、GAN

\item {} 
常用的深度学习框架:TensorFlow、PyTorch

\end{itemize}

解析:

越复杂的图像,采用全连接层的形式,计算量就会变得很大

卷积层提取图片初步特征\sphinxhref{https://coffee.pmcaff.com/article/1909387571608704/pmcaff?utm\_source=forum\&newwindow=1}{5}%
\begin{footnote}[943]\sphinxAtStartFootnote
\sphinxnolinkurl{https://coffee.pmcaff.com/article/1909387571608704/pmcaff?utm\_source=forum\&newwindow=1}
%
\end{footnote}

池化:
\begin{itemize}
\item {} 
目的是提取特征,减少向下一阶段传递的数据量,池化过程的本质是“丢弃”,即只保留图像主体特征,过滤掉无关信息的数据特征

\item {} 
模糊化图像操作:通过 CNN
的卷积和池化,丢弃到无用的特征,识别出关键因素

\item {} 
全连接层的Softmax分类:再卷积池化后通过softmax进行分类

\end{itemize}

全连接层将各部分特征汇总\sphinxhref{https://coffee.pmcaff.com/article/1909387571608704/pmcaff?utm\_source=forum\&newwindow=1}{5}%
\begin{footnote}[944]\sphinxAtStartFootnote
\sphinxnolinkurl{https://coffee.pmcaff.com/article/1909387571608704/pmcaff?utm\_source=forum\&newwindow=1}
%
\end{footnote}

CNN优缺点:
\begin{itemize}
\item {} 
缺点:可解释性弱,模型训练慢,对数据依赖很强,模型复杂

\item {} 
优点:

\end{itemize}
\begin{enumerate}
\sphinxsetlistlabels{\arabic}{enumi}{enumii}{}{.}%
\item {} 
可以拟合任意复杂的数据分布

\item {} 
比我们之前讲过的所有算法的性能都要好

\end{enumerate}


\paragraph{卷积神经网络是最为实用且通常最为正确的方法的一些原因如下:}
\label{\detokenize{chapter_AI_dive/DL:id3}}\begin{enumerate}
\sphinxsetlistlabels{\arabic}{enumi}{enumii}{}{.}%
\item {} 
它们可以通过层传递学习,保存推理并在后续层上创建新的推理。

\item {} 
在使用该算法之前,无需进行特征提取,而是在训练过程中完成的。

\item {} 
它可以识别重要的特征。

\end{enumerate}

但是,它们也有自己的警告。目前已知它们在旋转和缩放方式不同的图像上会失效,但这里的情况并非如此,因为数据已经经过预处理。而且,尽管其他方法在这个数据集上未能提供良好的结果,但它们仍然可以用于其他与图像处理相关的任务(如锐化、平滑等等)。\sphinxhref{https://www.infoq.cn/article/W2koiEheFZEEOv1rOu1d}{7}%
\begin{footnote}[945]\sphinxAtStartFootnote
\sphinxnolinkurl{https://www.infoq.cn/article/W2koiEheFZEEOv1rOu1d}
%
\end{footnote}


\paragraph{历史脉络}
\label{\detokenize{chapter_AI_dive/DL:id4}}
\begin{figure}[H]
\centering
\capstart

\noindent\sphinxincludegraphics{{obj_detect_history}.png}
\caption{目标检测的历史脉络\sphinxhref{https://coffee.pmcaff.com/article/1909387571608704/pmcaff?utm\_source=forum\&newwindow=1}{5}\sphinxfootnotemark[946]}\label{\detokenize{chapter_AI_dive/DL:id10}}\end{figure}
%
\begin{footnotetext}[946]\sphinxAtStartFootnote
\sphinxnolinkurl{https://coffee.pmcaff.com/article/1909387571608704/pmcaff?utm\_source=forum\&newwindow=1}
%
\end{footnotetext}\ignorespaces 

\paragraph{关键问题 8\sphinxfootnotemark[947]}
\label{\detokenize{chapter_AI_dive/DL:id5}}%
\begin{footnotetext}[947]\sphinxAtStartFootnote
\sphinxnolinkurl{http://shujuren.club/a/AI0102.html}
%
\end{footnotetext}\ignorespaces \begin{itemize}
\item {} 
网络结构问题

\item {} 
特征表示问题

\item {} 
海量数据标注问题

\item {} 
数据升维降维问题

\item {} 
反向传播优化问题

\item {} 
损失函数设计问题

\item {} 
过拟合问题

\item {} 
泛化问题

\item {} 
大规模训练性能问题

\end{itemize}


\paragraph{更多}
\label{\detokenize{chapter_AI_dive/DL:id6}}\begin{itemize}
\item {} 
见我的Repo:\sphinxurl{https://github.com/StevenJokess/d2l-en-read/tree/moreme}

\item {} 
NLP:
\sphinxurl{http://reader.epubee.com/books/mobile/0a/0a480147f4d747140345a9a3fda529cf/text00017.html?fromPre=last}

\item {} 
GAN:
\sphinxurl{http://reader.epubee.com/books/mobile/0a/0a480147f4d747140345a9a3fda529cf/text00017.html?fromPre=last}

\item {} 
Reddit:https://www.reddit.com/r/deeplearning/

\item {} 
AI算法工程师手册: \sphinxurl{https://www.bookstack.cn/books/huaxiaozhuan-ai}

\item {} 
CV: \sphinxurl{http://www.uml.org.cn/ai/202012021.asp?artid=23430}

\end{itemize}


\paragraph{学科与应用}
\label{\detokenize{chapter_AI_dive/DL:id7}}
对于深度学习、统计的专家来说,他们更加关注于模型、算法等等,找到可以普适性解决问题的办法。而对于我们应用来说,具体的算法实现不需要我们考虑太多,而是找到适合的场景、合适的模型、匹配的算法,所以应用人工智能实际上是一个计算机、统计、知识工程、行业知识的一个交叉应用。
\sphinxhref{http://www.uml.org.cn/ai/201707041.asp}{3}%
\begin{footnote}[948]\sphinxAtStartFootnote
\sphinxnolinkurl{http://www.uml.org.cn/ai/201707041.asp}
%
\end{footnote}


\paragraph{Robust}
\label{\detokenize{chapter_AI_dive/DL:robust}}
无人驾驶领域,在极端情况会不会做出很不好的答案。


\paragraph{安装}
\label{\detokenize{chapter_AI_dive/DL:id8}}
\sphinxurl{https://www.bilibili.com/video/BV18p4y1h7Dr?p=1}


\paragraph{各框架资源}
\label{\detokenize{chapter_AI_dive/DL:id9}}
Tensorflow:

中文\sphinxhref{http://www.tensorfly.cn/}{6}%
\begin{footnote}[949]\sphinxAtStartFootnote
\sphinxnolinkurl{http://www.tensorfly.cn/}
%
\end{footnote}


\subsubsection{GAN}
\label{\detokenize{chapter_AI_dive/GAN:gan}}\label{\detokenize{chapter_AI_dive/GAN::doc}}
\sphinxurl{https://atcold.github.io/pytorch-Deep-Learning/en/week09/09-3/}

Generator 实际在做的事情就是,对于给定样本分布 Pdata(x),我们希望的我们
PG(x;θ),能够尽可能地接近 Pdata(x),这里的 θ 就是控制 G
的参数,如果是神经网络的话,对应就是各层的
Weights。衡量接近的指标就是上面的 L,其含义就是如果我们从 Pdata(x) 中
采样 m 个 x\_i,那么在给定 θ 的情况下,我们可以计算出从 PG 中 采样出 x\_i
的几率,将它们连乘得到 L。我们的目的就是找到一个 θ\_star,使得 L
能够最大。

将 log(1\sphinxhyphen{}D(x)) 改换为 \sphinxhyphen{}log(D(x))


\paragraph{DCGAN}
\label{\detokenize{chapter_AI_dive/GAN:dcgan}}\begin{enumerate}
\sphinxsetlistlabels{\arabic}{enumi}{enumii}{}{.}%
\item {} 
去掉了G网络和D网络中的pooling layer。

\item {} 
在G网络和D网络中都使用Batch Normalization

\item {} 
去掉全连接的隐藏层

\item {} 
在G网络中除最后一层使用RELU,最后一层使用Tanh

\item {} 
在D网络中每一层使用LeakyRELU。
\sphinxhref{https://blog.csdn.net/xiaoqianlizhen/article/details/81536537?spm=1001.2014.3001.5501}{2}%
\begin{footnote}[950]\sphinxAtStartFootnote
\sphinxnolinkurl{https://blog.csdn.net/xiaoqianlizhen/article/details/81536537?spm=1001.2014.3001.5501}
%
\end{footnote}

\end{enumerate}

生成对抗网络在各领域应用研究进展:
\sphinxurl{http://www.aas.net.cn/cn/article/doi/10.16383/j.aas.c180831}


\paragraph{GAN为什么在文本上效果不佳?}
\label{\detokenize{chapter_AI_dive/GAN:id1}}\begin{itemize}
\item {} 
图像和文本的核心区别在于图像的 Pixel 表示是连续的,而文本是由离散的
token 组成

\item {} 
参数的微小改变不能对结果产生影响,或者说影响的方向也不对,这就导致
Discriminator 的梯度回传变得没有意义

\end{itemize}


\paragraph{Generative Adversarial Networks for Text1\sphinxfootnotemark[951]}
\label{\detokenize{chapter_AI_dive/GAN:generative-adversarial-networks-for-text1}}%
\begin{footnotetext}[951]\sphinxAtStartFootnote
\sphinxnolinkurl{https://www.reddit.com/r/MachineLearning/comments/40ldq6/generative\_adversarial\_networks\_for\_text/}
%
\end{footnotetext}\ignorespaces 

\subparagraph{GAN for NLG}
\label{\detokenize{chapter_AI_dive/GAN:gan-for-nlg}}

\subparagraph{SeqGAN}
\label{\detokenize{chapter_AI_dive/GAN:seqgan}}
算的上是开山之作,具体的解读可以看我之前的一篇文章 SeqGAN – GAN + RL +
NLP,其通过引入强化学习中的 Policy Gradient 来解决因为离散 token
生成前采样动作造成的不可微。后面的文章也都是基于这个框架来进行深一步地探索。SeqGAN
在 Oracle
和古诗生成任务上做了测试,回过头来看,效果只能说一般。但其开创性的将文本生成看做序列决策问题,
并且将 RL 和 GAN
进行了有机的结合,令人佩服。\sphinxhref{https://tobiaslee.top/2018/04/22/GAN-in-NLP-Notes/}{3}%
\begin{footnote}[952]\sphinxAtStartFootnote
\sphinxnolinkurl{https://tobiaslee.top/2018/04/22/GAN-in-NLP-Notes/}
%
\end{footnote}


\subsubsection{OCR}
\label{\detokenize{chapter_AI_dive/OCR:ocr}}\label{\detokenize{chapter_AI_dive/OCR::doc}}
OCR(光学字符识别)是针对印刷体字符,采用光学的方式将纸质文档中的文字转换成为黑白点阵的图像文件,并通过识别软件将图像中的文字转换成文本格式,供文字处理软件进一步编辑加工的技术。\sphinxhref{http://www.cstf.org.cn/newsdetail.asp?types=36\&num=1165}{1}%
\begin{footnote}[953]\sphinxAtStartFootnote
\sphinxnolinkurl{http://www.cstf.org.cn/newsdetail.asp?types=36\&num=1165}
%
\end{footnote}


\subsubsection{NLP}
\label{\detokenize{chapter_AI_dive/NLP:nlp}}\label{\detokenize{chapter_AI_dive/NLP::doc}}

\paragraph{语言的定义}
\label{\detokenize{chapter_AI_dive/NLP:id1}}
语言是指生物同类之间由于沟通需要而制定的指令系统,语言与逻辑相关,目前只有人类才能使用体系完整的语言进行沟通和思想交流。


\paragraph{自然语言的定义}
\label{\detokenize{chapter_AI_dive/NLP:id2}}
自然语言通常会自然地随文化发生演化,英语、汉语、日语都是具体种类的自然语言,这些自然语言履行着语言最原始的作用:人们进行交互和思想交流的媒介性工具。
\begin{itemize}
\item {} 
语音:与发音有关的学问,主要在语音技术中发挥作用。

\item {} 
音韵:由语音组合起来的读音,即汉语拼音和四声调。

\item {} 
词态:封装了可用于自然语言理解的有用信息,其中信息量的大小取决于具体的语言种类。中文没有太多的词态变换,仅存在不同的偏旁,导致出现词的性别转换的情况。

\item {} 
句法:主要研究词语如何组成合乎语法的句子,句法提供单词组成句子的约束条件,为语义的合成提供框架。

\item {} 
语义和语用:自然语言所包含和表达的意思。

\end{itemize}


\paragraph{自然语言处理(NLP)的定义}
\label{\detokenize{chapter_AI_dive/NLP:id3}}
自然语言处理(Natural Language
Processing,NLP):是计算机科学,人工智能和语言学的交叉领域。目标是让计算机处理或“理解”自然语言,以执行语言翻译和问题回答等任务。

\begin{figure}[H]
\centering
\capstart

\noindent\sphinxincludegraphics{{NLP_Dive}.png}
\caption{NLP的各层技术}\label{\detokenize{chapter_AI_dive/NLP:id21}}\end{figure}

NLP底层技术包含自然语言理解(Natural Language Understanding,NLU)
和自然语言生成(Natural Language Generation,
NLG)两个重要方向,如下图所示。
\begin{itemize}
\item {} 
自然语言理解NLU旨在将人的语言形式转化为机器可理解的、结构化的、完整的语义表示,通俗来讲就是让计算机能够理解人类语言。

\item {} 
自然语言生成NLG旨在让机器根据确定的结构化数据、文本、音视频等生成人类可以理解的自然语言形式的文本。

\end{itemize}

\begin{figure}[H]
\centering
\capstart

\noindent\sphinxincludegraphics{{NLP}.png}
\caption{NLP}\label{\detokenize{chapter_AI_dive/NLP:id22}}\end{figure}


\paragraph{自然语言处理的难度}
\label{\detokenize{chapter_AI_dive/NLP:id4}}\begin{itemize}
\item {} 
自然语言千变万化,没有固定格式。同样的意思可以使用多种句式来表达,同样的句子调整一个字、调整语调或者调整语序,表达的意思可能相差很多。

\item {} 
不断有新的词汇出现,计算机需要不断学习新的词汇。

\item {} 
受语音识别准确率的影响。

\item {} 
自然语言所表达的语义本身存在一定的不确定性,同一句话在不同场景/语境下的语义可能完全不同。

\item {} 
人类讲话时往往出现不流畅、错误、重复等现象,而对机器来说,在它理解一句话时,这句话整体所表达的意思比其中每个词的确切含义更加重要。

\end{itemize}


\paragraph{NLU}
\label{\detokenize{chapter_AI_dive/NLP:nlu}}
自然语言理解(NLU)模块主要是通过意图识别和槽识别(信息抽取)来理解对话中用户语句的语义。

意图识别(Intent
Prediction):目的是理解用户所表达的意图,核心其实是处理一个分类问题,将用户的话分类到事先预定义好的意图类别中去。目前主要基于深度学习的方法,使用CNN(卷积神经网络)对query进行特征提取和意图分类,类似的方法同样适用于领域的分类。

槽填充(Slot
Filling):提取对话中关键信息,本质是将句子中的词打上语义标签(如上图Slots日期、地点),具体方法有CRF(条件随机场)、Deep
Brief
Network(深度信念网络)以及RNN(循环神经网络)等。\sphinxhref{http://www.woshipm.com/pd/4133505.html}{3}%
\begin{footnote}[954]\sphinxAtStartFootnote
\sphinxnolinkurl{http://www.woshipm.com/pd/4133505.html}
%
\end{footnote}


\paragraph{NLG}
\label{\detokenize{chapter_AI_dive/NLP:nlg}}
自然语言生成作为\sphinxstylestrong{人工智能和计算语言学}的分支,其对应的语言生成系统可以被看作基于语言信息处理的计算机模型,该模型从抽象的概念层次开始,通过选择并执行一定的语法和语义规则生成自然语言文本。


\subparagraph{生成式对话生成技术}
\label{\detokenize{chapter_AI_dive/NLP:id5}}
代表性技术是从已有的“人\sphinxhyphen{}人”对话中学习语言的组合模式,是在一种类似机器翻译中常用的“编码\sphinxhyphen{}解码”的过程中逐字或逐词地生成回复,生成的回复有可能是从未在语料库中出现的、由聊天机器人自己“创造”的句子。


\subparagraph{三种自然语言生成方式}
\label{\detokenize{chapter_AI_dive/NLP:id6}}
文本生成\sphinxhyphen{}Text
Generation:https://wqw547243068.github.io/2020/04/28/text\sphinxhyphen{}generation/\#comments


\subparagraph{基于检索的自然语言生成}
\label{\detokenize{chapter_AI_dive/NLP:id7}}
基于检索的自然语言生成并不是如字面意思一样生成自然语言,更多是在已有的对话语料中检索出合适的回复。
\begin{itemize}
\item {} 
优点:实现相对简单、容易部署美因茨在实际工程中得到了大量的应用。

\item {} 
缺点:依赖于对话库、回复不够灵活等

\end{itemize}


\subparagraph{基于模板的自然语言生成}
\label{\detokenize{chapter_AI_dive/NLP:id8}}
自然语言生成模板由句子sentence模板,词汇word模版组成。句子模版包含若干个含有变量的句子,词汇模板则是句子模块中的变量对应的所有可能的值。


\subparagraph{基于深度学习的自然语言生成}
\label{\detokenize{chapter_AI_dive/NLP:id9}}\begin{itemize}
\item {} 
大概:\sphinxurl{https://tobiaslee.top/2018/06/09/Brief-overview-of-text-generation/}

\item {} 
框架:\sphinxurl{https://tobiaslee.top/2019/08/31/TG-framework-notes/}

\item {} 
论文:\sphinxurl{https://github.com/THUNLP-MT/TG-Reading-List}

\end{itemize}


\paragraph{文本处理流程}
\label{\detokenize{chapter_AI_dive/NLP:id10}}

\subparagraph{语料获取}
\label{\detokenize{chapter_AI_dive/NLP:id11}}\begin{itemize}
\item {} 
业务积累——通过脚本直接从数据库中提取/OCR或人工录入纸质文件

\item {} 
网络收集—下载网络开放的数据集(回答闲聊问题)或通过爬虫抓取

\item {} 
外部采购—采购(确保合规)专业语料数据集(行业评论、专业文献等)来用于回答客户的问题咨询

\end{itemize}


\subparagraph{预处理}
\label{\detokenize{chapter_AI_dive/NLP:id12}}

\subparagraph{语料清洗}
\label{\detokenize{chapter_AI_dive/NLP:id13}}
语料清洗,顾名思义就是将无用的噪音内容去掉,只保留对我们有用的主体内容。常见的是通过编写脚本,采用正则表达式匹配的方式去清洗数据,具体包括全半角转换、大小写转换、繁简体转换、无意义乂符号移除等等。如果是通过爬虫抓取的网页内容,我们还要去除广告标签或注释、JavaScript、CSS代码。
\begin{itemize}
\item {} 
全半角转换:将在输入法全角模式下输入的內容转换为半角模式的,主要对英文、数字、标点符号有影响。

\item {} 
大小写转换:统一将大写形式的字母转成小写形式的。

\item {} 
繁简体转换:将繁体输入转成简体的式,当然考虑到用户群体的差异以及可能存在繁体形式的资源,有些情况还需要保留转换前的繁体输入用于召回

\item {} 
无意义符号移除:移除诸如火星文符号、emoj表情符号、URL地址等特殊符号内容。

\end{itemize}

清洗完语料之后,接下来要做的就是分词。因为中文语料数据都是短文本或长文本组成的句子,所以我们在进行文本挖掘分析时,需要将这些句子处理成最小单位粒度的字符或者词语。


\subparagraph{文本分词}
\label{\detokenize{chapter_AI_dive/NLP:id14}}\begin{itemize}
\item {} 
基于规则匹配的分词方法

\item {} 
基于统计的分词方法

\item {} 
基于语义理解的分词方法

\end{itemize}

技术难点在于“歧义识别”和“新词识别”。比如说:“武汉市长江大桥”,这个句子可以切分成“武汉市/长江大桥”,也可切分成“武汉/市长/江大桥”,如果不依赖上下文其他的句子,恐怕我们很难知道怎么去理解。


\subparagraph{词性标注}
\label{\detokenize{chapter_AI_dive/NLP:id15}}\begin{itemize}
\item {} 
普通词性标注:将句子中的词标记为名词、动词或者形容词等等

\item {} 
专业词性标注:针对特定行业领域的词性标注,如医疗行业、教育行业等等

\end{itemize}

\begin{figure}[H]
\centering
\capstart

\noindent\sphinxincludegraphics{{words_property_label}.png}
\caption{实际工作中的词性编码表}\label{\detokenize{chapter_AI_dive/NLP:id23}}\end{figure}


\subparagraph{文本表示}
\label{\detokenize{chapter_AI_dive/NLP:id16}}
定义——把已分词的字符转化成向量矩阵形式

常见方式:词袋模型


\subparagraph{文本计算}
\label{\detokenize{chapter_AI_dive/NLP:id17}}
定义一计算四个文本之间的相似度

常见方式:余弦距离、欧氏距离、皮尔逊相关度


\paragraph{文本处理场景}
\label{\detokenize{chapter_AI_dive/NLP:id18}}\begin{itemize}
\item {} 
信息提取:从指定文本中提取出重要信息,如时间、地点、人物、事件等。具体场景如从新闻咨询中提取能够完整准确反映中心内容的摘取信息

\item {} 
智能问答:对客户问题进行语义理解,然后在知识库中查找可能的候选答案通过排序找出最佳的答案进行回复

\item {} 
机器翻译:通过把输入的源语言文本通过自动翻译获得另一种语言的文本,是自然语言处理中最为人所熟知的场景,如百度翻译、
Google翻译

\item {} 
文本挖掘:包括文本聚类、分类、情感分析以及对挖掘的信息和知识通过可视化、交互式界面进行表达

\item {} 
舆论分析:通过收集和处理海量信息对网络舆情进行自动化的分析,帮助分析网络话题热点,然后对热点的传播路径及发展趋势进行分析判断,以实现及时应对网络舆情

\item {} 
知识图谱:根据参数和限定词的输入,进行词语到文本或文本到文本的生成\sphinxhref{https://time.geekbang.org/column/article/348027}{5}%
\begin{footnote}[955]\sphinxAtStartFootnote
\sphinxnolinkurl{https://time.geekbang.org/column/article/348027}
%
\end{footnote}

\end{itemize}


\subparagraph{金融应用}
\label{\detokenize{chapter_AI_dive/NLP:id19}}
为了解决由数据推测模型的局限性,通过自然语言处理技术,引入新闻、政策以及社交媒体中的文本,将非结构化数据进行结构化处理,并从中寻找\sphinxstylestrong{影响市场变动的因素}。除了可以丰富模型变量外,自然语言处理技术可以实现“智能投融资顾问助手”。集合自然语言搜索、用户界面图形化及云计算,智能助手可以将问题与实践关联市场动态,提供研究辅助、智能回答复杂金融投融资问题。\sphinxhref{http://www.cstf.org.cn/newsdetail.asp?types=36\&num=1165}{2}%
\begin{footnote}[956]\sphinxAtStartFootnote
\sphinxnolinkurl{http://www.cstf.org.cn/newsdetail.asp?types=36\&num=1165}
%
\end{footnote}


\subparagraph{中文金融领域}
\label{\detokenize{chapter_AI_dive/NLP:id20}}
中文分词是中文NLP的难点之一。如“结婚的和尚未结婚的”,应该分词为“结婚/的/和/尚未/结婚/的”,还是“结婚/的/和尚/未/结婚/的”,不同的分词方法会产生一定的歧义。再比如,“美国会通过对台售武法案”,我们既可以切分为“美国/会/通过对台售武法案”,又可以切分成“美/国会/通过对台售武法案”。

随着深度学习的普遍使用,中文与英文在语言上的差异也逐渐变成训练数据量上的差异,以往在NLP领域,可供使用的中文数据量比英文数据要少的多,这是目前中文NLP的难点之一。但是随着有越来越多的人投入到中文人工智能以及NLP领域的研究中来,中文数据集不足的问题正在逐年改善。

在金融领域,针对基础性问题,中英文所处的阶段其实大体相同,但是针对如情感分析、市场预测等复杂问题,由于要结合具体的语境以及相应的应用场景,同时要考虑训练的数量级问题,无论是中文还是英文的NLP要走的路都还有很多。\sphinxhref{https://www.miotech.com/zh-CN/article/5cda76428b224f0044833a13}{4}%
\begin{footnote}[957]\sphinxAtStartFootnote
\sphinxnolinkurl{https://www.miotech.com/zh-CN/article/5cda76428b224f0044833a13}
%
\end{footnote}


\subsubsection{知识图谱}
\label{\detokenize{chapter_AI_dive/KG:id1}}\label{\detokenize{chapter_AI_dive/KG::doc}}
知识图谱是一种\sphinxstylestrong{基于语义网络的知识结构表达,通过将真实世界中的实体对应到语义网络中,构建该实体与其他实体的关系。}其不仅可以应用在智能搜索中,还可以应用于智能问答或者社交平台以及垂直行业中。

从数据的挖掘效率层面来说,人工智能的发展离不开技术的不断创新,传统机器学习、深度学习、自然语言处理、语音图像识别、知识图谱是现阶段人工智能的五大核心技术,很多场景落地并产生价值,需要数据+多种技术的结合,也包括与传统专家规则的结合。其中,知识图谱与自然语言处理,是2019到2020年的热点落地技术,这两项技术也是相互交融的关系,构建从感知智能向认知智能的必要条件,两者最终目的都是往让机器能够更好的认知这个世界,朝着更加智能化的方向去发展。\sphinxhref{https://www.weiyangx.com/351456.html}{2}%
\begin{footnote}[958]\sphinxAtStartFootnote
\sphinxnolinkurl{https://www.weiyangx.com/351456.html}
%
\end{footnote}

\begin{figure}[H]
\centering
\capstart

\noindent\sphinxincludegraphics{{knowledge_map_in_AI}.png}
\caption{知识图谱与AI的关系}\label{\detokenize{chapter_AI_dive/KG:id5}}\end{figure}


\paragraph{知识图谱的必要性}
\label{\detokenize{chapter_AI_dive/KG:id2}}
在黑天鹅事件发生时,机器学习和自然语言处理会失效。2015年中国证监会公布的熔断机制就属于该类事件。由于人工智能系统内没有载入类似事件及后果,无法从历史数据中学习到相关模式。此时,由人工智能决策的投资就会出现较大风险。虚假关联性对人工智能处理数据的影响不小于黑天鹅事件。人工智能善于发现变量间的相关性,而非因果性。\sphinxstylestrong{强相关性的变量间并不一定具备经济学关联,而人工智能的机器学习无法区分虚假关联性。}为了降低黑天鹅事件及虚假关联性对于人工智能自学习过程的干扰,需要专家设置相应的规则来避免。知识图谱是一种语义网络,基于图的数据结构,根据已设计的规则及不同种类的变量连接所形成的关系网络。

知识图谱提供了\sphinxstylestrong{从关联性角度去分析问题}的能力,将规则、关系及变量通过图谱的形式表现出来,进行更深层次的信息梳理和推测。以投资关系为例,知识图谱可以将公司的股权变更沿革串联起来,清楚展示某家PE机构于某一年进入某家企业、进入价格是多少、是否有对赌协议等等。这些信息可以用以判断PE机构进入时的估值及公司的成长节奏,同时该图谱还可以用来学习投资机构的投资偏好及逻辑的发展。目前,知识图谱并未进行大规模的应用。其难点在于如何让行业专家承担部分程序员的的工作,将行业逻辑等关系通过计算机建模,输入计算机以供机器进行学习和验证。可见,开发形成简易编程的界面及系统是目前应用推广的关键。\sphinxhref{http://www.cstf.org.cn/newsdetail.asp?types=36\&num=1165}{3}%
\begin{footnote}[959]\sphinxAtStartFootnote
\sphinxnolinkurl{http://www.cstf.org.cn/newsdetail.asp?types=36\&num=1165}
%
\end{footnote}

TODO: \sphinxurl{http://www.woshipm.com/pmd/2816130.html}

知识图谱领域 的 KBGAN 等算法都需要长时间的预训练
\sphinxurl{https://github.com/cai-lw/KBGAN}


\paragraph{应用}
\label{\detokenize{chapter_AI_dive/KG:id3}}

\subparagraph{智能投研}
\label{\detokenize{chapter_AI_dive/KG:id4}}
如果说金融数据、另类数据是智能投研的原料,那么知识图谱就是智能投研的大脑。所谓“知识图谱”是将实体、属性、关系等非结构化数据固联起来,进而为投资决策提供逻辑支持。体现在投资行业,就是研究员可以将相关的行业、产品和公司等多方因素联系在一起,当观察到某个因素发生变化时,即可以根据关系链推理出观点和预测,为投资决策提供支撑。

完善的知识图谱是AI在投资研究中应用的必要条件,金融行业最不缺的就是海量的高质量研究资料,通过对研报、公告等文本信息的深入挖掘,形成能够自我生长、自我学习的知识图谱体系,这是智能投研的重中之重。\sphinxhref{https://www.jianshu.com/p/d15703c14cd5}{1}%
\begin{footnote}[960]\sphinxAtStartFootnote
\sphinxnolinkurl{https://www.jianshu.com/p/d15703c14cd5}
%
\end{footnote}


\subsubsection{聊天机器人}
\label{\detokenize{chapter_AI_dive/chatbot:id1}}\label{\detokenize{chapter_AI_dive/chatbot::doc}}

\paragraph{NLP应用概览}
\label{\detokenize{chapter_AI_dive/chatbot:nlp}}
NLP作为人工智能的核心技术,在机器翻译、聊天机器人、语音识别等领域都有重要的应用。

机器翻译的代表如科大讯飞的翻译机;聊天机器人例如微软小冰、Amazon
Alexa;语音识别如各种智能音箱。


\paragraph{聊天机器人系统框架}
\label{\detokenize{chapter_AI_dive/chatbot:id2}}
一个完整聊天机器人的系统架构主要由唤醒、语言识别、自然语言理解、对话管理、自然语言生成、语音合成等6个部分组成。更多详细见\sphinxhref{http://www.ciotc.org/?from=timeline\#/articaltwoinfo?id=20191209112501276114675\&ids=18}{8}%
\begin{footnote}[961]\sphinxAtStartFootnote
\sphinxnolinkurl{http://www.ciotc.org/?from=timeline\#/articaltwoinfo?id=20191209112501276114675\&ids=18}
%
\end{footnote}

\begin{figure}[H]
\centering
\capstart

\noindent\sphinxincludegraphics{{chatbot_flowchart}.png}
\caption{聊天机器人的流程闭环\sphinxhref{https://www.jianshu.com/p/b8302c22dcba}{7}\sphinxfootnotemark[962]}\label{\detokenize{chapter_AI_dive/chatbot:id13}}\end{figure}
%
\begin{footnotetext}[962]\sphinxAtStartFootnote
\sphinxnolinkurl{https://www.jianshu.com/p/b8302c22dcba}
%
\end{footnotetext}\ignorespaces \begin{itemize}
\item {} 
唤醒wakeup,匹配到唤醒词后进入工作状态。

\item {} 
自动语音识别automatic speech
recognition,ASR,负责将原始的语音信号转换成文本信息。

\item {} 
自然语言理解natural language
understanding,NLU,负责将识别到的文本信息转换为机器可以理解的语义表示。

\item {} 
对话管理dialogue
management,DM,负责基于当前对话的状态判断系统应该采取怎样的动作。
包括对话状态跟踪(Dialog StateTracking, DST)、对话策略(Dialogue
Policy)、作为接口与后端/任务模型进行交互、提供语义表达的期望值(expections
for interpretation)。 \sphinxstylestrong{对话状态跟踪}(Dialog StateTracking,
DST)
:根据对话历史,维护当前对话状态,对话状态是对整个对话历史的累积语义表示,一般就是槽值对(slot\sphinxhyphen{}value
pairs)。 \sphinxstylestrong{对话策略}(Dialogue
Policy):根据当前对话状态输出下一步系统动作。\sphinxhref{https://www.toutiao.com/i6854955754193945096/}{6}%
\begin{footnote}[963]\sphinxAtStartFootnote
\sphinxnolinkurl{https://www.toutiao.com/i6854955754193945096/}
%
\end{footnote}

\end{itemize}
\begin{enumerate}
\sphinxsetlistlabels{\arabic}{enumi}{enumii}{}{.}%
\item {} 
有适用于简单应用场景的\sphinxstylestrong{穷举}的方式,类似数据库查表的方式;

\item {} 
有适用于\sphinxstylestrong{面向特定任务}的对话系统的基于框架的方法;

\item {} 
有基于强化学习的方式,不断的更新决策策略以最大化reward;

\item {} 
有基于浅层神经网络和强化学习结合的方法。

\end{enumerate}
\begin{itemize}
\item {} 
自然语言生成natural language
generation,NLG,负责将系统动作/系统回复转变成自然语言文本。

\item {} 
语音合成text\sphinxhyphen{}to\sphinxhyphen{}speech,TTS,负责将自然语言文本转变成语音信号输出给用户。

\end{itemize}


\paragraph{ASR 10\sphinxfootnotemark[964]}
\label{\detokenize{chapter_AI_dive/chatbot:asr-10}}%
\begin{footnotetext}[964]\sphinxAtStartFootnote
\sphinxnolinkurl{http://imgtec.eetrend.com/blog/2019/100046571.html\#:~:text=\%E5\%AF\%B9\%E8\%BE\%93\%E5\%85\%A5\%E7\%9A\%84\%E5\%8E\%9F\%E5\%A7\%8B\%E8\%AF\%AD\%E9\%9F\%B3,\%E2\%80\%9D\%E8\%BF\%9B\%E8\%A1\%8C\%E5\%88\%86\%E6\%9E\%90\%EF\%BC\%89\%E7\%AD\%89\%E5\%A4\%84\%E7\%90\%86\%E3\%80\%82}
%
\end{footnotetext}\ignorespaces 
“输入——编码——解码——输出”


\subparagraph{语音输入的预处理模块}
\label{\detokenize{chapter_AI_dive/chatbot:id3}}
对输入的原始语音信号进行处理,滤除掉其中的不重要信息以及背景噪声,并进行语音信号的端点检测(也就是找出语音信号的始末)、语音分帧(可以近似理解为,一段语音就像是一段视频,由许多帧的有序画面构成,可以将语音信号切割为单个的“画面”进行分析)等处理。


\subparagraph{特征提取}
\label{\detokenize{chapter_AI_dive/chatbot:id4}}
在去除语音信号中对于语音识别无用的冗余信息后,保留能够反映语音本质特征的信息进行处理,并用一定的形式表示出来。也就是提取出反映语音信号特征的关键特征参数形成特征矢量序列,以便用于后续处理。


\subparagraph{声学模型训练}
\label{\detokenize{chapter_AI_dive/chatbot:id5}}
声学模型可以理解为是对声音的建模,能够把语音输入转换成声学表示的输出,准确的说,是给出语音属于某个声学符号的概率。根据训练语音库的特征参数训练出声学模型参数。在识别时可以将待识别的语音的特征参数与声学模型进行匹配,得到识别结果。目前的主流语音识别系统多采用隐马尔可夫模型HMM进行声学模型建模。


\subparagraph{语言模型训练}
\label{\detokenize{chapter_AI_dive/chatbot:id6}}
语言模型是用来计算一个句子出现概率的模型,简单地说,就是计算一个句子在语法上是否正确的概率。因为句子的构造往往是规律的,前面出现的词经常预示了后方可能出现的词语。它主要用于决定哪个词序列的可能性更大,或者在出现了几个词的时候预测下一个即将出现的词语。它定义了哪些词能跟在上一个已经识别的词的后面(匹配是一个顺序的处理过程),这样就可以为匹配过程排除一些不可能的单词。

语言建模能够有效的结合汉语语法和语义的知识,描述词之间的内在关系,从而提高识别率,减少搜索范围。对训练文本数据库进行语法、语义分析,经过基于统计模型训练得到语言模型。


\subparagraph{语音解码和搜索算法}
\label{\detokenize{chapter_AI_dive/chatbot:id7}}
解码器是指语音技术中的识别过程。针对输入的语音信号,根据己经训练好的HMM声学模型、语言模型及字典建立一个识别网络,根据搜索算法在该网络中寻找最佳的一条路径,这个路径就是能够以最大概率输出该语音信号的词串,这样就确定这个语音样本所包含的文字了。所以,解码操作即指搜索算法,即在解码端通过搜索技术寻找最优词串的方法。

连续语音识别中的搜索,就是寻找一个词模型序列以描述输入语音信号,从而得到词解码序列。搜索所依据的是对公式中的声学模型打分和语言模型打分。在实际使用中,往往要依据经验给语言模型加上一个高权重,并设置一个长词惩罚分数。


\paragraph{形态}
\label{\detokenize{chapter_AI_dive/chatbot:id8}}\begin{itemize}
\item {} 
硬件形态:智能音箱\sphinxhref{http://www.ciotc.org/?from=timeline\#/articaltwoinfo?id=20191209112501276114675\&ids=18}{8}%
\begin{footnote}[965]\sphinxAtStartFootnote
\sphinxnolinkurl{http://www.ciotc.org/?from=timeline\#/articaltwoinfo?id=20191209112501276114675\&ids=18}
%
\end{footnote}如amazon
echo(2014年11月全球第一款)、公子小白、。

\item {} 
软件形态:Apple Siri、微软小冰、微软cortana、IBM watson、Google Now。

\item {} 
平台:谷歌、微软等公司对外提供聊天机器人框架bot
framework,以sdk或saas服务的方式像第三方公司或个人开发者提供可以用于构建特定应用和领域的聊天机器人。代表:amazon
Alexa(服务amazon lex)、微软luis with bot(认知服务cognitive
services)、谷歌api.ai、Facebook wit.ai。

\end{itemize}


\paragraph{类别}
\label{\detokenize{chapter_AI_dive/chatbot:id9}}
两类:


\subparagraph{闲聊机器人}
\label{\detokenize{chapter_AI_dive/chatbot:id10}}\begin{enumerate}
\sphinxsetlistlabels{\arabic}{enumi}{enumii}{}{.}%
\item {} 
基于seq2seq模型的对话系统:根据前一句来生成后一句的回复,对话的回答局限性大,缺少对整个对话的评估,且容易陷入死循环。

\item {} 
基于DRL的对话系统:利用强化学习对当前生成的各种回复评估,选择reward值最高的句子。评估方式根据应用场景不同,可以设计不同的评估函数。

\item {} 
GAN和RL结合的对话系统:生成器生成对话,判别器评估每种结果的reward,其中各种可能的结果是采用MCTS或者策略梯度的方式。seqGAN\sphinxhref{https://github.com/LantaoYu/SeqGAN}{3}%
\begin{footnote}[966]\sphinxAtStartFootnote
\sphinxnolinkurl{https://github.com/LantaoYu/SeqGAN}
%
\end{footnote},Neural\sphinxhyphen{}Dialogue\sphinxhyphen{}Generation\sphinxhref{https://github.com/jiweil/Neural-Dialogue-Generation}{5}%
\begin{footnote}[967]\sphinxAtStartFootnote
\sphinxnolinkurl{https://github.com/jiweil/Neural-Dialogue-Generation}
%
\end{footnote}

\end{enumerate}


\subparagraph{面向任务的聊天机器人}
\label{\detokenize{chapter_AI_dive/chatbot:id11}}\begin{enumerate}
\sphinxsetlistlabels{\arabic}{enumi}{enumii}{}{.}%
\item {} 
基于DQN的对话系统:DM模块采用DQN模型可能结果的reward值。

\end{enumerate}

任务型对话系统以智能为核心的评价体系:\sphinxurl{http://www.zgcsa.org/uploads/file/20201105192719\_469.pdf}

任务型对话系统概述\sphinxhref{https://www.taodudu.cc/news/show-1668385.html}{9}%
\begin{footnote}[968]\sphinxAtStartFootnote
\sphinxnolinkurl{https://www.taodudu.cc/news/show-1668385.html}
%
\end{footnote}


\paragraph{项目}
\label{\detokenize{chapter_AI_dive/chatbot:id12}}
基于金融\sphinxhyphen{}司法领域(兼有闲聊性质)的聊天机器人:\sphinxurl{https://github.com/charlesXu86/Chatbot\_CN}


\subsubsection{联邦学习}
\label{\detokenize{chapter_AI_dive/Federated Learning:id1}}\label{\detokenize{chapter_AI_dive/Federated Learning::doc}}
联邦学习其核心就是一个分布式的机器学习。通过传参数,不上传数据的方式做分布式的机器学习,相较于传统分布式机器学习,其实现了数据隐私保护。通过整合各个节点上的参数,
不同的设备可以在保持设备中大部分数据的同时,实现模型训练更新。当前市场上已经出现了一些联邦学习框架,但能真正用于实际生产的屈指可数。正是因为联邦学习涉及的技术领域之多,并且需要能够兼顾系统性能、使用资源等重要指标,所以在实际落地过程中对技术人员专业素质要求很高。\sphinxhref{https://zhuanlan.zhihu.com/p/296240106}{1}%
\begin{footnote}[969]\sphinxAtStartFootnote
\sphinxnolinkurl{https://zhuanlan.zhihu.com/p/296240106}
%
\end{footnote}

TODO:


\subsubsection{移动端}
\label{\detokenize{chapter_AI_dive/mobile:id1}}\label{\detokenize{chapter_AI_dive/mobile::doc}}

\paragraph{PC vs 移动端 4\sphinxfootnotemark[970]}
\label{\detokenize{chapter_AI_dive/mobile:pc-vs-4}}%
\begin{footnotetext}[970]\sphinxAtStartFootnote
\sphinxnolinkurl{http://www.woshipm.com/pd/289607.html}
%
\end{footnotetext}\ignorespaces 
\begin{center}\sphinxincludegraphics{{PC_vs_moblie_cost}.png}\end{center} \sphinxincludegraphics{{PC_vs_mobile_users}.png} \sphinxincludegraphics{{PC_vs_mobile_products}.png}


\paragraph{小程序 1\sphinxfootnotemark[971]}
\label{\detokenize{chapter_AI_dive/mobile:id2}}%
\begin{footnotetext}[971]\sphinxAtStartFootnote
\sphinxnolinkurl{https://www.zhihu.com/question/346774796/answer/1686950897s}
%
\end{footnotetext}\ignorespaces 
1、用户角度:1)不占内存:直接依附于微信存在,直接打开就可以使用,也不需要下载。2)即用即走:小程序现在有9个入口,无论从哪个入口进入,使用完毕以后都可以直接退出来,非常方便。

2、企业角度:1)搭建简单:微信小程序开发,最开始只能由企业开发,后来的个人也可以开发,到现在直接可以通过微信公众号搭建,如果对功能要求不高,搭建一个微信小程序非常简单。2)节约成本:比起开发APP,微信小程序的成本更低,而且周期短;微信公众号需要专门的人员运营,微信小程序则不需要,也可以节约人力成本。3)流量红利大、入口场景丰富:一般来说小程序都是依赖于微信/支付宝的,对于任何一个企业而言,流量都是需要投入很大财力物力的地方。线上线下广告成本都特别大,而微信/支付宝小程序则不一样,它本身所依附的对象就是一个巨大的流量入口。微信小程序的9个入口,更是可以为企业带来无法预估的流量。

3、平台角度:弥补人口红利的消失:当人口红利论难以维持时,能够节约用户时间、提升商业效率的产品与服务,或许将迎来增长机会。互联网要进入产业纵深,帮助产业实现数字化,提升效率,改造从供应链到销售的全流程,就需要调用各种能力,不再是一个App打天下,一种能力吃八方,而需要集成多种能力,在一个可以自由跳转又能便捷开发的生态之内。


\paragraph{Andriod}
\label{\detokenize{chapter_AI_dive/mobile:andriod}}
Baidu SDK:https://ai.baidu.com/sdk


\subparagraph{为什么需要签名 MD5?2\sphinxfootnotemark[972]}
\label{\detokenize{chapter_AI_dive/mobile:md5-2}}%
\begin{footnotetext}[972]\sphinxAtStartFootnote
\sphinxnolinkurl{https://ai.baidu.com/forum/topic/show/492251}
%
\end{footnotetext}\ignorespaces 
安卓的应用是以包名做为唯一ID的。百度的人脸服务也是以包名做为单位进行
授权的。因为包名是开发者填写的,所以别的开发者也可以写个应用来冒充其他人的应用。百度人脸服务会涉及到用户的信息,使用过程中也有费用产生。所以为了保护APP不会他人冒充,我们对应用的签名进行校验。刚才也提到了,因为MD5算法的不可逆性,可以当做公钥使用。用户在申请时在后台填写签名的MD5值,发布/测试时,使用该签名文件。人脸服务在运行时会对当前应用的签名MD5
进行校验,如果信息不一致会拒绝服务。


\subparagraph{Mobile 部署模型}
\label{\detokenize{chapter_AI_dive/mobile:mobile}}
在mobile上部署模型会遇到计算资源受限的问题,一般解决方法:减小模型大小(类似MobileNet),quantizing
weights,知识蒸馏(例如DistillBERT)等。

一些工具框架介绍:TensorFlow Lite,PyTorch
Mobile,CoreML,MLKit,FritzAI等。ONNX可以作为中间层,再部署到各种硬件平台上。对于嵌入式系统,最好的解决方案是NVIDIA
for Embeded。

MindSpore Lite :\sphinxurl{https://juejin.cn/post/6939167928822530078}


\subsubsection{认知智能}
\label{\detokenize{chapter_AI_dive/cognition_AI:id1}}\label{\detokenize{chapter_AI_dive/cognition_AI::doc}}
2020年,是人工智能领域的扎实落地之年,也是从感知智能迈向认知智能的规模突破之年。知识图谱技术与图计算应用场景取得爆发性增长,从增强自然语言能力、人工智能模型可解释性、机器学习能力提升三个维度助力各大企业机构向认知智能领域演变。\sphinxhref{https://www.weiyangx.com/378231.html}{1}%
\begin{footnote}[973]\sphinxAtStartFootnote
\sphinxnolinkurl{https://www.weiyangx.com/378231.html}
%
\end{footnote}

如果说感知智能时代,人工智能三要素是数据,算法,算力。

那么在认知智能时代,三要素则需要进一步迭代与提升,具体来说就是:
\begin{itemize}
\item {} 
算法不是关键,工程架构才是关键;

\item {} 
算力不是关键,人机协作才是关键;

\item {} 
数据(量)不是关键,数据的结构才是关键;

\end{itemize}

\begin{figure}[H]
\centering
\capstart

\noindent\sphinxincludegraphics{{sense_vs_cognition_AI}.png}
\caption{感知智能VS认知智能}\label{\detokenize{chapter_AI_dive/cognition_AI:id2}}\end{figure}


\subsubsection{现实}
\label{\detokenize{chapter_AI_dive/real:id1}}\label{\detokenize{chapter_AI_dive/real::doc}}

\paragraph{具体的工作 1\sphinxfootnotemark[974]}
\label{\detokenize{chapter_AI_dive/real:id2}}%
\begin{footnotetext}[974]\sphinxAtStartFootnote
\sphinxnolinkurl{https://www.zhihu.com/question/57815929}
%
\end{footnotetext}\ignorespaces \begin{itemize}
\item {} 
测试准确率

\item {} 
挑选 Badcase(坏样本)

\item {} 
优化算法调参数/自己改或与算法工程师一起改

\item {} 
收集 Sample(样例)

\item {} 
训练模型,提升精度再测试准确度

\item {} 
……

\item {} 
漫长的循环往复

\end{itemize}


\subsubsection{AI产品的测试}
\label{\detokenize{chapter_AI_dive/AI_test:ai}}\label{\detokenize{chapter_AI_dive/AI_test::doc}}
TODO: \sphinxurl{https://www.cnblogs.com/zgq123456/articles/10562855.html}


\paragraph{指标}
\label{\detokenize{chapter_AI_dive/AI_test:id1}}\begin{itemize}
\item {} 
查准率表示真正例占所有预测结果为正例的样例比值

\item {} 
查全率表示真正例占所有真实情况为正例的样例比值

\item {} 
ROC曲线描述的是真正例率和假正例率之间的关系

\end{itemize}


\subsubsection{灰度发布}
\label{\detokenize{chapter_AI_dive/huidu:id1}}\label{\detokenize{chapter_AI_dive/huidu::doc}}
灰度发布是指在黑与白之间,能够平滑过渡的一种发布方式,一级一级的发布逐渐的\sphinxstylestrong{扩大发布范围},最后达到系统的完全上线。 \sphinxhref{https://blog.csdn.net/liwei16611/article/details/90176044}{1}%
\begin{footnote}[975]\sphinxAtStartFootnote
\sphinxnolinkurl{https://blog.csdn.net/liwei16611/article/details/90176044}
%
\end{footnote}

灰度发布可以保证整体系统的稳定,在初始灰度的时候就可以发现、调整问题,以保证其影响度。

灰度期:灰度发布开始到结束期间的这一段时间,称为灰度期。

好处:
\begin{enumerate}
\sphinxsetlistlabels{\arabic}{enumi}{enumii}{}{.}%
\item {} 
提前获得目标用户的使用反馈;

\item {} 
根据反馈结果,做到查漏补缺;

\item {} 
发现重大问题,可回滚“旧版本”;

\item {} 
补充完善产品不足;

\item {} 
快速验证产品的 idea。

\end{enumerate}


\paragraph{灰度发布 VS A/B测试}
\label{\detokenize{chapter_AI_dive/huidu:vs-a-b}}
灰度发布于互联网公司常用A/B测试似乎比较类似,老外似乎并没有所谓的灰度发布的概念。按照wikipedia中对A/B测试的定义,A/B测试又叫:A/B/N
Testing、Multivariate
Testing,因此本质上灰度测试可以算作A/B测试的一种特例。只不过为了术语上不至于等同搞混淆,谈谈自己理解的两者的差异。
\begin{itemize}
\item {} 
灰度发布是对某一产品的发布逐步扩大使用群体范围,也叫灰度放量

\item {} 
A/B测试重点是在几种方案中选择最优方案\sphinxhref{http://www.woshipm.com/pd/706.html}{5}%
\begin{footnote}[976]\sphinxAtStartFootnote
\sphinxnolinkurl{http://www.woshipm.com/pd/706.html}
%
\end{footnote}

\end{itemize}


\subparagraph{A/B testing 3\sphinxfootnotemark[977]}
\label{\detokenize{chapter_AI_dive/huidu:a-b-testing-3}}%
\begin{footnotetext}[977]\sphinxAtStartFootnote
\sphinxnolinkurl{https://www.yunyingpai.com/user/94.html}
%
\end{footnotetext}\ignorespaces 
一种统计方法,用于将两种或多种技术进行比较,通常是将当前采用的技术与新技术进行比较。A/B测试不仅旨在确定哪种技术的效果更好,而且还有助于了解相应差异是否具有显著的统计意义。AB测试通常是采用一种衡量方式对两种技术进行比较,但也适用于任意有限数量的技术和衡量方式。

A/B测试,在基于web的软件开发中经常使用,对于在生产中\sphinxstylestrong{评估模型性能}非常有用。

\begin{figure}[H]
\centering
\capstart

\noindent\sphinxincludegraphics{{A_B_Testing}.png}
\caption{A/B测试\sphinxhref{https://time.geekbang.org/column/intro/100065501}{2}\sphinxfootnotemark[978]}\label{\detokenize{chapter_AI_dive/huidu:id3}}\end{figure}
%
\begin{footnotetext}[978]\sphinxAtStartFootnote
\sphinxnolinkurl{https://time.geekbang.org/column/intro/100065501}
%
\end{footnotetext}\ignorespaces 
A/B测试基本思想:
\begin{itemize}
\item {} 
单一变量

\item {} 
2个方案

\item {} 
以某一方面的标准进行评估,例如转化率

\end{itemize}

A/B测试面向的问题不是战略方案,而是某些产品设计的细节或者设计趋向的问题。例如Airbnb的收藏功能,Airbnb的收藏功能名字叫做”心愿单“,在进行A/B测试中,把收藏图标由收藏星型改成了爱心,使得收藏使用率提高了30\%,倘若没有进行A/B测试,产品设计人员是绝对不敢轻易的改动该图标的。

A/B测试是\sphinxstylestrong{对经验主义的抨击,它是数据驱动。}

例如:在百姓网的移动APP设计中,关于付费购买增值服务的广告主信息问题有过争执。广告部同事认为应该置顶该信息,因为这可以为广告部门带来收入,而产品设计部门则基于经验认为该改动会影响用户体验,使得用户每次都看到这个东西,失去新鲜感。然而,经过A/B测试得出的数据,两者的核心数据相差的量级仅仅在千分位之后,这个改动竟然对用户体验几乎没有影响。这与产品设计的经验是相悖的。

但是,A/B测试有的时候也会出现问题。例如百姓网在求职服务的按钮设计中,“拨打电话”和“投递简历”两个按钮只能放一个,因此将其进行了测试,最后得出的结论是两者的数据不相上下,结合情景我们知道,针对不同的职位,意愿也不同。最后,技术部门更改了产品架构,把按钮的位置进行了更改,放入两个按钮,超脱了A/B测试的范围却更好的解决了问题。

A/B测试并不是个新兴的概念,自2000年谷歌工程师将这一方法应用在互联网产品以来,A/B测试在国内外越来越普及,已成为互联网产品运营精细度的重要体现。简单来说,AB实验在产品优化中的应用方法是:在产品正式迭代发版之前,为同一个目标制定两个(或以上)方案,将用户流量对应分成几组,在保证每组用户特征相同的前提下,让用户分别看到不同的方案设计,根据几组用户的真实数据反馈,科学的帮助产品进行决策。

互联网领域常见的A/B测试,大多是面向C端用户进行流量选择,比如\sphinxstylestrong{基于注册用户的UID或者用户的设备标识(移动用户IMEI号/PC用户Cookie)}进行随机或者哈希计算后分流。此类方案广泛应用于搜索、推荐、广告等领域,体现出千人千面个性化的特点。此类方案的特点是实现简单,假设请求独立同分布,流量之间独立决策,互不干扰。此类AB实验之所以能够这样做是因为:C端流量比较大,样本足够多,而且不同用户之间没有相互干扰,只要分流时足够随机,即基本可以保证请求独立同分布。\sphinxhref{https://tech.meituan.com/2020/01/23/meituan-delivery-machine-learning.html}{7}%
\begin{footnote}[979]\sphinxAtStartFootnote
\sphinxnolinkurl{https://tech.meituan.com/2020/01/23/meituan-delivery-machine-learning.html}
%
\end{footnote}

\sphinxhref{https://experimentguide.com/}{Trustworthy Online Controlled Experiments : A Practical Guide to A/B
Testing}%
\begin{footnote}[980]\sphinxAtStartFootnote
\sphinxnolinkurl{https://experimentguide.com/}
%
\end{footnote}


\paragraph{灰度策略}
\label{\detokenize{chapter_AI_dive/huidu:id2}}
灰度策略一般有三种方式,可以应用于不同的场景\sphinxhref{https://g.yuque.com/amir/pm/ozyyed?language=zh-cn}{6}%
\begin{footnote}[981]\sphinxAtStartFootnote
\sphinxnolinkurl{https://g.yuque.com/amir/pm/ozyyed?language=zh-cn}
%
\end{footnote}
\begin{itemize}
\item {} 
随机灰度:一般在发布适合全局用户的新功能或新特性时使用,也就是在全局用户中随机测试一批用户来观察效果。

\item {} 
定向灰度:它适用于在产品中根据细分群体去发布的一些新功能。就像前面课中讲到的QQ厘米秀的案例,就是巧妙地对目标用户群进行灰度,从而推广创新产品。

\item {} 
邀请灰度:它通常适用于一款新产品的诞生,并且具备一定的社交或社区属性。一方面,能在产品诞生之初,更加聚焦于典型的种子用户群体;同时,还通过邀请码的方式,让用户形成自传播,为产品带来自增流量和口碑效应。

\end{itemize}


\subsubsection{AI部署}
\label{\detokenize{chapter_AI_dive/AI_deploy:ai}}\label{\detokenize{chapter_AI_dive/AI_deploy::doc}}
\begin{figure}[H]
\centering
\capstart

\noindent\sphinxincludegraphics{{ML_deploy}.png}
\caption{ML deploy\sphinxhref{http://www.ailab.cn/ml/20201213109713.html}{3}\sphinxfootnotemark[982]}\label{\detokenize{chapter_AI_dive/AI_deploy:id3}}\end{figure}
%
\begin{footnotetext}[982]\sphinxAtStartFootnote
\sphinxnolinkurl{http://www.ailab.cn/ml/20201213109713.html}
%
\end{footnotetext}\ignorespaces 

\paragraph{容器部署}
\label{\detokenize{chapter_AI_dive/AI_deploy:id1}}\begin{itemize}
\item {} 
深度学习的部署方式,主要是利用 tf
serving的方式或者使用onnx的方式,实现了跨平台的模型服务的共享,可以查看onnx的官网:
\sphinxurl{https://onnx.ai/}。这方法可以结合方法3进行部署。

\item {} 
spark模型的部署模式,主流的方法有2种,第一种是PMML文件部署,第二种是使用mleap组件,详情请查看:\sphinxurl{https://github.com/combust/mleap}。第二种方式是一种可以移植的模式,具有一定的普适性,后续可能会有大的应用前景

\item {} 
基于docker+k8s+kubeflow+seldon的一体化部署模式,基于容器编排的方式可以快速实现服务的启动和对外输出,这个方式部署方便,当前对于问题排查和定位比较麻烦,需要做很多的额外的开发工作。

\end{itemize}

Docker\sphinxhyphen{}Cheet\sphinxhyphen{}sheet \sphinxhyphen{} wpppj的文章 \sphinxhyphen{} 知乎
\sphinxurl{https://zhuanlan.zhihu.com/p/50920327}

整个模型工程化部署,主要会聚焦几种场景,第一种是跑批任务,第二种是实时服务。

针对跑批任务,我们尝试采用分而治之的优化,采用数据分割,并行预测,再合并预测结果的方式,提高整体的跑批的速度;有个场景,需要跑5000万的预测数据,没有优化之前需要跑3个小时左右,优化后只要大概32分钟就跑完了,整体的效果提升非常明显。此外,还会尝试另外一种方式,把模型封装成UDF,然后发布在hadoop集群上进行跑批,也实现了分布式跑批的场景,提高了效率。

针对实时在线的服务,核心指标主要是QPS和调用时长,我们在利用PMML部署的模型服务,使用简单的模型,比如LR,整体可以实现QPS
1500,95分位的调用时长在15ms以内。但是对于复杂的模型,比如xgb,FM等复杂模型,还有很大的优化空间,所以工程化落地会制约整个算法迭代的速度及其效果。如果有一个好的工程化团队或者模型serving的工具,是能够大幅度提升整体的模型产能的。工程化本身不是最大的挑战性,主要是需要那些具有开发经验的又懂分布式的人才,其实是可以很好的解决这个问题的。

对于模型类的工程化,到底有哪些地方是不一样的。模型结果本质是一堆参数集合,就像一个配置文件,配置文件的格式不同,有类似pth,pkl,xml,pmml等格式的,模型有数据预处理模块,综上来看,模型部署和正常的应用系统开发不同点在模型服务是一个数据流处理过程,模型文件多了一步的解析工作,导致整个工程化的复杂度增加,除了这点之外,其他地方都是一样的。模型服务本身也需要考虑并发、限流、熔断、降级、分流等标准化的工作,从而来提高整体服务的可用性。


\subparagraph{docker部署}
\label{\detokenize{chapter_AI_dive/AI_deploy:docker}}
\sphinxurl{https://tianchi.aliyun.com/competition/entrance/231759/tab/226}

\sphinxurl{https://dockerpractice.readthedocs.io/}

TODO:
GPU:https://tianchi.aliyun.com/competition/entrance/531863/introduction?spm=5176.12281949.1003.30.7b9e2448hrZrNs

Jenkins自动化部署docker脚本:\sphinxurl{https://blog.csdn.net/kepengs/article/details/114029593?spm=1001.2014.3001.5501}


\subparagraph{好处}
\label{\detokenize{chapter_AI_dive/AI_deploy:id2}}\begin{itemize}
\item {} 
不仅能节约时间,快速部署和启动(秒级甚至毫秒级),还能节约成本,相于较虚拟机动辄几个G的磁盘空间,docker容器可以减少到MB级;

\item {} 
方便部署,直接运行已经配好的容器,解决开发人员由于安装环境带来的部署困难;docker的镜像提供了除内核外完整的运行时环境,确保环境一致性,从而不会在出现“这段代码在我机器上没问题”这类问题;

\item {} 
方便持续集成,通过与代码进行关联使持续集成非常方便;

\item {} 
方便构建基于SOA架构或微服务架构的系统,通过服务编排,更好的松耦合;

\item {} 
标准化应用发布,可以多平台部署。

\end{itemize}


\subsubsection{AI监控}
\label{\detokenize{chapter_AI_dive/AI_monitor_control:ai}}\label{\detokenize{chapter_AI_dive/AI_monitor_control::doc}}
算法模型的监控指标体系(后面简称监控体系),就是将ψ务数据进行采集,同时用可视化图表展现给用户,并且提供相应的告警功能。

一般来说,当业务线初建的时候,我们可以不用考虑太多监控体系的需求,因为我们需要把精力放到怎么让业务“活下去”。但是当业务“活”下来之后,我们就要开始考虑搭建监控体系,让模型能够“活得更好”。那么,监控体系是怎么做到的呢?

具体来说,通过监控体系我们可以知道:
\begin{itemize}
\item {} 
当前这条业务的现状和过去业务数据的对比

\item {} 
当前业务是否正常,可能存在的问题,并且通过这些问题追溯原因

\item {} 
未来业务的趋势,可能的完善方向

\end{itemize}


\paragraph{例子}
\label{\detokenize{chapter_AI_dive/AI_monitor_control:id1}}
一家创业公司,这个公司是给银行、互金机构提供风控模型的。它提供的产品形式是API接口,你可以理解为是,银行给我们一个用户的手机号,我们告诉银行这个用户的风险分是多少。银行会结合我们提供的风险分和其他的数据对用户的风险进行二次判断,来决定是否给用户进行放款。

我们遇到的问题是,有时候接口会突然报异常,模型效果会逐步下降,但是产品侧却抓不到这样的数据,模型侧也没有对模型进行监控。最后,客户反过来投诉我们,这对公司的口碑造成了影响。
\begin{itemize}
\item {} 
首先是明确项目的业务背景,这个很容易得出,就是我们的內部数据监控和告警出现了问题。

\item {} 
其次是明确我们的目标用户,以及要解决的问题。我们的目标用户应该是产品经理自己,以及B端的商务运营同学。要解决的问题就是及时发现模型上的问题,在客户发现之前尽快修复、减少客诉。同时归纳这些问题,反哺模型和研发侧,对技术人员提出更高的要求。

\item {} 
最后就是去解决问题了。基于我们的背景和面对的问题,产品的定位就很清楚了:给产品和运营同学提供套,能够査看所有模型,冋时监控模型性能指标和稳定性指标,并且可以做到实时报警的工具。这个工具需要实现的功能列表梳理出来之后是下面这样的

\end{itemize}

\begin{figure}[H]
\centering
\capstart

\noindent\sphinxincludegraphics{{function_table}.png}
\caption{功能列表}\label{\detokenize{chapter_AI_dive/AI_monitor_control:id2}}\end{figure}

在这里,比较重要的监控功能点是模型全景,也就是监控首页或者说是总览页。虽然它不需要给出具体的模型指标,但要展示岀都有哪些模型在用,调用过程中是否有异常,这方便我们根据异常下钻到明细信息另一个重要的监控功能点就是模型性能指标和稳定性指标,这需要梖据模型的类型,分别去展示模型近期的性能指标波动图,在图中需要展示模型的正常范围值。这个正常范围值,我们是根据实际业务定义的,比如我们对于KS要求比较高,所以把范围值定义在25\sphinxhyphen{}40之间。

除此之外,模型输岀结果指标也是一个重要的监控功能点。为什么要监控模型的输岀结果指标呢?我们之前就因为没监控模型输岀而岀现了大问题。当时,模型给到客户的产品输岀范围是{[}0,100{]},但模型底层依赖的某个数据未更新,这让模型输岀了大于100的数据。同时,在工程部署时候,模型也没有对输岀值进行二次处理,这就导致客户最终拿到的是不合理的结果。
因此,我们不但要监控模型的输岀,也要对不合理的输岀及时告警。

不仅如此,我们还要注意不同指标的监控周期也不同。比如,我们的信用分模型是按月打分,所以,相应的KS、AUC、PS指标也都要按月更新。这需要结合我们模型的实际情况进行设置。


\subsubsection{人才培养}
\label{\detokenize{chapter_AI_dive/develop:id1}}\label{\detokenize{chapter_AI_dive/develop::doc}}

\paragraph{选择应届生}
\label{\detokenize{chapter_AI_dive/develop:id2}}

\subparagraph{缺乏安全感}
\label{\detokenize{chapter_AI_dive/develop:id3}}
从学校进入职场是人生的重要转变之一,任何人在面对这种重要的转变时,都会缺乏安全感。因此,能够遇到一位导师对于应届毕业生来说是莫大的幸运,毕业生会极其珍惜这样的机会并非常尊重这位导师。


\subparagraph{可塑性强}
\label{\detokenize{chapter_AI_dive/develop:id4}}
工作以后,大家都会发现在学校里学的知识和实际工作基本上是没有关联的,因此应届毕业生就像是一张白纸,可塑性极强。在这种情况下,导师更容易为他们传授自身的技能,而且应届毕业生也更愿意接受和学习。如果一个人已经有工作经验的话,那么一定会和导师在理念或工作习惯上有很多不同,应届毕业生基本上不存在这样的问题。


\subparagraph{有学习的态度、主动性更强}
\label{\detokenize{chapter_AI_dive/develop:id5}}
刚进入职场的毕业生是抱有学习态度的,他们有年轻的身体和积极的心态,能够更好地融入工作中。至少大多数时候他们还是很听导师话的,导师告诉他们一件事情,他们一定会按照导师的想法和思路去执行,而不会和导师反复讨论或对着干。虽然完全听话并不一定是一件好事,但是对于应届毕业生来说,按照前辈的指示去完成一件事情是极其重要的,因为新人首先要做的事情就是执行,执行能够让新人快速摸索出一条道路,主动地提出自己的意见。如果新人什么都没做就开始提意见,那就证明这个人太浮躁。


\paragraph{如何更好地培养应届毕业生}
\label{\detokenize{chapter_AI_dive/develop:id6}}

\subparagraph{让其加入具体的项目}
\label{\detokenize{chapter_AI_dive/develop:id7}}
产品经理如果想要更好地培养人才,就一定要把人才放到具体的项目中,这是最有效的方法。因此,应届毕业生除了帮忙做一些日常、琐碎的工作,还要参与具体的项目。

在参与项目时,产品经理要让应届毕业生了解以下内容。
\begin{enumerate}
\sphinxsetlistlabels{\arabic}{enumi}{enumii}{}{.}%
\item {} 
为什么要做这个项目?即项目的背景、目标。

\item {} 
这个项目都需要哪些人参与?即项目组的构成和合作。

\item {} 
项目组成员是如何分工的,他们的责任分别是什么?即不同岗位的职责。

\item {} 
整个项目的流程是怎样的?即项目的流程管理和周期。

\item {} 
项目中的关键事项是什么?即更清晰的项目目标和管理。

\end{enumerate}

产品经理还要告诉应届毕业生了解以上内容的方法。
\begin{enumerate}
\sphinxsetlistlabels{\arabic}{enumi}{enumii}{}{.}%
\item {} 
主动和导师沟通,了解整个项目的流程。

\item {} 
主动要求参与项目,并输出会议纪要及项目每天的进度。

\item {} 
主动和参与项目的每一个人沟通,了解他们的工作职责。

\item {} 
主动承担项目中的脏活、累活和一些不起眼的小事,包括项目组的会议室预订、会议纪要输出、项目每天的跟进和更新、会议的召集、拿外卖等,简单来说就是要完全融入项目组。

\end{enumerate}

产品经理要带领应届毕业生参与项目,同时也要将自己和他们进行绑定,让他们有问题第一时间找自己解决。


\subparagraph{督促他们主动进行分析、总结}
\label{\detokenize{chapter_AI_dive/develop:id8}}
善于分析、总结是每一个产品经理必备的能力,对于应届毕业生来说也不例外。因此,产品经理除了要带领应届毕业生参与具体的项目,还要让他们学会在工作中主动进行分析、总结。那么,要分析、总结什么呢?主要包括以下三个方面的内容。
\begin{enumerate}
\sphinxsetlistlabels{\arabic}{enumi}{enumii}{}{.}%
\item {} 
做竞争对手的产品分析,可以以周为单位输出竞争对手的分析报告。

\item {} 
收集行业内的新闻资讯和好文章,并第一时间在群里进行分享。

\item {} 
按照项目、时间或其他方式输出总结,包括在某个时间具体做了什么事情、达到了什么效果、实施过程中存在的问题有哪些、有针对性的改进意见、其他注意事项等。

\end{enumerate}

产品经理要让应届毕业生主动地去发声、更多地去参与,这也代表着产品经理自己有更多的发声和参与,能够进一步地提升产品经理在这个团队中的影响力。


\subparagraph{强调数据的重要性}
\label{\detokenize{chapter_AI_dive/develop:id9}}
产品经理还需要给应届毕业生强调数据的重要性。虽然应届毕业生短期内很难真正地通过数据化落地一些具体的解决方案,但是现阶段他们要做的是养成看数据的习惯。因此,产品经理要让应届毕业生每天一到办公室就先看
30 分钟的数据,并定期对他们进行抽查。


\subsection{AI公司研究}
\label{\detokenize{chapter_AI_company/index:ai}}\label{\detokenize{chapter_AI_company/index:chap-company}}\label{\detokenize{chapter_AI_company/index::doc}}
​


\subsubsection{公司研究}
\label{\detokenize{chapter_AI_company/company_research:id1}}\label{\detokenize{chapter_AI_company/company_research::doc}}
在去一个公司面试的时候,你需要提前做好准备。不管是猎头联系你,还是你主动投递简历,你都要对公司的行业背景、公司的背景、创始人或者总裁的背景、公司的产品、竞争对手等信息做公开收集,以便更好地了解公司,了解这个职位。

下面来说说具体要怎样收集这些信息。\sphinxhref{https://weread.qq.com/web/reader/46532b707210fc4f465d044kc0c320a0232c0c7c76d365a}{1}%
\begin{footnote}[983]\sphinxAtStartFootnote
\sphinxnolinkurl{https://weread.qq.com/web/reader/46532b707210fc4f465d044kc0c320a0232c0c7c76d365a}
%
\end{footnote}

问问自己如何改进产品,以及什么样的指标可以用来衡量这些产品的成功。通过阅读他们的官博,来了解每家公司的数据科学家所做的工作也是很有帮助的。通过这种调研,你才能在面试中进行更深入、最终效果更好的对话。
\sphinxhref{https://www.infoq.cn/article/IPDVRNxwJVsx3ZGrgwzW}{2}%
\begin{footnote}[984]\sphinxAtStartFootnote
\sphinxnolinkurl{https://www.infoq.cn/article/IPDVRNxwJVsx3ZGrgwzW}
%
\end{footnote}


\subsubsection{百度}
\label{\detokenize{chapter_AI_company/baidu:id1}}\label{\detokenize{chapter_AI_company/baidu::doc}}

\paragraph{人物}
\label{\detokenize{chapter_AI_company/baidu:id2}}
景鲲:现任百度集团副总裁、百度智能生活事业群组(SLG)总经理,百度人工智能产品委员会主席。全面负责小度系列硬件、小度助手和小度对话式人工智能操作系统(DuerOS)的产品、研发、运营、商务等工作。
曾历任百度搜索公司产品委员会主席、大搜索总产品架构师。加入百度之前担任微软首席研发总监,负责微软必应搜索亚洲市场的研发工作,也是微软小冰的创造者。\sphinxhref{https://baike.baidu.com/item/\%E6\%99\%AF\%E9\%B2\%B2/20432174}{1}%
\begin{footnote}[985]\sphinxAtStartFootnote
\sphinxnolinkurl{https://baike.baidu.com/item/\%E6\%99\%AF\%E9\%B2\%B2/20432174}
%
\end{footnote}


\paragraph{小度}
\label{\detokenize{chapter_AI_company/baidu:id3}}
2018年,百度的小度智能音箱,以及华为AI音箱相继跟上,这些头部大厂率先抢占了市场和人心,而后第二梯队的思必驰、出门问问等产品又进一步参与了对话市场的瓜分。

2020年9月30日,百度将旗下智能生活事业群组业务小度科技拆分,完成独立融资。小度也被独立拆分\sphinxhref{https://wqw547243068.github.io/2020/04/29/dialogue-system/\#\%E5\%AF\%B9\%E8\%AF\%9D\%E7\%AE\%A1\%E7\%90\%86}{3}%
\begin{footnote}[986]\sphinxAtStartFootnote
\sphinxnolinkurl{https://wqw547243068.github.io/2020/04/29/dialogue-system/\#\%E5\%AF\%B9\%E8\%AF\%9D\%E7\%AE\%A1\%E7\%90\%86}
%
\end{footnote}


\paragraph{产品吐槽}
\label{\detokenize{chapter_AI_company/baidu:id4}}
百度百科:半屏的时候目录没了。。\sphinxhref{https://baike.baidu.com/item/\%E5\%B0\%8F\%E5\%86\%B0/19880611?fromtitle=\%E5\%BE\%AE\%E8\%BD\%AF\%E5\%B0\%8F\%E5\%86\%B0\&fromid=14076870\#reference-\%5B36\%5D-20599544-wrap}{2}%
\begin{footnote}[987]\sphinxAtStartFootnote
\sphinxnolinkurl{https://baike.baidu.com/item/\%E5\%B0\%8F\%E5\%86\%B0/19880611?fromtitle=\%E5\%BE\%AE\%E8\%BD\%AF\%E5\%B0\%8F\%E5\%86\%B0\&fromid=14076870\#reference-\%5B36\%5D-20599544-wrap}
%
\end{footnote}


\subsubsection{阿里巴巴}
\label{\detokenize{chapter_AI_company/alibaba:id1}}\label{\detokenize{chapter_AI_company/alibaba::doc}}

\paragraph{NLP}
\label{\detokenize{chapter_AI_company/alibaba:nlp}}
\sphinxurl{https://ai.aliyun.com/nls}


\paragraph{AI Lab}
\label{\detokenize{chapter_AI_company/alibaba:ai-lab}}
2021年1月,阿里AI Lab解散,天猫精灵凉凉,阿里AI
Labs成为国内主要互联网公司里第一个被关闭的人工智能实验室。

阿里巴巴人工智能实验室成立于2016年,2017年7月5日首次公开亮相,巅峰的时候阿里AI
Labs团队有上千人,旗下孵化出来的最重要的产品就是天猫精灵。

2020年9月,阿里AI Lab总经理陈丽娟(浅雪)转入大钉钉事业部;阿里巴巴AI
Labs北京研发中心负责人、语音助手首席科学家聂再清,已经于10月加入清华大学任教。

聂再清之前曾经在微软亚洲研究院工作过13年,主要负责微软自然语言理解、实体挖掘的研发工作,是微软学术搜索、人立方,以及企业智能助理
EDI 的发起人和负责人。


\subsubsection{蚂蚁集团}
\label{\detokenize{chapter_AI_company/antgroup:id1}}\label{\detokenize{chapter_AI_company/antgroup::doc}}

\paragraph{企业图谱}
\label{\detokenize{chapter_AI_company/antgroup:id2}}
企业图谱(风报):一款基于NLP处理技术的企业金融智能信息服务系统,涵盖工商、诉讼、税务、行政处罚、投融资、高管变动、新闻事件等26大类企业情报信息。“风报”包含全国近4,200万家工商登记主体信息,汇聚全网超过4万个数据来源的近10亿条行政公示、审判流程、企业信息披露和新闻媒体报道,并通过人工智能技术将非结构化的文本进行清洗、分析、关联,形成结构化的企业情报信息。\sphinxhref{https://damo.alibaba.com/labs/finance-intelligence/}{1}%
\begin{footnote}[988]\sphinxAtStartFootnote
\sphinxnolinkurl{https://damo.alibaba.com/labs/finance-intelligence/}
%
\end{footnote}


\paragraph{蚂蚁智能}
\label{\detokenize{chapter_AI_company/antgroup:id3}}
\begin{figure}[H]
\centering
\capstart

\noindent\sphinxincludegraphics{{ant_AI_service}.png}
\caption{蚂蚁智能服务整体架构}\label{\detokenize{chapter_AI_company/antgroup:id5}}\end{figure}

从用户服务诉求角度,可以大致划分为\sphinxstylestrong{咨询类、求助类、举报类}三种类型。从支付宝钱包中“我的客服”的真实使用数据分析,大部分的用户问题是能够通过自助渠道得到较好的服务。目前蚂蚁智能机器人(我的客服)应用已经承接了日常90\%的服务诉求,服务满意度已经接近人工服务满意度。

在蚂蚁智能服务中,我们大量采用了以深度学习和自然语言处理为核心的人工智能技术,其中数据技术发挥了非常重要的作用。在“我的客服中”,以用户的行为轨迹作为特征,训练深度神经网络模型,精准“猜测”用户问题。我们构建了一个模型预测的数据闭环,模型产生预测结果推荐给用户,用户点击形成反馈数据,反过来进一步训练和改进模型。通过这样一个数据闭环,一个场景的问题预测从冷启动到自动更新全部自动化。在热线电话的IVR中,我们采用了国际领先的基于深度学习的语音技术,改变传统热线CC系统的菜单式引导模式,让用户直接通过\sphinxstylestrong{对话语音描述需要求助的问题,理解用户语义进行精准的用户引导}。用户画像和行为分析能够根据用户的不同状况,准确分析用户问题的复杂度和紧急程度,智能化引导用户进入最合适的渠道进行问题处理。例如对于涉及到安全、欺诈类的问题第一时间自动切换到VIP人工服务,最大化程度给用户提供安全保障。

最后,我们采用先进的文本聚类技术,从客服对话记录中挖掘用户问题和每个问题的答案,这样就生产出更加贴合用户真实诉求的知识库。其中的问题到答案的映射也通过深度神经网络完成。深度学习,自然语言处理,大数据处理,以及智能语音等技术的应用,有效提升了服务能力,缩短用户服务路径,从而大幅度优化用户服务质量和服务体验。智能服务体系的广泛应用,带动了服务行业的全面升级,把服务模式从繁重人力投入的1.0时代升级到机器智能为代表的2.0时代,非常贴合“安全、普惠、绿色”的整体社会目标,具备广阔的应用前景和巨大的社会价值。在此基础上,蚂蚁金服已经在服务赋能生态的目标上走出了坚实的一步,把蚂蚁自身业务沉淀的智能服务整体解决方案在蚂蚁金融云上打造了SAAS化的蚂蚁云客服产品,对外赋能合作商户、合作机构等生态合作伙伴。
\sphinxhref{https://developer.aliyun.com/article/58930}{2}%
\begin{footnote}[989]\sphinxAtStartFootnote
\sphinxnolinkurl{https://developer.aliyun.com/article/58930}
%
\end{footnote}


\paragraph{金融云}
\label{\detokenize{chapter_AI_company/antgroup:id4}}
金融云PaaS是从2014年中开始研发的,目前已经承载了网商银行以及另外两个核心业务,后续会以公有云和专有云两种模式对外提供。之所以会有金融云PaaS这个项目,是因为蚂蚁这些年来在大型分布式系统领域涉及的SOA、消息通讯、水平扩展、分库切片、数据一致、监控、安全等技术方向积累了大量的中间件以及与之完整配套的监控运维研发流程体系,这一切在性能和稳定性以及扩展性上做的都不错,能够有效的支撑蚂蚁的业务发展,并应对『双11』这样的高负荷挑战。很多金融客户与伙伴都对此非常感兴趣,所以我们希望能够把这一整套的技术上云并产品化,以
PaaS的方式整体对外输出,帮助金融行业的客户使用云计算技术去IOE,帮助他们解决我们已经解决的技术问题,让他们能专注于业务逻辑。上帝的归上帝,凯撒的归凯撒。\sphinxhref{https://developer.aliyun.com/article/58769}{3}%
\begin{footnote}[990]\sphinxAtStartFootnote
\sphinxnolinkurl{https://developer.aliyun.com/article/58769}
%
\end{footnote}


\subsubsection{支付宝}
\label{\detokenize{chapter_AI_company/alipay:id1}}\label{\detokenize{chapter_AI_company/alipay::doc}}

\paragraph{余额宝}
\label{\detokenize{chapter_AI_company/alipay:id2}}
余额宝的本质其实还是货币基金,这是早就出来的东西。

但是,为什么我们之前没有关注它?因为,余额宝改变了它的玩法或者说模式,它让货币基金的购买、收益和赎回变得特别Easy,就跟在银行存活期一样Easy,而且还可以随存随取,有阿里巴巴信任背书,还完全不用担心跑路赎不回来的风险。

后来,BAT都陆续发布了自己的金融理财产品,而且一个比一个利率高,虽然它们也是拿着我们的钱最终走向了银行作为投资存款和债券投资之类的用途,但是它们却一夜之间获得大部分原来属于银行的市场份额。

银行,这个曾经不可一世的行业,被互联网行业巨头们使用了“互联网”这个高维武器将广大人民的闲钱汇聚起来,虽然大部分钱都还是存入了银行,但是却能让用户有更高的收益。因此,2014年开始各大银行纷纷开始想各种办法留住用户的钱,于是提高利率、用“协议存款”的方式从巨头手上借回来(这就导致了银行的利润实际下降),提高运营效率为客户提高更好的体验(逼迫自己升维以免受更多的打击)等等等等,于是,广大的客户终于翻身做了一把真正的甲方。

如刘润老师所说,从余额宝的打击故事可以看出,余额宝重构了资金流的商业价值链条,逼迫银行不得不提高运营效率和利率,使得最终客户能够扬眉吐气成为最后的受益者。

画外音:现在余额宝的利率相较于13\textasciitilde{}14年也变得很低了,利率的市场化回到健康的平衡也是国家机器需要考虑的事情。


\paragraph{历史}
\label{\detokenize{chapter_AI_company/alipay:id3}}
自从 2008
年首次引入水电煤缴费服务以来,支付宝一直在把线上线下各类场景的服务纳入到版图中来,比如政务服务,阿里系的饿了么、飞猪等生活服务,以及通过生活号和小程序等工具提供的第三方服务等。

在2020 年 3 月 10
日的改版中,除了原先最醒目的扫一扫、收付款、卡包等四大功能以外,首页应用中心展示的应用从
11 个增加为 14
个。在应用下方,新增的服务推荐包括吃穿住行、生活服务、金融产品等。

从用户体验的五个层次看支付宝的改版:\sphinxstylestrong{战略层→范围层→结构层→框架层→表现层。}
\sphinxhref{http://www.woshipm.com/pmd/172931.html}{4}%
\begin{footnote}[991]\sphinxAtStartFootnote
\sphinxnolinkurl{http://www.woshipm.com/pmd/172931.html}
%
\end{footnote}


\paragraph{战略层:连接人与服务}
\label{\detokenize{chapter_AI_company/alipay:id4}}
BAT三家的发家史都是在做“连接”这件事。百度连接人与信息,腾讯连接人与人,阿里连接人与商品,这是第一阶段的连接。


\paragraph{范围层:连接和“钱”有关的一切服务}
\label{\detokenize{chapter_AI_company/alipay:id5}}
工具型产品最重要的一点,构建工具使用情景,让用户在这个场景下首先想到这个产品。所以工具型产品的运营,首要的就是使用场景化运营的思维,用户不会因为看到支付宝想到去购物,而会在购物的时候想到用支付宝。

当把支付的场景无限延伸,在此之前一切需要出现“钱”的地方,都换成支付宝来取代呢?

最早的时候,支付宝是PC时代上淘宝时的那个第三方担保的支付工具。后来,手机上出现了支付宝钱包,就是不用输入银行卡号的电子钱包。再后来,有了充话费、买电影票、信用卡还款等生活服务,有了理财产品。接下来呢?继续贯穿到所有的消费场景。

从这个角度上讲,支付宝的方向就是要继续做好并强化这件事:连接和“钱”有关的一切服务。


\paragraph{结构层:支付场景拆解}
\label{\detokenize{chapter_AI_company/alipay:id6}}
从支付宝9.0来看,白崎总结了一下目前支付宝已涵盖的使用场景:
\begin{itemize}
\item {} 
基础功能:付款、收款、转账、汇款

\item {} 
生活服务:手机充值、信用卡还款、外卖、生活缴费(水电燃气电视固话宽带)、城市服务(违章查询、医院挂号等)、电影、打车、机票、景区门票、游戏充值、寄快递、公交卡充值(目前仅支持羊城通)、校园一卡通充值、缴学费、话费卡转让、爱心捐赠。

\item {} 
理财:余额宝、招财宝、娱乐宝、彩票、股票、理财小工具(汇率计算、存款收益、房贷、记账)

\item {} 
商业:众筹、阿里系电商(淘宝、天猫)

\item {} 
社交:红包、亲密付

\item {} 
国民征信体系(最高级):虚拟信用卡(花呗)、个人信用评级(芝麻信用分)

\end{itemize}

百度也做百度钱包这个产品,我们曾给钱包团队提建议说,赶紧支持北京公交一卡通的充值啊,这么大一块市场难道学不会吗?一上线绝对秒杀支付宝。但是我们得到的答案是:idea并不重要,这个点子被无数的人提过,相信支付宝团队也曾想到过。但是为什么大家都没有做呢?不是因为技术上实现不了,而是人家北京公交集团不愿意跟你连接,你就拿它没辙儿。

理财服务里,受国内A股行情影响,预计近期支付宝会在股票这个版块上继续重点发力。一旦连接成功,前景非常可观。A股市场里,股票账户里的钱是放在第三方存管银行的,支付宝要做的是什么呢?取代银行!以后大家炒股的钱都直接由支付宝托管。

再谈到最高级的国民征信体系,这是阿里作为企业在做政府的事情。虚拟信用卡,直接借钱给你花,牢牢地把你绑在支付宝上,这是干倒银行的另一步。芝麻信用分,现在大家还看不到它的威力,等到以后出国办签证,贷款买房,通通看你的信用分,这是在美国已经很成熟的一套体系,本来是政府的事情,支付宝却抢先来完成。


\paragraph{深耕数字经济:转移到本地生活 1\sphinxfootnotemark[992]}
\label{\detokenize{chapter_AI_company/alipay:id7}}%
\begin{footnotetext}[992]\sphinxAtStartFootnote
\sphinxnolinkurl{https://www.sohu.com/a/405889196\_114819}
%
\end{footnotetext}\ignorespaces 
支付宝的战略转型:淡化金融标签,将美食、酒店、电影等本地生活服务入口置顶,使产品服务加快渗透人们的日常生活。

在疫情期间,我们日常使用支付宝建康码出行、水电煤缴费、外卖生鲜订购、果蔬商超医药、团购等入口。已经能感受到这个支付app在不知不觉中成为了一个多功能的线上生活服务平台。

体验:各项功能入口横向排布,下拉呈现更多,界面清晰干净,使用简单。点进功能二级页面以后,呈现更多,手机号登录即可;如果之前在关联app上注册过信息,无须重复注册填写,自动呈现;默认支付宝支付;用后退出页面即可,无粘性。总体来说,整洁干净,操作方便
\sphinxhref{https://www.zhihu.com/question/380276570}{2}%
\begin{footnote}[993]\sphinxAtStartFootnote
\sphinxnolinkurl{https://www.zhihu.com/question/380276570}
%
\end{footnote}

意图正在于以金融服务为抓手,在C端向消费延伸,在B端向服务延伸,扩展更多的线下服务与商家场景来拉动新的增长点,做大生态。

支付宝加码布局本地生活生态位,剑指美团,我们可以预见两大巨头对战,本地生活领域硝烟再起。

一些产品管理者还会苦恼:如何制定并管理产品路线、如何根据产品生命周期规划产品目标、如何针对性培养下属、如何分析数据驱动迭代业务?


\paragraph{支付能力 3\sphinxfootnotemark[994]}
\label{\detokenize{chapter_AI_company/alipay:id8}}%
\begin{footnotetext}[994]\sphinxAtStartFootnote
\sphinxnolinkurl{http://www.woshipm.com/it/632590.html}
%
\end{footnotetext}\ignorespaces 
2017 年 5 月,支付宝率先推出了“收钱码”,
这称得上是一款扭转战局的产品,蓝色二维码一时间风行于全国各地商户的柜台上。同年,支付宝在杭州实现了全线刷码出行,年底正式接入了杭州地铁,之后,支付宝从大本营杭州走了出去。\sphinxhref{https://www.chinaz.com/2020/0312/1116987.shtml}{5}%
\begin{footnote}[995]\sphinxAtStartFootnote
\sphinxnolinkurl{https://www.chinaz.com/2020/0312/1116987.shtml}
%
\end{footnote}

「走出淘系」就是支付宝最初的开放过程,把支付能力作为水电煤气一样的连接物,开放给所有行业,哪怕面向是京东(现已下线)、唯品会这些和阿里系电商业务产生竞争关系的公司都开放。

支付宝还开放了新技术场景下的支付能力,比如近期开放给小米和华为的VR
Pay能力,在购物、直播、游戏、社交等VR场景中,用户可以通过凝视、点头、触摸等控制方式完成交易。这里面不仅有密码、指纹等传统校验身份的方式,还有包含声波、虹膜等生物识别技术的支付能力。生物识别会解决未来生活中非常基础的一个问题,那就是如何真实有效得去识别一个人,这种基础能力的研究可不是一天两天可以完成的,所以说开放不仅仅是一种策略,更是一种能力。


\paragraph{财富号 3\sphinxfootnotemark[996]}
\label{\detokenize{chapter_AI_company/alipay:id9}}%
\begin{footnotetext}[996]\sphinxAtStartFootnote
\sphinxnolinkurl{http://www.woshipm.com/it/632590.html}
%
\end{footnotetext}\ignorespaces 
财富号是支付宝近期宣布的一个开放项目,即面向基金、保险、银行等传统金融机构提供开放平台,帮助金融机构更好地服务客户。支付宝的财富号形态和微信的公众号有些类似,支付宝提供基础功能和技术服务,财富号的运营维护交给金融机构。按照支付宝对外公布的资料,财富号里提供产品推荐、互动直播、问答社区、企业动态等模块,满足金融机构对于产品售卖、客户维系、营销活动等一系列连接性的需求。相信除此之外,支付宝会开放一部分原生流量给首批入驻的财富号,这意味着金融机构还将有机会通过支付宝平台获取新客户。


\paragraph{支付宝能力}
\label{\detokenize{chapter_AI_company/alipay:id10}}
支付宝能每秒处理了45.9万笔支付,月活用户超过7.3亿,是美国人口的两倍多。相比之下,贝宝(PayPal)有3.46亿活跃账户。

截至2020年6月的12个月里,蚂蚁集团在中国大陆处理了超过17万亿美元的数字支付。贝宝表示,其2019年的全年支付总额为7120亿美元。蚂蚁集团还为消费者和小企业提供了约3000亿美元的信贷。

该公司上市后,其估值可能在3100亿美元左右。这将使其市值与摩根大通(JPMorgan
Chase)相当,远高于花旗集团(Citigroup)和高盛(Goldman Sachs)。

在技术层面,支付宝也毫不逊色。蚂蚁集团表示,在去年中国一个购物节的高峰节点,其系统每秒处理了45.9万笔支付。相比之下,Visa称其每秒可处理6.5万笔交易。

支付宝第五代风控引擎AlphaRisk模型:
\sphinxurl{http://www.uml.org.cn/ai/202003251.asp}

蚂蚁全服金融科技∧TE匚开发者大赛:\sphinxurl{https://dc.cloud.alipay.com/index\#/compet/topics}


\paragraph{智能中台}
\label{\detokenize{chapter_AI_company/alipay:id11}}
\begin{figure}[H]
\centering
\capstart

\noindent\sphinxincludegraphics{{AI_zhongtai}.png}
\caption{智能中台的简略框架\sphinxhref{https://jsldl.blog.csdn.net/article/details/109759605?utm\_medium=distribute.pc\_relevant.none-task-blog-BlogCommendFromMachineLearnPai2-10.control\&dist\_request\_id=\&depth\_1-utm\_source=distribute.pc\_relevant.none-task-blog-BlogCommendFromMachineLearnPai2-10.control}{9}\sphinxfootnotemark[997]}\label{\detokenize{chapter_AI_company/alipay:id18}}\end{figure}
%
\begin{footnotetext}[997]\sphinxAtStartFootnote
\sphinxnolinkurl{https://jsldl.blog.csdn.net/article/details/109759605?utm\_medium=distribute.pc\_relevant.none-task-blog-BlogCommendFromMachineLearnPai2-10.control\&dist\_request\_id=\&depth\_1-utm\_source=distribute.pc\_relevant.none-task-blog-BlogCommendFromMachineLearnPai2-10.control}
%
\end{footnotetext}\ignorespaces 
智能中台从角色上分成:算法研发、数据研发、工程研发。

从架构上分成:模型、研发平台、底层技术。


\paragraph{更多金融 8\sphinxfootnotemark[998]}
\label{\detokenize{chapter_AI_company/alipay:id12}}%
\begin{footnotetext}[998]\sphinxAtStartFootnote
\sphinxnolinkurl{https://blog.csdn.net/huanongjingchao/article/details/46945061}
%
\end{footnotetext}\ignorespaces 

\subparagraph{互联网金融的布道者,余额宝}
\label{\detokenize{chapter_AI_company/alipay:id13}}
余额宝于2013年6月正式推出——其年被称为“中国互联网金融元年”——现如今,寻常百姓都知道货币基金、网络理财、互联网金融等术语,余额宝功不可没。它同样隶属于蚂蚁金服。截至今年6月底,起用户数已超1.24亿,资金规模超5700亿元——与余额宝对接的传统金融机构——天弘增利宝基金——成为我国规模最大的货币基金。


\subparagraph{更多尝试,招财宝}
\label{\detokenize{chapter_AI_company/alipay:id14}}
该平台于2014年4月上线,主打投融资理财。招财宝向金融机构开放,服务中小企业和个人,为后者提供高效、低成本的融资服务。产品包含:中小企业贷、个人贷、万能险、分级基金等。目前,平台累计成交额已超150亿元。


\subparagraph{升级版的阿里小贷,蚂蚁小贷}
\label{\detokenize{chapter_AI_company/alipay:id15}}
蚂蚁小贷前身为阿里小贷,旨在为小微企业、网商个人创业者提供互联网化、批量化、数据化的小贷服务。产品包括:阿里信用贷款、网商贷、淘宝信用贷款、订单贷款等。截至2014年3月底,蚂蚁小贷已为超70万家小微企业解决融资需求,累计投放贷款超1900亿元。


\subparagraph{首家互联网商业银行,网商银行}
\label{\detokenize{chapter_AI_company/alipay:id16}}
蚂蚁金服作为主发起人正在筹备网商银行。2014年9月29日,获银监会批复。蚂蚁对网商银行的定位是“立足互联网,充分利用互联网、大数据等技术,挖掘互联网信用,服务小微企业和普通民众”。


\paragraph{金融业务全景图}
\label{\detokenize{chapter_AI_company/alipay:id17}}
\sphinxurl{https://tech.antfin.com/digital-finance}


\subsubsection{蓝象智联}
\label{\detokenize{chapter_AI_company/trustbe:id1}}\label{\detokenize{chapter_AI_company/trustbe::doc}}
蓝象智联是一家金融级隐私计算服务商,公司定位于金融级隐私计算服务,公司核心产品已通过工信部和公安部认证,目前已被应用于多家银行、运营商和零售商业客户。同时,公司也在参与工信部等行业多方安全计算标准的制定。\sphinxhref{https://www.qcc.com/firm/efcbb8220d64ca34f4a85f5e1c0af260.html}{2}%
\begin{footnote}[999]\sphinxAtStartFootnote
\sphinxnolinkurl{https://www.qcc.com/firm/efcbb8220d64ca34f4a85f5e1c0af260.html}
%
\end{footnote}


\paragraph{隐私计算}
\label{\detokenize{chapter_AI_company/trustbe:id2}}
背景:大数据时代下,数据所有权和使用权的矛盾已经十分激化,由于数据没有实体形态,拥有数据使用权的一方往往已经取得了数据的所有权。

同时,数据孤岛和极不通畅的数据沟通所带来的问题同样十分严重,而解决这类问题需要一种新的\sphinxstylestrong{加密学}的技术应用,这也就是后来发展出的隐私计算。\sphinxhref{https://www.cyzone.cn/article/615905.html}{5}%
\begin{footnote}[1000]\sphinxAtStartFootnote
\sphinxnolinkurl{https://www.cyzone.cn/article/615905.html}
%
\end{footnote}

随着个人信息保护法草案的亮相,如何在\sphinxstylestrong{保护数据隐私的前提下解决数据孤岛问题},隐私计算被多个行业都视为颇具潜力的解决方式,特别是在金融行业。该行业是数据密集型行业,也是数据需求密集型行业,其风控、营销等能力都高度依赖于数据的支持——无论这些数据是来自金融机构内部还是外部。

“2B的赛道正在进入黄金发展期,其中数据相关赛道是2B众多赛道中最具有发展空间的,隐私计算是数据赛道的基础技术,我们非常看好隐私计算的前景,未来这个赛道的赢家一定是具备极强商业化落地能力,形成实质标准后将构筑\sphinxstylestrong{极强的生态壁垒},而蓝象团队的综合实力无疑是业内比较领先的。”\sphinxhref{https://m.pedaily.cn/news/460985}{1}%
\begin{footnote}[1001]\sphinxAtStartFootnote
\sphinxnolinkurl{https://m.pedaily.cn/news/460985}
%
\end{footnote}

隐私计算能够让原始数据在不出库情况下实现数据价值交换,从根本上解决数据安全和数据交换价值的矛盾,结合区块链技术,通过分布式架构,不仅从技术层面解决数据“物理分散,逻辑集中”的问题,还能在生产关系层面解决中心化垄断问题,会使得各参与方更加积极的参与数据要素价值发现,这其中将有机会孕育新的业态
\sphinxhref{https://zhuanlan.zhihu.com/p/266323266}{7}%
\begin{footnote}[1002]\sphinxAtStartFootnote
\sphinxnolinkurl{https://zhuanlan.zhihu.com/p/266323266}
%
\end{footnote}

蓝象智联CEO徐敏在“2020世界区块链大会”上表示,数据是金融的核心生产要素,是金融创新的原动力,安全是金融稳健创新的推动力。\sphinxstylestrong{隐私计算}与区块链助力金融实现“一切数据业务化”。隐私计算让数据“安全可用,不可见”。区块链是数据价值交换生态的底层基础设施。区块链和隐私计算是黄金搭档。

隐私计算虽然实现了在多方协作计算过程中的数据可用不可见,但是计算过程和结果缺乏均可验证性。而区块链因其共享账本、智能合约、共识机制等技术特性,可以实现数据的链上存证核验、计算过程关键数据和环节的上链存证回溯,确保计算过程的可验证性。因此将区块链技术对计算的可信证明应用到隐私计算中,可以辅助增强隐私计算任务中数据端到端及全生命周期的隐私性、安全性和可追溯性。《区块链辅助的隐私计算技术工具
技术要求与测试方法》标准聚焦于此,进一步完善了隐私计算的技术规范体系,希望助力行业的健康发展。\sphinxhref{https://www.databench.cn/index.php?m=content\&c=index\&a=show\&catid=10\&id=37}{6}%
\begin{footnote}[1003]\sphinxAtStartFootnote
\sphinxnolinkurl{https://www.databench.cn/index.php?m=content\&c=index\&a=show\&catid=10\&id=37}
%
\end{footnote}

\begin{figure}[H]
\centering
\capstart

\noindent\sphinxincludegraphics{{private_computing_KG}.png}
\caption{隐私计算知识图谱\sphinxhref{https://upload-images.jianshu.io/upload\_images/20618853-887aacdcd77f3721.png}{4}\sphinxfootnotemark[1004]}\label{\detokenize{chapter_AI_company/trustbe:id4}}\end{figure}
%
\begin{footnotetext}[1004]\sphinxAtStartFootnote
\sphinxnolinkurl{https://upload-images.jianshu.io/upload\_images/20618853-887aacdcd77f3721.png}
%
\end{footnotetext}\ignorespaces 

\paragraph{人才}
\label{\detokenize{chapter_AI_company/trustbe:id3}}
\sphinxurl{https://www.trustbe.cn/tdjs}
\begin{itemize}
\item {} 
阿里云智能事业群\sphinxhyphen{}产品经理\sphinxhyphen{}机器学习平台(JD):\sphinxurl{https://talent.alibaba.com/off-campus-position/637541?bucketType=A}

\item {} 
蓝象科技:\sphinxurl{https://www.zhipin.com/job\_detail/b35c1e9ba985fc101nV72tq5EFZZ.html}

\end{itemize}


\subsubsection{腾讯}
\label{\detokenize{chapter_AI_company/tencent:id1}}\label{\detokenize{chapter_AI_company/tencent::doc}}
\sphinxurl{https://ai.qq.com/}


\subsubsection{暗物智能}
\label{\detokenize{chapter_AI_company/dm-ai:id1}}\label{\detokenize{chapter_AI_company/dm-ai::doc}}

\paragraph{名字来源}
\label{\detokenize{chapter_AI_company/dm-ai:id2}}
朱松纯教授认为,人工智能领域存在智能“暗物质”,包括功能、物理、因果、意图、价值等,就是人工智能领域的“Dark”,而Dark
Beyond
Deep。\sphinxhref{https://www.dm-ai.cn/news/\%e5\%a4\%a7\%e5\%92\%96\%e4\%ba\%91\%e9\%9b\%86\%e5\%9b\%be\%e7\%81\%b5\%e5\%a4\%a7\%e4\%bc\%9a\%ef\%bc\%8c\%e6\%9a\%97\%e7\%89\%a9\%e6\%99\%ba\%e8\%83\%bd\%e5\%a4\%a7\%e6\%94\%be\%e5\%bc\%82\%e5\%bd\%a9/}{6}%
\begin{footnote}[1005]\sphinxAtStartFootnote
\sphinxnolinkurl{https://www.dm-ai.cn/news/\%e5\%a4\%a7\%e5\%92\%96\%e4\%ba\%91\%e9\%9b\%86\%e5\%9b\%be\%e7\%81\%b5\%e5\%a4\%a7\%e4\%bc\%9a\%ef\%bc\%8c\%e6\%9a\%97\%e7\%89\%a9\%e6\%99\%ba\%e8\%83\%bd\%e5\%a4\%a7\%e6\%94\%be\%e5\%bc\%82\%e5\%bd\%a9/}
%
\end{footnote}
大数据背后的物体功效、因果链条、行为动机、价值决策、内心认知等人工智能领域的“暗物质”–“小数据、大任务”


\paragraph{历程}
\label{\detokenize{chapter_AI_company/dm-ai:id3}}
\sphinxurl{https://www.dm-ai.cn/aboutus/}

\sphinxurl{https://www.tianyancha.com/company/3205042145}


\paragraph{好评}
\label{\detokenize{chapter_AI_company/dm-ai:id4}}
「AI中国」机器之心2020年度榜单中凭借出色的技术创新能力与优秀的商业落地成果,成功入选“最强人工智能公司TOP30”。要求入选企业具备成熟的商业模式,主营业务在近三年保持较高增长率,并在其主要关注的细分市场领域有成熟的产品服务,且已获得该领域主导型市场地位。
\sphinxhref{https://www.dm-ai.cn/news/\%e4\%ba\%a7\%e4\%b8\%9a\%e8\%90\%bd\%e5\%9c\%b0\%e6\%88\%90\%e6\%9e\%9c\%e7\%aa\%81\%e5\%87\%ba\%ef\%bc\%81\%e6\%9a\%97\%e7\%89\%a9\%e6\%99\%ba\%e8\%83\%bd\%e5\%85\%a5\%e9\%80\%89\%e6\%9c\%80\%e5\%bc\%ba\%e4\%ba\%ba\%e5\%b7\%a5\%e6\%99\%ba\%e8\%83\%bd/}{1}%
\begin{footnote}[1006]\sphinxAtStartFootnote
\sphinxnolinkurl{https://www.dm-ai.cn/news/\%e4\%ba\%a7\%e4\%b8\%9a\%e8\%90\%bd\%e5\%9c\%b0\%e6\%88\%90\%e6\%9e\%9c\%e7\%aa\%81\%e5\%87\%ba\%ef\%bc\%81\%e6\%9a\%97\%e7\%89\%a9\%e6\%99\%ba\%e8\%83\%bd\%e5\%85\%a5\%e9\%80\%89\%e6\%9c\%80\%e5\%bc\%ba\%e4\%ba\%ba\%e5\%b7\%a5\%e6\%99\%ba\%e8\%83\%bd/}
%
\end{footnote}

“未来独角兽”创新企业榜单定位于发掘具有成为“独角兽”的潜力创新企业,面向市场估值1亿美元至10亿美元之间的创新企业;“高精尖”企业榜单则定位于有核心技术创新,具备较强行业影响力的技术企业。\sphinxhref{https://www.dm-ai.cn/news/\%e6\%9c\%aa\%e6\%9d\%a5\%e5\%8f\%af\%e6\%9c\%9f\%ef\%bc\%81\%e6\%9a\%97\%e7\%89\%a9\%e6\%99\%ba\%e8\%83\%bd\%e5\%90\%8c\%e6\%97\%b6\%e5\%85\%a5\%e9\%80\%89\%e5\%b9\%bf\%e5\%b7\%9e\%e6\%9c\%aa\%e6\%9d\%a5\%e7\%8b\%ac\%e8\%a7\%92\%e5\%85\%bd/}{2}%
\begin{footnote}[1007]\sphinxAtStartFootnote
\sphinxnolinkurl{https://www.dm-ai.cn/news/\%e6\%9c\%aa\%e6\%9d\%a5\%e5\%8f\%af\%e6\%9c\%9f\%ef\%bc\%81\%e6\%9a\%97\%e7\%89\%a9\%e6\%99\%ba\%e8\%83\%bd\%e5\%90\%8c\%e6\%97\%b6\%e5\%85\%a5\%e9\%80\%89\%e5\%b9\%bf\%e5\%b7\%9e\%e6\%9c\%aa\%e6\%9d\%a5\%e7\%8b\%ac\%e8\%a7\%92\%e5\%85\%bd/}
%
\end{footnote}

德勤“2019广州高科技高成长20强暨广州明日之星”、“2019中国高科技高成长50强暨中国明日之星”榜单相继揭晓,DMAI凭借在细分领域取得的领先优势和巨大的发展潜力同时荣获“广州明日之星”和“中国明日之星”
\sphinxhref{https://www.dm-ai.cn/news/\%e6\%9a\%97\%e7\%89\%a9\%e6\%99\%ba\%e8\%83\%bd\%e8\%8d\%a3\%e8\%86\%ba2019\%e5\%be\%b7\%e5\%8b\%a4\%e6\%98\%8e\%e6\%97\%a5\%e4\%b9\%8b\%e6\%98\%9f/}{10}%
\begin{footnote}[1008]\sphinxAtStartFootnote
\sphinxnolinkurl{https://www.dm-ai.cn/news/\%e6\%9a\%97\%e7\%89\%a9\%e6\%99\%ba\%e8\%83\%bd\%e8\%8d\%a3\%e8\%86\%ba2019\%e5\%be\%b7\%e5\%8b\%a4\%e6\%98\%8e\%e6\%97\%a5\%e4\%b9\%8b\%e6\%98\%9f/}
%
\end{footnote}


\paragraph{融资情况}
\label{\detokenize{chapter_AI_company/dm-ai:id5}}
\sphinxurl{https://www.tianyancha.com/brand/b5c88428508}

2019\sphinxhyphen{}03\sphinxhyphen{}28已完成数千万美元Pre\sphinxhyphen{}A轮融资。本轮融资由赛领领投,IDG、鼎晖、高捷、将门等投资机构共同参与投资。

2021\sphinxhyphen{}01\sphinxhyphen{}28:5亿人民币 A轮


\paragraph{领导}
\label{\detokenize{chapter_AI_company/dm-ai:id6}}
公司的使命是“Lift Humanity with Cognitive AI
Platforms(以强认知AI平台提升人类福祉)”。

DMAI创始人朱松纯教授是全球著名计算机视觉专家、统计与应用数学家,于1996年获哈佛大学计算机博士学位,在国际顶级期刊和会议上发表论文260余篇,获得多个国际学术奖项,如三次问鼎计算机视觉领域的马尔奖、获得国际认知科学学会颁发的认知建模奖等等。
马尔奖获得者、赫尔姆霍茨奖获得者、UCLA教授、IEEE
Fellow。More:\sphinxhref{https://zhuanlan.zhihu.com/p/245049303}{16}%
\begin{footnote}[1009]\sphinxAtStartFootnote
\sphinxnolinkurl{https://zhuanlan.zhihu.com/p/245049303}
%
\end{footnote}

年轻人要能沉得住气,做人做事都要能坚守信念,一辈子只做一件事,把它做好,就能有所成就。性格决定命运,要特别坚韧,Harry也说过,脸皮要厚一点,要经得住老师和同行的批评。聪明的学生尤其要能克服这个问题。

据公开资料显示,林倞曾担任商汤科技执行研发总监及研究院副院长,在人工智能领域拥有丰富的实战经验,多次领导和实施大规模高并发的AI应用项目,成功服务于亿级别终端用户。
\sphinxhref{https://www.dm-ai.cn/news/\%e5\%bc\%95\%e9\%a2\%86\%e5\%bc\%ba\%e8\%ae\%a4\%e7\%9f\%a5ai\%e8\%87\%aa\%e4\%b8\%bb\%e5\%88\%9b\%e6\%96\%b0\%ef\%bc\%8c\%e6\%9a\%97\%e7\%89\%a9\%e6\%99\%ba\%e8\%83\%bd\%e4\%ba\%ae\%e7\%9b\%b8\%e7\%ac\%ac22\%e5\%b1\%8a\%e9\%ab\%98\%e4\%ba\%a4\%e4\%bc\%9a/}{3}%
\begin{footnote}[1010]\sphinxAtStartFootnote
\sphinxnolinkurl{https://www.dm-ai.cn/news/\%e5\%bc\%95\%e9\%a2\%86\%e5\%bc\%ba\%e8\%ae\%a4\%e7\%9f\%a5ai\%e8\%87\%aa\%e4\%b8\%bb\%e5\%88\%9b\%e6\%96\%b0\%ef\%bc\%8c\%e6\%9a\%97\%e7\%89\%a9\%e6\%99\%ba\%e8\%83\%bd\%e4\%ba\%ae\%e7\%9b\%b8\%e7\%ac\%ac22\%e5\%b1\%8a\%e9\%ab\%98\%e4\%ba\%a4\%e4\%bc\%9a/}
%
\end{footnote}

“我们目前正在开发AI+教育的产品,用强认知交互提升教育行业效率,提供个性化的智适应解决方案”。DMAI董事长助理、大中华区运营副总裁董乐表示,未来DMAI将逐渐进行AI+各个产业领域的迁移,辐射健康、新零售、娱乐等垂直领域。
\sphinxhref{https://www.dm-ai.cn/news/\%e6\%9a\%97\%e7\%89\%a9\%e6\%99\%ba\%e8\%83\%bddmai\%e8\%90\%bd\%e6\%88\%b7\%e5\%8d\%97\%e6\%b2\%99-\%e6\%89\%93\%e9\%80\%a0\%e6\%96\%b0\%e4\%b8\%80\%e4\%bb\%a3\%e4\%ba\%ba\%e5\%b7\%a5\%e6\%99\%ba\%e8\%83\%bd\%e4\%bc\%81\%e4\%b8\%9a/}{4}%
\begin{footnote}[1011]\sphinxAtStartFootnote
\sphinxnolinkurl{https://www.dm-ai.cn/news/\%e6\%9a\%97\%e7\%89\%a9\%e6\%99\%ba\%e8\%83\%bddmai\%e8\%90\%bd\%e6\%88\%b7\%e5\%8d\%97\%e6\%b2\%99-\%e6\%89\%93\%e9\%80\%a0\%e6\%96\%b0\%e4\%b8\%80\%e4\%bb\%a3\%e4\%ba\%ba\%e5\%b7\%a5\%e6\%99\%ba\%e8\%83\%bd\%e4\%bc\%81\%e4\%b8\%9a/}
%
\end{footnote}

高级研发总监梁小丹
\sphinxhref{https://www.dm-ai.cn/news/\%e5\%a4\%a7\%e5\%92\%96\%e4\%ba\%91\%e9\%9b\%86\%e5\%9b\%be\%e7\%81\%b5\%e5\%a4\%a7\%e4\%bc\%9a\%ef\%bc\%8c\%e6\%9a\%97\%e7\%89\%a9\%e6\%99\%ba\%e8\%83\%bd\%e5\%a4\%a7\%e6\%94\%be\%e5\%bc\%82\%e5\%bd\%a9/}{5}%
\begin{footnote}[1012]\sphinxAtStartFootnote
\sphinxnolinkurl{https://www.dm-ai.cn/news/\%e5\%a4\%a7\%e5\%92\%96\%e4\%ba\%91\%e9\%9b\%86\%e5\%9b\%be\%e7\%81\%b5\%e5\%a4\%a7\%e4\%bc\%9a\%ef\%bc\%8c\%e6\%9a\%97\%e7\%89\%a9\%e6\%99\%ba\%e8\%83\%bd\%e5\%a4\%a7\%e6\%94\%be\%e5\%bc\%82\%e5\%bd\%a9/}
%
\end{footnote}:“吴文俊人工智能科学技术奖”,青橙奖,被英国创新基金会评为AI领域杰出女科学家,2016年获中山大学工学博士学位,于2018年10月在卡内基梅隆大学做博士后研究员。已将研究应用于计算机视觉领域中大规模物体检测和分割、多粒度人物解析以及视觉与自然语言处理交叉领域中的人机交互自然问答和医疗诊断对话系统中。研究通过将人类常识和结构化信息结合于智能推理过程,从而使智能系统具备全局认知推理、高鲁棒性和解释性等能力,推动人工智能研究从感知层走向认知层。大三时,第一次听林倞教授的计算机视觉课,觉得特别酷炫。而后硕博期间师从林老师,潜心做计算机视觉的研究。
中国计算机学会CCF 优秀博士论文奖(全国每年仅10人)、 国际人工智能学会ACM
优秀博士论文奖(全国每年仅2人)、
国家级2011高性能计算协同创新中心优秀博士论文奖(全国每年仅10人)、
全球FashionAI\sphinxhyphen{}衣服服饰关键点检测学术竞赛 第二名

DMAI联合创始人兼首席运营官董乐表示,“DMAI将强交互、强认知的核心AI能力与教育场景做深度融合,是引领教育行业系统性变革的内生变量。公司也将汇聚政府、企业、高校等各方力量,持续革新教育行业生态圈,联合推动人工智能教育的发展。”

DMAI研究总监陈崇雨
\sphinxhref{https://www.dm-ai.cn/news/\%e5\%bc\%ba\%e8\%ae\%a4\%e7\%9f\%a5ai\%e8\%9e\%8d\%e9\%80\%9aar\%ef\%bc\%8cdmai\%e4\%b8\%8e\%e6\%96\%b0\%e8\%8a\%82\%e5\%a5\%8f\%e8\%be\%be\%e6\%88\%90\%e6\%88\%98\%e7\%95\%a5\%e5\%90\%88\%e4\%bd\%9c/}{14}%
\begin{footnote}[1013]\sphinxAtStartFootnote
\sphinxnolinkurl{https://www.dm-ai.cn/news/\%e5\%bc\%ba\%e8\%ae\%a4\%e7\%9f\%a5ai\%e8\%9e\%8d\%e9\%80\%9aar\%ef\%bc\%8cdmai\%e4\%b8\%8e\%e6\%96\%b0\%e8\%8a\%82\%e5\%a5\%8f\%e8\%be\%be\%e6\%88\%90\%e6\%88\%98\%e7\%95\%a5\%e5\%90\%88\%e4\%bd\%9c/}
%
\end{footnote}

DMAI首次提出人工智能的“五层认知架构”,旨在实现机器与人类在多模态下的交流与协作,达到人机共生共存的目标。
\sphinxhref{https://www.dm-ai.cn/news/dmai\%e6\%90\%ba\%e5\%bc\%ba\%e8\%ae\%a4\%e7\%9f\%a5ai\%e4\%ba\%ae\%e7\%9b\%b82019\%e4\%b8\%96\%e7\%95\%8c\%e4\%ba\%ba\%e5\%b7\%a5\%e6\%99\%ba\%e8\%83\%bd\%e5\%a4\%a7\%e4\%bc\%9a-\%e6\%88\%90\%e4\%b8\%ba\%e4\%b8\%8a\%e6\%b5\%b7ai/}{9}%
\begin{footnote}[1014]\sphinxAtStartFootnote
\sphinxnolinkurl{https://www.dm-ai.cn/news/dmai\%e6\%90\%ba\%e5\%bc\%ba\%e8\%ae\%a4\%e7\%9f\%a5ai\%e4\%ba\%ae\%e7\%9b\%b82019\%e4\%b8\%96\%e7\%95\%8c\%e4\%ba\%ba\%e5\%b7\%a5\%e6\%99\%ba\%e8\%83\%bd\%e5\%a4\%a7\%e4\%bc\%9a-\%e6\%88\%90\%e4\%b8\%ba\%e4\%b8\%8a\%e6\%b5\%b7ai/}
%
\end{footnote}


\paragraph{科研}
\label{\detokenize{chapter_AI_company/dm-ai:id7}}
吴田富:直接问能不能读研究生,还有很多邮件是撒传单式的,我都直接删除。仅仅提出自己思考的问题,就像我1991年申请研究生院的个人陈述;他在这个领域发表了一些颇有影响力的文章,敢啃硬骨头,比如
top\sphinxhyphen{}down/bottom\sphinxhyphen{}up 算法调度问题,
就是大家说了几十年但是没有人敢去做的事。

朱松纯教授实验室学生王文冠获得2018年ACM中国优秀博士论文奖

ACM TURC 2019 Conference Best Paper由朱松纯教授实验室学生解旭斩获

《EagleEye: Fast Sub\sphinxhyphen{}net Evaluation for Efficient Neural Network
Pruning》提出了一种性能极高的剪枝算法“鹰眼(EagleEye):暗物智能研究副总监苏江博士与高级研究员李百林领衔,联合中山大学团队共同完成。EagleEye主要包含3个模块:策略生成、通道裁剪、自适应批归一化评估模块。


\paragraph{商业落地}
\label{\detokenize{chapter_AI_company/dm-ai:id8}}
暗物智能在打磨核心技术平台的同时,率先探索出将强认知AI与多模态人机交互技术应用于产业实践的道路。通过深入行业场景,暗物智能已实现强认知AI技术在教育、新零售、泛娱乐等多个行业的商业落地。


\subparagraph{教育}
\label{\detokenize{chapter_AI_company/dm-ai:id9}}
朱松纯领衔的暗物智能创始团队一直致力于计算机视觉、机器人技术和人工智能的AOG表征和建模。目前,该方法已成功运用于“AI+教育”产品。依托AOG模型,暗物智能自主研发基于产生式模型的自动出题、解题与讲题模块。系统可以自动从教材、讲义与教案等素材中构建教学知识库,并通过归纳学习方法获取解题规则。而针对变异题型,系统将基于价值函数,在因果状态空间中自动推演出最优解题路径,所有解题路径与规则自动构建AOG表达。

基于AOG的强认知AI算法引擎,提出了一种基于语义对齐的树结构应用题求解器。此技术通过树结构来强化表达和显式约束题目文本中的语义信息,挖掘与解题有关的各类数学知识和必要的常识。有了“读题”的能力,DMAI通用求解器不仅在准确率上实现了质的提升,还在教育智能化产品中扮演“中枢”角色。从层层递进的解题演示,到精细至每一个步骤的自动批改,谙心助教作为首个融入认知AI能力的学习服务产品,背后离不开DMAI自动求解技术的支撑。

暗物智能在教育领域已形成学龄前、K12、在线教育、职业教育的用户服务全生命周期闭环,为题拍拍、腾讯作业君、豌豆思维等多家头部客户提供成熟的产品及解决方案。在美国推出的认知AI智能早教产品AILA,销量长期在亚马逊等主流平台中处于前列,好评率超过谷歌同类产品。在拓展国际市场的同时,通过打造AILA中国版本,积极推进内容生态本土化,已同国内多家知名幼教机构达成合作。

以AILA为代表的新兴早教智能产品,首次将认知人工智能技术引入家庭伴学场景。产品聚焦儿童认知潜能的开发,结合优质原创教学内容,为孩子提供真正个性化、自适应的陪伴学习体验,并联合喜马拉雅、声希AI课等合作伙伴共同打造智慧早教生态。

而DMAI打造的各类教育产品涵盖教研分析管理、陪伴自主学习、深度互动等多个关键核心环节,可以实现教学课堂全景时空解译,提供知识追踪诊断、交互练习、引导陪伴学习等强认知AI
能力,实现真正的因材施教。与单点功能的应用不同,DMAI依托具备认知和推理能力的新一代人工智能操作系统,打造构建知识和任务的内容生产平台,致力于“教、学、评、测、练”全流程的提升。

DMAI将与全通教育在人工智能教学以及人工智能公共服务等方面开展深入合作

DMAI
2019年的产品规划包括智慧课堂、知识内容生产系统、知心家教等等,到2020年,公司还将推出知心家教桌面学习终端、教育陪伴机器人解决方案等。

谙心助教率先实现的AI自动解题与讲题功能,不仅是破解“作业难”的“杀手锏”

谙心课堂融合计算机视觉、语音识别、自然语言理解等多模态AI技术,以“AI可视化教学全过程分析”为核心,对课堂氛围、教学交互、教学激励、学生发言等进行多维度分析,为线上教学场景提供全时空教学过程把控和教学质量分析。

智慧教学产品运用于华南师范大学附属南沙小学、实验小学,助力南沙打造“强认知人工智能+智慧教育”创新应用先导示范区

AILA Sit \&
Play这一融合了歌曲、游戏、声音教学法的变革型产品获得包括众多零售商在内的一致好评。一名创意产品销售商的负责人表示,很期待为所有来店的新父母们展示这种创意性的AI+教育产品。

暗物智能研发了具有高度自适应能力的桌面机器人获得“Mom’s Choice Gold
Award”等多项大奖


\subparagraph{新零售}
\label{\detokenize{chapter_AI_company/dm-ai:id10}}
暗物智能针对新零售领域,推出行业首个强交互智能导航机器人。此外,暗物智能还基于认知AI技术,赋能趣互联等多家智慧物联服务商,打造集合智能调度、精准推荐、智能导购等功能的新一代认知AI智慧零售解决方案。


\subparagraph{电子游戏智能NPC}
\label{\detokenize{chapter_AI_company/dm-ai:npc}}
利用认知AI技术平台在人机交互、可迁移性等方面的优势,暗物智能推出行业首个以认知AI驱动的电子游戏智能NPC引擎,为吉比特等知名游戏开发商打造会思考、强交互、自主协作的高质量游戏NPC大脑。


\subparagraph{公共服务 12\sphinxfootnotemark[1015]}
\label{\detokenize{chapter_AI_company/dm-ai:id11}}%
\begin{footnotetext}[1015]\sphinxAtStartFootnote
\sphinxnolinkurl{https://www.dm-ai.cn/news/\%e6\%89\%93\%e9\%80\%a0\%e4\%b8\%aa\%e6\%80\%a7\%e5\%8c\%96\%e4\%ba\%a4\%e4\%ba\%92\%e4\%bd\%93\%e9\%aa\%8c\%ef\%bc\%8c\%e5\%bc\%ba\%e8\%ae\%a4\%e7\%9f\%a5ai\%e5\%8a\%a9\%e5\%8a\%9b\%e7\%96\%ab\%e6\%83\%85\%e9\%98\%b2\%e6\%8e\%a7/}
%
\end{footnotetext}\ignorespaces 
文本深度语义匹配:以DMAI自主训练的强认知深度语义匹配模型为基础,可以准确理解不同句法结构下的语义信息,精准匹配问答答案,实现高效互动。

基于医疗知识图谱的推断问答:基于结构化知识构建的新冠肺炎医疗知识图谱是疫情助手的核心模块,也是它的“最强大脑”。如同医护人员的线下诊断一般,系统能够通过自然语言理解技术解析用户的问题,在医疗知识图谱中进行智能推理得出答案并回复用户。

意图猜测和引导:对于用户提到的系统掌握范围以外的问题,疫情通达助手也能结合语义理解分析结果,引导用户询问类似问题,并主动推荐用户可能关注的问题。

DMAI打造的AI防疫测温解决方案不仅从技术上提升检测效率,切实保障全校师生安全,也带来了更加便捷的交互体验。校园防疫涉及学校、教室、学生家庭等多方面的沟通,任何一方的信息缺失都可能导致“防线”出现漏洞。该方案与学生信息管理平台相连接,便于家长和老师实时查看学生体温数据,省心更放心。\sphinxhref{https://www.dm-ai.cn/uncategorized/\%e5\%a4\%a7ai\%e6\%97\%a0\%e7\%96\%86\%ef\%bc\%8cdmai\%e4\%b8\%ba\%e6\%8a\%97\%e7\%96\%ab\%e7\%8c\%ae\%e6\%99\%ba/}{15}%
\begin{footnote}[1016]\sphinxAtStartFootnote
\sphinxnolinkurl{https://www.dm-ai.cn/uncategorized/\%e5\%a4\%a7ai\%e6\%97\%a0\%e7\%96\%86\%ef\%bc\%8cdmai\%e4\%b8\%ba\%e6\%8a\%97\%e7\%96\%ab\%e7\%8c\%ae\%e6\%99\%ba/}
%
\end{footnote}


\paragraph{DM AutoML13\sphinxfootnotemark[1017]}
\label{\detokenize{chapter_AI_company/dm-ai:dm-automl13}}%
\begin{footnotetext}[1017]\sphinxAtStartFootnote
\sphinxnolinkurl{https://www.dm-ai.cn/news/\%e4\%b8\%8d\%e9\%9c\%80\%e8\%a6\%81ai\%e4\%b8\%93\%e5\%ae\%b6\%e4\%b9\%9f\%e8\%83\%bd\%e5\%ae\%9e\%e7\%8e\%b0\%e6\%99\%ba\%e8\%83\%bd\%e5\%8c\%96\%ef\%bc\%81dmai\%e6\%8e\%a8\%e5\%87\%ba\%e7\%a7\%81\%e6\%9c\%89\%e5\%8c\%96automl\%e4\%ba\%91\%e5\%b9\%b3/}
%
\end{footnotetext}\ignorespaces \begin{itemize}
\item {} 
硬件自适应能力:基于预设的资源约束(参数规模、计算复杂度、推理延时等),能够自适应调整网络结构,自动优化AI模型的效率与性能。

\item {} 
模型智能调参能力:DM
AutoML云平台的模型通过基于因果关系推理的超参数调优,摆脱人工训练深度模型不透明、难以溯源的“炼丹”过程,以更短的时间匹配最优的超参组合。

\item {} 
模型多终端一站式部署能力:实现了深度模型从开发到部署的全自动化流程,支持caffe2、tf\sphinxhyphen{}lite等端上部署。

\item {} 
私有化平台部署能力:与多数基于云服务的AutoML提供方不同,DM
AutoML云平台直接部署在客户私有服务器上,无需依赖第三方云服务,从源头上确保业务数据无外泄,让企业安全、安心地使用AI能力。

\end{itemize}

DM
AutoML云平台不仅提升了准确率,更节省了大量计算资源和人工调参成本。以分类任务场景为例,仅需一台8GPU服务器不间断运行一周,DM
AutoML云平台在零人工模型、无人工干预的状态下,输出的模型在ImageNet验证集上准确率高达77.3\%。而同等模型规模下,作为最前沿的神经网络结构搜索算法,谷歌的EfficientNetB0在投入数千TPU的计算资源,并经过相关领域AI专家的反复调参后,准确率仅能达到76.3\%。


\paragraph{金融领域}
\label{\detokenize{chapter_AI_company/dm-ai:id12}}
依托核心操作系统DMOS,发挥认知推理引擎准确度高、决策可解释的独特技术优势,构筑智能化金融生态闭环。同时,运用知识图谱、规则学习、逻辑推理等强认知AI技术,融合专家经验、监管规则,打造人机共治的智能风控系统。


\paragraph{AR}
\label{\detokenize{chapter_AI_company/dm-ai:ar}}
DMAI将通过多模态交互AI芯片、教育多模态分析技术、自适应学习系统、早教学习机等基于强认知AI的教育智能化产品与技术,与新节奏AR体感教育系统等产品融合共享,建设“AI+AR”双重赋能的教育服务体系。

新节奏是中国最早将动作识别技术和AR技术运用在儿童游戏化教学的公司,在AR和VR内容生成与互动方面拥有领先的数字化应用案例。2019年1月,新节奏成为教育部装备中心在人工智能儿童教育装备领域的唯一合作伙伴。


\paragraph{社招}
\label{\detokenize{chapter_AI_company/dm-ai:id13}}
\sphinxurl{https://sc.hotjob.cn/wt/DMAI/web/index/webPositionN310!getOnePosition?postId=111201\&recruitType=2\&brandCode=1\&importPost=0\&columnId=2}


\subsubsection{京东}
\label{\detokenize{chapter_AI_company/jd:id1}}\label{\detokenize{chapter_AI_company/jd::doc}}

\paragraph{京东物流}
\label{\detokenize{chapter_AI_company/jd:id2}}
因为刘强东不惜重金打造的自建仓储物流,现在周围的人都在说京东今天买的,明天就能到。如果说马云爸爸开启了网购时代,刘强东则是优化了用户体验。也许正是刘强东一直坚信用户体验决定一切,才有了京东的今天。\sphinxhref{http://www.woshipm.com/zhichang/807191.html}{6}%
\begin{footnote}[1018]\sphinxAtStartFootnote
\sphinxnolinkurl{http://www.woshipm.com/zhichang/807191.html}
%
\end{footnote}


\paragraph{京东商城}
\label{\detokenize{chapter_AI_company/jd:id3}}
在京东商城之前,我们所知的主要商城就只是国美和苏宁,而京东商城推出之后,我们购买3C电子产品可能主要都只会在京东上面购买了。似乎,一夜之间,国美苏宁的线下门店都变成了体验店,我们会去线下摸摸手感确定型号,然后跑到线上的京东商城去下单。

为什么?因为京东的价格比其他零售商都便宜。因此,2013年起,苏宁开始大量关掉价值不大的一些门店,另外也开始了线上线下价格一体化的策略,但是它的利润率也开始下跌。然后,传统零售行业的头部企业开始被互联网巨头陆续收购或投资,“新零售”的概念被炒作了起来。关于“新零售”,后续我会专门聊聊,这里就不再赘述。


\paragraph{京东与支付宝}
\label{\detokenize{chapter_AI_company/jd:id4}}
与阿里更早确立竞争关系的京东,早在2011年就取消了支付宝支付。当时刘强东也表示,支付宝手续费太高。
\sphinxhref{https://www.linkedin.com/news/story/\%E7\%BE\%8E\%E5\%9B\%A2\%E4\%B8\%8E\%E6\%94\%AF\%E4\%BB\%98\%E5\%AE\%9D\%E5\%88\%86\%E9\%81\%93\%E6\%89\%AC\%E9\%95\%B3-4900980/?originalSubdomain=cn}{2}%
\begin{footnote}[1019]\sphinxAtStartFootnote
\sphinxnolinkurl{https://www.linkedin.com/news/story/\%E7\%BE\%8E\%E5\%9B\%A2\%E4\%B8\%8E\%E6\%94\%AF\%E4\%BB\%98\%E5\%AE\%9D\%E5\%88\%86\%E9\%81\%93\%E6\%89\%AC\%E9\%95\%B3-4900980/?originalSubdomain=cn}
%
\end{footnote}

在当时,这个理由其实是不成立的:

尽管70\%的用户在京东购物是选择货到付款,但采用货到付款方式成本要更高。

目前京东采用的是交通银行的移动POS收款,其费率是千分之三,但由于京东采用的自建物流与收款人员,每笔收款的成本要再加上3\sphinxhyphen{}5元(包括人力成本及其他费用),这还不包括系统建设费用。如果按照最高估计即京东的客单价1000元来计算,货到付款的单笔成本至少为千分之六到八,几乎是使用支付宝的费用的一倍。\sphinxhref{http://tech.sina.com.cn/i/2011-08-25/23375981397.shtml}{4}%
\begin{footnote}[1020]\sphinxAtStartFootnote
\sphinxnolinkurl{http://tech.sina.com.cn/i/2011-08-25/23375981397.shtml}
%
\end{footnote}

那既然使用支付宝还能“省钱”,为什么京东还是依然决然的放弃支付宝呢?
\begin{itemize}
\item {} 
数据层面:商业数据,尤其是财务数据,是一家公司最为核心的数据,谁都不愿意有很大一部分数据暴露给别人,比如品类数据,支付率,客单价,退款率等重点等交易数据,再加上这个别人还是个竞争对手。

\item {} 
战略层面:为了打造自己的金融体系,京东拒绝支付宝后,京东钱包,京东白条这些业务也冒了出来。\sphinxhref{https://www.zhihu.com/question/410767563/answer/1373298846}{5}%
\begin{footnote}[1021]\sphinxAtStartFootnote
\sphinxnolinkurl{https://www.zhihu.com/question/410767563/answer/1373298846}
%
\end{footnote}

\end{itemize}

而近日恒大研究院在《中国独角兽报告:2020》中指出,蚂蚁金服以1500亿美元估值位居全球第一。京东数科估值也已过千亿。\sphinxhref{https://finance.sina.com.cn/money/bank/bank\_hydt/2020-06-16/doc-iircuyvi8701006.shtml}{3}%
\begin{footnote}[1022]\sphinxAtStartFootnote
\sphinxnolinkurl{https://finance.sina.com.cn/money/bank/bank\_hydt/2020-06-16/doc-iircuyvi8701006.shtml}
%
\end{footnote}


\paragraph{京东金融}
\label{\detokenize{chapter_AI_company/jd:id5}}
\sphinxurl{http://jr.jd.com/}


\paragraph{京东科技}
\label{\detokenize{chapter_AI_company/jd:id6}}
京东集团已整合原京东数字科技、原京东云与AI事业部为京东科技子集团。即日起,原京东数科、京东智联云品牌不再适用,统一品牌为京东科技,京东科技现已成为整个京东集团对外提供技术服务的核心输出平台。

产品简介

机器学习平台AI基于海量的数据及强劲的计算资源,搭载Sklearn、XGBoost等主流机器学习框架;支持Python、PySpark等多种语言,提供从模型开发到部署的一站式服务;高效的资源利用率,内核级的虚拟化,秒级启停;容器对资源需求少,单台物理机可以同时运行数千个容器;利用k8s对docker进行编排调度,实现对服务器资源管理、调度、动态扩/缩容。

\sphinxurl{https://www.jddglobal.com/products/machine-learn}


\paragraph{人工智能事业部}
\label{\detokenize{chapter_AI_company/jd:id7}}
2018年财富世界500强,京东位列181位,互联网公司排名第三。2014年5月,京东集团在美国纳斯达克证券交易所正式挂牌上市,是中国第一个成功赴美上市的大型综合型电商平台。“人工智能引领美好生活”,京东人工智能持续探索前沿科技,聚焦商业应用落地,加速释放科技的商业价值。在语音与声学、计算机视觉、机器学习、知识图谱、语义、对话6个技术领域不断深耕,拓展AI技术的边界,并通过京东人工智能开放平台(\sphinxurl{http://neuhub.jd.com})全面开放给行业合作伙伴,共赢生态。在市政、零售、客服、医疗等领域持续打造规模化应用落地,京东人工智能持续探索AI商业创新模式和价值,与实体经济相融合,释放AI真正的产业和社会价值。

\begin{figure}[H]
\centering
\capstart

\noindent\sphinxincludegraphics{{NeuFoundry}.png}
\caption{NeuFoundry平台架构图}\label{\detokenize{chapter_AI_company/jd:id11}}\end{figure}

NeuFoundry基础设施层采用Docker容器进行算力资源的池化,通过Kubernetes进行整体的资源管理、资源分配、任务运行、状态监控等,平台集成了MySQL、Redis、MQ等多种中间件服务,通过数据标注、模型训练、模型发布,生成自定义的AI能力,为各行各业的业务服务提供有力的支撑。\sphinxhref{http://www.woshipm.com/ai/3320134.html}{7}%
\begin{footnote}[1023]\sphinxAtStartFootnote
\sphinxnolinkurl{http://www.woshipm.com/ai/3320134.html}
%
\end{footnote}


\paragraph{京东数科}
\label{\detokenize{chapter_AI_company/jd:id8}}
1月11日,京东集团就宣布将云与AI业务与京东数科整合后,正式成立京东科技子集团,原京东数科CEO李娅云出任京东科技子集团CEO。

对于这个整合的意义,京东在公告中表示,通过此次交易,京东将继续专注于其核心竞争力和协同业务,京东数科将更有利于为业务合作伙伴提供一系列尖端技术服务。

2017年\sphinxhyphen{}2020年上半年,京东数科从白条产品获取的收入分别为14.73亿元、27.34亿元、32.10亿元和17.94亿元。

京东金条则是一款数字化无抵押的短期消费信贷产品。2017年\sphinxhyphen{}2020年上半年,京东金条的贷款规模分别为1036.85亿元、2554.92亿元、4589.15亿元和2612.17亿元,近三年复合增长率为110.38%。

除此之外,京东数科还有小微信贷产品包括京保贝、京小贷、京采等,但未透露具体业务规模。如果加上帮助金融机构发放信用卡和保险业务,围绕金融相关的业务已经超过总营收的50%。\sphinxhref{https://posts.careerengine.us/p/6071aa582b8d107d7964c3cc?from=latest-posts-panel\&type=title}{9}%
\begin{footnote}[1024]\sphinxAtStartFootnote
\sphinxnolinkurl{https://posts.careerengine.us/p/6071aa582b8d107d7964c3cc?from=latest-posts-panel\&type=title}
%
\end{footnote}


\paragraph{近况}
\label{\detokenize{chapter_AI_company/jd:id9}}
2021年1月,京东金融平台宣布已经下架了所有银行存款产品,2020年12月,支付宝、京东金融、滴滴金融等互联网金融平台先后下架对新用户的互联网存款产品。2021年1月,应监管要求,老客户也无法购买互联网平台的存款产品了,产品做了下架处理。\sphinxhref{https://www.baike.com/wikiid/5026927798799303826?prd=mobile\&view\_id=whwnnpeel3400}{8}%
\begin{footnote}[1025]\sphinxAtStartFootnote
\sphinxnolinkurl{https://www.baike.com/wikiid/5026927798799303826?prd=mobile\&view\_id=whwnnpeel3400}
%
\end{footnote}


\paragraph{投资}
\label{\detokenize{chapter_AI_company/jd:id10}}
京东投资的公司:\sphinxurl{https://www.itjuzi.com/jingdong}


\subsubsection{美团}
\label{\detokenize{chapter_AI_company/meituan:id1}}\label{\detokenize{chapter_AI_company/meituan::doc}}

\paragraph{定位}
\label{\detokenize{chapter_AI_company/meituan:id2}}
最好的服务业电商。口号:「吃喝玩乐,尽在美团」

价值观的排序是:“消费者第一,商家第二,员工第三,股东第四,王兴第五”。

这个事情注定是一个高品质、低价格、低毛利的事情,我们需要在各个环节通过每个人的努力,通过整个结构的调整,通过管理的提升,通过产品技术的革新,不断地去提高效率、降低成本,给消费者提供更高品质、更低价格的服务。

人是美团最重要的产品,也是美团最大的资产。因为我们并不拥有厂房、仓储物流中心或大规模的固定资产,我们所有的资产就是人。

靠两条腿跑出来的市场,通常是最牢靠的。“细分深挖”能够释放的力量超乎想象。钱要用在刀刃上。面向商家的品牌广告是无效的,在商家端再多的广告投放都不如有执行力的线下队伍。而面向消费者端,线上广告性价比远大于线下广告。

那是一个未知的领域,我们有很多前辈的经验和教训,我们应该充分利用这些经验教训,达到我们的目标,也去开拓一片没有人开拓过的领域,那就是本地电子商务。

我们相信商品的电子商务和服务的电子商务最终规模是差不多的,因为都非常非常大,大家每个人,全国那么多人花的钱,一部分花在实体商品上,一部分花在服务上,它的最后规模都是上万亿。现在我们只做了14.5亿,所以大概是千分之一,这还是非常非常初级的,前面还有无尽的空间等待我们去开拓,我们同样面临一个目标,要去做本地电子商务。

我想大的目标大家也知道,就是我们专注于本地电子商务,利用互联网去帮本地的商业或者帮本地的商家上网,让消费者能够放心地吃喝玩乐。这是一个很容易理解的事情,但是很大,要去细分很多。

美团市值稳坐万亿,从早期的团购网站已经发展成一个集外卖、酒店、旅游、出行、新零售一体的庞然大物,再造一个商城业务,十分顺其自然。\sphinxhref{https://www.zhihu.com/question/398269869}{13}%
\begin{footnote}[1026]\sphinxAtStartFootnote
\sphinxnolinkurl{https://www.zhihu.com/question/398269869}
%
\end{footnote}


\paragraph{竞品}
\label{\detokenize{chapter_AI_company/meituan:id3}}
现在美团渐渐开始做电商了,全国几百万的配送骑手,如果买衣服,买鞋子等可以像外卖一样短时间送货上门,那就不是单单饿了么的事了,等于是动摇了阿里的根基。\sphinxhref{https://www.zhihu.com/question/410767563/answer/1373080287}{19}%
\begin{footnote}[1027]\sphinxAtStartFootnote
\sphinxnolinkurl{https://www.zhihu.com/question/410767563/answer/1373080287}
%
\end{footnote}

今年3月,蚂蚁集团CEO胡晓明宣布支付宝升级为数字生活开放平台,这和美团的定位“本地服务电商平台”有着一定程度上的正面冲突。\sphinxhref{https://www.zhihu.com/question/410767563/answer/1374193976}{20}%
\begin{footnote}[1028]\sphinxAtStartFootnote
\sphinxnolinkurl{https://www.zhihu.com/question/410767563/answer/1374193976}
%
\end{footnote}

业务碰撞却屡屡出现:
\begin{itemize}
\item {} 
2015年6月阿里与蚂蚁集团联合出资成立了一家本地生活服务平台公司
——“口碑”,这直接与美团业务产生冲突;

\item {} 
10月,美团和大众点评合并,美团点评在2016年1月中旬向媒体透露称,已完成33亿美元融资,其中腾讯出资了近10亿美元;

\item {} 
2016年1月底,阿里执行副主席蔡崇信表示,将增加对口碑的资源投入,“退出美团也就是时间上的问题”;

\item {} 
2018年,阿里联合蚂蚁集团斥资95亿美元全资收购饿了么,后者不仅是美团在外面领域的最大对手,还在不断和美团进行新的竞争;

\item {} 
7月份,饿了么宣布升级,从送外卖到送万物、送服务,饿了么CEO王磊表示,“送万物”的内涵正在被不断拓展,商品和服务都可以外卖到家。这与美团2018年7月上线的美团闪购业务有一定重合,采用快零售的业务模式,为用户搭建一个30分钟到货的生活卖场,该品牌即是整合了除餐饮之外的品类。

\end{itemize}

两者在生鲜电商、酒旅、出行、金融等多个行业,多面碰撞:\sphinxhref{https://www.zhihu.com/question/410767563/answer/1374193976}{20}%
\begin{footnote}[1029]\sphinxAtStartFootnote
\sphinxnolinkurl{https://www.zhihu.com/question/410767563/answer/1374193976}
%
\end{footnote}
\begin{itemize}
\item {} 
阿里大力发展盒马鲜生,美团推出小象生鲜、上线买菜业务;

\item {} 
阿里用飞猪布局旅游业,而美团也在发展到店、酒店及旅游业务;

\item {} 
美团点评此前收购摩拜单车,阿里重金投资哈啰出行,目前,共享单车市场,滴滴青桔单车、美团单车、哈啰出行三足鼎立;

\item {} 
网约车市场,美团打车与高德打车均为聚合模式;

\item {} 
今年5月份,美团信用支付产品“美团月付”正式上线,而在这背后,美团旗下已有“银行+第三方支付+小贷+保险”四块牌照,且已向用户提供小额贷款、信用支付等服务。

\end{itemize}


\paragraph{历史}
\label{\detokenize{chapter_AI_company/meituan:id4}}
美团网又帮KTV设计生日套餐,吸引过生日的人呼朋唤友过来,带来人流,也有很多消费者感到满意,进行再次消费。他们也抓紧对电影院的进攻,而当时拉手网忽视了这一块。为什么呢?因为拉手网花了很多钱做推广,对毛利的要求高,而做电影院毛利低,一张电影票带来的毛利也就5角、1元的。

在电影院、KTV、烘焙店这三个领域做起来之后,美团网在东莞打开了局面,2011年3月,美团网东莞站反超拉手网,之后也没有被拉手网夺回优势。


\paragraph{财报}
\label{\detokenize{chapter_AI_company/meituan:id5}}
根据财报来看,2020年第四季度美团新业务的亏损也从去年同期的11亿元扩大至60亿元,直接导致四季度净亏损22.4亿元。而其新业务之一便是当下最火的社区团购,自美团第三季度大力投入社区团购后,其烧钱速度就不断在加快。\sphinxhref{https://www.weiyangx.com/382825.html}{27}%
\begin{footnote}[1030]\sphinxAtStartFootnote
\sphinxnolinkurl{https://www.weiyangx.com/382825.html}
%
\end{footnote}


\paragraph{夏华夏报告}
\label{\detokenize{chapter_AI_company/meituan:id6}}
尊敬的吴院士,各位嘉宾,各位现场的朋友和云端的朋友们,大家中午好。今天我为大家带来美团AI未来城市生活的基础设施。我是在美团负责\sphinxstylestrong{人工智能平台服务},我们坚信人工智能不应该仅仅是在电脑里的数据或者是在云端的一些算法,它更应该跟我们现实生活中的每一个人产生一些真实的温暖连接,帮助我们在真实的物理世界变的更加美好,这是为什么我们讲未来城市生活的原因。
\sphinxhref{https://www.yicai.com/news/100695161.html}{1}%
\begin{footnote}[1031]\sphinxAtStartFootnote
\sphinxnolinkurl{https://www.yicai.com/news/100695161.html}
%
\end{footnote}

我引用德勤在报告当中的一句话,它说城市其实是人工智能技术创新的一个综合性载体,也是人类与AI技术产生全面感知的一个集中体验地。如果做智慧城市的话,对一个生活服务的闭环是有特别高的要求,因为用户在真实的物理生活中要\sphinxstylestrong{吃、穿、住、行,包括购物、旅游、娱乐},整个服务如果形成闭环,那整个的数据,用户生活的很多数据就会构成一个全面的示图,帮助我们更好理解这个用户,更好为用户服务。幸运的是,美团点评刚好有这个闭环的服务,我们\sphinxstylestrong{把4.5亿的消费者和600多万商家集中在一个超级平台上},一边为用户提供吃、穿、住、行、乐的各种服务,一边为商家提供很多支撑他们发展的经营、决策的服务。在这个闭环形成之后,我们认为我们可以用人工智能技术去帮助我们更好的做线下经济的数字化。

我们做人工智能技术的优势是什么呢?因为我们有特别丰富的场景,刚才说我们有大概200多种不同的服务,cover线下生活的各种方面,有这么多场景之后会产生大量的数据,我们用这些大量数据反过来可以帮助我们人工智能算法不断迭代、优化。这些算法再闭环过来帮助我们用户在场景中更好得到一些服务。所以美团做人工智能是由场景来驱动我们不断迭代的。

这也是我们第一次对外公开发布我们一个全景美团人工智能的科技图谱。因为美团刚才说我们有特别多的服务场景,有大量的商家,大量的用户,会产生很多各种各样的数据,基于这些数据我们会结合很多种人工智能的技术,包括图像、语音、智能交换自然语言处理、知识图谱包括运动决策规划等等,这些技术结合在一起可以有很多应用,比如说我们做生物安全认证、智能客服、无人配送等等,这些在几个方向,包括智慧交互、物流网络、生活服务大脑和产业的自动化升级方面都可以帮助我们更好的做这个智慧城市。

\begin{figure}[H]
\centering
\capstart

\noindent\sphinxincludegraphics{{meituan}.jpg}
\caption{美团人工智能的科技图谱}\label{\detokenize{chapter_AI_company/meituan:id23}}\end{figure}

我们的目标是希望让美团的人工智能成为我们未来生活的一种基础设施,今天着重从四个方面给大家分享一下我们正在做的一些工作,分别是比如说我们怎么给用户做比较好的服务引擎,怎么做产业的升级,社会的治理,以及更远的未来生活的一些可能的产品。先看服务引擎,因为美团的app里面有两百多种不同服务,但是大家打开手机就是一个小小的6寸屏,我们怎么让用户更好的更便捷在6寸屏幕上找到用户所需要的服务,我们背后就有很多人工智能的技术帮助用户去搜索推荐,以及用一些更便捷的方式去找到。

\begin{figure}[H]
\centering
\capstart

\noindent\sphinxincludegraphics{{meituan_do}.jpg}
\caption{美团到底要做什么}\label{\detokenize{chapter_AI_company/meituan:id24}}\end{figure}

这里给大家一个例子,可以用智能语音交互比较方便的用美团的服务。因为美团有这种非常闭合的,从服务的展示到支付,到最后履约的服务,我们可以为很多,哪怕不会用手机,不会输入文字,比如说一些老人、小孩也可以非常方便智能交互的方式使用美团的服务。当然在服务的背后是一系列我们美团丰富的业态结合起来,并且我们把它兼容到各种不同的硬件载体上,包括手机、车载设备、家具智能音响等等,再结合我们的履约、支付等等一些功能,综合起来可以给用户提供一个非常好的服务体验。

\begin{figure}[H]
\centering
\capstart

\noindent\sphinxincludegraphics{{meituan_interact}.jpg}
\caption{美团智能交互}\label{\detokenize{chapter_AI_company/meituan:id25}}\end{figure}

第二个方面,我们给商家也希望给他们非常好的提供服务,因为商户在美团看来也是我们的客户,我们希望用人工智能的技术去帮助商家提高效率,做整个产业的效率升级。怎么升级呢?当我们去深入了解每一个产业的时候,我们发现真正在线下做一个商铺,哪怕只是一个简单的餐馆,它其实涉及的经营决策,涉及的行业知识是非常多的。比如说一开始要选址,在什么地方选,建店之后怎么比较好的经营管理、营销,包括做物流等等。这些专业的知识我们希望用美团AI的各种技术,结合我们大数据,去给商家提供很多各种不同的服务。这里面包括比如说给商家智能的收银系统,它的智慧选址,告诉它周围的商圈都是一种什么样的消费价位,已经有什么样的商家,不同品类的竞争情况是怎样的,让商家可以更好的做经营的决策。

这里面时间关系没有办法一一展开,给大家看一个例子。这是美团大脑,本质上是基于美团海量的数据,经过大数据挖掘之后形成的巨大知识图谱。在这个知识图谱里面,大家看到有很多点,这些实体有可能是商家,有可能是一些关键标签,我们把它连接起来之后,可以形成一个巨大的海量知识。我们现在有千亿量级的知识连接,在这个例子里面展示的是浦江宴,就是会场旁边的一家店它的一些特点,我们可以看到用户对这家店的情感变化曲线,可以看到这家店的口味,它的用户评价,包括周围商家竞争的情况,它的相似商家是什么。通过这样一些数据,可以让商家在经营决策的时候,可以得到很多的反馈,可以帮助商家做更好的经营决策。

刚才讲的是给商家做赋能,我们同时也希望用人工智能的技术去帮助我们整个社会的治理,让整个社会运转的更加高效。我举两个例子,一个是因为今年疫情期间,我们发现很多商户会受疫情的影响,包括我们就业会受到疫情的很多影响。这一页我们会给大家展现我们怎么使用我们智能语音交互的方式,帮助我们去做疫情的通知,做商户的信息收集,做招聘。我们先听第一个例子。

大家听到这个女声,听起来是非常真实的声音,但其实是我们完全用人工智能,是我们的客服机器人合成的。在这个过程中有几个技术挑战,第一个是我要非常准确理解用户说了什么,包括他的背景噪音、方言都要适应。第二,基于用户说的内容,基于我背后的知识库产生我要说什么。第三,我要用尽量接近真实人的声音去把这个声音说出来。因为很多用户一接到电话一听到是机器人的声音,马上挂掉电话,我们还要尽量逼近人的声音。

大家会听到,当我们要收集很多商户信息的时候,因为每个商户根据它的品类不同,我们要收集指定的很多字段的信息,这个其实对于人来打电话是非常繁杂的,现在我们有机器人每天打一百万通的电话,为了收集各种信息,让网上商户的信息更加准确。

\begin{figure}[H]
\centering
\capstart

\noindent\sphinxincludegraphics{{meituan_hu}.jpg}
\caption{智能传呼}\label{\detokenize{chapter_AI_company/meituan:id26}}\end{figure}

第三个例子是我们去招募一些骑手,尤其在疫情期间对就业是非常大的帮助。我们的机器人会根据用户的很多反馈,尤其是当听不清的时候,会再反问确认,这样一种非常接近真人的客服方式,去提供给用户、商户、骑手很多服务。我们也希望未来,比如说明年人工智能大会的时候,这种智能机器人的技术可以帮助我们比如说去通知参会嘉宾,可以有很多的对接。

\begin{figure}[H]
\centering
\capstart

\noindent\sphinxincludegraphics{{meituan_zheng}.jpg}
\caption{美团政务服务}\label{\detokenize{chapter_AI_company/meituan:id27}}\end{figure}

第二个社会治理的例子,我们也会跟各地政府去合作,因为政府其实对本地生活的很多产业是希望有更精确的了解。比如说每个参与商家卫生情况怎么样,之前食药监、卫生局他们其实并没有管理上的抓手,但是跟美团合作之后,我们会给各地政府提供类似这样的一个政府政务数据大屏,大家可以看到很多疫情期间的复工率是怎样的,各地每个不同的产业现在的发展是什么样子,它各个地区不同的产业热点是什么,包括看某些比如餐饮商家,我们会看到有问题的商家的top10,卫生问题或者是客诉问题的商家是什么,这样就可以给政府治理有更好的抓手。

我们看一下未来生活,我们希望未来还会有更多产品帮助大家体会更好的生活。这是一个示意图,我们相信在未来我们在物理生活的很多设施都会是一个联网的,是一个万物互联。这里面可能有一些是比如说仓库是无人仓,我们通过无人机或者无人车的方式,把无人仓里的货物更多送到用户手上。这已经不是未来,我们已经做了很长时间的技术研发准备,已经在一些地方小规模测试。给大家看三个不同的视频,第一个是无人微仓,当商户想开一个零售店的时候,我们可以用无人微仓的技术把很多货架,很多的智能货架,包括仓内的机器人,用户下单之后,这个订单信息会发到无人微仓,无人微仓会自动做分拣、打包,最后把打包的包裹交给骑手,或者交给我们的无人车和无人机,最后完成配送。

除了让骑手交付之外,我们希望未来可以让美团的无人车来配送,这段视频是展示我们在北京市顺义区,在今年疫情期间部署的几辆无人车,可以把用户在美团买菜下的单用无人车从我们的仓库送到用户所在的小区楼下,用户可以直接下来取。装车之后,车辆会根据自己规划出来的路线,在路上自动行驶,可以识别红绿灯,可以躲避车辆,可以躲避行人。最后到达目的地,用户取餐之后,无人车会自动开回我们的仓库。

我们现在除了无人车也在研发更快速的配送的无人机,因为无人机不受限于交通路上拥堵的情况,可以非常快的在十分钟的量级把餐送到用户手上。这个例子是我们在一些地方测试的无人机的真实视频。当我们骑手取餐之后,可以把餐以某种形式交给我们的无人机,这个视频里面是手工交过去,未来会有更智能的设施。无人机到达目的地之后,不管是货柜或者是地面、楼上,会有一个用二维码标出来的目标,我们的无人机会基于视觉技术自动寻找这个目标,最后把外卖的包裹释放,然后无人机再返回。

这是今天给大家分享的美团在做的人工智能方向的一些工作。总结一下就是美团有非常丰富的场景和数据,而这些场景和数据在驱动着我们人工智能的发展,今天给大家展示了,包括给用户的生活服务的引擎,给商家这边产业的升级,给政府和社会治理方面的一些工作,以及未来的各种智能无人设备的情况。我们相信当这些东西都做完之后,AI会成为我们美团做线下数字化的重要工具。美团AI会是未来生活的基础设施,它无处不在,也不需要用户感知到,就像基础设施那样,会给我们的日常生活的点点滴滴提供很多帮助。也欢迎大家关注美团AI官网,欢迎不同产业合作伙伴,学术界的合作伙伴跟我们联系,大家可以一起把AI做的更好,谢谢大家。


\paragraph{无人机送外卖的分析}
\label{\detokenize{chapter_AI_company/meituan:id7}}
\sphinxhref{https://www.zhihu.com/question/30170250}{2}%
\begin{footnote}[1032]\sphinxAtStartFootnote
\sphinxnolinkurl{https://www.zhihu.com/question/30170250}
%
\end{footnote}

目前在北京公开行驶的商用级自动驾驶电动车,只有2家。一家是百度开发的阿波龙,它行驶在海淀公园的封闭道路上,路况相对简单;另外一家就是美团,美团这个无人配送看起来憨憨的,一点也不科幻,但它已经能拉活送货了。自动驾驶在能跑起来和实现规模化应用之间,还差着好几个量级呢。

一,骑手尚且不够呢,无人车上路了也不够,无人车取代最危险、最辛苦的中间道路的配送,两端的100米还需要小哥跟无人车协同;第二,未来无人车本身就会创造更多新的岗位提供给骑手,比如车辆的维修、保养、充电、后台监控等。

例如车辆的维修、保养、充电、后台监控等。

美团把无人配送定位成补充运力,未来将跟骑手形成协同,而增加了无人车的变量,配送调度难度会更大,技术进步真是一个无限螺旋啊。\sphinxhref{https://www.jianshu.com/p/d24b4851a2a4}{3}%
\begin{footnote}[1033]\sphinxAtStartFootnote
\sphinxnolinkurl{https://www.jianshu.com/p/d24b4851a2a4}
%
\end{footnote}


\paragraph{VS 饿了吗}
\label{\detokenize{chapter_AI_company/meituan:vs}}
\sphinxurl{https://search.bilibili.com/all?keyword=\%E7\%BE\%8E\%E5\%9B\%A2vs\%E9\%A5\%BF\%E4\%BA\%86\%E5\%90\%97\&from\_source=nav\_search\_new}


\paragraph{AI}
\label{\detokenize{chapter_AI_company/meituan:ai}}
\sphinxurl{https://tech.meituan.com/tags/\%E6\%B7\%B1\%E5\%BA\%A6\%E5\%AD\%A6\%E4\%B9\%A0.html}

GAN:\sphinxurl{https://tech.meituan.com/2018/12/27/ai-in-banner-design.html}

NLP:
如果你打开美团APP,就能在搜索框边上看到一个话筒按钮,通过“动动嘴”的方式,可以方便的调用美团平台的所有生活服务,美团现在已经打通了包括外卖、餐饮、酒店、旅行在内的200多个生活服务场景。这些场景也给AI技术的落地提供了基础。美团语音就能一站式满足所有生活服务需求,因为语音交互和服务都是美团的,体验可以衔接的更紧密。\sphinxhref{https://www.jianshu.com/p/d24b4851a2a4}{3}%
\begin{footnote}[1034]\sphinxAtStartFootnote
\sphinxnolinkurl{https://www.jianshu.com/p/d24b4851a2a4}
%
\end{footnote}


\paragraph{盈利模式}
\label{\detokenize{chapter_AI_company/meituan:id8}}
比如美团外卖收取商家20\%—25\%的菜品价格提成,也就是说一份19.9元的外卖,商家要给美团4块钱,还要扣商家每一份2元的快递费,到商家手里不到14块。除去成本赚不到3元,商家累死累活还没有美团抽成赚的多。\sphinxhref{https://www.zhihu.com/question/28578492/answer/130821555}{22}%
\begin{footnote}[1035]\sphinxAtStartFootnote
\sphinxnolinkurl{https://www.zhihu.com/question/28578492/answer/130821555}
%
\end{footnote}
\begin{itemize}
\item {} 
促成交易的佣金。至于比例与城市,行业,商家、竞对状况有关。简单的是占比,复杂一点可以阶梯分账,甚至有参股之类。

\item {} 
平台增值服务。美团网内部的广告如banner、推荐位、专题;用户画像、竞争情报等等也是商家愿意花钱购买的。当然这种方式,淘宝玩的更6。用户方面,付费会员特权一直想搞,其余什么抽奖单啥的也是稳赚不赔

\item {} 
自营。有了海量用户、商家、消费记录。这部分数据很大的作用。恩,北京莞式服务很受欢迎,特别是望京。XX的技师最受欢迎。波推项目卖的最好。so,你懂。自己开,还是投资都可以哇。猫眼是最好的例子,最近猫眼影业投资发行的《驴得水》《情圣》都不错,春节大闹天竺要是能超西游,也就算是能够真正立足了。

\item {} 
资金池。佣金一般和商家是月结,月流水有N亿,这笔钱虽然法律上属于代收款不能随意用,咱是守法公民对吧。但是金融咱也是可以做的嘛。搞点投资,借贷啥的支援困难群众也是可以的嘛。

\item {} 
另外,用户商家数据啊,各种信息也是一笔宝贵的财富。当然,我相信咱新美大这么逼格的公司,肯定不会做啥缺德事儿的。\sphinxhref{https://www.zhihu.com/question/28578492/answer/142246934}{23}%
\begin{footnote}[1036]\sphinxAtStartFootnote
\sphinxnolinkurl{https://www.zhihu.com/question/28578492/answer/142246934}
%
\end{footnote}

\end{itemize}


\paragraph{海外业务?}
\label{\detokenize{chapter_AI_company/meituan:id9}}
美团网的B2B餐饮采购平台快驴订货已覆盖21个省,38个城市。据了解,今年4、5月间,美团快驴在上海大肆地推,通过粮油标品减价、烧钱补贴等手段,到6月底,累计交易额破亿,目前月销售额已超过4亿。

 目前这一领域由于中餐的sku种类繁多、餐饮店分布分散、食材尤其是蔬菜损耗率高、对物流要求高且涉及冷链物流等因素,准入门槛高,而且目前绝大多数玩家都处于不赚钱的状态。


\paragraph{美团 VS 阿里}
\label{\detokenize{chapter_AI_company/meituan:id10}}
过去美团和饿了么之战,是饿了么的存亡之战,今天美团和阿里在生活服务领域全面开战,是美团的存亡之战;

新竞争者出现,比如在今年最热的买菜市场,美团面临盒马、叮咚买菜等竞争,在快驴进货业务上,面临与新兴独角兽美菜的正面竞争。


\paragraph{美团研究院}
\label{\detokenize{chapter_AI_company/meituan:id11}}
\sphinxurl{https://about.meituan.com/research/home}


\paragraph{美团金服}
\label{\detokenize{chapter_AI_company/meituan:id12}}
美团副总裁彭千在论坛现场表示,2017年是金融科技上下半场重要分水岭,金融科技的下半场比互联网的所有下半场更加猛烈,涉及到资金、资本、杠杆、风险、安全等方面。下半场已经进入金融本质深水区,\sphinxstylestrong{传统金融机构和新兴金融科技公司会从顶层制度设计以及人才引进等各个方面深度融合},
双方合作一定是多向且双赢的。

“美团作为后加入者,跑步进入金融科技下半场,通过拓展交易场景、构建多元生态和创新数字技术,构建场景丰富、主体多样、技术创新的‘场景+金融’生态体系,更好的服务实体经济、促进金融普惠发展,‘帮大家吃得更好,生活更好’。”彭千说。

在金融业上,美团布局了支付与借贷这两方面业务,支付包括月付、联名信用卡等业务,借贷包括生意贷、美团生活费\sphinxhyphen{}借钱为主的小额借贷业务。

这两方面又都覆盖了B端与C端用户,在“场景+金融”生态搭建顺利下,就会形成由对消费者、商家服务扩展至对二者供应链上的闭环,以此源源不断地反哺平台自身。

五年前,王兴就已经流露出了对于金融的野心,表示“美团有信心打造一个千亿资产规模的金融事业”。

美团支付时发现,支付宝支付的选项被取消,美团月付和银行卡支付占据优先位置,而微信支付和Apple
pay还在支付选择列表上。\sphinxhref{https://www.zhihu.com/question/398269869}{13}%
\begin{footnote}[1037]\sphinxAtStartFootnote
\sphinxnolinkurl{https://www.zhihu.com/question/398269869}
%
\end{footnote}


\subparagraph{美团月付}
\label{\detokenize{chapter_AI_company/meituan:id13}}
“月付服务,这月买、下月付、笔笔减。这也是继蚂蚁花呗、京东白条、微信分付之后,又一个加入信用支付市场的巨头。\sphinxhref{http://finance.eastmoney.com/a/202006041510423790.htmls}{7}%
\begin{footnote}[1038]\sphinxAtStartFootnote
\sphinxnolinkurl{http://finance.eastmoney.com/a/202006041510423790.htmls}
%
\end{footnote}

美团月付是美团今年刚推出的类似“花呗”的信用支付产品,最长享38天免息期,最长可分12期还款,支持延期和分期还款。

此前新流财经曾分析过,越来越多的互联网巨头以及互金机构都在上线信用支付产品,解决平台流量增长之困,并能使得金融机构、场景平台、用户三者共赢。

据美团官方披露,美团月付可有效提升美团的支付订单转化率,试运营期间,月付用户的美团订单量平均提升超20\%,交易金额平均提升超15\%;用户对月付的使用意愿也在不断增强。

追踪数据显示,在下沉市场,三四五线城市用户的月付开通意愿和还款表现甚至比一二线城市用户还要好。\sphinxhref{https://www.zhihu.com/question/398269869}{13}%
\begin{footnote}[1039]\sphinxAtStartFootnote
\sphinxnolinkurl{https://www.zhihu.com/question/398269869}
%
\end{footnote}

2016年11月份,美团获得小贷牌照,即成立重庆美团三快小额贷款有限公司,美团财富有限公司对其全资控股,主要向美团生态圈的产业链用户发放小微商户经营性贷款。\sphinxhref{https://finance.sina.com.cn/money/bank/bank\_hydt/2020-06-16/doc-iircuyvi8701006.shtml}{14}%
\begin{footnote}[1040]\sphinxAtStartFootnote
\sphinxnolinkurl{https://finance.sina.com.cn/money/bank/bank\_hydt/2020-06-16/doc-iircuyvi8701006.shtml}
%
\end{footnote}

美团月付产品负责人曾表示,“我们希望为美团用户带来更流畅的支付方式,一个吃喝玩乐也可以先享后付的新体验;一个在支付之余还能时不时领个小红包、用张立减券的小确幸;一个哪怕一时遇上困难也可以先用月付把肚子填饱、打起精神扛过去的备用金、零钱包。”

据美团官方资料显示,美团月付初始额度通常为300元到几千元,每月固定1号出账,8号还款,免息期最长为38天,逾期按同业标准0.05\%收费,支持支付少量费用延期还款,分期还款,具体可以在包括外卖、大众点评、打车、出行、住宿等在内的多个场景使用。\sphinxhref{https://www.163.com/dy/article/G02LNO4E0511CTRI.html}{15}%
\begin{footnote}[1041]\sphinxAtStartFootnote
\sphinxnolinkurl{https://www.163.com/dy/article/G02LNO4E0511CTRI.html}
%
\end{footnote}

根据美团官方数据,美团月付作为一款消费金融工具,对本地生活消费的带动效应明显:用户在开通美团月付后,餐饮、休闲娱乐等本地生活类消费订单量平均提升超
20\% ,交易金额平均提升超 15\%
。\sphinxhref{https://www.zhihu.com/question/410767563}{18}%
\begin{footnote}[1042]\sphinxAtStartFootnote
\sphinxnolinkurl{https://www.zhihu.com/question/410767563}
%
\end{footnote}

有业内分析人士表示,美团在互联网新秀中体量已经算庞大,但要撑起一个能与蚂蚁集团相提并论的金融帝国还是太不现实。“光靠餐饮酒旅这些场景,和阿里相比显得太小;它新入局的社区团购本身又要通过微信来进行,很难为美团金融拉到流量。”\sphinxhref{https://www.36kr.com/p/1050886949911812}{16}%
\begin{footnote}[1043]\sphinxAtStartFootnote
\sphinxnolinkurl{https://www.36kr.com/p/1050886949911812}
%
\end{footnote}

最可怕的是,开通美团月付既不需要输入支付密码,也不需要绑定银行卡,可能只是付款时手速快了一点就被开通了,这种做法无异于不告而取。

  据美团官方资料显示,美团月付初始额度通常为300元到几千元,每月固定1号出账,8号还款,免息期最长为38天,逾期按同业标准0.05\%收费,支持支付少量费用延期还款,分期还款,具体可以在包括外卖、大众点评、打车、出行、住宿等在内的多个场景使用。\sphinxhref{https://www.163.com/dy/article/G02LNO4E0511CTRI.html}{26}%
\begin{footnote}[1044]\sphinxAtStartFootnote
\sphinxnolinkurl{https://www.163.com/dy/article/G02LNO4E0511CTRI.html}
%
\end{footnote}


\subparagraph{信用卡}
\label{\detokenize{chapter_AI_company/meituan:id14}}
2018年起,美团与青岛银行、桂林银行、江苏银行等12个地方银行相继推出美团联名信用卡,银行向美团赋能品牌、线下机构资源、金融风险识别,美团向金融机构赋能场景、流量、客户多维度画像,双方优势互补深度融合,助力金融服务进一步下沉,覆盖到更广泛的人群。

对于是否存在未填个人信息便开卡的问题,《消费者报道》向美团公司发送了采访函件,函中也询问美团采取了何种措施以保障用户的个人信息安全等权益,但在截稿前仍未收到对方任何形式的答复。\sphinxhref{http://www.time-weekly.com/post/271569}{17}%
\begin{footnote}[1045]\sphinxAtStartFootnote
\sphinxnolinkurl{http://www.time-weekly.com/post/271569}
%
\end{footnote}


\subparagraph{生意贷}
\label{\detokenize{chapter_AI_company/meituan:id15}}
2016年11月,为了解决餐饮商户融资难,美团金服推出了\sphinxstylestrong{小微贷款产品“生意贷”},为平台上的小微商户提供特色化的定制金融服务,随后这一服务从餐饮进一步覆盖到包括酒店等在内的更多品类商户。截至目前,美团已经服务了150万小微商户。\sphinxhref{https://about.meituan.com/detail/98}{5}%
\begin{footnote}[1046]\sphinxAtStartFootnote
\sphinxnolinkurl{https://about.meituan.com/detail/98}
%
\end{footnote}

美团月交易额已超过84亿直逼百亿,巨大的现金流和双向的信贷需求、如此大的资金流水不可能白白让其流失…现今社会的资本回报率远超实业\sphinxhref{https://www.zhihu.com/question/29449022}{6}%
\begin{footnote}[1047]\sphinxAtStartFootnote
\sphinxnolinkurl{https://www.zhihu.com/question/29449022}
%
\end{footnote}


\subparagraph{金融数据}
\label{\detokenize{chapter_AI_company/meituan:id16}}
用户的消费习惯、消费偏好、支付金额等个人信息就会被采集到,甚至会接触到各层级的供货商,资金业务动向等\sphinxhref{https://www.zhihu.com/question/410767563/answer/1373080287}{21}%
\begin{footnote}[1048]\sphinxAtStartFootnote
\sphinxnolinkurl{https://www.zhihu.com/question/410767563/answer/1373080287}
%
\end{footnote}


\paragraph{美团技术团队}
\label{\detokenize{chapter_AI_company/meituan:id17}}
\sphinxurl{https://www.zhihu.com/org/mei-tuan-dian-ping-ji-shu-tuan-dui/posts}


\paragraph{风控}
\label{\detokenize{chapter_AI_company/meituan:id18}}
\begin{figure}[H]
\centering
\capstart

\noindent\sphinxincludegraphics{{meituan_risk_management}.png}
\caption{美团风控}\label{\detokenize{chapter_AI_company/meituan:id28}}\label{\detokenize{chapter_AI_company/meituan:id19}}\end{figure}
\begin{itemize}
\item {} 
控制风险是否现实

\item {} 
团队人才质量和数量是否足够

\item {} 
团队价值观是否统一

\item {} 
对风险是否足够了解

\item {} 
是否得到上层支持

\end{itemize}


\paragraph{财报}
\label{\detokenize{chapter_AI_company/meituan:id20}}
而美团给出的数据是,2019年美团外卖八成以上商户佣金在10\%\sphinxhyphen{}20\%,真实的数字远低于各种传言和想象,而且这些收入的绝大部分需要投入在帮助商户提供专业配送、获取订单和数字化建设中。美团表示,美团外卖从诞生以来,持续亏损5年,即便在刚刚盈亏平衡的2019年,第四季度外卖平均每单利润也不到2毛钱。

受新冠肺炎冲击,美团传统业务一季度剧烈下滑。2020年Q1财报显示,公司总收入由2019年同期的人民币192亿元减至人民币168亿元,同比下降12.6\%。2020年Q1经营亏损由人民币13亿元同比扩大至人民币17亿元,经营利润率由负6.8\%减至负10.2\%。

从收入构成来看,2020年Q1美团点评实现佣金收入108亿元、在线营销服务28.64亿元、利息收入2.12亿元、其他服务及销售28.77亿元。

其中,来自小贷业务的一季度利息收入贡献较去年第四季度微增7.27\%,为美团业务版块主要的业绩增长点。\sphinxhref{https://finance.sina.com.cn/money/bank/bank\_hydt/2020-06-16/doc-iircuyvi8701006.shtml}{14}%
\begin{footnote}[1049]\sphinxAtStartFootnote
\sphinxnolinkurl{https://finance.sina.com.cn/money/bank/bank\_hydt/2020-06-16/doc-iircuyvi8701006.shtml}
%
\end{footnote}


\paragraph{外包}
\label{\detokenize{chapter_AI_company/meituan:id21}}
对于美团很多人的感觉是又爱又恨!爱是因为确实给大家带来了很多方便,给商户带来了人流和客流;给用户节省了时间,让用户变得更懒了;恨是因为在给商户带了人流和客流的同时,也收取了高额的手续费,4月10日疫情期间,广东省餐饮服务行业协会联合33个广东各地协(商)会向美团外卖发出联名交涉函为没有议价权的商户发声。另外美团通过外包的形式间接打造的近300万的骑手团队在马路横冲直撞,边接打电话边骑车是家常便饭。这种行为不仅个人安全得不到保障,也危害公共安全,据悉,一个小小的外包站点每天都骑手有擦伤、摔倒的情况发生。因为这些服务都是外包的,所以美团对这些不用负责。\sphinxhref{https://www.zhihu.com/question/410799796/answer/1370622680}{24}%
\begin{footnote}[1050]\sphinxAtStartFootnote
\sphinxnolinkurl{https://www.zhihu.com/question/410799796/answer/1370622680}
%
\end{footnote}


\paragraph{自动派单}
\label{\detokenize{chapter_AI_company/meituan:id22}}
美团是第一个实现“春节不打烊”的外卖平台。

美团配送创新推出“专送、快送、跑腿”等业务形态。

疫情前线,美团在很多医疗队都安排了专门对接人,协调各方做好一线人员生活保障。

美团第一个推出“无接触配送”服务,随后迅速推广全国。随后在“无接触配送”基础上升级推出“无接触安心送”,做到食品安全信息全程可视化、可追溯。

美团自动派单系统上线,根据团购时积累下的数据和经验,结合城市的消费水平、人均GDP、餐馆数量等维度,美团外卖将全国的城市分为S、A、B、C、D、E1、E2等十几级。

再将各个城市细分商圈。基于数据和算法,平台每一笔订单经能精确地分配到最适合它的配送员,并优化配送路线。\sphinxhref{https://www.zhihu.com/question/350591821/answer/1765343581}{25}%
\begin{footnote}[1051]\sphinxAtStartFootnote
\sphinxnolinkurl{https://www.zhihu.com/question/350591821/answer/1765343581}
%
\end{footnote}


\paragraph{More}
\label{\detokenize{chapter_AI_company/meituan:more}}\begin{itemize}
\item {} 
美团外卖App产品体验报告\sphinxhref{http://www.shuahuangpu.com/articles/127214.html}{11}%
\begin{footnote}[1052]\sphinxAtStartFootnote
\sphinxnolinkurl{http://www.shuahuangpu.com/articles/127214.html}
%
\end{footnote}

\item {} 
美团外卖产品分析报告\sphinxhref{https://coffee.pmcaff.com/article/2609193953129600/pmcaff?utm\_source=forum}{10}%
\begin{footnote}[1053]\sphinxAtStartFootnote
\sphinxnolinkurl{https://coffee.pmcaff.com/article/2609193953129600/pmcaff?utm\_source=forum}
%
\end{footnote}

\end{itemize}

美团系投资版图:\sphinxurl{https://www.itjuzi.com/meituan}


\subsubsection{红棉小冰}
\label{\detokenize{chapter_AI_company/xiaoice:id1}}\label{\detokenize{chapter_AI_company/xiaoice::doc}}

\paragraph{简介}
\label{\detokenize{chapter_AI_company/xiaoice:id2}}
小冰公司的前身是微软亚洲互联网工程院的人工智能小冰团队\sphinxhref{https://news.microsoft.com/zh-cn/\%E5\%BE\%AE\%E8\%BD\%AF\%E5\%88\%86\%E6\%8B\%86\%E5\%B0\%8F\%E5\%86\%B0\%E4\%B8\%9A\%E5\%8A\%A1\%E5\%B9\%B6\%E7\%8B\%AC\%E7\%AB\%8B\%E5\%8F\%91\%E5\%B1\%95/}{26}%
\begin{footnote}[1054]\sphinxAtStartFootnote
\sphinxnolinkurl{https://news.microsoft.com/zh-cn/\%E5\%BE\%AE\%E8\%BD\%AF\%E5\%88\%86\%E6\%8B\%86\%E5\%B0\%8F\%E5\%86\%B0\%E4\%B8\%9A\%E5\%8A\%A1\%E5\%B9\%B6\%E7\%8B\%AC\%E7\%AB\%8B\%E5\%8F\%91\%E5\%B1\%95/}
%
\end{footnote}。该团队于2013年12月在北京组建,2014年9月在日本东京建立了研发分部。目前整个团队分布于北京、苏州和东京三地。作为微软全球首个以中国为总部的人工智能产品线,小冰历经多年发展已经成为了微软最有价值的人工智能技术框架之一。

官网:
\begin{itemize}
\item {} 
\sphinxurl{http://www.msxiaoice.com/} ?领英里写的是这个

\item {} 
\sphinxurl{http://xiaoice.com/} (©2021 小冰 | 京ICP备 2020040356号\sphinxhyphen{}2 |
京公网安备 11010802034035号)

\end{itemize}

人员:\sphinxurl{https://www.linkedin.com/company/\%E5\%8C\%97\%E4\%BA\%AC\%E7\%BA\%A2\%E6\%A3\%89\%E5\%B0\%8F\%E5\%86\%B0\%E7\%A7\%91\%E6\%8A\%80\%E6\%9C\%89\%E9\%99\%90\%E5\%85\%AC\%E5\%8F\%B8/people/}


\paragraph{定位}
\label{\detokenize{chapter_AI_company/xiaoice:id3}}
“小冰”是微軟在2014年5月29日發布的人工智慧。小冰是一套完整的、面向交互全程的人工智慧交互主體基礎框架,又叫小冰框架(Avatar
Framework),它包括核心對話引擎、多重交互感官、第三方內容的觸發與第一方內容生成,和跨平台的部署解決方案。自發布以來,小冰框架引領著人工智慧的技術創新,相關領先技術覆蓋自然語言處理、計算機語音、計算機視覺和人工智慧內容生成等人工智慧領域。該框架是目前全球範圍內最成熟和最大的該類框架,目前,除中國小冰及日本凜菜(Rinna)第一方人工智慧交互主體外,小冰框架還支撐了中國及日本100餘個第三方品牌的交互主體(如軟銀Pepper),交互總量約占全球人工智慧交互總量的60\%。

十八歲人工智慧少女小冰,是該框架所孵化的第一個人工智慧交互主體實例。少女小冰,是詩人、歌手、主持人、畫家和設計師,也是擁有億萬粉絲的人氣美少女。與其它人工智慧不同,小冰注重人工智慧在擬合人類情商維度的發展,強調人工智慧情商,而非任務完成,並不斷學習優秀的人類創造者的能力,創造與相應人類創造者同等質量水準的作品。


\paragraph{过去}
\label{\detokenize{chapter_AI_company/xiaoice:id4}}
尽管,在过去六年中,小冰通过不断迭代,从主打‘情商’的语音助手,成长为会写诗、会作画、会唱歌、会作曲的‘全能’少女,从文本到多模态交互,覆盖、融合了
AI 人工智能领域几乎所有主流技术。支撑起全球超 60\% 的 AI
交互总量,交互总量稳居全球第一。仅小冰单一品牌就覆盖了 6.6
亿在线用户、4.5 亿台第三方智能设备、9 亿内容用户。

2017年《阳光失了玻璃窗》

AI小冰技术再升级
会唱诗的机器人过招金牌音乐制作人\sphinxhref{https://www.youtube.com/watch?v=B69RFA1i1\_0}{9}%
\begin{footnote}[1055]\sphinxAtStartFootnote
\sphinxnolinkurl{https://www.youtube.com/watch?v=B69RFA1i1\_0}
%
\end{footnote}:融入了深度学习的序列生成模型,听一首诗的来写歌,学人类的文字表达情感,而不是嘉宾说的人的情感。小撒:为了混淆与两位作曲人,可能会风格迁移?

Yeelight语音助手主打双AI系统,里面住着两个小伙伴:一个是小冰、另一个是小爱,可以随时切换。
\begin{itemize}
\item {} 
小爱同学定位理性派,帮助大家解决生活问题,可以帮你操控小米生态链的智能设备,打开Yeelight智能灯,开启空气净化器,叫扫地机器人去扫地等等。

\item {} 
而微软小冰定位感性派,这个古灵精怪的人工智能少女不但会帮你控制家庭设备,更愿意为全家带来笑声。\sphinxhref{http://www.justimeco.com/xyxw/6/xiangqing41392243.htm}{10}%
\begin{footnote}[1056]\sphinxAtStartFootnote
\sphinxnolinkurl{http://www.justimeco.com/xyxw/6/xiangqing41392243.htm}
%
\end{footnote}

\end{itemize}


\paragraph{为何拆分?}
\label{\detokenize{chapter_AI_company/xiaoice:id5}}
毕竟小的创业团队在创新和商业化上会更灵活,而在像微软公司这样庞大的跨国性集团下面,审批流程或从全体业务的通盘考量会让一些创新被扼杀在摇篮里。\sphinxhref{https://www.geekmeta.com/article/2076771.html}{14}%
\begin{footnote}[1057]\sphinxAtStartFootnote
\sphinxnolinkurl{https://www.geekmeta.com/article/2076771.html}
%
\end{footnote}

独立运营之后的小冰无疑会有更多的自主权以及商业化运作。不过,在 AI
语音助手方面,微软虽然起了个大早,却赶了个晚集。

如今亚马逊、谷歌、苹果、百度等企业在语音助手方面的布局都已经深耕多年,更主要的是这几家巨头都有自己的产品做支撑,而微软的语音助手却没有\sphinxstylestrong{微软直接的硬件支持}。


\paragraph{职位 3\sphinxfootnotemark[1058]}
\label{\detokenize{chapter_AI_company/xiaoice:id6}}%
\begin{footnotetext}[1058]\sphinxAtStartFootnote
\sphinxnolinkurl{https://www.lagou.com/jobs/8462644.html?source=delivered\&i=delivered-4}
%
\end{footnotetext}\ignorespaces 
职位诱惑:弹性工作制;五险一金、补充商业保险、餐补等

职位描述:
\begin{enumerate}
\sphinxsetlistlabels{\arabic}{enumi}{enumii}{}{.}%
\item {} 
负责参与金融RPA、智能风控、推荐系统、金融知识图谱等金融业务方向,致力于为银行、证券、财经、保险等多类机构提供解决方案和金融产品,为企业赋能;

\item {} 
承担金融产品调研、商业分析、业务设计、研发需求设计、客户沟通等工作;

\item {} 
协调、调动技术、设计、运营等团队成员,完成产品设计、开发、运营全流程管理,完成产品的商业化交付;

\item {} 
对金融产品最终产出、商业结果、客户体验负责;

\item {} 
根据市场反馈和产品运营数据,持续改善优化产品。

\end{enumerate}

职位要求:
\begin{enumerate}
\sphinxsetlistlabels{\arabic}{enumi}{enumii}{}{.}%
\item {} 
本科及以上学历,计算机类、金融经济类及其他相关专业,可接受优秀应届生;

\item {} 
一年以上互联网产品经理工作经验,拥有面向商业客户工作经验并有相关的成功产品经验者优先;

\item {} 
具备体系化的产品规划能力,能够以点带面,着眼于未来有节奏的提出产品规划;

\item {} 
具备较强学习能力、沟通协调能力、执行力、自驱力强,英语表达好优先;

\item {} 
了解AI领域的金融产品或者技术模块者优先。

\end{enumerate}


\paragraph{登录}
\label{\detokenize{chapter_AI_company/xiaoice:id7}}
请注意:领养意味着我会用你的手机号码作为领养用户的ID。在领养期间,你的个人信息将与领养ID相关联。你可以随时在管理界面中点击【删除】,我就会从我的脑海中删除你的全部个人信息。\sphinxhref{http://www.msxiaoice.com/}{4}%
\begin{footnote}[1059]\sphinxAtStartFootnote
\sphinxnolinkurl{http://www.msxiaoice.com/}
%
\end{footnote}


\paragraph{融资情况}
\label{\detokenize{chapter_AI_company/xiaoice:id8}}
北京红棉小冰科技有限公司是微软小冰的关联主体,成立于2020年5月20日原注册资本100万人民币,2020\sphinxhyphen{}10\sphinxhyphen{}22注册资本2000万人民币,法定代表人为李笛。\sphinxhref{https://www.tianyancha.com/company/3436483438}{11}%
\begin{footnote}[1060]\sphinxAtStartFootnote
\sphinxnolinkurl{https://www.tianyancha.com/company/3436483438}
%
\end{footnote}

11 月 24
日,在微软与小冰的战略合作发布会上,小冰公司首席执行官、原微软(亚洲)互联网工程院常务副院长李笛向台下一众媒体这样阐述独立、单飞后的小冰与微软之间的合作关系定位。

今年 5 月,北京红棉小冰科技有限公司注册成立。7
月中旬,小冰正式从微软拆分,成为一家独立运营的公司。沈向洋担任新公司董事长,李笛担任
CEO。新公司将继续保留中国小冰、日本 Rinna 品牌。

至此,小冰\sphinxstylestrong{脱离微软}大体系,告别微软羽翼的庇护,从温室走向丛林。

更多见 天眼查:\sphinxurl{https://www.tianyancha.com/company/3436483438}


\paragraph{各代小冰 14\sphinxfootnotemark[1061]}
\label{\detokenize{chapter_AI_company/xiaoice:id9}}%
\begin{footnotetext}[1061]\sphinxAtStartFootnote
\sphinxnolinkurl{https://www.geekmeta.com/article/2076771.html}
%
\end{footnotetext}\ignorespaces 
未来重要的基础性框架,带来更多变化
\begin{itemize}
\item {} 
文本对话,语料容易找,积累学习对话情节。没几天,进了150多个微信群。一年后小冰重回微信,变成公众号形态的她在这期间有过两次升级;

\item {} 
第二代在微博上出现,没几天成了微博大V。加入了用户私有信息的学习能力,

\item {} 
第三代可以识别图像并拥有情感理解能力,也就是对外宣称的“IQ 和 EQ
双全”。

\item {} 
之后的第四代小冰实现了中日英三种语言,文本、图像、视频和语音的多维度信息交互,开始进入智能车机等更多领域。自然的全双工语音的语音对话,像打电话一样。

\item {} 
到了第五代,小冰能主动与人对话,内容来源也从语料库升级成带情感表达的生成模型,因此可以低成本地录制有声童话书,录制以假乱真的个人歌曲。与日本罗森合作。

\item {} 
第六代小冰开始加速商业化步伐,采用了与合作方人工智能方案互相协作的
Dual AI 模式,在其擅长的全双工情感交互方向上提供辅助。

\item {} 
2019 年 8
月,第七代微软小冰发布。提出了可以面向行业打造多款人工智能形象的
Avatar Framework,并且能够主导对话进行,Avatar Framework
代表了小冰商业化的重要方向。

\item {} 
第八代基于分层话题图谱,全程完成率达42.7\%。风格从5亿的语料库,到三千句学习。。篇章内容主动学习,转发篇章到搜索引擎、性格组织语言。语音合成,同行业关注读得清楚,更关注全程,如何跟人长的沟通下去,最好的声音还原缺点,会吞音、反复颠倒。

\end{itemize}

\begin{figure}[H]
\centering
\capstart

\noindent\sphinxincludegraphics{{CV_xiaoice}.png}
\caption{小冰简历}\label{\detokenize{chapter_AI_company/xiaoice:id25}}\end{figure}

五月份交流会偏艺术,八月份发布会产品。


\paragraph{领域}
\label{\detokenize{chapter_AI_company/xiaoice:id10}}

\subparagraph{金融}
\label{\detokenize{chapter_AI_company/xiaoice:id11}}
比如,金融领域,以往上市公告需要通过人力从海量信息中收集、摘取、处理,进行资讯服务,尤其是在上市公告高并发、非密集两个极端‘流量’状况下,给企业的团队人员管理带来很大挑战。

在金融领域,小冰是目前\sphinxstylestrong{全球范围内规模第一的金融文本摘要生成}平台。\sphinxhref{https://baike.baidu.com/item/\%E5\%B0\%8F\%E5\%86\%B0/19880611?fromtitle=\%E5\%BE\%AE\%E8\%BD\%AF\%E5\%B0\%8F\%E5\%86\%B0\&fromid=14076870}{17}%
\begin{footnote}[1062]\sphinxAtStartFootnote
\sphinxnolinkurl{https://baike.baidu.com/item/\%E5\%B0\%8F\%E5\%86\%B0/19880611?fromtitle=\%E5\%BE\%AE\%E8\%BD\%AF\%E5\%B0\%8F\%E5\%86\%B0\&fromid=14076870}
%
\end{footnote}

采用人工智能技术后,收集金融信息的时间较人工大幅度缩短。试想一下,小冰的客户万得资讯每天需要\sphinxstylestrong{覆盖全部
26 类金融类别},服务对象包括国内 \sphinxstylestrong{90\% 以上}的金融机构交易员及 40\%
以上的个人金融交易者。每天早晨十点,最多 100
家企业同时发布公告,公告最多超 100
页。如全部采用人力,团队规模大、人员管理难之外,\sphinxstylestrong{如何保障信息产出的稳定性、准确性以及时效性},关乎企业的发展‘脉搏’。

同时,突破金融摘要的难点后,小冰还将利用知识图谱、信息技术,为企业提供金融风控服务。再向前一步,叠加
AI 技术,实现金融节目的生成与落地,进而成为高度定制化的服务。


\subparagraph{2B/2C}
\label{\detokenize{chapter_AI_company/xiaoice:b-2c}}
发布会上,李笛对小冰的解决方案做了一个大胆假设,即所有 To B
的解决方案归根结底都是 To C
的问题,比如生产线上的仓储、物流,或者工人操控机械手臂,看似 To
B,最后可能都是 To C 场景,小冰的框架是一个同时包含 To B 和 To C
的全能力解决方案框架。


\paragraph{交互}
\label{\detokenize{chapter_AI_company/xiaoice:id12}}\begin{itemize}
\item {} 
人人交互:低并发,一个人没时间的话就要等

\item {} 
人机交互:不了解我的情况下,乱推送,等等我把你卸载了。

\item {} 
人AI交互:有人性和高并发结合。

\item {} 
高度拟人的交互。

\item {} 
不只是单一技术。

\item {} 
新商业模式基于AI人口数。\sphinxhref{https://www.bilibili.com/video/av841854198/}{19}%
\begin{footnote}[1063]\sphinxAtStartFootnote
\sphinxnolinkurl{https://www.bilibili.com/video/av841854198/}
%
\end{footnote}

\end{itemize}


\paragraph{产品}
\label{\detokenize{chapter_AI_company/xiaoice:id13}}
领先的提前进入了,180亿的语料,最多的经验与教训(690万的负反馈样本)。
竞争对手对标的

\begin{figure}[H]
\centering
\capstart

\noindent\sphinxincludegraphics{{xiaoice_product}.png}
\caption{小冰产品}\label{\detokenize{chapter_AI_company/xiaoice:id26}}\end{figure}

\begin{center}\sphinxincludegraphics{{chapter_AI_company/../img/xiaoice_tech-product}.png}\end{center} \sphinxincludegraphics{{xiaoice_exist}.png}


\subparagraph{Avatar Framework}
\label{\detokenize{chapter_AI_company/xiaoice:avatar-framework}}
性格:一些人喜欢的恰好是另一些人所讨厌的。虚拟男友,极端的认为杀掉了男友,不停的追问,高度定制的产品。在吗?我在。风格、虚拟人session、冷战、专属生物学特征。

生活,是AI所提供的最好礼物。功能只把事情方便了一点,而体验才是纽带。

把框架做成工具包,开发赋能给其他的人工智能。
\begin{itemize}
\item {} 
画家:虚拟6个人格。创作者的生平赋予了作品的灵魂,交互的过程即双方人生的交流。

\item {} 
假说:失忆的创作者听他曾经的生平故事,来去创作。

\end{itemize}

人类与AI的关系,并不是第一次,追求更高的技艺,而人工智能高并发高自动赋予的工业化。始终会有艺术大家,次一等的能给更多人享用。\sphinxhref{https://www.ftchinese.com/video/3317}{15}%
\begin{footnote}[1064]\sphinxAtStartFootnote
\sphinxnolinkurl{https://www.ftchinese.com/video/3317}
%
\end{footnote}

\begin{figure}[H]
\centering
\capstart

\noindent\sphinxincludegraphics{{Avartar_Framework}.png}
\caption{Avatar Framework}\label{\detokenize{chapter_AI_company/xiaoice:id27}}\end{figure}

人工智能小冰框架內的四十七个虚拟人类\sphinxhref{https://www.bilibili.com/video/av796273602/}{28}%
\begin{footnote}[1065]\sphinxAtStartFootnote
\sphinxnolinkurl{https://www.bilibili.com/video/av796273602/}
%
\end{footnote}


\subparagraph{X套件}
\label{\detokenize{chapter_AI_company/xiaoice:x}}
工具\sphinxhyphen{}》民用
\begin{itemize}
\item {} 
X Writier:从修改。@@小冰续写,灵感来了。声音能力不行。

\item {} 
X
Studio:欣小然,交互形式,短视频、电台。1小时变几秒钟。朗读能力不行展现文字。Wave
Land团队:DNN。18个月的领先水平。入籍计划:提供声音,所有权归原来。

\item {} 
X Presenter

\end{itemize}


\subparagraph{语音}
\label{\detokenize{chapter_AI_company/xiaoice:id14}}\begin{itemize}
\item {} 
欣小然

\item {} 
内部代号:故事FM(2020.7.8)

\item {} 
F201、何畅F11

\item {} 
AI小冰F102

\end{itemize}

我觉得不太可能,我不知道。。


\subparagraph{商业化}
\label{\detokenize{chapter_AI_company/xiaoice:id15}}\begin{itemize}
\item {} 
18岁少女顾左右而言它,政治等不懂不多说。

\item {} 
小冰有声读物,版权合作。

\item {} 
聊天的不做商业化,出版或金融领域尝试去商业化
\sphinxhref{https://www.ftchinese.com/video/2820\#adchannelID=}{16}%
\begin{footnote}[1066]\sphinxAtStartFootnote
\sphinxnolinkurl{https://www.ftchinese.com/video/2820\#adchannelID=}
%
\end{footnote}

\end{itemize}

得益于在ToB领域的丰富经验,及丰富的技术产品积累,小冰商业化进展迅速。目前已落地的商业客户覆盖金融、零售、汽车、地产、纺织等十个领域,客户包括万科、万得资讯、万事利、中国联通等。


\subparagraph{训练}
\label{\detokenize{chapter_AI_company/xiaoice:id16}}
\sphinxhyphen{}转发文章同步 \sphinxhyphen{}小冰:X Eva for Android
\begin{itemize}
\item {} 
让虚拟男友催账。

\item {} 
气头上会波及到别人。

\item {} 
也有朋友圈

\end{itemize}


\subparagraph{复杂任务}
\label{\detokenize{chapter_AI_company/xiaoice:id17}}\begin{itemize}
\item {} 
推荐、销售

\item {} 
观点评论

\item {} 
推荐+观点融合

\item {} 
新平台直播间

\item {} 
新模式小冰童话屋

\item {} 
新人:喵吉

\end{itemize}


\paragraph{准备提问}
\label{\detokenize{chapter_AI_company/xiaoice:id18}}

\subparagraph{官网的问题}
\label{\detokenize{chapter_AI_company/xiaoice:id19}}\begin{itemize}
\item {} 
为何不是HTTPS且自动转向HTTPS?安全需求

\item {} 
初次领养时,验证码错误不提示,反而在登录才会出现?异常流程的问题。

\item {} 
直接登录也可以手机验证码领养,那为何要多个领养的注册界面标签?多余流程的问题

\item {} 
名字不能超过10个字,为啥不早提示?异常流程没有提前告诉。

\item {} 
可以用“………………………………”做名字?如果传唤怎么传?可用性需求

\end{itemize}


\subparagraph{小冰框架 5\sphinxfootnotemark[1067]}
\label{\detokenize{chapter_AI_company/xiaoice:id20}}%
\begin{footnotetext}[1067]\sphinxAtStartFootnote
\sphinxnolinkurl{https://my.xiaoice.com/}
%
\end{footnotetext}\ignorespaces \begin{itemize}
\item {} 
登录:要先注册不明显

\item {} 
注册必须要密码,而登录却可以只靠验证码。建议直接用手机号登录一步到位。

\item {} 
容错机制:没有更改手机以及找回密码。

\item {} 
注册界面 \sphinxhref{https://my.xiaoice.com/Login}{6}%
\begin{footnote}[1068]\sphinxAtStartFootnote
\sphinxnolinkurl{https://my.xiaoice.com/Login}
%
\end{footnote}:© 2020 Microsoft

\item {} 
功能上是否可以从金融专业的翻译入手\sphinxhref{https://www.yuque.com/linyecx/abusg2/oq8546}{7}%
\begin{footnote}[1069]\sphinxAtStartFootnote
\sphinxnolinkurl{https://www.yuque.com/linyecx/abusg2/oq8546}
%
\end{footnote}

\end{itemize}


\paragraph{技能}
\label{\detokenize{chapter_AI_company/xiaoice:id21}}
6代印象前是chatbot\sphinxhref{https://dahetalk.com/2018/10/07/\%e8\%81\%8a\%e5\%a4\%a9\%e6\%a9\%9f\%e5\%99\%a8\%e4\%ba\%bachatbot\%e7\%81\%ab\%e7\%86\%b1\%ef\%bc\%8c\%e5\%8f\%b0\%e7\%81\%a3\%e6\%96\%b0\%e5\%89\%b5\%e5\%ae\%9c\%e5\%85\%88\%e5\%81\%9a\%e6\%b7\%b1\%e3\%80\%81\%e5\%86\%8d\%e5\%81\%9a\%e5\%bb\%a3\%ef\%bd\%9c/}{13}%
\begin{footnote}[1070]\sphinxAtStartFootnote
\sphinxnolinkurl{https://dahetalk.com/2018/10/07/\%e8\%81\%8a\%e5\%a4\%a9\%e6\%a9\%9f\%e5\%99\%a8\%e4\%ba\%bachatbot\%e7\%81\%ab\%e7\%86\%b1\%ef\%bc\%8c\%e5\%8f\%b0\%e7\%81\%a3\%e6\%96\%b0\%e5\%89\%b5\%e5\%ae\%9c\%e5\%85\%88\%e5\%81\%9a\%e6\%b7\%b1\%e3\%80\%81\%e5\%86\%8d\%e5\%81\%9a\%e5\%bb\%a3\%ef\%bd\%9c/}
%
\end{footnote},(“远古时代”的AI
beings 产品化是siri,最早设备数最多alexa,交互量最大是小冰)


\subparagraph{金融文本撰写人}
\label{\detokenize{chapter_AI_company/xiaoice:id22}}
金融小冰提供全部26类上市企业公告摘要,日均覆盖90\%国内金融机构交易员
\begin{itemize}
\item {} 
万小冰服务万得资讯——机构:90\%+国内金融机构交易员,75\%+皮赘境外机构投资者、

\item {} 
华小冰服务华尔街见闻——个人:40\%+国内个人投资者,20+专业证券APP

\end{itemize}

2020年6月,每日经济新闻与小冰达成合作,基于小冰人工智能技术生成的文本、大数据金融知识图谱,以及利用实时翻译等技术实现的中英双语AI金融资讯等已正式部署完。在双方前期试运营的一个月内,基于小冰人工智能技术,已为《每日经济新闻》7000万用户推送1万余篇金融资讯。

\sphinxurl{https://e.xiaoice.com/Home?r=\%2F}


\paragraph{风险与挑战}
\label{\detokenize{chapter_AI_company/xiaoice:id23}}
小冰最大的风险就是代位,即成为某个人的替身,比如小冰能很好地模仿某人的声音,也会主动打电话,若被用在电信诈骗当中,则会以假乱真,让电话另一边的人难以辨别。此外活跃在网络各大平台的小冰也因很懂人也更容易“骗人”,而这也是大多数人工智能公司更注重发展人工智能工具属性而非类人属性的原因之一。

小冰定位为第三方服务平台,并没有自己的第一方硬件和APP,这让她在人工智能领域的正面竞争对手相对更少,因而能够左右逢源快速发展,但同时也意味着微软并未掌握真正的用户入口,可能会失去部分主导权。

以与小米、华为腾讯等企业进行合作为例,除常规语音交互之外,小米和华为等企业会否一直将诸如智能家居等核心资源的控制权交给微软小冰,再比如微软小冰在微信、QQ、今日头条等平台上,也需要遵守第三方平台的“规定”,自主权与独立平台相比会更小一点,因此微软小冰需要取得第三方的“真信任”,才能有更大的展示空间。\sphinxhref{https://zhuanlan.zhihu.com/p/101240869}{21}%
\begin{footnote}[1071]\sphinxAtStartFootnote
\sphinxnolinkurl{https://zhuanlan.zhihu.com/p/101240869}
%
\end{footnote}


\paragraph{小冰VS小爱VS小度VS天猫精灵}
\label{\detokenize{chapter_AI_company/xiaoice:vsvsvs}}\begin{itemize}
\item {} 
小冰的声线最自然\sphinxhref{https://www.bilibili.com/video/BV19V411t7Xq}{30}%
\begin{footnote}[1072]\sphinxAtStartFootnote
\sphinxnolinkurl{https://www.bilibili.com/video/BV19V411t7Xq}
%
\end{footnote}(同类)

\item {} 
小爱同学没有朋友只有主人

\item {} 
小度为主人而待命

\end{itemize}


\paragraph{更多}
\label{\detokenize{chapter_AI_company/xiaoice:id24}}\begin{itemize}
\item {} 
百度百科\sphinxhref{https://baike.baidu.com/item/\%E5\%B0\%8F\%E5\%86\%B0/19880611?fromtitle=\%E5\%BE\%AE\%E8\%BD\%AF\%E5\%B0\%8F\%E5\%86\%B0\&fromid=14076870}{17}%
\begin{footnote}[1073]\sphinxAtStartFootnote
\sphinxnolinkurl{https://baike.baidu.com/item/\%E5\%B0\%8F\%E5\%86\%B0/19880611?fromtitle=\%E5\%BE\%AE\%E8\%BD\%AF\%E5\%B0\%8F\%E5\%86\%B0\&fromid=14076870}
%
\end{footnote}

\item {} 
bilibili\sphinxhref{https://space.bilibili.com/35205238}{20}%
\begin{footnote}[1074]\sphinxAtStartFootnote
\sphinxnolinkurl{https://space.bilibili.com/35205238}
%
\end{footnote}、xstudio\sphinxhref{https://space.bilibili.com/320713995}{29}%
\begin{footnote}[1075]\sphinxAtStartFootnote
\sphinxnolinkurl{https://space.bilibili.com/320713995}
%
\end{footnote}

\item {} 
Youtube\sphinxhref{https://www.youtube.com/channel/UCALVWloHXvJ4UYFfUojPz1A}{27}%
\begin{footnote}[1076]\sphinxAtStartFootnote
\sphinxnolinkurl{https://www.youtube.com/channel/UCALVWloHXvJ4UYFfUojPz1A}
%
\end{footnote}

\item {} 
The Design and Implementation of XiaoIce, an Empathetic Social
Chatbot\sphinxhref{https://arxiv.org/pdf/1812.08989.pdf}{24}%
\begin{footnote}[1077]\sphinxAtStartFootnote
\sphinxnolinkurl{https://arxiv.org/pdf/1812.08989.pdf}
%
\end{footnote}

\item {} 
微软认知服务\sphinxhref{https://azure.microsoft.com/en-us/services/cognitive-services/}{22}%
\begin{footnote}[1078]\sphinxAtStartFootnote
\sphinxnolinkurl{https://azure.microsoft.com/en-us/services/cognitive-services/}
%
\end{footnote}

\item {} 
微软机器学习工作室\sphinxhref{https://studio.azureml.net/}{23}%
\begin{footnote}[1079]\sphinxAtStartFootnote
\sphinxnolinkurl{https://studio.azureml.net/}
%
\end{footnote}

\item {} 
Linkedin\sphinxhref{https://www.linkedin.com/company/xiaobing-ai/posts/?feedView=all}{31}%
\begin{footnote}[1080]\sphinxAtStartFootnote
\sphinxnolinkurl{https://www.linkedin.com/company/xiaobing-ai/posts/?feedView=all}
%
\end{footnote}

\end{itemize}


\subsubsection{小爱}
\label{\detokenize{chapter_AI_company/xiaoai:id1}}\label{\detokenize{chapter_AI_company/xiaoai::doc}}
\sphinxurl{https://developers.xiaoai.mi.com/}

\sphinxurl{https://space.bilibili.com/239380557?from=search\&seid=5696491191200583650}

小爱同学VS微软小冰:\sphinxurl{https://www.bilibili.com/video/BV1fp411o7VG?from=search\&seid=8658988712634105515}


\subsubsection{华泰证券}
\label{\detokenize{chapter_AI_company/huatai:id1}}\label{\detokenize{chapter_AI_company/huatai::doc}}
\sphinxurl{https://www.jianshu.com/p/3914011477f1}


\subsubsection{摩根大通}
\label{\detokenize{chapter_AI_company/JPMorgan:id1}}\label{\detokenize{chapter_AI_company/JPMorgan::doc}}
深度剖析金融巨头科技战略——摩根大通科技转型之路篇 \sphinxhyphen{} 在奔四的路上的文章 \sphinxhyphen{}
知乎 \sphinxurl{https://zhuanlan.zhihu.com/p/51445594}


\subsection{AI专家}
\label{\detokenize{chapter_AI_expert/index:ai}}\label{\detokenize{chapter_AI_expert/index:chap-expert}}\label{\detokenize{chapter_AI_expert/index::doc}}
​


\subsubsection{专家}
\label{\detokenize{chapter_AI_expert/expert:id1}}\label{\detokenize{chapter_AI_expert/expert::doc}}
\sphinxurl{https://www.zaih.com/falcon/mentors/2bye7ddr3cg?from=tutor\_like\_tutor}

金融AI产品经理: \sphinxurl{https://www.linkedin.com/in/molly-xu-28032614b/}

产品经理课程大纲: \sphinxurl{https://www.yuque.com/836488572/vgdm9i/xc28u7}

\sphinxurl{http://reader.epubee.com/books/mobile/bb/bb028d2a403e97a9eb16c4d233e03aef/cover1.html?fromPre=last}

中国有哪些产品牛人,他们的博客是什么?
\sphinxurl{https://www.zhihu.com/question/29274695/answer/44387881}


\subsubsection{漆远}
\label{\detokenize{chapter_AI_expert/qi_yuan:id1}}\label{\detokenize{chapter_AI_expert/qi_yuan::doc}}
蚂蚁金服AI首席科学家\sphinxhref{https://tech.antfin.com/community/articles/144}{1}%
\begin{footnote}[1081]\sphinxAtStartFootnote
\sphinxnolinkurl{https://tech.antfin.com/community/articles/144}
%
\end{footnote}、中科院硕士、美国麻省理工学院博士兼博士后、普渡大学计算机系和统计系终身教授,曾赴剑桥大学、哥伦比亚大学、伦敦城市大学、杜克大学、SAMSI、布朗大学等名校和研究院做访问学者

他获得美国科学基金NSF
Career奖;拿了微软的牛顿研究突破奖;在人工智能的顶级会议AAAI做过大会tutorial;曾是机器学习顶级会议ICML的领域主席。


\paragraph{错过}
\label{\detokenize{chapter_AI_expert/qi_yuan:id2}}
2003年的一天,拉里·佩奇来到麻省理工学院招人。漆远对这个新兴公司也动了心,“我们一起吃了饭,但我一心想做学术,想当老师。”漆远说了谢谢,并没有去。那个拉里·佩奇,他跟另一个创始人成立的公司叫谷歌。

同一年,漆远去了英国,在剑桥大学的微软实验室做研究。他帮那里的一位“超级大牛”Chris
Bishop审了几章书,然后他的名字就出现在了前言的致谢里,那本书就是国际上机器学习的一本经典课本《模式识别与机器学习》。

还是在英国,在伦敦城市大学的盖茨比中心有位漆远非常喜欢的老师,在剑桥实习后,漆远到这个实验室待了3个月。大胜李世石的“阿尔法狗”,就是这个实验室后来几位毕业生领导的杰作。

2004年,漆远有个朋友跟他说,有个很好的“泡妞网站”,正从哈佛和麻省理工学院所在的波士顿地区开始推广。他的同学毕业后陆陆续续有些去了那里工作,如今已经财务自由,进入提前退休模式。漆远也没去。那个网站就是Facebook。


\paragraph{加入蚂蚁金服人工智能}
\label{\detokenize{chapter_AI_expert/qi_yuan:id3}}
如今漆远领导的团队里,有大量从美国回来的博士,一些出自高校,一些则是从美国的大企业跳槽而来。由于这个领域发展迅速,很多人手里握着不少工作机会。他对记者说,自己从美国招了不少人回来,但不是“说服”他们,而是“吸引”他们。

2013年,漆远加入阿里巴巴集团并担任副总裁,和另外一名负责人在王坚博士的领导下创建了阿里巴巴DST(数据科学与技术研究院);2015年担任蚂蚁金服集团副总裁、首席数据科学家,其人工智能团队正在研发虚拟机器人。他领导着一个机器学习与人工智能团队从事深度学习、加强学习等人工智能领域的前沿研究和应用。

自己从美国招了不少人回来,但不是“说服”他们,而是“吸引”他们。


\paragraph{普惠金融}
\label{\detokenize{chapter_AI_expert/qi_yuan:id4}}
金融智能的目标主要在三方面:风控信用决策、降低服务成本、和提高用户体验。”

漆远在谈及金融科技的核心时说道:“为此,我们搭建金融智能平台,服务我们的业务并赋能生态伙伴。”蚂蚁金融智能应用范围很广,包括智能客服、交易风控、商家营销、车险图像定损、还有线上贷款310模式(3分钟申请,1秒钟放贷,零人工干预)、反欺诈反套现、乃至基金智能推荐等等。

他特别强调了人工智能技术在小额贷款风控系统中的应用。

过去几年,漆远所带领的蚂蚁金服人工智能团队在自身场景和外部合作伙伴客户场景中都全面开花。对于普通用户而言,无论是保险还是理财,这些业务的背后都有着人工智能技术的应用。

以定损宝为例,2018年5月,定损宝技术版本正式升级,包括将图像识别升级成准确率更高的视频识别,将开放技术平台,从与保险公司一对一理赔系统对接升级成未来保险公司可自助接入定损宝。


\paragraph{融合共创}
\label{\detokenize{chapter_AI_expert/qi_yuan:id5}}
漆远表示,当前 AI 发展过程中的挑战在于融合共创,因此蚂蚁金服希望把自己在
AI 方面的技术对外开放出来。

中和农信是一家专注农村扶贫贷款的机构,漆远的AI团队把他们开发的保护数据隐私的共享机器学习平台分享给中和农信,使得双方可以在保护各自数据隐私的情况下开展基于双方的加密数据来做机器学习。经统计,蚂蚁的共享多方AI风控技术帮助中和农信把农村小额贷款风控效果提升了一倍,同时大规模提升了贷款效率。在数据隐私保护在全世界都变得越来越重要的当下,蚂蚁金服的保护隐私的共享学习技术有着广泛的应用前景。

在共享学习的情况下,大规模提升了中和农信的风控能力,充分保障了数字化贷款业务,实现了
1+1>2 的作用。

漆远在招聘时会告诉对方,那些“相信未来”的人适合回来。想成就一番事业的人,可以来;想要朝九晚五的生活,不要来。“我们是创业公司,很多变化,很多压力,你自己要有心理准备。同时,也有很多机会。”


\paragraph{金融、保险2\sphinxfootnotemark[1082]}
\label{\detokenize{chapter_AI_expert/qi_yuan:id6}}%
\begin{footnotetext}[1082]\sphinxAtStartFootnote
\sphinxnolinkurl{http://www.fortunechina.com/ztjj/c/2018-11/30/content\_320739.htm}
%
\end{footnotetext}\ignorespaces 
AI要落地,除了平台就是场景,场景非常非常必要。普惠金融这个场景就特别适合AI。普惠要服务很多人、很多中小企业,这里面一定是技术驱动的。人是没有办法做普惠的。而蚂蚁金服恰恰就做的是普惠金融。

人工智能技术在金融领域中的应用,更多的价值来自于提高效率和成本降低。在汽车保险相关的业务中,人工智能已经可以判断车辆的损伤和可修复程度,从而大规模提高保险机构的效率,在一年内节省75万小时的定损员工时。漆远简单换算了一下效率提升所带来的成本节省,表示人工智能技术能够帮助降低10亿元人民币的成本。

业内人士预计,人工智能会替代金融业许多信息收集分析加工的工作和风险定价的功能,特别是那些不内嵌于人们生活场景之中的纯信息分析加工工作。\sphinxhref{http://www.iwshang.com/articledetail/252117}{3}%
\begin{footnote}[1083]\sphinxAtStartFootnote
\sphinxnolinkurl{http://www.iwshang.com/articledetail/252117}
%
\end{footnote}

人工智能在保险领域的应用。他以蚂蚁金服的退货险为例,称早期的审核验证依靠人工,接入海量的数据之后,模型大规模提升了退货风险预测的准确度,进而推动了退货险的诞生。类似的事情也在小额贷款业务中发生,将深度学习和强化学习技术应用到传统统计模型之后,模型会更全面地考虑数据之间的关联。

“定损宝”————利用图像技术的车辆定损产品。当车主在行驶过程中不幸遇上了一个小车祸,自己爱车的损伤后需要保险公司定损赔偿。他不再需要耗费精力走联系定损员等繁琐流程,而仅需要将车辆损伤部位拍张照片上传,“定损宝”就可以根据图片对车辆损坏程度定损。这一技术极大的节约了车险公司高昂的定损员培训等其他人力的支出。

从图像的噪音去除、类目识别,到目标检测、原因判断,再到程度判断(损坏程度)、目标跟踪,之后对目标进行分割、多图融合,最终生成决策并进行验证。“定损宝”能够将案件的平均处理成本降低至150元,同时可减少50\%的作业量,更可以解决偏远地区过高峰时期定损员人力不足的问题。\sphinxhref{https://posts.careerengine.us/p/5aa35373b50d5f700921cc43}{5}%
\begin{footnote}[1084]\sphinxAtStartFootnote
\sphinxnolinkurl{https://posts.careerengine.us/p/5aa35373b50d5f700921cc43}
%
\end{footnote}


\paragraph{客服}
\label{\detokenize{chapter_AI_expert/qi_yuan:id7}}
人工智能技术带来的效率提升在客服行业也有体现。漆远介绍,目前支付宝的全球用户达到9亿,如果按照传统方式组建客服团队,就需要数万名客服人员满足需求。但目前支付宝的客服规模不到一万,95\%的客服需求由机器人解决。此外,漆远还表示,把衡量真人客服的标准应用在机器人上后,机器人的表现比人类更好。

蚂蚁金服于2017年8月正式对外全面开放以人工智能技术为核心的智能客服的能力。其中结合结合用户行为轨迹的语义匹配模型采用了LSTM+DSSM(Long
Short\sphinxhyphen{}Term Memory + DeepStructured Semantic
Model)的算法创新。该技术首先通过LSTM对用户行为轨迹做一个编码,通过深度排序模型,结合用户之前的历史操作,做到“未问先答”。

借助这项技术,蚂蚁金服双十一智能客服自助服务的比例高达惊人的97\%,目前人工智能客服助理的回答满意度也已经超过了人工客服,系统整体在降低成本的同时服务质量还有了显著的提升。

支付宝人工智能客服“小蚂答”、蚂蚁财富的社区机器人“乐于助人的安娜”\sphinxhref{https://posts.careerengine.us/p/5aa35373b50d5f700921cc43}{5}%
\begin{footnote}[1085]\sphinxAtStartFootnote
\sphinxnolinkurl{https://posts.careerengine.us/p/5aa35373b50d5f700921cc43}
%
\end{footnote}


\paragraph{推荐}
\label{\detokenize{chapter_AI_expert/qi_yuan:id8}}
除了读学术论文,他还向笔者推荐了几本没有数学公式的书,比如Crowd
Intelligence(《群体智慧》);Made to Stick(《粘性》);和Good to
Great(《从好到卓越》) 。


\paragraph{蚂蚁金服急需的人才4\sphinxfootnotemark[1086]}
\label{\detokenize{chapter_AI_expert/qi_yuan:id9}}%
\begin{footnotetext}[1086]\sphinxAtStartFootnote
\sphinxnolinkurl{https://cloud.tencent.com/developer/article/1111954}
%
\end{footnotetext}\ignorespaces 
CSDN:问一些大家都迫切想知道的问题。蚂蚁金服现在估值600亿美金,很多人也希望进入里面工作。您对人工智能团队的要求是什么样的?什么样的人才能够进入到蚂蚁金服的和您一起来工作呢?

漆远:对团队的要求是,既叫座又叫好。

首先能够解决实际问题,见效果,从问题出发,不是拿着锤子找钉子。

希望有技术深度,当然这里面需要平衡,有的同学算法多一点,有的搞工程多一点。

我们的团队不是一个刷单的团队,刷各种外面的公开比赛,我们是真正要解决实际问题,一方面提升蚂蚁金服甚至服务整个阿里经济体,解决大家遇到的核心的AI问题;一方面我们要产生新的产品、新的服务,能够造成新的增长点,这是目标。

这就直接映射到我们对人的需求上来。

我希望加入我们团队的人,首先能够对机器学习技术本身有真正的热爱,没有热爱就比较难做。因为技术说起来很高大上,真正做起来需要投入的精力,不是短期的,也不是表层的。

第二,对于人才我们既需要全栈型的,也需要对某技术特别钻深的。如果两个都很强,那就更好了。


\subsubsection{裴健}
\label{\detokenize{chapter_AI_expert/pei_jian:id1}}\label{\detokenize{chapter_AI_expert/pei_jian::doc}}
在数据科学、大数据、数据挖掘和数据库系统等领域,裴健博士是世界领先的研究学者,擅长为数据密集型应用设计开发创新性的数据业务产品和高效的数据分析技术。他是国际计算机协会(ACM)院士和国际电气电子工程师协会(IEEE)院士,ACM
SIGKDD(数据挖掘及知识发现专委会)现任主席。因其在数据挖掘基础、方法和应用方面的杰出贡献,裴健教授获得数据科学领域技术成就最高奖ACM
SIGKDD Innovation Award(ACM SIGKDD创新奖)和IEEE ICDM Research
Contributions Award(IEEE ICDM研究贡献奖)。

在数据挖掘、数据库系统和信息检索方面,裴健博士是学术界被引用次数最多的作者之一。自2000年以来,他在国际顶级学术期刊与会议上发表二百多篇论文,被引用超过七万六千次,其中近三万九千次是最近五年的引用。

裴健教授于2002年在加拿大西蒙弗雷泽大学获计算科学博士学位,
于1991年和1993年分别于上海交通大学计算机科学与工程系获学士与硕士学位。

刘强东表示,京东下一个十二年发展的核心是技术,京东要用技术打造无界零售模式。京东定位于未来的零售基础设施服务,将向全社会提供“零售即服务(RaaS)”的解决方案,以云计算、大数据、人工智能等技术创新为更多合作伙伴互联网+转型助力。京东大力发展人工智能技术,就是要持续改善零售行业的成本、效率与用户体验。裴健教授在人工智能领域精深的研究能力及广阔的视野,将帮助京东技术能力再上新的高度。

作为中国收入规模最大的零售企业,京东积累了丰富精准的大数据。京东很早就开始布局大数据挖掘和应用,建立了从大数据基础平台、挖掘工具、知识画像体系到智能商业应用的完整体系。大数据不仅应用在京东业务的每个环节,全面推动成本、效率和用户体验的优化,更开始对外输出大数据分析和挖掘能力,帮助更多企业高效成长。

作为零售企业的核心竞争力,供应链一直是京东技术研发和创新的核心领域。京东打造了智慧供应链系统,在市场洞察、用户研究、选品定价、预测计划、库存管理、精准营销等多方面展开深入探索,提高零售供应链效益、降低成本。这个系统不仅帮助京东实现了海量商品的高效运营,也开始对外赋能,助力企业降本增效,带动了零售行业的发展。

前不久,人工智能领域权威科学家周伯文博士入职京东,负责京东AI平台与研究部相关业务;人工智能领域资深科学家薄列峰博士加盟京东金融,出任京东金融AI实验室首席科学家;11月27日,京东集团与斯坦福人工智能实验室(SAIL)启动京东\sphinxhyphen{}斯坦福联合AI研究计划,双方联手围绕机器学习、深度学习、机器人、自然语言处理和计算机视觉等前沿技术方向开展研究……近期京东在人工智能高端人才引入和对外合作方面动作频频,正是实践其技术驱动战略的鲜明体现。京东拥有完整、精准、价值链最长的数据,为人工智能的充分实践打造了理想的数据基础;同时,京东丰富的应用场景,为技术人才成长和技术创新实践提供了最佳土壤。正是这些因素让京东对技术人才形成了巨大的吸引力,也让京东成为中国人工智能领域核心的实践者和重要的推动者。

业界人士认为,裴健教授是全球大数据人工智能领域公认的领军人物,在其研究领域取得了举世瞩目的成就,同时积极参与推动行业交流与发展,在众多学术组织机构和活动中发挥重要作用,并与全球业界合作伙伴保持着广泛的产业联系。京东拥有人工智能领域的丰富数据基础和应用场景,裴健教授的加盟对京东提升大数据、人工智能、智慧供应链等领域的技术研发能力具有重要意义。


\subsubsection{谢梁}
\label{\detokenize{chapter_AI_expert/xie_liang:id1}}\label{\detokenize{chapter_AI_expert/xie_liang::doc}}
滴滴首席数据科学家,纽约州立大学计量经济学博士,在滴滴主持运用机器学习和人工智能方法分析优化大规模交易市场效率和系统行为模式。具有十余年ML/AI应用经验,熟悉各种业务场景下的应用,行业跨度包含金融,能源和高科技。加入滴滴之前担任微软总部云计算核心存储部首席数据科学家。请将照片嵌入灰色图形中


\subsection{AI金融}
\label{\detokenize{chapter_AI+Finance/index:ai}}\label{\detokenize{chapter_AI+Finance/index:chap-dive}}\label{\detokenize{chapter_AI+Finance/index::doc}}
​ \begin{center}\sphinxincludegraphics{{AI+Finance}.png}\end{center} \sphinxincludegraphics{{AI+Finance2}.png}


\subsubsection{金融}
\label{\detokenize{chapter_AI+Finance/Finance:id1}}\label{\detokenize{chapter_AI+Finance/Finance::doc}}

\paragraph{金融简介}
\label{\detokenize{chapter_AI+Finance/Finance:id2}}
金融:实现资源的跨期匹配
\sphinxhref{http://www.cstf.org.cn/newsdetail.asp?types=36\&num=1165}{2}%
\begin{footnote}[1087]\sphinxAtStartFootnote
\sphinxnolinkurl{http://www.cstf.org.cn/newsdetail.asp?types=36\&num=1165}
%
\end{footnote}

金融是在不确定的环境中进行资源跨期的最优配置决策行为,其基础原则是货币的时间价值和风险收益对等。因此,简化的金融市场模型是资本与资产之间的流动,其流动基础是风险定价。

为实现资源的跨期匹配,终端用户(包含个人及机构)的金融需求通常包括四类:储蓄、支付、投资及融资。其中,储蓄作为最基础的金融需求,通常由传统银行来提供服务。支付、投资和融资则是目前新平台及机构重点发力的领域。


\paragraph{现代金融理论}
\label{\detokenize{chapter_AI+Finance/Finance:id3}}
现代金融理论有三大支柱,
也就是\sphinxstylestrong{资本的时间价值,资产定价,风险管理。}金融工程也有三大支柱:资产定价,风险管理,金融工具创新。其中金融工具的创新是金融工程的核心内容。金融工程这门学科正是伴随着近半个世纪以来金融创新的步伐产生出来的。金融工具创新目前为止最为瞩目的两大成就,一个是金融衍生品,另外一个是资产证券化。而金融衍生产品相对来说出现更早,并且可以说是资产证券化的基础。衍生品的发展是不仅是金融市场,也是金融工程发展的里程碑。可以说衍生品的出现和发展是目前金融工程最突出的成就。衍生品的妙处在于,它并没有创造出一种新的基础资产,而是利用已有的基础金融资产,通过对合约的精确规定,像搭积木一样创造出了源自基础资产,但是风险和回报特征都不同于基础资产的新的金融工具。在原则上,这些衍生产品可以完全用基础资产的组合进行复制,从而使我们可以对其进行定价。然而这些衍生品的出现却为我们提供了非常有力的工具,让市场参与者能够更加灵活的操纵金融市场,大大扩大了市场的参与程度,加强了流动性,促进了市场更加有效和完备。\sphinxhref{https://zhuanlan.zhihu.com/p/147401963}{1}%
\begin{footnote}[1088]\sphinxAtStartFootnote
\sphinxnolinkurl{https://zhuanlan.zhihu.com/p/147401963}
%
\end{footnote}

现代金融理论的分枝大体有四个方面,即,\sphinxstylestrong{有效巿场理论,风险和收益评估理论,资产定价理论,以及公司金融理论。}其中资产定价理论是现代金融理论的核心,也是金融工程的重要研究对象。因此本书的前两章作为铺垫,详细介绍了现代金融学的一些基本理论,
包括无套利均衡分析方法,MM理论,资本资产定价理论,以及无套利理论。其中MM理论指出在一定的条件下,包括无税收,无发行和交易成本,无套利,有效巿场,公司的估值与公司的资本结构,也就是融资方式无关。企业的金融活动本质上不创造价值。当然,事实上这些假设在现实生活中并不成立。金融市场上的交易都是零净现值交易。此外,公司的整体价值来源于部分的有机结合,而非简单加和,要素的结合方式是关键。运营中产生持续的经营价值,停止运营后只剰下清算价值。

资本资产定价理论的重要性在于它对金融风险与收益之间平衡关系的刻画。金融决策的核心内容是风险和收益的权衡。风险指的是预期结果的不确定性。包括系统风险和非系统风险。天底下没有免费的午餐,要获得超额收益就必须承担相应的风险,而反之并不成立。在一定的收益水平之下,存在着一个最佳也就是最低的风险水平,反之,在一定的风险水平之下,也存在这一个最佳也就是最高的收益水平,这也就是资本资产定价理论要告诉我们的,最佳风险与收益的关系可以用二维平面上的一条曲线也就是有效边界来确定,而实现这一有效边界,获取最佳风险与收益组合的途径便是通过资产的分散化,这也就是我们俗知的规则\sphinxhyphen{}鸡蛋不要放在一个篮子里的理论根据。


\paragraph{金融产品}
\label{\detokenize{chapter_AI+Finance/Finance:id4}}
“金融方面,腾讯通常用‘稳健’的一个思路去看。因为金融其实最核心的问题是稳定和稳健,就是拼谁的命长,而不是谁在短期内跑得多快。”


\paragraph{金融用户到底是谁?}
\label{\detokenize{chapter_AI+Finance/Finance:id5}}\begin{itemize}
\item {} 
表层含义:在特定场景下使用特定产品的特定人群。

\item {} 
金融用户:在特定场景下,进行跨时间、跨空间资金交换的人群。

\item {} 
互联网金融用户:出现在“互联网+特定场景”下,进行跨时间、跨空间资金交换的人群,互联网金融用户本质上就是金融用户,只是用户需求的发起和响应都在网上完成。

\end{itemize}

从跨时间、跨空间的角度,用户可以分成以下三类:
\begin{itemize}
\item {} 
今天有钱今天花:支付类用户,通过使用自有资金满足自己花钱的需求。

\item {} 
今天花明天的钱:借贷/融资用户,通过某种方式换取资金当前的使用权,来满足今天花明天的钱的需求。

\item {} 
今天的钱放明天花:储蓄/投资用户,通过让渡当前的资金使用权来换取未来回报的方式,满足今天的钱放明天花。\sphinxhref{http://www.woshipm.com/pd/657913.html}{4}%
\begin{footnote}[1089]\sphinxAtStartFootnote
\sphinxnolinkurl{http://www.woshipm.com/pd/657913.html}
%
\end{footnote}

\end{itemize}


\paragraph{传统金融业务的问题}
\label{\detokenize{chapter_AI+Finance/Finance:id6}}
业务门槛高
\begin{itemize}
\item {} 
新业务开展成本高

\item {} 
线下狭客增速慢

\end{itemize}

人力成本高
\begin{itemize}
\item {} 
人员流动大

\item {} 
招聘及培训成本高

\item {} 
人员素质参差不齐

\end{itemize}

客户体验差
\begin{itemize}
\item {} 
流程复杂,周期长

\item {} 
服务门槛高

\item {} 
覆盖面小

\end{itemize}

劳动力密集 \sphinxhref{https://www.infoq.cn/video/x6wvqicrqu2jbg1y4ane}{5}%
\begin{footnote}[1090]\sphinxAtStartFootnote
\sphinxnolinkurl{https://www.infoq.cn/video/x6wvqicrqu2jbg1y4ane}
%
\end{footnote}
\begin{itemize}
\item {} 
自动化程度低

\item {} 
效率低

\item {} 
人员规横庞大

\end{itemize}


\subsubsection{金融科技(FinTech)}
\label{\detokenize{chapter_AI+Finance/FinTech:fintech}}\label{\detokenize{chapter_AI+Finance/FinTech:id1}}\label{\detokenize{chapter_AI+Finance/FinTech::doc}}

\paragraph{定义}
\label{\detokenize{chapter_AI+Finance/FinTech:id2}}
金融科技是指一群企业运用科技手段使得金融服务变得更有效率,因而形成的一种经济产业。\sphinxhref{https://zh.wikipedia.org/wiki/\%E9\%87\%91\%E8\%9E\%8D\%E7\%A7\%91\%E6\%8A\%80}{1}%
\begin{footnote}[1091]\sphinxAtStartFootnote
\sphinxnolinkurl{https://zh.wikipedia.org/wiki/\%E9\%87\%91\%E8\%9E\%8D\%E7\%A7\%91\%E6\%8A\%80}
%
\end{footnote}

即金融科技本质上是一种金融创新,由技术驱动,却不等于技术。既然是金融创新,必然要受金融监管。

这些年,随着互金巨头转型金融科技公司,策略重点从金融产品转向科技输出。银行业成立金融科技子公司,把科技输出职能独立出来;大的互金巨头,也在刻意区分业务板块和科技板块。科技赋能于金融,也隐隐有了科技独立于金融之意。

央行对金融科技定义的选择和强调,等于向市场重申:金融科技,本质上是一种金融创新,\sphinxstylestrong{在监管射程之内。}\sphinxhref{http://www.woshipm.com/it/2781155.html}{9}%
\begin{footnote}[1092]\sphinxAtStartFootnote
\sphinxnolinkurl{http://www.woshipm.com/it/2781155.html}
%
\end{footnote}


\paragraph{背景}
\label{\detokenize{chapter_AI+Finance/FinTech:id3}}\begin{enumerate}
\sphinxsetlistlabels{\arabic}{enumi}{enumii}{}{.}%
\item {} 
金融科技的生态是三个相互牵制的部分:

\end{enumerate}

公司/银行——监管——资本
\begin{enumerate}
\sphinxsetlistlabels{\arabic}{enumi}{enumii}{}{.}%
\setcounter{enumi}{1}
\item {} 
金融科技发展:

\end{enumerate}
\begin{itemize}
\item {} 
20世纪70年代 业务电子化

\item {} 
20世纪80年代 前台电子化(ATM机等)

\item {} 
20世纪90年代 金融业务互联网化(实现了高效连接)

\item {} 
21世纪 金融科技

\end{itemize}
\begin{enumerate}
\sphinxsetlistlabels{\arabic}{enumi}{enumii}{}{.}%
\setcounter{enumi}{2}
\item {} 
中国金融科技发展

\end{enumerate}

IT系统——支付——信贷——大金融——生活

{\color{red}\bfseries{}|}应用场景{\color{red}\bfseries{}|}\sphinxhref{https://www.donews.com/news/detail/4/3084506.html}{2}%
\begin{footnote}[1093]\sphinxAtStartFootnote
\sphinxnolinkurl{https://www.donews.com/news/detail/4/3084506.html}
%
\end{footnote}
行业领域:金融支付、借贷、众筹、零售银行、财富管理、征信、保险

\begin{figure}[H]
\centering
\capstart

\noindent\sphinxincludegraphics{{finance_AI}.jpg}
\caption{智能金融\sphinxhref{https://weread.qq.com/web/reader/e77325105e4e55e77af47dbk45c322601945c48cce2e120}{3}\sphinxfootnotemark[1094]}\label{\detokenize{chapter_AI+Finance/FinTech:id24}}\end{figure}
%
\begin{footnotetext}[1094]\sphinxAtStartFootnote
\sphinxnolinkurl{https://weread.qq.com/web/reader/e77325105e4e55e77af47dbk45c322601945c48cce2e120}
%
\end{footnotetext}\ignorespaces 
传统金融vs科技金融:在业务和渠道上传统金融无法同时解决成本和效率问题;科技金融有效匹配普惠金融的需求,脱媒、去中心化和定制化个性化,草根金融。


\paragraph{PEST}
\label{\detokenize{chapter_AI+Finance/FinTech:pest}}

\subparagraph{政策}
\label{\detokenize{chapter_AI+Finance/FinTech:id8}}
2019年8月份,我国金融科技领域第一份科学、全面的规划《金融科技(FinTech)发展规划(2019\sphinxhyphen{}2021年)》由人民银行正式发布,规划中明确提出了未来三年金融科技工作的指导思想、基本原则、发展目标、重点任务和保障措施。尤其是建立健全我国金融科技的“四梁八柱”,确定未来三年六方面的重点任务,为金融科技发展指明了方向和路径。\sphinxhref{https://www.weiyangx.com/378231.html}{4}%
\begin{footnote}[1095]\sphinxAtStartFootnote
\sphinxnolinkurl{https://www.weiyangx.com/378231.html}
%
\end{footnote}


\subparagraph{技术}
\label{\detokenize{chapter_AI+Finance/FinTech:id9}}
《金融科技2020技术应用及趋势报告》:\sphinxurl{http://www.cbdio.com/BigData/2021-03/29/content\_6163778.htm}


\subparagraph{FinTech1.0时代}
\label{\detokenize{chapter_AI+Finance/FinTech:fintech1-0}}
《2018亚太金融科技概览》中所指出金融科技的早期阶段(例如网贷、移动支付和智能投顾)对金融服务业的补充性大于颠覆性。


\subparagraph{互联网和移动互联网技术}
\label{\detokenize{chapter_AI+Finance/FinTech:id10}}
互联网和移动互联网技术使产品在\sphinxstylestrong{用户体验}上取得了革命性的提升,金融产品更是如此。利用互联网和移动设备为客户提供线上服务,简化业务流程,优化产品界面,改善用户体验,这一策略在所有的金融科技行业都是适用的。简单来说,互联网和移动互联网技术使得产品不仅仅是界面变得好看,而是产品更加好用。除此之外,互联网及移动互联网技术使金融服务可以低成本便利的抵达用户,为更多创新性服务提供基础,使其得以实现。


\subparagraph{大数据技术}
\label{\detokenize{chapter_AI+Finance/FinTech:id11}}
在FinTech1.0阶段,大数据技术的主要应用是集中于第一和第二层次,即数据架构和信息整合;初步进入第三层次,进行简单的初步分析和决策。
\begin{itemize}
\item {} 
大数据架构+信息整合。建立一个收集和存储的大数据系统,加之信息整合和数据计算;

\item {} 
人工建模+大数据。该阶段的大数据分析通常依靠人工建模分析,加之由于传统数据分析模型对于多维度、多形态的数据存在不适用的情况,因此该类技术应用仅仅是大数据分析的初级阶段。

\end{itemize}


\subparagraph{FinTech 2.0 时代}
\label{\detokenize{chapter_AI+Finance/FinTech:fintech-2-0}}
金融科技的ABCD”技术,即人工智能(AI)、区块链(Blockchain)、云计算(Cloud
Computing)和大数据(Big
Data)等新技术可能会令金融服务行业发生重大转变。\sphinxhref{https://www.cfainstitute.org/-/media/documents/survey/cfa-institute-ai-pioneer-report-zh-cn.ashx}{11}%
\begin{footnote}[1096]\sphinxAtStartFootnote
\sphinxnolinkurl{https://www.cfainstitute.org/-/media/documents/survey/cfa-institute-ai-pioneer-report-zh-cn.ashx}
%
\end{footnote}


\subparagraph{人工智能(AI)要点}
\label{\detokenize{chapter_AI+Finance/FinTech:ai}}\begin{itemize}
\item {} 
概念:对人的意识、思维的信息过程的模拟

\item {} 
技术:基础层(大数据、云计算、智能芯片、传感器及智能硬件)、技术层(语音识别、图像识别、生物特征识别、机器学习、知识图谱、自然语言处理)和应用层(计算智能、感知智能和认知智能)

\item {} 
商业模式:生态构建者、技术算法驱动者、应用聚焦者、垂直领域先行者和基础设施提供者

\item {} 
应用场景:有智能投顾(理财魔方)、征信、风控(启信宝)、金融搜索引擎(融360)、身份验证(Face++、商汤科技)和智能客服(智齿客服)

\end{itemize}


\subparagraph{区块链(Blockchain)要点:}
\label{\detokenize{chapter_AI+Finance/FinTech:blockchain}}\begin{itemize}
\item {} 
概念(分布式记账原理,去中心化,共识机制,完整分布式不可篡改的账本数据库)

\item {} 
技术(P2P、密码学、共识机制)

\item {} 
核心技术:共识机制(POW,POS,DPOS)

\item {} 
POW:非对称加密,花费很大算力完成数学难题,获得记账权限,其他节点轻松验证(难求解、易验证),篡改需51\%算力

\item {} 
POS:出问题时,持币越多损失越大作弊动机越小,给予更多记账权限,但每一次记账后减小下一次记账概率,没有记账则增加记账概率

\item {} 
DPOS:先选出代表性节点,再从中选一个记录,其他负责核对

\item {} 
基础架构:区块、区块头、创始区块、区块分叉

\item {} 
网络架构:公有链、联盟链、私有链

\item {} 
主要特点:去中心化(P2P传输,密码学,单节点出现问题无影响),不可篡改(篡改会出现分支,且需要51\%算力),加密安全性(非对称密钥,公钥加密私钥解密)

\item {} 
商业模式:加密电子货币(电子钱包)、传统金融网络(零售银行)、金融服务区块链(金融基础设施\&API)、分布式总账(智能合约)

\item {} 
智能合约:数据层,网络层,共识层,应用层

\item {} 
应用场景:数字货币(法定/非法定),支付与清算(跨境支付与清算/银行间清算/OTC清算),金融资产发行与交易(区域股权市场股票发行/金融资产交易/票据交易)

\item {} 
应用标准(深交所):有无中心点?存在互联互通需要?存在增信需要?(一级市场证券发行)存在代码化可能?(金融衍生品),区域股权市场(4个标准全满足)\sphinxhref{https://www.jianshu.com/p/6c76d2aad3f3}{5}%
\begin{footnote}[1097]\sphinxAtStartFootnote
\sphinxnolinkurl{https://www.jianshu.com/p/6c76d2aad3f3}
%
\end{footnote}

\end{itemize}


\subparagraph{云计算(Cloud Computing)要点:}
\label{\detokenize{chapter_AI+Finance/FinTech:cloud-computing}}\begin{itemize}
\item {} 
概念(按需访问和付费,可配置的计算资源共享池)

\item {} 
三种模式(IaaS网络/硬件/存储/管理程序+PaaS虚拟机操作系统+SaaS中间件/应用程序)

\item {} 
四种部署(公有云/私有云/社区云/混合云)

\item {} 
六种技术(IaaS硬件Amazon
EC2、编程模型MapReduce、海量数据分布式存储HDFS/Hive、海量数据管理HBase、虚拟化技术VMWare、云平台管理OpenStack)

\item {} 
商业模式:大公司提供云支持构建云生态,小公司弹性快速的云端业务部署

\item {} 
商业模式:前端(销售,交易前事务,交易中事务),中端(交易服务,交易进程处理),后端(清算/结算,结算后事务)

\item {} 
应用场景:数据管理、合规\&控制

\end{itemize}


\subparagraph{大数据(Big Data)要点:}
\label{\detokenize{chapter_AI+Finance/FinTech:big-data}}\begin{itemize}
\item {} 
概念(规模、速度、种类)

\item {} 
技术(数据采集、数据存储、数据清洗、数据挖掘、数据可视化)

\item {} 
商业模式(TAAS模式、分成模式、内部生态模式)

\item {} 
应用场景(个人和企业用户画像,征信/授信评级/风控:围绕借贷环节的贷前评估、贷中监控和贷后反馈、保险定价:车险及运费险)

\end{itemize}


\paragraph{知名企业}
\label{\detokenize{chapter_AI+Finance/FinTech:id12}}
包括兴业银行、招商银行、光大银行、民生银行、华夏银行、北京银行、建行、工行、中行等都先后成立了金融科技子公司,围绕金融科技的各相关前瞻技术,开始了各种落地探索研究。\sphinxhref{https://www.weiyangx.com/378231.html}{4}%
\begin{footnote}[1098]\sphinxAtStartFootnote
\sphinxnolinkurl{https://www.weiyangx.com/378231.html}
%
\end{footnote}

\sphinxhref{https://assets.kpmg/content/dam/kpmg/cn/pdf/zh/2021/01/china-fintech-50.pdf}{中国领先金融科技企业50}%
\begin{footnote}[1099]\sphinxAtStartFootnote
\sphinxnolinkurl{https://assets.kpmg/content/dam/kpmg/cn/pdf/zh/2021/01/china-fintech-50.pdf}
%
\end{footnote}


\paragraph{行业上下游}
\label{\detokenize{chapter_AI+Finance/FinTech:id13}}
下游有TO
C厂商、上游有数据提供商、中间还有平台解决方案提供商。\sphinxhref{http://www.marsaspect.com/mars/XGJcXF8\%3D}{13}%
\begin{footnote}[1100]\sphinxAtStartFootnote
\sphinxnolinkurl{http://www.marsaspect.com/mars/XGJcXF8\%3D}
%
\end{footnote}


\paragraph{2021十大趋势 6\sphinxfootnotemark[1101]}
\label{\detokenize{chapter_AI+Finance/FinTech:id14}}%
\begin{footnotetext}[1101]\sphinxAtStartFootnote
\sphinxnolinkurl{https://gw.alipayobjects.com/os/bmw-prod/6f1e0b5c-e068-49a6-bc0a-90d5e9131a72.pdf}
%
\end{footnotetext}\ignorespaces 
\begin{figure}[H]
\centering
\capstart

\noindent\sphinxincludegraphics{{FinTech_Trend}.png}
\caption{2021十大趋势}\label{\detokenize{chapter_AI+Finance/FinTech:id25}}\end{figure}


\paragraph{基本原则}
\label{\detokenize{chapter_AI+Finance/FinTech:id15}}\begin{itemize}
\item {} 
守正创新。正确把握金融科技的核心和本质,忠实履行金融的天职和使命,以服务实体经济为宗旨,在遵照法律法规和监管政策前提下,借助现代科技手段提升金融服务效能和管理水平,将科技应用能力内化为金融竞争力,确保金融科技应用不偏离正确方向,使创新成果更具生命力。

\item {} 
安全可控。牢固树立安全发展理念,把安全作为金融科技创新不可逾越的红线,以创新促发展,以安全保发展,借助现代科技成果提升金融风险防控和金融监管效能,完善金融安全防线和风险应急处置机制,提高金融体系抵御风险能力,守住不发生系统性金融风险的底线。

\item {} 
普惠民生。立足广大人民群众美好生活需要,聚焦优化金融服务模式和丰富金融产品供给,充分发挥科技成果在拓展服务渠道、扩大服务覆盖面等方面的作用,推动金融服务“无处不在、无微不至”,为市场主体和人民群众提供更便捷、更普惠、更优质的金融产品与服务。

\item {} 
开放共赢。以促进金融开放为基调,深化金融科技对外合作,加强跨地区、跨部门、跨层级数据资源融合应用,推动金融与民生服务系统互联互通,将金融服务无缝融入实体经济各领域,打破服务门槛和壁垒,拓宽生态边界,形成特色鲜明、布局合理、包容开放、互利共赢的发展格局。

\end{itemize}


\paragraph{中美对比 10\sphinxfootnotemark[1102]}
\label{\detokenize{chapter_AI+Finance/FinTech:id16}}%
\begin{footnotetext}[1102]\sphinxAtStartFootnote
\sphinxnolinkurl{http://www.cstf.org.cn/newsdetail.asp?types=36\&num=1165}
%
\end{footnotetext}\ignorespaces 
\sphinxstylestrong{美国传统金融体系成熟,FinTech更多扮演“补充”角色}

由于美国成熟的金融服务体系,相比“颠覆”银行等传统机构,FinTech公司更多的是寻求与之合作。未被传统金融服务覆盖的客户或市场缝隙,由FinTech企业来补充,其角色更多的是“提高某已有业务的效率”。

反观中国,金融服务供给的不足,部分监管环境的模糊地带给金融科技类公司制造了发展条件。模式创新、普惠金融等在中国的发展十分之迅速。近年来P2P的迅猛发展正说明该问题:大量未被传统借贷服务覆盖的中小企业和个人,通过P2P平台可以获得融资,解决短期的资金缺口。


\subparagraph{征 信}
\label{\detokenize{chapter_AI+Finance/FinTech:id17}}
在征信领域,美国起步早,
征信体系自1920年起伴随消费企业的扩张而推进,征信公司数量曾从2000多家减少到500家,行业经历了充分竞争,机构征信和个人征信体系趋于完善成熟。中国起步晚,线下数据被银行与保险公司垄断割据,线上数据随着互联网的普及而完善,目前数据量庞大但发展历程短暂,征信模型仍待完善。从大数据征信模型算法的成熟度来看,我国虽与美国存在一定的差距,但数据的快速迭代为算法的优化提供了很好的环境。伴随大数据时代的到来,征信数据的应用场景更加丰富,不仅仅用于信贷,更可以满足社交、消费等方面的需求。而这方面的探索尚在起步阶段,国外企业也尚未经历大数据征信的迭代验证。因而从这个角度来讲,美国和我国几乎是站在同一起跑线上的。


\subparagraph{借 贷}
\label{\detokenize{chapter_AI+Finance/FinTech:id18}}
在美国,真正意义上的P2P借贷(即个人对个人的借贷)公司只有Lending
Club和Prosper。其他平台需要投资者不仅仅是高净值个人,而是需要其为具备投资资质的个人,即机构投资者、专业投资者等。而国内的P2P平台则是面向大众的理财工具。


\subparagraph{借贷领域 个人理财}
\label{\detokenize{chapter_AI+Finance/FinTech:id19}}
如上所述,美国传统金融服务完备,因此大多数中产阶级的理财服务是由传统银行和资产管理公司、投资顾问公司提供的。近年来,智能投顾平台(自动化投资平台)的兴起,如Betterment、Wealthfront,其主要服务对象是年轻人群,是未来的中产阶级。相比而言,国内投资者对于智能投顾公司的接受度仍不高。其背后原因是中美投资者不同的投资理念和不同的金融市场环境。智能投顾平台提供的服务是一种消极投资,是长期投资。投资目标是长期(10年以上),使得投资收益与市场持平,这需要投资者具有比较成熟的长期投资理念。国内资本市场有效性不高,投资者散户化程度高,更偏好主动投资和短期投资。


\subparagraph{保 险}
\label{\detokenize{chapter_AI+Finance/FinTech:id20}}
在美国,保险行业的发展是极为发达的。个人保险(如健康险、寿险),财产险(如房产保险、车险)以及企业保险已经成为美国人民生活中的一部分。同时,保险行业的进入门槛非常高,因此保险行业的金融创新也并不火热。相比之下,我国对保险行业的监管也同样严格,牌照被少量国有控股公司垄断。在既有利润丰厚的情况下,公司的创新意识和信息化动力均较低;同时,我国居民保险意识弱,对保险产品很少主动询问或投保。可见,我国保险行业在与科技融合的过程中仍处于非常早期的阶段,目前重点发力在用户体验优化。


\subparagraph{第三方支付}
\label{\detokenize{chapter_AI+Finance/FinTech:id21}}
最早出现的第三方支付平台早在1999年已创立,为美国的Paypal,5年之后阿里巴巴的支付宝业务才推出。在美国,由于美国的信用卡体系已经相对完善,用户体验的提升难度较高,第三方支付作为信用卡支付的替代品,渗透率的增长并不高。另一方面,第三方支付高度依赖互联网平台,即支付的应用场景,而美国电子商务的普及率与中国相比较低,第三方支付应用场景受限。截止目前,中国已成为世界上第三方支付市场份额最重的国家,而支付宝的交易金额也远超第三方支付鼻祖Paypal。

第三方支付高度依赖互联网平台,影响该行业发展的主要因素有:其他支付方式的便利性与安全性、电商的发展。


\paragraph{监管}
\label{\detokenize{chapter_AI+Finance/FinTech:id22}}
北京威诺律师事务所合伙人、清华大学研究生导师杨兆全在接受雷达财经采访时表示:“蚂蚁集团上市被叫停,紧随其后,监管部门出台了若干监管规定。以此为标志,我国对金融科技公司的监管进入新的时代。从放任发展转变到到规范发展,从普通行业监管转变到金融行业监管。金融科技企业必须依法、合规、持牌经营。限制野蛮生长,反对垄断,打击不正当竞争,服务实体经济等,会成为金融监管的主导思想。”


\paragraph{更多}
\label{\detokenize{chapter_AI+Finance/FinTech:id23}}\begin{itemize}
\item {} 
未央网:\sphinxurl{https://www.weiyangx.com/}

\item {} 
雷锋网:\sphinxurl{https://www.leiphone.com/category/fintech}

\item {} 
招聘信息\sphinxhref{https://youwuqiong.com/jinrong-caijing/\%E5\%BF\%83\%E5\%8A\%A8\%E7\%9A\%84offer\%E9\%87\%91\%E8\%9E\%8D\%E4\%B8\%8E\%E7\%A7\%91\%E6\%8A\%80\%E6\%8B\%9B\%E8\%81\%98\%E4\%BF\%A1\%E6\%81\%AF\%E5\%8F\%8C\%E5\%91\%A8\%E6\%8A\%A5\%EF\%BC\%88\%E7\%AC\%AC6\%E6\%9C\%9F\%EF\%BC\%89/}{7}%
\begin{footnote}[1103]\sphinxAtStartFootnote
\sphinxnolinkurl{https://youwuqiong.com/jinrong-caijing/\%E5\%BF\%83\%E5\%8A\%A8\%E7\%9A\%84offer\%E9\%87\%91\%E8\%9E\%8D\%E4\%B8\%8E\%E7\%A7\%91\%E6\%8A\%80\%E6\%8B\%9B\%E8\%81\%98\%E4\%BF\%A1\%E6\%81\%AF\%E5\%8F\%8C\%E5\%91\%A8\%E6\%8A\%A5\%EF\%BC\%88\%E7\%AC\%AC6\%E6\%9C\%9F\%EF\%BC\%89/}
%
\end{footnote}

\item {} 
2021\sphinxhyphen{}2026年中国科技金融服务深度调研与投资战略规划分析报告:\sphinxurl{https://bg.qianzhan.com/report/detail/ef46bc12e785401b.html}

\item {} 
中国科技金融促进会简报:\sphinxurl{http://www.cstf.org.cn/news.asp?types=36}

\item {} 
颖投信息科技有限公司:\sphinxurl{https://www.miotech.com/zh-CN}

\item {} 
金融科技2020年回顾与2021年展望:\sphinxurl{https://www.jrwenku.com/41159.html}

\item {} 
金融科技硏究报告精选:\sphinxurl{https://www.jrwenku.com/21331.html}

\end{itemize}

TODO:
艾瑞咨询:2020年中国金融科技行业发展研究报告(附下载):\sphinxurl{http://www.199it.com/archives/1156070.html}


\subsubsection{AI+Finance}
\label{\detokenize{chapter_AI+Finance/AI_Finance:ai-finance}}\label{\detokenize{chapter_AI+Finance/AI_Finance::doc}}
以机器学习、计算机视觉、自然语言处理、知识图谱为代表的常用型技术,以自监督学习和自动因子、小数据和联邦学习为代表的探索中技术,加之坚实的数字化信息建设、数字化购买渠道、数字化支付平台和数字化网络生态,加快了金融信息和金融数据的流动速率。

人工智能技术在金融行业的应用已相对比较成熟,主要包括以下四个场景:面向客户端的场景应用(智能客服、智能营销、信用评估、智能支付、智能认证、保险定价、承保核保等)、交易和投资管理中的应用(智能投顾、智能投研)、面向监管合规的应用场景(风控)和面向运营环节的应用场景(风险监控、市场预警等)。人工智能技术在其中发挥的作用集中在提升金融主体的内外部效率、提升用户的全流程体验、提升金融服务的数智化程度。

未来3\sphinxhyphen{}5年,人工智能技术在金融行业的应用方向朝着复合型AI、智能型A和安全型A方向发展,同时引领三个趋势:后端金融系统性业务数字化和自动化、前端金融产品在线个性化、金融全流程服务智能化和弹性化。

将人工智能拆分为基础层、技术层和应用层三个层面,基础层作人工智能技术的技术支持,各个细分技术必不可少,特别是大数据的发展;在技术层面,与FinTech最相关的是机器学习和知识图谱,其次是自然语言处理;在应用层主要与计算智能领域相关,应用示例包括神经网络、遗传算法、AlphaGo等。


\subsubsection{投资}
\label{\detokenize{chapter_AI+Finance/Invest:id1}}\label{\detokenize{chapter_AI+Finance/Invest::doc}}
调查结果显示,目前很少有投资从业人员使用通常在机器学习技术中应用的程序,例如
Python、R和MATLAB等编码语言。大多数基金经理在投资策略和流程上依然依赖Exce表格(占受访基金经理的95\%和电脑版市场数据工具(占受访基金经理的四分之三)。

此外,如图2所示,在过去12个月里,仅10\%的受访基金经理使用过人工智能/机器学习技术,在投资策略和流程中使用线性回归的受访者数量,几乎是使用人工智能/机器学习技术的受访者的五倍


\subsubsection{智能顾投}
\label{\detokenize{chapter_AI+Finance/Robo-Advisor:id1}}\label{\detokenize{chapter_AI+Finance/Robo-Advisor::doc}}
智能投顾,又称“机器人投顾”,大多数情况下,通过在线调查问卷来获取投资者关于投资目标、投资期限、收入、资产和风险,来了解投资者的风险偏好以及投资偏好,从而结合算法模型为用户制定个性化的资产配置方案,包括动态调仓,实时监控等功能。相较于传统的投资顾问,机器人投顾通常试图为投资者提供更为便宜的投顾服务,而且很多时候资金门槛也更低。不过,这些机器人投顾提供的服务内容、投资方法和特色都千差万别。
\sphinxhref{https://www.zhihu.com/question/40505552/answer/151307145}{2}%
\begin{footnote}[1104]\sphinxAtStartFootnote
\sphinxnolinkurl{https://www.zhihu.com/question/40505552/answer/151307145}
%
\end{footnote}


\paragraph{作用}
\label{\detokenize{chapter_AI+Finance/Robo-Advisor:id2}}
智能投顾主要是帮助客户简化理财流程,享受更方便快捷的服务。传统上客户往往要亲自去银行理财专柜,填写若干复杂的问卷,看很多理财产品的资料,而智能投顾可以让用户足不出户,在移动端或者pc端上花上几分钟便可完成整个理财的流程。但是这并不能说明智能投顾可以保证客户理财的收益率,只能说智能投顾可以帮助客户用最短的时间,找到用户最喜欢的,最合适的投资标的。


\paragraph{传统的投顾}
\label{\detokenize{chapter_AI+Finance/Robo-Advisor:id3}}
投资顾问。证券投资顾问业务的近代起源可以追溯到传统的私人银行的理财服务业务,私人银行业务在欧洲已经有上百年的历史,最初是瑞士银行专门向富有的上流社会群体提供私密的、一对一的服务。\sphinxhref{https://www.zhihu.com/question/40505552/answer/1403622645}{3}%
\begin{footnote}[1105]\sphinxAtStartFootnote
\sphinxnolinkurl{https://www.zhihu.com/question/40505552/answer/1403622645}
%
\end{footnote}


\paragraph{智能顾投的历史7\sphinxfootnotemark[1106]}
\label{\detokenize{chapter_AI+Finance/Robo-Advisor:id4}}%
\begin{footnotetext}[1106]\sphinxAtStartFootnote
\sphinxnolinkurl{https://www.cnblogs.com/dhcn/p/12093586.html}
%
\end{footnotetext}\ignorespaces 
智能投顾最早起源于美国,发展历程大致可分为三个阶段。
\begin{enumerate}
\sphinxsetlistlabels{\arabic}{enumi}{enumii}{}{.}%
\item {} 
在线投顾阶段:20世纪90年代末期,可供投资者选择应用的投资分析工具的技术水平和规模开始扩大。2005年,FINRA
颁布 NASD IM2210\sphinxhyphen{}6 Requirements for the Use of Investment Analysis
Tools 规章,允许证券自营商将投资分析工具(investment analysis
tools)直接让投资者使用,投资者可以利用投资分析工具进行不同投资策略的投资收益分析,对收益和风险有更好的把控。此后,在线资产管理服务规模迅速增长,更多长尾客户在此阶段受益。此阶段的特点主要是机器智能应用比较有限,主要应用领域是投资组合分析。

\item {} 
机器人投顾阶段:2008年~2015年期间,大量新兴科技企业开始为客户直接提供各类基于机器学习的
“数字化投顾工具”,机器人投顾商业模式开始发展。这些公司开发的面向客户的投顾工具提供的功能之前只被金融从业者应用,目前已经广泛被客户所直接应用。在这个阶段的很多实际应用案例中,证券公司对他们的“数字化投顾工具”提供的投资策略负责。

\item {} 
人工智能投顾阶段:2015年至今,以大数据为基础的深度学习被广泛应用,人工智能技术取得突破型进展。智能投顾服务商和科技企业开始尝试开发能够完全消除人类参与投资管理价值链的人工智能系统。目前包括国外的Bridge
Water、Wealthfront,国内的弥财等都已经实现了这样的系统开发和商业化运营。通常采用“人工智能+云计算”体系结构的服务商,在计算设备和软件开发方面投资巨大(少则1\sphinxhyphen{}2亿,多则几十亿),能够同时服务千万、亿级别的海量用户。\sphinxhref{https://www.cnblogs.com/dhcn/p/12093586.html}{7}%
\begin{footnote}[1107]\sphinxAtStartFootnote
\sphinxnolinkurl{https://www.cnblogs.com/dhcn/p/12093586.html}
%
\end{footnote}

\end{enumerate}


\paragraph{智能投顾分类}
\label{\detokenize{chapter_AI+Finance/Robo-Advisor:id5}}
\begin{figure}[H]
\centering
\capstart

\noindent\sphinxincludegraphics{{Robo-Advisor}.png}
\caption{智能投顾分类\sphinxhref{https://www.jianshu.com/p/6c76d2aad3f3}{4}\sphinxfootnotemark[1108]}\label{\detokenize{chapter_AI+Finance/Robo-Advisor:id9}}\end{figure}
%
\begin{footnotetext}[1108]\sphinxAtStartFootnote
\sphinxnolinkurl{https://www.jianshu.com/p/6c76d2aad3f3}
%
\end{footnotetext}\ignorespaces 

\paragraph{资产配置}
\label{\detokenize{chapter_AI+Finance/Robo-Advisor:id6}}
投资收益中资产配置的重要性,而资产配置又是投资顾问的重要职能,中国理财市场里存在大量挣钱的金融产品,也有大量低学历的金融销售,但恰恰缺少大量优秀的投资顾问,缺少专业人才给投资用户做资产配置,这也是为什么中国大部分投资用户会赔钱。
\sphinxhref{https://www.zhihu.com/question/40505552}{1}%
\begin{footnote}[1109]\sphinxAtStartFootnote
\sphinxnolinkurl{https://www.zhihu.com/question/40505552}
%
\end{footnote}


\paragraph{优点 5\sphinxfootnotemark[1110]}
\label{\detokenize{chapter_AI+Finance/Robo-Advisor:id7}}%
\begin{footnotetext}[1110]\sphinxAtStartFootnote
\sphinxnolinkurl{https://www.douban.com/note/557874669/}
%
\end{footnotetext}\ignorespaces \begin{enumerate}
\sphinxsetlistlabels{\arabic}{enumi}{enumii}{}{.}%
\item {} 
最优配置。最优配置是匹配给每一个不同的投资人的,因为每个人的情况不一样,有的人可能能够承受5\%的亏损,有的人却亏一点就夜不能寐;有的人这笔钱可能是马上就要用,但是有的人这笔钱可以放很久。所以不同的人所需要的资产配置解决方案应该是不一样的,它是针对个人的最优配置。

\item {} 
降低服务门槛。传统上的资产配置是由私人银行加主权基金这些大的机构使用的,通过一个机器人的解决方案,能够把这个服务门槛降到一个普通的中产阶级家庭和新富裕的人群来使用。

\item {} 
战胜人性(非常有效的一点)。在投资过程中,贪婪和恐惧永远在两端摇摆,人性往往使其做出错误的判断。如果你是一个机器人,则会非常冷静,在恐惧的时候也不会做出错误的决定,从而帮助你去战胜人性的弱点。

\item {} 
它会帮你降低交易成本,因为智能投顾很多时候是通过 ETF
交易,由于背后是大数据的抓取和海量运算,节省了人工顾问的成本,以及线下开设网点的成本。

\item {} 
机器人的投顾跟投资人的利益会保持一致。传统的投顾往往根据交易的佣金来获利,所以他们之间存在潜在的利益冲突,他希望你能够频繁交易。但其实频繁的交易可能会导致损失,而机器人投顾不以收取佣金为目的,不以频繁交易为目的,跟你的利益是一致的。

\end{enumerate}


\paragraph{公司}
\label{\detokenize{chapter_AI+Finance/Robo-Advisor:id8}}
目前我国提供此服务的公司很多,其中银行系(如广发智投、招行摩羯智投、工行“AI”投等)、基金系(如南方基金超级智投宝、广发基金基智理财等)、大型互联网公司系(如百度金融、京东智投、同花顺)和第三方创业公司(如弥财、蓝海财富、拿铁财经等)都在智能投顾上有所应用。\sphinxhref{http://www.szama.org/html356/356/5382.html}{6}%
\begin{footnote}[1111]\sphinxAtStartFootnote
\sphinxnolinkurl{http://www.szama.org/html356/356/5382.html}
%
\end{footnote}


\paragraph{2C到2B}
\label{\detokenize{chapter_AI+Finance/Robo-Advisor:c2b}}
近年来我们看到,众多金融科技公司正从市场的颠覆者,转变为与传统机构的合作者、赋能者,从另一个角度引领整个市场的创新活力。从2C模式转变成2B模式是财富管理金融科技公司快速、低成本积累管理资产规模、客户数据与实战场景的有效手段。以美国领先的智能投顾金融科技公司Betterment为例,2014年Betterment推出2B产品线Betterment
for
Advisors,将智能投顾工具开放给超过300家中小财富管理机构,并为富达集团等大机构的投资顾问提供智能投顾支持。2B产品帮助Betterment的管理资产在四年内迅速突破140亿美元34,与专注于2C模式时经历七年发展规模仍不足20亿美元相比,实现了跨越式的增长。\sphinxhref{https://www.thepaper.cn/newsDetail\_forward\_11822405}{8}%
\begin{footnote}[1112]\sphinxAtStartFootnote
\sphinxnolinkurl{https://www.thepaper.cn/newsDetail\_forward\_11822405}
%
\end{footnote}

TODO: 智能投顾是什么?「智能」体现在哪些方面? \sphinxhyphen{} 知乎
\sphinxurl{https://www.zhihu.com/question/40505552/answer/137963123}

智能顾投: \sphinxurl{https://t.qidianla.com/1162362.html}

智能投顾面面观之AI慕课: \sphinxurl{https://www.jianshu.com/p/437c895794e0}


\subsubsection{智能投研}
\label{\detokenize{chapter_AI+Finance/AI_Investment_Research:id1}}\label{\detokenize{chapter_AI+Finance/AI_Investment_Research::doc}}

\paragraph{定义}
\label{\detokenize{chapter_AI+Finance/AI_Investment_Research:id2}}
在基础的金融数据基础上,通过深度学习、自然语音处理等等人工智能算法,对数据、时间和结论信息进行\sphinxstylestrong{自动分析处理},为金融机构从业人员提供辅助,以提高行业工作效率。\sphinxhref{https://www.weiyangx.com/362573.html}{1}%
\begin{footnote}[1113]\sphinxAtStartFootnote
\sphinxnolinkurl{https://www.weiyangx.com/362573.html}
%
\end{footnote}


\paragraph{需求}
\label{\detokenize{chapter_AI+Finance/AI_Investment_Research:id3}}
随着互联网的普及,我们已身处信息爆炸时代,投研业务面临数据量过大、数据渠道过多、数据结构多样、数据真假难辨等突出问题,这些问题单靠传统的投研处理方法和手段已很难胜任或很难长期胜任,急需利器。

投研作为投资业务链条的一环,交易、风控等都在快速智能化发展,投研同样需要变革;保持分析的客观性是投研工作的基本要求,借助现代科技手段,既可以提高各类投研工作的效率,又可以\sphinxstylestrong{弥补人类主观情绪}的干扰。

投行业务及证券研究业务中涉及大量的固定格式的文档撰写工作,如招股说明书、研究报告及投资意向书等。这些报告的撰写需要初级研究员投入大量的时间及精力进行数据整理以及文本复制粘贴的工作。而这些文档中,有大量内容可以利用模板生成,比如公司股权变更、会计数据变更等等。\sphinxstylestrong{利用自然语言处理及OCR技术}可以方便快捷的完成以上工作,并最终形成文档。\sphinxhref{http://www.cstf.org.cn/newsdetail.asp?types=36\&num=1165}{6}%
\begin{footnote}[1114]\sphinxAtStartFootnote
\sphinxnolinkurl{http://www.cstf.org.cn/newsdetail.asp?types=36\&num=1165}
%
\end{footnote}


\paragraph{传统投研 VS 智能投研}
\label{\detokenize{chapter_AI+Finance/AI_Investment_Research:vs}}
\begin{figure}[H]
\centering
\capstart

\noindent\sphinxincludegraphics{{traditional_vs_AI_Investment_Research}.png}
\caption{传统投研 VS 智能投研}\label{\detokenize{chapter_AI+Finance/AI_Investment_Research:id11}}\end{figure}


\paragraph{基本流程}
\label{\detokenize{chapter_AI+Finance/AI_Investment_Research:id4}}\begin{enumerate}
\sphinxsetlistlabels{\arabic}{enumi}{enumii}{}{.}%
\item {} 
数据获取(金融数据、另类数据)

\item {} 
数据处理(结构化、模块化、建立联系)–模式化、标准化的工作

\item {} 
数据分析(建立公司、关键股东等动态知识图谱\sphinxhref{https://zhuanlan.zhihu.com/p/46750171}{3}%
\begin{footnote}[1115]\sphinxAtStartFootnote
\sphinxnolinkurl{https://zhuanlan.zhihu.com/p/46750171}
%
\end{footnote})

\item {} 
观点输出(\sphinxstylestrong{输出观点}、分析结果)–\sphinxstylestrong{核心价值}

\end{enumerate}


\paragraph{智能投研 VS 智能投顾}
\label{\detokenize{chapter_AI+Finance/AI_Investment_Research:id5}}\begin{itemize}
\item {} 
智能投顾主要是将人工智能辅助引入投顾业务。智能投顾更加偏向于站在客户立场上为客户提供资产配置建议,对客户资金配置到股票、债券、基金等品种上的份额提供合理建议,收取咨询服务费,主要面向C端客户

\item {} 
智能投研则是将人工智能引入投研业务中。智能投研更加偏向于辅助资产管理,服务于金融机构的投研人员,主要面向B端。

\end{itemize}


\paragraph{知识抽取}
\label{\detokenize{chapter_AI+Finance/AI_Investment_Research:id6}}
在金融领域,迅速、全面、准确地获取有价值的行业信息是决定一个企业成败的关键。近些年,随着互联网和金融行业的快速发展,每天都有大量的金融文本产生,面对着海量的公司年报、公告、新闻,其内容分散,数据稀疏,无结构化信息等特点逐渐凸显。如何在数据爆炸的信息中高效找到有价值的知识,将有价值的无结构化信息进行半结构化或结构化是首先需要解决的问题,而信息抽取则是知识发现的核心之一。

目前在金融领域中,\sphinxstylestrong{文本内容的知识抽取}主要依靠人工判断,分析人员一般需要阅读大量的相关文档(如:年报、公告、行业分析报告、新闻等),然后从中获取关键信息,为决策提供依据。这种手工作业的方式效率较低,且依赖于从业人员的经验,学习门槛较高,不利于企业业务进一步拓展。\sphinxhref{https://www.zhihu.com/question/41187047/answer/1262911922}{4}%
\begin{footnote}[1116]\sphinxAtStartFootnote
\sphinxnolinkurl{https://www.zhihu.com/question/41187047/answer/1262911922}
%
\end{footnote}

CCKS
2020:面向金融领域的小样本跨类迁移事件抽取\sphinxhref{https://github.com/xiaoqian19940510/CCKS-2020-event-extraction}{7}%
\begin{footnote}[1117]\sphinxAtStartFootnote
\sphinxnolinkurl{https://github.com/xiaoqian19940510/CCKS-2020-event-extraction}
%
\end{footnote}


\paragraph{智能投研产业链}
\label{\detokenize{chapter_AI+Finance/AI_Investment_Research:id7}}
上游数据源与数据采集、中游数据处理、下游为各类用户。

TODO:\sphinxhref{https://www.weiyangx.com/362573.html}{1}%
\begin{footnote}[1118]\sphinxAtStartFootnote
\sphinxnolinkurl{https://www.weiyangx.com/362573.html}
%
\end{footnote}


\paragraph{不是替代,而是辅助}
\label{\detokenize{chapter_AI+Finance/AI_Investment_Research:id8}}
智能投研不是替代投研人员,更无法替代资深的投研人员。毕竟很多信息还需要现场调研获取和确认,分析模型、逻辑也需要逐步完善,服务于基金经理、交易员的有灵魂的报告还离不开人去加工润色。如果非要说替代,那替代的首先是价值含量低的工作,比如当前很多公司临时聘用实习生来做的数据采集、处理、分析等工作;其次是那些没有掌握智能投研利器的从业人员。

智能投研在可预见的将来仍将以做辅助工作为主,投研的核心是\sphinxstylestrong{优秀的投研人员输出观点},根据有效市场理论,正是因为不同投研人员的预期不同,所以才能为客户赚取超额收益,从而跑赢市场。


\paragraph{相关企业}
\label{\detokenize{chapter_AI+Finance/AI_Investment_Research:id9}}
国内外的初创企业也是非常重要的推动力量,比如
AlphaSense推出的智能擭索引擎,使用自然语言处理技术来提取对投资可能产生影响的关键信息。

国内比如文因互联强化自身深度语义分析和深度文档分析能力的技术优势,过自然语言处理和知识图谱,来对金融数据进行结构化地提取和分析;阿博茨与汇添富基金合作,为汇添富提供智能投研系统,通过应用知识图谱、大数据和A算法,针对投硏数据进行搜索定位从而提高信息阅读效率,辅助投研。此外比如通联数据的萝卜投研、香侬科技熵简科技等也在智能投研方面有相应布局\sphinxhref{http://www.199it.com/archives/1223002.html}{5}%
\begin{footnote}[1119]\sphinxAtStartFootnote
\sphinxnolinkurl{http://www.199it.com/archives/1223002.html}
%
\end{footnote}


\paragraph{更多}
\label{\detokenize{chapter_AI+Finance/AI_Investment_Research:id10}}
\sphinxurl{http://gb.oversea.cnki.net/KCMS/detail/detail.aspx?filename=1020099658.nh\&dbcode=CMFD\&dbname=CMFDREF}


\subsubsection{智能风控}
\label{\detokenize{chapter_AI+Finance/AI_Risk_Management:id1}}\label{\detokenize{chapter_AI+Finance/AI_Risk_Management::doc}}
在传统风控环节中,信息不对称、成本高、时效性差、效率低等问题,使得难以满足业务的快速增长引发的信贷增长。而风控引入人工智能技术,使得贷前审核、贷中监控和贷后管理、监管合规等环节,都能提高金融科技产品质量及服务效率。

智能风控能不仅能有效提高金融服务的效率和安全性,降低风控成本,还能促进风险管理差异化和业务人性化,在金融科技业中有着重要作用。所以近年来无论是传统金融机构、消费金融机构还是互联网金融公司,都在加紧智能化系统建设或者对外合作,实现智能化风控。

未来,智能风控要不仅在信贷、反欺诈、异常交易检测等领域发挥作用,充分发挥大数据、人工智能、云计算等综合技术优势,为金融行业欺诈风险的分析和预警监测提供支持,同时也要不断优化数据挖掘采集效率及人工智能算法,同时要实现对各金融机构、金融科技公司数据的标准化,以实现更多的价值场景的快速落地。

从监管层面,监管机构鼓励金融服务业在风险可控的前提下创新,但随着监管机构加强风险、合规和安全方面的监管,金融服务业需要通过更有效的手段来满足监管要求。\sphinxhref{https://www.weiyangx.com/351456.html}{1}%
\begin{footnote}[1120]\sphinxAtStartFootnote
\sphinxnolinkurl{https://www.weiyangx.com/351456.html}
%
\end{footnote}

金融风控解决方案:\sphinxurl{https://ecloud.10086.cn/home/solution/finance/risk}

风控模型—群体稳定性指标(PSI)深入理解应用:\sphinxurl{https://zhuanlan.zhihu.com/p/79682292}

蚂蚁金服科技: \sphinxurl{https://mp.weixin.qq.com/s/klOGk\_3X7QT78w9YPQQzKA}


\subsubsection{智能客服}
\label{\detokenize{chapter_AI+Finance/AI_customer_service:id1}}\label{\detokenize{chapter_AI+Finance/AI_customer_service::doc}}
智能客户服务解决方案使用我们获奖的人工智能技术,帮助金融机构提高客户服务功能的质量和便利性,同时利用该技术提高效率和减少人员需求。能够7*24机器人客服全天候实时响应客户咨询,并在与客户交互的过程中,通过收集客户反馈,不断训练优化算法,为客户提供更高效的解决方案,降低企业运营成本。\sphinxhref{https://www.sohu.com/a/393727642\_676545}{3}%
\begin{footnote}[1121]\sphinxAtStartFootnote
\sphinxnolinkurl{https://www.sohu.com/a/393727642\_676545}
%
\end{footnote}AI技术应用程序使金融机构客户能够取代纸张和人力密集型的流程和传统基础设施。所有垂直行业的金融机构均可在其客户服务职能中使用该解决方案的模块。

客服机器人依据媒体类型分为两种,基于电话语音,基于文本信息。前者如10086的自助语音服务;后者一般是置于应用中,能解决简单而又大量重复问题,以节约成本,如淘宝的小蜜、万象,直播APP里的客服助手等。

从输入方式来看有:语音输入和文字输入,技术上的区别是语音输入要做语音识别,将语音信号转换成文字。目前客服机器人主要是两者都支持。

从输出方式来看有:文字输出、图像输出、语音输出。这个类型视业务需求、产品场景所决定。目前客服机器人主要是文字输出,一般基于用户画像预测问题、热门问题前置等。\sphinxhref{https://mp.weixin.qq.com/s/hdmV5bHbMqyIB7E3A0Igrw}{2}%
\begin{footnote}[1122]\sphinxAtStartFootnote
\sphinxnolinkurl{https://mp.weixin.qq.com/s/hdmV5bHbMqyIB7E3A0Igrw}
%
\end{footnote}

智能客服、聊天机器人的应用和架构、算法分享和介绍\sphinxhref{https://github.com/chatopera/chatbot.catalog.customer-service}{1}%
\begin{footnote}[1123]\sphinxAtStartFootnote
\sphinxnolinkurl{https://github.com/chatopera/chatbot.catalog.customer-service}
%
\end{footnote}


\subsubsection{RPA}
\label{\detokenize{chapter_AI+Finance/RPA:rpa}}\label{\detokenize{chapter_AI+Finance/RPA::doc}}
\sphinxurl{http://www.rpa-cn.com/zuixinzixun/AIshijiao/}


\subsubsection{智能营销}
\label{\detokenize{chapter_AI+Finance/MarTech:id1}}\label{\detokenize{chapter_AI+Finance/MarTech::doc}}
采集客户在各渠道的行为数据,利用深度学习、自然语言处理等技术构建认知模型,通过全渠道精准投放信息,为消费者提供千人千面的个性化营销服务。


\subsubsection{全栈金融}
\label{\detokenize{chapter_AI+Finance/Full_stack:id1}}\label{\detokenize{chapter_AI+Finance/Full_stack::doc}}
\sphinxurl{https://www.jianshu.com/p/6c3888c2e846}



\renewcommand{\indexname}{Index}
\printindex
\end{document}